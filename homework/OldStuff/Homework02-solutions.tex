\documentclass[11pt]{exam}

\usepackage{amssymb, amsmath, amsthm, mathrsfs, multicol, graphicx} 
\usepackage{tikz}
\usepackage[bottom=.25in, top=.75in]{geometry}

 \def\d{\displaystyle}
\def\?{\reflectbox{?}}
\def\b#1{\mathbf{#1}}
\def\f#1{\mathfrak #1}
\def\c#1{\mathcal #1}
\def\s#1{\mathscr #1}
\def\r#1{\mathrm{#1}}
\def\N{\mathbb N}
\def\Z{\mathbb Z}
\def\Q{\mathbb Q}
\def\R{\mathbb R}
\def\C{\mathbb C}
\def\F{\mathbb F}
\def\A{\mathbb A}
\def\X{\mathbb X}
\def\E{\mathbb E}
\def\O{\mathbb O}
\def\U{\mathcal U}
\def\pow{\mathcal P}
\def\inv{^{-1}}
\def\nrml{\triangleleft}
\def\st{:}
\def\~{\widetilde}
\def\rem{\mathcal R}
\def\sigalg{$\sigma$-algebra }
\def\Gal{\mbox{Gal}}
\def\iff{\leftrightarrow}
\def\Iff{\Leftrightarrow}
\def\land{\wedge}
\def\And{\bigwedge}
\def\AAnd{\d\bigwedge\mkern-18mu\bigwedge}
\def\Vee{\bigvee}
\def\VVee{\d\Vee\mkern-18mu\Vee}
\def\imp{\rightarrow}
\def\Imp{\Rightarrow}
\def\Fi{\Leftarrow}

%\def\={\equiv}
\def\var{\mbox{var}}
\def\mod{\mbox{Mod}}
\def\Th{\mbox{Th}}
\def\sat{\mbox{Sat}}
\def\con{\mbox{Con}}
\def\bmodels{=\joinrel\mathrel|}
\def\iffmodels{\bmodels\models}
\def\dbland{\bigwedge \!\!\bigwedge}
\def\dom{\mbox{dom}}
\def\rng{\mbox{range}}
\DeclareMathOperator{\wgt}{wgt}


\def\bar{\overline}


\newcommand{\vtx}[2]{node[fill,circle,inner sep=0pt, minimum size=4pt,label=#1:#2]{}}
\newcommand{\va}[1]{\vtx{above}{#1}}
\newcommand{\vb}[1]{\vtx{below}{#1}}
\newcommand{\vr}[1]{\vtx{right}{#1}}
\newcommand{\vl}[1]{\vtx{left}{#1}}
\renewcommand{\v}{\vtx{above}{}}

\def\circleA{(-.5,0) circle (1)}
\def\circleAlabel{(-1.5,.6) node[above]{$A$}}
\def\circleB{(.5,0) circle (1)}
\def\circleBlabel{(1.5,.6) node[above]{$B$}}
\def\circleC{(0,-1) circle (1)}
\def\circleClabel{(.5,-2) node[right]{$C$}}
\def\twosetbox{(-2,-1.4) rectangle (2,1.4)}
\def\threesetbox{(-2.5,-2.4) rectangle (2.5,1.4)}
\newcommand{\twoline}[2]{\begin{pmatrix}#1 \\ #2 \end{pmatrix}}


%\pointname{pts}
\pointsinmargin
\marginpointname{pts}
\addpoints
\pagestyle{head}
\printanswers

\firstpageheader{Math 228}{\bf Homework 2\\ Solutions}{Spring 2013}


\begin{document}


\begin{questions}
\question[6] 
\begin{parts}
  \part Make a truth table for the statement $P \imp (\neg Q \vee R)$.
  \begin{solution}
    \begin{center}
  \begin{tabular}{c|c|c||c|c}
    $P$ & $Q$ & $R$ & $\neg Q \vee R$ & $P \imp (\neg Q \vee R)$\\ \hline
    T & T & T & T & T\\
    T & T & F & F & F \\
    T & F & T & T & T \\
    T & F & F & T & T \\
    F & T & T & T & T \\
    F & T & F & F & T \\
    F & F & T & T & T \\
    F & F & F & T & T
  \end{tabular}
\end{center}
  \end{solution}

  \part If Tommy {\bf lies} when he says, ``if I ate pizza, then either I didn't eat cucumber sandwiches or I did eat raisins,'' what can you conclude about what Tommy ate?  Explain.
  \begin{solution}
    The statement made is the same as the one we made a truth table for above.  If the statement is a lie, then we are in the case(s) in which the statement is false.  This turns out to be only the second case, so we see that $P$ and $Q$ are true and $R$ is false.  Therefore Tommy ate pizza a cucumber sandwiches, but not raisins.
  \end{solution}

\end{parts}
  
\question[6] Can you distribute conjunctions over disjunctions?  Disjunctions over conjunctions?  Let's find out.  Remember, two statements are logically equivalent if they are true in exactly the same cases.
\begin{parts}
  \part Are the statements $P \vee (Q \and R)$ and $(P \vee Q) \and (P \vee R)$ logically equivalent?  
  \begin{solution}
    Yes they are.  We prove this by showing that their truth tables are identical:
    \begin{center}
        \begin{tabular}{c|c|c||c||c}
    $P$ & $Q$ & $R$ & $P \vee (Q \and R)$ & $(P \vee Q) \and (P \vee R)$\\ \hline
    T & T & T & T & T\\
    T & T & F & T & T \\
    T & F & T & T & T \\
    T & F & F & T & T \\
    F & T & T & T & T \\
    F & T & F & F & F \\
    F & F & T & F & F \\
    F & F & F & F & F
  \end{tabular}
    \end{center}
  \end{solution}

  \part Are the statements $P \and (Q \vee R)$ and $(P \and Q) \vee (P \and R)$ logically equivalent?
  \begin{solution}
    It works again.  Here are the two truth tables which prove it:
        \begin{center}
        \begin{tabular}{c|c|c||c||c}
    $P$ & $Q$ & $R$ & $P \and (Q \vee R)$ & $(P \and Q) \vee (P \and R)$\\ \hline
    T & T & T & T & T\\
    T & T & F & T & T \\
    T & F & T & T & T \\
    T & F & F & F & F \\
    F & T & T & F & F \\
    F & T & F & F & F \\
    F & F & T & F & F \\
    F & F & F & F & F
  \end{tabular}
    \end{center}
  \end{solution}

\end{parts}

\question[6] Use De Morgan's Laws, and any other logical equivalence facts you know to simplify the following statements.  Show all your steps, justifying each.  Your final statements should have negations only appear directly next to the propositional variables ($P$, $Q$, etc.), and no double negations.  
\begin{parts}
  \part $\neg((\neg P \and Q) \vee \neg(R \vee \neg S))$.
  \begin{solution}
    $\neg((\neg P \and Q) \vee \neg(R \vee \neg S))$\\
    $\neg(\neg P \and Q) \and \neg\neg(R \vee \neg S)$ by De Morgan's law.\\
    $\neg(\neg P \and Q) \and (R \vee \neg S)$ by double negation.\\
    $(\neg\neg P \vee \neg Q) \and (R \vee \neg S)$ by De Morgan's law.\\
    $(P \vee \neg Q) \and (R \vee \neg S)$ by double negation.
  \end{solution}

  \part $\neg((\neg P \imp \neg Q) \and (\neg Q \imp R))$ (careful with the implications).
  \begin{solution}
    We will need to convert the implications to disjunctions so we can apply De Morgan's law:
    
    $\neg((\neg P \imp \neg Q) \and (\neg Q \imp R))$\\
    $\neg((\neg \neg P \vee \neg Q) \and (\neg\neg Q \vee R))$ by implication/disjunction equivalence.\\
    $\neg((P \vee \neg Q) \and (Q \vee R))$ by double negation.\\
    $\neg(P \vee \neg Q) \vee \neg (Q \vee R)$ by De Morgan's law.\\
    $(\neg P \and \neg \neg Q) \vee (\neg Q \and \neg R)$ by De Morgan's law.\\
    $(\neg P \and Q) \vee (\neg Q \and \neg R)$ by double negation.
  \end{solution}

\end{parts}

% \question[4] Find a statement which has the following truth table.  You final answer should contain only one instance of each of the variables $P$, $Q$ and $R$.
% 
% \begin{center}
%   \begin{tabular}{c|c|c||c}
%     $P$ & $Q$ & $R$ & ???\\ \hline
%     T & T & T & F \\
%     T & T & F & F \\
%     T & F & T & T \\
%     T & F & F & T \\
%     F & T & T & T \\
%     F & T & F & F \\
%     F & F & T & T \\
%     F & F & F & T
%   \end{tabular}
% \end{center}
% \begin{solution}
%   One possible statement is this: $\neg Q \vee (\neg P \and R)$.  Of course, there are many statements logically equivalent to this such as $Q \imp (\neg P \and R)$ or $Q \imp \neg(R \imp P)$.  
%   
%   To find the solution, you have lots of choices.  Notice that whenever $Q$ is false, the statement is true.  But also the statement is true in one case when $Q$ is true - when $P$ is false and $R$ is true.  The other case which has $P$ false and $R$ true (and $Q$ false) also results in a true statement, so we to get truth we must have either $Q$ false (so $\neg Q$ true) or $\neg P$ and $R$ both true (so $P$ is false and $R$ is true).
%   
%   An alternate method of arriving at the solution is to notice that the statement is false more than it is true, so first find a the negation of the statement.  This leads to $\neg ((P \and Q) \vee (\neg P \and Q \and \neg R)$.  If you simplify this, you get the same answer as above.
% \end{solution}

\question[6] Can you chain implications together?  That is, if $P \imp Q$ and $Q \imp R$, does that means the $P \imp R$?  Can you chain more implications together?  Let's find out:
\begin{parts}
  \part Prove that the following is a valid argument form:
  \begin{tabular}{rc}
    & $P \imp Q$ \\
    & $Q \imp R$ \\ \hline
    $\therefore$ & $P \imp R$
  \end{tabular}
  
  \begin{solution}
    Consider the truth table:
    
    \begin{tabular}{c|c|c||c|c|c}
      $P$ & $Q$ & $R$ & $P\imp Q$ & $Q \imp R$ & $P \imp R$ \\ \hline
      T   &  T  &  T  &     T     &      T     &      T \\
      T   &  T  &  F  &     T     &      F     &      F \\
      T   &  F  &  T  &     F     &      T     &      T \\
      T   &  F  &  F  &     F     &      T     &      F \\
      F   &  T  &  T  &     T     &      T     &      T \\
      F   &  T  &  F  &     T     &      F     &      T \\
      F   &  F  &  T  &     T     &      T     &      T \\
      F   &  F  &  F  &     T     &      T     &      T \\
    \end{tabular}
    Notice that both $P \imp Q$ and $Q \imp R$ are true in rows 1, 5, 7 and 8.  In each of these rows, $P \imp R$ is also true.  So whenever the premises are true, so in the conclusion.  Thus the argument form is valid.
  \end{solution}

% \part Prove that the following is a valid argument form:
%   \begin{tabular}{rc}
%     & $P \imp Q$ \\
%     & $Q \imp R$ \\
%     & $R \imp S$ \\ \hline
%     $\therefore$ & $P \imp S$
%   \end{tabular}
\part Prove that the following is a valid argument form: 
\begin{tabular}{rc}
  & $P_1 \imp P_2$\\
  & $P_2 \imp P_3$ \\ 
  & $\vdots$ \\
  & $P_8 \imp P_9$ \\ \hline
  $\therefore$ & $P_1 \imp P_9$.
\end{tabular}
I suggest you don't go through the trouble of writing out a 512 row truth table.  You should still be able to explain why this is argument form is valid (using part (a)).
\begin{solution}
  From the first two premises, we can conclude that $P_1 \imp P_3$, using the same reasoning as in part (a).  The next line of the argument form is $P_3 \imp P_4$.  Since we know that $P_1 \imp P_3$ is true as well, we can conclude (using part (a) again) that $P_1 \imp P_4$. And so on.  Eventually we will find that $P_1 \imp P_8$.  This combined with the last premise ($P_8 \imp P_9$) allows us to conclude that $P_1 \imp P_9$.
\end{solution}

\end{parts}

\question[6] Consider the statement, ``if you study logic, then you will be happy.''
\begin{parts}
  \part Rephrase the implication in at least 3 different ways.  At least one of the ways should use necessary/sufficient language.
  \begin{solution}
    The implication is equivalent to each of the following:
    \begin{itemize}
      \item You will be happy if you study logic.
      \item You will study logic only if you are happy.
      \item To be happy it is sufficient to study logic.
      \item Being happy necessarily follows from studying logic.
      \item If you are unhappy, then you have not studied logic.
      \item Either you don't study logic or you are happy.
    \end{itemize}

  \end{solution}

  \part State the converse of the implication, and rephrase the converse in at least 3 different ways.
  \begin{solution}
    The converse is, ``if you are happy, then you study logic.''  This is equivalent to each of the following:
    \begin{itemize}
      \item You study logic if you are happy.
      \item You are happy only if you study logic.
      \item To be happy it is necessary to study logic.
      \item To study logic it is sufficient to be happy.
      \item If you don't study logic, then you are not happy.
      \item Either are you not happy or you do not study logic.
    \end{itemize}

  \end{solution}

\end{parts}


\end{questions}




\end{document}


