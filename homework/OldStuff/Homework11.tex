\documentclass[11pt]{exam}

\usepackage{amssymb, amsmath, amsthm, mathrsfs, multicol, graphicx} 
\usepackage{tikz}

\def\d{\displaystyle}
\def\?{\reflectbox{?}}
\def\b#1{\mathbf{#1}}
\def\f#1{\mathfrak #1}
\def\c#1{\mathcal #1}
\def\s#1{\mathscr #1}
\def\r#1{\mathrm{#1}}
\def\N{\mathbb N}
\def\Z{\mathbb Z}
\def\Q{\mathbb Q}
\def\R{\mathbb R}
\def\C{\mathbb C}
\def\F{\mathbb F}
\def\A{\mathbb A}
\def\X{\mathbb X}
\def\E{\mathbb E}
\def\O{\mathbb O}
\def\pow{\mathscr P}
\def\inv{^{-1}}
\def\nrml{\triangleleft}
\def\st{:}
\def\~{\widetilde}
\def\rem{\mathcal R}
\def\iff{\leftrightarrow}
\def\Iff{\Leftrightarrow}
\def\and{\wedge}
\def\And{\bigwedge}
\def\AAnd{\d\bigwedge\mkern-18mu\bigwedge}
\def\Vee{\bigvee}
\def\VVee{\d\Vee\mkern-18mu\Vee}
\def\imp{\rightarrow}
\def\Imp{\Rightarrow}
\def\Fi{\Leftarrow}

\def\={\equiv}
\def\var{\mbox{var}}
\def\mod{\mbox{Mod}}
\def\Th{\mbox{Th}}
\def\sat{\mbox{Sat}}
\def\con{\mbox{Con}}
\def\bmodels{=\joinrel\mathrel|}
\def\iffmodels{\bmodels\models}
\def\dbland{\bigwedge \!\!\bigwedge}
\def\dom{\mbox{dom}}
\def\rng{\mbox{range}}
\DeclareMathOperator{\wgt}{wgt}

\def\circleA{(-.5,0) circle (1)}
\def\circleAlabel{(-1.5,.6) node[above]{$A$}}
\def\circleB{(.5,0) circle (1)}
\def\circleBlabel{(1.5,.6) node[above]{$B$}}
\def\circleC{(0,-1) circle (1)}
\def\circleClabel{(.5,-2) node[right]{$C$}}
\def\twosetbox{(-2,-1.5) rectangle (2,1.5)}
\def\threesetbox{(-2,-2.5) rectangle (2,1.5)}


\def\bar{\overline}

%\pointname{pts}
\pointsinmargin
\marginpointname{pts}
\marginbonuspointname{pts-bns}
\addpoints
\pagestyle{headandfoot}
%\printanswers

\firstpageheader{Math 228}{\bf Homework 11}{} %Due: Wed, May 1}
%\firstpagefooter{}{}{\footnotesize \em More $\rightarrow$}
\runningfooter{}{}{}

\def\vertexsize{4pt}
\newcommand{\vtx}[2]{node[fill,circle,inner sep=0pt, minimum size=\vertexsize,label=#1:#2]{}}
\newcommand{\va}[1]{\vtx{above}{#1}}
\newcommand{\vb}[1]{\vtx{below}{#1}}
\newcommand{\vr}[1]{\vtx{right}{#1}}
\newcommand{\vl}[1]{\vtx{left}{#1}}
\renewcommand{\v}{\vtx{above}{}}

\begin{document}
\noindent \textbf{Instructions}: Same rules as usual - turn in your work on separate sheets of paper.  You must justify all your answers for full credit.

\begin{questions}
\question[3] Prove that if $a \mid b$ and $b \mid c$, then $a \mid c$.

\begin{solution}
\begin{proof}
Suppose that $a \mid b$ and $b \mid c$.  This means that $b$ is a multiple of $a$, and that $c$ is a multiple of $b$.  In other words, $a = kb$ and $b = jc$ for some integers $k$ and $j$.  But then $a = kjc$, so $a$ is a multiple of $c$.  In other words, $a \mid c$.
\end{proof}
\end{solution}

\question[3] Prove that if $a \equiv b \pmod{n}$ and $c \equiv d \pmod{n}$, then $ac \equiv bd \pmod{n}$.  Hint: rewrite each congruence as an equation, then multiply the equations.

\begin{solution}
\begin{proof}
	Suppose $a \equiv b \pmod{n}$ and $c \equiv d \pmod{n}$.  That means $a = b + kn$ and $c = d + jn$ for some integers $k$ and $j$.  Then
	\[ac = (b+kn)(d+jn) = bd + bjn + dkn + kjn^2 = bd + (bj+dk+kjn)n\]
	In other words, $ac = bd + hn$ for some integer $h$, so $ac \equiv bd \pmod{n}$.
\end{proof}
\end{solution}

\question[6] Find the remainder when $42^{2013} + 2013^{42}$ is divided by 5.  Use modular arithmetic, and explain how you got your answer.

\begin{solution}
We want to reduce $42^{2013} + 2013^{42}$ modulo 5.  First, note that $42 \equiv 2 \pmod 5$ and $2013 \equiv 3 \pmod 5$, so we know:
\[42^{2013} + 2013^{42} \equiv 2^{2013} + 3^{42} \pmod 5\]
Let's simplify the $2^{2013}$ first.  Look for a power of 2 which is congruent to 1 modulo 5.  We could use $2^4 = 16$.  2013 is not a multiple of 4, so first pull out a 2:

\[2^{2013} = 2(2^4)^53 \equiv 2\cdot 1^52 \pmod{5}\]
so $2^{2013} \equiv 2 \pmod 5$.  Now for the $3^42$.  We could again look at $3^4 = 81$, or we could simply do $3^2 = 9 \equiv -1 \pmod 5$.  We get
\[3^{42} = 9^{21} \equiv (-1)^{21} \pmod 5\]
Of course $(-1)^21 = -1$.  

Putting this all together we get:

\[42^{2013} + 2013^{42} \equiv 2 - 1 \pmod 5\]
Thus the remainder is simply 1.

\end{solution}

\question[6] Solve the following congruences for $x$.  Give the general solution, as well as all solutions with $0 \le x \le 20$.
\begin{parts}
	\part $75x \equiv 41 \pmod{4}$
	\begin{solution}
		First reduce everything modulo 4.  We get,
		\[3x \equiv 1 \pmod{4}\]
		Then continue to add 4 to the right side (which is like adding 0 to both sides, since $0 \equiv 4 \pmod{4}$) until we get a multiple of 3.  So $1 + 4 = 5$, $5+4 = 9$ - bingo.
		\[3x \equiv 9 \pmod{4}\]
		Divide both sides by 3:
		\[x \equiv 3 \pmod{4}\]
		The general solution is $x = 3 + 4k$.  The solutions between 0 and 20 are:
		\[x \in \{3, 7, 11, 15, 19\} \]
	\end{solution} 
	
	\part $10x + 7 \equiv 3 \pmod{12}$
	\begin{solution}
		Subtract 7 from both sides:
		\[10x \equiv -4 \pmod{12}\]
		Of course $-4 \equiv 8 \equiv 20 \pmod{12}$.  Now when we divide both sides by 10, we must also reduce the modulus, since $10$ and $12$ have a common divisor.  We get:
		\[x \equiv 2 \pmod{6}\]
		So the general solution is $x = 2 + 6k$, and the set of solutions between 0 and 20 is:
		\[x \in \{2, 8, 14, 20\}\]
	\end{solution}
\end{parts}

\question[6] Solve the following Diophantine equations.  Describe all solutions.

\begin{parts}
	\part $7x + 46 y = 100$.
	\begin{solution}
		First convert to a congruence modulo 7:
		\[46 y \equiv 100 \pmod 7\]
		This reduces to 
		\[4y \equiv 2 \pmod 7\]
		Since $2 \equiv 16 \pmod 7$, if we divide by 4, we get
		\[y \equiv 4 \pmod 7\]
		so $y = 4 + 7k$.  Then $7x + 46(4+7k) = 100$, so $x = -12 + 46k$	
	\end{solution}
	
	\part $55x + 42y = 47$
	\begin{solution}
		We can rewrite this as
		\[55x \equiv 47 \pmod{42}\]
		This reduces to 
		\[13x \equiv 5 \pmod{42}\]
		Now we could keep adding 42 to 5 until we get a multiple of 13.  Or we can switch back to a Diophantine equation:
		\[13x = 5 + 42z\]
		Now reduce modulo 13:
		\[0 \equiv 5 + 42z \pmod{13}\]
		This is the same as:
		\[-3z \equiv 5 \pmod{13}\]
		which becomes $-3z \equiv 18 \pmod{13}$ reducing to $z \equiv -6 \pmod{13}$.  So $z = -6 + 13k$.  Now go back and find $x$ and $y$:
		\[13 x = 5 + 42(-6+13k)\]
		so $x = -19 + 42k$.  Then plug in to find $y$:
		\[55(-19+42k) + 42y = 47\]
		so $y = 26 - 55k$.

	\end{solution}
\end{parts}



\question[6] The hugely popular math-rock band {\em Fibonacci's Rabbits} recently performed on campus.  Tickets were \$21, unless you knew the password (the closed formula for the $n$th Fibonacci number), in which case tickets were only \$13.  All together, the revenues from ticket sales came to \$12855.  How many total tickets were sold, assuming the number of discounted tickets was as large as possible.

\begin{solution}
The total ticket sales revenue is some multiple of 34 plus some multiple of 21.  That is, we are trying to solve the Diophantine equation:
\[21x + 13y = 12855\]
Let's reduce this modulo 13.  We get
\[21x + 13y \equiv 12855 \pmod{13}\]
Simplifying gives:
\[8x \equiv 11 \pmod{13}\]
But $11 \equiv 24 \pmod{13}$, so
\[8x \equiv 24 \pmod{13}\]
\[x \equiv 3 \pmod{13}\]
So $x = 3 + 13k$.  We can then solve for $y$: 
\[21(3+13k) + 13y = 12855\]
\[y = 984 - 21k\]
We want to make $x$, the number of full price tickets, as small as possible, which in this case would be to take $k = 0$.  So $x = 3$ and $y = 984$.  Therefore the total number of tickets sold was $987$ (which, of course, is the 16th Fibonacci number).
\end{solution}


\end{questions}
\end{document}

