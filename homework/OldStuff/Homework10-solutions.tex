\documentclass[11pt]{exam}

\usepackage{amssymb, amsmath, amsthm, mathrsfs, multicol, graphicx} 
\usepackage{tikz}

\def\d{\displaystyle}
\def\?{\reflectbox{?}}
\def\b#1{\mathbf{#1}}
\def\f#1{\mathfrak #1}
\def\c#1{\mathcal #1}
\def\s#1{\mathscr #1}
\def\r#1{\mathrm{#1}}
\def\N{\mathbb N}
\def\Z{\mathbb Z}
\def\Q{\mathbb Q}
\def\R{\mathbb R}
\def\C{\mathbb C}
\def\F{\mathbb F}
\def\A{\mathbb A}
\def\X{\mathbb X}
\def\E{\mathbb E}
\def\O{\mathbb O}
\def\pow{\mathscr P}
\def\inv{^{-1}}
\def\nrml{\triangleleft}
\def\st{:}
\def\~{\widetilde}
\def\rem{\mathcal R}
\def\iff{\leftrightarrow}
\def\Iff{\Leftrightarrow}
\def\and{\wedge}
\def\And{\bigwedge}
\def\AAnd{\d\bigwedge\mkern-18mu\bigwedge}
\def\Vee{\bigvee}
\def\VVee{\d\Vee\mkern-18mu\Vee}
\def\imp{\rightarrow}
\def\Imp{\Rightarrow}
\def\Fi{\Leftarrow}

\def\={\equiv}
\def\var{\mbox{var}}
\def\mod{\mbox{Mod}}
\def\Th{\mbox{Th}}
\def\sat{\mbox{Sat}}
\def\con{\mbox{Con}}
\def\bmodels{=\joinrel\mathrel|}
\def\iffmodels{\bmodels\models}
\def\dbland{\bigwedge \!\!\bigwedge}
\def\dom{\mbox{dom}}
\def\rng{\mbox{range}}
\DeclareMathOperator{\wgt}{wgt}

\def\circleA{(-.5,0) circle (1)}
\def\circleAlabel{(-1.5,.6) node[above]{$A$}}
\def\circleB{(.5,0) circle (1)}
\def\circleBlabel{(1.5,.6) node[above]{$B$}}
\def\circleC{(0,-1) circle (1)}
\def\circleClabel{(.5,-2) node[right]{$C$}}
\def\twosetbox{(-2,-1.5) rectangle (2,1.5)}
\def\threesetbox{(-2,-2.5) rectangle (2,1.5)}


\def\bar{\overline}

%\pointname{pts}
\pointsinmargin
\marginpointname{pts}
\marginbonuspointname{pts-bns}
\addpoints
\pagestyle{headandfoot}
\printanswers

\firstpageheader{Math 228}{\bf Homework 10 Solutions}{Spring 2013}
%\firstpagefooter{}{}{\footnotesize \em More $\rightarrow$}
\runningfooter{}{}{}

\def\vertexsize{4pt}
\newcommand{\vtx}[2]{node[fill,circle,inner sep=0pt, minimum size=\vertexsize,label=#1:#2]{}}
\newcommand{\va}[1]{\vtx{above}{#1}}
\newcommand{\vb}[1]{\vtx{below}{#1}}
\newcommand{\vr}[1]{\vtx{right}{#1}}
\newcommand{\vl}[1]{\vtx{left}{#1}}
\renewcommand{\v}{\vtx{above}{}}

\begin{document}
%\noindent \textbf{Instructions}: Same rules as usual - turn in your work on separate sheets of paper.  You must justify all your answers for full credit.

\begin{questions}

\question[4] Find a generating function for the sequence $3, 4, 6, 10, 18, 34, 66, \ldots$.  Hint: find the generating function for the difference between terms. Explain why your answer is correct.
\begin{solution}
We use differencing:
  \begin{align*}
    A & = 3 + 4x + 6x^2 + 10x^3 + 18x^4 + \cdots \\
    \underline{ - ~~xA } & \underline{ = ~~~~~~ 3x + 4x^2 + 6x^3 + 10x^4 + \cdots }\\
    (1-x)A & = 3 + x + 2x^2 + 4x^3 + 8x^4 + \cdots
  \end{align*}
  Since the generating function for $x + 2x^2 + 4x^3 + 8x^4 + \cdots$ is $\dfrac{x}{1-2x}$ we have
  \[A = \frac{3}{1-x} + \frac{x}{(1-2x)(1-x)}\]
\end{solution}


\question[4] Find the generating function for the sequence $1, 4, 11, 34, 101, 304, \ldots$ using the fact that the sequence is recursively defined by $a_n = 2 a_{n-1} + 3a_{n-2}$ with $a_0 = 1$ and $a_1 = 4$.  

\begin{solution}
  \begin{align*}
    A & = 1 + 4x + 11x^2 + 34x^3 + \cdots \\
    2xA & = ~~~~~ 2x + 8x^2 + 22x^3 + \cdots \\
    \underline{- ~~ 3x^2A } & \underline{ = ~~~~~~~~~~~~~3x^2 + 12x^3 + \cdots} \\
    (1-2x-3x^2)A & = 1 + 2x
  \end{align*}
(The other terms will all equal 0 by the recurrence relation.)  Thus
\[A = \frac{1+2x}{1-2x-3x^2}\]
\end{solution}




\question[4] Zombie Euler and Zombie Cauchy - two famous zombie mathematicians - have just signed up for Twitter accounts.  After one day, Zombie Cauchy has more followers than Zombie Euler.  Each day after that, the number of new followers of Zombie Cauchy is exactly the same as the number of new followers of Zombie Euler (and neither lose any followers).  Explain how a proof by mathematical induction can show that on every day after the first day, Zombie Cauchy will have more followers than Zombie Euler.  That is, explain what the base case and inductive case are, and why they together prove that Zombie Cauchy will have more followers on the 4th day.

\begin{solution}
  The idea here is that because we know Zombie Cauchy starts ahead, and each day increases by the same amount as Zombie Euler, he will always be ahead.
  
   The base case is that Zombie Cauchy has more followers than Zombie Euler on day 1.  We know this is true because it says so in the problem.
    
    The inductive case is that {\em if} Zombie Cauchy has more followers on day $k$, then he will still have more followers on day $k+1$.  We know this is true because each day, the Zombies receive an equal number of new followers. 
    
    Together, the base case and inductive case show that on the 4th day, Zombie Cauchy will be ahead: he is ahead on day 1, and because on day 1 he is ahead, by the inductive case he will also be ahead on day 2.  By the inductive case again, he will be ahead on day 3 since he is ahead on day 2, and since he is ahead on day 3, he will also be ahead on day 4.  Of course we could keep doing this up to any day.

\end{solution}


\uplevel{{\bf Special Induction Instructions}: For the rest of the homework problems, you should first give a rough sketch of the argument (i.e., say {\em why} induction will work in this case) and then also give a formal proof by induction (starting with, ``Let $P(n)$ be the statement\ldots'').}


\question[6] Find the largest number of points which a football team cannot get exactly using just 3-point field goals and 7-point touchdowns (ignore the possibilities of safeties, missed extra points, and two point conversions).  Prove your answer is correct by mathematical induction.

\begin{solution}
  First note that it is impossible to make 11 points - if only field goals are made, the points must be a multiple of 3, if 1 touchdown is made, the possible point totals are 7, 10, 13, \ldots and two touchdowns are already too much.
  
  We will prove that 11 is the largest number of points which cannot be made.  In other words, any number of points greater than or equal to 12 can be made.
  
  \begin{proof}
    Let $P(n)$ be the statement ``it is possible to make $n$ points using touchdowns and field goals.''  We will prove $P(n)$ is true for all $n \ge 12$.
    
    First the base case: You can make 12 points with 4 field goals, so $P(12)$ is true.
    
    Now the inductive case: Assume $P(k)$ is true for some fixed $k \ge 12$.  That is, it is possible to make $k$ points.  Since $k \ge 12$, we must have made the $k$ points using either at least 2 field goals or at least 2 touchdowns, or both (because if we used just one of each we would have only 10 points).  Now if the $k$ points were accomplished with 2 (or more) field goals, then replace 2 field goals with 1 touchdown.  This increases to point total by 1, giving $k + 1$ points.  On the other hand, if the $k$ points were accomplished with $2$ (or more) touchdowns, replace 2 touchdowns with 5 field goals, again increasing the point total by 1, giving $k+1$ points.  Using one of these two substitutions, we can make $k+1$ points, so $P(k+1)$ is true, establishing the inductive case.
    
    Therefore by the principle of mathematical induction, $P(n)$ is true for all $n \ge 12$.
  \end{proof}
\end{solution}



\question[6] Prove, by mathematical induction, that $F_0 + F_1 + F_2 + \cdots + F_{n} = F_{n+2} - 1$, where $F_n$ is the $n$th Fibonacci number ($F_0 = 0$, $F_1 = 1$ and $F_n = F_{n-1} + F_{n-2}$).
\begin{solution}
 This is saying that if we add up the first $n$ Fibonacci numbers, we will get another Fibonacci number (specifically, the $(n+2)$th one).  Induction is a good idea here because it will be easy to just add one more Fibonacci number to the sum we already have.  If we already have $F_{k+2}$ and we add $F_{k+1}$ we can use the recurrence relation to simplify this, becoming $F_{k+3}$.  
 
  \begin{proof}
    Let $P(n)$ be the statement $F_0 + F_1 + F_2 + \cdots + F_n = F_{n+2} - 1$.  We will prove that $P(n)$ is true for all $n \ge 0$.  
    
    Base case: $P(0)$ states that $F_0 = F_2 - 1$, which is true because $F_0 = 0$ and $F_2 = 1$.
    
    Inductive case:  Assume $P(k)$ is true for an arbitrary fixed $k \ge 0$.  That is, \[F_0 + F_1 + F_2 + \cdots + F_k = F_{k+2} - 1\]
    We must prove that $P(k+1)$ is true as well (i.e. that $F_0 + F_1 + \cdots +F_{k+1} = F_{k+3} - 1$).  Start with the left hand side:
    \begin{align*}
      F_0 + F_1 + F_2 + \cdots + F_k + F_{k+1} & = F_{k+2} - 1 + F_{k+1} & \mbox{ by the inductive hypothesis}\\
      & = F_{k+3} - 1 & \mbox{ by the definition of the Fibonacci numbers}
    \end{align*}
    Thus $P(k+1)$ is true.
    
    Therefore by the principle of mathematical induction, $P(n)$ is true for all $n \ge 0$.
  \end{proof}

\end{solution}


\question[6] Prove that every natural number is either a Fibonacci number or can be written as the sum of distinct Fibonacci numbers.  Use strong induction.  Hint: To write 32 as the sum of distinct Fibonacci numbers, you can first look for the largest Fibonacci number less than 32.  In this case, that's 21, so if you can write 11 as the sum of distinct Fibonacci numbers, you will be done.

\begin{solution}
	The idea: for any number $n$ which is not already a Fibonacci number, you can subtract off the largest Fibonacci number less than $n$, which reduces you to an earlier case.  The inductive hypothesis will tell you how to write the difference correctly.
	
	\begin{proof}
		Let $P(n)$ be the statement, ``$n$ is either a Fibonacci number or can be written as the sum of distinct Fibonacci numbers.''  We will prove $P(n)$ is true for all $n \in \N$.
		
		Base case: $P(0)$ is true, because $0$ is a Fibonacci number.
		
		Inductive case: Consider an arbitrary natural number $n$, and assume $P(k)$ is true for all $k < n$.  We will prove $P(n)$ is true as well.  If $n$ is a Fibonacci number, then we are done.  If not, then let $x$ be the largest Fibonacci number less than $n$.  Consider $n-x$.  This is a number less than $n$, so $P(n-x)$ will be true.  Thus $n-x$ is either a Fibonacci number or can be written as the sum of distinct Fibonacci numbers.  Note that in either case, $n-x < x$, so it is impossible for $x$ to be used twice. After decomposing $x$, we will have written $n$ as the sum of distinct Fibonacci numbers.
		
		Therefore, by strong induction, $P(n)$ is true for all $n \in \N$.
	\end{proof} 
\end{solution}
\end{questions}
\end{document}

