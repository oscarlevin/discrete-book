\documentclass[11pt]{exam}

\usepackage{amssymb, amsmath, amsthm, mathrsfs, multicol, graphicx} 
\usepackage{tikz}

\def\d{\displaystyle}
\def\?{\reflectbox{?}}
\def\b#1{\mathbf{#1}}
\def\f#1{\mathfrak #1}
\def\c#1{\mathcal #1}
\def\s#1{\mathscr #1}
\def\r#1{\mathrm{#1}}
\def\N{\mathbb N}
\def\Z{\mathbb Z}
\def\Q{\mathbb Q}
\def\R{\mathbb R}
\def\C{\mathbb C}
\def\F{\mathbb F}
\def\A{\mathbb A}
\def\X{\mathbb X}
\def\E{\mathbb E}
\def\O{\mathbb O}
\def\pow{\mathscr P}
\def\inv{^{-1}}
\def\nrml{\triangleleft}
\def\st{:}
\def\~{\widetilde}
\def\rem{\mathcal R}
\def\iff{\leftrightarrow}
\def\Iff{\Leftrightarrow}
\def\and{\wedge}
\def\And{\bigwedge}
\def\AAnd{\d\bigwedge\mkern-18mu\bigwedge}
\def\Vee{\bigvee}
\def\VVee{\d\Vee\mkern-18mu\Vee}
\def\imp{\rightarrow}
\def\Imp{\Rightarrow}
\def\Fi{\Leftarrow}

\def\={\equiv}
\def\var{\mbox{var}}
\def\mod{\mbox{Mod}}
\def\Th{\mbox{Th}}
\def\sat{\mbox{Sat}}
\def\con{\mbox{Con}}
\def\bmodels{=\joinrel\mathrel|}
\def\iffmodels{\bmodels\models}
\def\dbland{\bigwedge \!\!\bigwedge}
\def\dom{\mbox{dom}}
\def\rng{\mbox{range}}
\DeclareMathOperator{\wgt}{wgt}


\def\bar{\overline}

%\pointname{pts}
\pointsinmargin
\marginpointname{pts}
\addpoints
\pagestyle{head}
\printanswers

\firstpageheader{Math 228}{\bf Homework 1\\Solutions}{Spring 2013}


\begin{document}

\begin{questions}
  \question While walking through a fictional forest, you encounter three trolls.  Each is either a {\em knight}, who always tells the truth, or a {\em knave}, who always lies.  The trolls will not let you pass until you correctly identify each as either a knight or a knave.  Each troll makes a single statement:
  \begin{itemize}
   \item[] Troll 1: Only one of us is a knave.
   \item[] Troll 2: No, only one of us is a knight.
   \item[] Troll 3: We are all knaves.
  \end{itemize}
  Which troll is which?
  
  \begin{solution}
    We can conclude that Troll 1 and Troll 3 are both knaves, while troll 2 is a knight.
    
    \begin{proof}
      First consider Troll 3. Suppose he was actually a knight.  Then his statement, that all three trolls are knaves, would be true.  That would, in particular mean that he is a knave.  But this is impossible since we assumed he was a knight.  Therefore, since our assumption led to a contradiction, our assumption must be wrong.  In other words, we know that Troll 3 is a knave.
      
      We also know, since troll 3 is a knave, that at least one of the trolls in the group must be a knight (it is false that all the trolls are knaves).  Further, since the statements of Troll 1 and Troll 2 cannot simultaneously be true, we know that exactly one of them must be telling the truth.
      
      We have concluded that among the three trolls, there is only one knight.  This is Troll 2's statement, so we see that Troll 2 is telling the truth, so is the knight.  Troll 1 must therefore be a knave.
    \end{proof}

  \end{solution}

\end{questions}




\end{document}


