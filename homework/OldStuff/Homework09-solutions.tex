\documentclass[11pt]{exam}

\usepackage{amssymb, amsmath, amsthm, mathrsfs, multicol, graphicx} 
\usepackage{tikz}

\def\d{\displaystyle}
\def\?{\reflectbox{?}}
\def\b#1{\mathbf{#1}}
\def\f#1{\mathfrak #1}
\def\c#1{\mathcal #1}
\def\s#1{\mathscr #1}
\def\r#1{\mathrm{#1}}
\def\N{\mathbb N}
\def\Z{\mathbb Z}
\def\Q{\mathbb Q}
\def\R{\mathbb R}
\def\C{\mathbb C}
\def\F{\mathbb F}
\def\A{\mathbb A}
\def\X{\mathbb X}
\def\E{\mathbb E}
\def\O{\mathbb O}
\def\pow{\mathscr P}
\def\inv{^{-1}}
\def\nrml{\triangleleft}
\def\st{:}
\def\~{\widetilde}
\def\rem{\mathcal R}
\def\iff{\leftrightarrow}
\def\Iff{\Leftrightarrow}
\def\and{\wedge}
\def\And{\bigwedge}
\def\AAnd{\d\bigwedge\mkern-18mu\bigwedge}
\def\Vee{\bigvee}
\def\VVee{\d\Vee\mkern-18mu\Vee}
\def\imp{\rightarrow}
\def\Imp{\Rightarrow}
\def\Fi{\Leftarrow}

\def\={\equiv}
\def\var{\mbox{var}}
\def\mod{\mbox{Mod}}
\def\Th{\mbox{Th}}
\def\sat{\mbox{Sat}}
\def\con{\mbox{Con}}
\def\bmodels{=\joinrel\mathrel|}
\def\iffmodels{\bmodels\models}
\def\dbland{\bigwedge \!\!\bigwedge}
\def\dom{\mbox{dom}}
\def\rng{\mbox{range}}
\DeclareMathOperator{\wgt}{wgt}

\def\circleA{(-.5,0) circle (1)}
\def\circleAlabel{(-1.5,.6) node[above]{$A$}}
\def\circleB{(.5,0) circle (1)}
\def\circleBlabel{(1.5,.6) node[above]{$B$}}
\def\circleC{(0,-1) circle (1)}
\def\circleClabel{(.5,-2) node[right]{$C$}}
\def\twosetbox{(-2,-1.5) rectangle (2,1.5)}
\def\threesetbox{(-2,-2.5) rectangle (2,1.5)}


\def\bar{\overline}

%\pointname{pts}
\pointsinmargin
\marginpointname{pts}
\marginbonuspointname{pts-bns}
\addpoints
\pagestyle{headandfoot}
\printanswers

\firstpageheader{Math 228}{\bf Homework 9 Solutions}{Spring 2013}
%\firstpagefooter{}{}{\footnotesize \em More $\rightarrow$}
\runningfooter{}{}{}

\def\vertexsize{4pt}
\newcommand{\vtx}[2]{node[fill,circle,inner sep=0pt, minimum size=\vertexsize,label=#1:#2]{}}
\newcommand{\va}[1]{\vtx{above}{#1}}
\newcommand{\vb}[1]{\vtx{below}{#1}}
\newcommand{\vr}[1]{\vtx{right}{#1}}
\newcommand{\vl}[1]{\vtx{left}{#1}}
\renewcommand{\v}{\vtx{above}{}}

\begin{document}
\noindent \textbf{Instructions}: Same rules as usual - turn in your work on separate sheets of paper.  You must justify all your answers for full credit.

\begin{questions}

\question[6] Solve the recurrence relation $a_n = a_{n-1} + 3$ using:

\begin{parts}
\part Telescoping.  Show your work.
\begin{solution}
  \begin{align*}
    a_1 - a_0 & = 3\\
    a_2 - a_1 & = 3 \\
    a_3 - a_2 & = 3 \\
    \vdots\qquad & \quad ~ \vdots \\
    \underline{ +~~ a_n - a_{n-1}} & \underline{\, = 3_{}}\\
    a_n - a_0 & = 3n
  \end{align*}
  Thus the solution is $a_n = 3n + a_0$.
\end{solution}

\part Iteration.  Show you work.
\begin{solution}
 \begin{align*}
 	a_1 & = a_0 + 3 \\
 	a_2 & = a_1 + 3 = (a_0 + 3) + 3 \\
 	a_3 & = a_2 + 3 = (a_0 + 3 + 3) + 3\\
 	a_4 & = a_3 + 3 = (a_0 + 3 + 3 + 3) + 3\\
 	\vdots & = ~~~ \vdots \\
 	a_n & = a_{n-1} + 3 = (a_0 + 3 + 3 + \cdots + 3) + 3
 \end{align*}
	Every iteration just adds another 3.  So the $n$th iteration takes $a_0$ and adds $n$ 3's.  Thus $a_n = a_0 + 3n$.
\end{solution}

\end{parts}



\newpage
\question Let $a_n$ be the number of  $1 \times n$ tile designs can you make using $1 \times 1$ tiles available in 4 colors and $1 \times 2$ tiles available in 5 colors.
\begin{parts}
  \part[3] First, find a recurrence relation to describe the problem.  Explain why the recurrence relation is correct (in the context of the problem).
  \begin{solution}
    $a_n = 4a_{n-1} + 5a_{n-2}$.  Each path of length $n$ must either start with one of the 4 $1\times 1$ tiles, in each case there are then $a_{n-1}$ ways to finish the path, or start with one of the 5 $1\times 2$ tiles, in each case there are then $a_{n-2}$ ways to finish the path.
  \end{solution}

  \part[2] Write out the first 6 terms of the sequence $a_1, a_2, \ldots$.
  \begin{solution}
    4, 21, 104, 521, 2604, 13021
  \end{solution}

  \part[3] Solve the recurrence relation.  That is, find a closed formula for $a_n$.
  \begin{solution}
    The characteristic equation is $x^2 - 4x - 5 = 0$ so the characteristic roots are $x = 5$ and $x = -1$.  Therefore the general solution is 
    \[a_n = a 5^n + b (-1)^n\]
    We solve for $a$ and $b$ using the fact that $a_1 = 4$ and $a_2 = 21$.  We get $a = \frac{5}{6}$ and $b = \frac{1}{6}$.  Therefore the solution is
    \[a_n = \frac{5}{6} 5^n + \frac{1}{6}(-1)^n\]
  \end{solution}

\end{parts}


\question[6] Consider the recurrence relation $a_n = 4a_{n-1} - 4a_{n-2}$.
\begin{parts}
  \part Find the general solution to the recurrence relation (beware the repeated root).
  \begin{solution}
    The characteristic polynomial is $x^2 - 4x + 4$ which factors as $(x -2)^2$, so the only characteristic root is $x = 2$.  Thus the general solution is
    \[a_n = a2^n + bn2^n\]
  \end{solution}

  \part Find the solution when $a_0 = 1$ and $a_1 = 2$.
  \begin{solution}
    Since $1 = a2^0 + b\cdot 0 \cdot 2^0$ have have $a = 1$.  Then $2 = 2^1 + b 2^1$ so $b = 0$.  We have the solution 
    \[a_n = 2^n\]
  \end{solution}

  \part Find the solution when $a_0 = 1$ and $a_1 = 8$.
  \begin{solution}
    Again, we have $a = 1$.  Now when we plug in $n = 1$ we bet $8 = 2 + 2b$ so $b = 3$.  The solution:
    \[a_n = 2^n + 3n2^n\]
  \end{solution}

\end{parts}



\question[10] Write down first 6 or so terms of the sequences generated by each of the following generating functions, using the fact that $\frac{1}{1-x}$ generates $1,1,1,1,\ldots$.  In each case, briefly explain how you arrived at your answer.
\begin{parts}
  \part $\d\frac{5}{1-x}$
  \begin{solution}
    $5,5,5,5,5,5,\ldots$.  We multiplied the power series by 5 - each term got multiplied by 5.
  \end{solution}

  \part $\dfrac{1}{1+2x}$
  \begin{solution}
    $1, -2, 4, -8, 16, -32,\ldots$.  We substituted $-2x$ in for $x$.  This gives the power series $1 + (-2x) + (-2x)^2 + (-2x)^3 + \cdots$.  Thus we get the powers of $-2$ as coefficients.
  \end{solution}

  \part $\d\frac{1}{(1-x^2)^2}$
  \begin{solution}
    $1, 0, 2, 0, 3, 0, 4, 0, 5, \ldots$.  We know $\frac{1}{(1-x)^2}$ generates $1,2,3,4,\ldots$ (by taking the derivative of $\frac{1}{1-x}$ for example).  Then substituting $x^2$ for $x$ spaces out the sequence - we have the same coefficients, but now only on even powers of $x$ - the coefficients of odd powers of $x$ are odd.
  \end{solution}

  \part$\d\frac{1}{1+2x}+\frac{5}{1-x}$
  \begin{solution}
    $6, 3, 9, -3, 21, -27, \ldots$.  Here we just added the sequences from above, term by term.
  \end{solution}

  \part$\d\frac{1}{1+2x}\cdot\frac{5}{1-x}$
  \begin{solution}
    $5, -5, 15, -25, 55, -105,\ldots$.  When multiplying two generating functions, the $n$th term is the sum of the first $n$ terms of one, each multiplied by the first $n$ terms in reverse order of the other.  So here we get this sequence like this: $(1\cdot 5), (1\cdot 5 + (-2)\cdot 5), (1 \cdot 5 + (-2)\cdot 5 + 4 \cdot 5)$ and so on.  
    
    Alternatively, multiplying a generating function by $\frac{1}{1-x}$ gives the sequence of partial sums.  Then multiply each term by $5$.
  \end{solution}

\end{parts}

\end{questions}
\end{document}

