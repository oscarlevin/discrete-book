\documentclass[11pt]{exam}

\usepackage{amssymb, amsmath, amsthm, mathrsfs, multicol, graphicx} 
\usepackage{tikz}

\def\d{\displaystyle}
\def\?{\reflectbox{?}}
\def\b#1{\mathbf{#1}}
\def\f#1{\mathfrak #1}
\def\c#1{\mathcal #1}
\def\s#1{\mathscr #1}
\def\r#1{\mathrm{#1}}
\def\N{\mathbb N}
\def\Z{\mathbb Z}
\def\Q{\mathbb Q}
\def\R{\mathbb R}
\def\C{\mathbb C}
\def\F{\mathbb F}
\def\A{\mathbb A}
\def\X{\mathbb X}
\def\E{\mathbb E}
\def\O{\mathbb O}
\def\pow{\mathscr P}
\def\inv{^{-1}}
\def\nrml{\triangleleft}
\def\st{:}
\def\~{\widetilde}
\def\rem{\mathcal R}
\def\iff{\leftrightarrow}
\def\Iff{\Leftrightarrow}
\def\and{\wedge}
\def\And{\bigwedge}
\def\AAnd{\d\bigwedge\mkern-18mu\bigwedge}
\def\Vee{\bigvee}
\def\VVee{\d\Vee\mkern-18mu\Vee}
\def\imp{\rightarrow}
\def\Imp{\Rightarrow}
\def\Fi{\Leftarrow}

\def\={\equiv}
\def\var{\mbox{var}}
\def\mod{\mbox{Mod}}
\def\Th{\mbox{Th}}
\def\sat{\mbox{Sat}}
\def\con{\mbox{Con}}
\def\bmodels{=\joinrel\mathrel|}
\def\iffmodels{\bmodels\models}
\def\dbland{\bigwedge \!\!\bigwedge}
\def\dom{\mbox{dom}}
\def\rng{\mbox{range}}
\DeclareMathOperator{\wgt}{wgt}

\def\circleA{(-.5,0) circle (1)}
\def\circleAlabel{(-1.5,.6) node[above]{$A$}}
\def\circleB{(.5,0) circle (1)}
\def\circleBlabel{(1.5,.6) node[above]{$B$}}
\def\circleC{(0,-1) circle (1)}
\def\circleClabel{(.5,-2) node[right]{$C$}}
\def\twosetbox{(-2,-1.5) rectangle (2,1.5)}
\def\threesetbox{(-2,-2.5) rectangle (2,1.5)}


\def\bar{\overline}

%\pointname{pts}
\pointsinmargin
\marginpointname{pts}
\marginbonuspointname{pts-bns}
\addpoints
\pagestyle{head}
%\printanswers

\firstpageheader{Math 228}{\bf Homework 7}{} %Due: Wed, March 13}

\def\vertexsize{4pt}
\newcommand{\vtx}[2]{node[fill,circle,inner sep=0pt, minimum size=\vertexsize,label=#1:#2]{}}
\newcommand{\va}[1]{\vtx{above}{#1}}
\newcommand{\vb}[1]{\vtx{below}{#1}}
\newcommand{\vr}[1]{\vtx{right}{#1}}
\newcommand{\vl}[1]{\vtx{left}{#1}}
\renewcommand{\v}{\vtx{above}{}}

\begin{document}
\noindent \textbf{Instructions}: Same rules as usual - turn in your work on separate sheets of paper.  You must justify all your answers for full credit.

\begin{questions}
\question After a late night of math studying, you and your friends decide to go to your favorite tax-free fast food Mexican restaurant, {\em Burrito Chime}.  You decide to order off of the dollar menu, which has 7 items.  Your group has \$16 to spend (and will spend all of it). 
\begin{parts}
  \part[2] How many different orders are possible?  Explain. (The {\em order} in which the order is place does not matter - just which and how many of each item that is ordered.)
  \begin{solution}
    $\d{22 \choose 6}$ - there are 16 stars and 6 bars.
  \end{solution}

  \part[2] How many different orders are possible if you want to get at least one of each item? Explain.
  \begin{solution}
    $\d{15 \choose 6}$ - buy one of each item, using \$7.  This leaves you \$11 to distribute to the 7 items, so 11 stars and 6 bars.
  \end{solution}

  \part[4] How many different orders are possible if you don't get more than 4 of any one item?  Explain. Hint: get rid of the bad orders using PIE.
  \begin{solution}
    \[{22 \choose 6} - \left[{7 \choose 1}{17 \choose 6} - {7 \choose 2}{12 \choose 6} + {7 \choose 3}{7 \choose 6} \right]\]
  \end{solution}

\end{parts}


\question[6] Consider functions $f:\{1,2,3,4,5\} \to \{0,1,2,\ldots,9\}$.  
\begin{parts}
	\part How many of these functions are strictly increasing?  Explain.  (A function is strictly increasing provided if $a < b$, then $f(a) < f(b)$.)
	\begin{solution}
		${10 \choose 5}$.  Note that a strictly increasing function is automatically injective.  So the five outputs must all be different.  So let's first pick which five outputs we will use: there are ${10 \choose 5}$ ways to do this.  Now how many ways are there to assign those outputs to the inputs $1$ through 5?  Only one way, since there is only one way to arrange numbers in increasing order.		
	\end{solution}
	\part How many of the functions are non-decreasing?  Explain.  (A function is non-decreasing provided if $a < b$, then $f(a) \le f(b)$.)
	\begin{solution}
		${14 \choose 5}$.  This is in fact a stars and bars problem.  The stars are the 5 inputs and the bars are the 9 spots between the 10 possible outputs.  Think of it this way - we will specify $f(1)$, then $f(2)$, then $f(3)$, and so on in that order.  Start with the possible output 0.  We can use it as the output of $f(1)$, or we can switch to 1 as a potential output.  Say we put $f(1) = 1$.  Now we are at 1 (can't go back to 0).  Should $f(2) = 1$?  If yes, then we are putting down another star.  If no, put down a bar and switch to 2.  Maybe you switch to 3, then assign $f(2) = 3$ and $f(3) = 3$ (two more stars) before switching to 4 as a possible output.  And so on.  
	\end{solution}
\end{parts}

\question[6] Rephrase each of the following counting problems in terms of counting functions.  That is, describe a collection of functions so that each function corresponds to an outcome of the counting problem.  Specifically, say what the domain and codomain of the functions are in terms of the counting problem, and say whether you are counting all functions, injective functions, or surjective functions.
\begin{parts}
	\part The 10 members of Math Club all decide to each pick one of the 15 math club meetings to give a presentation (each of which will take the entire meeting time).  How many ways can this happen?
	\begin{solution}
		We are assigning club members to meetings (some meetings will be left out - so the meetings cannot be the domain).  The set of 10 club members is the domain, and the set of 15 meetings is the codomain.  Since no two members can pick the same meeting, we are asking for the number of injective functions (which in this case is $P(15,10)$).
	\end{solution}
	\part Over the next seven days, you plan to finish a box of 24 different types of chocolates, having at least one chocolate each day.  How many ways can this happen?
	\begin{solution}
		Let the set of chocolates be the domain and the set of days be the codomain.  This works because each chocolate gets eaten on exactly one day.  Since we want a chocolate on each day, we are asking for the number of surjective functions (which you would use PIE to count).
	\end{solution}
	\part At the end of the semester, you assign grades to each student in your class.  There are 12 possible grades, and you have 30 students in your class.  How many ways are there to do this?
	\begin{solution}
		In this case, every student gets a grade, but not every grade must be assigned to a student.  So the domain is the set of students and the codomain is the set of grades.  We are looking just for the number of functions - none of them will be injective because the domain is larger than the codomain, and they might not be surjective (if for example everyone gets an A).  
	\end{solution}
\end{parts}

\question The Grinch sneaks into a room with 6 Christmas presents to 6 different people.  He proceeds to switch the name-labels on the presents.  How many ways could he do this if:
\begin{parts}
  \part[4] No present is allowed to end up with its original label?  Explain what each term in your answer represents.
  \begin{solution}
    \[6! - \left[{6 \choose 1}5! - {6 \choose 2}4! + {6 \choose 3}3! - {6 \choose 4}2! + {6 \choose 5}1! - {6 \choose 6}0!\right]\]
  \end{solution}

  \part[4] Exactly 2 presents keep their original labels? Explain.
  \begin{solution}
    \[{6 \choose 2}\left(4! - \left[{4\choose 1}3! - {4 \choose 2}2! + {4 \choose 3}1! - {4 \choose 4}0!\right]\right)\]
  \end{solution}

  \part[2] Exactly 5 presents keep their original labels? Explain.
  \begin{solution}
    0.  Once 5 presents have their original label, there is only one present left and only one label left, so the 6th present must get its own label.
  \end{solution}

\end{parts}



\end{questions}




\end{document}


