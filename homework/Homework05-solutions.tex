\documentclass[11pt]{exam}

\usepackage{amssymb, amsmath, amsthm, mathrsfs, multicol, graphicx} 
\usepackage{tikz}

\def\d{\displaystyle}
\def\?{\reflectbox{?}}
\def\b#1{\mathbf{#1}}
\def\f#1{\mathfrak #1}
\def\c#1{\mathcal #1}
\def\s#1{\mathscr #1}
\def\r#1{\mathrm{#1}}
\def\N{\mathbb N}
\def\Z{\mathbb Z}
\def\Q{\mathbb Q}
\def\R{\mathbb R}
\def\C{\mathbb C}
\def\F{\mathbb F}
\def\A{\mathbb A}
\def\X{\mathbb X}
\def\E{\mathbb E}
\def\O{\mathbb O}
\def\pow{\mathscr P}
\def\inv{^{-1}}
\def\nrml{\triangleleft}
\def\st{:}
\def\~{\widetilde}
\def\rem{\mathcal R}
\def\iff{\leftrightarrow}
\def\Iff{\Leftrightarrow}
\def\and{\wedge}
\def\And{\bigwedge}
\def\AAnd{\d\bigwedge\mkern-18mu\bigwedge}
\def\Vee{\bigvee}
\def\VVee{\d\Vee\mkern-18mu\Vee}
\def\imp{\rightarrow}
\def\Imp{\Rightarrow}
\def\Fi{\Leftarrow}

\def\={\equiv}
\def\var{\mbox{var}}
\def\mod{\mbox{Mod}}
\def\Th{\mbox{Th}}
\def\sat{\mbox{Sat}}
\def\con{\mbox{Con}}
\def\bmodels{=\joinrel\mathrel|}
\def\iffmodels{\bmodels\models}
\def\dbland{\bigwedge \!\!\bigwedge}
\def\dom{\mbox{dom}}
\def\rng{\mbox{range}}
\DeclareMathOperator{\wgt}{wgt}

\def\circleA{(-.5,0) circle (1)}
\def\circleAlabel{(-1.5,.6) node[above]{$A$}}
\def\circleB{(.5,0) circle (1)}
\def\circleBlabel{(1.5,.6) node[above]{$B$}}
\def\circleC{(0,-1) circle (1)}
\def\circleClabel{(.5,-2) node[right]{$C$}}
\def\twosetbox{(-2,-1.5) rectangle (2,1.5)}
\def\threesetbox{(-2,-2.5) rectangle (2,1.5)}


\def\bar{\overline}

%\pointname{pts}
\pointsinmargin
\marginpointname{pts}
\marginbonuspointname{pts-bns}
\addpoints
\pagestyle{head}
\printanswers

\firstpageheader{Math 228}{\bf Homework 5}{Due: Wed, Feb 27}

\def\vertexsize{4pt}
\newcommand{\vtx}[2]{node[fill,circle,inner sep=0pt, minimum size=\vertexsize,label=#1:#2]{}}
\newcommand{\va}[1]{\vtx{above}{#1}}
\newcommand{\vb}[1]{\vtx{below}{#1}}
\newcommand{\vr}[1]{\vtx{right}{#1}}
\newcommand{\vl}[1]{\vtx{left}{#1}}
\renewcommand{\v}{\vtx{above}{}}

\begin{document}
%\noindent \textbf{Instructions}: Same rules as usual - turn in your work on separate sheets of paper.  You must justify all your answers for full credit.

\begin{questions}



\question[4] In a recent survey, 30 students reported whether they liked their potatoes Mashed, French-fried, or Twice-baked. 15 liked them mashed, 20 liked French fries, and 9 liked twice baked potatoes. Additionally, 12 students liked both mashed and fried potatoes, 5 liked French fries and twice baked potatoes, 6 liked mashed and baked, and 3 liked all three styles. How many students
{\em hate} potatoes?  Explain why your answer is correct.

	\begin{solution}
	  Using the principle of inclusion/exclusion, the number of students who like their potatoes in at least one of the ways described is \[15 + 20 + 9 - 12 - 5 - 6 + 3 = 24.\]  Therefore there are $30-24 = 6$ students who do not like potatoes.  You can also do this problem with a Venn diagram.
	\end{solution}


\question[4] For how many three digit numbers (100 to 999) is the {\em sum of the digits} even? (For example, $343$ has an even sum of digits: $3+4+3 = 10$ which is even.)  Explain.

\begin{solution}
  There are multiple ways to do this.
  \begin{enumerate}
    \item An even sum can occur in 4 ways: EEE, EOO, OEO, and OOE.  There are $4 \cdot 5 \cdot 5$ ways to build numbers of the first two types (there are only 4 choices for a starting even number - it cannot be 0) and $5 \cdot 5 \cdot 5$ ways to build the second two types.  This gives a total of 450 numbers.
    \item To build a 3 digit number with an even sum, you can choose any of 9 digits for the first digit, any of 10 digits for the second digit.  Then the last digit must either be even (if the sum of the first two digits are even) or odd (if the sum of the first two digits are odd).  Luckily there are the same number of even last digits and odd last digits - 5.  So there are a total of $9 \cdot 10 \cdot 5 = 450$ numbers with an even sum of digits.
  \end{enumerate}
\end{solution}


\question[8] Let $A = \{1,2,3,\ldots,9\}$.  
\begin{parts}
	\part How many subsets of $A$ are there?  That is, find $|\pow(A)|$. Explain.
	\begin{solution}
		There are $512$ subsets.  This is $2^9$, which makes sense because we are deciding yes or no on whether to include each element of $A$ in the subset.
	\end{solution}
	\part How many subsets of $A$ contain exactly 5 elements?  Explain.
		\begin{solution}
			Of the nine elements in $A$, we must choose five of them to be in the subset.  So ${9 \choose 5} = 126$.
		\end{solution}
		
  \part How many subsets of $A$ contain only even numbers? Explain.
  \begin{solution}
    For each of the 9 elements from $A$, we must decide yes or no on whether to include them in the subset.  However, for the odd numbers, we only have one choice: no.  So there are only 4 elements we have two choices for, so the answer is $2^4 = 16$.  (Note, if you wish to exclude the empty set - it does not contain odd numbers, but no evens either - then you could subtract 1).
  \end{solution}

  \part How many subsets of $A$ contain an even number of elements? Explain.
  \begin{solution}
    Count the number of subsets with each possible even cardinality:
    \[{9 \choose 0} + {9 \choose 2} + {9\choose 4} + {9 \choose 6} + {9 \choose 8} = 256\]
  \end{solution}

\end{parts}




\question[8] How many $9$-bit strings (that is, bit strings of length 9) are there which:
\begin{parts}
  \part Start with the sub-string 101? Explain.
  \begin{solution}
    $2^6 = 64$.  You have 2 choices for each of the remaining 6 bits.
  \end{solution}

  \part Have weight 5 (i.e., contain exactly five 1's) and start with the sub-string 101? Explain.
  \begin{solution}
    ${6 \choose 3} = 20$.  You need to choose 3 of the remaining 6 bits to be 1's.
  \end{solution}

	\part Either start with $101$ or end with $11$ (or both)?  Explain.
		\begin{solution}
			176.  There are 64 strings that start with 101, and another 128 which end with 11 (we choose 1 or 0 for 7 bits, so $2^7$).  However, we count the strings that start with 101 and end with 11 twice - there are $16$ such strings ($2^4$).  So using PIE, we have $64 + 128 - 16 = 176$ 
		\end{solution}

	\part Have weight 5 and either start with 101 or end with 11 (or both)?  Explain.
	\begin{solution}
	 	51. There are ${6 \choose 3} = 20$ strings of weight 5 which start with 101, and another ${7 \choose 3} = 35$ which end with 11.  We have over counted again - there are weight 5 strings which both start with 101 and end with 11, in fact ${4 \choose 1} = 4$ of them.  So all together we have $20 + 35 - 4 = 51$ strings.
	\end{solution}
	
	
\end{parts}

\question[6] How many triangles are there with vertices from the points shown below?  Note, we are not allowing degenerate triangles - ones with all three vertices on the same line, but we do allow non-right triangles.  Explain why your answer is correct. (HINT: you need at exactly two points on either the $x$- or $y$-axis, but don't over-count the right triangles.)

\begin{center}
  \begin{tikzpicture}
    \foreach \i in {0,...,6} {
      \fill (\i,0) circle (2pt);
    }
    \foreach \i in {1,...,4} {
      \fill (0,\i) circle (2pt);
    }
  \end{tikzpicture}
\end{center}

\begin{solution}
  There are 120 triangles.  Here are two ways (there are others as well) to get this:
  
  \begin{enumerate}
    \item First count the triangles with the base on the $x$-axis.  There are ${7 \choose 2}$ ways to pick the base.  The third vertex of the triangle must be one of the 4 dots on the $y$-axis (not the origin) so there are a total of ${7 \choose 2}4$ of these triangles.  The triangles with base on the $y$ axis can be counted similarly: ${5 \choose 2}6$.  However, we have counted all the right triangles twice - they have a base on the $x$-axis and also on the $y$-axis.  There are $4 \cdot 6$ right triangles.  Thus the total number of triangles is:
    \[{7 \choose 2}4 + {5 \choose 2}6 - 6\cdot 4 = 120\]
    \item We must select 3 of the 11 dots.  This can be done in ${11 \choose 3}$ ways.  However, this will also give us degenerate triangles when all three vertices are on the $x$-axis or on the $y$-axis.  There are ${7 \choose 3}$ ways we could have picked all three vertices on the $x$-axis.  There are ${5 \choose 3}$ ways we could have picked all three vertices on the $y$-axis.  Therefore the total number of triangles is
    \[{11 \choose 3} - {7 \choose 3} - {5 \choose 3} = 120\]
  \end{enumerate}

\end{solution}
\end{questions}




\end{document}


