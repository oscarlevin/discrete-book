\documentclass[11pt]{exam}

\usepackage{amssymb, amsmath, amsthm, mathrsfs, multicol, graphicx} 
\usepackage{tikz}

 \def\d{\displaystyle}
\def\?{\reflectbox{?}}
\def\b#1{\mathbf{#1}}
\def\f#1{\mathfrak #1}
\def\c#1{\mathcal #1}
\def\s#1{\mathscr #1}
\def\r#1{\mathrm{#1}}
\def\N{\mathbb N}
\def\Z{\mathbb Z}
\def\Q{\mathbb Q}
\def\R{\mathbb R}
\def\C{\mathbb C}
\def\F{\mathbb F}
\def\A{\mathbb A}
\def\X{\mathbb X}
\def\E{\mathbb E}
\def\O{\mathbb O}
\def\U{\mathcal U}
\def\pow{\mathcal P}
\def\inv{^{-1}}
\def\nrml{\triangleleft}
\def\st{:}
\def\~{\widetilde}
\def\rem{\mathcal R}
\def\sigalg{$\sigma$-algebra }
\def\Gal{\mbox{Gal}}
\def\iff{\leftrightarrow}
\def\Iff{\Leftrightarrow}
\def\land{\wedge}
\def\And{\bigwedge}
\def\AAnd{\d\bigwedge\mkern-18mu\bigwedge}
\def\Vee{\bigvee}
\def\VVee{\d\Vee\mkern-18mu\Vee}
\def\imp{\rightarrow}
\def\Imp{\Rightarrow}
\def\Fi{\Leftarrow}

%\def\={\equiv}
\def\var{\mbox{var}}
\def\mod{\mbox{Mod}}
\def\Th{\mbox{Th}}
\def\sat{\mbox{Sat}}
\def\con{\mbox{Con}}
\def\bmodels{=\joinrel\mathrel|}
\def\iffmodels{\bmodels\models}
\def\dbland{\bigwedge \!\!\bigwedge}
\def\dom{\mbox{dom}}
\def\rng{\mbox{range}}
\DeclareMathOperator{\wgt}{wgt}


\def\bar{\overline}


\newcommand{\vtx}[2]{node[fill,circle,inner sep=0pt, minimum size=4pt,label=#1:#2]{}}
\newcommand{\va}[1]{\vtx{above}{#1}}
\newcommand{\vb}[1]{\vtx{below}{#1}}
\newcommand{\vr}[1]{\vtx{right}{#1}}
\newcommand{\vl}[1]{\vtx{left}{#1}}
\renewcommand{\v}{\vtx{above}{}}

\def\circleA{(-.5,0) circle (1)}
\def\circleAlabel{(-1.5,.6) node[above]{$A$}}
\def\circleB{(.5,0) circle (1)}
\def\circleBlabel{(1.5,.6) node[above]{$B$}}
\def\circleC{(0,-1) circle (1)}
\def\circleClabel{(.5,-2) node[right]{$C$}}
\def\twosetbox{(-2,-1.4) rectangle (2,1.4)}
\def\threesetbox{(-2.5,-2.4) rectangle (2.5,1.4)}
\newcommand{\twoline}[2]{\begin{pmatrix}#1 \\ #2 \end{pmatrix}}


\def\circleA{(-.5,0) circle (1)}
\def\circleAlabel{(-1.5,.6) node[above]{$A$}}
\def\circleB{(.5,0) circle (1)}
\def\circleBlabel{(1.5,.6) node[above]{$B$}}
\def\circleC{(0,-1) circle (1)}
\def\circleClabel{(.5,-2) node[right]{$C$}}
\def\twosetbox{(-2,-1.5) rectangle (2,1.5)}
\def\threesetbox{(-2,-2.5) rectangle (2,1.5)}

%\pointname{pts}
\pointsinmargin
\marginpointname{pts}
\addpoints
\pagestyle{head}
\printanswers

\firstpageheader{Math 228}{\bf Homework 4\\Solutions}{Due: Wednesday, Feb 11}


\begin{document}
%\noindent \textbf{Instructions}: Same rules as usual - turn in your work on separate sheets of paper.  You must justify all your answers for full credit.

\begin{questions}

\question[6] Suppose you own $x$ fezzes and $y$ bow ties.  Of course, $x$ and $y$ are both greater than 1.
\begin{parts}
  \part How many combinations of fez and bow tie can you make?  You can wear only one fez and one bow tie at a time.  Explain.
  \begin{solution}
    You have $x$ choices for the fez, and for each choice of fez you have $y$ choices for the bow tie.  Thus you have $x \cdot y$ choices for fez and bow tie combination.
  \end{solution}

  \part Explain why the answer is {\em also} ${x+y \choose 2} - {x \choose 2} - {y \choose 2}$.  (If this is what you claimed the answer was in part (a), try it again.)
  \begin{solution}
    Line up all $x+y$ quirky clothing items - the $x$ fezzes and $y$ bow ties.  Now pick 2 of them.  This can be done in ${x+y \choose 2}$ ways.  However, we might have picked 2 fezzes, which is not allowed.  There are ${x \choose 2}$ ways to pick 2 fezzes.  Similarly, the ${x+y \choose 2}$ ways to pick two items includes ${y \choose 2}$ ways to select 2 bow ties, also not allowed.  Thus the total number of ways to pick a fez and a bow ties is
    \[{x+y \choose 2} - {x \choose 2} - {y \choose 2}\]
  \end{solution}

  \part Use your answers to parts (a) and (b) to give a combinatorial proof of the identity
  \[{x+y \choose 2} - {x \choose 2} - {y \choose 2} = xy\]
  \begin{solution}
  \begin{proof}
       The question is how many ways can you select one of $x$ fezzes and one of $y$ bow ties.  We answer this question in two ways.  First, the answer could be $a\cdot b$. This is correct as described in part (a) above.  Second, the answer could be ${x+y \choose 2} - {x \choose 2} - {y \choose 2}$.  This is correct as described in part (b) above.  Therefore 
    \[{x+y \choose 2} - {x \choose 2} - {y \choose 2} = xy\]
  \end{proof}
  \end{solution}

\end{parts}


\question[6] Consider the identity:
\[k{n\choose k} = n{n-1 \choose k-1}\]
\begin{parts}
  \part Is this true?  Try it for a few values of $n$ and $k$.
  \begin{solution}
    Yes.  For example, if $n = 7$ and $k = 4$, we have \[4 \cdot {7 \choose 4} = 4 \cdot 35 = 140 = 7 \cdot 20 = 7 \cdot {6 \choose 3}\]
  \end{solution}

  \part Use the formula for ${n \choose k}$ to give an algebraic proof of the identity.
  \begin{solution}
    \[k{n \choose k} = k \frac{n!}{(n-k)!\,k!} = \frac{n!}{(n-k)!(k-1)!} = n\frac{(n-1)!}{(n-1-(k-1))!(k-1)!} = n {n-1 \choose k-1}\]
  \end{solution}

  \part Give a combinatorial proof of the identity. Hint: How many ways can you select a team of $k$ people from a group of $n$ people {\em and} select one of them to be the team captain?  
  \begin{solution}
    \begin{proof}
      Question: How many ways can you select a chaired committee of $k$ people from a group of $n$ people?  That is, you need to select $k$ people to be on the committee and one of them needs to be in charge.  How many ways can this happen?
      
      Answer 1: First select $k$ of the $n$ people to be on the committee.  This can be done in ${n \choose k}$ ways.  Now select one of those $k$ people to be in charge - this can be done in $k$ ways.  So there are a total of $k {n \choose k}$ ways to select the chaired committee.
      
      Answer 2: First select the chair of the committee.  You have $n$ people to choose from, so this can be done in $n$ ways.  Now fill the rest of the committee.  There are $n-1$ people to choose from (you cannot select the person you picked to be the chair) and $k-1$ spots to fill (the chair's spot is already taken).  So this can be done in ${n-1 \choose k-1}$ ways.  Therefore there are $n{n-1 \choose k-1}$ ways to select the chaired committee.
    \end{proof}

  \end{solution}

\end{parts}



\question[6] After a late night of math studying, you and your friends decide to go to your favorite tax-free fast food Mexican restaurant, {\em Burrito Chime}.  You decide to order off of the dollar menu, which has 7 items.  Your group has \$16 to spend (and will spend all of it). 
\begin{parts}
  \part How many different orders are possible?  Explain. (The {\em order} in which the order is placed does not matter - just which and how many of each item that is ordered.)
  \begin{solution}
    $\d{22 \choose 6}$ - there are 16 stars and 6 bars.
  \end{solution}

  \part How many different orders are possible if you want to get at least one of each item? Explain.
  \begin{solution}
    $\d{15 \choose 6}$ - buy one of each item, using \$7.  This leaves you \$11 to distribute to the 7 items, so 11 stars and 6 bars.
  \end{solution}

  \part How many different orders are possible if you don't get more than 4 of any one item?  Explain. Hint: get rid of the bad orders using PIE.
  \begin{solution}
    \[{22 \choose 6} - \left[{7 \choose 1}{17 \choose 6} - {7 \choose 2}{12 \choose 6} + {7 \choose 3}{7 \choose 6} \right]\]
  \end{solution}

\end{parts}


\question[6] Consider functions $f:\{1,2,3,4,5\} \to \{0,1,2,\ldots,9\}$.  
\begin{parts}
	\part How many of these functions are strictly increasing?  Explain.  (A function is strictly increasing provided if $a < b$, then $f(a) < f(b)$.)
	\begin{solution}
		${10 \choose 5}$.  Note that a strictly increasing function is automatically injective.  So the five outputs must all be different.  So let's first pick which five outputs we will use: there are ${10 \choose 5}$ ways to do this.  Now how many ways are there to assign those outputs to the inputs $1$ through 5?  Only one way, since there is only one way to arrange numbers in increasing order.		
	\end{solution}
	\part How many of the functions are non-decreasing?  Explain.  (A function is non-decreasing provided if $a < b$, then $f(a) \le f(b)$.)
	\begin{solution}
		${14 \choose 5}$.  This is in fact a stars and bars problem.  The stars are the 5 inputs and the bars are the 9 spots between the 10 possible outputs.  Think of it this way - we will specify $f(1)$, then $f(2)$, then $f(3)$, and so on in that order.  Start with the possible output 0.  We can use it as the output of $f(1)$, or we can switch to 1 as a potential output.  Say we put $f(1) = 1$.  Now we are at 1 (can't go back to 0).  Should $f(2) = 1$?  If yes, then we are putting down another star.  If no, put down a bar and switch to 2.  Maybe you switch to 3, then assign $f(2) = 3$ and $f(3) = 3$ (two more stars) before switching to 4 as a possible output.  And so on.  
	\end{solution}
\end{parts}



\question[6] The Grinch sneaks into a room with 6 Christmas presents to 6 different people.  He proceeds to switch the name-labels on the presents.  How many ways could he do this if:
\begin{parts}
  \part No present is allowed to end up with its original label?  Explain what each term in your answer represents.
  \begin{solution}
    \[6! - \left[{6 \choose 1}5! - {6 \choose 2}4! + {6 \choose 3}3! - {6 \choose 4}2! + {6 \choose 5}1! - {6 \choose 6}0!\right]\]
  \end{solution}

  \part Exactly 2 presents keep their original labels? Explain.
  \begin{solution}
    \[{6 \choose 2}\left(4! - \left[{4\choose 1}3! - {4 \choose 2}2! + {4 \choose 3}1! - {4 \choose 4}0!\right]\right)\]
  \end{solution}

  \part Exactly 5 presents keep their original labels? Explain.
  \begin{solution}
    0.  Once 5 presents have their original label, there is only one present left and only one label left, so the 6th present must get its own label.
  \end{solution}

\end{parts}

\end{questions}
\end{document}


