\documentclass[11pt]{exam}

\usepackage{amssymb, amsmath, amsthm, mathrsfs, multicol, graphicx} 
\usepackage{tikz}

 \def\d{\displaystyle}
\def\?{\reflectbox{?}}
\def\b#1{\mathbf{#1}}
\def\f#1{\mathfrak #1}
\def\c#1{\mathcal #1}
\def\s#1{\mathscr #1}
\def\r#1{\mathrm{#1}}
\def\N{\mathbb N}
\def\Z{\mathbb Z}
\def\Q{\mathbb Q}
\def\R{\mathbb R}
\def\C{\mathbb C}
\def\F{\mathbb F}
\def\A{\mathbb A}
\def\X{\mathbb X}
\def\E{\mathbb E}
\def\O{\mathbb O}
\def\U{\mathcal U}
\def\pow{\mathcal P}
\def\inv{^{-1}}
\def\nrml{\triangleleft}
\def\st{:}
\def\~{\widetilde}
\def\rem{\mathcal R}
\def\sigalg{$\sigma$-algebra }
\def\Gal{\mbox{Gal}}
\def\iff{\leftrightarrow}
\def\Iff{\Leftrightarrow}
\def\land{\wedge}
\def\And{\bigwedge}
\def\AAnd{\d\bigwedge\mkern-18mu\bigwedge}
\def\Vee{\bigvee}
\def\VVee{\d\Vee\mkern-18mu\Vee}
\def\imp{\rightarrow}
\def\Imp{\Rightarrow}
\def\Fi{\Leftarrow}

%\def\={\equiv}
\def\var{\mbox{var}}
\def\mod{\mbox{Mod}}
\def\Th{\mbox{Th}}
\def\sat{\mbox{Sat}}
\def\con{\mbox{Con}}
\def\bmodels{=\joinrel\mathrel|}
\def\iffmodels{\bmodels\models}
\def\dbland{\bigwedge \!\!\bigwedge}
\def\dom{\mbox{dom}}
\def\rng{\mbox{range}}
\DeclareMathOperator{\wgt}{wgt}


\def\bar{\overline}


\newcommand{\vtx}[2]{node[fill,circle,inner sep=0pt, minimum size=4pt,label=#1:#2]{}}
\newcommand{\va}[1]{\vtx{above}{#1}}
\newcommand{\vb}[1]{\vtx{below}{#1}}
\newcommand{\vr}[1]{\vtx{right}{#1}}
\newcommand{\vl}[1]{\vtx{left}{#1}}
\renewcommand{\v}{\vtx{above}{}}

\def\circleA{(-.5,0) circle (1)}
\def\circleAlabel{(-1.5,.6) node[above]{$A$}}
\def\circleB{(.5,0) circle (1)}
\def\circleBlabel{(1.5,.6) node[above]{$B$}}
\def\circleC{(0,-1) circle (1)}
\def\circleClabel{(.5,-2) node[right]{$C$}}
\def\twosetbox{(-2,-1.4) rectangle (2,1.4)}
\def\threesetbox{(-2.5,-2.4) rectangle (2.5,1.4)}
\newcommand{\twoline}[2]{\begin{pmatrix}#1 \\ #2 \end{pmatrix}}


\def\circleA{(-.5,0) circle (1)}
\def\circleAlabel{(-1.5,.6) node[above]{$A$}}
\def\circleB{(.5,0) circle (1)}
\def\circleBlabel{(1.5,.6) node[above]{$B$}}
\def\circleC{(0,-1) circle (1)}
\def\circleClabel{(.5,-2) node[right]{$C$}}
\def\twosetbox{(-2,-1.5) rectangle (2,1.5)}
\def\threesetbox{(-2,-2.5) rectangle (2,1.5)}

%\pointname{pts}
\pointsinmargin
\marginpointname{pts}
\addpoints
\pagestyle{head}
\printanswers

\firstpageheader{Math 228}{\bf Homework 7\\ Solutions}{Due: Wednesday, Mar 11}


\begin{document}
%\noindent \textbf{Instructions}: Same rules as usual - turn in your work on separate sheets of paper.  You must justify all your answers for full credit.

\begin{questions}
\question[6] Zombie Euler and Zombie Cauchy - two famous zombie mathematicians - have just signed up for Twitter accounts.  After one day, Zombie Cauchy has more followers than Zombie Euler.  Each day after that, the number of new followers of Zombie Cauchy is exactly the same as the number of new followers of Zombie Euler (and neither lose any followers).  Explain how a proof by mathematical induction can show that on every day after the first day, Zombie Cauchy will have more followers than Zombie Euler.  That is, explain what the base case and inductive case are, and why they together prove that Zombie Cauchy will have more followers on the 4th day.

\begin{solution}
  The idea here is that because we know Zombie Cauchy starts ahead, and each day increases by the same amount as Zombie Euler, he will always be ahead.
  
   The base case is that Zombie Cauchy has more followers than Zombie Euler on day 1.  We know this is true because it says so in the problem.
    
    The inductive case is that {\em if} Zombie Cauchy has more followers on day $k$, then he will still have more followers on day $k+1$.  We know this is true because each day, the Zombies receive an equal number of new followers. 
    
    Together, the base case and inductive case show that on the 4th day, Zombie Cauchy will be ahead: he is ahead on day 1, and because on day 1 he is ahead, by the inductive case he will also be ahead on day 2.  By the inductive case again, he will be ahead on day 3 since he is ahead on day 2, and since he is ahead on day 3, he will also be ahead on day 4.  Of course we could keep doing this up to any day.
\end{solution}

\uplevel{{\bf Special Induction Instructions}: For the rest of the homework problems, you should first give a rough sketch of the argument (i.e., say {\em why} induction will work in this case) and then also give a formal proof by induction (starting with, ``Let $P(n)$ be the statement\ldots'').}

\question[8] Find the largest number of points which a football team cannot get exactly using just 3-point field goals and 7-point touchdowns (ignore the possibilities of safeties, missed extra points, and two point conversions).  Prove your answer is correct by mathematical induction.

\begin{solution}
  First note that it is impossible to make 11 points - if only field goals are made, the points must be a multiple of 3, if 1 touchdown is made, the possible point totals are 7, 10, 13, \ldots and two touchdowns are already too much.
  
  We will prove that 11 is the largest number of points which cannot be made.  In other words, any number of points greater than or equal to 12 can be made.
  
  \begin{proof}
    Let $P(n)$ be the statement ``it is possible to make $n$ points using touchdowns and field goals.''  We will prove $P(n)$ is true for all $n \ge 12$.
    
    First the base case: You can make 12 points with 4 field goals, so $P(12)$ is true.
    
    Now the inductive case: Assume $P(k)$ is true for some fixed $k \ge 12$.  That is, it is possible to make $k$ points.  Since $k \ge 12$, we must have made the $k$ points using either at least 2 field goals or at least 2 touchdowns, or both (because if we used just one of each we would have only 10 points).  Now if the $k$ points were accomplished with 2 (or more) field goals, then replace 2 field goals with 1 touchdown.  This increases to point total by 1, giving $k + 1$ points.  On the other hand, if the $k$ points were accomplished with $2$ (or more) touchdowns, replace 2 touchdowns with 5 field goals, again increasing the point total by 1, giving $k+1$ points.  Using one of these two substitutions, we can make $k+1$ points, so $P(k+1)$ is true, establishing the inductive case.
    
    Therefore by the principle of mathematical induction, $P(n)$ is true for all $n \ge 12$.
  \end{proof}
\end{solution}

%\question[6] Prove that the sum of $n$ squares can be found as follows\[1^2 +2^2 +3^2+...+n^2 = \frac{n(n+1)(2n+1)}{6}\]
%\begin{solution}
%This question is asking us to show that the sum of squares for $n$ numbers can be found using the formula $\frac{n(n+1)(2n+1)}{6}$. We can definitely see this is true for the first few instances, but we are really taking it on faith that it is true for the first $n$ squares. So, it seems like induction would be a good place to start.
%\begin{proof}
%First, let $P(n)$ be the statement that is given.
%
%Base case: We must now show that our base case $P(1)$ is true \[1^2 = \frac{1(1+1)(2(1)+1)}{6} = \frac{6}{6} =1\]
%
%Inductive case: Now, we assume that for some $k\leq n$ that $P(k)$ is true and show that $P(k+1)$ is true. Namely, assume that $1^2 +2^2 +3^2+...+k^2 = \frac{k(k+1)(2k+1)}{6}$ and show that \[1^2 +2^2 +3^2+...+k^2+{(k+1)}^2 = \frac{(k+1)((k+1)+1)(2(k+1)+1)}{6}\]
%
%At this point it is advantageous to start on one side of the equality. So, choosing the left hand side to start I shall manipulate it so that it looks like the right hand side.
%
%\begin{align*}
%1^2 +2^2 +3^2+...+k^2+{(k+1)}^2 = \\
%= & \frac{k(k+1)(2k+1)}{6} +(k+1)^2 \mbox{ by our inductive hypothesis}\\
%= & \frac{k(k+1)(2k+1)}{6} +\frac{6(k+1)^2}{6} \\
%= & \frac{k(k+1)(2k+1)+6(k+1)^2}{6} \\
%= & \frac{(k+1)[k(2k+1)+6(k+1)]}{6} \mbox{ factoring out $k+1$ from each term}\\
%= & \frac{(k+1)[2k^2+k+6k+6]}{6}\\
%= & \frac{(k+1)[2k^2+7k+6]}{6}\\
%= & \frac{(k+1)[(2k+3)(k+2)]}{6}\\
%= & \frac{(k+1)(2(k+1)+1)((k+1)+1)}{6}
%\end{align*}
%Thus, $P(k+1)$ is true.
%
%Therefore, by the principle of mathematical induction $P(n)$ is true for all $n \geq 1$ 
%\end{proof}
%\end{solution}
\question[8] Prove, by mathematical induction, that $F_0 + F_1 + F_2 + \cdots + F_{n} = F_{n+2} - 1$, where $F_n$ is the $n$th Fibonacci number ($F_0 = 0$, $F_1 = 1$ and $F_n = F_{n-1} + F_{n-2}$).
\begin{solution}
This is saying that if we add up the first $n$ Fibonacci numbers, we will get another Fibonacci number (specifically, the $(n+2)$th one).  Induction is a good idea here because it will be easy to just add one more Fibonacci number to the sum we already have.  If we already have $F_{k+2}$ and we add $F_{k+1}$ we can use the recurrence relation to simplify this, becoming $F_{k+3}$.  
 
  \begin{proof}
    Let $P(n)$ be the statement $F_0 + F_1 + F_2 + \cdots + F_n = F_{n+2} - 1$.  We will prove that $P(n)$ is true for all $n \ge 0$.  
    
    Base case: $P(0)$ states that $F_0 = F_2 - 1$, which is true because $F_0 = 0$ and $F_2 = 1$.
    
    Inductive case:  Assume $P(k)$ is true for an arbitrary fixed $k \ge 0$.  That is, \[F_0 + F_1 + F_2 + \cdots + F_k = F_{k+2} - 1\]
    We must prove that $P(k+1)$ is true as well (i.e. that $F_0 + F_1 + \cdots +F_{k+1} = F_{k+3} - 1$).  Start with the left hand side:
    \begin{align*}
      F_0 + F_1 + F_2 + \cdots + F_k + F_{k+1} & = F_{k+2} - 1 + F_{k+1} & \mbox{ by the inductive hypothesis}\\
      & = F_{k+3} - 1 & \mbox{ by the definition of the Fibonacci numbers}
    \end{align*}
    Thus $P(k+1)$ is true.
    
    Therefore by the principle of mathematical induction, $P(n)$ is true for all $n \ge 0$.
  \end{proof}

\end{solution}

\question[8] Prove that for any integer $n\geq 2$, $n$ is either prime or a product of primes. Use strong induction.  Hint: If $n > 2$ and is {\bf not} prime, then $n = a\cdot b$ where $a$ and $b$ are less than $n$.

\begin{solution}
This question is asking us to prove that every integer is a product of primes. That is, we are basically showing the existence of the prime factorization for every integer. Since the question asks us to use strong induction, we are going to first show a base case is true and then assume that if $P(k)$ with $k\geq 2$ is true then $P(k+1)$ is also true.
\begin{proof}
Let $P(n)$ be the statement ``for any integer $n\geq 2$, $n$ is either a prime or a product of primes''

Base case: $P(2)$ is then our first case (because $n\geq 2$) and $P(2)$ states that $n=2$ is either a prime or a product of primes. Well, $2$ is in fact a prime number. So, our base case is satisfied.

Inductive case: Assume that for all $k\geq 2$ that $P(k)$ is true. Namely, $k$ is either a prime or a product of primes. Now, since we are using strong induction, we need to show that $P(k+1)$ holds. That is, we need to show that $k+1$ is either prime or a product of primes. If $k+1$ is a prime, then we are done. So, assume that $k+1$ is not prime. Then we need to show that $k+1$ can be written as a product of primes. Since $k+1$ is not a prime, $k+1=ab$ where $a,b$ are two integers such that $2 \leq a, b<k+1$. Now, if $a=2$ then the largest possible value for $b$ is $\frac{k+1}{2}$, so $a,b< \frac{k+1}{2} <k$. Therefore by our inductive case, $a$ and $b$ are both either a prime or a product of primes. Namely,

\[a=p_1p_2\cdots p_i \text{ where each $p_i$ is a prime or } a \text{ is prime} \] 
\[b=r_1r_2\cdots r_j \text{ where each $r_j$ is a prime or } b \text{ is prime}\]
Thus, $k+1=ab$ is a product of primes.

Therefore by the strong induction principle $P(n)$ is true for all $n\geq 2$
 \end{proof}
 \end{solution}
 
 

 
\end{questions}
\end{document}


