\documentclass[12pt]{article}

\usepackage{discrete}

\def\thetitle{Mathematical Logic} % will be put in the center header on the first page only.
\def\lefthead{Math 228 Notes} % will be put in the left header
\def\righthead{Logic} % will be put in the right header


\begin{document}

\begin{activity}
Holmes owns two suits: one blue and one brown.  He always wears either a blue suit or white socks.  Whenever he wears his blue suit and a blue shirt, he also wears a blue tie.  He never wears the blue suit unless he is also wearing either a blue shirt or white socks.  Whenever he wears white socks, he also wears a blue shirt.  Today, Holmes is wearing a gold tie.  What else is he wearing?\footnote{Adapted from \textit{Problem Solving Through Recreational Mathematics} by Averbach and Chein, Dover 1999.}
\end{activity}


Logic is the study of consequence.  Given a few mathematical statements or facts, we would like to be able to draw some conclusions.  For example, if I told you that a particular real valued function was continuous on the interval $[0,1]$, and $f(0) = -1$  and $f(1) = 5$, can we conclude that there is some point between $[0,1]$ where the graph of the function crosses the $x$-axis? Yes, we can, thanks to the Intermediate Value Theorem from Calculus.  Can we conclude that there is exactly one point?  No.  Whenever we find an ``answer'' in math, we really have a (perhaps hidden) argument - given the situation we are in, we can conclude the answer is the case.  Of course real mathematics is about proving general statements (like the Intermediate Value Theorem), and this too is done via an argument, usually called a proof.  We start with some given conditions - these are the premises of our argument.  From these we find a consequence of interest - our conclusion.  

The problem is, as you no doubt know from arguing with friends, not all arguments are \textit{good} arguments.  A ``bad'' argument is one in which the conclusion does not follow from the premises - the conclusion is not a consequence of the premises.  Logic is the study of what makes an argument good or bad.  In other words, logic aims to determine in which cases a conclusion is, or is not, a consequence of a set of premises.

By the way, ``argument'' is actually a technical term in math (and philosophy -- another discipline which studies logic):
\begin{definition}
 An {\em argument} is a set of statements, one of which is called the {\em conclusion} and the rest of which are called {\em premises}.  An argument is said to be {\em valid} if the conclusion must be true whenever the premises are all true.  An argument is {\em invalid} if it is not valid; it is possible for all the premises to be true and the conclusion to be false.
\end{definition}

For example, consider the following two arguments:

\vskip 1em
 \begin{tabular}{ll}
  & If Edith eats her vegetables, then she can have a cookie.\\
  & Edith eats her vegetables. \\ \hline
  $\therefore$ & Edith gets a cookie.
 \end{tabular}
\vskip 1em
 \begin{tabular}{ll}
  & Florence must eat her vegetables in order to get a cookie. \\
  & Florence eats her vegetables. \\ \hline
  $\therefore$ & Florence gets a cookie.
 \end{tabular}
\vskip 1em
(The symbol ``$\therefore$'' means ``therefore'')


Are these arguments valid?  Hopefully you agree that the first one is but the second one is not.  Logic tells us why.  How?  By analyzing the structure of the statements in the argument.  Notice that the two arguments above look almost identical.  Edith and Florence both eat their vegetables.  In both cases there is a connection between eating of vegetables and cookies.  But we claim that it is valid to conclude that Edith gets a cookie, but not that Florence does.  The difference must be in the connection between eating vegetables and getting cookies.  We need to be good at reading and comprehending these sentences.  Do the two sentences mean the same thing?  Unfortunately, when talking in everyday language, we are often sloppy, and you might be tempted to say they are equivalent.  But notice that just because Florence {\em must} eat her vegetables, we have not said that doing so would be enough - she might also need to clean her room, for example.  In everyday (non-mathematical) practice, you might 
say that the ``other direction'' was implied.  We don't ever get to say that.

Our goal in studying logic is to gain intuition for which arguments are valid and which are invalid.  This will require us to become better at reading and writing mathematics -- a worthy goal in its own right.  So let's get started.


\end{document}
