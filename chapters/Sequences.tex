\documentclass[12pt]{article}

\usepackage{../discrete}

\heading{Math 228}{}{Sequences Notes}



\begin{document}
As the legend goes, there is a monastery in Hanoi with a great hall containing 3 tall pillars.  Resting on the first pillar are 64 giant disks (or washers) - all different sizes, stacked from largest to smallest.  The monks are charged with the following task: they must move the entire stack of disks to the third pillar.  However, due to the size of the disks, the monks cannot move more than one at a time.  Each disk must be placed on one of the pillars before the next disk is moved.  And because the disks are so heavy and fragile, the monks may never place a larger disk on top of a smaller disk.

When the monks finally complete their task, the world shall come to an end.  Your task: figure out how long before we need to start worrying about the end of the world.


This puzzle is called the {\em Tower of Hanoi}.  You are tasked with finding the minimum number of moves to complete the puzzle.  This certainly sounds like a counting problem.  Perhaps you have an answer?  If not, what else could we try?  The answer depends on the number of disks you need to move.  In fact, we could answer the puzzle first for 1 disk, then 2, then 3 and so on.  If we list out all of the answers for each number of disks, we will get a {\em sequence} of numbers.  The $n$th term in the sequence is the answer to the question, ``what is the smallest number of moves required to complete the Tower of Hanoi puzzle with $n$ disks?''  You might wonder why we would create such a sequence, instead of just answering the question.  The reason is, but looking at how the sequence of numbers grows, we gain insight into the problem.  It is easy to count the number of moves required for small numbers of disks.  We can then look for a pattern among the first few terms of the sequence.  Hopefully this will suggest a method for finding the $n$th term - the answer to our question.  Of course we will also need to verify that our suspected pattern is correct, and that this correct pattern really does give us the $n$th term we think it does, but it is impossible to prove that your formula is correct without having a formula to start with.  

Of course sequences are also interesting mathematical objects to study in their own right. Let's see why.


\section{Defining Sequences}

A sequence is simply an ordered list of numbers.  For example, here is a sequence: 0, 1, 2, 3, 4, 5, \ldots.  This is different from the set $\N$, because while the sequence is a complete list of every element in the set of natural numbers, in the sequence, we very much care what order the numbers come in.  For this reason, when we use variables to represent terms in a sequence, they will look like this:
\[a_0, a_1, a_2, a_3, \ldots\]
We might replace the $a$ with another letter, and sometimes we omit $a_0$, starting with $a_1$ instead.  The numbers in the subscripts are called {\em indices} (the plural of {\em index}).  We can think of the terms in the sequence as the outputs of a function with domain $\N$: the $n$th term in the sequence is $f(n) = a_n$.


\begin{example}
Can you find the next term in the following sequences?
\begin{enumerate}


\begin{multicols}{2}
		\item $7,7,7,7,7, \ldots$
		\item $3, -3, 3, -3, 3, \ldots$
		\item $1, 5, 2, 10, 3, 15, \ldots$
		\item $1, 2, 4, 8, 16, 32, \ldots$
		\item $1, 4, 9, 16, 25, 36, \ldots$
		\item $1, 2, 3, 5, 8, 13, 21, \ldots$
		\item $1, 3, 6, 10, 15, 21, \ldots$
		\item $2, 3, 5, 7, 11, 13, \ldots$
		\item $3, 2, 1, 0, -1, \ldots$
		\item $1, 1, 2, 6, \ldots$ 
		\end{multicols}
	\end{enumerate}
	\begin{solution}
	 No you cannot.  You might guess that the next terms are: 
	 \begin{enumerate}
	  \begin{multicols}{4}
	  \item $7$
	  \item $-3$
	  		\item $4$
		\item 64
		\item 49
		\item 34
		\item 28
		\item 17
		\item $-2$
		\item $24$
	  \end{multicols}

	 \end{enumerate}
	 In fact, those are the next terms of the sequences I had in mind when I made up the example.  But there is no way to be sure they are correct.  
	 
	 That said, we will often do this.  Given the first few terms of a sequence, we can ask what the pattern in the sequence suggests the next terms are.  
	\end{solution}

\end{example}

Given that no number of initial terms in a sequence is enough to say for certain which sequence we are dealing with, we need to find another way to specify a sequence.  There are two ways to do this.

\begin{defbox}{Closed formula}
 A {\em closed formula} for a sequence $a_0, a_1, a_2,\ldots$ is a formula for $a_n$ using a fixed finite number of operations on $n$.  This is what you normally think of as a formula in $n$.  
\end{defbox}

\begin{defbox}{Recursive definition}
 A {\em recursive definition} (sometimes called an {\em inductive definition}) for a sequence $a_0, a_1, a_2, \ldots$ consists of a {\em recurrence relation}: an equation relating a term of the sequence to previous terms (terms with smaller index) and an {\em initial condition}: a list of a few terms of the sequence (one less than the number of terms in the recurrence relation).
\end{defbox}

It is easier to understand what is going on here with an example:

\begin{example}
 Here are a few closed formulas for sequences:
 \begin{itemize}
  \item $a_n = n^2$
  \item $a_n = \frac{n(n+1)}{2}$
  \item $a_n = \frac{(\frac{1 + \sqrt 5}{2})^n - 1/(\frac{1 + \sqrt 5}{2})^n}{5}$.
 \end{itemize}
 Note in each case, if you are given $n$, you can calculate $a_n$ directly - just plug in $n$.
 
 Here are a few recursive definitions for sequences:
 \begin{itemize}
  \item $a_n = 2a_{n-1}$ with $a_0 = 1$.
  \item $a_n = 2a_{n-1}$ with $a_0 = 27$.
  \item $a_n = a_{n-1} + a_{n-2}$ with $a_0 = 0$ and $a_1 = 1$.
 \end{itemize}
  In these cases, if you are given $n$, you cannot calculate $a_n$ directly, you first need to find $a_{n-1}$ or $a_{n-1}$ and $a_{n-2}$.  
\end{example}


You might wonder why we would bother with recursive definitions for sequences - after all, it is harder to find $a_n$ with a recursive definition than with a closed formula.  This is true, but it is also harder to find a closed formula for a sequence than it is to find a recursive definition.  So to find a useful closed formula, we might first find the recursive definition, then use that to find the closed formula.  

This is not to say that recursive definitions aren't useful in finding $a_n$.  You can always calculate $a_n$ given a recursive definition - it might just take a while.

\begin{example}
 Find $a_6$ in the sequence defined by $a_n = 2a_{n-1} - a_{n-2}$ with $a_0 = 3$ and $a_1 = 4$.  
 \begin{solution}
  We know that $a_6 = 2a_5 - a_4$.  So to find $a_6$ we need to find $a_5$ and $a_4$.  Well \[a_5 = 2a_4 - a_3 \qquad \mbox{and} \qquad a_4 = 2a_3 - a_2\]
  so if we can only find $a_3$ and $a_2$ we would be set.  Of course
  \[a_3 = 2a_2 - a_1 \qquad \mbox{and} \qquad a_2 = 2a_1 - a_0\]
  so we only need to find $a_1$ and $a_0$.  But we are given these.  Thus
  \begin{align*}
   a_0 & = 3 \\
   a_1 & = 4 \\
   a_2 & = 2\cdot 4 - 3 = 5\\
   a_3 & = 2\cdot 5 - 4 = 6\\
   a_4 & = 2\cdot 6 - 5 = 7\\
   a_5 & = 2\cdot 7 - 6 = 8\\
   a_6 & = 2\cdot 8 - 7 = 9.
  \end{align*}
 Note that now we can guess a closed formula for the $n$th term of the sequence: $a_n = n+3$.  To be sure this will always work, we could plug in this formula into the recurrence relation: \[2a_{n-1} - a_{n-2} = 2((n-1) + 3) - ((n-2) + 3) = 2n + 4 - n - 1 = n + 3 = a_n\]
 Since $a_0 = 0 + 3 = 3$ and $a_1 = 1+3 = 4$ are the correct initial conditions, we have lucked upon the correct closed formula.
 \end{solution}

\end{example}


Finding closed formulas, or even recursive definitions, for sequences is not trivial - there is no one method for doing this.  Just like in evaluating integrals or solving differential equations, it is useful to have a bag of tricks you can apply, but sometimes there is no easy answer.

One useful trick to keep in your bag is relating a given sequence to another sequence for which we already know the closed formula.

\begin{example}
  Use the formulas $T_n = \frac{n(n+1)}{2}$ and $a_n = 2^n$ to find closed formulas for the following sequences.  Assume the first term has index 1 (not 0).
%  \begin{multicols}{2}
  \begin{enumerate}
    \item 2, 4, 7, 11, 16, 22, \ldots
    \item 2, 3, 5, 9, 17, 33,\ldots
    \item 2, 6, 12, 20, 30, 42,\ldots
    \item 6, 10, 15, 21, 28, \ldots
    \item 1, 3, 7, 15, 31, \ldots
    \item 3, 6, 12, 24, 48, \ldots
    \item 6, 10, 18, 34, 66, \ldots
    \item 15, 33, 57, 87, 123, \ldots
  \end{enumerate}
%  \end{multicols}

\begin{solution}
  Before you say this is impossible, what we are asking for is simply to find a closed formula which agrees with all of the initial terms of the sequences.  Of course there is no way to read into the mind of the person who wrote the numbers down, but we can at least do this.
  
  Now the first few terms of $T_n$, starting with $T_1$ are $1, 3, 6, 10, 15, 21, \ldots$ (these are the triangular numbers).  The first few terms of $a_n$ (starting this time with $a_0$) are $1, 2, 4, 8, 16, \ldots$.  Now let's try to find formulas for the given sequences.
  
    \begin{enumerate}
    \item 2, 4, 7, 11, 16, 22, \ldots - Note that if subtract 1 from each term, we get the sequence $T_n$.  So this sequence is $T_n + 1$.  Therefore a closed formula is $\frac{n(n+1)}{2} + 1$.  A quick check of the first few $n$ confirms we have it right.
    \item 2, 3, 5, 9, 17, 33,\ldots - This sequence is the result of adding 1 to each term in $a_n$.  So we might guess the closed formula $2^n + 1$.  If we try this though, we get the first terms $2^1 + 1 = 3$ and $2^2 + 1 = 5$.  We are off, because $a_n$ started with $n = 0$, and now we are starting with $n = 1$.  So we shift: the closed formula is $2^{n-1} + 1$.
    \item 2, 6, 12, 20, 30, 42,\ldots - Notice that all these terms are even.  What happens if we factor out a 2?  We get $T_n$!  So this sequence has closed formula $n(n+1)$.
    \item 6, 10, 15, 21, 28, \ldots - These are all triangular numbers.  However, we are starting with 6 as our first term instead of as our third term.  So if we could plug in 3 instead of 1 into the formula for $T_n$, we would be set.  Therefore the closed formula is $\frac{(n+2)(n+3)}{2}$ (where $n+3$ came from $(n+2)+1$)  
    \item 1, 3, 7, 15, 31, \ldots - Try adding one to each term and we get powers of 2.  You might guess this because each term is approximately twice the previous term.  Closed formula: $2^{n} - 1$.
    \item 3, 6, 12, 24, 48, \ldots - These numbers are all multiples of 3.  Let's try dividing each by 3.  Doing so gives 1, 2, 4, 8, \ldots.  Aha.  We get the closed formula $3\cdot 2^{n-1}$.  
    \item 6, 10, 18, 34, 66, \ldots - To get from one term to the next, we almost double each term.  So maybe we can relate this back to $2^n$. Yes, each term is 2 more than a power of 2.  So we get $2^{n+1} + 2$ (the $n+1$ is because the first term is 2 more than $2^2$, not $2^1$).  Alternatively, we could have related this sequence to the second sequence in this example: starting with 3, 5, 9, 17, \ldots we see that this sequence is twice the terms from that sequence (omitting the 2).  That sequence had closed formula $2^{n-1} + 1$.  To make the 3 first, we would write $2^{n} + 1$.  Our sequence would be twice this, so $2(2^n + 1)$, which of course is the same as we got before.
    \item 15, 33, 57, 87, 123, \ldots - Try dividing each term by 3.  That gives the sequence $5, 11, 19, 29, 41,\ldots$.  Now add one: $12, 20, 30, 42, \ldots$ - which is sequence 3 in this example, except starting with 6 instead of 2.  So let's start with the formula for sequence 3: $n(n+1)$.  To start with the 6, we shift: $(n+1)(n+2)$.  But this is one too many, so subtract 1: $(n+1)(n+2) - 1$.  That gives us our sequence, but divided by 3.  So we want $3((n+1)(n+2) - 1)$. 
  \end{enumerate}
\end{solution}

\end{example}






\section{Arithmetic and Geometric Sequences}
We now turn to the question of finding closed formulas for particular types of sequences.


\begin{defbox}{Arithmetic Sequences}
  If the terms of a sequence differ by a constant, we say the sequence is {\em arithmetic}.
  
  If the first term ($a_1$) of the sequence is $a$ and the common difference is $d$, then we have,
  
  Recursive definition: $a_n = a_{n-1} + d$ with $a_1 = a$.
  
  Closed formula: $a_n = a + d(n-1)$.
\end{defbox}

How do we know this? For the recursive definition, we need to specify $a_1$.  Then we need to express $a_n$ in terms of $a_{n-1}$.  If we call the first term $a$, then $a_1 = a$.  For the recurrence relation, by the definition of an arithmetic sequence, the difference between successive terms is some constant, say $d$.  So $a_n - a_{n-1} = d$, or in other words, \[ a_1 = a \qquad a_n=a_{n-1}+d\]

To find a closed formula, first try writing out the first write out the sequence using the recursive definition without simplifying:
\[a, a+d, a+d+d, a+d+d+d, \ldots\]

We see that to find the $n$th term, we need to start with $a$ and then add $d$ a bunch of times. In fact, add it $n-1$ times.  Thus $a_n = a+d(n-1)$.  

\begin{example}
  Find recursive definitions and closed formulas for the sequences:
  \begin{enumerate}
  \item $2, 5, 8, 11, 14, \ldots$.
    \item $50, 43, 36, 29, \ldots$.
    
  \end{enumerate}
  \begin{solution}
  First we should check that these sequences really are arithmetic.  Doing so will reveal the common difference $d$.
    \begin{enumerate}
      \item $5-2 = 3$, $8-5 = 3$, etc.  To get from each term to the next, we add three, so $d = 3$.  The recursive definition is therefore $a_n = a_{n-1} + 3$ with $a_1 = 2$.  The closed formula is $a_n = 2 + 3(n-1)$.
      \item Here the common difference is $-7$, since we add $-7$ to 50 to get 43, and so on.  Thus we have a recursive definition of $a_n = a_{n-1} - 7$ with $a_1 = 50$.  The closed formula is $a_n = 50 - 7(n-1)$.
    \end{enumerate}

  \end{solution}

\end{example}


What about sequences like $2, 6, 18, 54, \ldots$?  This is not arithmetic, because the difference between terms is not constant.  However, the {\em ratio} between successive terms is constant.  We call such sequences {\em geometric}.

The recursive definition for the geometric progression with first term $a$ and common ratio $r$ is 
\[a_1 = a \qquad a_n = a_{n-1}\cdot r \]
This makes sense.  To get the next term we multiply the previous term by $r$.  We can find the closed formula like we did for the arithmetic progression.  Write $a_1 = a$, $a_2 = a\cdot r$, $a_3 = a_2 \cdot r = a\cdot r \cdot r$ and so on.  We must multiply the first term $a$ by $r$ a number of times - $n -1$ to be precise.  We get $a_n = a\cdot r^{n-1}$.

\begin{defbox}{Geometric Sequences}
  A sequence is called {\em geometric} if the ratio between successive terms is constant.
  Suppose the first term $a_1$ is $a$ and the common ratio is $r$.  Then we have,
  
  Recursive definition: $a_n = ra_{n-1}$ with $a_1 = a$.
  
  Closed formula: $a_n = a\cdot r^{n-1}$.
\end{defbox}


\begin{example}
  Find the recursive and closed formula for the sequences:
  \begin{enumerate}
    \item $3, 6, 12, 24, 48, \ldots$
    \item $27, 9, 3, 1, 1/3, \ldots$
  \end{enumerate}
  \begin{solution}
    Again, we should first check that these sequences really are geometric - but doing so will tell us what $r$ is.
    \begin{enumerate}
      \item $6/3 = 2$, $12/6 = 2$, $24/12 = 2$, etc.  Yes, to get from any term to the next, we multiply by $r = 2$.  So the recursive definition is $a_n = 2a_{n-1}$ with $a_1 = 3$.  The closed formula is $a_n = 3\cdot 2^{n-1}$.
      \item $9/27 = 1/3$ and $3/9 = 1/3$ and so on.  The common ratio is $r = 1/3$.  So the sequence has recursive definition $a_n = \frac{1}{3}a_{n-1}$ with $a_1 = 27$ and closed formula $a_n = 27\cdot \frac{1}{3}^{n-1}$.
    \end{enumerate}
  \end{solution}
\end{example}





\subsection{Sums of Arithmetic and Geometric Sequences}

Look at the sequence $1, 3, 6, 10, 15,\ldots$.  We have a formula for this already (they are the triangular numbers) but let's see if we can derive the formula anew.  First, is this sequence arithmetic?  No, since $3-1 = 2$ and $6-3 = 3 \ne 2$, so there is no common difference.  Is the sequence geometric?  No.  $3/1 = 3$ but $6/3 = 2$, so there is no common ratio.  What to do?  

Notice thought that the differences between terms are a arithmetic sequence: $2, 3, 4, 5, 6,\ldots$.  This says that the $n$th term of the sequence $1,3,5,10,15,\ldots$ is the {\em sum} of the first $n$ terms in the sequence $1,2,3,4,5,\ldots$.  If we know how to add up the terms of an arithmetic sequence, we could use this to find a closed formula for a sequence whose differences are the terms of that arithmetic sequence.  In fact, this is how we found the formula $\frac{n(n+1)}{2}$ for the triangular numbers.

We could use a similar technique to find a closed formula for the sequence $2, 3, 5, 9, 17, \ldots$ - the differences are $1, 2, 4, 8, \ldots$, a geometric sequence.  If we had a method for summing geometric sequences, we could get a formula for sequences with geometric differences.

\subsubsection*{Summing Arithmetic Sequences: Reverse and Add}
Here is a trick that allows us to quickly find the sum of an arithmetic sequence. 

\begin{example}
  Find the sum:  $2 + 5 + 8 + 11 + 14 + \cdots + 470$.
  
  \begin{solution}
    The idea is to mimic the trick we used to find the formula for triangular numbers.  If we add the first and last, we get 472.  The second term and second-to-last term also add up to 472.  To make this clearer, we might express this as follows.  Call the sum $S$.  Then,
    
    \begin{center}
    \begin{tabular}{rccccccccc}
      $S  =  $& $2 $&$ + $& $5$ & $ + $ & $8$ & $+ \cdots + $ & $467$ &$ + $ & 470 \\
     $+ \quad S  = $& $470$ & $+ $ & $467$ & $ + $ & $464$& $+ \cdots + $&$ 5$ & $+$ & 2 \\ \hline
     $2S  = $& $472$ & $+ $ & $472$ & $ + $ & $472$& $+ \cdots + $&$472$ & $+$ & $472$ \\
    \end{tabular}
    \end{center}
    
    To find $2S$ then we add 472 to itself a number of times.  What number?  We need to decide how many terms (summands) are in the sum.  The $n$th term in the sum can be expressed as $2 + 3(n-1)$ - this is an arithmetic sequence after all.  We want to find $n$ when $2 + 3(n-1) = 470$.  Solving for $n$ gives $n = 157$.  So $n$ ranges from 1 to 157, giving 157 terms in the sum.  This is the number of 472's in the sum for $2S$.  Thus
    \[2S = 157\cdot 472 = 74104\]
    It is now easy to find $S$:
    \[S = 74104/2 = 37052\]
  \end{solution}
\end{example}

This will work in general, for any sum of {\em arithmetic} sequences.  Call the sum $S$.  Reverse and add.  This produces a single number added to itself many times.  Find the number of times.  Multiply.  Divided by 2.  Done.

\begin{example} 
  Find a closed formula for $6 + 10 + 14 + \cdots + (4n - 2)$.
  \begin{solution}
    Again, we have a sum of an arithmetic sequence.  We need to know how many terms are in the sequence.  Clearly each term in the sequence has the form $4k -2$ (as evidenced by the last term).  For which values of $k$ though?  To get 6, $k = 2$.  To get $4n-2$ take $k = n$.  So to find the number of terms, we need to know how many integers are in the range $2,3,\ldots, n$.  The answer is $n-1$.  (There are $n$ numbers from 1 to $n$, so one less if we start with 2.)
    
    Now use the reverse and add trick:
    
        \begin{center}
    \begin{tabular}{rccccccccc}
      $S  =  $& $6 $&$ + $& $10$ & $ + $ & $14$ & $+ \cdots + $ & $4n-6$ &$ + $ & $4n-2$ \\
     $+ \quad S  = $& $4n-2$ & $+ $ & $4n-6$ & $ + $ & $4n-10$& $+ \cdots + $& $10$ & $+$ & 6 \\ \hline
     $2S  = $& $4n+4$ & $+ $ & $4n+4$ & $ + $ & $4n+4$& $+ \cdots + $&$4n+4$ & $+$ & $4n+4$ \\
    \end{tabular}
    \end{center}
    
    Since there are $n-2$ terms, we get
    \[2S = (n-2)(4n+4)\qquad \mbox{ so }\qquad S = \frac{(n-2)(4n+4)}{2}\]
  \end{solution}

\end{example}


\subsubsection*{Summing Geometric Sequences: Multiply, Shift and Subtract}

To find the sum of a geometric sequence, we cannot use the previous trick.  Do you see why?  The reason we got the same term added to itself many times is because there was a constant difference.  So as we added that difference in one direction, we subtracted the difference going the other way, leaving a constant total.  For geometric sums, we have a different trick.

\begin{example} What is $3 + 6 + 12 + 24 + \cdots + 12288$? 
\begin{solution}
  Multiply each term by the common ratio (2).  You get $2S = 6 + 12 + 24 + \cdots + 24576$.   Now subtract: $2S - S = -3 + 24576 = 24573$.  Since $2S - S = S$, we have our answer. 
  \end{solution}
\end{example} 

To see what happened in the above example, try writing it this way:

\begin{center}
\begin{tabular}{rl}
  $S $& $= 3 + 6 + 12 + 24 + \cdots + 12288$ \\
 $-~~2S$ & $= ~~~~~~6 + 12 + 24 + \cdots + 12288 + 24576 $\\ \hline
 $-S$ &$ = 3 + 0 ~+~ 0 ~+~ 0 ~ +  \cdots + ~~0 ~~ - 24576$
\end{tabular}
\end{center}

Then divide both sides by $-1$ and we have the same result for $S$.  The idea is, by multiplying the sum by the common ratio, each term becomes the next term.  We shift over the sum to get the subtraction to mostly cancel out, leaving just the first term and new last term.

\begin{example}
  Find a closed formula for $S(n) = 2 + 10 + 50 + \cdots + 2\cdot 5^n$.
  \begin{solution}
    The common ratio is 5.  So we have
    
    \begin{center}
\begin{tabular}{rl}
  $S $& $= 2 + 10 + 50 + \cdots + 2\cdot 5^n$ \\
 $-~~5S$ & $= ~~~~~~10 + 50 + \cdots + 2\cdot 5^n + 2\cdot5^{n+1} $\\ \hline
 $-4S$ &$ = 2  - 2\cdot5^{n+1}$
\end{tabular}
\end{center}

Thus $S = \dfrac{2-2\cdot 5^{n+1}}{-4}$
  \end{solution}
\end{example}

Even though this might seem like a new technique, you have probably used it before.  

\begin{example}
  Express $0.464646\ldots$ as a fraction.
  
  \begin{solution}
    Let $N = 0.46464646\ldots$.  Consider $100N$.  We get:
    
    \begin{center}
    \begin{tabular}{rl}
     $N$ &$ = 0.4646464\ldots$\\
     $-0.01N$ &$ = 0.00464646\ldots$\\ \hline
     $0.99N$ & $ = 0.46$
    \end{tabular}
    \end{center}
    So $N = \frac{46}{99}$.  What have we done?  We viewed the repeating decimal $0.464646\ldots$ as a sum of the geometric sequence $0.46, 0.0046, 0.000046, \ldots$  The common ratio is $.01$.  The only real difference is that we are now computing an {\em infinite} geometric sum, we we do not have the extra term on the end to consider.
  \end{solution}

\end{example}


\subsubsection*{$\sum$ and $\prod$ notation}

To simplify writing out sums, we will use notation like $\d\sum_{k=1}^n a_k$.  This means add up the $a_k$s where $k$ changes from 1 to $n$.

\begin{example}
  Use $\sum$ notation to rewrite the sums:
  \begin{enumerate}
    \item $1 + 2 + 3 + 4 + \cdots + 100$
    \item $1 + 2 + 4 + 8 + \cdots + 2^{50}$
    \item $6 + 10 + 14 + \cdots + (4n - 2)$.
  \end{enumerate}
  \begin{solution}
  \begin{multicols}{3}
    \begin{enumerate}
      \item $\d\sum_{k=1}^{100} k$
      \item $\d\sum_{k=0}^{50} 2^k$
      \item $\d\sum_{k=2}^{n} (4k -2)$
    \end{enumerate}
    \end{multicols}
  \end{solution}
\end{example}


If we want to multiply the $a_k$ instead, we would write $\d\prod_{k=1}^n a_k$.  For example, $\d\prod_{k=1}^n k = n!$.



\section{Polynomial Fitting}

So far we have seen methods for finding the closed formulas for certain types of sequences - arithmetic and geometric.  Since we know how to compute the sum of the first $n$ terms of arithmetic and geometric sequences, we can compute the closed formulas for sequences which have an arithmetic (or geometric) sequence of differences between terms.  But what if we consider a sequence which is the sum of the first $n$ terms of a sequence which is the sum of an arithmetic sequence!

Before we get too carried away, let's consider an example: How many squares (of all sizes) are there on a chessboard?  A chessboard consists of $64$ squares, but we also want to consider squares of longer side length.  Even though we are only considering an $8 \times 8$ board, there is already a lot to count.  So instead, let us build a sequence: the first term will be the number of squares on a $1 \times 1$ board, the second term will be the number of squares on a $2 \times 2$ board, and so on.  After a little thought, we arrive at the sequence
\[1,5,14,30, 55,\ldots\]
This sequence is not arithmetic (or geometric for that matter), but perhaps it's sequence of differences is.  For differences we get
\[4, 9, 16, 25, \ldots\]
Not a huge surprise: one way to count the number of squares in a $4 \times 4$ chessboard is to notice that there are $16$ squares with side length 1, 9 with side length 2, 4 with side length 3 and 1 with side length 4.  So the original sequence is just the sum of squares.  Now this sequence of differences is not arithmetic since it's sequence of differences (the differences of the differences of the original sequence) is not constant.  In fact, this sequence of {\em second differences} is
\[5, 7, 9, \ldots\]
which {\em is} an arithmetic sequence - {\em its} sequence of differences is constant (2, 2, 2,\ldots).  Notice that our original sequence had {\em third differences} (that is, differences of differences of differences of the original) constant.  We will call such a sequence $\Delta^3$.  The sequence $1, 4, 9, 16, \ldots$ has second differences constant, so it will be a $\Delta^2$ sequence. In general, we will say a sequence is a $\Delta^k$ sequence if the $k$th differences are constant.

Now $\Delta^0$ sequences are constant, so a closed formula for them is easy to compute (it's just the constant). $\Delta^1$ sequences are arithmetic - we have a method for finding closed formulas for them as well.  $\Delta^2$ sequences are sums of arithmetic sequences - we can find formulas for these as well.  But notice that the format of the closed formula for a $\Delta^2$ sequence is always quadratic - the squares are $\Delta^2$ with closed formula $a_n= n^2$, the triangular numbers (also $\Delta^2$) have closed formula $a_n = \frac{n(n+1)}{2}$, which when multiplied out gives you an $n^2$ term as well.  It appears that every time we increase the complexity of the sequence - that is, increase the number of differences before we get constants - we also increase the degree of the polynomial used for the closed formula.  We go from constant to linear to quadratic.  This makes sense.  The sequence of differences between terms tells us something about the rate of growth of the sequence.  If a sequence is growing at a constant rate, then the formula for the sequence will be linear.  If the sequence is growing at a rate which itself is growing at a constant rate, then the formula is quadratic.  You have seen this elsewhere - if a function has a constant second derivative (rate of change) then the function must be quadratic.

This works in general:

\begin{defbox}{Finite Differences}
The closed formula for a sequence will be a degree $k$ polynomial if and only if the sequence is $\Delta^k$ - that is, the $k$th sequence of differences is constant.
\end{defbox}

This tells us that the sequence $1, 5, 14, 30, 55, \ldots$ will have a cubic (degree 3 polynomial) for its closed formula.  

Now once we know what format the closed formula for a sequence will take, it is much easier to actually find the closed formula.  In the case that the closed formula is a degree $k$ polynomial, we just need $k+1$ data points to ``fit'' the polynomial to the date.

\begin{example}
  Find a formula for the sequence $3, 7, 14, 24,\ldots$. Assume $a_1 = 3$.  
  \begin{solution}
    First, check to see if the formula has constant differences at some level.  The sequence of first differences is $4, 7, 10, \ldots$ which is arithmetic, so the sequence of second differences is constant.  The sequence is $\Delta^2$, so the formula for $a_n$ will be a degree 2 polynomial.  That is, we know that for some constants $a$, $b$, and $c$,
    \[a_n = an^2 + bn + c\]
    Now to find $a$, $b$, and $c$.  First, it would be nice to know what $a_0$ is, since plugging in $n = 0$ simplifies the above formula greatly.  In this case, $a_0 = 2$ (work backwards from the sequence of constant differences).  Thus
    \[a_0 = 2 = a\cdot 0^2 + b \cdot 0 + c\]
    so $c - 2$.  Now plug in $n =1$ and $n = 2$.  We get 
    \[a_1 = 3 = a + b + 2\]
    \[a_2 = 7 = a4 + b 2 + 2.\]  At this point we have two (linear) equations and two unknowns, so we can solve the system for $a$ and $b$.  We find $a = \frac{3}{2}$ and $b = \frac{-1}{2}$, so $a_n = \frac{3}{2} n^2 - \frac{1}{2}n + 2$.
  \end{solution}

\end{example}


\begin{example}
  Find a closed formula for the number of squares on an $n \times n$ chessboard.
  \begin{solution}
    We have seen that the sequence $1, 5, 14, 30, 55, \ldots$ is $\Delta^3$, so we are looking for a degree 3 polynomial.  That is, 
    \[a_n = an^3 + bn^2 + cn + d\]
    We can find $d$ if we know what $a_0$ is.  Working backwards from the third differences, we find $a_0 = 0$ (which makes sense - there are no squares on a $0\times 0$ chessboard).  Thus $d = 0$.  Now plug in $n = 1$, $n =2$, and $n =3$:
    \begin{align*}
      1 = & a + b + c \\
      5 = & 8a + 4b + 2c \\
      14 = & 27a + 9b + 3c
    \end{align*}
    If we solve this system of equations (using elimination, or an augmented matrix, or a computer) we get $a = \frac{1}{3}$, $b = \frac{1}{2}$ and $c = \frac{1}{6}$.  Therefore the number of squares on an $n \times n$ chessboard is $a_n = \frac{1}{3}n^3 + \frac{1}{2}n^2 + \frac{1}{6}n$.
  \end{solution}
  Note: since the squares on a chessboard problem really is asking for the sum of squares, we now have a nice formula for $\d\sum_{k=1}^n k^2$.
\end{example}

Not all sequences will have polynomials as their closed formula.  That is, this process doesn't always work.

\begin{example}
  Determine whether the following sequences can be described by a polynomial, and if so, of what degree.
  \begin{enumerate}
    \item $1, 2, 4, 8, 16, \ldots$
    \item $0, 7, 50, 183, 484, 1055, \ldots$
    \item $1,1,2,3,5,8,13,\ldots$
  \end{enumerate}
\begin{solution}
  \begin{enumerate}
    \item The sequence of first differences is $1, 2, 4, 8, 16\ldots$.  This is identical to the original sequence, so taking any additional differences will not give anything different either.  So there is no number of differences you could take to get constants, so the sequence is not $\Delta^k$ for any $k$.  Therefore the closed formula for the sequence is not a polynomial.  (In fact, we know the closed formula is $a_n = 2^{n-1}$, not a polynomial.)
    \item First differences: $7, 43, 133, 301, 571,\ldots$.  Second differences: $36, 90, 168, 270,\ldots$.  Third difference: $54, 78, 102,\ldots$.  Fourth differences: $24, 24, \ldots$ - constant.  Thus the sequence is $\Delta^4$, so the closed formula is a degree 4 polynomial.
    \item This is the Fibonacci sequence.  The first differences are $0, 1, 1, 2, 3, 5, 8, \ldots$, the second differences are $1, 0, 1, 1, 2, 3,5\ldots$ - we notice that after the first few terms, we get the original sequence back.  So there will never be constant differences, so the closed formula for the Fibonacci sequence is not a polynomial.
  \end{enumerate}

\end{solution}

\end{example}


\section{Solving Recurrence Relations}

We have seen that it is often easier to find recursive definitions than closed formulas.  Lucky for us, there are a few techniques for converting recursive definitions to closed formulas.  Doing so is called solving a {\em recurrence relation}.  Recall that the recurrence relation is the part of a recursive definition besides the initial conditions.  For example, the recurrence relation for the Fibonacci sequence is $F_n = F_{n-1} + F_{n-2}$.  (This, together with the initial conditions $F_0 = 0$ and $F_1 = 1$ give the entire recursive definition for the sequence.)  
 
\begin{example}
  Find a recurrence relation and initial conditions for the sequence $1, 5, 17, 53, 161, 485\ldots$. 
  \begin{solution}
    Finding the recurrence relation would be easier if we had some context for the problem (like the Tower of Hanoi, for example).  Alas, we have only the sequence.  Remember, the recurrence relation tells you how to get from previous terms to future terms.  What is going on here?  We could look at the differences between terms: $4, 12, 36, 108, \ldots$.  Notice that these are growing by a factor of 3.  Is the original sequence as well?  $1\cdot 3 = 3$, $5 \cdot 3 = 15$, $17 \cdot 3 = 51$ and so on.  It appears that we always end up with 2 less than the next term.  Aha!  
    
    So $a_n = 3a_{n-1} + 2$ is our recurrence relation.  The initial condition is $a_1 = 1$.
  \end{solution}

\end{example}

 
We are going to try to {\em solve} the recurrence relations.  By this we mean something very similar to solving differential equations: we want to find a function of $n$ (a closed formula) which satisfies the recurrence relation, as well as the initial condition.  (Note: recurrence relations are sometimes called difference equations since they can describe the difference between terms - this highlights the relation to differential equations further.) Just like for differential equations, finding a solution might be tricky, but checking that the solution is correct is easy - just plug it in.
 
 \begin{example}
    Check that $a_n = 2^n + 1$ is a solution to the recurrence relation $a_n = 2a_{n-1} - 1$ with $a_1 = 3$. 
    \begin{solution}
      First, it is easy to check the initial condition: $a_1$ should be $2^1 + 1$ according to our closed formula.  But $2^1 + 1 = 3$, which is what we want.  To check that our proposed solution satisfies the the recurrence relation, try plugging it in.
      \[2a_{n-1} - 1 = 2(2^{n-1} + 1) - 1 = 2^n + 2 - 1 = 2^n +1 = a_n\]
      That's what our recurrence relation says!  We have a solution.
    \end{solution}

 \end{example}

 
We will consider three techniques for solving recurrence relations: Telescoping, Iteration and the Characteristic Root Technique.

\subsection{Telescoping}

Telescoping refers to the phenomenon when many terms in a large sum cancel out - so the sum ``telescopes.''  For example:
\[(2 - 1) + (3 - 2) + (3 - 4) + \cdots + (100 - 99) + (101 - 100) = -1 + 101\]
because every third term looks like: $2 + -2 = 0$, the $3 + -3 = 0$ and so on.

We can use this behavior to solve recurrence relations.  Here is an example.

\begin{example}
  Solve the recurrence relation $a_n = a_{n-1} + n$ with initial term $a_0 = 4$.  
  
  \begin{solution}
    To get a feel for the recurrence relation, write out the first few terms of the sequence: $4, 5, 7, 10, 14, 19, \ldots$.  Look at the difference between terms.  $a_1 - a_0 = 1$ and $a_2 - a_1 = 2$ and so on.  The key thing here is that the difference between terms is $n$.  We can write this explicitly: $a_n - a_{n-1} = n$.  Of course, we could have arrived at this conclusion directly from the recurrence relation - just subtract $a_{n-1}$ from both sides.
    
    Now use this equation over and over again, changing $n$ each time:
    
    \begin{align*}
      a_1 - a_0 &= 1\\
      a_2 - a_1 &= 2\\
      a_3 - a_2 & = 3\\
      \vdots \quad & \quad \vdots \\
      a_n - a_{n-1} & = n
    \end{align*}
  Now add all these equations together.  On the right hand side, we get the sum $1 + 2 + 3 + \cdots + n$.  We already know that that can be simplified to $\frac{n(n+1)}{2}$.  What happens on the left hand side?  We get 
  \[(a_1 - a_0) + (a_2 - a_1) + (a_3 - a_2) + \cdots (a_{n-1} - a_{n-2})+ (a_n - a_{n-1})\]
  This sum telescopes.  We are left with only the $-a_0$ from the first equation and the $a_n$ from the last equation.  Putting this all together we have $-a_0 + a_n = \frac{n(n+1)}{2}$ or $a_n = \frac{n(n+1)}{2} + a_0$.  But we know that $a_0 = 4$.  So the solution to the recurrence relation, subject to the initial condition is
  \[a_n = \frac{n(n+1)}{2} + 4\]
  (Now that we know that, we should notice that yes, the sequence is indeed the result of adding 4 to each of the triangular numbers.)
  \end{solution}

\end{example}

The above example shows a way to solve recurrence relations of the form $a_n = a_{n-1} + f(n)$ where $\sum_{k = 1}^n f(k)$ has a known closed formula.  If you rewrite the recurrence relation as $a_n - a_{n-1} = f(n)$, and then add up all the different equations with $n$ ranging between 1 and $n$, the left hand side will always give you $a_n - a_0$.  The right hand side will be $\sum_{k = 1}^n f(k)$, which is why we need to know the closed formula for that sum.

However, telescoping will not help us with a recursion such as $a_n = 3a_{n-1} + 2$ - the left hand side will not telescope, since you will have $-3a_{n-1}$'s but only one $a_{n-1}$.  We need another method for this.  

\subsection{Iteration}

We have already seen an example of iteration when we found the closed formula for arithmetic and geometric sequences.  The idea is, we {\em iterate} the process of finding the next term, starting with the known initial condition, up until we have $a_n$.  Then we simplify.  In the arithmetic sequence example, we simplified by multiplying $d$ by the number of times we add it to $a$ when we get to $a_n$, to get from $a_n = a + d + d + d + \cdots + d$ to $a_n = a + d(n-1)$.

To see how this works, let's go through the same example we used for telescoping, but this time use iteration.

\begin{example}
  Use iteration to solve the recurrence relation $a_n = a_{n-1} + n$ with $a_0 = 4$.
  
  \begin{solution}
    Again, start by writing down the recurrence relation when $n = 1$.  This time, don't subtract the $a_{n-1}$ terms to the other side.
    \[a_1 = a_0 + 1\]
    Now $a_2 = a_1 + 2$, but we know what $a_1$ is.  We get
    \[a_2 = (a_0 + 1) + 2\]
    Now go to $a_3 = a_2 + 3$, using our known value of $a_2$:
    \[a_3 = ((a_0 + 1) + 2) + 3\]
    We notice a pattern.  Each time, we take the previous term and add the current index.  So
    \[a_n = ((((a_0 + 1) +2)+3)+\cdots + n-1) + n\]
    Regrouping terms, we notice that $a_n$ is just $a_0$ plus the sum of the integers from $1$ to $n$.  So, since $a_0 = 4$, 
    \[a_n = 4 + \frac{n(n+1)}{2}\]
  \end{solution}

\end{example}

 Of course in this case we still needed to know formula for the sum of $1,\ldots,n$.  Let's try iteration with a sequence for which telescoping doesn't work.
 
 \begin{example}
   Solve the recurrence relation $a_n = 3a_{n-1} + 2$ subject to $a_0 = 1$.
   \begin{solution}
     Again, we iterate the recurrence relation, building up to the index $n$.
     \begin{align*}
      a_1 &= 3a_0 + 2\\
      a_2 &= 3(a_1) + 2 = 3(3a_0 + 2) + 2\\
      a_3 &= 3[a_2] + 2 = 3[3(3a_0 + 2) + 2] + 2\\
      \vdots & \qquad \vdots \qquad \qquad \vdots \\
      a_n &= 3(a_{n-1}) + 2 = 3(3(3(3\cdots(3a_0 + 2) + 2) + 2)\cdots + 2)+ 2
     \end{align*}
	 It is difficult to see what is happening here because we have a lot of distributing the 3's to do.  Let' try again, this time simplifying a bit as we go.
	      \begin{align*}
      a_1 &= 3a_0 + 2\\
      a_2 &= 3(a_1) + 2 = 3(3a_0 + 2) + 2 = 3^2a_0 + 2\cdot 3 + 2\\
      a_3 &= 3[a_2] + 2 = 3[3^2a_0 + 2\cdot 3 + 2] + 2 = 3^3 a_0 + 2 \cdot 3^2 + 2 \cdot 3 + 2\\
      \vdots & \qquad \vdots \qquad \qquad \vdots \\
      a_n &= 3(a_{n-1}) + 2 = 3(3^{n-1}a_0 + 2 \cdot 3^{n-2} + \cdots +2)+ 2\\
      & \qquad \qquad = 3^n a_0 + 2\cdot 3^{n-1} + 2 \cdot 3^{n-2} + \cdots + 2\cdot 3 + 2
     \end{align*}
    Now we simplify.  $a_0 = 1$, so we have $3^n + (\cdots)$.  Note that every other term has a 2 in it - in fact, we have a geometric sum with first term $2$ and common ratio $3$.  We have seen how to simplify $2 + 2\cdot 3 + 2 \cdot 3^2 + \cdots + 2\cdot 3^{n-1}$.  We get $\frac{2-2\cdot 3^n}{-2}$ which simplifies to $3^n - 1$.  Putting this together with the first $3^n$ term gives our closed formula:
    \[a_n = 2\cdot 3^n - 1\]
   \end{solution}
 \end{example}


 Iteration can be messy, but when the recurrence relation only refers to one previous term (and maybe some function of $n$) it can work well.  However, trying to iterate a recurrence relation such as $a_n = 2 a_{n-1} + 3 a_{n-2}$ will be way to complicated - we would need to keep track of two sets of previous terms, each of which were expressed by two previous terms, and so on.  The length of the formula would grow exponentially (double each time, in fact).  And anyway, there happens to be a third method for solving recurrence relations which works very well on relations like this.



\subsection{The Characteristic Root Technique}

Suppose we want to solve a recurrence relation expressed as a combination of the two previous terms, such as $a_n = a_{n-1} + 6a_{n-2}$. In other words, we want to find a function of $n$ which satisfies $a_n - a_{n-1} - 6a_{n-2} = 0$.  Now iteration is too complicated, but think just for a second what would happen if we {\em did} iterate.  In each step, we would, among other things, multiply a previous iteration by 6.   So our closed formula would have include $6$ multiplied some number of times.  Thus it is reasonable to guess the solution will contain parts that look geometric.  So perhaps the solution will take the form $r^n$. for some constant $r$.
 
The nice thing is, we know how to check whether a formula is actually a solution to a recurrence relation - plug it in.  What happens if we plug in $r^n$ into the recursion above? We get  \[r^n - r^{n-1} - 6r^{n-2} = 0\]  Now solve for $r$: \[r^{n-2}(r^2 - r - 6) = 0\]
so by factoring, $r = -2$ or $r = 3$.  This tells us that $a_n = (-2)^n$ is a solution to the recurrence relation, as is $a_n = 3^n$.  Which one is correct?  They both are, unless we specify initial conditions.  Notice we could also have $a_n = (-2)^n + 3^n$.  Or $a_n = 7(-2)^n + 4\cdot 3^n$.  In fact, for any $a$ and $b$, $a_n = a(-2)^n + b 3^n$ is a solution - try plugging that into the recurrence relation.  To find the values of $a$ and $b$, use the initial conditions.
 
The points us in the direction of a more general technique for solving recurrence relations.  Notice will will always be able to factor out the $r^{n-2}$ as we did above.  So we really only care about the other part.  We call this other part the {\em characteristic equation} for the recurrence relation.  We are interested in finding the roots of the characteristic equation, which are called (surprise) the characteristic roots.  

\begin{defbox}{Characteristic Roots}
 Given a recurrence relation $a_n + \alpha a_{n-1} + \beta a_{n-2} = 0$, the characteristic polynomial is
 \[x^2 + \alpha x + \beta\]
 giving the {\em characteristic equation}:
 \[x^2 + \alpha x + \beta = 0\]
 If $r_1$ and $r_2$ are two distinct roots of the characteristic polynomial (i.e, solutions to the characteristic equation), then the solution to the recurrence relation is
 \[a_n = ar_1^n + br_2^n\]
 where $a$ and $b$ are constants determined by the initial conditions.
\end{defbox}

\begin{example}
  Solve the recurrence relation $a_n = 7a_{n-1} - 10 a_{n-2}$ with $a_0 = 2$ and $a_1 = 3$. 
  \begin{solution}
   Rewrite the recurrence relation $a_n - 7a_{n-1} + 10a_{n-2} = 0$.  Now form the characteristic equation:
   \[x^2 - 7x + 10 = 0\]
   and solve for $x$: 
   \[(x - 2) (x - 5) = 0\]
   so $x = 2$ and $x = 5$ are the characteristic roots.  We therefore know that the solution to the recurrence relation will have the form
   \[a_n = a 2^n + b 5^n\]  
   To find $a$ and $b$, plug in $n =0$ and $n = 1$ to get a system of two equations with two unknowns:
   \begin{align*}
    2 & = a 2^0 + b 5^0 = a + b\\
    3 & = a 2^1 + b 5^1 = 2a + 5b
   \end{align*}
  Solving this system gives $a = \frac{7}{3}$ and $b = -\frac{1}{3}$ so the solution to the recurrence relation is
  \[a_n = \frac{7}{3}2^n - \frac{1}{3} 3^n\]
  \end{solution}

\end{example}

Perhaps the most famous recurrence relation is $F_n = F_{n-1} + F_{n-2}$, which together with the initial conditions $F_0 = 0$ and $F_1= 1$ defines the Fibonacci sequence.  But notice that this is precisely the type of recurrence relation on which we can use the characteristic root technique.  When you do, the only thing that changes is that the characteristic equation does not factor - you need to use the quadratic formula to find the characteristic roots.  In fact, doing so gives the third most famous irrational number - $\varphi$, the golden ratio.

Before leaving the characteristic root technique, we should think about what all might happen when you solve the characteristic equation.  Above we have an example in which the characteristic polynomial has two distinct roots.  These roots can be integers, or perhaps irrational numbers (requiring the quadratic formula to find them).  In these cases, we know what the solution to the recurrence relation looks like.  

However, it is possible for the characteristic polynomial to only have one root - say the characteristic polynomial factors as $(x - r)^2$.  It is still the case that $r^n$ would be a solution to the recurrence relation, but we won't be able to find solutions for all initial conditions using the general form $a_n = ar_1^n + br_2^n$, since we can't distinguish between $r_1^n$ and $r_2^n$ - they are the same (repeated) root.  We are in luck though:

\begin{defbox}{Characteristic Root Technique for Repeated Roots}
 Suppose the recurrence relation $a_n = \alpha a_{n-1} + \beta a_{n-2}$ has a characteristic polynomial with only one root $r$.  Then the solution to the recurrence relation is
 \[a_n = ar^n + bnr^n\]
 where $a$ and $b$ are constants determined by the initial conditions.
\end{defbox}

Notice the extra $n$ in $bnr^n$.  This allows us to solve for the constants $a$ and $b$ from the initial conditions.

\begin{example}
 Solve the recurrence relation $a_n = 6a_{n-1} - 9a_{n-2}$ with initial conditions $a_0 = 1$ and $a_1 = 4$.  
 \begin{solution}
  The characteristic polynomial is $x^2 - 6x + 9$.  We solve the characteristic equation
  \[x^2 - 6x + 9 = 0\]
  by factoring:
  \[(x - 3)^2 = 0\]
  so $x =3$ is the only characteristic root.  Therefore we know that the solution to the recurrence relation has the form
  \[a_n = a 3^n + bn3^n\]
  for some constants $a$ and $b$.  Now use the initial conditions:
  \begin{align*}
   a_0 = 1 &= a 3^0 + b\cdot 0 \cdot 3^0 = a\\
   a_1 = 4 &= a\cdot 3 + b\cdot 1 \cdot3 = 3a + 3b\\
  \end{align*}
  Since $a = 1$, we find that $b = \frac{1}{3}$.  Therefore the solution to the recurrence relation is
  \[a_n = 3^n + \frac{1}{3}n3^n\]
 \end{solution}


 
\end{example}

 Although we will not consider examples more complicated than these, this characteristic root technique can be applied to much more complicated recurrence relations.  For example, $a_n = 2a_{n-1} + a_{n-2} - 3a_{n-3}$ has characteristic polynomial $x^3 - 2 x^2 - x + 3$.  Assuming you see how to factor such a degree 3 (or more) polynomial you can easily find the characteristic roots and as such solve the recurrence relation (the solution would look like $a_n = ar_1^n + br_2^n + cr_3^n$ if there were 3 distinct roots).  It is also possible to solve recurrence relations of the form $a_n = \alpha a_{n-1} + \beta a_{n-2} + C$ for some constant $C$.  Additionally, if the characteristic roots could be complex numbers - this is acceptable as well (and there are tricks to deal with this).



\end{document}
