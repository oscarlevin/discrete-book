\documentclass[12pt]{article}

\usepackage{discrete}

\def\thetitle{Introduction to Graph Theory} % will be put in the center header on the first page only.
\def\lefthead{Math 228 Notes} % will be put in the left header
\def\righthead{\thetitle} % will be put in the right header



\begin{document}
Graph Theory is a relatively new area of mathematics, first studied by the super famous mathematician Leonhard Euler in 1735.  Since then it has blossomed in to a powerful tool used in nearly every branch of science and is currently an active area of mathematics research.  We will begin our study with the problem that started it all: The Seven Bridges of K\"onigsberg.
\todo{fix activity environment to make spacing above environment consistent for sections and for paragraphs}

\begin{activity} 
In the time of Euler, in the town of K\"onigsberg in Prussia, there was a river containing two islands.  The islands were connected to the banks of the river by seven bridges (as seen below).  The bridges were very beautiful, and on their days off, townspeople would spend time walking over the bridges.  As time passed, a question arose: was it possible to plan a walk so that you cross each bridge once and only once?  Euler was able to answer this question.  Are you?

\centerline{\includegraphics[width=5in]{images/7bridgescolor.png}}


Try finding a path which uses each bridge exactly once.  
\end{activity}
\todo{recreate this image with tikz}
Here is another problem:  below is a drawing of four dots connected by some lines.  Is it possible to trace over each line once and only once (without lifting up your pencil)?  You must start and end on one of the dots.

\centerline{\begin{tikzpicture}[scale=1.5, yscale=.5]
 \draw[thick] (-1,-2) \v to [out=120, in=240] (-1,0) \v to [out=120, in=240] (-1,2) \v to [out=300, in=60] (-1,0) to [out=300, in=60] (-1,-2);
  \draw[thick] (1,0) \v -- (-1,2) (-1,0) -- (1,0) -- (-1,-2);
  \end{tikzpicture}}

There is an obvious connection between these two problems - any path through the dot and line drawing corresponds exactly to a path over the bridges of K\"onigsberg.  


Pictures like the dot and line drawing are called {\em graphs}.  Graphs are made up of a collection of dots (called {\em vertices}) and lines connecting those dots (called {\em edges}).  The nice thing about looking at graphs instead of pictures of rivers, islands and bridges is that we now have a mathematical object to study.  A {\em set} of vertices and a set of edges (in fact, we can take the set of edges to be a set of two element subsets from the set of vertices).  We have distilled the ``important'' parts of the bridge picture for the purposes of the problem - it does not matter how big the islands are, what the bridges are made out of, if the river contains alligators, etc.  All that matters is which land masses are connected to which other land masses, and how many times.  This was the great insight that Euler had.

We will return to the question of finding paths through graphs later.  But first, here are a few other situations you can represent with graphs.

\begin{example}
  Al, Bob, Cam, Dan, and Euclid are all members of the social networking website {\em Facebook}.  The site allows members to be ``friends'' with each other.  It turns out that Al and Cam are friends, as are Bob and Dan.  Euclid is friends with everyone.  Represent this situation with a graph.
  \begin{solution}
    Each person will be represented by a vertex and each friendship will be represented by an edge - that is, there will be an edge between two vertices if and only if the people represented by those vertices are friends.  We get the following graph:
    
    \begin{center}
      \begin{tikzpicture}[scale=0.75]
        \draw[thick] (-1, 0) \vl{C} -- (0,1) \vb{E} -- (-1,2) \vl{A} -- (-1,0)(1,0) \vr{D} -- (0,1)  -- (1,2) \vr{B} -- (1,0);
      \end{tikzpicture}
    \end{center}
  \end{solution}

\end{example}


\begin{example}
  Each of three houses must be connected to each of three utilities.  Is it possible to do this without any of the utility lines crossing?
  \begin{solution}
    We will answer this question later.  For now, notice how we would ask this question in the context of graph theory.  We are really asking whether it is possible to redraw the graph below without any edges crossing (except at vertices).
    
    \begin{center}
      \begin{tikzpicture}[yscale=1.2]
        \draw[thick] (-1,1) \v -- (-1,0)\v  -- (0,1) \v -- (0,0) \v -- (1,1) \v -- (1,0) \v -- (0,1) -- (-1,0) -- (1,1) (1,0) -- (-1,1) -- (0,0);
      \end{tikzpicture}
    \end{center}

  \end{solution}

\end{example}

\section{Basics}


\begin{activity}

\begin{questions}
\question Which (if any) of the graphs below are the same?


\begin{tikzpicture}
\coordinate (A) at (-1,0);
\coordinate (B) at (0,0);
\coordinate (C) at (1,0);
\coordinate (D) at (-.5,1);
\coordinate (E) at (.5,1);

\draw (A) -- (D) -- (B) -- (E) -- (C) -- (D)  (A) -- (E);
\foreach \x in {(A), (B), (C), (D), (E)}{
	\fill \x \v;
}
\end{tikzpicture}
\hfill
\begin{tikzpicture}
\coordinate (A) at (90+360/5:1);
\coordinate (B) at (90+2*360/5:1);
\coordinate (C) at (90+3*360/5:1);
\coordinate (D) at (90+4*360/5:1);
\coordinate (E) at (90:1);

\draw (A) -- (B) -- (C) -- (D) -- (E) -- (A);
\foreach \x in {(A), (B), (C), (D), (E)}{
	\fill \x \v;
}
\end{tikzpicture}
\hfill
\begin{tikzpicture}
\coordinate (A) at (-1,0);
\coordinate (B) at (0,1);
\coordinate (C) at (0,0);
\coordinate (D) at (0,-1);
\coordinate (E) at (1,0);

\draw (A) -- (B) -- (E) -- (C) -- (A) -- (D) -- (E);
\foreach \x in {(A), (B), (C), (D), (E)}{
	\fill \x \v;
}
\end{tikzpicture}
\hfill
\begin{tikzpicture}
\coordinate (A) at (90+360/5:1);
\coordinate (B) at (90+2*360/5:1);
\coordinate (C) at (90+3*360/5:1);
\coordinate (D) at (90+4*360/5:1);
\coordinate (E) at (90:1);

\draw (A) -- (C) -- (E) -- (B) -- (D) -- (A);
\foreach \x in {(A), (B), (C), (D), (E)}{
	\fill \x \v;
}
\end{tikzpicture}
\hfill
\begin{tikzpicture}
\coordinate (A) at (-1,0);
\coordinate (B) at (0,1);
\coordinate (C) at (0,0);
\coordinate (D) at (0,-1);
\coordinate (E) at (1,0);

\draw (A) -- (C) -- (B) (D) -- (C) -- (E);
\foreach \x in {(A), (B), (C), (D), (E)}{
	\fill \x \v;
}
\end{tikzpicture}
\hfill
~



\question The graphs above are unlabeled.  Usually we think of a graph as having a specific set of vertices.  Which (if any) of the graphs below are the same?

\begin{tikzpicture}
\coordinate (A) at (-1,0);
\coordinate (B) at (-1, 1);
\coordinate (C) at (0,0);
\coordinate (D) at (0,1);
\coordinate (E) at (1,0);
\coordinate (F) at (1,1);

\draw (A) node[below] {$a$} -- (B) node[above] {$b$} -- (C) node[below] {$c$} -- (D) node[above] {$d$} -- (E) node[below] {$e$} -- (F) node[above] {$f$} -- (C) (A) -- (D);
\foreach \x in {(A), (B), (C), (D), (E), (F)}{
	\fill \x \v;
}
\end{tikzpicture}
\hfill
\begin{tikzpicture}
\coordinate (A) at (-1,0);
\coordinate (B) at (-1, 1);
\coordinate (C) at (0,1);
\coordinate (D) at (0,0);
\coordinate (E) at (1,0);
\coordinate (F) at (1,1);

\draw (A) node[below] {$a$} -- (B) node[above] {$b$} -- (C) node[above] {$c$} -- (D) node[below] {$d$} -- (E) node[below] {$e$} -- (F) node[above] {$f$} -- (C) (A) -- (D);
\foreach \x in {(A), (B), (C), (D), (E), (F)}{
	\fill \x \v;
}
\end{tikzpicture}
\hfill
\begin{tikzpicture}
\coordinate (A) at (-1,0);
\coordinate (B) at (-1, 1);
\coordinate (C) at (0,0);
\coordinate (D) at (0,1);
\coordinate (E) at (1,0);
\coordinate (F) at (1,1);

\draw (A) node[below] {$a$} -- (B) node[above] {$c$} -- (C) node[below] {$e$} -- (D) node[above] {$b$} -- (E) node[below] {$d$} -- (F) node[above] {$f$} -- (C) (A) -- (D);
\foreach \x in {(A), (B), (C), (D), (E), (F)}{
	\fill \x \v;
}
\end{tikzpicture}
\hfill
\begin{tikzpicture}
\coordinate (A) at (-1,0);
\coordinate (B) at (-1, 1);
\coordinate (C) at (0,1);
\coordinate (D) at (0,0);
\coordinate (E) at (1,0);
\coordinate (F) at (1,1);

\draw (A) node[below] {$v_6$} -- (B) node[above] {$v_1$} -- (C) node[above] {$v_2$} -- (D) node[below] {$v_5$} -- (E) node[below] {$v_4$} -- (F) node[above] {$v_3$} -- (C) (A) -- (D);
\foreach \x in {(A), (B), (C), (D), (E), (F)}{
	\fill \x \v;
}
\end{tikzpicture}



\question Actually, all the graphs we have seen above are just {\em drawings} of graphs.  A graph is really an abstract mathematical object consisting of two sets $V$ and $E$ where $E$ is a set of 2-element subsets of $V$.

Are the graphs below the same or different?

Graph 1: $V = \{a, b, c, d, e\}$, $E = \{\{a,b\}, \{a, c\}, \{a,d\}, \{a,e\}, \{b,c\}, \{d,e\}\}$.


Graph 2: $V = \{v_1, v_2, v_3, v_4, v_5\}$,\\ $E = \{\{v_1, v_3\}, \{v_1, v_5\}, \{v_2, v_4\}, \{v_2, v_5\}, \{v_3, v_5\}, \{v_4, v_5\}\}$.
\end{questions}

\end{activity}


While we almost always think of graphs as pictures, these are really just visual representations of mathematical objects.  In fact, a graph is simply a set of vertices some pairs of which are ``related'' by an edge.  For example, we can describe a particular graph like this: the vertices are the letters $\{a,b,c,d\}$ and the edges are the pairs $\{\{a,b\}, \{a,c\}, \{b,c\}, \{b,d\}, \{c,d\}\}$.  %Technically the edges are 2-elements subsets of the set of vertices, so we should write $(a,b)$ as $\{a,b\}$ but we use parentheses to ease reading.  
\todo{Make sure edge sets are written consistently throughout chapter}
We could have described the graph in words as follows: we have four vertices, $a$, $b$, $c$, and $d$, and $a$ is connected (by an edge) to $b$ and $c$, $b$ is connected to both $c$ and $d$, and $c$ is also connected to $d$.  One way to draw this graph is this:
\begin{center}
  \begin{tikzpicture}[scale=0.7]
    \draw[thick] (-1,1) \vl{$a$} -- (1,1) \vr{$b$} (-1,1) -- (-1,-1) \vl{$c$} -- (1,-1) \vr{$d$} -- (1,1) -- (-1,-1);
  \end{tikzpicture}
\end{center}

However we could also have drawn the graph differently.  For example either of these:

\begin{center}
  \begin{tikzpicture}[scale=0.7]
    \draw[thick] (-1,1) \vl{$a$} -- (1,-1) \vr{$b$} (-1,1) -- (-1,-1) \vl{$c$} -- (1,1) \vr{$d$} -- (1,-1) -- (-1,-1);
  \end{tikzpicture}
  \hspace{1in}
    \begin{tikzpicture}[scale=0.7]
    \draw[thick] (-1.5,0) \vb{$a$} -- (-.5,0) \vb{$b$} (-1.5,0) .. controls (-.5,1) .. (.5,0) \vb{$c$} -- (1.5,0) \vb{$d$} .. controls (.5,1) .. (-.5,0) -- (.5,0);
  \end{tikzpicture}
\end{center}

Viewed as pictures, the graphs above are the same whether or not the vertices are labeled as they are, or at all.  In fact, two graphs are equal precisely when there is a way to label the vertices so that all the pairs of vertices which have edges in one graph also have edges in the other graph, and \textit{vice versa}.  Two vertices which are connected by an edge are called \emph{adjacent}.

Notice that the graph above has the property that no pair of vertices is connected more than once, and no vertex is connected to itself.  Graphs like these are sometimes called {\em simple}, although we will just call them {\em graphs}.  This is because our definition for a graph says that the edges form a set of 2-element subsets of the vertices.  Remember that it doesn't make sense to say a set contains an element more than once.  So no pair of vertices can be connected by an edge more than once.  And since each edge must be a set containing two vertices, we cannot have a single vertex connected to itself by an edge.  If we did allow either of these to happen, then our set of edges would in fact be a \emph{multiset}.  Thus graphs which contain multiple edges between a single pair of vertices are often called \emph{multigraphs}.

There are times we want to consider double (or more) edges and single edge loops.  For example, the ``graph'' we drew for the Bridges of K\"onigsberg problem had double edges because there really are two bridges connecting a particular island to the near shore.  We will call these objects {\em multigraphs}.  This is a good name: a multiset is a set in which we are allowed to include a single element multiple times.

The graph above is also {\em connected} - you can get from any vertex to any other vertex by following some path of edges ($a$ and $d$ are connected by a path two edges long).  A graph that is not connected can be thought of as two separate graphs drawn close together.  Unless otherwise stated, we will assume all graphs are connected.  

The graph above does not have an edge between $a$ and $d$.  Thus it is possible to add an edge to the graph.  If we add all possible edges, then the resulting graph is said to be {\em complete}.  That is, a graph is complete if every pair of vertices is connected by an edge.  Since a graph is determined completely by which vertices are adjacent to which other vertices, there is only one complete graph with a given number of vertices.  We give these a special name: \gls{Kn} is the complete graph on $n$ vertices.

Each vertex in $K_n$ is adjacent to $n-1$ other vertices.  We call the number of edges emanating from a given vertex the {\em degree} of that vertex.  So every vertex in $K_n$ has degree $n-1$.  How many edges does $K_n$ have?  One might think the answer should be $n(n-1)$, since we count $n-1$ edges $n$ times (once for each vertex).  However, each edge is adjacent to 2 vertices, so we count every edge exactly twice in this way.  Thus there are $n(n-1)/2$ edges in $K_n$.  
Alternatively, we can say there are ${n \choose 2}$ edges, since to draw an edge we must choose 2 of the $n$ vertices.

In general, if we know the degrees of all the vertices in a graph we can find the number of edges.  The sum of the degrees of all vertices will always be twice the number of edges, since each edge adds to the degree of two vertices.  Notice this means that the sum of the degrees of all vertices in any graph must be even!

\begin{example}
  At a recent math seminar, 9 mathematicians greeted each other by shaking hands.  Is it possible that each mathematician shook hands with exactly 7 people at the seminar?
  \begin{solution}
    It seems like this should be possible - each mathematician chooses one person to not shake hands with.  But it is impossible.  We are asking whether a graph with 9 vertices can have each vertex have degree 7.  If such a graph existed, the sum of the degrees of the vertices would be $9\cdot 7 = 63$.  This would be twice the number of edges (handshakes) so this says that the graph would have $31.5$ edges.  That is impossible - edges only come in whole numbers.  Thus at least one (in fact an odd number) of the mathematicians must have shaken hands with an {\em even} number of people at the seminar.
  \end{solution}

\end{example}

One final definition: we say a graph is {\em bipartite} if the vertices can be divided into two sets, $A$ and $B$, with no two vertices in $A$ adjacent and no two vertices in $B$ adjacent.  Of course the vertices in $A$ can be adjacent to some or all of the vertices in $B$.  If all vertices in $A$ and adjacent to all the vertices in $B$, then the graph is a {\em complete bipartite graph}, and gets a special name: $K_{m,n}$, where $|A| = m$ and $|B| = n$.  The graph in the houses and utilities puzzle is $K_{3,3}$.

\begin{defbox}{Named Graphs}
  Some graphs are used more than others, and get special names.
  \begin{itemize}
    \item \gls{Kn}: the complete graph on $n$ vertices.
    \item $K_{m,n}$: the complete bipartite graph with sets of $m$ and $n$ vertices.
    \item $C_n$: the cycle graph on $n$ vertices - just one big loop.
    \item $P_n$: the path graph on $n$ vertices - just one long path.
  \end{itemize}
 % Here are some typical examples:
  
\def\sb{.6}
\begin{center}
\hfill
\begin{tikzpicture}[scale=\sb+.05]
  \path (0,.9) +(18:1) coordinate (a);
  \path (0,.9) +(90:1) coordinate (b);
  \path (0,.9) +(162:1) coordinate (c);
  \path (0,.9) +(234:1) coordinate (d);
  \path (0,.9) +(306:1) coordinate (e);
  \draw[thick] (a) \v -- (b) \v -- (c) \v -- (d) \v -- (e) \v -- (a) -- (c) -- (e) -- (b) -- (d) -- (a);
  \draw (0,-.5) node[below]{\large $K_5$};
\end{tikzpicture}
\hfill
\begin{tikzpicture}[scale=\sb, xscale=1.5]
 \draw[thick] (-1, 0) \v -- (-.5,2) \v -- (0,0) \v -- (.5, 2) \v -- (1,0) \v -- (-.5,2) (.5,2) -- (-1,0);
 \draw (0,-.5) node[below]{\large $K_{2,3}$};
  \end{tikzpicture}
\hfill
\begin{tikzpicture}[scale=\sb]
  \draw[thick] (0:1) \v -- (60:1) \v -- (120:1) \v -- (180:1) \v -- (240:1) \v -- (300:1) \v -- cycle;
  \draw (270:1.5) node[below]{\large $C_6$};
\end{tikzpicture}
\hfill
\begin{tikzpicture}[scale=\sb]
  \draw[thick] (-2,0) \v -- (-1,.5) \v -- (0,0) \v -- (1,.75) \v -- (.5,1.5) \v -- (2,2) \v;
  \draw (0,-.5) node[below]{\large $P_6$};
\end{tikzpicture}
\hfill
~
\end{center}
\end{defbox}
\newpage
\begin{defbox}{Graph Theory Definitions}

  \begin{itemize}
    \item[] \textbf{Graph}: A collection of {\em vertices}, some of which are connected by {\em edges}.  More precisely, a pair of sets $V$ and $E$ where $V$ is a set of vertices and $E$ is a set of 2-element subsets of $V$.
    \item[] \textbf{Adjacent}: Two vertices are {\em adjacent} if they are connected by an edge.  Two edges are {\em adjacent} if they share a vertex.
    \item[] \textbf{Bipartite graph}: A graph for which it is possible to divide the vertices into two disjoint sets such that there are no edges between any two vertices in the same set.
    \item[] \textbf{Complete bipartite graph}: A bipartite graph for which every vertex in the first set is adjacent to every vertex in the second set.
    \item[] \textbf{Complete graph}: A graph with edges connecting every pair of vertices.
    \item[] \textbf{Connected}: A graph is {\em connected} if there is a path from any vertex to any other vertex.     
 %   \item[] \textbf{Chromatic number}: The minimum number of colors required in a proper vertex coloring of the graph.
    \item[] \textbf{Cycle}: A path (see below) that starts and stops at the same vertex, but contains no other repeated vertices.
    \item[] \textbf{Degree of a vertex}: The number of edges connected to a vertex is called the {\em degree} of the vertex.
    \item[] \textbf{Euler path}: A path which uses each edge exactly once.
    \item[] \textbf{Euler circuit}: An Euler path which starts and stops at the same vertex. 
    \item[] \textbf{Multigraph}: A {\em multigraph} is just like a graph but can contain multiple edges between two vertices as well as single edge loops (that is an edge from a vertex to itself).
    \item[] \textbf{Path}: A sequence of vertices such that consecutive vertices (in the sequence) are adjacent (in the graph).  A path in which no vertex is repeated is called {\em simple}.
    \item[] \textbf{Planar}: A graph is planar if it is possible to draw it (in the plane) without any edges crossing.
    \item[] \textbf{Subgraph}: We say that $H$ is a subgraph of $G$ if every vertex and edge of $H$ is also a vertex or edge of $G$.  We say $H$ is an {\em induced} subgraph of $G$ if every vertex of $H$ is a vertex of $G$ and for pair of vertices in $H$ are adjacent in $H$ if and only if they are adjacent in $G$.
    \item[] \textbf{Tree}: A (connected) graph with no cycles.  (A non-connected graph with no cycles is called a {\em forest}.)  The vertices in a tree with degree 1 are called {\em leaves}.
%    \item[] \textbf{Vertex coloring}: An assignment of colors to each of the vertices of a graph. A vertex coloring is {\em proper} if adjacent vertices are always colored differently.
  \end{itemize}

\end{defbox}

\begin{activity}
For each of the following, try to give two \underline{different} unlabeled graphs with the given properties, or explain why doing so is impossible.
\begin{questions}
\question Two different trees with the same number of vertices and the same number of edges.

\question Two different graphs with 8 vertices all of degree 2.

\question Two different graphs with 5 vertices all of degree 4.
\question Two different graphs with 5 vertices all of degree 3.




\end{questions}
\end{activity}
\end{document}


