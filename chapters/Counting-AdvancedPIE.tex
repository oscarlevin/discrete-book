\documentclass[12pt]{article}

\usepackage{discrete}

\def\thetitle{Advanced Counting} % will be put in the center header on the first page only.
\def\lefthead{Math 228 Notes} % will be put in the left header
\def\righthead{\thetitle} % will be put in the right header




\begin{document}




\section{Advanced Counting using PIE}\label{sec-advPIE}
\begin{activity}

\begin{questions}
\question You have 11 identical mini key-lime pies to give to 4 children.  However, you don't want any kid to get more than 3 pies.  How many ways can you distribute the pies?
\begin{parts}
\part How many ways are there to distribute the pies without any restriction?

\part Let's get rid of the ways that one or more kid gets too many pies.  How many ways are there to distribute the pies if Al gets too many pies?  What if Bruce gets too many?  Or Cat?  Or Dent?

\part What if two kids get too many pies?  How many ways can this happen?  Does it matter which two kids you pick to overfeed?

\part Is it possible that three kids get too many pies?  If so, how many ways can this happen?

\part How should you combine all the numbers you found above to answer the original question?

\end{parts}
\question Suppose now you have 13 pies and 7 children.  No child can have more than 2 pies.  How many ways can you distribute the pies?



\end{questions}
\end{activity}


Stars and Bars allows us to count the number of ways to distribute 10 cookies to 3 kids and natural number solutions to $x+y+z = 10$, for example.  A relatively easy modification allows us to put a \emph{lower bound} restriction on these problems: perhaps each kid must get at least two cookies or $x,y,z \ge 2$.  This was done by first assigning each kid (or variable) 2 cookies (or units) and then distributing the rest using stars and bars.

What if we wanted an \emph{upper bound} restriction?  For example, we might insist that no kid gets more than 3 cookies or that $x, y, z \le 3$.  It turns out this is considerably harder, but still possible.  The idea is to count all the distributions and then remove those that violate the condition.  In other words, we must count the number of ways to distribute 10 cookies to 3 kids in which \emph{one or more} of the kids gets more than 3 cookies.  For any particular kid, this is not a problem; we do this using stars and bars.  But how to combine the number of ways for kid A, or B or C?  We must use the PIE.  


The Principle of Inclusion/Exclusion (PIE) gives a method for finding the cardinality of the union of not necessarily disjoint sets.  We saw in Section \ref{sec-PIE} how this works with three sets.  To find how many things are in {\em one or more} of the sets $A$, $B$, and $C$, we should just add up the number of things in each of these sets.  However, if there is any overlap among the sets, those elements are counted multiple times.  So we subtract the things in each intersection of a pair of sets.  But doing this removes elements which are in all three sets once too often, so we need to add it back in.  In terms of cardinality of sets, we have
\[|A \cup B \cup C| = |A| + |B| + |C| - |A \cap B| - |A \cap C| - |B \cap C| + |A\cap B \cap C|\]

\begin{example}
Three kids, Alberto, Bernadette, and Carlos, decide to share 10 cookies.  They wonder how many ways they could split the cookies up provided that none of them receive more than 3 cookies (someone receiving no cookies is for some reason acceptable to these kids).

\begin{solution}
Without the ``no more than 3'' restriction, the answer would be ${12 \choose 10}$, using 10 stars and 2 bars (separating the three kids).  Now count the number of ways that one or more of the kids violates the condition, i.e., gets at least 4 cookies.

Let $A$ be the set of outcomes in which Alberto gets more than 3 cookies.  Let $B$ be the set of outcomes in which Bernadette gets more than 3 cookies.  Let $C$ be the set of outcomes in which Carlos gets more than 3 cookies.  We then are looking (for the sake of subtraction) for the size of the set $A \cup B \cup C$.  Using PIE, we must find the sizes of $|A|$, $|B|$, $|C|$, $|A\cap B|$ and so on.  Here is what we find.

\begin{itemize}
\item[] $|A| = {8 \choose 6}$.  First give Alfred 4 cookies, then distribute the remaining 6 to the three kids without restrictions, using 6 stars and 2 bars.
\item[] $|B| = {8 \choose 6}$.  Just like above, only now Bernadette gets 4 cookies at the start.
\item[] $|C| = {8 \choose 6}$.  Carlos gets 4 cookies.

\item[] $|A \cap B| = {4 \choose 2}$.  Give Alfred and Bernadette 4 cookies each, leaving 2 (stars) to distribute to the three kids (2 bars).

\item[] $|A \cap C| = {4 \choose 2}$.  Alfred and Carlos get 4 cookies first.  

\item[] $|B \cap C| = {4 \choose 2}$.  Bernadette and Carlos get 4 cookies first.

\item[] $|A \cap B \cap C| = 0$.  It is not possible for all three kids to get 4 or more cookies.
\end{itemize}

Combining all of these we see
\[|A \cup B \cup C| = {8 \choose 6} + {8 \choose 6} + {8 \choose 6} - {4 \choose 2} - {4 \choose 2} - {4 \choose 2} + 0 = 66\]

Thus the answer to the original question is ${12 \choose 10} - 66 = 66 - 66 = 0$.  At first this might seem perplexing, but of course if 3 kids split 10 cookies, at least one of them must get more than 3 cookies (otherwise only 9 cookies have been consumed).  We could have found the answer much quicker through this observation, but the point of the example is to illustrate that PIE works!
\end{solution}
\end{example}


For four or more sets, we do not write down a formula for PIE.  Instead, we just think of the principle: add up all the elements in single sets, then subtract out things you counted twice (elements in the intersection of a {\em pair} of sets), then add back in elements you removed too often (elements in the intersection of groups of three sets), then take back out elements you added back in too often (elements in the intersection of groups of four sets), then add back in, take back out, add back in, etc.  This would be very difficult if it wasn't for the fact that in these problems, all the cardinalities of the single sets are equal, as are all the cardinalities of the intersections of two sets, and that of three sets, and so on.  Thus we can group all of these together and multiply by how many different combinations of 1, 2, 3,\ldots sets there are.


\begin{example}
How many ways can you distribute 7 cookies to 4 kids so that no kid gets more than 2 cookies?  

\begin{solution}
To answer this, we will subtract all the outcomes in which a kid gets 3 or more cookies.  How many outcomes are there like that?  We can force kid A to eat 3 or more cookies by giving him 3 cookies before we start.  Doing so reduces the problem to one in which we have 4 cookies to give to 4 kids without any restrictions.  In that case, we have 4 stars (the 4 remaining cookies) and 3 bars (one less than the number of kids) so we can distribute the cookies in ${7 \choose 3}$ ways.  Of course we could choose any one of the 4 kids to give too many cookies, so it would appear that there are ${4 \choose 1}{7 \choose 3}$ ways to distribute the cookies giving too many to one kid.  But in fact, we have over counted.  We must get rid of the outcomes in which two kids have too many cookies.  There are ${4 \choose 2}$ ways to select to 2 kids to give extra cookies.  It takes 6 cookies to do this, leaving only 1 cookie.  So we have 1 star and still 3 bars.  The remaining 1 cookie can thus be distributed in ${4 \choose 3}$ ways (for each of the ${4 \choose 2}$ choices of which 2 kids to over-feed).  We could continue in this fashion (using PIE) but there are no ways to give too many cookies to 3 kids - you don't have enough cookies.

All together we get that the number of ways to distribute 7 cookies to 4 kids without giving any kid more than 2 cookies is:
\[{10 \choose 3} - \left[{4 \choose 1}{7 \choose 3} - {4 \choose 2}{4 \choose 3}\right] = 120 - [ 140 - 24] = 4\]
This makes sense: the only way to distribute cookies with this restriction is for 3 kids to get 2 cookies and one kid to get 1.  There are 4 choices for which kid gets the 1 cookie.
\end{solution}
\end{example}


\begin{example}
Earlier we counted the number of solutions to the equation
\[x_1 + x_2 + x_3 + x_4 + x_5 = 13.\]

How many of those solutions have $0 \le x_i \le 3$ for each $x_i$? 


\begin{solution}
 We must subtract off the number of solutions in which one or more of the variables has a value greater than 3.  We will need to use PIE because counting the number of solutions for which each of the five variables separately are greater than 3 counts solutions multiple times.  Here is what we get:
    \begin{itemize}
      \item Total solutions: ${17 \choose 4}$
      \item Solutions where $x_1 > 3$: ${13 \choose 4}$ - give $x_1$ 4 units first, then distribute the remaining 9 units to the 5 variables.
      \item Solutions where $x_1 > 3$ and $x_2 > 3$: ${9 \choose 4}$ - after you give 4 units to $x_1$ and another 4 to $x_2$, you only have 5 units left to distribute.
      \item Solutions where $x_1 > 3$, $x_2 > 3$ and $x_3 > 3$: ${5 \choose 4}$.
      \item Solutions where $x_1 > 3$, $x_2 > 3$, $x_3 > 3$, and $x_4 > 3$: 0.
    \end{itemize}
    We also need to account for the fact that we could choose any of the five variables in the place of $x_1$ above, any pair of variables in the place of $x_1$ and $x_2$ and so on.  It is because of this that the double counting occurs, so we need to use PIE.  All together we have that the number of solutions with $0 \le x_i \le 3$ is
    \[{17 \choose 4} - \left[{5\choose 1}{13 \choose 4} - {5 \choose 2}{9 \choose 4} + {5 \choose 3}{5 \choose 4}\right]\]
  \end{solution}
\end{example}





\subsection{Counting Derangements}

This advanced use of PIE has uses beyond stars and bars.  A {\em derangement} of $n$ elements $\{1,2,3,\ldots, n\}$ is a permutation in which no element is fixed.  For example, there are $6$ permutations of the three elements $\{1,2,3\}$:
\[123 ~~ 132 ~~ 213 ~~ 231 ~~ 312 ~~ 321\]
but most of these have one or more elements fixed: $123$ has all three elements fixed, $132$ has the first element fixed (1 is in its original first position), and so on.  In fact, the only derangements of three elements are
\[231 \mbox{  and  }312\]
If we go up to 4 elements, there are 24 permutations (because we have 4 choices for the first element, 3 choices for the second, 2 choices for the third leaving only 1 choice for the last - $4! = 24$).  How many of these are derangements?  If you list out all 24 permutations and eliminate those which are not derangements, you will be left with just 9 derangements.  Let's see how we can get that number using PIE.

\begin{example}
  How many derangements are there of 4 elements?
  \begin{solution}
    We count all permutations, and subtract those which are not derangements.  There are $4! = 24$ permutations of 4 elements.  Now for a permutation to {\em not} be a derangement, at least one of the 4 elements must be fixed.  There are ${4 \choose 1}$ choices for which single element we fix.  Once fixed, we need to find a permutation of the other three elements - there are $3!$ permutations on 3 elements.  But now we have counted too many non-derangements, so we must subtract those permutations which fix two elements.  There are ${4 \choose 2}$ choices for which two elements we fix, and then for each pair, $2!$ permutations of the remaining elements.  But this subtracts too many, so add back in permutations which fix 3 elements, all ${4 \choose 3}1!$ of them.  Finally subtract the ${4 \choose 4}0!$ permutations (recall $0! = 1$) which fix all four elements.  All together we get that the number of derangements of 4 elements is:
    \[4! - \left[{4 \choose 1}3! - {4 \choose 2}2! + {4 \choose 3} 1! - {4 \choose 4}0!\right] = 24 - 15 = 9\]
    
  \end{solution}

\end{example}

Of course we can use a similar formula to count the derangements on any number of elements.  However, the more elements we have, the longer the formula gets.  Here is another example.

\begin{example}
  Five gentlemen attend a party, leaving their hats at the door.  At the end of the party, they hastily grab hats on their way out.  How many different ways could this happen so that none of the gentlemen leave with their own hat?
  
  \begin{solution}
    We are counting derangements on 5 elements.  There are $5!$ ways for the gentlemen to grab hats in any order - but many of these permutations will result in someone getting their own hat.  So we subtract all the ways in which one or more of the men get their own hat.  In other words, we subtract the non-derangements. Doing so requires PIE.  Thus the answer is:
    
    \[5! - \left[{5 \choose 1}4! - {5 \choose 2}3! + {5 \choose 3}2! - {5 \choose 4}1! + {5 \choose 5}0!\right]\]
  \end{solution}

\end{example}







\end{document}


