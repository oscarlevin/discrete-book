%This is counting.tex.  This very large chapter will cover a myriad of counting topics, including Pascal's triangle, combinations and permutations, and PIE.


\subsection*{Sum and Product Rule}
\begin{itemize}
\item What did you find for $\#P(A)$?  It should be $2^{\#A}$.  But how do we know this is correct?

\item We need to learn how to count.  

\ex At Red Dogs and Donuts, there are 14 varieties of donuts, and 16 types of hot-dogs.  If you want either a donut or a dog, how many options do you have?

\item This is an easy question - you just add 14 and 16.  Will that always work?  What is important here?

\item The {\em Sum Rule} states that if an event $A$ can occur in $m$ ways, and event $B$ can occur in $n$ {\em disjoint} ways, then the event ``$A$ or $B$'' can occur in $m + n$ ways.  Or in set theory lingo: if $A \cap B = \emptyset$, then $\#(A \cup B) = \#A + \#B$

\ex In a standard deck of 52 cards, how many cards are either red or a face card?  The sum rule doesn't work!  Why?

\ex What if you have more than two sets/events?  Say you would also consider eating one of 15 waffles?  How many choices do you have now?

\ex Suppose you are going for some FroYo - you can pick one of 6 yogurt choices, and one of 4 toppings.  How many choices do you have?  Break your choices up into disjoint sets:  $A$ is the set of choices with the first toping, $B$ is the set of choices featuring the second toping, and so on.  So we have four sets.  Each set has 6 elements (one for each yogurt flavor).  The sets are disjoint, so the total number of choices is 6 + 6 + 6 + 6.

\item Note that we could rewrite this as $6 \cdot 4$.  This gives us...

\item The {\em Product Rule}: if event $A$ can occur in $m$ ways, and each possibility for $A$ allows for exactly $n$ ways for event $B$, then the event ``$A$ and $B$'' can occur in $m \cdot n$ ways.

\ex How many two letter ``words'' are there which start with a vowel?  How many standard Colorado license plates are there (\#\#\#-aaa)?  How many FroYo's are there if you can pick one of 6 yogurts, one of 4 toppings, and one of 5 syrups?  Note the product rule generalizes to more than two events.

\item Careful: ``and'' doesn't mean ``times.''  For example, how many playing cards are both red and a face card?  Not $26 \cdot 12$!  On the other hand, how many two card hands are there made up of an ace and a jack?  If order matters?  If order doesn't?  Those work.  But what if you want the first card to be red and the second card to be an ace?  Now there is a problem.

\item How do you think about $\#P(A)$ now? There are two choices for each element in $A$.

\end{itemize}

\subsection*{Counting functions}
\begin{itemize}
\item The main question today is: How many functions are there with domain $A$ and codomain $B$?
\item If we have $\#A = m$ and $\#B = n$, how many functions $f: A \to B$ are there?
\item How many of those are onto?
\item How many are one-to-one?
\item How many are both?  Neither?
\item How many permutations are there on a set $A$?  Recall, a permutation is a one-to-one function from $A$ {\em onto} itself.
\end{itemize}




\subsection*{Bit Strings}

\begin{itemize}

\item We have spent some time trying to count the number of subsets of a given set.  One thing we have yet to figure out is how many subsets have a particular cardinality.

\item To tackle this problem, we will create a ``code'' for each subset.  

\item Suppose we have a set $\{a, b, c, d\}$.  We will use a 4 digit string to encode each subset. For example, the subset $\{a,c,d\}$ gets the code 1011. 

\item Give a few more examples, and then try to guess the rule.  Note that we are actually giving a bijection between $P(S)$ and the set of all {\em bit strings} with length $\#P(S)$.

\item The advantage?  Now we are dealing with a new type of mathematical object.  We can do things with bit strings that maybe don't make sense to do with subsets.

\item For example, we can combine two bit strings by concatenation.  More commonly, we can prefix a string $\alpha$ with a 0 or 1: $0\alpha$ is the string produced by placing a 0 in front of $\alpha$.  Similarly, if $\alpha = 01101$, what is $1\alpha$?

\item First, a few definitions: the {\em length} of a bit string is the number of digits in it. The weight of a bit string $\alpha$, denoted $\wgt \alpha$, is the number of 1's in $\alpha$.

\item We use $\b B^n$ to denote the set of all bit strings of length $n$.  Use $\b B^n_k$ to denote the set of all $n$-bit strings with weight $k$.

\ex Write out all the elements of $\b B^3$.  Then separate them into separate weights.

\item What order should we write these down in?  Take all the bit strings of length $n-1$, write them down twice.  Prefix all the first set with a zero, all the second set with a 1.

\ex Illustrate how to get the standard order for $\b B^3$ from the standard orders on $\b B^1$ and then $\b B^2$.

\item For the class: write down all bit strings in $\b B^4$ in the standard order.  Then separate them into their various weights.  What is $\#\b B^4_2$.  Can you relate this to the number of 3-bit strings with various weights?  Use this to find $\#\b B^5_2$.

\item Another way to count $\#\b B^n_k$: we need to select $k$ of the $n$ bits to be 1.  We use the symbol ${n \choose k}$ to denote the number of ways to choose $k$ elements from a set of $n$ elements.  Such numbers are usually called {\em binomial coefficients}

\item Give some examples.  Find the values of some small ones using the bit strings we already have counted.

\item Explain why ${n \choose 0 } + {n \choose 1} + \cdots + {n \choose n} = 2^n$.

\item Other formulas: Border formulas: ${n \choose 0} = 1 = {n \choose n}$  Why?  
\item Recursion formula: ${n \choose k}  = {n-1 \choose k} + {n-1 \choose k-1}$  Why?
\item Symmetry formula: ${n \choose k} = {n \choose n-k}$.  Why?
\end{itemize}



\subsection*{Lattice Path review and the Binomial Theorem}
\begin{itemize}
\item First, any questions on the homework (due Friday).
\item Some comments on lattice paths: why do lattice paths relate to ${n \choose k}$?
\item Proofs with lattice paths: Explain why ${n \choose k} = {n-1 \choose k} + {n-1 \choose k-1}$.
\item Proof of $\sum_{k=0}^n{n\choose k} = 2^n$ using paths.
\item Proof of the Hockey Stick Theorem: all paths to a point compared to all the different paths starting at points $(k, 1)$.
\subsubsection*{Identities from the Binomial Theorem}
\item Let's prove $\sum_{k=0}^n 2^k{n\choose k} = 3^n$.  (First expand this).
\item Remember the Binomial Theorem: how do you expand $(x + y)^n$?  In particular, what is the coefficient of the term $x^{n-k}y^k$?
\item Now what happens if you substitute $x = y = 1$.  You get our familiar $\sum_{k=0}^n {n\choose k}$. 
\item What if $x = 1$ and $y = 2$?  You prove the desired identity.
\item What if $x = 1$ and $y = -1$?  Try it for $n = 6$.  This gives alternating sum of a row on Pascal's Triangle.
\item Another way to look at it: the sum of the even entries equals the sum of the odd entries.  And each are half of the sum: so equal to $2^{n-1}$, or the sum of the previous row.  Does this make sense?
\item One more challenge problem: What is the coefficient of $x^{15}$ in $(x+1)^{20} + x^4(x+3)^{16}$?
\end{itemize}



\subsection*{Permutations and PIE}

\begin{itemize}


\item First, the find the coefficient problem.

\item Now on to permutations.  A permutation is a (possible) rearrangement of a list of numbers.  How many permutations are there of $\{1,2,3,4\}$?

\item How may permutations are there of $\{1,2,3,4,5,6\}$?  How many permutations are there if we only want to take 4 of those elements?

\item In general, we will let $P(n,k)$ denote the number of $k$-permutations of $n$ elements.  That is, the number of permutations of $k$ of the $n$ total elements.

\item Can we find a formula for $P(n,k)$.  How about $\frac{n!}{(n-k)!}$.  

\item Another way to think about it: We need to select $k$ out of the $n$ elements.  There are ${n \choose k}$ ways to do this.  Once we have chosen the $k$ elements, how many ways are there to permute them?  $k!$.  

\item Thus $P(n,k) = {n \choose k}\cdot k!$.  But we also have $P(n,k) = \frac{n!}{(n-k)!}$.

\item Combining these, we find a formula for ${n \choose k}$:
\[{n\choose k} = \frac{n!}{k!(n-k)!}\]

\ex You deal 5 cards from a deck of 52.  How many ways can this transpire?  Compare: how many 5 card hands are there?  How are these numbers related?

\subsubsection*{PIE}

\item Try the problem on the slide: An examination in three subjects, Algebra, Biology, and Chemistry, was taken
by 41 students. The following table shows how many students failed in each
single subject and in their various combinations.
\begin{center}
\begin{tabular}{|l|c|c|c|c|c|c|c|}
\hline
 Subject: & A & B & C & AB & AC & BC & ABC\\
\hline
Number Failed: & 12 & 5 & 8 & 2 & 6 & 3 & 1\\
\hline
\end{tabular}
\end{center}

How many students passed all their exams?

\item To find the number of students who failed at least one exam, you need to find \\$\#(A \cup B \cup C)$.  For this, we need PIE: the Principle of Inclusion/Exclusion.

\item Recall $\#(A \cup B) = \#A + \#B - \#(A \cap B)$.  Why did we subtract?

\item Use the Venn diagram to derive 
\[\#(A \cup B \cup C) = \#A + \#B +\#C - \#(A \cap B) - \#(A \cap C) - \#(B \cap C) + \#(A \cap B \cap C)\]


\end{itemize}



\subsection*{Applications to PIE}

\begin{itemize}
  \item Recall the formula for the three set version of PIE:
  \[\#(A \cup B \cup C) = \#A + \#B +\#C - \#(A \cap B) - \#(A \cap C) - \#(B \cap C) + \#(A \cap B \cap C)\]
  
  \ex How many numbers less than 1000 are divisible by either 3, 5, or 7?  Note: 1000/3 = 333.33, 1000/5 = 200, 1000/7 = 142.8, 1000/15 = 66.6, 1000/21 = 47.6, 1000/35 = 28.6, and 1000/105 = 9.5
  
  \item Four set version?  Probably best not to write it out, but give a scheme?
  \item We can use PIE with any number of sets, however with more than 3 sets it gets difficult - unless all the intersections of $k$ sets have the same cardinality.
  
  \ex How many derangements are there of 5 elements?  First, what does a derangement mean $(d_n)$? How many derangements are there with 1, 2, 3 elements?  With 4?  Use PIE.
  
  \begin{itemize}
    \item Start with all the permutations of 4 elements and then subtract the permutations which leave at least one element fixed.  How many are there of those?
    \item With one element fixed?  This is your $\#A + \#B + \#C + \#D$.  Note though that they are all the same.
    \item For the pairs of intersections, we ask how many permutations leave two elements fixed.  How many ways are there to pick the two elements?
    \item For the triples of intersections, how many permutations leave three elements fixed?
    \item Finally, the intersection of all the sets corresponds to fixing all elements.  
    \item Write all these out using factorials, and simplify.
    \item What should you get for $d_5$?
  \end{itemize}
  
  \item Now use the rest of the time to try to count onto functions.  How many onto functions are there from $\{1,2,3,4,5\}$ to $\{a,b,c\}$?  Hint: count all the functions, subtract those which are not onto - the ones for which at least one of $a$, $b$, or $c$ is not in the range.
\end{itemize}



\subsection*{More applications of counting}

\begin{itemize}
  \item Start with the worksheet on compositions of integers on page 110.
  \item Our goal: how many solutions are there to the equation $x_1 + x_2 + \cdots + x_k = n$ where each $x_i$ is
  \begin{itemize}
    \item a positive integer?
    \item a non-negative integer?
  \end{itemize}
  \item Starting with the positive integer version: how can we think of this problem as choosing some things out of a group of things?
  \item Here's one way to write $n$.  $1+1+1+\cdots+1$ (a total of $n$ 1's).  We need to regroup these so that there are exactly $k$ groups.  Each group must have at least one 1 in it.
  \item Where can we put the group dividers?  How many dividers do we need?
  \item Thus, then number of solutions in positive integers is ${n-1 \choose k-1}$.
  \subsubsection*{Non-negative integer version:}
  \item Now the groups can be empty.  We still need $k-1$ dividers.  They can go more places now.  
  \item Think of each place as being occupied by either a 1 or a divider.  There are $n$ 1's and $k-1$ dividers, so a total of $n + k - 1$ spots.  
  \item Thus the number of solutions in non-negative integers is ${n+k-1 \choose k-1}$
  \subsubsection*{Other models}
  \item Another way to ask this question: How many ways can you place $10$ identical objects into $6$ boxes - either so that each box must contain at least one element, or could be empty.  What do the objects represent, what do the boxes?
  \item Another: If you have 5 types flavors of ice-cream, and you want to make a 9 scoop sundae, how many ways can this be done?  In terms of compositions of integers, what are the flavors, and what are the number of scoops?
  \subsubsection*{Adding PIE}
  \item Now suppose you have a constraint: none of the integers can go above a certain value:
  
  \ex How many solutions are there to $x + y + z = 25$ in which each $x, y, z$ is in $[1,\ldots,9]$?  First find all solutions regardless of constraint.  Then subtract off those in which $x \ge 10$ (which is the number of solutions to $x' + y + z = 6$), $y \ge 10$, and $z \ge 10$.  Then add back in solutions in which both $x$ and $y$ are $\ge 10$, etc.
\end{itemize}

\subsection*{Combinatorial Proofs}

\begin{itemize}
  \item Today we look at a very specific method of proof - in fact, one we have seen before.
  \item {\em Combinatorial proof} is a method of proving certain identities by equating both sides of the equation with a third quantity.
  
  \ex Prove $1\cdot 12 + 2\cdot 11 + 3\cdot 10 + \cdots + 12 \cdot 1 = {14 \choose 3}$.
  
  \item Let's try to figure out what we are counting on the right hand side.  ${14 \choose 3}$ counts the number of 3-element subsets of $\{1,2,3,\ldots,14\}$. Now let's count that same quantity in another way.
  
  \item We need to fine a 3-element subset $\{a,b,c\}$ with $1 \le a < b < c \le 14$.  What could $b$ be?  The smallest it could be is 2.  If it was 2, then there is one choice for $a$ and 12 choices for $c$.  Or we could have $b = 3$, in which case there are two choices for $a$, and for each choice, there are 11 choices for $c$.  And so on.  
  
  \item The logic of what we just did: We wanted to show $X = Y$.  We did so by arguing that $X = Z$ and $Y = Z$.  Therefore $X = Z = Y$ so $X = Y$.
  
  \ex Here is another example which we have actually seen before: Prove ${n \choose k} = {n-1 \choose k-1} + {n-1 \choose k}$.  Remember what we did?  We said ${n \choose k}$ is the number of $n$-bit strings of weight $k$.  We think of counting that in another way too - all bit strings which start with a 1 followed by a $n-1$-bit string of weight $k-1$ together with all bit strings starting with a 0 followed by a $n-1$-bit string of weight $k$.  Of course there are ${n-1\choose k-1} + {n-1 \choose k}$ of those strings.
  
  \item Now you try some:
  
  \ex Prove ${5 \choose 0} + {5 \choose 1} + \cdots + {5\choose 5} = 2^5$.
  
  \ex Prove ${m+n \choose 2} - {m \choose 2} - {n \choose 2} = mn$  (Hint: try counting the number of edges in $K_{m,n}$; or the number of dancing pairs among $m$ men and $n$ women)
\end{itemize}



