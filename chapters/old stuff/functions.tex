%This is functions.tex.  In this section we will discuss functions.

\section{Functions}

\subsection*{Functions, part 1}

\begin{itemize}
\item What is a function?
\item We write $f: X \to Y$.  What are $X$ and $Y$ called?
\item Since $Y$ is not the range, what is the range.  We prefer to call the range the {\em image set}.  The set $\{f(x) \st x \in X\}$.
\item What does it mean for a function to be {\em onto}?  How can we phrase this in terms of the codomain and the image set?
\item What does it mean for a function to be one-to-one?  How does this relate to the definition of a function?
\item What is the {\em complete inverse image} $f\inv(y)$?  Note this is a set!
\item What is the relationship between the inverse image and whether a function is one-to-one?  Onto?
\item Find examples or say why they can't exist.  First, let $X = \{1, 2, 3, 4\}$ and $Y = \{a, b, c, d, e\}$.  Find:
\renewcommand{\labelenumi}{(\alph{enumi})}
\begin{enumerate}
 \item A function $f: X \to Y$ which is one-to-one.
 \item A function $f: X \to Y$ which is onto.
  \item A function $f: X \to Y$ which is neither one-to-one, nor onto.
 \item A function $f: X \to Y$ such that $\#f\inv(b) = 0$.
 \end{enumerate}
 \begin{enumerate}
  \item A set $X$ and a function $f: X \to X$ which is one-to-one but not onto.
  \item A set $X$ and a function $f: X \to X$ which is onto, but not one-to-one.
  
  \item A set $X$ and a function $F: X \to \{0,1,2, \ldots\}$ such that $f\inv(0) \cup f\inv(1) = X$
  
  \item A set $X$ and a function $F: X\to\{0,1,2,\ldots\}$ such that $f\inv(0) \cup f\inv(1) = X$ and $f$ is one-to-one.
  
\end{enumerate}

\end{itemize}



\subsection*{Functions, part 2}
\begin{itemize}
\item Spend 5 minutes agreeing on the worksheet from last time in groups.  Then discuss as a class.
\item Our next task: how to prove that a function is one-to-one and/or onto.

\ex Let $f: \N \to \N$ be given by $f(n) = 2n - 1$.  Is $f$ 1-1?  Is $f$ onto?  Prove your answers.
\begin{itemize}
\item $f$ is one-to-one.  Proof: Suppose $f(x) = f(y)$.  Then $2x - 1 = 2y - 1$.  Solving, we get $x = y$.  Thus every time two elements of the domain have the same image, they are in fact the same element.  
\item $f$ is NOT onto.  Proof: There is no $n \in \N$ for which $f(n) = 2$.  If there were, then $2 = 2x - 1$ for some $x$, so $x = 3/2$.  But $3/2$ is not in $\N$.
\item What is the image of $f$?  If we can find this, then we can change the codomain of the function to get a new function which is onto.
\item Consider the function $g: \N \to \O$ given by $g(n) = 2n -1$.  We will show $g$ is onto.  Consider an element $y \in \O$.  Since $y$ is odd, $y + 1$ is even, so $(y+1)/2$ is in $\N$.  Also, $g((y+1)/2) = 2((y+1)/2) -1 = y + 1 - 1 = y$, so $g$ is onto.
\end{itemize}

\item Try this one on your own (or in groups).  

\ex Let $f: \Z \to \Z$ be given by \[f(n) = \begin{cases}
\frac{n}{2} & \mbox{ if $n$ is even}\\
3n + 1 & \mbox{ if $n$ is odd}
\end{cases}\]  Is $f$ one-to-one?  Onto?  Prove or disprove.  If $f$ is not one-to-one, can we change something to make a new similar function which is?  What about the function $g : \E \to \Z$ given by the same rule?

\item Finish next time?
\end{itemize}

\subsection{Functions and cardinality}


\subsection{Composition of functions}



