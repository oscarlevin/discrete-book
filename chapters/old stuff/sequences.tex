% This is sequences.tex.  Here we will discuss sequences of natural numbers: how to describe them and why they are useful.  Perhaps also include Pascal's Triangle here.  Also series (finding formulas for the partial sums).

\section*{Sequences}
\begin{itemize}

\item Find the next term in the following sequences:
	\begin{itemize}
	\begin{multicols}{2}
		\item $a_1, a_2, a_3, \ldots$
		\item $7,7,7,7,7, \ldots$
		\item $3, -3, 3, -3, 3, \ldots$
		\item $1, 5, 2, 10, 3, 15, \ldots$
		\item $1, 2, 4, 8, 16, 32, \ldots$
		\item $1, 4, 9, 16, 25, 36, \ldots$
		\item $1, 2, 3, 5, 8, 13, 21, \ldots$
		\item $1, 3, 6, 10, 15, 21, \ldots$
		\item $2, 3, 5, 7, 11, 13, \ldots$
		\item $2, 6, 18, 54, \ldots$ 
		\item $3, 2, 1, 0, -1, \ldots$
		\item $1, 1, 2, 6, \ldots$ ($a_n = n!$)
		\end{multicols}
	\end{itemize}


\item Our goal is find a way to represent a given sequence.  We can do some either with a closed formula or with a recursive (or inductive) definition.

\item A closed formula gives the value of $a_n$ as a function of $n$.  

\item A recursive definition gives the value of $a_n$ as a function of previous terms in the sequence.  The recursive definition also must specify enough first values of the sequence so the process can get started.  The first few terms are called the \underline{basis} and the rest is called the \underline{inductive part}.

\item Sometimes it is easier to give either a closed formula or a recursive definition, but not the other:

\noindent{\bf Ex:}	The Fibonacci numbers: $1, 1, 2, 3, 5, 8, 13, \ldots$ have a recursive formula: $f_1 = 1$, $f_2 = 1$ and $f_n = f_{n-1} + f_{n-2}$.  However, the closed formula is very complicated: $f_n = \frac{\varphi^n - 1/\varphi^n}{5}$ where $\varphi = \frac{1 + \sqrt 5}{2}$

\noindent{\bf Ex:}	The squares: $1, 4, 9, 16, 25, \ldots$ have closed formula $a_n = n^2$.  This is much simpler than $a_1 = 1$, $a_n = a_{n-1} + 2n - 1$, which isn't really a recursive formula (it mentions $n$).

\subsection*{Arithmetic Progressions}

\item If the terms of a sequence differ by a constant, we say the sequence is an {\em arithmetic progression}.  Let's find recursive and closed formulas for such sequences.

\item How about we start with an example: make one up.

\item For the recursive definition, we need to specify $a_1$.  Then we need to express $a_n$ in terms of $a_{n-1}$.  In general, if the first term is $a$ and the common difference is $d$, then we have \[ a_1 = a \qquad a_n=a_{n-1}+d\]

\item To find a closed formula, first try writing out the first write out the sequence using the recursive definition without simplifying:
\[a, a+d, a+d+d, a+d+d+d, \ldots\]

\item So it looks like $a_n = a+d(n-1)$.  

\noindent{\bf Ex:} Write the sequence of all positive numbers which are 3 more than a multiple of 7.  Give a recursive definition and a closed formula.

\noindent{\bf Ex:} Find recursive definitions and closed formulas for the sequence $50, 43, 36, 29, \ldots$.  What is the 12th term of the sequence? 

\subsection*{Geometric Progressions}

\item What about sequences like $2, 6, 18, 54, \ldots$?  This is not arithmetic, because the difference is not constant.  However, the {\em ratio} between successive terms is constant.  We call such sequences {\em geometric progressions}.

\item The recursive definition for the geometric progression with first term $a$ and common ration $r$ is 
\[a_1 = a \qquad a_n = a_{n-1}\cdot r \]
\item We can find the closed formula like we did for the arithmetic progression.  We get $a_n = a\cdot r^{n-1}$.

\noindent{\bf Ex:} Find recursive and closed formulas for the sequence $96, 48, 24, 12, \ldots$.

\end{itemize}





\subsection*{Sums of Arithmetic and Geometric Progressions}
\begin{itemize}
\item The sum of the slide: $7 + 10 + 13 + \cdots + 3421$.  Describe the terms in the sequence: $4 + 3n$.  
\item What is $n$ for the last term?  Solving gives $n = 1139$.  So there are that many terms total.  
\item Now reverse and add: we get $3428\times 1139 = 3904492$.  But this is the sum twice over.  So the original sum is $3904492/2 = 1952246$.
\item Let's examine what we did.  We found the sum of the terms of an arithmetic progression.  We have done this before: the triangular numbers were formed by taking $T(n) = 1 + 2 + \cdots + n$.
\item Repeat the process to find a formula for $T(n)$.  
\item The general process.  Call the sum $S$.  Reverse and add.  This produces a single number added to itself many times.  Find the number of times.  Multiply.  Divided by 2.

\noindent{\bf Ex:} Find a closed formula for $6 + 10 + 14 + \cdots + (4n - 2)$.

\subsubsection*{Summing Geometric Progressions}

\item What is $3 + 6 + 12 + 24 + \cdots + 12288$?  Here we are summing up a geometric progression.
\item Our old trick does not work.  But we can still prevail.  Multiply each term by the common ration (2).  You get $2S = 6 + 12 + 24 + \cdots + 24576$.  
\item Now subtract: $2S - S = -3 + 24576 = 24573$.  Since $2S - S = S$, we have our answer.
\item Another: Find a closed formula for $S(n) = 2 + 10 + 50 + \cdots + 2\cdot 5^n$.
\item This is a useful trick: Express $0.464646\ldots$ as a fraction.  First write it as the sum of a geometric progression.

\noindent{\bf Ex:} Suppose you drop a ball from 50 ft, and on each bounce, it reaches 60\% of its previous height.  How far has the ball traveled after it reaches its peak of the 10th bounce?  How far will it travel if it bounces forever (with no air resistance, etc.)?

\subsubsection*{$\sum$ and $\prod$ notation}
\item To simplify writing out sums, we will use notation like $\d\sum_{i=1}^n a_i$.  This means add up the $a_i$s where $i$ changes from 1 to $n$.
\item If we want to multiply the $a_i$ instead, we would write $\d\prod_{i=1}^n a_i$.  For example, $\prod_{i=1}^n i = n!$.
\end{itemize}




\begin{itemize}
\item Start with a review question from last class: Express $0.464646\ldots$ as a fraction.


\subsection*{Finite Differences}
\item What if we want to form a sequence which is the sum of the terms of the sequence which is the sum of an arithmetic sequence, for example.

\ex How many squares (of all sizes) are there in a $8\times 8$ chessboard?  Start small.  Can we find a formula for the number of squares in an $n\times n$ board?

\item The example above is the sum of squares.  Squares are themselves sums of odd number (an arithmetic progression).

\item What if we look at the differences between terms.  The differences are squares.  What about the differences between the differences.  They are always odd.  And the differences between those - constant.

\item Let's figure out how to give a closed formula for a sequence whose differences, or second differences, etc, are eventually constant.

\item If the ``first'' differences are constant, we have an arithmetic sequence: $a + d(n-1)$.  This is linear.  If the second differences are constant, then we get something like $n(n+1)/2$ - it has an $n^2$ term. 

\item Call $\Delta$ the first difference, $\Delta^2$ the second, etc.  If $\Delta^k$ is constant, we will need a degree $k$ polynomial.  Compare $k$th difference being constant with $k$th derivative being constant.

\item If the $k$th difference is constant, then the closed formula with have the form $a_k n^k + a_{k-1} n^{k-1} +\cdots + a_2 n^2 + a_1 n + a_0$.

\ex Find a formula for the sequence $3, 7, 14, 24,\ldots$. Assume $a_1 = 3$.  It is easier to include the term $a_0$.  In this case, $a_0 = 2$ (why?).  Now we can check that this sequence has constant second difference.  So the formula will look like $a_n = a n^2 + b n + c$.  What are $a, b, c$?  Plug some values of $n$ in. $a_0 = 2 = a\cdot 0^2 + b \cdot 0 + c$.  Then $a_1 = 3 = a + b + c$ and $a_2 = 7 = a2^2 + b 2 + c$.  We have three equations and three unknowns, so we can solve.  We get $a_n = \frac{3}{2} n^2 - \frac{1}{2}n + 2$.

\ex Here's another way to solve the previous example.  Note that $3, 7, 14, 24,\ldots$ is the sequence you get for $S(n) = 2 + (1 + 4 + 7 + \cdots + 3n - 2)$.  We can solve this using our reverse and add trick.  

\item Now we have two ways to solve sequences with constant second difference.  For sequences with constant third (or higher) differences, we should use the ``method of finite differences'' outlined above.  Note that we will need to solve a system of (linear) equations, possibly with many variables.  This could be done with linear algebra (row reducing a matrix).

\item Now return to the question about counting squares on a chessboard.  It has constant third differences.  What is the formula? $a_n = \frac{1}{3}n^3 + \frac{1}{2}n^2 + \frac{1}{6}n = \frac{n(n+1)(2n+1)}{6}$

\end{itemize}



\begin{itemize}
\subsubsection*{from last time...}

\item Start by finding a closed formula for $\sum_{k=1}^n k^2$.  The first few terms are $1, 5, 14, 30, 55,\ldots$.  Use the method of finite differences to solve: $a_n = \frac{1}{3}n^3 + \frac{1}{2}n^2 + \frac{1}{6}n = \frac{n(n+1)(2n+1)}{6}$.

\item When does finite differences fail?  What about $2, 4, 8, 16, \ldots$?  What about $1, 1, 2, 3, 5, 8, \ldots$?

\subsubsection*{Summary of sequence techniques}

\item Here are our techniques for finding {\em closed} formulas:
\begin{itemize}
\item There are a few recognizable sequences we know the formulas for already.
\item If the sequence is arithmetic or geometric, we have formulas. What are they?
\item If the sequence is not arithmetic or geometric, perhaps the differences are.  If so, the sequence is the sum of an arithmetic or geometric progression, and we have tricks.  (They are?)
\item If the differences are not nice, perhaps the second (or third) differences are.  Then we can fit the sequence to a polynomial.
\item Before or after checking any of these, it's possible that the sequence is related to a sequence we know (or can find) a formula for.  Try adding/subtracting one/two to each term.  Or doubling, or factoring, or shifting, etc. 
\end{itemize} 

\item What about techniques for finding recursive formulas?

\begin{itemize}
\item Often you can just ``see'' it by looking for the pattern.  Ask yourself how you would find the next term.
\item The application can guide you.  What is the recursion for the Tower of Hanoi?  What about the total number of gold coins?  Note that we get two different recursions (both work) for the same sequence.
\item Another example: The colored tile problem from activity 10: You have $1 \times 1$ tiles in two colors, and $1\times 2$ tiles in 3 colors.  How many paths $1\times n$ paths are there?  Work through this example.
\item If there is no context, you can still sometimes find recursive formulas.  For example, try for the sequence 0, 1, 2, 5, 12, 29, 70, 169.  Look at differences, and then second differences (which are doubles of terms in the sequence).  How do you write the second differences, using the original $a_1, a_2, a_3,\ldots$.  We should get $a_n = 2a_{n-1} + a_{n-2}$
\item Don't forget, you need to specify the first few terms?  How many?  As many as the inductive part mentions.
\end{itemize}

\end{itemize}

\subsection*{Recurrence Relations}

\begin{itemize}
 \item Way back at the beginning of the course, we saw one way to define a sequence was {\em recursively} (or inductively).  Today we look at these ``definitions'' a little closer.
 
 \item A recurrence relation is the equation which gives the inductive part of the definition of the sequence.  For example, the recurrence relation for the Fibonacci numbers is $F_n = F_{n-1} + F_{n-2}$.  This, together with the initial conditions $F_0 = 0$ and $F_1 = 1$ define the sequence.
 
 \ex Find a recurrence relation and initial conditions for the sequence $1, 5, 17, 53, 161, 485\ldots$.  Look at the approximate ration between terms for a hint.  You should get $a_n = 3a_{n-1} + 2$ with initial term $a_0 = 1$.
 
 \item We are going to try to {\em solve} the recurrence relations.  By this we mean something very similar to solving differential equations: we want to find a function of $n$ which satisfies the recurrence relation, as well as the initial condition.  (Note: recurrence relations are sometimes called difference equations since they can describe the difference between terms - but this highlights the relation to differential equations further.)
 
 \item Just like for differential equations, finding a solution might be tricky, but checking that the solution is correct is easy - just plug it in.
 
 \ex Check that $a_n = 2^n + 1$ is a solution to the recurrence relation $a_n = 2a_{n-1} - 1$ with $a_1 = 3$.  Note that this recurrence relation looks familiar: we saw it with the Tower of Hanoi problem.  But there our initial condition was different.  Compare the terms of each.
 
 \item So how do you solve these recurrence relations?  There are a variety of methods.
 
 \ex Solve the recurrence relation $a_n = a_{n-1} + n$ with initial term $a_0 = 4$.  First, let's write out the sequence: $4, 5, 7, 10, 14, 19, \ldots$.  Look at the difference between terms.  $a_1 = a_0 + 1$ and $a_2 = a_1 + 2$ and so on.  The key thing here is that the difference between terms is $n$.  We can write this explicitly: $a_n - a_{n-1} = n$ (we get this from the recurrence relation by algebra).  
 
 If we use this equation over an over again, changing $n$, we have $a_1 - a_0 = 1$, $a_2 - a_1 = 2$, $a_3 - a_2 = 3$, \ldots, $a_n - a_{n-1} = n$.  Now add up all these equations.  On the right hand side we get $1 + 2 + 3 + \ldots + n$.  We know a formula for this sum: $\frac{n(n+1)}{2}$.  On the left hand side, the sum telescopes, leaving only $a_n - a_0$.  Now solve for $a_n$.
 
 \item The above example shows a way to solve recurrence relations of the form $a_n = a_{n-1} + f(n)$ where $\sum_{k = 1}^n f(k)$ has a known closed formula.  This will not help with our original example.  What can we do there?
 
 \item Let's try to solve the previous example in another way.  This time, don't first rearrange the recurrence relation $a_n = a_{n-1} + n$.  Instead, do some substitution: $a_n = (a_{n-2} + n-1) + n = ((a_{n-3} + n-2) + n-1) + n = \cdots = ((\ldots((a_0 + 1) + 2) + \cdots + n-1) + n = 4 + \frac{n(n+1)}{2}$.  
 
 \item Of course in this case we still needed to know formula for the sum of $1,\ldots,n$.  Let's try it with $a_n = 3a_{n-1} + 2$:
 \[a_n = 3a_{n-1} + 2 = 3(3a_{n-2} + 2) + 2 = 3(3(3a_{n-3} + 2) + 2) + 2 = \cdots = ?\]
 It helps to multiply this out while you go.  We get
 \[a_n = 3^n a_0 + 2(1 + 3 + 3^2 + \cdots + 3^{n-1}) = 3^n\cdot 1 + 2\frac{3^n - 1}{3-1} = 2\cdot 3^n - 1 \]
 We should check that this is really a solution.
 
 \item Next time, we will figure out how to solve recurrence relations where two previous terms are given, such as $a_n = 2 a_{n-1} + 3 a_{n-2}$.
\end{itemize}



\subsubsection*{More fun with recursions}

\begin{itemize}
  \item Last time we started to solve the recurrence relation $a_n = 3a_{n-1} + 2$ with $a_0 = 1$.  The sequence is: $1, 5, 17, 53, 161, 485\ldots$.   We saw that the telescoping technique does not work.  So we try another technique: iteration:
  
  \item The idea is: we keep $a_n$ on the left hand side of the equation, but replace all the other $a_k$ terms repeatedly until we get down to the known initial conditions.  So we get: 
   \[a_n = 3a_{n-1} + 2 = 3(3a_{n-2} + 2) + 2 = 3(3(3a_{n-3} + 2) + 2) + 2 = \cdots = ?\]
 It helps to multiply this out while you go.  This gives:
 \[a_n = 3^n a_0 + 2(1 + 3 + 3^2 + \cdots + 3^{n-1}) = 3^n\cdot 1 + 2\frac{3^n - 1}{3-1} = 2\cdot 3^n - 1 \]
 
 \item Iteration works on some recurrence relations for which telescoping fails.  But sometimes it is no help.  For example, what happens if you try to iterate $a_n = a_{n-1} + 6a_{n-2}$?  It gets messy real fast.
 
 \item Instead, we use a third technique: the Characteristic Root Technique.
 
 \item The idea is, we want to find a function which satisfies $a_n - a_{n-1} - 6a_{n-2} = 0$.  While iteration was too complicated, we notice that every time you iterate, you multiply by a constant.  So it is reasonable to guess the solution will be geometric - that is, $r^n$'s.
 
 \item What happens if we plug in $r^n$ into the recursion?   $r^n - r^{n-1} - 6r^{n-2} = 0$.  Solve for $r$: $r^{n-2}(r^2 - r - 6) = 0$.  So by factoring, $r = -2$ or $r = 3$.  Which one is it?
 
 \item That depends on the initial conditions.  Notice we could also have $a_n = (-2)^n + 3^n$.  Or $a_n = 7(-2)^n + 4\cdot 3^n$.  In fact, for any $a$ and $b$, $a_n = a(-2)^n + b 3^n$ is a solution.  To find the values of $a$ and $b$, use the initial conditions.
 
 \ex Solve the recursion $a_n = 7a_{n-1} - 10 a_{n-2}$ with $a_0 = 2$ and $a_1 = 3$.  First find the characteristic equation: $x^n - 7x + 10 = 0$.  The roots: $x = 2$ and $x = 5$ (these are the characteristic roots).  So $a_n = a 2^n + b 5^n$.  Plug in $n = 0$ and $n = 1$ to find $a$ and $b$.
 
 \item Now you try some.  Do the worksheet of Solving Recurrence Relations.
 
\end{itemize}




\subsection*{Wrapping up Recurrence Relations}
\begin{itemize}
 \item Recall one of the questions we looked at in the very beginning of the course: how many 1 by $n$ paths can you design out of 1 by 1 tiles which come in two colors, and 1 by 2 tiles which come in 3 colors?
 
 \item Guessing a closed formula for this problem looks hard.  However, we can quite easily build a recurrence relation: each path 1 by $n$ path can be constructed either by adding one of two 1 by 1 tiles to a path of length $n-1$ or by adding one of three 1 by 2 tiles to a path of length 1 by $n-2$.  If we already know how many paths there are of shorter length, we can find the paths of length $n$.
 
 \item Thus the recurrence relation is $a_n = 2a_{n-1} + 3a_{n-2}$.  Initial conditions?  Well there are no paths of length 0, so we can make a path of length 0 in only one way.  So $a_0 = 1$ and only two paths of length 1, so $a_1 = 2$.  Check that this works to generate the correct next two terms of the sequence.
 
 \item Can we find a closed formula?  You betcha.  Use the characteristic root technique.
\end{itemize}

\subsection*{Generating Functions}

\begin{itemize}
 \item We end the semester with a look at powerful tool sometimes used in discrete mathematics - a method of manipulating sequences: the generating function.
 \item The idea is this: instead of an infinite sequence (for example: $2, 3, 5, 8, 12, \ldots$) we look at a single function which encodes the sequence.  But not a function which gives the $n$th term as output.  Instead, a function whose power series (like from calculus 2) ``displays'' the terms of the sequence.  So for example, we would look at the power series $2 + 3x + 5x^2 + 8x^3 + 12x^4 + \cdots$.
 
 \item Ideally, we would like to find a closed formula for the power series - sometimes this works, sometimes not.  But when it does, we then have a single function which ``encodes'' the sequence.  Or, if you like, {\em generates} the sequence.  The function is therefore called a {\em generating function}.
 
 \item Let's see what the generating functions are for some very simple sequences.  The simplest of all: 1, 1, 1, 1, 1, \ldots.  What does the {\em generating series} look like?  Simply $1 + x + x^2 + x^3 + x^4 + \cdots$.  Now, can we find a closed formula for this power series?
 
 \item Yes!  Note that our series is really just a geometric series with common ration $x$.  So shift and subtract.  We get \[1 + x + x^2 + \cdots = \dfrac{1}{1-x}\]
 
 \item Now let's use this basic generating function to find generating functions for more sequences.  What if we replace $x$ by $-x$.  We get 
 \[\frac{1}{1+x} = 1 - x + x^2 - x^3 + \cdots \mbox{ which generates } 1, -1, 1, -1, \ldots\]
 
 \item Or if we replace $x$ by $3x$ we get
 \[\frac{1}{1-3x} = 1 + 3x + 9x^2 + 27x^3 + \cdots \mbox{ which generates } 1, 3, 9, 27, \ldots\]
 
 \item How could we get a generating function for $2, 2, 2, 2, \ldots$.  Notice we just multiplied each term of $1, 1, 1, \ldots$ by 2.  So:
 \[\frac{2}{1-x} = 2 + 2x + 2x^2 + 2x^3 + \cdots \mbox{ generates } 2, 2, 2, 2, \ldots\]
 
 \item Similarly, we get generate the sequence $3, 9, 27, 81, \ldots$ by $\frac{3}{1-3x}$
 
 \item What about the sequence $2, 4, 10, 28, 82, \ldots$?  The terms are always 1 more than powers of 3.  We have added the sequences $1,1,1,1,\ldots$ and $1,3,9, 27,\ldots$ term by term.  We get a generating function by adding the respective generating functions.
 \[2 + 4x + 10x^2 + 28x^3 + \cdots  = (1 + 1) + (1 + 3)x + (1 + 9)x^2 + (1 + 27)x^3 + \cdots = \frac{1}{1-x} + \frac{1}{1-3x}\]
 
 \item If we replace $x$ in our original generating function by $x^2$ we get
 \[\frac{1}{1-x^2} = 1 + x^2  + x^4 + x^6\cdots \mbox{ which generates } 1, 0, 1, 0, 1, 0, \ldots\]
 
 \item How could we get $0,1,0,1,0,1,\ldots$?  Start with the previous sequence and {\em shift} it over by 1.  Get get:
 \[\frac{x}{1-x^2} = x + x^3 + x^5 + \cdots \mbox{ generating } 0, 1, 0, 1, 0 , 1 \ldots\]
 
 \item We know how to add two sequences now.  If we add the previous two, we should get $1,1,1,1,1\ldots$.  This works out:
 \[\frac{1}{1-x^2} + \frac{x}{1-x^2} = \frac{1}{1-x}\]
 
 \item Here's a tricky one: What happens if you take the derivative of $\frac{1}{1-x}$.  We, we simply get $\frac{1}{(1-x)^2}$.  But if you differentiate term by term in the power series, we get $(1 + x + x^2 + x^3 + \cdots)' = 1 + 2x + 3x^2 + 4x^3 + \cdots $ which is the generating function for $1, 2, 3, 4, \ldots$.
 
 \item What happens if we take a second derivative: $\frac{2}{(1-x)^3} = 2 + 6x + 12x^2 + 20x^3 + \cdots$.  So $\frac{1}{(1-x)^3} = 1 + 3x + 6x^2 + 10x^3 + \cdots$ is a generating function for the triangular numbers.
 
 
\end{itemize}



\subsection*{More Generating Functions}
\begin{itemize}
  \item Find the generating function for $1, 3, 9, 27, \ldots$.  Then use this to find the generating function for $2, 4, 10, 28, \ldots$.  Now find the generating function for the sequence $0, 2, 0, 4, 0, 10, 0 , 28, \ldots$
  
  \item So far we have seen how to find generating functions from $\frac{1}{1-x}$ using multiplication (by a constant or by $x$), substitution, addition, and differentiation.  What else might we try?  
  
  \item We have a generating function for $2, 4, 10, 28, \ldots$.  How could we find a generating function for the sequence of first differences: $2, 6, 18, 54,\ldots$?  We want to subtract 2 from the 4, 4 from the 10, 10 from the 28, and so on.  So if we subtract (term by term) the sequence $0, 2, 4, 10, 28,\ldots$ from $2, 4, 10, 28\ldots$, we will be set.  Of course it is easy to find the generating function for $0,2,4,10,28\ldots$ - then just subtract.
  
  \item This ``differencing'' technique is useful for finding a generating function for the original sequence as well.  For example, let's find a generating function for $3, 5, 7, 9,\ldots$.  Note that the first differences are constant.  We don't know the generating function for the original, so call it $A$.  Now $(1-x)A = 3 + 2x + 2x^2 + \cdots = 1 + \frac{2}{1-x}$.  So the generating function for $3, 5, 7, 9\ldots$ must be $\frac{1}{1-x} + \frac{2}{(1-x)(1-x)}$
  
  \ex Use differencing to find the generating function for $1, 4, 9, 16, \ldots$.
  
  \ex Find a generating function for $0, 1, 1, 2, 3, 5, 8, 13, \ldots$.
  
  \item What happens to the sequences when you multiply two generating functions?  Let's see: $A = a_0 + a_1x + a_2x^2 + \cdots$ and $B = b_0 + b_1x + b_2x^2 + \cdots$.  To multiply $A$ and $B$, we need to do a lot of distributing (infinite FOIL?) but keep in mind we will regroup and only need to write down the first few terms to see the pattern.  What is the constant term?  $a_0b_0$.  What is the coefficient of $x$?  $a_0b_1 + a_1b_0$.  And so on.  We get:
  \[AB = a_0b_0 + (a_0b_1 + a_1b_0)x + (a_0b_2 + a_1b_1 + a_2b_0)x^2 + (a_0b_3 + a_1b_2 + a_2b_1 + a_3b_0)x^3 + \cdots\]
  
  \ex ``Multiply'' the sequence $1, 2, 3, 4 \ldots$ and $2, 4, 8, 16, \ldots$.  What new sequence do you get?
  
  \item What happens when you multiply a sequence by $1, 1, 1, \ldots$?  Try it with $1, 2, 3, 4, 5\ldots$?  Now multiply it again!  Is this a surprise?  What would differencing give you?
  
  \item The point is, if you need to find a generating function for the sum of the first $n$ terms of a particular sequence, and you know the generating function for {\em that} sequence, you can multiply it by $\frac{1}{1-x}$.
  
  \subsubsection*{Finding closed formulas via Generating Functions}
  
  \item Last time we found a closed formula for the number of paths problem.  The sequence was $1, 2, 7, 20, 61, \ldots$ (we had the recurrence relation $a_n = 2a_{n-1} + 3a_{n-2}$.  Let's find a generating function for this sequence.  The recurrence relation will help.
  
  \item Call the generating function $A$.  Now find $A - 2xA - 3x^2A$.  Every thing should cancel out (because of the recurrence relation) except for 1.  So the generating function is just 
  \[\frac{1}{1-2x -3x^2} = \frac{1}{(1+x)(1-3x)} = \frac{a}{1+x} + \frac{b}{1-3x}\]
  The last step by partial fraction decomposition.
  
  \item We can now solve for $a$ and $b$ (it works out that $a = 1/4$ and $b = 3/4$).  But why is this helpful?
  
  \item What is generated by $\frac{3/4}{1-3x} = 3/4(1 + 3x + 9x^2 + 27x^3+ \cdots)$?  The $n$th term of the sequence is $(3/4)3^n$.  Similarly, the $n$th term of the sequence generated by $\frac{1/4}{1+x}$ is $(1/4)(-1)^n$.  So the $n$th term of our original sequence is $(3/4)3^n + (1/4)(-1)^n$.
  
  \item Thus we can add generating functions to our list of methods for solving recurrence relations - although you do need to know how to do partial fraction decomposition.
\end{itemize}
