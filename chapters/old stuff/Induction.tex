\documentclass[12pt]{article}

\usepackage{../discrete}

\heading{Math 228}{}{Induction Notes}



\begin{document}
\section{Mathematical Induction}
Mathematical induction is a proof technique, not unlike direct proof or proof by contradiction.  However, while both direct proof and proof by contradiction are somewhat natural styles of proof - they arise from the way we think of the statements we want to prove - proof by mathematical induction is different and a little bit strange.  Let's start with an example.

\subsection{Stamps Revisited}

One of the problems we considered the first day involved making different amounts of postage:

\begin{quote}
  You need to mail a package, but don't yet know how much postage you will need. You have
a large supply of 8-cent stamps and 5-cent stamps. Which amounts of postage can you make
exactly using these stamps? Which amounts are impossible to make?
\end{quote}

If you have not thought about this problem, do so now.  Otherwise, read on.  Perhaps in investigating the problem you picked some amounts of postage, and then figured out whether you could make that amount using just 8-cent and 5-cent stamps.  Perhaps you did this in order - can you make 1 cents of postage?  Can you make 2 cents?  3 cents? And so on.  If this is what you did, you were actually answering a {\em sequence} of questions.  Well now we have methods for dealing with sequences.  Let's see if that helps.

Actually, we will not make a sequence of questions, but rather a sequence of statements.  Let $P(n)$ be the statement ``you can make $n$ cents of postage using just 8-cent and 5-cent stamps.''  Since each $P(n)$ is a statement, it is either true or false.  So if we form the sequence of statements
\[P(1), P(2), P(3), P(4), \ldots\]
the sequence will consist of $T$'s (for true) and $F$'s (for false).  In our particular case the sequence starts
\[F,F,F,F,T,F,F,T,F,F,T,F,F,T,\ldots\]
because $P(1), P(2), P(3), P(4)$ are all false (you cannot make 1, 2, 3, or 4 cents of postage) but $P(5)$ is true (use one 5-cent stamp), and so on.  

Let's think a bit about how we could find the value of $P(n)$ for some specific value of $n$ (the ``value'' with be either $T$ or $F$).  Well, how did we find the value of the $n$th term of a sequence of numbers? How did we find $a_n$?  There were two ways we could do this: either there was a closed formula for $a_n$, so we could plug in $n$ into the formula and get our output value, or we had a recursive definition for the sequence, so we could use the previous terms of the sequence to compute the $n$th term.  Now when dealing with sequences of statements, we could use either of these techniques as well.  Maybe there is a way to use $n$ itself to determine whether we can make $n$ cents of postage.  That would be something like a closed formula.  Or instead we could use the previous terms in the sequence (of statements) to determine whether we can make $n$ cents of postage.  That is, if we know the value of $P(n-1)$, can we get from that to the value of $P(n)$?  That would be something like a recursive definition for the sequence.  Remember, finding recursive definitions for sequences was often easier than finding closed formulas.  The same is true here.

Suppose I told you that $P(43)$ was true (it is).  Can you determine from this fact the value of $P(44)$ (is it true or false)?  Yes you can. Even if we don't know how exactly we made 43 cents out of the 5-cent and 8-cent stamps, we do know that there was some way to do it.  What if that way used at least three 5-cent stamps  (making 15 cents)?  We could replace those three 5-cent stamps with two 8-cent stamps (making 16 cents).  The total postage has gone up by 1, so we have a way to make 44 cents, so $P(44)$ is true.  Of course, we assumed that we had at least three 5-cent stamps.  What if we didn't? Then we must have at least three 8-cent stamps (making 24 cents).  If we replace those three 8-cent stamps with five 5-cent stamps (making 25 cents) then again we have bumped up our total by 1 cent so we can make 44 cents, so $P(44)$ is true.  

Notice that we have not said how to make 44 cents, just that we can, on the basis that we can make 43 cents.  How do we know we can make 43 cents?  Perhaps because we know we can make 42 cents, which we know we can do because we know we can make 41 cents, and so on.  It's a recursion!  As with a recursive definition of a numerical sequence, we must specify our initial value.  In this case, the initial value is ``$P(1)$ is false.''  That's not good, since our recurrence relation just says why $P(k+1)$ is true when we know that $P(k)$ is true.  So instead, we might want to use ``$P(31)$ is true'' as the initial condition.

Putting this all together we arrive at the following fact: it is possible to (exactly) make any amount of postage greater than 27 cents using just 5-cent and 8-cent stamps.\footnote{This is not claiming that there are no amounts less than 27 cents which can also be made - there are.}  In other words, $P(k)$ is true for any $k \ge 28$.  To prove this, we could do the following: 

\begin{enumerate}
  \item Demonstrate that $P(28)$ is true.
  \item Prove that if $P(k)$ is true, then $P(k+1)$ is true (for any $k \ge 28$).
\end{enumerate}
  
Suppose we have done this.  Then we know that the 28th term of the sequence above is a $T$ (using (1) - the initial condition), and that every term after the 28th is $T$ also (using (2) - the recursive part - in effect (2) shows that $P(k+1) = P(k)$).  Here is what the proof would actually look like.

\begin{proof}
  Let $P(n)$ be the statement ``it is possible to make exactly $n$ cents of postage using 5-cent and 8-cent stamps.''  We will show $P(n)$ is true for all $n \ge 28$.
  
  First, we show that $P(28)$ is true: $28 =  4 \cdot 5+ 1\cdot 8$, so we can make $28$ cents using four 5-cent stamps and one 8-cent stamp.  
  
  Now suppose $P(k)$ is true for some arbitrary $k \ge 28$.  Then it is possible to make $k$ cents using 5-cent and 8-cent stamps.  Note that since $k \ge 28$, it cannot be that we use less than three 5-cent stamps {\em and} less than three 8-cent stamps: using two of each would give only 26 cents.  Now if we have made $k$ cents using at least three 5-cent stamps, replace three 5-cent stamps by two 8-cent stamps.  The replaces 15 cents of postage with 16 cents, moving from a total of $k$ cents to $k+1$ cents - so $P(k+1)$ is true.  On the other hand, if we have made $k$ cents using at least three 8-cent stamps, then we can replace three 8-cent stamps with five 5-cent stamps, moving from 24 cents to 25 cents, giving a total of $k+1$ cents of postage.  So in this case as well $P(k+1)$ is true.  
  
  Therefore, by the principle of mathematical induction, $P(n)$ is true for all $n \ge 28$.
\end{proof}


\subsection{The Proof Technique}

What we did in the stamp example above works for many types of problems.  Proof by induction is useful when trying to prove statements about all natural numbers, or all natural numbers after some fixed first case (like 28 in the example above), and in some other situations besides.  Here is the general structure of a proof by mathematical induction:

\begin{defbox}{Induction Proof Structure}
Start by saying what the statement is which you want to prove: ``Let $P(n)$ be the statement\ldots''
To prove that $P(n)$ is true for all $n \ge 0$, you must prove two facts:

\begin{enumerate}
  \item Base case: Prove that $P(0)$ is true.  You do this directly.  This is often easy.
  \item Inductive case: Prove that $P(k) \imp P(k+1)$ for all $k \ge 0$.  That is, prove that for any $k \ge 0$ if $P(k)$ is true, then $P(k+1)$ is true as well.  This is the proof of an if\ldots then\ldots statement, so you might do this directly by assuming $P(k)$ and deriving $P(k+1)$ or indirectly (by contrapositive for example).
\end{enumerate}

Assuming you are successful on both parts above, you can conclude, ``Therefore by the principle of mathematical induction, $P(n)$ is true for all $n \ge 0$.''
\end{defbox}

Sometimes the statement $P(n)$ will only be true for values of $n \ge 4$, for example, or some other value.  In such cases, replace all the 0's above with 4's (or the other value).

Why does induction work?  Think of a row of dominoes set up standing on their edges.  We want to argue that in a minute, all the dominoes will have fallen down.  For this to happen, you will need to push the first domino - that is the base case.  It will also have to be that the dominoes are close enough together that when any particular domino falls, it will cause the next domino to fall - that is the inductive case.  If both of these conditions are met - you push the first domino over and each domino will cause the next to fall, then all the dominoes will fall.

Induction is powerful!  Think how much easier it is to knock over dominoes when you don't have to push over each domino yourself.  You just start the chain reaction, and the rely on the relative nearness of the dominoes to take care of the rest.  

Think about our study of sequences.  It is easier to find recursive definitions for sequences than closed formulas - going from one case to the next is easier than going directly to a particular case.  That is what is so great about induction.  Instead of going directly to the (arbitrary) case for $n$, we just need to say how to get from one case to the next.  

\subsection{Examples}

Here are some examples of proof my mathematical induction.

\begin{example}
  Prove for each natural number $n \ge 1$ that $1 + 2 + 3 + \cdots + n = \frac{n(n+1)}{2}$
  \begin{proof}
    Let $P(n)$ be the statement $1 + 2 + 3 + \cdots + n = \frac{n(n+2)}{2}$.  We will show that $P(n)$ is true for all natural numbers $n \ge 1$.
    
    Base case: $P(1)$ is the statement $1 = \frac{1(1+1)}{2}$ which is clearly true.
    
    Inductive case: Let $k \ge 1$ be a natural number.  Assume (for induction) that $P(k)$ is true.  That means $1 + 2 + 3 + \cdots + k = \frac{k(k+1)}{2}$.  We will prove that $P(k+1)$ is true as well.  That is, we must prove that $1 + 2 + 3 + \cdots + k + (k+1) = \frac{(k+1)(k+2)}{2}$.  To prove this equation, start with the left hand side:
    \[1 + 2 + 3 + \cdots + k + (k+1) = \frac{k(k+1)}{2} + (k+1)\]
    by the induction hypothesis.  Now simplifying:
    \[\frac{k(k+1)}{2} + k+1 = \frac{k(k+1)}{2} + \frac{2(k+1)}{2} = \frac{k(k+1) + 2(k+1)}{2} = \frac{(k+2)(k+1)}{2}\]
    Thus $P(k+1)$ is true, so by the principle of mathematical induction $P(n)$ is true for all natural numbers $n \ge 1$.
  \end{proof}
\end{example}

 Note that in the part of the proof in which we proved $P(k+1)$ from $P(k)$, we used the equation $P(k)$ - this was the inductive hypothesis.  This is very easy to do with proving a fact about a sum like this.  Sometimes it is not as easy to see where to use the inductive hypothesis ($P(k)$).
 
 \begin{example}
   Prove that for all $n \in \N$, $6^n - 1$ is a multiple of 5.
   \begin{proof}
     Let $P(n)$ be the statement, ``$6^n - 1$ is a multiple of 5.''  We will prove that $P(n)$ is true for all $n \in \N$.
     
     Base case: $P(0)$ is true: $6^0 -1 = 0$ which is a multiple of 5.
     
     Inductive case: Let $k$ be an arbitrary natural number.  Assume, for induction, that $P(k)$ is true.  That is, $6^k - 1$ is a multiple of $5$.  Then $6^k - 1 = 5j$ for some integer $j$.  Now consider $6^{k+1} - 1$.  We have
     \[6^{k+1} - 1 = 6^{k+1} - 6 + 5 = 6(6^k -1) + 5\]
     By the inductive hypothesis, 
     \[6(6^k-1) + 5 = 6(5j) + 5 = 5(6j + 1)\]
     Therefore $6^{k+1} - 1$ is a multiple of 5, or in other words, $P(k+1)$ is true.  Thus, by the principle of mathematical induction $P(n)$ is true for all $n \in \N$.
   \end{proof}
 \end{example}

We had to be a little bit clever to locate the $6^k - 1$ inside of $6^{k+1} - 1$ before we could apply the inductive hypothesis.  This is what can make inductive proofs challenging.  

In the two examples above, we started with $n = 1$ or $n = 0$.  We can start later if we need to.

\begin{example}
  Prove that $n^2 < 2^n$ for all integers $n \ge 5$.
  \begin{proof}
    Let $P(n)$ be the statement $n^2 < 2^n$.\footnote{$P(n)$ is not the statement ``$n^2 < 2^n$ for all integers $n\ge 5$'' - you do not include the quantifier on $n$ in the statement of $P(n)$.}
    We will prove $P(n)$ is true for all integers $n \ge 5$.
    
    Base case: $P(5)$ is the statement $5^2 < 2^5$.  Since $5^2 = 25$ and $2^5 = 32$, we see that $P(5)$ is indeed true.
    
    Inductive case: Let $k \ge 5$ be an arbitrary integer.  Assume, for induction, that $P(k)$ is true.  That is, assume $k^2 < 2^k$.  We will prove that $P(k+1)$ is true, i.e., $(k+1)^2 < 2^{k+1}$.  To prove such an inequality, start with the left hand side and work towards the right hand side:
    \begin{align*}
      (k+1)^2 & = k^2 + 2k + 1 &\\
       & < 2^k + 2k + 1 &\mbox{\ldots by the inductive hypothesis}\\
       & < 2^k + 2^k  &\mbox{\ldots since $2k + 1 < 2^k$ for $k \ge 5$}\\
       & = 2^{k+1} &
    \end{align*}
    Following the equalities and inequalities through, we get $(k+1)^2 < 2^{k+1}$, in other words, $P(k+1)$.  Therefore by the principle of mathematical induction, $P(n)$ is true for all $n \ge 5$.
  \end{proof}

\end{example}

Warning: induction isn't magic.  It seems very powerful to be able to assume $P(k)$ is true - after all, we are trying to prove $P(n)$ is true (and the only difference is in the variable: $k$ vs. $n$).  Are we assuming that what we want to prove is true?  Not really.  We assume $P(k)$ is true only for the sake of proving that $P(k+1)$ is true.  

Still you might start to believe that you can prove anything with induction. Consider this incorrect ``proof'' that every Canadian has the same eye color: Let $P(n)$ be the statement that any $n$ Canadians have the same eye color.  $P(1)$ is true, since everyone has the same eye color as themselves.  Now assume $P(k)$ is true.  That is, assume that in any group of $k$ Canadians, everyone has the same eye color.  Now consider an arbitrary group of $k+1$ Canadians.  The first $k$ of these must all have the same eye color, since $P(k)$ is true.  Also, the last $k$ of these must have the same eye color, since $P(k)$ is true.  So in fact, everyone the group must have the same eye color.  Thus $P(k+1)$ is true.  So by the principle of mathematical induction, $P(n)$ is true for all $n$.

Clearly something went wrong.  The problem is that the proof that $P(k)$ implies $P(k+1)$ assumes that $k \ge 2$.  We have only shown $P(1)$ is true.  In fact, $P(2)$ is false.

\end{document}


