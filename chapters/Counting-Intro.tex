\documentclass[12pt]{article}

\usepackage{discrete}

\def\thetitle{Introduction to Counting} % will be put in the center header on the first page only.
\def\lefthead{Math 228 Notes} % will be put in the left header
\def\righthead{Counting} % will be put in the right header


\begin{document}


One of the first things you learn in mathematics is how to count.  Now we want to count large collections of things quickly and precisely.  For example:
\begin{itemize}
% \item How many different lottery tickets are possible?
 \item In a group of 10 people, if everyone shakes hands with everyone else exactly once, how many handshakes took place? %\footnote{We have solved this problem already.  Soon we will see how this problem relates to more complicated counting problems.}
 \item How many ways can you distribute 10 girl scout cookies to 7 boy scouts?
 \item How many anagrams are there of ``anagram''?
% \item How many functions $f: \{1,2,3,4\} \to \{1,2,3\}$ are surjective?  
 \item How many subsets of $\{1,2,3,\ldots, 10\}$ have cardinality 7?
\end{itemize}

Before tackling these difficult questions, let's look at the basics of counting.




\end{document}


