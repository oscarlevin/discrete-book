\documentclass[12pt]{article}

\usepackage{discrete}

\def\thetitle{Conclusion} % will be put in the center header on the first page only.
\def\lefthead{Math 228 Notes} % will be put in the left header
\def\righthead{\thetitle} % will be put in the right header




\begin{document}

\section{Chapter Summary}
\begin{activity}
\begin{questions}
\question Suppose you have a huge box of animal crackers containing a plenty of each of 10 different animals.  For the counting questions below, carefully examine their similarities and differences, and then give an answer.  The answers are all one of the following.

\begin{itemize}
\begin{multicols}{4}
\item[] $ P(10,6)$
\item[] ${10 \choose 6}$
\item[] $10^6$
\item[] ${15 \choose 9}$
\end{multicols}
\end{itemize}

\begin{parts}
\part How many animal parades can you line up containing 6 crackers?

\part How many animal parades of 6 crackers can you line up so that the animals appear in alphabetical order?

\part How many ways could you line up 6 different animals in alphabetical order?

\part How many ways could you line up 6 different animals if they can come in any order?

\part How many ways could you give 6 children one animal cracker each?

\part How many ways could you give 6 children one animal cracker each so that no two kids get the same animal?

\part How many ways could you give out 6 giraffes to 10 kids?

\part Write a question about giving animal crackers to kids that has the answer ${10\choose 6}$.
\end{parts}
\end{questions}
\end{activity}

With all the different counting techniques we have mastered in this last chapter, it might be difficult to keep know when to apply which technique.  Indeed, it is very easy to get mixed up and use the wrong counting method for a given problem.  You get better with practice.  And as you practice you start to notice some trends that can help you distinguish between types of counting problems.  Here are some suggestions that you might find helpful when deciding how to tackle a counting problem and checking whether your solution is correct.

\begin{itemize}
\item Remember that you are counting the number of items in some \emph{list of outcomes}.  Write down part of this list.  Write down an element in the middle of the list -- how are you deciding whether your element really is in the list.  Could you get this element more than once using your proposed answer?
\item If generating an element on the list involves selecting something (for example, picking a letter or picking a position to put a letter, etc), can the things you select be repeated?  Remember, permutations and combinations select objects from a set \emph{without} repeats.
\item Does order matter?  Be careful here and be sure you know what your answer really means.  We usually say that order matters when you get different outcomes when the same objects are selected in different orders.  Combinations and ``Stars \& Bars'' are used when order {\em does not} matter.
\item There are four possibilities when it comes to order and repeats.  If order matters and repeats are allowed, the answer will look like $n^k$.  If order matters and repeats are not allowed, we have $P(n,k)$.  If order doesn't matter and repeats are allowed, use Stars \& Bars.  If order doesn't matter and repeats are not allowed, use ${n\choose k}$.  But be careful: this only applies when you are selecting things, and you should make sure you know exactly what you are selecting before determining which case you are in.
\item Think about how you would represent your counting problem in terms of sets or functions.  We know how to count different sorts of sets and different types of functions.  %maybe provide some specific examples here.

\item As we saw with combinatorial proofs, you can often solve a counting problem in more than one way.  Do that, and compare your numerical answers.  If they don't match, something is amiss.
\end{itemize}

While we have covered many counting techniques, we have really only scratched the surface of the large subject of \emph{enumerative combinatorics}.  There are mathematicians doing original research in this area even as you read this.  Counting can be really hard.

In the next chapter, we will approach counting questions from a very different direction, and in doing so, answer infinitely many counting questions at the same time.  We will create \emph{sequences} of answers to related questions.  
 
\end{document}


