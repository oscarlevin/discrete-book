\documentclass[12pt]{article}

\usepackage{discrete}

\def\thetitle{Sets} % will be put in the center header on the first page only.
\def\lefthead{Math 228 Notes} % will be put in the left header
\def\righthead{Sets} % will be put in the right header


\begin{document}

The most fundamental objects we will use in our studies (and really in all of math) are \textit{sets}\index{set}.  Much of what follows might be review, but it is very important that you are fluent in the language of set theory.  Most of the notation we use below is standard, although some might be a little different than what you have seen before.

For us, a set will simply be an unordered collection of objects.  For example, we could consider the set of all actors who have played The Doctor on Doctor Who.  Or the set of natural numbers between 1 and 10 inclusive.  In the first case, Tom Baker is a element (or member) of the set, while Idris Elba, among many others, is not an element of the set.  Also, the two example are of different sets.  Two sets are equal exactly if they contain the exact same elements.



\subsection{Notation}

We need some notation to make talking about sets easier.  Consider,
\[ A = \{1, 2, 3\}.\]
This is read, ``$A$ is the set containing the elements 1, 2 and 3.''  We use curly braces ``$\{,~~ \}$'' to enclose elements of a set.  Some more notation:
\[ a \in \{a, b, c\}. \]
The symbol ``$\in$'' is read ``is in'' or ``is an element of.''  Thus the above means that $a$ is an element of the set containing the letters $a$, $b$, and $c$.  Note that this is a true statement.  It would also be true to say that $d$ is not in that set:
\[ d \not\in \{a, b, c\}.\]
Be warned: we say ``$x \in A$'' when we wish to express that ``one of the elements of the set $A$ is $x$.''  For example, consider the set,
\[A = \{1, b, \{x, y, z\}, \emptyset\}\]
This is a strange set, to be sure. It contains four elements: the number 1, the letter b, the set $\{x,y,z\}$ and the empty set ($\emptyset = \{ \}$, the set containing no elements).  Is $x$ in $A$?  The answer is no.  None of the four elements in $A$ are the letter $x$, so we must conclude that $x \notin A$.  Similarly, if we considered the set $B = \{1,b\}$, then again $B \notin A$.  Even though the elements of $B$ are also elements of $A$, we cannot say that the \emph{set} $B$ is one of the things in the collection $A$. 

If a set is \emph{finite}\index{finite}, then we can describe it by simply listing the elements.  Infinite sets exists though, so we need to be able to describe them as well.  For instance, if we want $A$ to be the set of all even natural numbers, would could write,
\[ A = \{0, 2, 4, 6, \ldots\},\]
but this is a little imprecise.  Better would be
\[ A = \{x \in \N \st \exists n\in \N ( x = 2 n)\}.\]
Breaking that down: $x \in \N$ means $x$ is in the set $\N$ (the set of natural numbers, starting with 0), $\st$ is read ``such that'' and $\exists n\in \N (x = 2n)$ is read ``there exists an $n$ in the natural numbers for which $x$ is two times $n$'' (in other words, $x$ is even).  Slightly easier might be,
\[ A = \{x \st \mbox{ $x$ is even}\}. \]

Note: sometime people use $|$ or $\backepsilon$ for the ``such that'' symbol instead of the colon.

Defining a set using this sort of notation is very useful, although it takes some practice to read them correctly.  It is a way to describe the set of all things that satisfy some condition (the condition is the logical statement after the ``:'' symbol).  Here are some more examples.  We use the symbols $\wedge$ for ``and'' and $\vee$ for ``or'' (which includes the ``or both'' for us).  

\begin{example}
Describe each of the following sets both in words and by listing out enough elements to see the pattern.

\begin{multicols}{2}
\begin{enumerate}
\item $\{x \st x + 3 \in \N\}$
\item $\{x \in \N \st x + 3 \in \N\}$
\item $\{x \st x \in \N \vee -x \in \N\}$
\item $\{x \st x \in \N \wedge -x \in \N\}$
\end{enumerate}
\end{multicols}

\begin{solution}
\begin{enumerate}
\item This is the set of all number which are 3 less than a natural number (i.e., that if you add 3 to them, you get a natural number).  The set could also be written as $\{-3, -2, -1, 0, 1, 2, \ldots\}$ (note that 0 is a natural number, so $-3$ is in this set because $-3 + 3 = 0$).
\item This is the set of all natural numbers which are 3 less than a natural number.  So here we just have $\{0, 1, 2,3 \ldots\}$. 
\item This is the set of all integers (positive or negative whole numbers, written $\Z$).  In other words, $\{\ldots, -2, -1, 0, 1, 2, \ldots\}$.  
\item Now we want all numbers $x$ such that $x$ and $-x$ are natural numbers.  There is only one: 0.  So we have the set $\{0\}$.
\end{enumerate}
\end{solution}
\end{example}


We already have a lot of notation, and there is more yet.  Below is a handy chart of symbols.  Some of these will be discussed in greater detail as we move forward.



\begin{defbox}{Set Theory Notation}
\noindent  \begin{tabular}{p{.75in} p{1.35in} p{3.15in}}
    Symbol: & Read: & Example: \\ \hline \\
    $\{$, $\}$ & braces & $\{1,2,3\}$.  The braces enclose the elements of a set.  This is the set which contains the numbers 1, 2 and 3.\\[1ex]
    $\st$ & such that & $\{x \st x > 2\}$ is the set of all $x$ such that $x$ is greater than 2.\\[1ex]
    $\in$ & is an element of & $2 \in \{1,2,3\}$ asserts that 2 is one of the elements in the set $\{1,2,3\}$.  However, $4 \notin\{1,2,3\}$.\\[1ex]
    $\subseteq$ & is a subset of & $A \subseteq B$ asserts that every element of $A$ is also an element of $B$.\\[1ex]
    $\subset$ & is a proper subset of & $A \subset B$ asserts that every element of $A$ is also an element of $B$, but $A \ne B$.\\[1ex]
    \gls{cap} & intersection & $A \cap B$ is the \emph{set} of all elements which are elements of both $A$ and $B$.\\[1ex]
    \gls{cup} & union & $A \cup B$ is the \emph{set} of all elements which are elements of $A$ or $B$ or both.\\[1ex]
    $\times$ & cross & $A \times B$ is the set of all ordered pairs $(a,b)$ with $a \in A$ and $b \in B$. \\[1ex]
    $\setminus$ & set difference & $A \setminus B$ is the \emph{set} of all elements of $A$ which are not elements of $B$.\\[1ex]
    \gls{Acomp} & compliment (of $A$) & $\bar A$ is the set of everything which is not an element of $A$.  The $A$ can be any set here.\\[1ex]
    $|A|$ & cardinality\index{cardinality} (of $A$)& $|\{4,5,6\}| = 3$ because there are 3 elements in the set.  Sometimes we say $|A|$ is the \emph{size} of $A$.\\[1ex]
\end{tabular}
\noindent\textbf{Logic symbols:}\\
\noindent  \begin{tabular}{p{.75in} p{1.35in} p{3.15in}}

    $\wedge$ & and & $x \in A \wedge x \notin B$ means $x$ is both in the set $A$ \textbf{and} also not in $B$. \\[1ex]
    $\vee$ & or & $x \in A \vee x \notin B$ asserts that $x$ is an element of $A$ \textbf{or} not an element of $B$, or both. \\[1ex]
    $\neg$ & not & Another way to write $x \notin A$ is $\neg x \in A$.\\[1ex]
    $\forall$ & for all & $\forall x (x \ge 0)$ claims that for every number is greater than 0. \\[1ex]
    $\exists$ & there exists & $\exists x (x < 0)$ claims that there are negative numbers (there exists a number such that it is less than 0). \\[1ex]
  \end{tabular}

\end{defbox}

\begin{defbox}{Special sets}

\begin{tabular}{l p{5in}}
  $\emptyset$ & The \emph{empty set} is the set which contains no elements.\\[1ex]
  $\U$ & The \emph{universe set} is the set of all elements.\\[1ex]
$\N$ & The set of natural numbers. That is, $\N = \{0, 1, 2, 3\ldots\}$ \\[1ex]
$\Z$ & The set of integers.  $\Z = \{\ldots, -2, -1, 0, 1, 2, 3, \ldots\}$\\[1ex]
$\Q$ & The set of rational numbers.\\[1ex]
$\R$ & The set of real numbers.\\[1ex]
$\pow(A)$ & The \emph{power set} of any set $A$ is the set of all subsets of $A$.
\end{tabular}
\end{defbox}

\begin{activity}
%The goal of this activity is to practice reading set theory notation and to start counting with sets. %\footnote{Throughout this text you will find ``Investigate!'' activities.  These are designed to give you a feel for the types of questions that follow.  You are encouraged to try your best to complete the investigation before moving on.  Even if you are not successful, the hope is that doing so will prime you to more easily understand the following material.}.
\begin{questions}
\question Find the cardinality of each set below.
\begin{parts}
\part $A = \{3,4,\ldots, 15\}$.

\part $B = \{n \in \N \st 2 < n \le 200\}$.
 
\part $C = \{n \le 100 \st n \in \N \wedge \exists m \in \N (n = 2m+1)\}$. 

\end{parts}

\question Find two sets $A$ and $B$ for which $|A| = 5$, $|B| = 6$ and $|A\cup B| = 9$.  What is $|A \cap B|$? 

\question Find sets $A$ and $B$ with $|A| = |B|$ such that $|A\cup B| = 7$ and $|A \cap B| = 3$.  What is $|A|$?

\question Let $A = \{1,2,\ldots, 10\}$.  Define $\mathcal{B}_2 = \{B \subseteq A \st |B| = 2\}$.  Find $|\mathcal{B}_2|$.
\question For any sets $A$ and $B$, define $AB = \{ab \st a\in A \wedge b \in B\}$.  If $A = \{1,2\}$ and $B = \{2,3,4\}$, what is $|AB|$?  What is $|A \times B|$?
\end{questions}
\end{activity}



\subsection{Relationships between sets}

We have already said what it means for two sets to be equal: they have exactly the same elements.  Thus, for example,
\[ \{1, 2, 3\} = \{2, 1, 3\}.\]
(Remember, the order the elements are written down in does not matter.)  Also,
\[ \{1, 2, 3\} = \{I, II, III\}.\]
Now what about the sets $A = \{1, 2, 3\}$ and $B = \{1, 2, 3, 4\}$?  Clearly $A \ne B$.  However, we can notice that every element of $A$ is also an element of $B$.  Because of this, we say that $A$ is a \emph{subset}\index{subset} of $B$, or in symbols $A \subset B$ or $A \subseteq B$.  (Both symbols are read ``is a subset of.'' The difference is that sometimes we want to say that $A$ is either equal to or a subset of $B$, in which case we use $\subseteq$.  Compare the difference between $<$ and $\le$.)

\begin{example}
 Let $A = \{1, 2, 3, 4, 5, 6\}$, $B = \{2, 4, 6\}$, $C = \{1, 2, 3\}$ and $D = \{7, 8, 9\}$.  Determine which of the following are true, false, or meaningless.
\begin{multicols}{3}
\begin{enumerate}
\item $A \subset B$
\item $B \subset A$
\item $B \in C$
\item $\emptyset \in A$
\item $\emptyset \subset A$
\item $A < D$
\item $3 \in C$
\item $3 \subset C$.
\item $\{3\} \subset C$
\end{enumerate}
\end{multicols}
\begin{solution}
 \begin{enumerate}
  \item False.
\item True: every element in $B$ is an element in $A$.
\item False: the elements in $C$ are 1, 2, and 3.  The \emph{set} $B$ is not equal to 1, 2, or 3.
\item False: $A$ has exactly 6 elements, and none of them are the empty set.
\item True: Everything in the empty set (nothing) is also an element of $A$.  Notice that the empty set is a subset of every set.
\item Meaningless.  A set cannot be less that another set.
\item True.
\item Meaningless.  $3$ is not a set, so it cannot be a subset of another set.
\item True.  $3$ is the only element of the set $\{3\}$, and is an element of $C$, so every element in $\{3\}$ is an element of $C$.
 \end{enumerate}
\end{solution}
\end{example}

In the example above, $B$ is a subset of $A$.  You might wonder what other sets are subsets of $A$.  If you collect all these subsets of $A$, they themselves for a set -- a set of sets.  We call the set of all subsets of $A$ the \emph{power set} of $A$, and write it $\pow(A)$.  

\begin{example}
  Let $A = \{1,2,3\}$.  Find $\pow(A)$.
  \begin{solution}
    $\pow(A)$ is a set of sets, all of which are subsets of $A$.  So
    \[\pow(A) = \{ \emptyset, \{1\}, \{2\}, \{3\}, \{1,2\}, \{1, 3\}, \{2,3\}, \{1,2,3\}\}\]
    Notice that while $2 \in A$, it is wrong to write $2 \in \pow(A)$ since none of the elements in $\pow(A)$ are numbers!  On the other hand we do have $\{2\} \in \pow(A)$ because $\{2\} \subseteq A$.  
    
    What does a subset of $\pow(A)$ look like?  Notice that $\{2\} \not\subseteq \pow(A)$ because not everything in $\{2\}$ is in $\pow(A)$.  But we do have $\{ \{2\} \} \subseteq \pow(A)$.  The only element of $\{\{2\}\}$ is the set $\{2\}$ which is also an element of $\pow(A)$.  We could take the collection of all subsets of $\pow(A)$ and call that $\pow(\pow(A))$.  Or even the power set of that set of sets of sets. 
  \end{solution}

\end{example}


Another way to compare sets is by their size.  Notice that in the example above, $A$ has 6 elements, $B$, $C$, and $D$ all have 3 elements.  The size of a set is called the set's \emph{cardinality}\index{cardinality}.  We would write $|A| = 6$, $|B| = 3$ and so on.  For sets that have a finite number of elements, the cardinality of the set is simply the number of elements in the set.  Note that the cardinality of $\{ 1, 2, 3, 2, 1\}$ is 3. We do not count repeats (in fact, $\{1, 2, 3, 2, 1\}$ is exactly the same set as $\{1, 2, 3\}$).  There are sets with infinite cardinality, such as $\N$, the set of rational numbers (written $\mathbb Q$), the set of even natural numbers, the set of real number ($\mathbb R$).  It is possible to distinguish between different infinite cardinalities, but that is beyond the scope of this text.  For us, a set will either be infinite, or finite, and if it is finite, we can determine it's cardinality by counting elements.

\begin{example}
\begin{enumerate}
\item Find the cardinality of $A = \{23, 24, \ldots, 37, 38\}$.
\item Find the cardinality of $B = \{1, \{2, 3, 4\}, \emptyset\}$.
\item If $C = \{1,2,3\}$, what is the cardinality of $\pow(C)$?
\end{enumerate}
  
\begin{solution}
\begin{enumerate}
\item Since $38 - 23 = 15$, we can conclude that the cardinality of the set is $|A| = 16$ (you need to add one since 23 is included).
\item Here $|B| = 3$.  The three elements are the number 1, the set $\{2,3,4\}$, and the empty set.  
\item We wrote out the elements of the power set $\pow(C)$ above, and there are 8 elements (each of which is a set).  So $|\pow(C)| = 8$.\footnotemark
\end{enumerate}
 
\end{solution}
\end{example}
\footnotetext{You might wonder if there is a relationship between $|A|$ and $|\pow(A)|$ for all sets $A$.  This is a good question which we will return to in \Cref{ch:counting}.}

\subsection{Operations on sets}

Is it possible to add two sets?  Not really, however there is something similar.  If we want to combine two sets to get the collection of objects that are in either set, then we can take the \emph{union}\index{union} of the two sets.  Symbolically,
\[ C = A \gls{cup} B\]
means $C$ is the union of $A$ and $B$.  Every element of $C$ is either an element of $A$ or an element of $B$ (or an element of both).  For example, if $A = \{1, 2, 3\}$ and $B = \{2, 3, 4\}$, then $A \cup B = \{1, 2, 3, 4\}$.

The other common operation on sets is \emph{intersection}\index{intersection}.  We write,
\[ C = A \cap B\]
to mean that $C$ is the intersection of $A$ and $B$; everything in $C$ is in both $A$ and in $B$.  So if $A = \{1, 2, 3\}$ and $B = \{2, 3, 4\}$, then $A \cap B = \{2, 3\}$.  

Often when dealing with sets, we will have some understanding as to what ``everything'' is.  Perhaps we are only concerned with natural numbers.  We would say that our \emph{universe} is $\N$.  Sometimes we denote this universe by $\U$.  Given this context, we might wish to speak of all the elements which are \emph{not} in a particular set.  We say $B$ is the \emph{compliment}\index{compliment} of $A$, and write,
\[ B = \bar A\]
when $B$ contains every element not contained in $A$.  So if our universe is $\{1, 2,\ldots, 9, 10\}$, and $A = \{2, 3, 5, 7\}$, then $\bar A = \{1, 4, 6, 8, 9,10\}$.

Of course we can perform more than one operation at a time.  Fore example, consider
\[A \cap \bar B.\]
This is the set of all element which are both elements of $A$ and not elements of $B$.  What have we done?  We've started with $A$ and removed all of the elements which were in $B$.  Another way to write this is the \emph{set difference}\index{set difference}:
\[A \cap \bar B = A \gls{setminus} B.\]

It is important to remember that these operations (union, intersection, compliment and difference) on sets produce other sets.  Don't confuse these with the symbols from the previous section (element of and subset of).  $A \cap B$ is a set, while $A \subseteq B$ is true or false.  This is the same difference as between $3 + 2$ (which is a number) and $3 \le 2$ (which is in this case false).

\begin{example}
 Let $A = \{1, 2, 3, 4, 5, 6\}$, $B = \{2, 4, 6\}$, $C = \{1, 2, 3\}$ and $D = \{7, 8, 9\}$.  The universe is $\U = \{1, 2, \ldots, 10\}$.  Find:
\begin{multicols}{3}
 \begin{enumerate}
  \item $A \cup B$
\item $A \cap B$
\item $B \cap C$
\item $A \cap D$
\item $\bar{B \cup C}$
\item $A \setminus B$
\item $(D \cap \bar C) \cup \bar{A \cap B}$
\item $\emptyset \cup C$
\item $\emptyset \cap C$
 \end{enumerate}
\end{multicols}
\begin{solution}
  \begin{enumerate}
  \item $A \cup B = \{1, 2, 3, 4, 5, 6\} = A$ since everything in $B$ is already in $A$.
\item $A \cap B = \{2, 4, 6\} = B$ since everything in $B$ is in $A$.
\item $B \cap C = \{2\}$ as the only element of both $B$ and $C$ is 2.
\item $A \cap D = \emptyset$ since $A$ and $D$ have no common elements
\item $\bar{B \cup C} = \{5, 7, 8, 9, 10\}$.  First we find that $B \cup C = \{1, 2, 3, 4, 6\}$, then we take everything not in that set.
\item $A \setminus B = \{1, 3, 5\}$.  Everything that is in $A$ which is not in $B$.  This is the same as $A \cap \bar B$.
\item $(D \cap \bar C) \cup \bar{A \cap B} = \{1, 3, 5, 7, 8, 9\}.$ The set contains all elements that are either in $D$ but not in $C$ or not in both $A$ and $B$.
\item $\emptyset \cup C = C$ since nothing is added by the emptyset.
\item $\emptyset \cap C = \emptyset$.  Nothing can be both in a set and in the emptyset.
 \end{enumerate}
\end{solution}
\end{example}

You might notice that the symbols for union and intersection slightly resemble the logic symbols for ``or'' and ``and.''  This is no accident.  What does it mean for $x$ to be an element of $A\cup B$?  It means that $x$ is an element of $A$ or $x$ is an element of $B$ (or both).  That is,
\[x \in A \cup B \qquad \Iff \qquad x \in A \vee x \in B.\]
Similarly,
\[x \in A \cap B \qquad \Iff \qquad x \in A \wedge x \in B.\]
Also,
\[x \in \bar A \qquad \Iff \qquad \neg (x \in  A)\]
which says $x$ is an element of the compliment of $A$ if $x$ is not an element of $A$.

There is one more way to combine sets which will be useful for us: the Cartesian product.  This sounds fancy but is nothing you haven't seen before.  When you graph a function in calculus, you graph it in the Cartesian plane.  This is the set of all ordered pairs of real numbers $(x,y)$.  We can do this for \emph{any} pair of sets, not just the real numbers with themselves.

Put another way, $A \times B = \{(a,b) \st a \in A \wedge b \in B\}$.  The first coordinate comes from the first set, the second coordinate comes from the second set.  Sometimes we will want to take the Cartesian product of a set with itself, and this is fine: $A \times A = \{(a,b) \st a, b \in A\}$ (we might also write $A^2$ for this set).  Notice that in $A \times A$, we still want \emph{all} ordered pairs, not just the ones where the first and second coordinate are the same.  We can also take products of 3 or more sets, getting ordered triples, or quadruples, and so on.

\begin{example}
Let $A = \{1,2\}$ and $B = \{3,4,5\}$.  Find $A \times B$ and $A \times A$.  How many elements do you expect to be in $B \times B$?
\begin{solution}
$A \times B = \{(1,3), (1,4), (1,5), (2,3), (2,4), (2,5)\}$.  \\ $A \times A = A^2 = \{(1,1), (1,2), (2,1), (2,2)\}$. \\
$|B\times B| = 9$.  There will be 3 pairs with first coordinate $3$, three more with first coordinate $4$ and a final three with first coordinate $5$.
\end{solution}
\end{example}
%Given all this, you should not be surprised to find out that there is a version of De Morgan's laws for sets:
%
%\begin{defbox}{De Morgan's Laws (for sets)}
%  For any sets $A$ and $B$ in some universe $\U$:
%  \[\bar{A \cap B} = \bar A \cup \bar B\]
%  \[\bar{A \cup B} = \bar A \cap \bar B\]
%\end{defbox}
%
%Do you believe these equations?  To check De Morgan's laws in logic we could just make a truth table for each statement.  We don't have truth tables for sets, but we do have\ldots.

\subsection{Venn Diagrams}
There is a very nice visual tool we can use to represent operations on sets.  Venn diagrams display sets as intersecting circles.  We can shade the region we are talking about when we carry out an operation.  We can also represent cardinality of a particular set by putting the number in the corresponding region.\\

\begin{center}
\begin{tikzpicture}[fill=gray!50]
 \draw[thick] \circleA \circleAlabel \circleB \circleBlabel \twosetbox;
\end{tikzpicture} \hspace{2in}
\begin{tikzpicture}[scale=.75, fill=gray!50]
 \draw[thick] \circleA \circleAlabel \circleB \circleBlabel \circleC \circleClabel \threesetbox;
\end{tikzpicture}\\
\end{center}

%\includegraphics[width=2in]{images/venn2blank.png} \hfill \includegraphics[width=2in]{images/venn3blank.png}\\

Each circle represents a set.  The rectangle containing the circles represents the universe.  To represent combinations of these sets, we shade the corresponding region.  For example, we could draw $A \cap B$ as:

\begin{center}
\begin{tikzpicture}[fill=gray!50]
	\begin{scope}
	\clip \circleA;
	\fill \circleB;
	\end{scope}
 \draw[thick] \circleA \circleAlabel \circleB \circleBlabel \twosetbox;
\end{tikzpicture}

%  \includegraphics[width=2in]{images/venn2AcapB.png} 
\end{center}

Here is a representation of $A \cap \bar B$, or equivalently $A \setminus B$:

\begin{center}
\begin{tikzpicture}[fill=gray!50]
	\begin{scope}
	\clip \twosetbox \circleB;
	\fill \circleA;
	\end{scope}
 \draw[thick] \circleA \circleAlabel \circleB \circleBlabel \twosetbox;
\end{tikzpicture}

% \includegraphics[width=2in]{images/venn2AcapbarB.png}
\end{center}



A more complicated example is $(B \cap C) \cup (C \cap \bar A)$, as seen below.

\begin{center}
\begin{tikzpicture}[fill=gray!50]
	\fill \circleC;
	\begin{scope}
	    \clip \circleC;
	    \fill[white] \circleA \circleB;
	  \end{scope}
	  \begin{scope}
	  	\clip \circleC;
	  	\fill \circleB;
	  \end{scope}
 \draw[thick] \circleA \circleAlabel \circleB \circleBlabel \circleC \circleClabel \threesetbox;
\end{tikzpicture}

% \includegraphics[width=2in]{images/venn3complex.png}
\end{center}

Notice that the shaded regions above could also be arrived at in another way.  We could have started with all of $C$, then excluded the region where $C$ and $A$ overlap outside of $B$.  That region is $(A \cap C) \cap \bar B$.  So the above Venn diagram also represents $C \cap \left(\bar{(A\cap C)\cap \bar B}\right).$  So using just the picture, we have determined that
\[ (B \cap C) \cup (C \cap \bar A) = C \cap \left(\bar{(A\cap C)\cap \bar B}\right).\]





\end{document}
