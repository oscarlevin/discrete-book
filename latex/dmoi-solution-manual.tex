%**************************************%
%*    Generated from PreTeXt source   *%
%*    on 2020-04-14T10:39:16-06:00    *%
%*                                    *%
%*      https://pretextbook.org       *%
%*                                    *%
%**************************************%
\documentclass[twoside,11pt,]{book}
%% Custom Preamble Entries, early (use latex.preamble.early)
%% Default LaTeX packages
%%   1.  always employed (or nearly so) for some purpose, or
%%   2.  a stylewriter may assume their presence
\usepackage{geometry}
%% Some aspects of the preamble are conditional,
%% the LaTeX engine is one such determinant
\usepackage{ifthen}
%% etoolbox has a variety of modern conveniences
\usepackage{etoolbox}
\usepackage{ifxetex,ifluatex}
%% Raster graphics inclusion
\usepackage{graphicx}
%% Color support, xcolor package
%% Always loaded, for: add/delete text, author tools
%% Here, since tcolorbox loads tikz, and tikz loads xcolor
\PassOptionsToPackage{usenames,dvipsnames,svgnames,table}{xcolor}
\usepackage{xcolor}
%% begin: defined colors, via xcolor package, for styling
%% end: defined colors, via xcolor package, for styling
%% Colored boxes, and much more, though mostly styling
%% skins library provides "enhanced" skin, employing tikzpicture
%% boxes may be configured as "breakable" or "unbreakable"
%% "raster" controls grids of boxes, aka side-by-side
\usepackage{tcolorbox}
\tcbuselibrary{skins}
\tcbuselibrary{breakable}
\tcbuselibrary{raster}
%% We load some "stock" tcolorbox styles that we use a lot
%% Placement here is provisional, there will be some color work also
%% First, black on white, no border, transparent, but no assumption about titles
\tcbset{ bwminimalstyle/.style={size=minimal, boxrule=-0.3pt, frame empty,
colback=white, colbacktitle=white, coltitle=black, opacityfill=0.0} }
%% Second, bold title, run-in to text/paragraph/heading
%% Space afterwards will be controlled by environment,
%% independent of constructions of the tcb title
%% Places \blocktitlefont onto many block titles
\tcbset{ runintitlestyle/.style={fonttitle=\blocktitlefont\upshape\bfseries, attach title to upper} }
%% Spacing prior to each exercise, anywhere
\tcbset{ exercisespacingstyle/.style={before skip={1.5ex plus 0.5ex}} }
%% Spacing prior to each block
\tcbset{ blockspacingstyle/.style={before skip={2.0ex plus 0.5ex}} }
%% xparse allows the construction of more robust commands,
%% this is a necessity for isolating styling and behavior
%% The tcolorbox library of the same name loads the base library
\tcbuselibrary{xparse}
%% Hyperref should be here, but likes to be loaded late
%%
%% Inline math delimiters, \(, \), need to be robust
%% 2016-01-31:  latexrelease.sty  supersedes  fixltx2e.sty
%% If  latexrelease.sty  exists, bugfix is in kernel
%% If not, bugfix is in  fixltx2e.sty
%% See:  https://tug.org/TUGboat/tb36-3/tb114ltnews22.pdf
%% and read "Fewer fragile commands" in distribution's  latexchanges.pdf
\IfFileExists{latexrelease.sty}{}{\usepackage{fixltx2e}}
%% shorter subnumbers in some side-by-side require manipulations
\usepackage{xstring}
%% Footnote counters and part/chapter counters are manipulated
%% April 2018:  chngcntr  commands now integrated into the kernel,
%% but circa 2018/2019 the package would still try to redefine them,
%% so we need to do the work of loading conditionally for old kernels.
%% From version 1.1a,  chngcntr  should detect defintions made by LaTeX kernel.
\ifdefined\counterwithin
\else
    \usepackage{chngcntr}
\fi
%% Text height identically 9 inches, text width varies on point size
%% See Bringhurst 2.1.1 on measure for recommendations
%% 75 characters per line (count spaces, punctuation) is target
%% which is the upper limit of Bringhurst's recommendations
\geometry{letterpaper,total={374pt,9.0in}}
%% Custom Page Layout Adjustments (use latex.geometry)
%% This LaTeX file may be compiled with pdflatex, xelatex, or lualatex executables
%% LuaTeX is not explicitly supported, but we do accept additions from knowledgeable users
%% The conditional below provides  pdflatex  specific configuration last
%% The following provides engine-specific capabilities
%% Generally, xelatex is necessary for non-Western fonts
\ifthenelse{\boolean{xetex} \or \boolean{luatex}}{%
%% begin: xelatex and lualatex-specific configuration
\ifxetex\usepackage{xltxtra}\fi
%% realscripts is the only part of xltxtra relevant to lualatex 
\ifluatex\usepackage{realscripts}\fi
%% fontspec package provides extensive control of system fonts,
%% meaning *.otf (OpenType), and apparently *.ttf (TrueType)
%% that live *outside* your TeX/MF tree, and are controlled by your *system*
%% (it is possible that a TeX distribution will place fonts in a system location)
\usepackage{fontspec}
%% We use Latin Modern (lmodern) as the default font
%% So we check that it is available as a system font
\IfFontExistsTF{Latin Modern Roman}{}{\GenericError{}{The font "Latin Modern Roman" requested by PreTeXt output is not available as a system font}{Consult the PreTeXt Guide for help with LaTeX fonts.}{}}
%% We then define various font family commands using a vanilla version,
%% with the intention of letting a style override these choices
%% \setmainfont can be re-issued, and \renewfontfamily can redefine others
\setmainfont{Latin Modern Roman}[SmallCapsFont={Latin Modern Roman Caps}, SlantedFont={Latin Modern Roman Slanted}]
\newfontfamily{\divisionfont}{Latin Modern Roman}
\newfontfamily{\blocktitlefont}{Latin Modern Roman}
\newfontfamily{\contentsfont}{Latin Modern Roman}
\newfontfamily{\pagefont}{Latin Modern Roman}[SlantedFont={Latin Modern Roman Slanted}]
\newfontfamily{\tabularfont}{Latin Modern Roman}[SmallCapsFont={Latin Modern Roman Caps}]
\newfontfamily{\xreffont}{Latin Modern Roman}
\newfontfamily{\titlepagefont}{Latin Modern Roman}
%% begin: font information supplied by "font-xelatex-style" template
%% end: font information supplied by "font-xelatex-style" template
%% 
%% Extensive support for other languages
\usepackage{polyglossia}
%% Set main/default language based on pretext/@xml:lang value
%% document language code is "en-US", US English
%% usmax variant has extra hypenation
\setmainlanguage[variant=usmax]{english}
%% Enable secondary languages based on discovery of @xml:lang values
%% Enable fonts/scripts based on discovery of @xml:lang values
%% Western languages should be ably covered by Latin Modern Roman
%% end: xelatex and lualatex-specific configuration
}{%
%% begin: pdflatex-specific configuration
\usepackage[utf8]{inputenc}
%% PreTeXt will create a UTF-8 encoded file
%% begin: font setup and configuration for use with pdflatex
%% Portions of a document, are, or may, be affected by font-changing commands
%% These are more robust when using  xelatex  but may be employed with  pdflatex
%% The following definitons are meant to be re-defined in a style with \renewcommand
\newcommand{\divisionfont}{\relax}
\newcommand{\blocktitlefont}{\relax}
\newcommand{\contentsfont}{\relax}
\newcommand{\pagefont}{\relax}
\newcommand{\tabularfont}{\relax}
\newcommand{\xreffont}{\relax}
\newcommand{\titlepagefont}{\relax}
%% begin: font information supplied by "font-pdflatex-style" template
\usepackage{lmodern}
\usepackage[T1]{fontenc}
%% begin: font information supplied by "font-pdflatex-style" template
%% end: font setup and configuration for use with pdflatex
%% end: pdflatex-specific configuration
}
%% Monospace font: Inconsolata (zi4)
%% Sponsored by TUG: http://levien.com/type/myfonts/inconsolata.html
%% Loaded for documents with intentional objects requiring monospace
%% See package documentation for excellent instructions
%% One caveat, seem to need full file name to locate OTF files
%% Loads the "upquote" package as needed, so we don't have to
%% Upright quotes might come from the  textcomp  package, which we also use
%% We employ the shapely \ell to match Google Font version
%% pdflatex: "varqu" option produces best upright quotes
%% xelatex,lualatex: add StylisticSet 1 for shapely \ell
%% xelatex,lualatex: add StylisticSet 2 for plain zero
%% xelatex,lualatex: we add StylisticSet 3 for upright quotes
%% 
\ifthenelse{\boolean{xetex} \or \boolean{luatex}}{%
%% begin: xelatex and lualatex-specific monospace font
\usepackage{zi4}
\setmonofont[BoldFont=Inconsolatazi4-Bold.otf,StylisticSet={1,3}]{Inconsolatazi4-Regular.otf}
%% end: xelatex and lualatex-specific monospace font
}{%
%% begin: pdflatex-specific monospace font
%% "varqu" option provides textcomp \textquotedbl glyph
%% "varl"  option provides shapely "ell"
\usepackage[varqu,varl]{zi4}
%% end: pdflatex-specific monospace font
}
%% Symbols, align environment, commutative diagrams, bracket-matrix
\usepackage{amsmath}
\usepackage{amscd}
\usepackage{amssymb}
%% allow page breaks within display mathematics anywhere
%% level 4 is maximally permissive
%% this is exactly the opposite of AMSmath package philosophy
%% there are per-display, and per-equation options to control this
%% split, aligned, gathered, and alignedat are not affected
\allowdisplaybreaks[4]
%% allow more columns to a matrix
%% can make this even bigger by overriding with  latex.preamble.late  processing option
\setcounter{MaxMatrixCols}{30}
%%
%%
%% Division Titles, and Page Headers/Footers
%% titlesec package, loading "titleps" package cooperatively
%% See code comments about the necessity and purpose of "explicit" option.
%% The "newparttoc" option causes a consistent entry for parts in the ToC 
%% file, but it is only effective if there is a \titleformat for \part.
%% "pagestyles" loads the  titleps  package cooperatively.
\usepackage[explicit, newparttoc, pagestyles]{titlesec}
%% The companion titletoc package for the ToC.
\usepackage{titletoc}
%% Fixes a bug with transition from chapters to appendices in a "book"
%% See generating XSL code for more details about necessity
\newtitlemark{\chaptertitlename}
%% begin: customizations of page styles via the modal "titleps-style" template
%% Designed to use commands from the LaTeX "titleps" package
%% Plain pages should have the same font for page numbers
\renewpagestyle{plain}{%
\setfoot{}{\pagefont\thepage}{}%
}%
%% Two-page spread as in default LaTeX
\renewpagestyle{headings}{%
\sethead%
[\pagefont\thepage]%
[]
[\pagefont\slshape\MakeUppercase{\ifthechapter{\chaptertitlename\space\thechapter.\space}{}\chaptertitle}]%
{\pagefont\slshape\MakeUppercase{\ifthesection{Section\space\thesection.\space\sectiontitle}{}}}%
{}%
{\pagefont\thepage}%
}%
\pagestyle{headings}
%% end: customizations of page styles via the modal "titleps-style" template
%%
%% Create globally-available macros to be provided for style writers
%% These are redefined for each occurence of each division
\newcommand{\divisionnameptx}{\relax}%
\newcommand{\titleptx}{\relax}%
\newcommand{\subtitleptx}{\relax}%
\newcommand{\shortitleptx}{\relax}%
\newcommand{\authorsptx}{\relax}%
\newcommand{\epigraphptx}{\relax}%
%% Create environments for possible occurences of each division
%% Environment for a PTX "preface" at the level of a LaTeX "chapter"
\NewDocumentEnvironment{preface}{mmmmmm}
{%
\renewcommand{\divisionnameptx}{Preface}%
\renewcommand{\titleptx}{#1}%
\renewcommand{\subtitleptx}{#2}%
\renewcommand{\shortitleptx}{#3}%
\renewcommand{\authorsptx}{#4}%
\renewcommand{\epigraphptx}{#5}%
\chapter*{#1}%
\addcontentsline{toc}{chapter}{#3}
\label{#6}%
}{}%
%% Environment for a PTX "chapter" at the level of a LaTeX "chapter"
\NewDocumentEnvironment{chapterptx}{mmmmmm}
{%
\renewcommand{\divisionnameptx}{Chapter}%
\renewcommand{\titleptx}{#1}%
\renewcommand{\subtitleptx}{#2}%
\renewcommand{\shortitleptx}{#3}%
\renewcommand{\authorsptx}{#4}%
\renewcommand{\epigraphptx}{#5}%
\chapter[{#3}]{#1}%
\label{#6}%
}{}%
%% Environment for a PTX "section" at the level of a LaTeX "section"
\NewDocumentEnvironment{sectionptx}{mmmmmm}
{%
\renewcommand{\divisionnameptx}{Section}%
\renewcommand{\titleptx}{#1}%
\renewcommand{\subtitleptx}{#2}%
\renewcommand{\shortitleptx}{#3}%
\renewcommand{\authorsptx}{#4}%
\renewcommand{\epigraphptx}{#5}%
\section[{#3}]{#1}%
\label{#6}%
}{}%
%% Environment for a PTX "subsection" at the level of a LaTeX "subsection"
\NewDocumentEnvironment{subsectionptx}{mmmmmm}
{%
\renewcommand{\divisionnameptx}{Subsection}%
\renewcommand{\titleptx}{#1}%
\renewcommand{\subtitleptx}{#2}%
\renewcommand{\shortitleptx}{#3}%
\renewcommand{\authorsptx}{#4}%
\renewcommand{\epigraphptx}{#5}%
\subsection[{#3}]{#1}%
\label{#6}%
}{}%
%% Environment for a PTX "exercises" at the level of a LaTeX "subsection"
\NewDocumentEnvironment{exercises-subsection}{mmmmmm}
{%
\renewcommand{\divisionnameptx}{Exercises}%
\renewcommand{\titleptx}{#1}%
\renewcommand{\subtitleptx}{#2}%
\renewcommand{\shortitleptx}{#3}%
\renewcommand{\authorsptx}{#4}%
\renewcommand{\epigraphptx}{#5}%
\subsection[{#3}]{#1}%
\label{#6}%
}{}%
%% Environment for a PTX "exercises" at the level of a LaTeX "subsection"
\NewDocumentEnvironment{exercises-subsection-numberless}{mmmmmm}
{%
\renewcommand{\divisionnameptx}{Exercises}%
\renewcommand{\titleptx}{#1}%
\renewcommand{\subtitleptx}{#2}%
\renewcommand{\shortitleptx}{#3}%
\renewcommand{\authorsptx}{#4}%
\renewcommand{\epigraphptx}{#5}%
\subsection*{#1}%
\addcontentsline{toc}{subsection}{#3}
\label{#6}%
}{}%
%% Environment for a PTX "subsubsection" at the level of a LaTeX "subsubsection"
\NewDocumentEnvironment{subsubsectionptx}{mmmmmm}
{%
\renewcommand{\divisionnameptx}{Subsubsection}%
\renewcommand{\titleptx}{#1}%
\renewcommand{\subtitleptx}{#2}%
\renewcommand{\shortitleptx}{#3}%
\renewcommand{\authorsptx}{#4}%
\renewcommand{\epigraphptx}{#5}%
\subsubsection[{#3}]{#1}%
\label{#6}%
}{}%
%%
%% Styles for six traditional LaTeX divisions
\titleformat{\part}[display]
{\divisionfont\Huge\bfseries\centering}{\divisionnameptx\space\thepart}{30pt}{\Huge#1}
[{\Large\centering\authorsptx}]
\titleformat{\chapter}[display]
{\divisionfont\huge\bfseries}{\divisionnameptx\space\thechapter}{20pt}{\Huge#1}
[{\Large\authorsptx}]
\titleformat{name=\chapter,numberless}[display]
{\divisionfont\huge\bfseries}{}{0pt}{#1}
[{\Large\authorsptx}]
\titlespacing*{\chapter}{0pt}{50pt}{40pt}
\titleformat{\section}[hang]
{\divisionfont\Large\bfseries}{\thesection}{1ex}{#1}
[{\large\authorsptx}]
\titleformat{name=\section,numberless}[block]
{\divisionfont\Large\bfseries}{}{0pt}{#1}
[{\large\authorsptx}]
\titlespacing*{\section}{0pt}{3.5ex plus 1ex minus .2ex}{2.3ex plus .2ex}
\titleformat{\subsection}[hang]
{\divisionfont\large\bfseries}{\thesubsection}{1ex}{#1}
[{\normalsize\authorsptx}]
\titleformat{name=\subsection,numberless}[block]
{\divisionfont\large\bfseries}{}{0pt}{#1}
[{\normalsize\authorsptx}]
\titlespacing*{\subsection}{0pt}{3.25ex plus 1ex minus .2ex}{1.5ex plus .2ex}
\titleformat{\subsubsection}[hang]
{\divisionfont\normalsize\bfseries}{\thesubsubsection}{1em}{#1}
[{\small\authorsptx}]
\titleformat{name=\subsubsection,numberless}[block]
{\divisionfont\normalsize\bfseries}{}{0pt}{#1}
[{\normalsize\authorsptx}]
\titlespacing*{\subsubsection}{0pt}{3.25ex plus 1ex minus .2ex}{1.5ex plus .2ex}
\titleformat{\paragraph}[hang]
{\divisionfont\normalsize\bfseries}{\theparagraph}{1em}{#1}
[{\small\authorsptx}]
\titleformat{name=\paragraph,numberless}[block]
{\divisionfont\normalsize\bfseries}{}{0pt}{#1}
[{\normalsize\authorsptx}]
\titlespacing*{\paragraph}{0pt}{3.25ex plus 1ex minus .2ex}{1.5em}
%%
%% Styles for five traditional LaTeX divisions
\titlecontents{part}%
[0pt]{\contentsmargin{0em}\addvspace{1pc}\contentsfont\bfseries}%
{\Large\thecontentslabel\enspace}{\Large}%
{}%
[\addvspace{.5pc}]%
\titlecontents{chapter}%
[0pt]{\contentsmargin{0em}\addvspace{1pc}\contentsfont\bfseries}%
{\large\thecontentslabel\enspace}{\large}%
{\hfill\bfseries\thecontentspage}%
[\addvspace{.5pc}]%
\dottedcontents{section}[3.8em]{\contentsfont}{2.3em}{1pc}%
\dottedcontents{subsection}[6.1em]{\contentsfont}{3.2em}{1pc}%
\dottedcontents{subsubsection}[9.3em]{\contentsfont}{4.3em}{1pc}%
%%
%% Begin: Semantic Macros
%% To preserve meaning in a LaTeX file
%%
%% \mono macro for content of "c", "cd", "tag", etc elements
%% Also used automatically in other constructions
%% Simply an alias for \texttt
%% Always defined, even if there is no need, or if a specific tt font is not loaded
\newcommand{\mono}[1]{\texttt{#1}}
%%
%% Following semantic macros are only defined here if their
%% use is required only in this specific document
%%
%% Used for inline definitions of terms
\newcommand{\terminology}[1]{\textbf{#1}}
%% Used for fillin answer blank
%% Argument is length in em
%% Length may compress for output to fit in one line
\newcommand{\fillin}[1]{\leavevmode\leaders\vrule height -1.2pt depth 1.5pt \hskip #1em minus #1em \null}
%% End: Semantic Macros
%% Division Numbering: Chapters, Sections, Subsections, etc
%% Division numbers may be turned off at some level ("depth")
%% A section *always* has depth 1, contrary to us counting from the document root
%% The latex default is 3.  If a larger number is present here, then
%% removing this command may make some cross-references ambiguous
%% The precursor variable $numbering-maxlevel is checked for consistency in the common XSL file
\setcounter{secnumdepth}{1}
%%
%% AMS "proof" environment is no longer used, but we leave previously
%% implemented \qedhere in place, should the LaTeX be recycled
\newcommand{\qedhere}{\relax}
%%
%% A faux tcolorbox whose only purpose is to provide common numbering
%% facilities for most blocks (possibly not projects, 2D displays)
%% Controlled by  numbering.theorems.level  processing parameter
\newtcolorbox[auto counter, number within=section]{block}{}
%%
%% This document is set to number PROJECT-LIKE on a separate numbering scheme
%% So, a faux tcolorbox whose only purpose is to provide this numbering
%% Controlled by  numbering.projects.level  processing parameter
\newtcolorbox[auto counter, number within=chapter]{project-distinct}{}
%% A faux tcolorbox whose only purpose is to provide common numbering
%% facilities for 2D displays which are subnumbered as part of a "sidebyside"
\newtcolorbox[auto counter, number within=tcb@cnt@block, number freestyle={\noexpand\thetcb@cnt@block(\noexpand\alph{\tcbcounter})}]{subdisplay}{}
%%
%% tcolorbox, with styles, for THEOREM-LIKE
%%
%% theorem: fairly simple numbered block/structure
\tcbset{ theoremstyle/.style={bwminimalstyle, runintitlestyle, blockspacingstyle, after title={\space}, } }
\newtcolorbox[use counter from=block]{theorem}[3]{title={{Theorem~\thetcbcounter\notblank{#1#2}{\space}{}\notblank{#1}{\space#1}{}\notblank{#2}{\space(#2)}{}}}, phantomlabel={#3}, breakable, parbox=false, after={\par}, fontupper=\itshape, theoremstyle, }
%% lemma: fairly simple numbered block/structure
\tcbset{ lemmastyle/.style={bwminimalstyle, runintitlestyle, blockspacingstyle, after title={\space}, } }
\newtcolorbox[use counter from=block]{lemma}[3]{title={{Lemma~\thetcbcounter\notblank{#1#2}{\space}{}\notblank{#1}{\space#1}{}\notblank{#2}{\space(#2)}{}}}, phantomlabel={#3}, breakable, parbox=false, after={\par}, fontupper=\itshape, lemmastyle, }
%% proposition: fairly simple numbered block/structure
\tcbset{ propositionstyle/.style={bwminimalstyle, runintitlestyle, blockspacingstyle, after title={\space}, } }
\newtcolorbox[use counter from=block]{proposition}[3]{title={{Proposition~\thetcbcounter\notblank{#1#2}{\space}{}\notblank{#1}{\space#1}{}\notblank{#2}{\space(#2)}{}}}, phantomlabel={#3}, breakable, parbox=false, after={\par}, fontupper=\itshape, propositionstyle, }
%% corollary: fairly simple numbered block/structure
\tcbset{ corollarystyle/.style={bwminimalstyle, runintitlestyle, blockspacingstyle, after title={\space}, } }
\newtcolorbox[use counter from=block]{corollary}[3]{title={{Corollary~\thetcbcounter\notblank{#1#2}{\space}{}\notblank{#1}{\space#1}{}\notblank{#2}{\space(#2)}{}}}, phantomlabel={#3}, breakable, parbox=false, after={\par}, fontupper=\itshape, corollarystyle, }
%%
%% tcolorbox, with styles, for AXIOM-LIKE
%%
%% conjecture: fairly simple numbered block/structure
\tcbset{ conjecturestyle/.style={bwminimalstyle, runintitlestyle, blockspacingstyle, after title={\space}, } }
\newtcolorbox[use counter from=block]{conjecture}[3]{title={{Conjecture~\thetcbcounter\notblank{#1#2}{\space}{}\notblank{#1}{\space#1}{}\notblank{#2}{\space(#2)}{}}}, phantomlabel={#3}, breakable, parbox=false, after={\par}, fontupper=\itshape, conjecturestyle, }
%%
%% tcolorbox, with styles, for REMARK-LIKE
%%
%% remark: fairly simple numbered block/structure
\tcbset{ remarkstyle/.style={bwminimalstyle, runintitlestyle, blockspacingstyle, after title={\space}, } }
\newtcolorbox[use counter from=block]{remark}[2]{title={{Remark~\thetcbcounter\notblank{#1}{\space\space#1}{}}}, phantomlabel={#2}, breakable, parbox=false, after={\par}, remarkstyle, }
%%
%% tcolorbox, with styles, for EXAMPLE-LIKE
%%
%% example: fairly simple numbered block/structure
\tcbset{ examplestyle/.style={bwminimalstyle, runintitlestyle, blockspacingstyle, after title={\space}, after upper={\space\space\hspace*{\stretch{1}}\(\square\)}, } }
\newtcolorbox[use counter from=block]{example}[2]{title={{Example~\thetcbcounter\notblank{#1}{\space\space#1}{}}}, phantomlabel={#2}, breakable, parbox=false, after={\par}, examplestyle, }
%%
%% tcolorbox, with styles, for PROJECT-LIKE
%%
%% investigation: fairly simple numbered block/structure
\tcbset{ investigationstyle/.style={bwminimalstyle, runintitlestyle, blockspacingstyle, after title={\space}, } }
\newtcolorbox[use counter from=project-distinct]{investigation}[2]{title={{Investigate!~\thetcbcounter\notblank{#1}{\space\space#1}{}}}, phantomlabel={#2}, breakable, parbox=false, after={\par}, investigationstyle, }
%%
%% xparse environments for introductions and conclusions of divisions
%%
%% introduction: in a structured division
\NewDocumentEnvironment{introduction}{m}
{\notblank{#1}{\noindent\textbf{#1}\space}{}}{\par\medskip}
%%
%% tcolorbox, with styles, for miscellaneous environments
%%
%% assemblage: fairly simple un-numbered block/structure
\tcbset{ assemblagestyle/.style={size=normal, colback=white, colbacktitle=white, coltitle=black, colframe=black, rounded corners, titlerule=0.0pt, center title, fonttitle=\blocktitlefont\bfseries, blockspacingstyle, } }
\newtcolorbox{assemblage}[2]{title={\notblank{#1}{#1}{}}, phantomlabel={#2}, breakable, parbox=false, assemblagestyle}
\NewDocumentEnvironment{solutionproof}{}
{\par\smallskip\noindent\textit{Proof}.\space\space}{\space\space\hspace*{\stretch{1}}\(\blacksquare\)\par\smallskip}
%% paragraphs: the terminal, pseudo-division
%% We use the lowest LaTeX traditional division
\titleformat{\subparagraph}[runin]{\normalfont\normalsize\bfseries}{\thesubparagraph}{1em}{#1}
\titlespacing*{\subparagraph}{0pt}{3.25ex plus 1ex minus .2ex}{1em}
\NewDocumentEnvironment{paragraphs}{mm}
{\subparagraph*{#1}\hypertarget{#2}{}}{}
%% proof: title is a replacement
\tcbset{ proofstyle/.style={bwminimalstyle, fonttitle=\blocktitlefont\itshape, attach title to upper, after title={\space}, after upper={\space\space\hspace*{\stretch{1}}\(\blacksquare\)},
} }
\newtcolorbox{proof}[2]{title={\notblank{#1}{#1}{Proof.}}, phantom={\hypertarget{#2}{}}, breakable, parbox=false, after={\par}, proofstyle }
%% back colophon, at the very end, typically on its own page
\tcbset{ backcolophonstyle/.style={bwminimalstyle, blockspacingstyle, before skip=5ex, left skip=0.15\textwidth, right skip=0.15\textwidth, fonttitle=\blocktitlefont\large\bfseries, center title, halign=center, bottomtitle=2ex} }
\newtcolorbox{backcolophon}[1]{title={Colophon}, phantom={\hypertarget{#1}{}}, breakable, parbox=false, backcolophonstyle}
%% begin: environments for duplicates in solutions divisions
%% Solutions to division exercises, not in exercise group
\tcbset{ divisionsolutionstyle/.style={bwminimalstyle, runintitlestyle, exercisespacingstyle, after title={\space}, breakable, parbox=false } }
\newtcolorbox{divisionsolution}[3]{divisionsolutionstyle, title={\hyperlink{#3}{#1}.\notblank{#2}{\space#2}{}}}
\tcbset{ investigationsolutionstyle/.style={bwminimalstyle, runintitlestyle, exercisespacingstyle, after title={\space}, breakable, parbox=false } }
\newtcolorbox{investigationsolution}[3]{investigationsolutionstyle, title={\hyperref[#3]{Investigate!~#1}\notblank{#2}{\space#2}{}}}
%% Divisional exercises (and worksheet) as LaTeX environments
%% Third argument is option for extra workspace in worksheets
%% Hanging indent occupies a 5ex width slot prior to left margin
%% Experimentally this seems just barely sufficient for a bold "888."
%% Division exercises, not in exercise group
\tcbset{ divisionexercisestyle/.style={bwminimalstyle, runintitlestyle, exercisespacingstyle, left=5ex, breakable, parbox=false } }
\newtcolorbox{divisionexercise}[4]{divisionexercisestyle, before title={\hspace{-5ex}\makebox[5ex][l]{#1.}}, title={\notblank{#2}{#2\space}{}}, phantom={\hypertarget{#4}{}}, after={\notblank{#3}{\newline\rule{\workspacestrutwidth}{#3\textheight}\newline}{}}}
%% Localize LaTeX supplied names (possibly none)
\renewcommand*{\chaptername}{Chapter}
\setcounter{chapter}{-1}
%% Equation Numbering
%% Controlled by  numbering.equations.level  processing parameter
%% No adjustment here implies document-wide numbering
\numberwithin{equation}{chapter}
%% "tcolorbox" environment for a single image, occupying entire \linewidth
%% arguments are left-margin, width, right-margin, as multiples of
%% \linewidth, and are guaranteed to be positive and sum to 1.0
\tcbset{ imagestyle/.style={bwminimalstyle} }
\NewTColorBox{image}{mmm}{imagestyle,left skip=#1\linewidth,width=#2\linewidth}
%% For improved tables
\usepackage{array}
%% Some extra height on each row is desirable, especially with horizontal rules
%% Increment determined experimentally
\setlength{\extrarowheight}{0.2ex}
%% Define variable thickness horizontal rules, full and partial
%% Thicknesses are 0.03, 0.05, 0.08 in the  booktabs  package
\newcommand{\hrulethin}  {\noalign{\hrule height 0.04em}}
\newcommand{\hrulemedium}{\noalign{\hrule height 0.07em}}
\newcommand{\hrulethick} {\noalign{\hrule height 0.11em}}
%% We preserve a copy of the \setlength package before other
%% packages (extpfeil) get a chance to load packages that redefine it
\let\oldsetlength\setlength
\newlength{\Oldarrayrulewidth}
\newcommand{\crulethin}[1]%
{\noalign{\global\oldsetlength{\Oldarrayrulewidth}{\arrayrulewidth}}%
\noalign{\global\oldsetlength{\arrayrulewidth}{0.04em}}\cline{#1}%
\noalign{\global\oldsetlength{\arrayrulewidth}{\Oldarrayrulewidth}}}%
\newcommand{\crulemedium}[1]%
{\noalign{\global\oldsetlength{\Oldarrayrulewidth}{\arrayrulewidth}}%
\noalign{\global\oldsetlength{\arrayrulewidth}{0.07em}}\cline{#1}%
\noalign{\global\oldsetlength{\arrayrulewidth}{\Oldarrayrulewidth}}}
\newcommand{\crulethick}[1]%
{\noalign{\global\oldsetlength{\Oldarrayrulewidth}{\arrayrulewidth}}%
\noalign{\global\oldsetlength{\arrayrulewidth}{0.11em}}\cline{#1}%
\noalign{\global\oldsetlength{\arrayrulewidth}{\Oldarrayrulewidth}}}
%% Single letter column specifiers defined via array package
\newcolumntype{A}{!{\vrule width 0.04em}}
\newcolumntype{B}{!{\vrule width 0.07em}}
\newcolumntype{C}{!{\vrule width 0.11em}}
%% Footnote Numbering
%% Specified by numbering.footnotes.level
%% Undo counter reset by chapter for a book
\counterwithout{footnote}{chapter}
\counterwithin*{footnote}{chapter}
%% Multiple column, column-major lists
\usepackage{multicol}
%% More flexible list management, esp. for references
%% But also for specifying labels (i.e. custom order) on nested lists
\usepackage[inline]{enumitem}
%% Package for tables spanning several pages
\usepackage{longtable}
%% hyperref driver does not need to be specified, it will be detected
%% Footnote marks in tcolorbox have broken linking under
%% hyperref, so it is necessary to turn off all linking
%% It *must* be given as a package option, not with \hypersetup
\usepackage[hyperfootnotes=false]{hyperref}
%% configure hyperref's  \url  to match listings' inline verbatim
\renewcommand\UrlFont{\small\ttfamily}
%% latex.print parameter set to 'yes', all hyperlinks black and inactive
\hypersetup{draft}
\hypersetup{pdftitle={Solution Manual}}
%% If you manually remove hyperref, leave in this next command
\providecommand\phantomsection{}
%% Graphics Preamble Entries
\usepackage{tikz, pgfplots}

\usetikzlibrary{positioning,matrix,arrows}

\usetikzlibrary{shapes,decorations,shadows,fadings,patterns}
\usetikzlibrary{decorations.markings}

\usepackage{skak} %for chessboards etc.

\def\circleA{(-.5,0) circle (1)}
\def\circleAlabel{(-1.5,.6) node[above]{$A$}}
\def\circleB{(.5,0) circle (1)}
\def\circleBlabel{(1.5,.6) node[above]{$B$}}
\def\circleC{(0,-1) circle (1)}
\def\circleClabel{(.5,-2) node[right]{$C$}}
\def\twosetbox{(-2,-1.4) rectangle (2,1.4)}
\def\threesetbox{(-2.5,-2.4) rectangle (2.5,1.4)}
\newcommand{\hexbox}[3]{
  \def\x{-cos{30}*\r*#1+cos{30}*#2*\r*2}
  \def\y{-\r*#1-sin{30}*\r*#1}
  \draw (\x,\y) +(90:\r) -- +(30:\r) -- +(-30:\r) -- +(-90:\r) -- +(-150:\r) -- +(150:\r) -- cycle;
  \draw (\x,\y) node{#3};
}

\tikzset{->-/.style={decoration={
  markings,
  mark=at position .5 with {\arrow{>}}},postaction={decorate}}}

  \newcommand{\onedot}{
    +(.5,.5) \v
  }
  \newcommand{\twodots}{
    +(.25,.25) \v +(.75,.75) \v
  }
  \newcommand{\threedots}{
  +(.25,.25) \v +(.5, .5) \v +(.75,.75) \v
  }
  \newcommand{\fourdots}{
    +(.25,.25) \v +(.25,.75) \v +(.75,.25) \v +(.75,.75) \v
  }
  \newcommand{\fivedots}{
    +(.5,.5) \v +(.25,.25) \v +(.25,.75) \v +(.75,.25) \v +(.75,.75) \v
  }
  \newcommand{\sixdots}{
    +(.25,.5) \v +(.75,.5) \v +(.25,.25) \v +(.25,.75) \v +(.75,.25) \v +(.75,.75) \v
  }
  \newcommand{\dominoborder}{
    \draw[thick, rounded corners] (0,0) rectangle (1,2);
    \draw[thin] (0,1) -- (1,1);
  }

%% Imported from https://upload.wikimedia.org/wikipedia/commons/3/32/Blank_US_Map.svg
%% Translated to TikZ using Inkscape 0.48
%% From https://gist.github.com/bordaigorl/fce575813ff943f47505

\tikzset{USA map/.cd,
state/.style={fill, draw=white, ultra thick},
HI/.style={}, AK/.style={}, FL/.style={}, NH/.style={}, MI/.style={}, MI/.style={}, SP/.style={}, VT/.style={}, ME/.style={}, RI/.style={}, NY/.style={}, PA/.style={}, NJ/.style={}, DE/.style={}, MD/.style={}, VA/.style={}, WV/.style={}, OH/.style={}, IN/.style={}, IL/.style={}, CT/.style={}, WI/.style={}, NC/.style={}, DC/.style={}, MA/.style={}, TN/.style={}, AR/.style={}, MO/.style={}, GA/.style={}, SC/.style={}, KY/.style={}, AL/.style={}, LA/.style={}, MS/.style={}, IA/.style={}, MN/.style={}, OK/.style={}, TX/.style={}, NM/.style={}, KS/.style={}, NE/.style={}, SD/.style={}, ND/.style={}, WY/.style={}, MT/.style={}, CO/.style={}, ID/.style={}, UT/.style={}, AZ/.style={}, NV/.style={}, OR/.style={}, WA/.style={}, CA/.style={}}

\tikzset{
every state/.style={USA map/state/.style={#1}},
HI/.style={USA map/HI/.style={#1}}, AK/.style={USA map/AK/.style={#1}}, FL/.style={USA map/FL/.style={#1}}, NH/.style={USA map/NH/.style={#1}}, MI/.style={USA map/MI/.style={#1}}, SP/.style={USA map/SP/.style={#1}}, VT/.style={USA map/VT/.style={#1}}, ME/.style={USA map/ME/.style={#1}}, RI/.style={USA map/RI/.style={#1}}, NY/.style={USA map/NY/.style={#1}}, PA/.style={USA map/PA/.style={#1}}, NJ/.style={USA map/NJ/.style={#1}}, DE/.style={USA map/DE/.style={#1}}, MD/.style={USA map/MD/.style={#1}}, VA/.style={USA map/VA/.style={#1}}, WV/.style={USA map/WV/.style={#1}}, OH/.style={USA map/OH/.style={#1}}, IN/.style={USA map/IN/.style={#1}}, IL/.style={USA map/IL/.style={#1}}, CT/.style={USA map/CT/.style={#1}}, WI/.style={USA map/WI/.style={#1}}, NC/.style={USA map/NC/.style={#1}}, DC/.style={USA map/DC/.style={#1}}, MA/.style={USA map/MA/.style={#1}}, TN/.style={USA map/TN/.style={#1}}, AR/.style={USA map/AR/.style={#1}}, MO/.style={USA map/MO/.style={#1}}, GA/.style={USA map/GA/.style={#1}}, SC/.style={USA map/SC/.style={#1}}, KY/.style={USA map/KY/.style={#1}}, AL/.style={USA map/AL/.style={#1}}, LA/.style={USA map/LA/.style={#1}}, MS/.style={USA map/MS/.style={#1}}, IA/.style={USA map/IA/.style={#1}}, MN/.style={USA map/MN/.style={#1}}, OK/.style={USA map/OK/.style={#1}}, TX/.style={USA map/TX/.style={#1}}, NM/.style={USA map/NM/.style={#1}}, KS/.style={USA map/KS/.style={#1}}, NE/.style={USA map/NE/.style={#1}}, SD/.style={USA map/SD/.style={#1}}, ND/.style={USA map/ND/.style={#1}}, WY/.style={USA map/WY/.style={#1}}, MT/.style={USA map/MT/.style={#1}}, CO/.style={USA map/CO/.style={#1}}, ID/.style={USA map/ID/.style={#1}}, UT/.style={USA map/UT/.style={#1}}, AZ/.style={USA map/AZ/.style={#1}}, NV/.style={USA map/NV/.style={#1}}, OR/.style={USA map/OR/.style={#1}}, WA/.style={USA map/WA/.style={#1}}, CA/.style={USA map/CA/.style={#1}}
}

\newcommand{\USA}[1][]{
    \begin{scope}[y=0.80pt,x=0.80pt,yscale=-1, inner sep=0pt, outer sep=0pt,
    #1
    ]

    % HI
    \path[USA map/state, USA map/HI, local bounding box=HI] (233.0875,519.3095) -- (235.0274,515.7529) -- (237.2907,515.4296) --
      (237.6140,516.2379) -- (235.5124,519.3095) -- (233.0875,519.3095) --
      cycle(243.2722,515.5913) -- (249.4153,518.1778) -- (251.5169,517.8545) --
      (253.1335,513.9747) -- (252.4869,510.5798) -- (248.2837,510.0948) --
      (244.2421,511.8731) -- (243.2722,515.5913) -- cycle(273.9878,525.6143) --
      (277.7060,531.1107) -- (280.1309,530.7874) -- (281.2625,530.3024) --
      (282.7175,531.5957) -- (286.4357,531.4341) -- (287.4057,529.9791) --
      (284.4958,528.2009) -- (282.5558,524.4826) -- (280.4542,520.9261) --
      (274.6344,523.8360) -- (273.9878,525.6143) -- cycle(294.1954,534.5056) --
      (295.4887,532.5657) -- (300.1769,533.5357) -- (300.8236,533.0507) --
      (306.9667,533.6973) -- (306.6434,534.9906) -- (304.0568,536.4456) --
      (299.6919,536.1222) -- (294.1954,534.5056) -- cycle(299.5303,539.6788) --
      (301.4702,543.5587) -- (304.5418,542.4270) -- (304.8651,540.8104) --
      (303.2485,538.7088) -- (299.5303,538.3855) -- (299.5303,539.6788) --
      cycle(306.4817,538.5472) -- (308.7450,535.6373) -- (313.4331,538.0622) --
      (317.7980,539.1938) -- (322.1628,541.9421) -- (322.1628,543.8820) --
      (318.6063,545.6603) -- (313.7565,546.6302) -- (311.3315,545.1753) --
      (306.4817,538.5472) -- cycle(323.1328,554.0666) -- (324.7494,552.7734) --
      (328.1443,554.3900) -- (335.7424,557.9465) -- (339.1373,560.0481) --
      (340.7539,562.4730) -- (342.6938,566.8379) -- (346.7353,569.4245) --
      (346.4120,570.7178) -- (342.5321,573.9510) -- (338.3290,575.4059) --
      (336.8740,574.7593) -- (333.8024,576.5375) -- (331.3775,579.7708) --
      (329.1143,582.6807) -- (327.3360,582.5190) -- (323.7794,579.9324) --
      (323.4561,575.4059) -- (324.1028,572.9810) -- (322.4862,567.3229) --
      (320.3846,565.5446) -- (320.2229,562.9580) -- (322.4862,561.9880) --
      (324.5878,558.9165) -- (325.0727,557.9465) -- (323.4561,556.1682) --
      (323.1328,554.0666) -- cycle;

    % AK
    \path[USA map/state, USA map/AK, local bounding box=AK] (158.0767,453.6750) -- (157.7534,539.0322) -- (159.3700,540.0021) --
      (162.4416,540.1638) -- (163.8965,539.0322) -- (166.4831,539.0322) --
      (166.6447,541.9420) -- (173.5962,548.7318) -- (174.0812,551.3184) --
      (177.4760,549.3785) -- (178.1227,549.2168) -- (178.4460,546.1452) --
      (179.9010,544.5286) -- (181.0326,544.3670) -- (182.9725,542.9120) --
      (186.0441,545.0136) -- (186.6907,547.9235) -- (188.6307,549.0551) --
      (189.7623,551.4801) -- (193.6422,553.2583) -- (197.0371,559.2398) --
      (199.7853,563.1197) -- (202.0486,565.8679) -- (203.5035,569.5861) --
      (208.5150,571.3644) -- (213.6882,573.4660) -- (214.6581,577.8308) --
      (215.1431,580.9024) -- (214.1732,584.2973) -- (212.3949,586.5605) --
      (210.7783,585.7522) -- (209.3233,582.6807) -- (206.5751,581.2257) --
      (204.7968,580.0941) -- (203.9885,580.9024) -- (205.4434,583.6507) --
      (205.6051,587.3689) -- (204.4735,587.8538) -- (202.5335,585.9139) --
      (200.4320,584.6206) -- (200.9169,586.2372) -- (202.2102,588.0155) --
      (201.4019,588.8238) .. controls (201.4019,588.8238) and (200.5936,588.5005) ..
      (200.1086,587.8538) .. controls (199.6236,587.2072) and (198.0070,584.4590) ..
      (198.0070,584.4590) -- (197.0371,582.1957) .. controls (197.0371,582.1957) and
      (196.7137,583.4890) .. (196.0671,583.1657) .. controls (195.4204,582.8423) and
      (194.7738,581.7107) .. (194.7738,581.7107) -- (196.5521,579.7708) --
      (195.0971,578.3158) -- (195.0971,573.3043) -- (194.2888,573.3043) --
      (193.4805,576.6992) -- (192.3489,577.1842) -- (191.3789,573.4660) --
      (190.7323,569.7478) -- (189.9240,569.2628) -- (190.2473,574.9209) --
      (190.2473,576.0526) -- (188.7923,574.7593) -- (185.2358,568.7778) --
      (183.1342,568.2928) -- (182.4876,564.5746) -- (180.8709,561.6647) --
      (179.2543,560.5331) -- (179.2543,558.2698) -- (181.3559,556.9765) --
      (180.8709,556.6532) -- (178.2844,557.2999) -- (174.8895,554.8750) --
      (172.3029,551.9650) -- (167.4531,549.3785) -- (163.4115,546.7919) --
      (164.7048,543.5587) -- (164.7048,541.9421) -- (162.9265,543.5587) --
      (160.0166,544.6903) -- (156.2984,543.5587) -- (150.6403,541.1338) --
      (145.1438,541.1338) -- (144.4972,541.6187) -- (138.0307,537.7389) --
      (135.9291,537.4155) -- (133.1809,531.5957) -- (129.6243,531.9191) --
      (126.0678,533.3740) -- (126.5528,537.9005) -- (127.6844,534.9906) --
      (128.6544,535.3139) -- (127.1994,539.6788) -- (130.4326,536.9306) --
      (131.0793,538.5472) -- (127.1994,542.9120) -- (125.9061,542.5887) --
      (125.4211,540.6488) -- (124.1279,539.8405) -- (122.8346,540.9721) --
      (120.0863,539.1938) -- (117.0148,541.2954) -- (115.2365,543.3970) --
      (111.8416,545.4986) -- (107.1534,545.3369) -- (106.6684,543.2353) --
      (110.3866,542.5887) -- (110.3866,541.2954) -- (108.1234,540.6488) --
      (109.0934,538.2238) -- (111.3566,534.3440) -- (111.3566,532.5657) --
      (111.5183,531.7574) -- (115.8831,529.4941) -- (116.8531,530.7874) --
      (119.6013,530.7874) -- (118.3081,528.2009) -- (114.5898,527.8775) --
      (109.5783,530.6258) -- (107.1534,534.0206) -- (105.3752,536.6072) --
      (104.2435,538.8705) -- (100.0403,540.3254) -- (96.9688,542.9120) --
      (96.6454,544.5286) -- (98.9087,545.4986) -- (99.7170,547.6002) --
      (96.9688,550.8334) -- (90.5023,555.0366) -- (82.7426,559.2398) --
      (80.6410,560.3714) -- (75.3062,561.5031) -- (69.9713,563.7663) --
      (71.7496,565.0596) -- (70.2947,566.5146) -- (69.8097,567.6462) --
      (67.0614,566.6762) -- (63.8282,566.8379) -- (63.0199,569.1011) --
      (62.0499,569.1011) -- (62.3733,566.6762) -- (58.8167,567.9695) --
      (55.9068,568.9395) -- (52.5119,567.6462) -- (49.6020,569.5861) --
      (46.3688,569.5861) -- (44.2672,570.8794) -- (42.6506,571.6877) --
      (40.5490,571.3644) -- (37.9624,570.2328) -- (35.6992,570.8794) --
      (34.7292,571.8494) -- (33.1126,570.7178) -- (33.1126,568.7778) --
      (36.1841,567.4845) -- (42.4889,568.1312) -- (46.8538,566.5146) --
      (48.9554,564.4130) -- (51.8653,563.7663) -- (53.6436,562.9580) --
      (56.3918,563.1197) -- (58.0084,564.4130) -- (58.9784,564.0896) --
      (61.2416,561.3414) -- (64.3132,560.3714) -- (67.7081,559.7248) --
      (69.0014,559.4015) -- (69.6480,559.8864) -- (70.4563,559.8864) --
      (71.7496,556.1682) -- (75.7911,554.7133) -- (77.7311,550.9951) --
      (79.9943,546.4686) -- (81.6110,545.0136) -- (81.9343,542.4270) --
      (80.3177,543.7203) -- (76.9228,544.3670) -- (76.2761,541.9421) --
      (74.9828,541.6187) -- (74.0129,542.5887) -- (73.8512,545.4986) --
      (72.3963,545.3369) -- (70.9413,539.5171) -- (69.6480,540.8104) --
      (68.5164,540.3254) -- (68.1931,538.3855) -- (64.1515,538.5472) --
      (62.0499,539.6788) -- (59.4634,539.3555) -- (60.9183,537.9005) --
      (61.4033,535.3139) -- (60.7566,533.3740) -- (62.2116,532.4040) --
      (63.5049,532.2424) -- (62.8582,530.4641) -- (62.8582,526.0993) --
      (61.8883,525.1293) -- (61.0800,526.5842) -- (54.9368,526.5842) --
      (53.4819,525.2909) -- (52.8352,521.4111) -- (50.7337,517.8545) --
      (50.7337,516.8846) -- (52.8352,516.0763) -- (52.9969,513.9747) --
      (54.1285,512.8430) -- (53.3202,512.3581) -- (52.0269,512.8430) --
      (50.8953,510.0948) -- (51.8653,505.0833) -- (56.3918,501.8501) --
      (58.9784,500.2335) -- (60.9183,496.5153) -- (63.6666,495.2220) --
      (66.2531,496.3536) -- (66.5765,498.7785) -- (69.0014,498.4552) --
      (72.2346,496.0303) -- (73.8512,496.6769) -- (74.8212,497.3236) --
      (76.4378,497.3236) -- (78.7010,496.0303) -- (79.5094,491.6654) .. controls
      (79.5094,491.6654) and (79.8327,488.7555) .. (80.4793,488.2705) .. controls
      (81.1260,487.7855) and (81.4493,487.3006) .. (81.4493,487.3006) --
      (80.3177,485.3606) -- (77.7311,486.1689) -- (74.4978,486.9772) --
      (72.5579,486.4923) -- (69.0014,484.7140) -- (63.9899,484.5523) --
      (60.4333,480.8341) -- (60.9183,476.9542) -- (61.5650,474.5293) --
      (59.4634,472.7511) -- (57.5234,469.0328) -- (58.0084,468.2245) --
      (64.7982,467.7396) -- (66.8998,467.7396) -- (67.8697,468.7095) --
      (68.5164,468.7095) -- (68.3547,467.0929) -- (72.2346,466.4463) --
      (74.8212,466.7696) -- (76.2761,467.9012) -- (74.8212,470.0028) --
      (74.3362,471.4578) -- (77.0844,473.0744) -- (82.0959,474.8526) --
      (83.8742,473.8827) -- (81.6110,469.5178) -- (80.6410,466.2846) --
      (81.6110,465.4763) -- (78.2161,463.5364) -- (77.7311,462.4047) --
      (78.2161,460.7881) -- (77.4078,456.9083) -- (74.4978,452.2201) --
      (72.0729,448.0169) -- (74.9828,446.0769) -- (78.2161,446.0769) --
      (79.9943,446.7236) -- (84.1975,446.5619) -- (87.9157,443.0054) --
      (89.0474,439.9338) -- (92.7656,437.5089) -- (94.3822,438.4789) --
      (97.1304,437.8322) -- (100.8486,435.7306) -- (101.9803,435.5690) --
      (102.9502,436.3773) -- (107.4767,436.2156) -- (110.2250,433.1441) --
      (111.3566,433.1441) -- (114.9132,435.5690) -- (116.8531,437.6706) --
      (116.3681,438.8022) -- (117.0148,439.9338) -- (118.6314,438.3172) --
      (122.5112,438.6405) -- (122.8346,442.3587) -- (124.7745,443.8137) --
      (131.8876,444.4603) -- (138.1924,448.6635) -- (139.6473,447.6936) --
      (144.8205,450.2801) -- (146.9221,449.6335) -- (148.8620,448.8252) --
      (153.7119,450.7651) -- (158.0767,453.6750) -- cycle(42.9739,482.6124) --
      (45.0755,487.9472) -- (44.9138,488.9172) -- (42.0039,488.5938) --
      (40.2257,484.5523) -- (38.4474,483.0974) -- (36.0225,483.0974) --
      (35.8608,480.5108) -- (37.6391,478.0859) -- (38.7707,480.5108) --
      (40.2257,481.9657) -- (42.9739,482.6124) -- cycle(40.3873,516.0763) --
      (44.1055,516.8846) -- (47.8237,517.8545) -- (48.6321,518.8245) --
      (47.0154,522.5427) -- (43.9439,522.3810) -- (40.5490,518.8245) --
      (40.3873,516.0763) -- cycle(19.6947,502.0117) -- (20.8263,504.5983) --
      (21.9580,506.2149) -- (20.8263,507.0232) -- (18.7247,503.9517) --
      (18.7247,502.0117) -- (19.6947,502.0117) -- cycle(5.9535,575.0826) --
      (9.3484,572.8193) -- (12.7433,571.8494) -- (15.3298,572.1727) --
      (15.8148,573.7893) -- (17.7548,574.2743) -- (19.6947,572.3344) --
      (19.3714,570.7178) -- (22.1196,570.0711) -- (25.0295,572.6577) --
      (23.8979,574.4360) -- (19.5330,575.5676) -- (16.7848,575.0826) --
      (13.0666,573.9510) -- (8.7017,575.4059) -- (7.0851,575.7292) --
      (5.9535,575.0826) -- cycle(54.9368,570.5561) -- (56.5535,572.4960) --
      (58.6550,570.8794) -- (57.2001,569.5861) -- (54.9368,570.5561) --
      cycle(57.8467,573.6276) -- (58.9784,571.3644) -- (61.0800,571.6877) --
      (60.2717,573.6276) -- (57.8467,573.6276) -- cycle(81.4493,571.6877) --
      (82.9042,573.4660) -- (83.8742,572.3344) -- (83.0659,570.3944) --
      (81.4493,571.6877) -- cycle(90.1790,559.2398) -- (91.3106,565.0596) --
      (94.2205,565.8679) -- (99.2320,562.9580) -- (103.5969,560.3714) --
      (101.9803,557.9465) -- (102.4652,555.5216) -- (100.3636,556.8149) --
      (97.4538,556.0066) -- (99.0704,554.8750) -- (101.0103,555.6833) --
      (104.8902,553.9050) -- (105.3751,552.4500) -- (102.9502,551.6417) --
      (103.7585,549.7018) -- (101.0103,551.6417) -- (96.3221,555.1983) --
      (91.4723,558.1082) -- (90.1790,559.2398) -- cycle(132.5342,539.3555) --
      (134.9592,537.9005) -- (133.9892,536.1222) -- (132.2109,537.0922) --
      (132.5342,539.3555) -- cycle;

    % FL
    \path[USA map/state, USA map/FL, local bounding box=FL] (759.8167,439.1428) -- (762.0824,446.4614) -- (765.8121,456.2037) --
      (771.1468,465.5800) -- (774.8650,471.8847) -- (779.7149,477.3812) --
      (783.7564,481.0994) -- (785.3730,484.0093) -- (784.2414,485.3025) --
      (783.4330,486.5958) -- (786.3429,494.0322) -- (789.2528,496.9421) --
      (791.8394,502.2769) -- (795.3959,508.0967) -- (799.9224,516.3413) --
      (801.2157,523.9394) -- (801.7007,535.9023) -- (802.3473,537.6805) --
      (802.0240,541.0754) -- (799.5991,542.3687) -- (799.9224,544.3086) --
      (799.2758,546.2485) -- (799.5991,548.6734) -- (800.0841,550.6134) --
      (797.3358,553.8466) -- (794.2643,555.3015) -- (790.3844,555.4632) --
      (788.9295,557.0798) -- (786.5046,558.0497) -- (785.2113,557.5648) --
      (784.0797,556.5948) -- (783.7564,553.6849) -- (782.9481,550.2900) --
      (779.5532,545.1169) -- (775.9967,542.8537) -- (772.1168,542.5303) --
      (771.3085,543.8236) -- (768.2370,539.4588) -- (767.5903,535.9023) --
      (765.0037,531.8608) -- (763.2255,530.7291) -- (761.6089,532.8307) --
      (759.8306,532.5074) -- (757.7290,527.4959) -- (754.8191,523.6161) --
      (751.9092,518.2813) -- (749.3227,515.2097) -- (745.7662,511.4915) --
      (747.8677,509.0666) -- (751.1009,503.5702) -- (750.9393,501.9536) --
      (746.4128,500.9836) -- (744.7962,501.6302) -- (745.1195,502.2769) --
      (747.7061,503.2468) -- (746.2511,507.7733) -- (745.4428,508.2583) --
      (743.6646,504.2168) -- (742.3713,499.3670) -- (742.0480,496.6188) --
      (743.5029,491.9306) -- (743.5029,482.3927) -- (740.4314,478.6745) --
      (739.1381,475.6029) -- (733.9649,474.3096) -- (732.0250,473.6630) --
      (730.4084,471.0764) -- (727.0135,469.4598) -- (725.8819,466.0649) --
      (723.1337,465.0950) -- (720.7088,461.3768) -- (716.5056,459.9219) --
      (713.5957,458.4669) -- (711.0092,458.4669) -- (706.9676,459.2752) --
      (706.8060,461.2151) -- (707.6143,462.1851) -- (707.1293,463.3167) --
      (704.0578,463.1551) -- (700.3396,466.7116) -- (696.7830,468.6515) --
      (692.9032,468.6515) -- (689.6700,469.9448) -- (689.3466,467.1966) --
      (687.7300,465.2566) -- (684.8202,464.1250) -- (683.2036,462.6701) --
      (675.1205,458.7902) -- (667.5225,457.0120) -- (663.1577,457.6586) --
      (657.1762,458.1436) -- (651.1948,460.2452) -- (647.7155,460.8581) --
      (647.4776,452.8084) -- (644.8910,450.8685) -- (643.1128,449.0902) --
      (643.4361,446.0186) -- (653.6207,444.7254) -- (679.1631,441.8155) --
      (685.9529,441.1688) -- (691.3889,441.4491) -- (693.9754,445.3290) --
      (695.4304,446.7839) -- (703.5285,447.2991) -- (714.3483,446.6525) --
      (735.8607,445.3592) -- (741.3064,444.6848) -- (746.4140,444.8893) --
      (746.8408,447.7992) -- (749.0738,448.6075) -- (749.3087,443.9775) --
      (747.7805,439.8046) -- (749.0889,438.3647) -- (754.6436,438.8195) --
      (759.8167,439.1428) -- cycle(772.3621,571.5479) -- (774.7870,570.9012) --
      (776.0803,570.6588) -- (777.5353,568.3147) -- (779.8793,566.6980) --
      (781.1726,567.1830) -- (782.8701,567.5064) -- (783.2742,568.5571) --
      (779.7985,569.7696) -- (775.5953,571.2246) -- (773.2512,572.4370) --
      (772.3621,571.5479) -- cycle(785.8608,566.5364) -- (787.0733,567.5872) --
      (789.8215,565.4856) -- (795.1563,561.2824) -- (798.8745,557.4025) --
      (801.3803,550.7744) -- (802.3502,549.0770) -- (802.5119,545.6821) --
      (801.7844,546.1671) -- (800.8145,548.9962) -- (799.3595,553.6035) --
      (796.1263,558.8575) -- (791.7614,563.0607) -- (788.3666,565.0006) --
      (785.8608,566.5364) -- cycle;

    % NH
    \path[USA map/state, USA map/NH, local bounding box=NH] (880.7990,142.4248) -- (881.6680,141.3483) -- (882.7582,138.0572) --
      (880.2152,137.1438) -- (879.7302,134.0722) -- (875.8503,132.9406) --
      (875.5270,130.1923) -- (868.2523,106.7515) -- (863.6508,92.2085) --
      (862.7538,92.2034) -- (862.1071,93.8200) -- (861.4605,93.3351) --
      (860.4905,92.3651) -- (859.0356,94.3050) -- (858.9871,99.3371) --
      (859.2987,105.0043) -- (861.2386,107.7525) -- (861.2386,111.7941) --
      (857.5204,116.8568) -- (854.9339,117.9885) -- (854.9339,119.1201) --
      (856.0655,120.8984) -- (856.0655,129.4664) -- (855.2572,138.6811) --
      (855.0955,143.5309) -- (856.0655,144.8242) -- (855.9038,149.3507) --
      (855.4188,151.1289) -- (856.3876,151.8382) -- (873.1753,147.4136) --
      (875.3502,146.8112) -- (877.1938,144.0378) -- (880.7990,142.4247) -- cycle;

    % % path57
    % \path[USA map/state, USA map/draw, local bounding box=draw[path57]=ca9a9a9,line width=1.600pt] (211.0000,493.0000) --
    %   (211.0000,548.0000) -- (247.0000,593.0000)(0.0000,425.0000) --
    %   (144.0000,425.0000) -- (211.0000,493.0000) -- (297.0000,493.0000) --
    %   (350.0000,547.0000) -- (350.0000,593.0000);

    \begin{scope}% MI
      % MI-
    \path[USA map/state, USA map/MI, local bounding box=MI] (697.8601,177.2369) -- (694.6269,168.9922)
        -- (692.3636,159.9392) -- (689.9387,156.7060) -- (687.3521,154.9277) --
        (685.7355,156.0594) -- (681.8557,157.8376) -- (679.9158,162.8491) --
        (677.1675,166.5673) -- (676.0359,167.2139) -- (674.5810,166.5673) .. controls
        (674.5810,166.5673) and (671.9944,165.1123) .. (672.1561,164.4657) .. controls
        (672.3177,163.8191) and (672.6410,159.4542) .. (672.6410,159.4542) --
        (676.0359,158.1609) -- (676.8442,154.7661) -- (677.4908,152.1795) --
        (679.9158,150.5629) -- (679.5924,140.5400) -- (677.9758,138.2767) --
        (676.6825,137.4684) -- (675.8742,135.3668) -- (676.6825,134.5585) --
        (678.2991,134.8818) -- (678.4608,133.2652) -- (676.0359,131.0020) --
        (674.7426,128.4154) -- (672.1561,128.4154) -- (667.6296,126.9605) --
        (662.1331,123.5656) -- (659.3849,123.5656) -- (658.7382,124.2123) --
        (657.7683,123.7273) -- (654.6967,121.4640) -- (651.7868,123.2423) --
        (648.8769,125.5055) -- (649.2003,129.0621) -- (650.1702,129.3854) --
        (652.2718,129.8704) -- (652.7568,130.6787) -- (650.1702,131.4870) --
        (647.5837,131.8103) -- (646.1287,133.5886) -- (645.8054,135.6901) --
        (646.1287,137.3067) -- (646.4520,142.8032) -- (642.8955,144.9048) --
        (642.2489,144.7431) -- (642.2489,140.5400) -- (643.5421,138.1151) --
        (644.1888,135.6901) -- (643.3805,134.8818) -- (641.4406,135.6901) --
        (640.4706,139.8933) -- (637.7224,141.0249) -- (635.9441,142.9649) --
        (635.7824,143.9348) -- (636.4291,144.7431) -- (635.7824,147.3297) --
        (633.5192,147.8147) -- (633.5192,148.9463) -- (634.3275,151.3712) --
        (633.1959,157.5143) -- (631.5793,161.5558) -- (632.2259,166.2440) --
        (632.7109,167.3756) -- (631.9026,169.8005) -- (631.5793,170.6088) --
        (631.2560,173.3570) -- (634.8125,179.3385) -- (637.7224,185.8049) --
        (639.1773,190.6547) -- (638.3690,195.3429) -- (637.3991,201.3243) --
        (634.9741,206.4974) -- (634.6508,209.2457) -- (631.3920,212.3308) --
        (635.8006,212.1688) -- (657.2191,209.9055) -- (664.4969,208.9184) --
        (664.5933,210.5848) -- (671.4452,209.3723) -- (681.7433,207.8692) --
        (685.5975,207.4083) -- (685.7356,206.8207) -- (685.8972,205.3658) --
        (687.9988,201.6476) -- (689.9994,199.9098) -- (689.7771,194.8579) --
        (691.3741,193.2609) -- (692.4647,192.9179) -- (692.6870,189.3614) --
        (694.2227,186.3303) -- (695.2735,186.9365) -- (695.4352,187.5832) --
        (696.2435,187.7448) -- (698.1834,186.7749) -- (697.8601,177.2369) -- cycle;

      % SP-
      \path[USA map/state, USA map/SP, local bounding box=SP] (581.6193,82.0590) -- (583.4483,80.0014) --
        (585.6202,79.2012) -- (590.9929,75.3146) -- (593.2791,74.7431) --
        (593.7363,75.2003) -- (588.5923,80.3443) -- (585.2773,82.2876) --
        (583.2197,83.2021) -- (581.6193,82.0590) -- cycle(667.7937,114.1872) --
        (668.4403,116.6929) -- (671.6736,116.8546) -- (672.9668,115.6421) .. controls
        (672.9668,115.6421) and (672.8860,114.1872) .. (672.5627,114.0255) .. controls
        (672.2394,113.8639) and (670.9461,112.1664) .. (670.9461,112.1664) --
        (668.7637,112.4089) -- (667.1470,112.5706) -- (666.8237,113.7022) --
        (667.7937,114.1872) -- cycle(567.4921,111.2132) -- (568.2084,110.6328) --
        (570.9566,109.8245) -- (574.5131,107.5612) -- (574.5131,106.5913) --
        (575.1598,105.9446) -- (581.1412,104.9747) -- (583.5661,103.0347) --
        (587.9310,100.9331) -- (588.0926,99.6399) -- (590.0325,96.7300) --
        (591.8108,95.9217) -- (593.1041,94.1434) -- (595.3673,91.8802) --
        (599.7322,89.4553) -- (604.4203,88.9703) -- (605.5519,90.1019) --
        (605.2286,91.0719) -- (601.5104,92.0418) -- (600.0555,95.1134) --
        (597.7922,95.9217) -- (597.3073,98.3466) -- (594.8824,101.5798) --
        (594.5590,104.1664) -- (595.3673,104.6513) -- (596.3373,103.5197) --
        (599.8938,100.6098) -- (601.1871,101.9031) -- (603.4504,101.9031) --
        (606.6836,102.8731) -- (608.1385,104.0047) -- (609.5934,107.0762) --
        (612.3417,109.8245) -- (616.2215,109.6628) -- (617.6765,108.6928) --
        (619.2931,109.9861) -- (620.9097,110.4711) -- (622.2030,109.6628) --
        (623.3346,109.6628) -- (624.9512,108.6928) -- (628.9927,105.1363) --
        (632.3876,104.0047) -- (639.0157,103.6814) -- (643.5421,101.7414) --
        (646.1287,100.4482) -- (647.5837,100.6098) -- (647.5837,106.2679) --
        (648.0687,106.5913) -- (650.9785,107.3996) -- (652.9185,106.9146) --
        (659.0616,105.2980) -- (660.1932,104.1664) -- (661.6481,104.6513) --
        (661.6481,111.6027) -- (664.8813,114.6743) -- (666.1746,115.3209) --
        (667.4679,116.2909) -- (666.1746,116.6142) -- (665.3663,116.2909) --
        (661.6481,115.8059) -- (659.5465,116.4526) -- (657.2833,116.2909) --
        (654.0501,117.7458) -- (652.2718,117.7458) -- (646.4520,116.4526) --
        (641.2789,116.6142) -- (639.3390,119.2008) -- (632.3876,119.8474) --
        (629.9627,120.6557) -- (628.8311,123.7273) -- (627.5378,124.8589) --
        (627.0528,124.6972) -- (625.5978,123.0806) -- (621.0714,125.5055) --
        (620.4247,125.5055) -- (619.2931,123.8889) -- (618.4848,124.0506) --
        (616.5449,128.4154) -- (615.5749,132.4569) -- (612.3938,139.4577) --
        (611.2170,138.4235) -- (609.8453,137.3922) -- (607.9045,127.1041) --
        (604.3600,125.7341) -- (602.3074,123.4479) -- (590.1871,120.7044) --
        (587.3318,119.6747) -- (579.1014,117.5020) -- (571.2114,116.3589) --
        (567.4921,111.2132) -- cycle;

    \end{scope}
    % VT
    \path[USA map/state, USA map/VT, local bounding box=VT] (844.4842,154.0579) -- (844.8009,148.7123) -- (841.9101,137.9281) --
      (841.2635,137.6048) -- (838.3536,136.3115) -- (839.1619,133.4016) --
      (838.3536,131.3000) -- (835.6536,126.6600) -- (836.6235,122.7802) --
      (835.8152,117.6070) -- (833.3903,111.1406) -- (832.5847,106.2181) --
      (859.0041,99.4863) -- (859.3128,105.0085) -- (861.2291,107.7507) --
      (861.2291,111.7922) -- (857.5219,116.8502) -- (854.9353,117.9929) --
      (854.9243,119.1135) -- (856.2343,120.6326) -- (855.9234,128.7305) --
      (855.3139,137.9894) -- (855.0860,143.5463) -- (856.0560,144.8396) --
      (855.8943,149.4103) -- (855.4093,151.1002) -- (856.4235,151.8274) --
      (848.9860,153.3341) -- (844.4842,154.0579) -- cycle;

    % ME
    \path[USA map/state, USA map/ME, local bounding box=ME] (922.8398,78.8307) -- (924.7797,80.9323) -- (927.0429,84.6505) --
      (927.0429,86.5904) -- (924.9413,91.2786) -- (923.0014,91.9252) --
      (919.6065,94.9968) -- (914.7567,100.4932) .. controls (914.7567,100.4932) and
      (914.1101,100.4932) .. (913.4635,100.4932) .. controls (912.8168,100.4932) and
      (912.4935,98.3916) .. (912.4935,98.3916) -- (910.7152,98.5533) --
      (909.7453,100.0082) -- (907.3204,101.4632) -- (906.3504,102.9181) --
      (907.9670,104.3731) -- (907.4820,105.0197) -- (906.9970,107.7679) --
      (905.0571,107.6063) -- (905.0571,105.9897) -- (904.7338,104.6964) --
      (903.2789,105.0197) -- (901.5006,101.7865) -- (899.3990,103.0798) --
      (900.6923,104.5347) -- (901.0156,105.6664) -- (900.2073,106.9596) --
      (900.5306,110.0312) -- (900.6923,111.6478) -- (899.0757,114.2344) --
      (896.1658,114.7193) -- (895.8425,117.6292) -- (890.5077,120.7008) --
      (889.2144,121.1858) -- (887.5978,119.7308) -- (884.5262,123.2873) --
      (885.4962,126.5206) -- (884.0412,127.8138) -- (883.8796,132.1787) --
      (882.7563,138.4380) -- (880.2941,137.2821) -- (879.8091,134.2105) --
      (875.9292,133.0789) -- (875.6059,130.3306) -- (868.3311,106.8898) --
      (863.6326,92.2501) -- (865.0531,92.1319) -- (866.5669,92.5418) --
      (866.5669,89.9553) -- (867.8752,85.4588) -- (870.4618,80.7706) --
      (871.9167,76.7291) -- (869.9768,74.3042) -- (869.9768,68.3228) --
      (870.7851,67.3528) -- (871.5934,64.6046) -- (871.4317,63.1497) --
      (871.2701,58.2998) -- (873.0483,53.4500) -- (875.9582,44.5587) --
      (878.0598,40.3555) -- (879.3531,40.3555) -- (880.6464,40.5172) --
      (880.6464,41.6488) -- (881.9397,43.9121) -- (884.6879,44.5587) --
      (885.4962,43.7504) -- (885.4962,42.7804) -- (889.5377,39.8705) --
      (891.3160,38.0923) -- (892.7709,38.2539) -- (898.7523,40.6788) --
      (900.6923,41.6488) -- (909.7453,71.5560) -- (915.7267,71.5560) --
      (916.5350,73.4959) -- (916.6967,78.3457) -- (919.6066,80.6090) --
      (920.4149,80.6090) -- (920.5765,80.1240) -- (920.0915,78.9924) --
      (922.8398,78.8307) -- cycle(901.9080,108.9783) -- (903.4438,107.4425) --
      (904.8179,108.4933) -- (905.3837,110.9182) -- (903.6863,111.8073) --
      (901.9080,108.9782) -- cycle(908.6169,103.0776) -- (910.3952,104.9367) ..
      controls (910.3952,104.9367) and (911.6885,105.0175) .. (911.6885,104.6942) ..
      controls (911.6885,104.3709) and (911.9310,102.6735) .. (911.9310,102.6735) --
      (912.8201,101.8652) -- (912.0118,100.0869) -- (909.9911,100.8144) --
      (908.6169,103.0776) -- cycle;

    % RI
    \path[USA map/state, USA map/RI, local bounding box=RI] (874.0700,178.8954) -- (870.3742,163.9394) -- (876.6435,162.0942) --
      (878.8346,164.0214) -- (882.1411,168.3420) -- (884.8290,172.7441) --
      (881.8297,174.3689) -- (880.5364,174.2072) -- (879.4048,175.9855) --
      (876.9799,177.9254) -- (874.0700,178.8954) -- cycle;

    % NY
    \path[USA map/state, USA map/NY, local bounding box=NY] (830.3794,188.7456) -- (829.2478,187.7756) -- (826.6612,187.6140) --
      (824.3980,185.6741) -- (822.7674,179.5449) -- (819.3089,179.6354) --
      (816.8652,176.9272) -- (797.4799,181.3092) -- (754.4781,190.0389) --
      (746.9485,191.2669) -- (746.2103,184.7985) -- (747.6384,183.6731) --
      (748.9317,182.5415) -- (749.9017,180.9249) -- (751.6799,179.7933) --
      (753.6198,178.0150) -- (754.1048,176.3984) -- (756.2064,173.6502) --
      (757.3380,172.6802) -- (757.1764,171.7103) -- (755.8831,168.6387) --
      (754.1048,168.4770) -- (752.1649,162.3339) -- (755.0748,160.5557) --
      (759.4396,159.1007) -- (763.4811,157.8074) -- (766.7143,157.3225) --
      (773.0191,157.1608) -- (774.9590,158.4541) -- (776.5756,158.6158) --
      (778.6772,157.3225) -- (781.2638,156.1908) -- (786.4369,155.7059) --
      (788.5385,153.9276) -- (790.3168,150.6944) -- (791.9334,148.7545) --
      (794.0350,148.7545) -- (795.9749,147.6228) -- (796.1365,145.3596) --
      (794.6816,143.2580) -- (794.3583,141.8031) -- (795.4899,139.7015) --
      (795.4899,138.2465) -- (793.7116,138.2465) -- (791.9334,137.4382) --
      (791.1251,136.3066) -- (790.9634,133.7200) -- (796.7832,128.2236) --
      (797.4298,127.4153) -- (798.8848,124.5054) -- (801.7947,119.9789) --
      (804.5429,116.2607) -- (806.6445,113.8358) -- (809.0596,112.0102) --
      (812.1409,110.7643) -- (817.6374,109.4710) -- (820.8706,109.6326) --
      (825.3971,108.1777) -- (832.9623,106.1065) -- (833.4821,111.0862) --
      (835.9070,117.5526) -- (836.7153,122.7258) -- (835.7453,126.6056) --
      (838.3319,131.1321) -- (839.1402,133.2337) -- (838.3319,136.1436) --
      (841.2418,137.4369) -- (841.8884,137.7602) -- (844.9600,148.7532) --
      (844.4237,153.8128) -- (843.9387,164.6441) -- (844.7470,170.1406) --
      (845.5553,173.6971) -- (847.0103,180.9719) -- (847.0103,189.0549) --
      (845.8787,191.3182) -- (847.7180,193.3109) -- (848.5145,194.9894) --
      (846.5746,196.7676) -- (846.8979,198.0609) -- (848.1912,197.7376) --
      (849.6462,196.4443) -- (851.9094,193.8577) -- (853.0410,193.2111) --
      (854.6576,193.8577) -- (856.9209,194.0194) -- (864.8422,190.1396) --
      (867.7521,187.3913) -- (869.0454,185.9364) -- (873.2486,187.5530) --
      (869.8537,191.1095) -- (865.9739,194.0194) -- (858.8608,199.3542) --
      (856.2742,200.3242) -- (850.4545,202.2641) -- (846.4130,203.3957) --
      (845.2382,202.8628) -- (844.9942,199.1743) -- (845.4792,196.4260) --
      (845.3175,194.3244) -- (842.5040,192.6254) -- (837.9775,191.6555) --
      (834.0976,190.5238) -- (830.3794,188.7456) -- cycle;

    % PA
    \path[USA map/state, USA map/PA, local bounding box=PA] (825.1237,224.6920) -- (826.4321,224.4211) -- (828.7616,223.1678) --
      (829.9735,220.6847) -- (831.5901,218.4215) -- (834.8233,215.3499) --
      (834.8233,214.5416) -- (832.3984,212.9250) -- (828.8419,210.5001) --
      (827.8719,207.9135) -- (825.1237,207.5902) -- (824.9620,206.4586) --
      (824.1537,203.7103) -- (826.4170,202.5787) -- (826.5787,200.1538) --
      (825.2854,198.8605) -- (825.4470,197.2439) -- (827.3870,194.1724) --
      (827.3870,191.1008) -- (830.0846,188.4549) -- (829.1643,187.7799) --
      (826.6402,187.5870) -- (824.3457,185.6471) -- (822.7958,179.5310) --
      (819.2912,179.6316) -- (816.8360,176.9282) -- (798.7450,181.1260) --
      (755.7432,189.8557) -- (746.8519,191.3106) -- (746.2312,184.7892) --
      (740.8687,189.8569) -- (739.5754,190.3419) -- (735.3731,193.3508) --
      (738.2839,212.4882) -- (740.7655,222.2176) -- (744.3373,241.4791) --
      (747.6066,240.8414) -- (759.5502,239.3389) -- (797.4768,231.6737) --
      (812.3531,228.8504) -- (820.6534,227.2280) -- (820.9205,226.9895) --
      (823.0221,225.3729) -- (825.1237,224.6920) -- cycle;

    % NJ
    \path[USA map/state, USA map/NJ, local bounding box=NJ] (829.6794,188.4602) -- (827.3569,191.1944) -- (827.3569,194.2660) --
      (825.4169,197.3375) -- (825.2553,198.9542) -- (826.5486,200.2474) --
      (826.3869,202.6724) -- (824.1237,203.8040) -- (824.9320,206.5522) --
      (825.0936,207.6838) -- (827.8419,208.0072) -- (828.8118,210.5937) --
      (832.3684,213.0187) -- (834.7933,214.6353) -- (834.7933,215.4436) --
      (831.8101,218.1401) -- (830.1934,220.4034) -- (828.7385,223.1516) --
      (826.4752,224.4449) -- (826.0128,226.0474) -- (825.7703,227.2598) --
      (825.1611,229.8666) -- (826.2533,232.1108) -- (829.4865,235.0206) --
      (834.3364,237.2839) -- (838.3779,237.9305) -- (838.5395,239.3855) --
      (837.7312,240.3554) -- (838.0545,243.1037) -- (838.8628,243.1037) --
      (840.9644,240.6788) -- (841.7727,235.8289) -- (844.5210,231.7874) --
      (847.5925,225.3210) -- (848.7241,219.8246) -- (848.0775,218.6929) --
      (847.9158,209.3166) -- (846.2992,205.9218) -- (845.1676,206.7301) --
      (842.4194,207.0534) -- (841.9344,206.5684) -- (843.0660,205.5984) --
      (845.1676,203.6585) -- (845.2307,202.5647) -- (844.8463,199.1308) --
      (845.4197,196.3826) -- (845.3022,194.4136) -- (842.4947,192.6632) --
      (837.4025,191.4875) -- (833.2651,190.1059) -- (829.6795,188.4602) -- cycle;

    % DE
    \path[USA map/state, USA map/DE, local bounding box=DE] (825.6261,228.2791) -- (825.9944,226.1322) -- (826.3695,224.4412) --
      (824.7465,224.8389) -- (823.1310,225.3065) -- (820.9248,227.0708) --
      (822.6449,232.1137) -- (824.9081,237.7718) -- (827.0097,247.4714) --
      (828.6263,253.7762) -- (833.6378,253.6145) -- (839.7799,252.4339) --
      (837.5157,245.0476) -- (836.5457,245.5326) -- (832.9892,243.1077) --
      (831.2109,238.4195) -- (829.2710,234.8630) -- (826.1239,231.9927) --
      (825.2597,229.8946) -- (825.6261,228.2791) -- cycle;

    % MD
    \path[USA map/state, USA map/MD, local bounding box=MD] (839.7917,252.4148) -- (833.7832,253.6186) -- (828.6403,253.7361) --
      (826.7967,246.8137) -- (824.8719,237.6444) -- (822.2993,231.4560) --
      (821.0109,227.0576) -- (813.5049,228.6800) -- (798.6287,231.5033) --
      (761.1773,239.0542) -- (762.3086,244.0659) -- (763.2785,249.7240) --
      (763.6018,249.4007) -- (765.7034,246.9758) -- (767.9667,244.3581) --
      (770.3916,243.7425) -- (771.8466,242.2876) -- (773.6248,239.7010) --
      (774.9181,240.3477) -- (777.8280,240.0243) -- (780.4146,237.9228) --
      (782.4215,236.4695) -- (784.2667,235.9845) -- (785.9110,237.1145) --
      (788.8209,238.5694) -- (790.7609,240.3477) -- (791.9733,241.8835) --
      (796.0957,243.5809) -- (796.0957,246.4908) -- (801.5921,247.7841) --
      (802.7366,248.3260) -- (804.1485,246.2977) -- (807.0304,248.2679) --
      (805.7523,250.7498) -- (804.9870,254.7355) -- (803.2087,257.3220) --
      (803.2087,259.4236) -- (803.8554,261.2019) -- (808.9193,262.5576) --
      (813.2304,262.4959) -- (816.3020,263.4659) -- (818.4035,263.7892) --
      (819.3735,261.6876) -- (817.9186,259.5860) -- (817.9186,257.8077) --
      (815.4937,255.7062) -- (813.3921,250.2097) -- (814.6854,244.8749) --
      (814.5237,242.7733) -- (813.2304,241.4800) .. controls (813.2304,241.4800) and
      (814.6854,239.8634) .. (814.6854,239.2168) .. controls (814.6854,238.5701) and
      (815.1703,237.1152) .. (815.1703,237.1152) -- (817.1103,235.8219) --
      (819.0502,234.2053) -- (819.5352,235.1753) -- (818.0802,236.7919) --
      (816.7869,240.5101) -- (817.1103,241.6417) -- (818.8885,241.9650) --
      (819.3735,247.4615) -- (817.2719,248.4314) -- (817.5952,251.9880) --
      (818.0802,251.8263) -- (819.2118,249.8864) -- (820.8285,251.6646) --
      (819.2118,252.9579) -- (818.8885,256.3528) -- (821.4751,259.7477) --
      (825.3549,260.2327) -- (826.9716,259.4244) -- (830.2081,263.6073) --
      (831.5665,264.1436) -- (838.2201,261.3466) -- (840.2277,257.3228) --
      (839.7917,252.4148) -- cycle(823.8222,261.4435) -- (824.9538,263.9492) --
      (825.1155,265.7275) -- (826.2471,267.5866) .. controls (826.2471,267.5866) and
      (827.1362,266.6975) .. (827.1362,266.3741) .. controls (827.1362,266.0508) and
      (826.4087,263.3026) .. (826.4087,263.3026) -- (825.6813,260.9585) --
      (823.8222,261.4435) -- cycle;

    % VA
    \path[USA map/state, USA map/VA, local bounding box=VA] (831.6389,266.0689) -- (831.4949,264.1219) -- (837.9484,261.5720) --
      (837.1780,264.7899) -- (834.2580,268.5690) -- (833.8399,273.1548) --
      (834.3017,276.5452) -- (832.4737,281.5234) -- (830.3094,283.4395) --
      (828.8391,278.7987) -- (829.2850,273.3496) -- (830.8720,269.1665) --
      (831.6389,266.0689) -- cycle(834.9790,294.3703) -- (776.8049,306.9457) --
      (739.3779,312.2248) -- (732.6996,311.8496) -- (730.1143,313.7760) --
      (722.7752,313.9967) -- (714.3931,314.9743) -- (703.4781,316.5890) --
      (713.9475,310.9778) -- (713.9344,308.9028) -- (715.4545,306.7567) --
      (726.0083,295.2553) -- (729.9550,299.7327) -- (733.7380,300.6967) --
      (736.2815,299.5564) -- (738.5187,298.2452) -- (741.0553,299.5887) --
      (744.9695,298.1610) -- (746.8462,293.6046) -- (749.4471,294.1447) --
      (752.3024,292.0134) -- (754.1016,292.5070) -- (756.9288,288.8304) --
      (757.2771,286.7473) -- (756.3134,285.4718) -- (757.3162,283.6051) --
      (762.5905,271.3280) -- (763.2072,265.5929) -- (764.4361,265.0694) --
      (766.6147,267.5122) -- (770.5505,267.2111) -- (772.4797,259.6374) --
      (775.2737,259.0766) -- (776.3235,256.3355) -- (778.9033,253.9886) --
      (781.6751,248.2934) -- (781.7600,243.2259) -- (791.5815,247.0487) .. controls
      (792.2624,247.3891) and (792.4144,241.9996) .. (792.4144,241.9996) --
      (796.0670,243.5979) -- (796.1353,246.5361) -- (801.9195,247.8355) --
      (804.0525,249.0117) -- (805.7124,251.0674) -- (805.0578,254.7161) --
      (803.1104,257.3071) -- (803.2202,259.3662) -- (803.8092,261.2191) --
      (808.7880,262.4875) -- (813.2392,262.5274) -- (816.3081,263.4860) --
      (818.2516,263.7953) -- (818.9664,266.8838) -- (822.1568,267.2863) --
      (823.0249,268.4863) -- (822.5854,273.1764) -- (823.9601,274.2790) --
      (823.4812,276.2094) -- (824.7106,276.9991) -- (824.4888,278.3837) --
      (821.7948,278.2888) -- (821.8838,279.9044) -- (824.1648,281.4472) --
      (824.2863,282.8591) -- (826.0594,284.6445) -- (826.5512,287.1686) --
      (823.9982,288.5499) -- (825.5704,290.0442) -- (831.3714,288.3584) --
      (834.9790,294.3703) -- cycle;

    % WV
    \path[USA map/state, USA map/WV, local bounding box=WV] (761.1855,238.9673) -- (762.2975,243.9118) -- (763.3810,249.9432) --
      (765.5113,247.3628) -- (767.7745,244.2913) -- (770.3129,243.6757) --
      (771.7678,242.2208) -- (773.5461,239.6342) -- (774.9911,240.2808) --
      (777.9010,239.9575) -- (780.4875,237.8559) -- (782.4944,236.4027) --
      (784.3397,235.9177) -- (785.6436,236.9342) -- (789.2868,238.7558) --
      (791.2268,240.5341) -- (792.6009,241.8273) -- (791.8392,247.3823) --
      (786.0042,244.8411) -- (781.7590,243.2190) -- (781.6579,248.3975) --
      (778.9102,253.9342) -- (776.3802,256.3609) -- (775.1881,259.1102) --
      (772.5445,259.6103) -- (771.6467,263.2122) -- (770.6034,267.1619) --
      (766.6352,267.5026) -- (764.3115,265.0638) -- (763.2403,265.6232) --
      (762.6076,271.0929) -- (761.2574,274.6274) -- (756.2990,285.5823) --
      (757.1956,286.7430) -- (756.9898,288.6516) -- (754.1811,292.5360) --
      (752.3726,291.9918) -- (749.4045,294.1515) -- (746.8622,293.5793) --
      (744.8629,298.1349) .. controls (744.8629,298.1349) and (741.6036,299.5651) ..
      (740.9400,299.5026) .. controls (740.7795,299.4875) and (738.4709,298.2535) ..
      (738.4709,298.2535) -- (736.1344,299.6329) -- (733.7246,300.6773) --
      (729.9799,299.7881) -- (728.8585,298.6199) -- (726.6663,295.5965) --
      (723.5237,293.6084) -- (721.8121,289.9851) -- (717.5273,286.5169) --
      (716.8806,284.2537) -- (714.2940,282.7987) -- (713.4857,281.1821) --
      (713.2432,275.9282) -- (715.4257,275.8474) -- (717.3656,275.0391) --
      (717.5273,272.2908) -- (719.1439,270.8359) -- (719.3055,265.8244) --
      (720.2755,261.9445) -- (721.5688,261.2979) -- (722.8620,262.4295) --
      (723.3470,264.2078) -- (725.1253,263.2378) -- (725.6103,261.6212) --
      (724.4787,259.8430) -- (724.4787,257.4180) -- (725.4486,256.1248) --
      (727.7119,252.7299) -- (729.0052,251.2749) -- (731.1068,251.7599) --
      (733.3700,250.1433) -- (736.4415,246.7484) -- (738.7048,242.8686) --
      (739.0281,237.2105) -- (739.5131,232.1990) -- (739.5131,227.5108) --
      (738.3815,224.4393) -- (739.3514,222.9843) -- (740.6349,221.6910) --
      (744.1262,241.5181) -- (748.7572,240.7670) -- (761.1855,238.9673) -- cycle;

    % OH
    \path[USA map/state, USA map/OH, local bounding box=OH] (735.3250,193.3283) -- (729.2314,197.3817) -- (725.3516,199.6449) --
      (721.9567,203.3631) -- (717.9152,207.2430) -- (714.6820,208.0513) --
      (711.7721,208.5362) -- (706.2756,211.1228) -- (704.1741,211.2845) --
      (700.7792,208.2129) -- (695.6061,208.8596) -- (693.0195,207.4046) --
      (690.6384,206.0538) -- (685.7459,206.7572) -- (675.5612,208.3738) --
      (664.3544,210.5585) -- (665.6477,225.1888) -- (667.4259,238.9300) --
      (670.0125,262.3708) -- (670.5783,267.2020) -- (674.7007,267.0729) --
      (677.1256,266.2646) -- (680.4894,267.7678) -- (682.5598,272.1326) --
      (687.6988,272.1155) -- (689.5905,274.2342) -- (691.3517,274.1689) --
      (693.8901,272.8274) -- (696.3943,273.1989) -- (701.8155,273.6816) --
      (703.5425,271.5489) -- (705.8882,270.2557) -- (707.9587,269.5748) --
      (708.6053,272.3230) -- (710.3836,273.2930) -- (713.8593,275.6371) --
      (716.0417,275.5563) -- (717.3748,275.0638) -- (717.5595,272.3023) --
      (719.1449,270.8473) -- (719.2441,266.0546) .. controls (719.2441,266.0546) and
      (720.2680,261.9455) .. (720.2680,261.9455) -- (721.5673,261.3442) --
      (722.8887,262.4920) -- (723.4268,264.1890) -- (725.1459,263.1516) --
      (725.5849,261.6908) -- (724.4682,259.7878) -- (724.5345,257.4733) --
      (725.2835,256.4010) -- (727.4363,253.0946) -- (728.4865,251.5512) --
      (730.5881,252.0362) -- (732.8513,250.4196) -- (735.9229,247.0247) --
      (738.6944,242.9460) -- (739.0147,237.8905) -- (739.4997,232.8790) --
      (739.3229,227.5721) -- (738.3681,224.6773) -- (738.7193,223.4875) --
      (740.5237,221.7374) -- (738.2349,212.6901) -- (735.3250,193.3283) -- cycle;

    % IN
    \path[USA map/state, USA map/IN, local bounding box=IN] (619.5695,299.9713) -- (619.6348,297.1127) -- (620.1198,292.5862) --
      (622.3831,289.6764) -- (624.1613,285.7965) -- (626.7479,281.5933) --
      (626.2629,275.7735) -- (624.4847,273.0253) -- (624.1613,269.7921) --
      (624.9697,264.2956) -- (624.4847,257.3442) -- (623.1914,241.3398) --
      (621.8981,225.9820) -- (620.9276,214.2620) -- (623.9987,215.1515) --
      (625.4536,216.1215) -- (626.5853,215.7982) -- (628.6868,213.8582) --
      (631.5164,212.2413) -- (636.6092,212.0792) -- (658.5951,209.8160) --
      (664.1708,209.2828) -- (665.6740,225.2390) -- (669.9253,262.0806) --
      (670.5238,267.8521) -- (670.1523,270.1154) -- (671.3802,271.9108) --
      (671.4766,273.2833) -- (668.9554,274.8828) -- (665.4159,276.4341) --
      (662.2138,276.9844) -- (661.6153,281.8514) -- (657.0406,285.1638) --
      (654.2442,289.1743) -- (654.5675,291.5510) -- (653.9862,293.0852) --
      (650.6597,293.0852) -- (649.0742,291.4686) -- (646.5809,292.7308) --
      (643.8979,294.2339) -- (644.0596,297.2884) -- (642.8658,297.5464) --
      (642.3979,296.5283) -- (640.2311,295.0251) -- (636.9807,296.3666) --
      (635.4294,299.3729) -- (633.9916,298.5646) -- (632.5366,296.9651) --
      (628.0723,297.4500) -- (622.4795,298.4200) -- (619.5696,299.9713) -- cycle;

    % IL
    \path[USA map/state, USA map/IL, local bounding box=IL] (619.5415,300.3424) -- (619.5727,297.1127) -- (620.1400,292.4668) --
      (622.4726,289.5509) -- (624.3392,285.4751) -- (626.5722,281.4798) --
      (626.2007,276.2274) -- (624.1955,272.6848) -- (624.0991,269.3382) --
      (624.7940,264.0687) -- (623.9686,256.8903) -- (622.9022,241.1128) --
      (621.6089,226.0955) -- (620.6867,214.4563) -- (620.4141,213.5349) --
      (619.6058,210.9483) -- (618.3126,207.2301) -- (616.6960,205.4519) --
      (615.2410,202.8653) -- (615.0074,197.3764) -- (569.2110,199.9746) --
      (569.4396,202.3466) -- (571.7259,203.0324) -- (572.6403,204.1755) --
      (573.0976,206.0045) -- (576.9842,209.4339) -- (577.6701,211.7201) --
      (576.9842,215.1494) -- (575.1552,218.8074) -- (574.4693,221.3222) --
      (572.1831,223.1512) -- (570.3541,223.8371) -- (565.0958,225.2088) --
      (564.4099,227.0378) -- (563.7241,229.0954) -- (564.4099,230.4672) --
      (566.2389,232.0675) -- (566.0103,236.1827) -- (564.1813,237.7831) --
      (563.4954,239.3834) -- (563.4954,242.1269) -- (561.6665,242.5841) --
      (560.0661,243.7273) -- (559.8375,245.0990) -- (560.0661,247.1566) --
      (558.3514,248.4712) -- (557.3226,251.2718) -- (557.7799,254.9298) --
      (560.0661,262.2457) -- (567.3820,269.7902) -- (572.8690,273.4482) --
      (572.6403,277.7920) -- (573.5548,279.1638) -- (579.9563,279.6210) --
      (582.6997,280.9928) -- (582.0139,284.6507) -- (579.7277,290.5949) --
      (579.0418,293.7956) -- (581.3280,297.6822) -- (587.7294,302.9405) --
      (592.3019,303.6264) -- (594.3595,308.6561) -- (596.4171,311.8568) --
      (595.5026,314.8289) -- (597.1030,318.9441) -- (598.9319,321.0017) --
      (600.3460,320.1210) -- (601.2536,318.0462) -- (603.4667,316.2990) --
      (605.5982,315.6846) -- (608.2007,316.8644) -- (611.8277,318.2401) --
      (613.0167,317.9419) -- (613.2165,315.6834) -- (611.9292,313.2717) --
      (612.2334,310.8949) -- (614.0718,309.5475) -- (617.0944,308.7372) --
      (618.3553,308.2787) -- (617.7427,306.8918) -- (616.9513,304.5374) --
      (618.3839,303.5565) -- (619.5414,300.3424) -- cycle;

    % CT
    \path[USA map/state, USA map/CT, local bounding box=CT] (874.0683,178.8629) -- (870.3909,163.9841) -- (865.6721,164.9044) --
      (844.4433,169.6475) -- (845.4435,172.8731) -- (846.8984,180.1479) --
      (847.0752,189.1148) -- (845.8552,191.2897) -- (847.7760,193.2220) --
      (852.0475,189.3164) -- (855.6040,186.0832) -- (857.5439,183.9816) --
      (858.3523,184.6282) -- (861.1005,183.1733) -- (866.2736,182.0417) --
      (874.0683,178.8629) -- cycle;

    % WI
    \path[USA map/state, USA map/WI, local bounding box=WI] (615.0659,197.3687) -- (614.9992,194.2112) -- (613.8201,189.6847) --
      (613.1734,183.5417) -- (612.0418,181.1167) -- (613.0118,178.0452) --
      (613.8201,175.1353) -- (615.2750,172.5487) -- (614.6284,169.1539) --
      (613.9817,165.5973) -- (614.4667,163.8191) -- (616.4066,161.3942) --
      (616.5683,158.6459) -- (615.7600,157.3526) -- (616.4066,154.7661) --
      (615.9541,150.5954) -- (618.7024,144.9373) -- (621.6122,138.1475) --
      (621.7739,135.8843) -- (621.4506,134.9143) -- (620.6423,135.3993) --
      (616.4391,141.7041) -- (613.6909,145.7456) -- (611.7510,147.5238) --
      (610.9427,149.7871) -- (608.9877,150.5954) -- (607.8561,152.5353) --
      (606.4011,152.2120) -- (606.2395,150.4337) -- (607.5328,148.0088) --
      (609.6343,143.3207) -- (611.4126,141.7040) -- (612.4034,139.3462) --
      (609.8430,137.4449) -- (607.8682,127.0779) -- (604.3207,125.7359) --
      (602.3744,123.4276) -- (590.2447,120.7059) -- (587.3688,119.6939) --
      (579.1557,117.5266) -- (571.2378,116.3678) -- (567.4726,111.2372) --
      (566.7222,111.7912) -- (565.5243,111.6295) -- (564.8777,110.4979) --
      (563.5437,110.7944) -- (562.4120,110.9561) -- (560.6338,111.9261) --
      (559.6638,111.2794) -- (560.3105,109.3395) -- (562.2504,106.2679) --
      (563.3820,105.1363) -- (561.4421,103.6814) -- (559.3405,104.4897) --
      (556.4306,106.4296) -- (548.9942,109.6628) -- (546.0843,110.3094) --
      (543.1745,109.8245) -- (542.1927,108.9462) -- (540.0760,111.7814) --
      (539.8474,114.5249) -- (539.8474,122.9839) -- (538.7043,124.5843) --
      (533.4460,128.4708) -- (531.1597,134.4150) -- (531.6170,134.6437) --
      (534.1318,136.7013) -- (534.8177,139.9020) -- (532.9887,143.1027) --
      (532.9887,146.9893) -- (533.4460,153.6193) -- (536.4181,156.5914) --
      (539.8474,156.5914) -- (541.6764,159.7922) -- (545.1057,160.2494) --
      (548.9923,165.9650) -- (556.0796,170.0802) -- (558.1372,172.8236) --
      (559.0517,180.2539) -- (559.7376,183.5689) -- (562.0238,185.1693) --
      (562.2524,186.5410) -- (560.1948,189.9703) -- (560.4234,193.1711) --
      (562.9383,197.0576) -- (565.4531,198.2007) -- (568.4252,198.6580) --
      (569.7676,200.0381) -- (615.0659,197.3687) -- cycle;

    % NC
    \path[USA map/state, USA map/NC, local bounding box=NC] (834.9815,294.3155) -- (837.0665,299.2329) -- (840.6231,305.6993) --
      (843.0480,308.1242) -- (843.6946,310.3875) -- (841.2697,310.5491) --
      (842.0780,311.1958) -- (841.7547,315.3989) -- (839.1681,316.6922) --
      (838.5215,318.7938) -- (837.2282,321.7037) -- (833.5100,323.3203) --
      (831.0851,322.9970) -- (829.6301,322.8353) -- (828.0135,321.5420) --
      (828.3369,322.8353) -- (828.3369,323.8053) -- (830.2768,323.8053) --
      (831.0851,325.0986) -- (829.1452,331.4033) -- (833.3483,331.4033) --
      (833.9950,333.0199) -- (836.2582,330.7567) -- (837.5515,330.2717) --
      (835.6116,333.8282) -- (832.5400,338.6781) -- (831.2468,338.6781) --
      (830.1151,338.1931) -- (827.3669,338.8397) -- (822.1938,341.2646) --
      (815.7273,346.5994) -- (812.3325,351.2876) -- (810.3926,357.7540) --
      (809.9076,360.1789) -- (805.2194,360.6639) -- (799.7663,362.0005) --
      (789.8199,353.7980) -- (777.2103,346.2000) -- (774.3004,345.3916) --
      (761.6909,346.8466) -- (757.4145,347.5967) -- (755.7979,344.3635) --
      (752.8275,342.2468) -- (736.3381,342.7318) -- (729.0634,343.5401) --
      (720.0104,348.0666) -- (713.8673,350.6532) -- (692.6897,353.2398) --
      (693.1898,349.1854) -- (694.9681,347.7305) -- (697.7163,347.0838) --
      (698.3630,343.3656) -- (702.5661,340.6174) -- (706.4460,339.1624) --
      (710.6492,335.6059) -- (715.0140,333.5043) -- (715.6606,330.4328) --
      (719.5405,326.5529) -- (720.1871,326.3913) .. controls (720.1871,326.3913) and
      (720.1871,327.5229) .. (720.9955,327.5229) .. controls (721.8038,327.5229) and
      (722.9354,327.8462) .. (722.9354,327.8462) -- (725.1986,324.2897) --
      (727.3002,323.6430) -- (729.5635,323.9664) -- (731.1801,320.4098) --
      (734.0900,317.8232) -- (734.5750,315.7217) -- (734.7625,312.0735) --
      (739.0390,312.0510) -- (746.2375,311.1952) -- (761.9948,308.9427) --
      (777.1308,306.8562) -- (798.7713,302.1368) -- (818.7546,297.8782) --
      (829.9316,295.4724) -- (834.9815,294.3156) -- cycle(839.2520,327.5221) --
      (841.8386,325.0164) -- (844.9909,322.4298) -- (846.5267,321.7831) --
      (846.6884,319.7624) -- (846.0417,313.6193) -- (844.5868,311.2752) --
      (843.9401,309.4161) -- (844.6676,309.1736) -- (847.4159,314.6701) --
      (847.8200,319.1157) -- (847.6584,322.5106) -- (844.2635,324.0464) --
      (841.4344,326.4713) -- (840.3028,327.6838) -- (839.2520,327.5221) -- cycle;

    % DC
    \path[USA map/state, USA map/DC, local bounding box=DC] (805.8194,250.8438) -- (803.9612,249.0197) -- (802.7285,248.3334) --
      (804.1715,246.3109) -- (807.0606,248.2594) -- (805.8194,250.8438) -- cycle;

    % MA
    \path[USA map/state, USA map/MA, local bounding box=MA] (899.6235,173.2539) -- (901.7954,172.5681) -- (902.2527,170.8534) --
      (903.2815,170.9677) -- (904.3103,173.2539) -- (903.0529,173.7112) --
      (899.1662,173.8255) -- (899.6235,173.2539) -- cycle(890.2499,174.0541) --
      (892.5362,171.4250) -- (894.1365,171.4250) -- (895.9655,172.9110) --
      (893.5650,173.9398) -- (891.3931,174.9686) -- (890.2499,174.0541) --
      cycle(855.4508,152.0659) -- (873.0977,147.4253) -- (875.3609,146.7786) --
      (877.2750,143.9829) -- (881.0118,142.3196) -- (883.9010,146.7324) --
      (881.4761,151.9056) -- (881.1528,153.3605) -- (883.0927,155.9471) --
      (884.2244,155.1388) -- (886.0026,155.1388) -- (888.2659,157.7253) --
      (892.1457,163.7068) -- (895.7023,164.1918) -- (897.9655,163.2218) --
      (899.7438,161.4435) -- (898.9355,158.6953) -- (896.8339,157.0787) --
      (895.3789,157.8870) -- (894.4090,156.5937) -- (894.8939,156.1087) --
      (896.9955,155.9471) -- (898.7738,156.7554) -- (900.7137,159.1803) --
      (901.6837,162.0902) -- (902.0070,164.5151) -- (897.8038,165.9700) --
      (893.9240,167.9099) -- (890.0441,172.4364) -- (888.1042,173.8914) --
      (888.1042,172.9214) -- (890.5291,171.4665) -- (891.0141,169.6882) --
      (890.2058,166.6167) -- (887.2959,168.0716) -- (886.4876,169.5266) --
      (886.9726,171.7898) -- (884.9063,172.7902) -- (882.1591,168.2631) --
      (878.7642,163.8983) -- (876.6937,162.0858) -- (870.1604,163.9620) --
      (865.0681,165.0128) -- (844.3929,169.6050) -- (843.7252,164.8371) --
      (844.3718,154.2484) -- (848.6611,153.3592) -- (855.4508,152.0659) -- cycle;

    % TN
    \path[USA map/state, USA map/TN, local bounding box=TN] (696.6779,318.2541) -- (644.7848,323.2656) -- (629.0252,325.0439) --
      (624.4040,325.5566) -- (620.5357,325.5289) -- (620.3147,329.6297) --
      (612.1293,329.8937) -- (605.1779,330.5403) -- (597.0871,330.4165) --
      (595.6733,337.4894) -- (593.9771,342.9694) -- (590.6839,345.7202) --
      (589.3352,350.1013) -- (589.0118,352.6879) -- (584.9703,354.9511) --
      (586.4253,358.5076) -- (585.4553,362.8725) -- (584.4869,363.6621) --
      (692.6455,353.2546) -- (693.0487,349.2996) -- (694.8595,347.8093) --
      (697.6936,347.0598) -- (698.3656,343.3428) -- (702.4642,340.6379) --
      (706.5111,339.1438) -- (710.5947,335.5735) -- (715.0308,333.5480) --
      (715.5520,330.4807) -- (719.6166,326.4957) -- (720.1674,326.3815) .. controls
      (720.1674,326.3815) and (720.1986,327.5132) .. (721.0070,327.5132) .. controls
      (721.8153,327.5132) and (722.9469,327.8677) .. (722.9469,327.8677) --
      (725.2101,324.2799) -- (727.2805,323.6333) -- (729.5556,323.9285) --
      (731.1539,320.3956) -- (734.1092,317.7517) -- (734.5308,315.8126) --
      (734.8398,312.1015) -- (732.6932,311.9017) -- (730.0916,313.9300) --
      (723.0983,313.9591) -- (704.7390,316.3460) -- (696.6779,318.2542) -- cycle;

    % AR
    \path[USA map/state, USA map/AR, local bounding box=AR] (593.8248,343.0530) -- (589.8449,343.7697) -- (584.7327,343.1356) --
      (585.1534,341.5336) -- (588.1332,338.9669) -- (589.0766,335.3106) --
      (587.2476,332.3385) -- (508.8300,334.8534) -- (510.4304,341.7121) --
      (510.4304,349.9425) -- (511.8021,360.9165) -- (512.0307,398.7534) --
      (514.3170,400.6967) -- (517.2891,399.3250) -- (520.0325,400.4681) --
      (520.7129,407.0414) -- (576.3341,405.9008) -- (577.4798,403.8104) --
      (577.1932,400.2609) -- (575.3675,397.2888) -- (576.9662,395.8036) --
      (575.3675,393.2921) -- (576.0517,390.7822) -- (577.4201,385.1768) --
      (579.9383,383.1142) -- (579.2524,380.8296) -- (582.9104,375.4578) --
      (585.6539,374.0894) -- (585.5404,372.5959) -- (585.1949,370.7702) --
      (588.0519,365.1715) -- (590.4549,363.9149) -- (590.8391,360.4873) --
      (592.6097,359.2456) -- (589.4662,358.7613) -- (588.1248,354.7509) --
      (590.9288,352.3742) -- (591.4791,350.3550) -- (592.7586,346.3083) --
      (593.8248,343.0530) -- cycle;

    % MO
    \path[USA map/state, USA map/MO, local bounding box=MO] (558.4402,248.1132) -- (555.9203,245.0259) -- (554.7772,242.7397) --
      (490.4200,245.1402) -- (488.1337,245.2545) -- (489.3912,247.7694) --
      (489.1626,250.0556) -- (491.6774,253.9422) -- (494.7638,258.0574) --
      (497.8502,260.8009) -- (500.0114,261.0295) -- (501.5082,261.9440) --
      (501.5082,264.9161) -- (499.6792,266.5164) -- (499.2219,268.8027) --
      (501.2795,272.2320) -- (503.7944,275.2041) -- (506.3092,277.0331) --
      (507.6810,288.6928) -- (507.9951,324.7650) -- (508.2237,329.4518) --
      (508.6810,334.8353) -- (531.1140,333.9685) -- (554.3200,333.2826) --
      (575.1246,332.4816) -- (586.7794,332.2513) -- (588.9488,335.6773) --
      (588.2646,338.9848) -- (585.1773,341.3878) -- (584.6050,343.2252) --
      (589.9834,343.6824) -- (593.8784,342.9966) -- (595.5956,337.5029) --
      (596.2470,331.6461) -- (598.3450,329.0910) -- (600.9411,327.6041) --
      (600.9925,324.5538) -- (602.0085,322.6174) -- (600.3143,320.0736) --
      (598.9833,321.0579) -- (596.9907,318.8306) -- (595.7057,314.0716) --
      (596.5067,311.5534) -- (594.5626,308.1258) -- (592.7319,303.5500) --
      (587.9325,302.7506) -- (580.9637,297.1519) -- (579.2449,293.0383) --
      (580.0442,289.8376) -- (582.1035,283.7799) -- (582.5624,280.9163) --
      (580.6133,279.8850) -- (573.7579,279.0873) -- (572.7299,277.3752) --
      (572.6181,273.1448) -- (567.1312,269.7138) -- (560.1557,261.9423) --
      (557.8695,254.6264) -- (557.6392,250.4011) -- (558.4402,248.1132) -- cycle;

    % GA
    \path[USA map/state, USA map/GA, local bounding box=GA] (672.2923,355.5518) -- (672.2923,357.7342) -- (672.4539,359.8358) --
      (673.1006,363.2307) -- (676.4955,371.1521) -- (678.9204,381.0134) --
      (680.3753,387.1565) -- (681.9919,392.0063) -- (683.4469,398.9577) --
      (685.5485,405.2625) -- (688.1350,408.6574) -- (688.6200,412.0522) --
      (690.5599,412.8605) -- (690.7216,414.9621) -- (688.9433,419.8119) --
      (688.4584,423.0452) -- (688.2967,424.9851) -- (689.9133,429.3499) --
      (690.2366,434.6847) -- (689.4283,437.1096) -- (690.0750,437.9179) --
      (691.5299,438.7262) -- (691.7346,441.9443) -- (693.9676,445.2939) --
      (696.2181,447.4559) -- (704.1395,447.6176) -- (714.9592,446.9709) --
      (736.4716,445.6777) -- (741.9173,445.0033) -- (746.4946,445.0310) --
      (746.6562,447.9409) -- (749.2428,448.7492) -- (749.5661,444.3843) --
      (747.9495,439.8578) -- (749.0811,438.2412) -- (754.9009,439.0495) --
      (759.8783,439.3673) -- (759.1029,433.0685) -- (761.3661,423.0456) --
      (762.8211,418.8424) -- (762.3361,416.2558) -- (765.6705,410.0115) --
      (765.1602,408.6599) -- (763.2468,409.3644) -- (760.6602,408.0711) --
      (760.0136,405.9695) -- (758.7203,402.4130) -- (756.4571,400.3114) --
      (753.8705,399.6648) -- (752.2539,394.8150) -- (749.3289,388.4800) --
      (745.1257,386.5400) -- (743.0241,384.6001) -- (741.7308,382.0135) --
      (739.6292,380.0736) -- (737.3660,378.7803) -- (735.1027,375.8704) --
      (732.0312,373.6072) -- (727.5047,371.8289) -- (727.0197,370.3740) --
      (724.5948,367.4641) -- (724.1098,366.0091) -- (720.7149,361.0386) --
      (717.1951,361.1378) -- (713.4401,358.7817) -- (712.0219,357.4884) --
      (711.6985,355.7102) -- (712.5693,353.7702) -- (714.7960,352.6601) --
      (714.1620,350.5629) -- (672.2923,355.5518) -- cycle;

    % SC
    \path[USA map/state, USA map/SC, local bounding box=SC] (764.9433,408.1649) -- (763.1662,409.1344) -- (760.5796,407.8411) --
      (759.9330,405.7395) -- (758.6397,402.1830) -- (756.3765,400.0814) --
      (753.7899,399.4347) -- (752.1733,394.5849) -- (749.4251,388.6035) --
      (745.2219,386.6635) -- (743.1203,384.7236) -- (741.8270,382.1370) --
      (739.7254,380.1971) -- (737.4622,378.9038) -- (735.1989,375.9939) --
      (732.1274,373.7307) -- (727.6009,371.9524) -- (727.1159,370.4975) --
      (724.6910,367.5876) -- (724.2060,366.1326) -- (720.8111,360.9595) --
      (717.4162,361.1211) -- (713.3747,358.6962) -- (712.0814,357.4029) --
      (711.7581,355.6247) -- (712.5664,353.6848) -- (714.8297,352.7148) --
      (714.3189,350.4257) -- (720.0870,348.0891) -- (729.2025,343.5001) --
      (736.9772,342.6918) -- (753.0916,342.2693) -- (755.7298,344.1468) --
      (757.4089,347.5050) -- (761.7113,346.8950) -- (774.3208,345.4400) --
      (777.2307,346.2484) -- (789.8402,353.8464) -- (799.9483,361.9681) --
      (794.5272,367.4264) -- (791.9406,373.5695) -- (791.4556,379.8743) --
      (789.8390,380.6826) -- (788.7074,383.4308) -- (786.2825,384.0775) --
      (784.1809,387.6340) -- (781.4327,390.3822) -- (779.1694,393.7771) --
      (777.5528,394.5854) -- (773.9963,397.9803) -- (771.0864,398.1419) --
      (772.0564,401.3751) -- (767.0449,406.8716) -- (764.9433,408.1649) -- cycle;

    % KY
    \path[USA map/state, USA map/KY, local bounding box=KY] (725.9944,295.2707) -- (723.7011,297.6724) -- (720.1229,301.6664) --
      (715.1983,307.1311) -- (713.9826,308.8469) -- (713.9201,310.9484) --
      (709.5402,313.1125) -- (703.8821,316.5074) -- (696.6502,318.3063) --
      (644.7823,323.2051) -- (629.0228,324.9834) -- (624.4016,325.4961) --
      (620.5332,325.4684) -- (620.3063,329.6887) -- (612.1269,329.8332) --
      (605.1755,330.4799) -- (597.1880,330.4197) -- (598.3958,329.0996) --
      (600.8953,327.5587) -- (601.1239,324.3580) -- (602.0384,322.5290) --
      (600.4316,319.9901) -- (601.2334,318.0833) -- (603.4967,316.3051) --
      (605.5983,315.6584) -- (608.3465,316.9517) -- (611.9030,318.2450) --
      (613.0347,317.9217) -- (613.1963,315.6584) -- (611.9030,313.2335) --
      (612.2264,310.9702) -- (614.1663,309.5153) -- (616.7529,308.8687) --
      (618.3695,308.2220) -- (617.5612,306.4437) -- (616.9145,304.5038) --
      (618.4211,303.5080) .. controls (618.4241,303.4709) and (619.6751,299.9857) ..
      (619.6594,299.8502) -- (622.7127,298.3715) -- (628.0324,297.4016) --
      (632.5265,296.9166) -- (633.9189,298.5440) -- (635.4472,299.4148) --
      (637.0380,296.3066) -- (640.2250,295.0240) -- (642.4301,296.5080) --
      (642.8407,297.5071) -- (644.0142,297.2430) -- (643.8525,294.2901) --
      (646.9834,292.5409) -- (649.1315,291.4674) -- (650.6609,293.1283) --
      (653.9790,293.0841) -- (654.5663,291.5128) -- (654.1988,289.2496) --
      (656.7994,285.2511) -- (661.5759,281.8132) -- (662.2819,276.9773) --
      (665.2069,276.5214) -- (668.9983,274.8757) -- (671.4417,273.1675) --
      (671.2433,271.6025) -- (670.1009,270.1476) -- (670.6667,267.1527) --
      (674.8516,267.0352) -- (677.1515,266.2894) -- (680.4989,267.7185) --
      (682.5530,272.0833) -- (687.6853,272.0941) -- (689.7363,274.3023) --
      (691.3517,274.1546) -- (693.9534,272.8765) -- (699.1905,273.4498) --
      (701.7654,273.6673) -- (703.4530,271.6111) -- (706.0709,270.1852) --
      (707.9527,269.4781) -- (708.5993,272.3147) -- (710.6428,273.3731) --
      (713.2855,275.4556) -- (713.4030,281.1288) -- (714.2113,282.7012) --
      (716.8010,284.2575) -- (717.5727,286.5520) -- (721.7325,289.9890) --
      (723.5379,293.6122) -- (725.9944,295.2707) -- cycle;

    % AL
    \path[USA map/state, USA map/AL, local bounding box=AL] (631.3065,460.4157) -- (629.8159,446.0942) -- (627.0676,427.3416) --
      (627.2293,413.2771) -- (628.0376,382.2382) -- (627.8759,365.5872) --
      (628.0410,359.1681) -- (672.5255,355.5487) -- (672.3777,357.7311) --
      (672.5394,359.8327) -- (673.1860,363.2276) -- (676.5809,371.1489) --
      (679.0058,381.0102) -- (680.4607,387.1534) -- (682.0773,392.0032) --
      (683.5323,398.9546) -- (685.6339,405.2593) -- (688.2205,408.6542) --
      (688.7054,412.0491) -- (690.6454,412.8574) -- (690.8070,414.9590) --
      (689.0287,419.8088) -- (688.5438,423.0420) -- (688.3821,424.9820) --
      (689.9987,429.3468) -- (690.3220,434.6816) -- (689.5137,437.1065) --
      (690.1604,437.9148) -- (691.6153,438.7231) -- (691.9435,441.6119) --
      (686.3458,441.2584) -- (679.5561,441.9050) -- (654.0137,444.8149) --
      (643.6021,446.2217) -- (643.3807,449.0991) -- (645.1590,450.8774) --
      (647.7456,452.8173) -- (648.3264,460.7527) -- (642.7844,463.3256) --
      (640.0361,463.0023) -- (642.7844,461.0624) -- (642.7844,460.0924) --
      (639.7128,454.1110) -- (637.4496,453.4643) -- (635.9946,457.8291) --
      (634.7013,460.5774) -- (634.0547,460.4157) -- (631.3065,460.4157) -- cycle;

    % LA
    \path[USA map/state, USA map/LA, local bounding box=LA] (607.9671,459.1612) -- (604.6824,455.9951) -- (605.6924,450.4949) --
      (605.0310,449.6018) -- (595.7693,450.6084) -- (570.7410,451.0673) --
      (570.0568,448.6726) -- (570.9696,440.2169) -- (574.2855,434.2711) --
      (579.3169,425.5800) -- (578.7428,423.1820) -- (579.9994,422.5012) --
      (580.4583,420.5487) -- (578.1721,418.4927) -- (578.0603,416.5503) --
      (576.2296,412.2048) -- (576.0826,405.8662) -- (520.6088,406.7902) --
      (520.6374,416.3637) -- (521.3233,425.7373) -- (522.0091,429.6238) --
      (524.5240,433.7390) -- (525.4385,438.7688) -- (529.7823,444.2557) --
      (530.0109,447.4564) -- (530.6968,448.1423) -- (530.0109,456.6013) --
      (527.0388,461.6310) -- (528.6392,463.6886) -- (527.9533,466.2035) --
      (527.2675,473.5194) -- (525.8957,476.7201) -- (526.0182,480.3365) --
      (530.7047,478.8164) -- (542.8180,479.0234) -- (553.1643,482.5799) --
      (559.6307,483.7116) -- (563.3489,482.2566) -- (566.5821,483.3882) --
      (569.8153,484.3582) -- (570.6236,482.2566) -- (567.3904,481.1250) --
      (564.8038,481.6100) -- (562.0556,479.9934) .. controls (562.0556,479.9934) and
      (562.2173,478.7001) .. (562.8639,478.5384) .. controls (563.5105,478.3768) and
      (565.9355,477.5685) .. (565.9355,477.5685) -- (567.7137,479.0234) --
      (569.4920,478.0534) -- (572.7252,478.7001) -- (574.1801,481.1250) --
      (574.5035,483.3882) -- (579.0299,483.7116) -- (580.8082,485.4898) --
      (579.9999,487.1064) -- (578.7066,487.9147) -- (580.3232,489.5313) --
      (588.7296,493.0879) -- (592.2861,491.7946) -- (593.2561,489.3697) --
      (595.8426,488.7230) -- (597.6209,487.2681) -- (598.9142,488.2381) --
      (599.7225,491.1479) -- (597.4592,491.9562) -- (598.1059,492.6029) --
      (601.5008,491.3096) -- (603.7640,487.9147) -- (604.5723,487.4298) --
      (602.4707,487.1064) -- (603.2790,485.4898) -- (603.1174,484.0349) --
      (605.2189,483.5499) -- (606.3506,482.2566) -- (606.9972,483.0649) .. controls
      (606.9972,483.0649) and (606.8355,486.1365) .. (607.6439,486.1365) .. controls
      (608.4522,486.1365) and (611.8470,486.7831) .. (611.8470,486.7831) --
      (615.8885,488.7230) -- (616.8585,490.1780) -- (619.7684,490.1780) --
      (620.9000,491.1479) -- (623.1633,488.0764) -- (623.1633,486.6214) --
      (621.8700,486.6214) -- (618.4751,483.8732) -- (612.6553,483.0649) --
      (609.4221,480.8017) -- (610.5537,478.0534) -- (612.8170,478.3768) --
      (612.9786,477.7301) -- (611.2004,476.7602) -- (611.2004,476.2752) --
      (614.4336,476.2752) -- (616.2119,473.2036) -- (614.9186,471.2637) --
      (614.5953,468.5155) -- (613.1403,468.6771) -- (611.2004,470.7787) --
      (610.5537,473.3653) -- (607.4822,472.7186) -- (606.5122,470.9404) --
      (608.2905,469.0005) -- (610.1938,465.5548) -- (609.1327,463.1426) --
      (607.9671,459.1612) -- cycle;

    % MS
    \path[USA map/state, USA map/MS, local bounding box=MS] (631.5588,459.3446) -- (631.3046,460.6007) -- (626.1314,460.6007) --
      (624.6765,459.7924) -- (622.5749,459.4691) -- (615.7851,461.4090) --
      (614.0069,460.6007) -- (611.4203,464.8039) -- (610.3178,465.5819) --
      (609.1939,463.0939) -- (608.0508,459.2074) -- (604.6215,456.0066) --
      (605.7646,450.4621) -- (605.0787,449.5476) -- (603.2498,449.7762) --
      (595.3318,450.6496) -- (570.7853,451.0230) -- (570.0156,448.7976) --
      (570.8890,440.4208) -- (574.0058,434.7480) -- (579.2329,425.6031) --
      (578.7871,423.1705) -- (580.0240,422.5142) -- (580.4599,420.5948) --
      (578.1424,418.5158) -- (578.0273,416.3743) -- (576.1915,412.2532) --
      (576.0825,406.2905) -- (577.4101,403.8095) -- (577.1868,400.3937) --
      (575.4173,397.3111) -- (576.9437,395.8289) -- (575.3731,393.3294) --
      (575.8303,391.6772) -- (577.4077,385.1508) -- (579.8937,383.1145) --
      (579.2520,380.7475) -- (582.9100,375.4450) -- (585.7419,374.0885) --
      (585.5209,372.4134) -- (585.2328,370.7323) -- (588.1088,365.1646) --
      (590.4545,363.9331) -- (590.6062,363.0401) -- (627.9496,359.1589) --
      (628.1345,365.4422) -- (628.2962,382.0933) -- (627.4879,413.1322) --
      (627.3262,427.1966) -- (630.0744,445.9493) -- (631.5588,459.3446) -- cycle;

    % IA
    \path[USA map/state, USA map/IA, local bounding box=IA] (569.1915,199.5843) -- (569.4559,202.3705) -- (571.6796,202.9478) --
      (572.6336,204.1731) -- (573.1336,206.0285) -- (576.9264,209.3871) --
      (577.6123,211.7786) -- (576.9380,215.2031) -- (575.3556,218.4351) --
      (574.5563,221.1768) -- (572.3836,222.7789) -- (570.6680,223.3513) --
      (565.0890,225.2115) -- (563.6976,229.0602) -- (564.4262,230.4319) --
      (566.2667,232.1145) -- (565.9838,236.1508) -- (564.2206,237.6887) --
      (563.4492,239.3318) -- (563.5764,242.1081) -- (561.6901,242.5654) --
      (560.0647,243.6703) -- (559.7859,245.0229) -- (560.0647,247.1378) --
      (558.5137,248.2539) -- (556.0431,245.1206) -- (554.7806,242.6707) --
      (489.0447,245.1856) -- (488.1267,245.3510) -- (486.0743,240.8351) --
      (485.8457,234.2050) -- (484.2453,230.0898) -- (483.5595,224.8315) --
      (481.2732,221.1735) -- (480.3588,216.3724) -- (477.6153,208.8279) --
      (476.4722,203.4552) -- (475.1004,201.2833) -- (473.5001,198.5399) --
      (475.4541,193.6960) -- (476.8258,187.9805) -- (474.0823,185.9229) --
      (473.6251,183.1794) -- (474.5396,180.6645) -- (476.2542,180.6645) --
      (558.9082,179.3951) -- (559.7425,183.5782) -- (561.9947,185.1392) --
      (562.2514,186.5622) -- (560.2219,189.9516) -- (560.4123,193.1571) --
      (562.9271,196.9553) -- (565.4539,198.2489) -- (568.5332,198.7519) --
      (569.1915,199.5843) -- cycle;

    % MN
    \path[USA map/state, USA map/MN, local bounding box=MN] (475.2378,128.8244) -- (474.7806,120.3653) -- (472.9516,113.0494) --
      (471.1226,99.5607) -- (470.6654,89.7299) -- (468.8364,86.3006) --
      (467.2360,81.2709) -- (467.2360,70.9829) -- (467.9219,67.0963) --
      (466.1009,61.6446) -- (496.2334,61.6799) -- (496.5567,53.4352) --
      (497.2033,53.2735) -- (499.4666,53.7585) -- (501.4065,54.5668) --
      (502.2148,60.0633) -- (503.6697,66.2064) -- (505.2863,67.8230) --
      (510.1362,67.8230) -- (510.4595,69.2779) -- (516.7642,69.6012) --
      (516.7642,71.7028) -- (521.6141,71.7028) -- (521.9374,70.4095) --
      (523.0690,69.2779) -- (525.3322,68.6313) -- (526.6255,69.6012) --
      (529.5354,69.6012) -- (533.4153,72.1878) -- (538.7501,74.6127) --
      (541.1750,75.0977) -- (541.6599,74.1277) -- (543.1149,73.6428) --
      (543.5999,76.5526) -- (546.1864,77.8459) -- (546.6714,77.3609) --
      (547.9647,77.5226) -- (547.9647,79.6242) -- (550.5513,80.5942) --
      (553.6228,80.5942) -- (555.2394,79.7859) -- (558.4726,76.5526) --
      (561.0592,76.0677) -- (561.8675,77.8459) -- (562.3525,79.1392) --
      (563.3224,79.1392) -- (564.2924,78.3309) -- (573.1837,78.0076) --
      (574.9620,81.0791) -- (575.6086,81.0791) -- (576.3223,79.9949) --
      (580.7622,79.6242) -- (580.1501,81.9037) -- (576.2113,83.7408) --
      (566.9656,87.8019) -- (562.1908,89.8088) -- (559.1193,92.3954) --
      (556.6944,95.9519) -- (554.4311,99.8318) -- (552.6529,100.6401) --
      (548.1264,105.6515) -- (546.8331,105.8132) -- (542.5053,108.5703) --
      (540.0424,111.7754) -- (539.8138,114.9668) -- (539.9082,123.0102) --
      (538.5322,124.6989) -- (533.4506,128.4589) -- (531.2205,134.4413) --
      (534.0923,136.6750) -- (534.7722,139.9020) -- (532.9169,143.1409) --
      (533.0877,146.8889) -- (533.4566,153.6193) -- (536.4848,156.6213) --
      (539.8138,156.6213) -- (541.7050,159.7539) -- (545.0841,160.2572) --
      (548.9433,165.9287) -- (556.0306,170.0454) -- (558.1737,172.9205) --
      (558.8449,179.3600) -- (477.6334,180.5048) -- (477.2955,144.8280) --
      (476.8382,141.8559) -- (472.7230,138.4265) -- (471.5799,136.5976) --
      (471.5799,134.9972) -- (473.6375,133.3968) -- (475.0092,132.0251) --
      (475.2379,128.8244) -- cycle;

    % OK
    \path[USA map/state, USA map/OK, local bounding box=OK] (380.3431,320.8215) -- (363.6589,319.5482) -- (362.7787,330.5006) --
      (383.2441,331.6575) -- (415.2997,332.9611) -- (412.9651,357.3797) --
      (412.5078,375.2123) -- (412.7364,376.8126) -- (417.0803,380.4706) --
      (419.1379,381.6137) -- (419.8237,381.3851) -- (420.5096,379.3275) --
      (421.8813,381.1565) -- (423.9389,381.1565) -- (423.9389,379.7847) --
      (426.6824,381.1565) -- (426.2252,385.0430) -- (430.3404,385.2717) --
      (432.8552,386.4148) -- (436.9704,387.1007) -- (439.4853,388.9296) --
      (441.7715,386.8720) -- (445.2009,387.5579) -- (447.7157,390.9872) --
      (448.6302,390.9872) -- (448.6302,393.2735) -- (450.9164,393.9593) --
      (453.2026,391.6731) -- (455.0316,392.3590) -- (457.5465,392.3590) --
      (458.4610,394.8738) -- (464.7620,396.9528) -- (466.1338,396.2669) --
      (467.9628,392.1517) -- (469.1059,392.1517) -- (470.2490,394.2093) --
      (474.3642,394.8952) -- (478.0221,396.2669) -- (480.9942,397.1814) --
      (482.8232,396.2669) -- (483.5091,393.7521) -- (487.8529,393.7521) --
      (489.9105,394.6666) -- (492.6540,392.6090) -- (493.7971,392.6090) --
      (494.4830,394.2093) -- (498.5982,394.2093) -- (500.1985,392.1517) --
      (502.0275,392.6090) -- (504.0851,395.1238) -- (507.2858,396.9528) --
      (510.4866,397.8673) -- (512.4277,398.9862) -- (512.0386,361.7692) --
      (510.6668,350.7952) -- (510.5063,341.9229) -- (509.0665,335.3852) --
      (508.2883,328.2055) -- (508.2202,324.3893) -- (496.0833,324.7081) --
      (449.6733,324.2508) -- (404.6344,322.1932) -- (380.3432,320.8215) -- cycle;

    % TX
    \path[USA map/state, USA map/TX, local bounding box=TX] (361.4642,330.5736) -- (384.1550,331.6595) -- (415.2477,332.8026) --
      (412.9131,356.2584) -- (412.6163,374.4120) -- (412.6844,376.4938) --
      (417.0283,380.3122) -- (419.0149,381.7593) -- (420.1991,381.1997) --
      (420.5725,379.3819) -- (421.7128,381.1856) -- (423.8245,381.2295) --
      (423.8215,379.7824) -- (425.4914,380.7496) -- (426.6301,381.1585) --
      (426.2708,385.1261) -- (430.3590,385.2197) -- (433.2843,386.4168) --
      (437.2391,386.9422) -- (439.6205,389.0212) -- (441.7446,386.9450) --
      (445.4695,387.5599) -- (447.6904,390.7849) -- (448.7654,391.1058) --
      (448.6049,393.0711) -- (450.8185,393.8634) -- (453.1487,391.8086) --
      (455.2817,392.4235) -- (457.5111,392.4590) -- (458.4441,394.8944) --
      (464.7722,397.0089) -- (466.3653,396.2420) -- (467.8547,392.0643) --
      (468.1955,392.0643) -- (469.1020,392.1458) -- (470.3310,394.2145) --
      (474.2609,394.8798) -- (477.5979,396.0026) -- (481.0235,397.1986) --
      (482.8641,396.2236) -- (483.5779,393.7088) -- (488.0311,393.7530) --
      (489.8398,394.6837) -- (492.6391,392.5772) -- (493.7427,392.6214) --
      (494.5937,394.2265) -- (498.6485,394.2265) -- (500.1673,392.1979) --
      (502.0347,392.6051) -- (503.9807,395.0084) -- (507.5013,397.0526) --
      (510.3601,397.8624) -- (511.8737,398.6622) -- (514.3204,400.6595) --
      (517.3634,399.3317) -- (520.0545,400.4706) -- (520.6183,406.5766) --
      (520.5785,416.2787) -- (521.2644,425.8127) -- (521.9666,429.4179) --
      (524.6419,433.8377) -- (525.5401,438.7884) -- (529.7560,444.3265) --
      (529.9520,447.4714) -- (530.6984,448.2572) -- (529.9683,456.6373) --
      (527.0962,461.6438) -- (528.6292,463.7967) -- (527.9991,466.1348) --
      (527.3296,473.5391) -- (525.8252,476.8771) -- (526.1201,480.3794) --
      (520.4552,481.9646) -- (510.5940,486.4911) -- (509.6240,488.4310) --
      (507.0374,490.3710) -- (504.9358,491.8259) -- (503.6425,492.6342) --
      (497.9844,497.9690) -- (495.2362,500.0706) -- (489.9014,503.3038) --
      (484.2433,505.7287) -- (477.9385,509.1236) -- (476.1603,510.5785) --
      (470.3405,514.1350) -- (466.9456,514.7817) -- (463.0658,520.2781) --
      (459.0243,520.6015) -- (458.0543,522.5414) -- (460.3176,524.4813) --
      (458.8626,529.9778) -- (457.5693,534.5043) -- (456.4377,538.3841) --
      (455.6294,542.9106) -- (456.4377,545.3355) -- (458.2160,552.2869) --
      (459.1859,558.4300) -- (460.9642,561.1782) -- (459.9942,562.6332) --
      (456.9227,564.5731) -- (451.2646,560.6933) -- (445.7681,559.5616) --
      (444.4748,560.0466) -- (441.2416,559.4000) -- (437.0384,556.3284) --
      (431.8653,555.1968) -- (424.2673,551.8019) -- (422.1657,547.9221) --
      (420.8724,541.4557) -- (417.6392,539.5157) -- (416.9925,537.2525) --
      (417.6392,536.6059) -- (417.9625,533.2110) -- (416.6692,532.5643) --
      (416.0226,531.5944) -- (417.3159,527.2295) -- (415.6993,524.9663) --
      (412.4660,523.6730) -- (409.0712,519.3082) -- (405.5146,512.6801) --
      (401.3115,510.0935) -- (401.4731,508.1536) -- (396.1383,495.8674) --
      (395.3300,491.6642) -- (393.5518,489.7243) -- (393.3901,488.2694) --
      (387.4087,482.9346) -- (384.8221,479.8630) -- (384.8221,478.7314) --
      (382.2355,476.6298) -- (375.4458,475.4982) -- (368.0094,474.8516) --
      (364.9379,472.5883) -- (360.4114,474.3666) -- (356.8548,475.8215) --
      (354.5916,479.0547) -- (353.6216,482.7729) -- (349.2568,488.9160) --
      (346.8319,491.3409) -- (344.2453,490.3710) -- (342.4671,489.2393) --
      (340.5271,488.5927) -- (336.6473,486.3295) -- (336.6473,485.6828) --
      (334.8690,483.7429) -- (329.6959,481.6413) -- (322.2595,473.8816) --
      (319.9963,469.1934) -- (319.9963,461.1104) -- (316.7631,454.6440) --
      (316.2781,451.8958) -- (314.6615,450.9258) -- (313.5298,448.8242) --
      (308.5184,446.7226) -- (307.2251,445.1060) -- (300.1120,437.1847) --
      (298.8187,433.9515) -- (294.1306,431.6882) -- (292.6756,427.3233) --
      (290.0890,424.4135) -- (288.1491,423.9285) -- (287.4999,419.2509) --
      (295.5018,419.9368) -- (324.5368,422.6802) -- (353.5718,424.2806) --
      (355.8054,404.8187) -- (359.6919,349.2637) -- (361.2923,330.5164) --
      (362.6641,330.5450)(461.6934,560.2077) -- (461.1276,553.0947) --
      (458.3794,545.9007) -- (457.8135,538.8685) -- (459.3493,530.6238) --
      (462.6634,523.7532) -- (466.1391,518.3375) -- (469.2915,514.7810) --
      (469.9381,515.0235) -- (465.1691,521.6516) -- (460.8043,528.1989) --
      (458.7835,534.8270) -- (458.4602,540.0001) -- (459.3493,546.1432) --
      (461.9359,553.3372) -- (462.4209,558.5103) -- (462.5825,559.9653) --
      (461.6934,560.2077) -- cycle;

    % NM
    \path[USA map/state, USA map/NM, local bounding box=NM] (288.1526,424.0131) -- (287.3771,419.2650) -- (296.0209,419.7904) --
      (326.1927,422.7363) -- (353.4608,424.4262) -- (355.6761,405.7188) --
      (359.5335,349.8428) -- (361.2711,330.4536) -- (362.8428,330.5821) --
      (363.6683,319.4187) -- (259.6638,308.7828) -- (242.1664,429.2176) --
      (257.6271,431.2067) -- (258.9204,421.1838) -- (288.1525,424.0131) -- cycle;

    % KS
    \path[USA map/state, USA map/KS, local bounding box=KS] (507.8806,324.3803) -- (495.2623,324.5847) -- (449.1732,324.1275) --
      (404.6158,322.0699) -- (379.9860,320.8124) -- (383.8798,256.2175) --
      (405.9633,256.8926) -- (446.2524,257.7340) -- (490.5536,258.7216) --
      (495.6493,258.7216) -- (497.8337,260.8840) -- (499.8513,260.8626) --
      (501.4916,261.8751) -- (501.4291,264.8843) -- (499.6001,266.6097) --
      (499.2679,268.8419) -- (501.1110,272.2442) -- (504.0633,275.4393) --
      (506.3907,277.0537) -- (507.6915,288.2945) -- (507.8806,324.3803) -- cycle;

    % NE
    \path[USA map/state, USA map/NE, local bounding box=NE] (486.0979,240.7006) -- (489.3285,247.7205) -- (489.1999,250.0230) --
      (492.6591,255.5169) -- (495.3784,258.6692) -- (490.3289,258.6692) --
      (446.8463,257.7305) -- (406.0595,256.8402) -- (383.8072,256.0564) --
      (384.8800,234.7285) -- (352.5618,231.8083) -- (356.9056,187.7984) --
      (372.4519,188.8272) -- (392.5707,189.9703) -- (410.4033,191.1134) --
      (434.1801,192.2566) -- (444.9253,191.7993) -- (446.9829,194.0855) --
      (451.7840,197.0576) -- (452.9271,197.9721) -- (457.2709,196.6004) --
      (461.1575,196.1431) -- (463.9010,195.9145) -- (465.7300,197.2863) --
      (469.7874,198.8866) -- (472.7595,200.4870) -- (473.2167,202.0873) --
      (474.1312,204.1449) -- (475.9602,204.1449) -- (476.7582,204.1911) --
      (477.6524,208.8730) -- (480.5727,217.3409) -- (481.1452,221.0976) --
      (483.6687,224.8718) -- (484.2383,229.9860) -- (485.8455,234.2263) --
      (486.0979,240.7006) -- cycle;

    % SD
    \path[USA map/state, USA map/SD, local bounding box=SD] (476.4469,204.0247) -- (476.3995,203.4438) -- (473.5038,198.5983) --
      (475.3640,193.8862) -- (476.8567,187.9997) -- (474.0748,185.9200) --
      (473.6897,183.1765) -- (474.4821,180.6222) -- (477.6706,180.6374) --
      (477.5475,175.6312) -- (477.2142,145.4570) -- (476.5965,141.6894) --
      (472.5242,138.3585) -- (471.5415,136.6815) -- (471.4790,135.0727) --
      (473.5012,133.5433) -- (475.0334,131.8776) -- (475.2783,129.2208) --
      (417.0212,127.6205) -- (362.2220,124.1714) -- (356.8968,187.8626) --
      (371.4870,188.7664) -- (391.4369,189.9720) -- (409.1799,190.9006) --
      (432.9567,192.2042) -- (444.9394,191.7795) -- (446.9057,194.0247) --
      (452.1003,197.2781) -- (452.8642,198.0008) -- (457.4057,196.5480) --
      (463.9462,195.9331) -- (465.6215,197.2694) -- (469.8260,198.8655) --
      (472.7711,200.5013) -- (473.1701,201.9851) -- (474.2096,204.2260) --
      (476.4469,204.0246) -- cycle;

    % ND
    \path[USA map/state, USA map/ND, local bounding box=ND] (475.3053,128.9185) -- (474.6904,120.4848) -- (473.0134,113.6689) --
      (471.1219,100.6446) -- (470.6647,89.6576) -- (468.9252,86.5805) --
      (467.1686,81.3861) -- (467.1998,70.9418) -- (467.8232,67.1177) --
      (465.9891,61.6500) -- (437.3468,61.0859) -- (418.7559,60.4393) --
      (392.2436,59.1460) -- (369.2972,57.0121) -- (362.3040,124.1890) --
      (417.2362,127.5326) -- (475.3052,128.9185) -- cycle;

    % WY
    \path[USA map/state, USA map/WY, local bounding box=WY] (360.3767,143.2759) -- (253.6341,129.8188) -- (239.5506,218.2768) --
      (352.8152,231.8623) -- (360.3767,143.2759) -- cycle;

    % MT
    \path[USA map/state, USA map/MT, local bounding box=MT] (369.2095,56.9691) -- (338.5352,54.1613) -- (309.2746,50.6048) --
      (280.0141,46.5633) -- (247.6820,41.2285) -- (229.2527,37.8336) --
      (196.5291,30.9009) -- (192.0500,52.2484) -- (195.4794,59.7929) --
      (194.1076,64.3654) -- (195.9366,68.9378) -- (199.1374,70.3096) --
      (203.7582,81.0790) -- (206.4533,84.2555) -- (206.9105,85.3987) --
      (210.3399,86.5418) -- (210.7971,88.5994) -- (203.7098,106.2033) --
      (203.7098,108.7182) -- (206.2247,111.9189) -- (207.1391,111.9189) --
      (211.9402,108.9468) -- (212.6261,107.8037) -- (214.2264,108.4896) --
      (213.9978,113.7479) -- (216.7413,126.3221) -- (219.7134,128.8370) --
      (220.6279,129.5228) -- (222.4569,131.8091) -- (221.9996,135.2384) --
      (222.6855,138.6677) -- (223.8286,139.5822) -- (226.1148,137.2960) --
      (228.8583,137.2960) -- (232.0590,138.8964) -- (234.5739,137.9819) --
      (238.6891,137.9819) -- (242.3470,139.5822) -- (245.0905,139.1250) --
      (245.5477,136.1529) -- (248.5198,135.4670) -- (249.8916,136.8388) --
      (250.3488,140.0395) -- (251.7747,140.8741) -- (253.6616,129.8394) --
      (360.4073,143.2683) -- (369.2095,56.9691) -- cycle;

    % CO
    \path[USA map/state, USA map/CO, local bounding box=CO] (380.0324,320.9646) -- (384.9357,234.6396) -- (271.5471,221.9956) --
      (259.3333,309.9348) -- (380.0324,320.9646) -- cycle;

    % ID
    \path[USA map/state, USA map/ID, local bounding box=ID] (148.4788,176.4839) -- (157.2497,141.2632) -- (158.6214,137.0337) --
      (161.1363,131.0895) -- (159.8788,128.8033) -- (157.3640,128.9176) --
      (156.5638,127.8888) -- (157.0211,126.7457) -- (157.3640,123.6593) --
      (161.8221,118.1723) -- (163.6511,117.7151) -- (164.7942,116.5720) --
      (165.3658,113.3713) -- (166.2803,112.6854) -- (170.1668,106.8555) --
      (174.0534,102.5117) -- (174.2821,98.7394) -- (170.8527,96.1103) --
      (169.3172,91.7093) -- (182.9421,28.3676) -- (196.4597,30.8957) --
      (192.0516,52.2787) -- (195.6119,59.7641) -- (194.0308,64.4249) --
      (196.0007,69.0661) -- (199.1389,70.3213) -- (202.9742,79.8779) --
      (206.4869,84.3151) -- (206.9942,85.4582) -- (210.3351,86.6013) --
      (210.7040,88.6984) -- (203.7330,106.0745) -- (203.5678,108.6404) --
      (206.1989,111.9621) -- (207.1040,111.9132) -- (212.0153,108.8876) --
      (212.6927,107.7926) -- (214.2550,108.4515) -- (213.9766,113.8052) --
      (216.7158,126.3879) -- (220.6337,129.5658) -- (222.3148,131.7313) --
      (221.5982,135.8151) -- (222.6644,138.6226) -- (223.7261,139.7138) --
      (226.2054,137.3624) -- (229.0535,137.4113) -- (231.9728,138.7465) --
      (234.7528,138.0646) -- (238.5471,137.9041) -- (242.5260,139.5045) --
      (245.2694,139.2077) -- (245.7662,136.1704) -- (248.6988,135.4056) --
      (249.9589,136.9215) -- (250.3999,139.8664) -- (251.8242,141.0797) --
      (243.4382,194.6883) .. controls (243.4382,194.6883) and (155.4722,177.9877) ..
      (148.4788,176.4840) -- cycle;

    % UT
    \path[USA map/state, USA map/UT, local bounding box=UT] (259.4984,310.1051) -- (175.7493,298.2328) -- (196.3369,185.6915) --
      (243.1173,194.4366) -- (241.6325,205.0670) -- (239.3208,218.2397) --
      (247.1285,219.1681) -- (263.5350,220.9729) -- (271.7460,221.8285) --
      (259.4984,310.1051) -- cycle;

    % AZ
    \path[USA map/state, USA map/AZ, local bounding box=AZ] (144.9112,382.6291) -- (142.2842,384.7874) -- (141.9609,386.2424) --
      (142.4459,387.2123) -- (161.3601,397.8819) -- (173.4847,405.4800) --
      (188.1958,414.0480) -- (205.0084,424.0709) -- (217.2946,426.4958) --
      (242.2458,429.2007) -- (259.5014,310.0737) -- (175.7658,298.1564) --
      (172.6734,314.5689) -- (171.0671,314.5842) -- (169.3524,317.2133) --
      (166.8376,317.0990) -- (165.5802,314.3556) -- (162.8367,314.0126) --
      (161.9222,312.8695) -- (161.0077,312.8695) -- (160.0932,313.4411) --
      (158.1499,314.4699) -- (158.0356,321.4429) -- (157.8070,323.1575) --
      (157.2354,335.7318) -- (155.7494,337.9037) -- (155.1778,341.2187) --
      (157.9213,346.1341) -- (159.1787,351.9640) -- (159.9789,352.9928) --
      (161.0077,353.5643) -- (160.8934,355.8505) -- (159.2930,357.2223) --
      (155.8637,358.9370) -- (153.9204,360.8803) -- (152.4344,364.5382) --
      (151.8628,369.4536) -- (149.0050,372.1971) -- (146.9474,372.8829) --
      (147.0831,373.7128) -- (146.6259,375.4275) -- (147.0831,376.2277) --
      (150.7411,376.7992) -- (150.1695,379.5427) -- (148.6835,381.7146) --
      (144.9112,382.6291) -- cycle;

    % NV
    \path[USA map/state, USA map/NV, local bounding box=NV] (196.3927,185.5755) -- (172.7538,314.3983) -- (170.9216,314.7474) --
      (169.3488,317.1536) -- (166.9759,317.1643) -- (165.5039,314.4208) --
      (162.8855,314.0424) -- (162.1145,312.9348) -- (161.0767,312.8808) --
      (158.2983,314.5251) -- (157.9881,321.3106) -- (157.6260,327.0877) --
      (157.2774,335.6805) -- (155.8303,337.7697) -- (153.3914,336.6957) --
      (84.3115,232.4945) -- (103.3006,164.9096) -- (196.3927,185.5756) -- cycle;

    % OR
    \path[USA map/state, USA map/OR, local bounding box=OR] (148.7218,175.5315) -- (157.5715,140.7300) -- (158.6223,136.5005) --
      (160.9767,130.8773) -- (160.3612,129.7144) -- (157.8463,129.6682) --
      (156.5647,127.9975) -- (157.0220,126.5334) -- (157.5254,123.2865) --
      (161.9835,117.7996) -- (163.8125,116.7004) -- (164.9556,115.5573) --
      (166.4417,111.9917) -- (170.4887,106.3223) -- (174.0543,102.4599) --
      (174.2830,99.0086) -- (171.0141,96.5399) -- (169.2307,91.8973) --
      (156.5669,88.2853) -- (141.4778,84.7417) -- (126.0458,84.8560) --
      (125.5886,83.4842) -- (120.1016,85.5418) -- (115.6435,84.9703) --
      (113.2430,83.3699) -- (111.9855,84.0558) -- (107.2988,83.8272) --
      (105.5841,82.4554) -- (100.3258,80.3978) -- (99.5256,80.5121) --
      (95.1818,79.0261) -- (93.2385,80.8551) -- (87.0657,80.5121) --
      (81.1215,76.3969) -- (81.8073,75.5968) -- (82.0360,67.8236) --
      (79.7497,63.9370) -- (75.6345,63.3654) -- (74.9487,60.8506) --
      (72.5947,60.3840) -- (66.7962,62.4428) -- (64.5330,68.9092) --
      (61.2998,78.9322) -- (58.0665,85.3986) -- (53.0551,99.4631) --
      (46.5887,113.0425) -- (38.5056,125.6521) -- (36.5657,128.5619) --
      (35.7574,137.1299) -- (36.1435,149.2102) -- (148.7218,175.5315) -- cycle;

    % WA
    \path[USA map/state, USA map/WA, local bounding box=WA] (102.0732,7.6118) -- (106.4381,9.0667) -- (116.1377,11.8149) --
      (124.7057,13.7549) -- (144.7516,19.4130) -- (167.7074,25.0711) --
      (182.9305,28.2783) -- (169.2981,91.8641) -- (156.8531,88.3388) --
      (141.3451,84.7681) -- (126.1158,84.8013) -- (125.6603,83.4566) --
      (120.0611,85.6359) -- (115.4656,84.8992) -- (113.3187,83.3151) --
      (112.0054,83.9731) -- (107.2698,83.8329) -- (105.5714,82.4832) --
      (100.3084,80.3709) -- (99.5734,80.5178) -- (95.1843,78.9934) --
      (93.2910,80.8108) -- (87.0251,80.5120) -- (81.0994,76.3863) --
      (81.8784,75.4536) -- (81.9996,67.7761) -- (79.7176,63.9364) --
      (75.6024,63.3294) -- (74.9250,60.8188) -- (72.6494,60.3618) --
      (69.0945,61.5924) -- (66.8313,58.3732) -- (67.1546,55.4633) --
      (69.9028,55.1400) -- (71.5194,51.0984) -- (68.9328,49.9668) --
      (69.0945,46.2486) -- (73.4593,45.6020) -- (70.7111,42.8538) --
      (69.2562,35.7407) -- (69.9028,32.8308) -- (69.9028,24.9094) --
      (68.1245,21.6762) -- (70.3878,12.2999) -- (72.4894,12.7849) --
      (74.9143,15.6948) -- (77.6625,18.2814) -- (80.8957,20.2213) --
      (85.4222,22.3229) -- (88.4938,22.9695) -- (91.4036,24.4245) --
      (94.7985,25.3944) -- (97.0618,25.2328) -- (97.0618,22.8079) --
      (98.3550,21.6762) -- (100.4566,20.3830) -- (100.7800,21.5146) --
      (101.1033,23.2928) -- (98.8400,23.7778) -- (98.5167,25.8794) --
      (100.2950,27.3344) -- (101.4266,29.7593) -- (102.0732,31.6992) --
      (103.5282,31.5375) -- (103.6898,30.2442) -- (102.7199,28.9510) --
      (102.2349,25.7177) -- (103.0432,23.9395) -- (102.3966,22.4845) --
      (102.3966,20.2213) -- (104.1748,16.6648) -- (103.0432,14.0782) --
      (100.6183,9.2284) -- (100.9416,8.4201) -- (102.0732,7.6118) --
      cycle(92.6165,13.5907) -- (94.6373,13.4291) -- (95.1223,14.8032) --
      (96.6581,13.1866) -- (99.0022,13.1866) -- (99.8105,14.7224) --
      (98.2747,16.4198) -- (98.9213,17.2281) -- (98.1939,19.2489) --
      (96.8197,19.6530) .. controls (96.8197,19.6530) and (95.9306,19.7339) ..
      (95.9306,19.4105) .. controls (95.9306,19.0872) and (97.3856,16.8240) ..
      (97.3856,16.8240) -- (95.6881,16.2581) -- (95.3648,17.7131) --
      (94.6373,18.3597) -- (93.1015,16.0965) -- (92.6165,13.5907) -- cycle;

    % CA
    \path[USA map/state, USA map/CA, local bounding box=CA] (144.6944,382.1981) -- (148.6345,381.7095) -- (150.1206,379.6981) --
      (150.6651,376.7570) -- (147.1136,376.1669) -- (146.5994,375.4986) --
      (147.0769,373.4663) -- (146.9176,372.8767) -- (148.8402,372.2571) --
      (151.8830,369.4244) -- (152.4645,364.4293) -- (153.8444,361.0272) --
      (155.7877,358.8609) -- (159.3066,357.2712) -- (160.9610,355.6664) --
      (161.0297,353.5576) -- (160.0363,352.9776) -- (159.0132,351.9048) --
      (157.8580,346.0564) -- (155.1728,341.2263) -- (155.7386,337.7213) --
      (153.3190,336.6920) -- (84.2577,232.5136) -- (103.1598,164.9121) --
      (36.0799,149.2141) -- (34.5730,153.9474) -- (34.4114,161.3838) --
      (29.2382,173.1850) -- (26.1667,175.7715) -- (25.8434,176.9032) --
      (24.0651,177.7115) -- (22.6102,181.9146) -- (21.8019,185.1478) --
      (24.5501,189.3510) -- (26.1667,193.5542) -- (27.2983,197.1107) --
      (26.9750,203.5771) -- (25.1967,206.6487) -- (24.5501,212.4685) --
      (23.5801,216.1867) -- (25.3584,220.0665) -- (28.1066,224.5930) --
      (30.3699,229.4428) -- (31.6632,233.4843) -- (31.3398,236.7175) --
      (31.0165,237.2025) -- (31.0165,239.3041) -- (36.6746,245.6089) --
      (36.1896,248.0338) -- (35.5430,250.2970) -- (34.8964,252.2369) --
      (35.0580,260.4816) -- (37.1596,264.1998) -- (39.0995,266.7864) --
      (41.8478,267.2714) -- (42.8177,270.0196) -- (41.6861,273.5761) --
      (39.5845,275.1927) -- (38.4529,275.1927) -- (37.6446,279.0726) --
      (38.1296,281.9825) -- (41.3628,286.3473) -- (42.9794,291.6821) --
      (44.4343,296.3702) -- (45.7276,299.4418) -- (49.1225,305.2616) --
      (50.5774,307.8481) -- (51.0624,310.7580) -- (52.6790,311.7280) --
      (52.6790,314.1529) -- (51.8707,316.0928) -- (50.0924,323.2059) --
      (49.6075,325.1458) -- (52.0324,327.8940) -- (56.2355,328.3790) --
      (60.7620,330.1573) -- (64.6419,332.2589) -- (67.5518,332.2589) --
      (70.4617,335.3304) -- (73.0482,340.1802) -- (74.1799,342.4435) --
      (78.0597,344.5451) -- (82.9095,345.3534) -- (84.3645,347.4550) --
      (85.0111,350.6882) -- (83.5562,351.3348) -- (83.8795,352.3048) --
      (87.1127,353.1131) -- (89.8609,353.2747) -- (93.0208,351.5879) --
      (96.9007,355.7911) -- (97.7090,358.0543) -- (100.2955,362.2575) --
      (100.6189,365.4907) -- (100.6189,374.8670) -- (101.1038,376.6453) --
      (111.1268,378.1002) -- (130.8494,380.8484) -- (144.6944,382.1981) --
      cycle(56.5592,338.4814) -- (57.8525,340.0172) -- (57.6908,341.3105) --
      (54.4576,341.2297) -- (53.8918,340.0173) -- (53.2452,338.5623) --
      (56.5592,338.4815) -- cycle(58.4992,338.4814) -- (59.7116,337.8348) --
      (63.2682,339.9364) -- (66.3397,341.1488) -- (65.4506,341.7955) --
      (60.9241,341.5530) -- (59.3075,339.9364) -- (58.4992,338.4814) --
      cycle(79.1918,358.2849) -- (80.9700,360.6290) -- (81.7783,361.5990) --
      (83.3141,362.1648) -- (83.8799,360.7098) -- (82.9100,358.9316) --
      (80.2426,356.9108) -- (79.1918,357.0725) -- (79.1918,358.2849) --
      cycle(77.7368,366.9338) -- (79.5151,370.0862) -- (80.7275,372.0261) --
      (79.2726,372.2686) -- (77.9793,371.0562) .. controls (77.9793,371.0562) and
      (77.2518,369.6012) .. (77.2518,369.1970) .. controls (77.2518,368.7929) and
      (77.2518,367.0146) .. (77.2518,367.0146) -- (77.7368,366.9338) -- cycle;

    \end{scope}
}
%% If tikz has been loaded, replace ampersand with \amp macro
\ifdefined\tikzset
    \tikzset{ampersand replacement = \amp}
\fi
%% tcolorbox styles for sidebyside layout
\tcbset{ sbsstyle/.style={raster before skip=2.0ex, raster equal height=rows, raster force size=false} }
\tcbset{ sbspanelstyle/.style={bwminimalstyle, fonttitle=\blocktitlefont} }
%% Enviroments for side-by-side and components
%% Necessary to use \NewTColorBox for boxes of the panels
%% "newfloat" environment to squash page-breaks within a single sidebyside
%% "xparse" environment for entire sidebyside
\NewDocumentEnvironment{sidebyside}{mmmm}
  {\begin{tcbraster}
    [sbsstyle,raster columns=#1,
    raster left skip=#2\linewidth,raster right skip=#3\linewidth,raster column skip=#4\linewidth]}
  {\end{tcbraster}}
%% "tcolorbox" environment for a panel of sidebyside
\NewTColorBox{sbspanel}{mO{top}}{sbspanelstyle,width=#1\linewidth,valign=#2}
%% extpfeil package for certain extensible arrows,
%% as also provided by MathJax extension of the same name
%% NB: this package loads mtools, which loads calc, which redefines
%%     \setlength, so it can be removed if it seems to be in the 
%%     way and your math does not use:
%%     
%%     \xtwoheadrightarrow, \xtwoheadleftarrow, \xmapsto, \xlongequal, \xtofrom
%%     
%%     we have had to be extra careful with variable thickness
%%     lines in tables, and so also load this package late
\usepackage{extpfeil}
%% Custom Preamble Entries, late (use latex.preamble.late)
%% Begin: Author-provided packages
%% (From  docinfo/latex-preamble/package  elements)
%% End: Author-provided packages
%% Begin: Author-provided macros
%% (From  docinfo/macros  element)
%% Plus three from MBX for XML characters
\renewcommand{\d}{\displaystyle}
\newcommand{\N}{\mathbb N}
\newcommand{\B}{\mathbf B}
\newcommand{\Z}{\mathbb Z}
\newcommand{\Q}{\mathbb Q}
\newcommand{\R}{\mathbb R}
\newcommand{\C}{\mathbb C}
\newcommand{\U}{\mathcal U}
\newcommand{\pow}{\mathcal P}
\newcommand{\inv}{^{-1}}
\newcommand{\st}{:}
\renewcommand{\iff}{\leftrightarrow}
\newcommand{\Iff}{\Leftrightarrow}
\newcommand{\imp}{\rightarrow}
\newcommand{\Imp}{\Rightarrow}
\newcommand{\isom}{\cong}

\renewcommand{\bar}{\overline}
\newcommand{\card}[1]{\left| #1 \right|}
\newcommand{\twoline}[2]{\begin{pmatrix}#1 \\ #2 \end{pmatrix}}

\newcommand{\vtx}[2]{node[fill,circle,inner sep=0pt, minimum size=4pt,label=#1:#2]{}}
\newcommand{\va}[1]{\vtx{above}{#1}}
\newcommand{\vb}[1]{\vtx{below}{#1}}
\newcommand{\vr}[1]{\vtx{right}{#1}}
\newcommand{\vl}[1]{\vtx{left}{#1}}
\renewcommand{\v}{\vtx{above}{}}
\newcommand{\lt}{<}
\newcommand{\gt}{>}
\newcommand{\amp}{&}
%% End: Author-provided macros
\begin{document}
\frontmatter
%% begin: half-title
\thispagestyle{empty}
{\titlepagefont\centering
\vspace*{0.28\textheight}
{\Huge Solution Manual}\\[2\baselineskip]
{\LARGE Discrete Mathematics: An Open Introduction, 3rd edition}\\
}
\clearpage
%% end:   half-title
%% begin: adcard
\thispagestyle{empty}
\null%
\clearpage
%% end:   adcard
%% begin: title page
%% Inspired by Peter Wilson's "titleDB" in "titlepages" CTAN package
\thispagestyle{empty}
{\titlepagefont\centering
\vspace*{0.14\textheight}
%% Target for xref to top-level element is ToC
\addtocontents{toc}{\protect\hypertarget{x:book:dmoi-solution-manual}{}}
{\Huge Solution Manual}\\[\baselineskip]
{\LARGE Discrete Mathematics: An Open Introduction, 3rd edition}\\[3\baselineskip]
{\Large Oscar Levin}\\[0.5\baselineskip]
{\Large University of Northern Colorado}\\[3\baselineskip]
{\Large April 14, 2020}\\}
\clearpage
%% end:   title page
%% begin: copyright-page
\thispagestyle{empty}
\hypertarget{g:colophon:idp140957718256}{}\vspace*{\stretch{2}}
\noindent{\bfseries Website}: \href{http:\slash{}\slash{}discrete.openmathbooks.org}{\mono{discrete.openmathbooks.org}}\par\medskip
\noindent\textcopyright{}2013\textendash{}2019\quad{}Oscar Levin\\[0.5\baselineskip]
 This work is licensed under the Creative Commons Attribution-ShareAlike 4.0 International License. To view a copy of this license, visit \href{http://creativecommons.org/licenses/by-sa/4.0/}{http:\slash{}\slash{}creativecommons.org\slash{}licenses\slash{}by-sa\slash{}4.0\slash{}}\par\medskip
\vspace*{\stretch{1}}
\null\clearpage
%% end:   copyright-page
%
%
\typeout{************************************************}
\typeout{Preface  Preface}
\typeout{************************************************}
%
\begin{preface}{Preface}{}{Preface}{}{}{x:preface:preface}
For updates and additional instructor resources, please contact the author.%
\end{preface}
%% begin: table of contents
%% Adjust Table of Contents
\setcounter{tocdepth}{2}
\renewcommand*\contentsname{Contents}
\tableofcontents
%% end:   table of contents
\mainmatter
\chapter*{0 Introduction and Preliminaries}
\addcontentsline{toc}{chapter}{0 Introduction and Preliminaries}
\chaptermark{0 Introduction and Preliminaries}
\section*{0.2 Mathematical Statements}
\addcontentsline{toc}{section}{0.2 Mathematical Statements}
\sectionmark{0.2 Mathematical Statements}
\subsection*{Exercises}
\addcontentsline{toc}{subsection}{Exercises}
\begin{divisionsolution}{0.2.1}{}{p:exercise:vWb}%
\end{divisionsolution}%
\begin{divisionsolution}{0.2.2}{}{p:exercise:cdk}%
Classify each of the sentences below as an atomic statement, a molecular statement, or not a statement at all. If the statement is molecular, say what kind it is (conjuction, disjunction, conditional, biconditional, negation).%
\begin{enumerate}[label=(\alph*)]
\item{}The sum of the first 100 odd positive integers.%
\item{}Everybody needs somebody sometime.%
\item{}The Broncos will win the Super Bowl or I'll eat my hat.%
\item{}We can have donuts for dinner, but only if it rains.%
\item{}Every natural number greater than 1 is either prime or composite.%
\item{}This sentence is false.%
\end{enumerate}
%
\par\smallskip%
\noindent\textbf{\blocktitlefont Solution}.\quad{}%
\begin{enumerate}[label=(\alph*)]
\item{}This is not a statement; it does not make sense to say it is true or false.%
\item{}This is an atomic statement (there are some quantifiers, but no connectives).%
\item{}This is a molecular statement, specifically a disjunction. Although if we read into it a bit more, what the speaker is really saying is that if the Broncos do not win the super bowl, then he will eat his hat, which would be a conditional.%
\item{}This is a molecular statement, a conditional.%
\item{}This is an atomic statement. Even though there is an ``or'' in the statement, it would not make sense to consider the two halves of the disjuction. This is because we quantified \emph{over} the disjunction. In symbols, we have \(\forall x (x > 1 \imp (P(x) \vee C(x)))\). If we drop the quantifier, we are not left with a statement, since there is a free variable.%
\item{}This is not a statement, although it certainly looks like one. Remember that statements must be true or false. If this sentence were true, that would make it false. If it were false, that would make it true. Examples like this are rare and usually arise from some sort of self-reference.%
\end{enumerate}
%
\end{divisionsolution}%
\begin{divisionsolution}{0.2.3}{}{p:exercise:Ikt}%
Suppose \(P\) and \(Q\) are the statements: \(P\): Jack passed math. \(Q\): Jill passed math.%
\begin{enumerate}[label=(\alph*)]
\item{}Translate ``Jack and Jill both passed math'' into symbols.%
\item{}Translate ``If Jack passed math, then Jill did not'' into symbols.%
\item{}Translate ``\(P \vee Q\)'' into English.%
\item{}Translate ``\(\neg(P \wedge Q) \imp Q\)'' into English.%
\item{}Suppose you know that if Jack passed math, then so did Jill. What can you conclude if you know that:%
\begin{enumerate}[label=\roman*.]
\item{}Jill passed math?%
\item{}Jill did not pass math?%
\end{enumerate}
%
\end{enumerate}
%
\par\smallskip%
\noindent\textbf{\blocktitlefont Solution}.\quad{}%
\begin{enumerate}[label=(\alph*)]
\item{}\(P \wedge Q\).%
\item{}\(P \imp \neg Q\).%
\item{}Jack passed math or Jill passed math (or both).%
\item{}If Jack and Jill did not both pass math, then Jill did.%
\item{}%
\begin{enumerate}[label=\roman*.]
\item{}Nothing else.%
\item{}Jack did not pass math either.%
\end{enumerate}
%
\end{enumerate}
%
\end{divisionsolution}%
\begin{divisionsolution}{0.2.4}{}{p:exercise:orC}%
\end{divisionsolution}%
\begin{divisionsolution}{0.2.5}{}{p:exercise:UyL}%
For each sentence below, decide whether it is an atomic statement, a molecular statement, or not a statement at all.%
\begin{enumerate}[label=(\alph*)]
\item{}Customers must wear shoes. \quad(\begin{itemize*}[label=$\square$,leftmargin=3em,itemjoin=\hspace{1em}]
\item{}atomic statement%

\item{}molecular statement%

\item{}not a statement%

\end{itemize*})\quad
%
\item{}The customers wore shoes. \quad(\begin{itemize*}[label=$\square$,leftmargin=3em,itemjoin=\hspace{1em}]
\item{}atomic statement%

\item{}molecular statement%

\item{}not a statement%

\end{itemize*})\quad
%
\item{}The customers wore shoes and they wore socks. \quad(\begin{itemize*}[label=$\square$,leftmargin=3em,itemjoin=\hspace{1em}]
\item{}atomic statement%

\item{}molecular statement%

\item{}not a statement%

\end{itemize*})\quad
%
\end{enumerate}
%
\par\smallskip%
\noindent\textbf{\blocktitlefont Answer 1}.\quad{}\(\text{not a statement}\)%
\par\smallskip%
\noindent\textbf{\blocktitlefont Answer 2}.\quad{}\(\text{atomic statement}\)%
\par\smallskip%
\noindent\textbf{\blocktitlefont Answer 3}.\quad{}\(\text{molecular statement}\)%
\par\smallskip%
\noindent\textbf{\blocktitlefont Solution}.\quad{}%
\begin{enumerate}[label=(\alph*)]
\item{}This is not a statement. It is an imperative sentence, but is not either true or false. It doesn't matter that this might actually be the rule or not. Note that ``The rule is that all customers must wear shoes'' \emph{is} a statement.%
\item{}This is a statement, as it is either true or false. It is an atomic statement because it cannot be divided into smaller statements.%
\item{}This is again a statement, but this time it is molecular. In fact, it is a conjunction, as we can write it as ``The customers wore shoes and the customers wore socks.''%
\end{enumerate}
%
\end{divisionsolution}%
\begin{divisionsolution}{0.2.6}{}{p:exercise:AFU}%
Determine whether each molecular statement below is true or false, or whether it is impossible to determine. Assume you do not know what my favorite number is (but you do know that 13 is prime).%
\begin{enumerate}[label=(\alph*)]
\item{}If 13 is prime, then 13 is my favorite number. \quad(\begin{itemize*}[label=$\square$,leftmargin=3em,itemjoin=\hspace{1em}]
\item{}True%

\item{}False%

\item{}Not enough information%

\end{itemize*})\quad
%
\item{}If 13 is my favorite number, then 13 is prime. \quad(\begin{itemize*}[label=$\square$,leftmargin=3em,itemjoin=\hspace{1em}]
\item{}True%

\item{}False%

\item{}Not enough information%

\end{itemize*})\quad
%
\item{}If 13 is not prime, then 13 is my favorite number. \quad(\begin{itemize*}[label=$\square$,leftmargin=3em,itemjoin=\hspace{1em}]
\item{}True%

\item{}False%

\item{}Not enough information%

\end{itemize*})\quad
%
\item{}13 is my favorite number or 13 is prime. \quad(\begin{itemize*}[label=$\square$,leftmargin=3em,itemjoin=\hspace{1em}]
\item{}True%

\item{}False%

\item{}Not enough information%

\end{itemize*})\quad
%
\item{}13 is my favorite number and 13 is prime. \quad(\begin{itemize*}[label=$\square$,leftmargin=3em,itemjoin=\hspace{1em}]
\item{}True%

\item{}False%

\item{}Not enough information%

\end{itemize*})\quad
%
\item{}7 is my favorite number and 13 is not prime. \quad(\begin{itemize*}[label=$\square$,leftmargin=3em,itemjoin=\hspace{1em}]
\item{}True%

\item{}False%

\item{}Not enough information%

\end{itemize*})\quad
%
\item{}13 is my favorite number or 13 is not my favorite number. \quad(\begin{itemize*}[label=$\square$,leftmargin=3em,itemjoin=\hspace{1em}]
\item{}True%

\item{}False%

\item{}Not enough information%

\end{itemize*})\quad
%
\end{enumerate}
%
\par\smallskip%
\noindent\textbf{\blocktitlefont Answer 1}.\quad{}\(\text{Not enough information}\)%
\par\smallskip%
\noindent\textbf{\blocktitlefont Answer 2}.\quad{}\(\text{True}\)%
\par\smallskip%
\noindent\textbf{\blocktitlefont Answer 3}.\quad{}\(\text{True}\)%
\par\smallskip%
\noindent\textbf{\blocktitlefont Answer 4}.\quad{}\(\text{True}\)%
\par\smallskip%
\noindent\textbf{\blocktitlefont Answer 5}.\quad{}\(\text{Not enough information}\)%
\par\smallskip%
\noindent\textbf{\blocktitlefont Answer 6}.\quad{}\(\text{False}\)%
\par\smallskip%
\noindent\textbf{\blocktitlefont Answer 7}.\quad{}\(\text{True}\)%
\par\smallskip%
\noindent\textbf{\blocktitlefont Solution}.\quad{}%
\begin{enumerate}[label=(\alph*)]
\item{}It is impossible to tell. The hypothesis of the implication is true. Thus the implication will be true if the conclusion is true (if 13 \emph{is} my favorite number) and false otherwise.%
\item{}This is true, no matter whether 13 is my favorite number or not. Any implication with a true conclusion is true.%
\item{}This is true, again, no matter whether 13 is my favorite number or not. Any implication with a false hypothesis is true.%
\item{}For a disjunction to be true, we just need one or the other (or both) of the parts to be true. Thus this is a true statement.%
\item{}We cannot tell. The statement would be true if 13 is my favorite number, and false if not (since a conjunction needs both parts to be true to be true).%
\item{}This is definitely false. 13 is prime, so its negation (13 is not prime) is false. At least one part of the conjunction is false, so the whole statement is false.%
\item{}This is true. Either 13 is my favorite number or it is not, but whichever it is, at least one part of the disjunction is true, so the whole statement is true.%
\end{enumerate}
%
\end{divisionsolution}%
\begin{divisionsolution}{0.2.7}{}{p:exercise:gNd}%
In my safe is a sheet of paper with two shapes drawn on it in colored crayon. One is a square, and the other is a triangle. Each shape is drawn in a single color. Suppose you believe me when I tell you that \emph{if the square is blue, then the triangle is green}. What do you therefore know about the truth value of the following statements?%
\begin{enumerate}[label=(\alph*)]
\item{}The square and the triangle are both blue. \quad(\begin{itemize*}[label=$\square$,leftmargin=3em,itemjoin=\hspace{1em}]
\item{}True%

\item{}False%

\item{}Not enough information%

\end{itemize*})\quad
%
\item{}The square and the triangle are both green. \quad(\begin{itemize*}[label=$\square$,leftmargin=3em,itemjoin=\hspace{1em}]
\item{}True%

\item{}False%

\item{}Not enough information%

\end{itemize*})\quad
%
\item{}If the triangle is not green, then the square is not blue. \quad(\begin{itemize*}[label=$\square$,leftmargin=3em,itemjoin=\hspace{1em}]
\item{}True%

\item{}False%

\item{}Not enough information%

\end{itemize*})\quad
%
\item{}If the triangle is green, then the square is blue. \quad(\begin{itemize*}[label=$\square$,leftmargin=3em,itemjoin=\hspace{1em}]
\item{}True%

\item{}False%

\item{}Not enough information%

\end{itemize*})\quad
%
\item{}The square is not blue or the triangle is green. \quad(\begin{itemize*}[label=$\square$,leftmargin=3em,itemjoin=\hspace{1em}]
\item{}True%

\item{}False%

\item{}Not enough information%

\end{itemize*})\quad
%
\end{enumerate}
%
\par\smallskip%
\noindent\textbf{\blocktitlefont Answer 1}.\quad{}\(\text{False}\)%
\par\smallskip%
\noindent\textbf{\blocktitlefont Answer 2}.\quad{}\(\text{Not enough information}\)%
\par\smallskip%
\noindent\textbf{\blocktitlefont Answer 3}.\quad{}\(\text{True}\)%
\par\smallskip%
\noindent\textbf{\blocktitlefont Answer 4}.\quad{}\(\text{Not enough information}\)%
\par\smallskip%
\noindent\textbf{\blocktitlefont Answer 5}.\quad{}\(\text{True}\)%
\par\smallskip%
\noindent\textbf{\blocktitlefont Solution}.\quad{}The main thing to realize is that we don't know the colors of these two shapes, but we do know that we are in one of three cases: We could have a blue square and green triangle. We could have a square that was not blue but a green triangle. Or we could have a square that was not blue and a triangle that was not green. The case in which the square is blue but the triangle is not green cannot occur, as that would make the statement false.%
\begin{enumerate}[label=(\alph*)]
\item{}This must be false. In fact, this is the negation of the original implication.%
\item{}This might be true or might be false.%
\item{}True. This is the contrapositive of the original statement, which is logically equivalent to it.%
\item{}We do not know. This is the converse of the original statement. In particular, if the square is not blue but the triangle is green, then the original statement is true but the converse is false.%
\item{}True. This is logically equivalent to the original statement.%
\end{enumerate}
%
\end{divisionsolution}%
\begin{divisionsolution}{0.2.8}{}{p:exercise:MUm}%
Consider the statement ``If Oscar eats Chinese food, then he drinks milk.''%
\begin{enumerate}[label=(\alph*)]
\item{}Write the converse of the statement.%
\item{}Write the contrapositive of the statement.%
\item{}Is it possible for the contrapositive to be false? If it was, what would that tell you?%
\item{}Suppose the original statement is true, and that Oscar drinks milk. Can you conclude anything (about his eating Chinese food)? Explain.%
\item{}Suppose the original statement is true, and that Oscar does not drink milk. Can you conclude anything (about his eating Chinese food)? Explain.%
\end{enumerate}
%
\par\smallskip%
\noindent\textbf{\blocktitlefont Solution}.\quad{}%
\begin{enumerate}[label=(\alph*)]
\item{}If Oscar drinks milk, then he eats Chinese food.%
\item{}If Oscar does not drink milk, then he does not eat Chinese food.%
\item{}Yes. The original statement would be false too.%
\item{}Nothing. The converse need not be true.%
\item{}He does not eat Chinese food. The contrapositive would be true.%
\end{enumerate}
\end{divisionsolution}%
\begin{divisionsolution}{0.2.9}{}{p:exercise:tbv}%
Again, suppose the statement ``if the square is blue, then the triangle is green'' is true. This time however, assume the converse is false. Classify each statement below as true or false (if possible).%
\begin{enumerate}[label=(\alph*)]
\item{}The square is blue if and only if the triangle is green. \quad(\begin{itemize*}[label=$\square$,leftmargin=3em,itemjoin=\hspace{1em}]
\item{}True%

\item{}False%

\item{}Not enough information%

\end{itemize*})\quad
%
\item{}The square is blue if and only if the triangle is not green. \quad(\begin{itemize*}[label=$\square$,leftmargin=3em,itemjoin=\hspace{1em}]
\item{}True%

\item{}False%

\item{}Not enough information%

\end{itemize*})\quad
%
\item{}The square is blue. \quad(\begin{itemize*}[label=$\square$,leftmargin=3em,itemjoin=\hspace{1em}]
\item{}True%

\item{}False%

\item{}Not enough information%

\end{itemize*})\quad
%
\item{}The triangle is green. \quad(\begin{itemize*}[label=$\square$,leftmargin=3em,itemjoin=\hspace{1em}]
\item{}True%

\item{}False%

\item{}Not enough information%

\end{itemize*})\quad
%
\end{enumerate}
%
\par\smallskip%
\noindent\textbf{\blocktitlefont Answer 1}.\quad{}\(\text{False}\)%
\par\smallskip%
\noindent\textbf{\blocktitlefont Answer 2}.\quad{}\(\text{True}\)%
\par\smallskip%
\noindent\textbf{\blocktitlefont Answer 3}.\quad{}\(\text{False}\)%
\par\smallskip%
\noindent\textbf{\blocktitlefont Answer 4}.\quad{}\(\text{True}\)%
\par\smallskip%
\noindent\textbf{\blocktitlefont Solution}.\quad{}The only way for an implication \(P\imp Q\) to be true but its converse to be false is for \(Q\) to be true and \(P\) to be false. Thus:%
\begin{enumerate}[label=(\alph*)]
\item{}False.%
\item{}True.%
\item{}False.%
\item{}True.%
\end{enumerate}
%
\end{divisionsolution}%
\begin{divisionsolution}{0.2.10}{}{p:exercise:ZiE}%
Write each of the following statements in the form, ``if \textellipsis{}, then \textellipsis{}.'' Careful, some of the statements might be false (which is alright for the purposes of this question).%
\begin{enumerate}[label=(\alph*)]
\item{}To lose weight, you must exercise.%
\item{}To lose weight, all you need to do is exercise.%
\item{}Every American is patriotic.%
\item{}You are patriotic only if you are American.%
\item{}The set of rational numbers is a subset of the real numbers.%
\item{}A number is prime if it is not even.%
\item{}Either the Broncos will win the Super Bowl, or they won't play in the Super Bowl.%
\end{enumerate}
%
\par\smallskip%
\noindent\textbf{\blocktitlefont Solution}.\quad{}%
\begin{enumerate}[label=(\alph*)]
\item{}If you have lost weight, then you exercised.%
\item{}If you exercise, then you will lose weight.%
\item{}If you are American, then you are patriotic.%
\item{}If you are patriotic, then you are American.%
\item{}If a number is rational, then it is real.%
\item{}If a number is not even, then it is prime. (Or the contrapositive: if a number is not prime, then it is even.)%
\item{}If the Broncos don't win the Super Bowl, then they didn't play in the Super Bowl. Alternatively, if the Broncos play in the Super Bowl, then they will win the Super Bowl.%
\end{enumerate}
%
\end{divisionsolution}%
\begin{divisionsolution}{0.2.11}{}{p:exercise:FpN}%
Which of the following statements are equivalent to the implication, ``if you win the lottery, then you will be rich,'' and which are equivalent to the converse of the implication?%
\begin{enumerate}[label=(\alph*)]
\item{}Either you win the lottery or else you are not rich.%
\item{}Either you don't win the lottery or else you are rich.%
\item{}You will win the lottery and be rich.%
\item{}You will be rich if you win the lottery.%
\item{}You will win the lottery if you are rich.%
\item{}It is necessary for you to win the lottery to be rich.%
\item{}It is sufficient to win the lottery to be rich.%
\item{}You will be rich only if you win the lottery.%
\item{}Unless you win the lottery, you won't be rich.%
\item{}If you are rich, you must have won the lottery.%
\item{}If you are not rich, then you did not win the lottery.%
\item{}You will win the lottery if and only if you are rich.%
\end{enumerate}
%
\par\smallskip%
\noindent\textbf{\blocktitlefont Solution}.\quad{}The statements are equivalent to the\textellipsis{}%
\begin{enumerate}[label=(\alph*)]
\item{}converse.%
\item{}implication.%
\item{}neither.%
\item{}implication.%
\item{}converse.%
\item{}converse.%
\item{}implication.%
\item{}converse.%
\item{}converse.%
\item{}converse (in fact, this \emph{is} the converse).%
\item{}implication (the statement is the contrapositive of the implication).%
\item{}neither.%
\end{enumerate}
%
\end{divisionsolution}%
\begin{divisionsolution}{0.2.12}{}{p:exercise:lwW}%
Consider the statement, ``If you will give me a cow, then I will give you magic beans.'' Decide whether each statement below is the converse, the contrapositive, or neither.%
\begin{enumerate}[label=(\alph*)]
\item{}If you will give me a cow, then I will not give you magic beans. \quad(\begin{itemize*}[label=$\square$,leftmargin=3em,itemjoin=\hspace{1em}]
\item{}Converse%

\item{}Contrapositive%

\item{}Neither%

\end{itemize*})\quad
%
\item{}If I will not give you magic beans, then you will not give me a cow. \quad(\begin{itemize*}[label=$\square$,leftmargin=3em,itemjoin=\hspace{1em}]
\item{}Converse%

\item{}Contrapositive%

\item{}Neither%

\end{itemize*})\quad
%
\item{}If I will give you magic beans, then you will give me a cow. \quad(\begin{itemize*}[label=$\square$,leftmargin=3em,itemjoin=\hspace{1em}]
\item{}Converse%

\item{}Contrapositive%

\item{}Neither%

\end{itemize*})\quad
%
\item{}If you will not give me a cow, then I will not give you magic beans. \quad(\begin{itemize*}[label=$\square$,leftmargin=3em,itemjoin=\hspace{1em}]
\item{}Converse%

\item{}Contrapositive%

\item{}Neither%

\end{itemize*})\quad
%
\item{}You will give me a cow and I will not give you magic beans. \quad(\begin{itemize*}[label=$\square$,leftmargin=3em,itemjoin=\hspace{1em}]
\item{}Converse%

\item{}Contrapositive%

\item{}Neither%

\end{itemize*})\quad
%
\item{}If I will give you magic beans, then you will not give me a cow. \quad(\begin{itemize*}[label=$\square$,leftmargin=3em,itemjoin=\hspace{1em}]
\item{}Converse%

\item{}Contrapositive%

\item{}Neither%

\end{itemize*})\quad
%
\end{enumerate}
%
\par\smallskip%
\noindent\textbf{\blocktitlefont Answer 1}.\quad{}\(\text{Neither}\)%
\par\smallskip%
\noindent\textbf{\blocktitlefont Answer 2}.\quad{}\(\text{Contrapositive}\)%
\par\smallskip%
\noindent\textbf{\blocktitlefont Answer 3}.\quad{}\(\text{Converse}\)%
\par\smallskip%
\noindent\textbf{\blocktitlefont Answer 4}.\quad{}\(\text{Neither}\)%
\par\smallskip%
\noindent\textbf{\blocktitlefont Answer 5}.\quad{}\(\text{Neither}\)%
\par\smallskip%
\noindent\textbf{\blocktitlefont Answer 6}.\quad{}\(\text{Neither}\)%
\par\smallskip%
\noindent\textbf{\blocktitlefont Solution}.\quad{}The converse is ``If I will give you magic beans, then you will give me a cow.'' The contrapositive is ``If I will not give you magic beans, then you will not give me a cow.'' All the other statements are neither the converse nor contrapositive.%
\end{divisionsolution}%
\begin{divisionsolution}{0.2.13}{}{p:exercise:REf}%
You have discovered an old paper on graph theory that discusses the \emph{viscosity} of a graph (which for all you know, is something completely made up by the author). A theorem in the paper claims that ``if a graph satisfies \emph{condition (V)}, then the graph is \emph{viscous}.'' Which of the following are equivalent ways of stating this claim? Which are equivalent to the \emph{converse} of the claim?%
\begin{enumerate}[label=(\alph*)]
\item{}A graph is viscous only if it satisfies condition (V). \quad(\begin{itemize*}[label=$\square$,leftmargin=3em,itemjoin=\hspace{1em}]
\item{}Original%

\item{}Converse%

\item{}Neither%

\end{itemize*})\quad
%
\item{}A graph is viscous if it satisfies condition (V). \quad(\begin{itemize*}[label=$\square$,leftmargin=3em,itemjoin=\hspace{1em}]
\item{}Original%

\item{}Converse%

\item{}Neither%

\end{itemize*})\quad
%
\item{}For a graph to be viscous, it is necessary that it satisfies condition (V). \quad(\begin{itemize*}[label=$\square$,leftmargin=3em,itemjoin=\hspace{1em}]
\item{}Original%

\item{}Converse%

\item{}Neither%

\end{itemize*})\quad
%
\item{}For a graph to be viscous, it is sufficient for it to satisfy condition (V). \quad(\begin{itemize*}[label=$\square$,leftmargin=3em,itemjoin=\hspace{1em}]
\item{}Original%

\item{}Converse%

\item{}Neither%

\end{itemize*})\quad
%
\item{}Satisfying condition (V) is a sufficient condition for a graph to be viscous. \quad(\begin{itemize*}[label=$\square$,leftmargin=3em,itemjoin=\hspace{1em}]
\item{}Original%

\item{}Converse%

\item{}Neither%

\end{itemize*})\quad
%
\item{}Satisfying condition (V) is a necessary condition for a graph to be viscous. \quad(\begin{itemize*}[label=$\square$,leftmargin=3em,itemjoin=\hspace{1em}]
\item{}Original%

\item{}Converse%

\item{}Neither%

\end{itemize*})\quad
%
\item{}Every viscous graph satisfies condition (V). \quad(\begin{itemize*}[label=$\square$,leftmargin=3em,itemjoin=\hspace{1em}]
\item{}Original%

\item{}Converse%

\item{}Neither%

\end{itemize*})\quad
%
\item{}Only viscous graphs satisfy condition (V). \quad(\begin{itemize*}[label=$\square$,leftmargin=3em,itemjoin=\hspace{1em}]
\item{}Original%

\item{}Converse%

\item{}Neither%

\end{itemize*})\quad
%
\end{enumerate}
%
\par\smallskip%
\noindent\textbf{\blocktitlefont Answer 1}.\quad{}\(\text{Converse}\)%
\par\smallskip%
\noindent\textbf{\blocktitlefont Answer 2}.\quad{}\(\text{Original}\)%
\par\smallskip%
\noindent\textbf{\blocktitlefont Answer 3}.\quad{}\(\text{Converse}\)%
\par\smallskip%
\noindent\textbf{\blocktitlefont Answer 4}.\quad{}\(\text{Original}\)%
\par\smallskip%
\noindent\textbf{\blocktitlefont Answer 5}.\quad{}\(\text{Original}\)%
\par\smallskip%
\noindent\textbf{\blocktitlefont Answer 6}.\quad{}\(\text{Converse}\)%
\par\smallskip%
\noindent\textbf{\blocktitlefont Answer 7}.\quad{}\(\text{Converse}\)%
\par\smallskip%
\noindent\textbf{\blocktitlefont Answer 8}.\quad{}\(\text{Original}\)%
\par\smallskip%
\noindent\textbf{\blocktitlefont Solution}.\quad{}%
\begin{enumerate}[label=(\alph*)]
\item{}Equivalent to the converse.%
\item{}Equivalent to the original theorem.%
\item{}Equivalent to the converse.%
\item{}Equivalent to the original theorem.%
\item{}Equivalent to the original theorem.%
\item{}Equivalent to the converse.%
\item{}Equivalent to the converse.%
\item{}Equivalent to the original theorem.%
\end{enumerate}
%
\end{divisionsolution}%
\begin{divisionsolution}{0.2.14}{}{p:exercise:xLo}%
Let \(P(x)\) be the predicate, ``\(3x+1\) is even.''%
\begin{enumerate}[label=(\alph*)]
\item{}Is \(P(5)\) true or false? \quad(\begin{itemize*}[label=$\square$,leftmargin=3em,itemjoin=\hspace{1em}]
\item{}True%

\item{}False%

\item{}Neither (not a statement)%

\end{itemize*})\quad
%
\item{}What, if anything, can you conclude about \(\exists x P(x)\) from the truth value of \(P(5)\text{?}\)%
\item{}What, if anything, can you conclude about \(\forall x P(x)\) from the truth value of \(P(5)\text{?}\)%
\end{enumerate}
%
\par\smallskip%
\noindent\textbf{\blocktitlefont Answer}.\quad{}\(\text{True}\)%
\par\smallskip%
\noindent\textbf{\blocktitlefont Solution}.\quad{}\(P(5)\) is the statement ``\(3\cdot 5 + 1\) is even'', which is true. Thus the statement \(\exists x P(x)\) is true (for example, 5 is such an \(x\)). However, we cannot tell anything about \(\forall x P(x)\) since we do not know the truth value of \(P(x)\) for \emph{all} elements of the domain of discourse. In this case, \(\forall x P(x)\) happens to be false (since \(P(4)\) is false, for example).%
\end{divisionsolution}%
\begin{divisionsolution}{0.2.15}{}{p:exercise:dSx}%
Suppose \(P(x,y)\) is some binary predicate defined on a very small domain of discourse: just the integers 1, 2, 3, and 4. For each of the 16 pairs of these numbers, \(P(x,y)\) is either true or false, according to the following table (\(x\) values are rows, \(y\) values are columns).%
\begin{sidebyside}{1}{0.25}{0.25}{0}%
\begin{sbspanel}{0.5}%
{\centering%
{\tabularfont%
\begin{tabular}{lllll}
\multicolumn{1}{rA}{}&1&2&3&4\tabularnewline\hrulethin
\multicolumn{1}{rA}{1}&T&F&F&F\tabularnewline[0pt]
\multicolumn{1}{rA}{2}&F&T&T&F\tabularnewline[0pt]
\multicolumn{1}{rA}{3}&T&T&T&T\tabularnewline[0pt]
\multicolumn{1}{rA}{4}&F&F&F&F
\end{tabular}
}%
\par}
\end{sbspanel}%
\end{sidebyside}%
\par
For example, \(P(1,3)\) is false, as indicated by the F in the first row, third column.%
\par\medskip
Let \(P(x)\) be the predicate, ``\(4x+1\) is even.''%
\begin{enumerate}[label=(\alph*)]
\item{}Is \(P(5)\) true or false? \quad(\begin{itemize*}[label=$\square$,leftmargin=3em,itemjoin=\hspace{1em}]
\item{}True%

\item{}False%

\item{}Not enough information%

\end{itemize*})\quad
%
\item{}What, if anything, can you conclude about \(\exists x P(x)\) from the truth value of \(P(5)\text{?}\)%
\item{}What, if anything, can you conclude about \(\forall x P(x)\) from the truth value of \(P(5)\text{?}\)%
\end{enumerate}
%
\par\smallskip%
\noindent\textbf{\blocktitlefont Answer}.\quad{}\(\text{False}\)%
\par\smallskip%
\noindent\textbf{\blocktitlefont Solution}.\quad{}Here \(P(5)\) is false, since it is the statement ``\(4\cdot 5 + 1\) is even'' (but 21 is odd). This does not tell us anything about the statement \(\exists x P(x)\text{,}\) since there still might be some \(x\) that makes \(P(x)\) true (although in this case, if our domain of discourse is the integers, there isn't, so in fact \(\exists x P(x)\) is false). On the other hand, \(\forall x P(x)\) is definitely false based on \(P(5)\) being false (we know that not all values of \(x\) make \(P(x)\) true).%
\end{divisionsolution}%
\begin{divisionsolution}{0.2.16}{}{p:exercise:JZG}%
Translate into symbols. Use \(E(x)\) for ``\(x\) is even'' and \(O(x)\) for ``\(x\) is odd.''%
\begin{enumerate}[label=(\alph*)]
\item{}No number is both even and odd.%
\item{}One more than any even number is an odd number.%
\item{}There is prime number that is even.%
\item{}Between any two numbers there is a third number.%
\item{}There is no number between a number and one more than that number.%
\end{enumerate}
%
\par\smallskip%
\noindent\textbf{\blocktitlefont Solution}.\quad{}%
\begin{enumerate}[label=(\alph*)]
\item{}\(\neg \exists x (E(x) \wedge O(x))\).%
\item{}\(\forall x (E(x) \imp O(x+1))\).%
\item{}\(\exists x(P(x) \wedge E(x))\) (where \(P(x)\) means ``\(x\) is prime'').%
\item{}\(\forall x \forall y \exists z(x \lt z \lt y \vee y \lt z \lt x)\).%
\item{}\(\forall x \neg \exists y (x \lt y \lt x+1)\).%
\end{enumerate}
%
\end{divisionsolution}%
\begin{divisionsolution}{0.2.17}{}{p:exercise:qgP}%
Translate into English:%
\begin{enumerate}[label=(\alph*)]
\item{}\(\forall x (E(x) \imp E(x +2))\).%
\item{}\(\forall x \exists y (\sin(x) = y)\).%
\item{}\(\forall y \exists x (\sin(x) = y)\).%
\item{}\(\forall x \forall y (x^3 = y^3 \imp x = y)\).%
\end{enumerate}
%
\par\smallskip%
\noindent\textbf{\blocktitlefont Solution}.\quad{}%
\begin{enumerate}[label=(\alph*)]
\item{}Any even number plus 2 is an even number.%
\item{}For any \(x\) there is a \(y\) such that \(\sin(x) = y\). In other words, every number \(x\) is in the domain of sine.%
\item{}For every \(y\) there is an \(x\) such that \(\sin(x) = y\). In other words, every number \(y\) is in the range of sine (which is false).%
\item{}For any numbers, if the cubes of two numbers are equal, then the numbers are equal.%
\end{enumerate}
%
\end{divisionsolution}%
\begin{divisionsolution}{0.2.18}{}{p:exercise:WnY}%
Suppose \(P(x)\) is some predicate for which the statement \(\forall x P(x)\) is true. Is it also the case that \(\exists x P(x)\) is true? In other words, is the statement \(\forall x P(x) \imp \exists x P(x)\) always true? Is the converse always true?  Assume the domain of discourse is non-empty.%
\par\smallskip%
\noindent\textbf{\blocktitlefont Hint}.\quad{}Try an example.  What if \(P(x)\) was the predicate, ``\(x\) is prime''?  What if it was ``if \(x\) is divisible by 4, then it is even''?  Of course examples are not enough to prove something in general, but that is entirely the point of this question.%
\par\smallskip%
\noindent\textbf{\blocktitlefont Solution}.\quad{}If \(P(x)\) is true of every \(x\), then in particular it is true of \(x = 0\) (or any fixed element of the universe). So then there is definitely some \(x\) (namely 0) for which \(P(x)\) holds. Thus \(\forall x P(x) \imp \exists x P(x)\) is always true. The converse is not always true though. Consider the predicate \(x = 5\). So \(P(x)\) is true if and only if \(x = 5\). Certainly it is true that \(\exists x P(x)\) (since we can take \(x = 5\)), but false that \(\forall x P(x)\).%
\par
Note that if the domain of discourse was empty, then no matter what we interpreted the predicate \(P(x)\) as, we would have \(\forall x P(x)\) true, but \(\exists x P(x)\) false.%
\end{divisionsolution}%
\begin{divisionsolution}{0.2.19}{}{p:exercise:Cvh}%
For each of the statements below, give a domain of discourse for which the statement is true, and a domain for which the statement is false.%
\begin{enumerate}[label=(\alph*)]
\item{}\(\forall x \exists y (y^2 = x)\).%
\item{}\(\forall x \forall y (x \lt y \imp \exists z (x \lt z \lt y))\).%
\item{}\(\exists x \forall y \forall z (y \lt z \imp y \le x \le z)\).%
\end{enumerate}
%
\par\smallskip%
\noindent\textbf{\blocktitlefont Hint}.\quad{}First figure out what each statement is saying.  For part (c), you don't need to assume the domain is an infinite set.%
\par\smallskip%
\noindent\textbf{\blocktitlefont Solution}.\quad{}%
\begin{enumerate}[label=(\alph*)]
\item{}This says that everything has a square root (every element is the square of something). This is true of the positive real numbers, and also of the complex numbers. It is false of the natural numbers though, as for \(x = 2\) there is no natural number \(y\) such that \(y^2 = 2\).%
\item{}This asserts that between every pair of numbers there is some number strictly between them. This is true of the rationals (and reals) but false of the integers. If \(x = 1\) and \(y = 2\), then there is nothing we can take for \(z\).%
\item{}Here we are saying that something is between every pair of numbers (note though that we don't mean \emph{strictly} between, as we allow 2 to be between 2 and 3). For almost every domain, this is false. In fact, if the domain contains \(\{1,2,3, 4\}\), then no matter what we take \(x\) to be, there will be a pair that \(x\) is \emph{not} between. However, the set \(\{1,2,3\}\) as our domain makes the statement true. Let \(x = 2\). Then no matter what \(y\) and \(z\) we pick, if \(y \lt z\), then 2 is between them.%
\end{enumerate}
%
\end{divisionsolution}%
\begin{divisionsolution}{0.2.20}{}{p:exercise:iCq}%
Consider the statement, ``For all natural numbers \(n\), if \(n\) is prime, then \(n\) is solitary.'' You do not need to know what \emph{solitary} means for this problem, just that it is a property that some numbers have and others do not.%
\begin{enumerate}[label=(\alph*)]
\item{}Write the converse and the contrapositive of the statement, saying which is which. Note: the original statement claims that an implication is true for all \(n\), and it is that implication that we are taking the converse and contrapositive of.%
\item{}Write the negation of the original statement. What would you need to show to prove that the statement is false?%
\item{}Even though you don't know whether 10 is solitary (in fact, nobody knows this), is the statement ``if 10 is prime, then 10 is solitary'' true or false? Explain.%
\item{}It turns out that 8 is solitary. Does this tell you anything about the truth or falsity of the original statement, its converse or its contrapositive? Explain.%
\item{}Assuming that the original statement is true, what can you say about the relationship between the \emph{set} \(P\) of prime numbers and the \emph{set} \(S\) of solitary numbers. Explain.%
\end{enumerate}
%
\par\smallskip%
\noindent\textbf{\blocktitlefont Solution}.\quad{}%
\begin{enumerate}[label=(\alph*)]
\item{}The converse: For all numbers \(n\), if \(n\) is solitary, then \(n\) is prime. The contrapositive: For all numbers \(n\), if \(n\) is not solitary, then \(n\) is not prime.%
\item{}The negation: There is a natural number \(n\) which is prime \emph{and} not solitary. So to prove the original statement false, we need to find one example of a number which is prime but not solitary.%
\item{}This statement is true. The hypothesis of the statement is false (10 is not prime), so the implication is automatically true.%
\item{}This does not tell you anything about the original implication or its contrapositive (since the contrapositive is equivalent to the original statement). All we can say is that there is a number which is not prime and not solitary. However, the converse is proved false by this example since there is a number (8) which is both solitary and not prime.%
\item{}We can say that \(P \subseteq S\), since this claims that every element of the set of primes is also an element of the set of solitary numbers. Thanks to the information that 8 is not solitary, we know that \(P \ne S\) so in fact \(P \subset S\) is also true.%
\end{enumerate}
%
\end{divisionsolution}%
\section*{0.3 Sets}
\addcontentsline{toc}{section}{0.3 Sets}
\sectionmark{0.3 Sets}
\subsection*{Exercises}
\addcontentsline{toc}{subsection}{Exercises}
\begin{divisionsolution}{0.3.1}{}{p:exercise:AUc}%
For a given predicate \(P(x)\text{,}\) you might believe that the statements \(\forall x P(x)\) or \(\exists x P(x)\) are either true or false. How would you decide if you were correct in each case?  You have four choices: you could give an example of an element \(n\) in the domain for which \(P(n)\) is true or for which \(P(n)\) if false, or you could argue that no matter what \(n\) is, \(P(n)\) is true or is false.%
\begin{enumerate}[label=(\alph*)]
\item{}What would you need to do to prove \(\forall x P(x)\) is true?%
\item{}What would you need to do to prove \(\forall x P(x)\) is false?%
\item{}What would you need to do to prove \(\exists x P(x)\) is true?%
\item{}What would you need to do to prove \(\exists x P(x)\) is false?%
\end{enumerate}
%
\par\smallskip%
\noindent\textbf{\blocktitlefont Solution}.\quad{}%
\begin{enumerate}[label=(\alph*)]
\item{}The claim that \(\forall x P(x)\) means that \(P(n)\) is true no matter what \(n\) you consider in the domain of discourse. Thus the only way to prove that \(\forall x P(x)\) is true is to check or otherwise argue that \(P(n)\) is true for all \(n\) in the domain.%
\item{}To prove \(\forall x P(x)\) is false all you need is one example of an element in the domain for which \(P(n)\) is false. This is often called a counterexample.%
\item{}We are simply claiming that there is some element \(n\) in the domain of discourse for which \(P(n)\) is true. If you can find one such element, you have verified the claim.%
\item{}Here we are claiming that no element we find will make \(P(n)\) true. The only way to be sure of this is to verify that \emph{every} element of the domain makes \(P(n)\) false. Note that the level of proof needed for this statement is the same as to prove that \(\forall x P(x)\) is true.%
\end{enumerate}
%
\end{divisionsolution}%
\begin{divisionsolution}{0.3.2}{}{p:exercise:hbl}%
Use the table to decide whether the following statements are true or false.%
\begin{enumerate}[label=(\alph*)]
\item{}\(\forall x \exists y P(x,y)\text{.}\) \quad(\begin{itemize*}[label=$\square$,leftmargin=3em,itemjoin=\hspace{1em}]
\item{}True%

\item{}False%

\item{}Not enough information%

\end{itemize*})\quad
%
\item{}\(\forall y \exists x P(x,y)\text{.}\) \quad(\begin{itemize*}[label=$\square$,leftmargin=3em,itemjoin=\hspace{1em}]
\item{}True%

\item{}False%

\item{}Not enough information%

\end{itemize*})\quad
%
\item{}\(\exists x \forall y P(x,y)\text{.}\) \quad(\begin{itemize*}[label=$\square$,leftmargin=3em,itemjoin=\hspace{1em}]
\item{}True%

\item{}False%

\item{}Not enough information%

\end{itemize*})\quad
%
\item{}\(\exists y \forall x P(x,y)\text{.}\) \quad(\begin{itemize*}[label=$\square$,leftmargin=3em,itemjoin=\hspace{1em}]
\item{}True%

\item{}False%

\item{}Not enough information%

\end{itemize*})\quad
%
\end{enumerate}
%
\par\smallskip%
\noindent\textbf{\blocktitlefont Answer 1}.\quad{}\(\text{False}\)%
\par\smallskip%
\noindent\textbf{\blocktitlefont Answer 2}.\quad{}\(\text{True}\)%
\par\smallskip%
\noindent\textbf{\blocktitlefont Answer 3}.\quad{}\(\text{True}\)%
\par\smallskip%
\noindent\textbf{\blocktitlefont Answer 4}.\quad{}\(\text{False}\)%
\par\smallskip%
\noindent\textbf{\blocktitlefont Solution}.\quad{}%
\begin{enumerate}[label=(\alph*)]
\item{}\(\forall x \exists y P(x,y)\) is false because when \(x = 4\text{,}\) there is no \(y\) which makes \(P(4,y)\) true.%
\item{}\(\forall y \exists x P(x,y)\) is true. No matter what \(y\) is (i.e., no matter what column we are in) there is some \(x\) for which \(P(x,y)\) is true. In fact, we can always take \(x\) to be \(3\text{.}\)%
\item{}\(\exists x \forall y P(x,y)\) is true. In particular \(x=3\) is such a number, so that no matter what \(y\) is, \(P(x,y)\) is true.%
\item{}\(\exists y \forall x P(x,y)\) is false. In fact, no matter what \(y\) (column) we look at, there is always some \(x\) (row) which makes \(P(x,y)\) false.%
\end{enumerate}
%
\end{divisionsolution}%
\begin{divisionsolution}{0.3.3}{}{p:exercise:Niu}%
Let \(A = \{1, 4, 9\}\) and \(B = \{1, 3, 6, 10\}\text{.}\) Find each of the following sets.%
\begin{enumerate}[label=(\alph*)]
\item{}\(A \cup B\text{.}\)%
\item{}\(A \cap B\text{.}\)%
\item{}\(A \setminus B\text{.}\)%
\item{}\(B \setminus A\text{.}\)%
\end{enumerate}
%
\par\smallskip%
\noindent\textbf{\blocktitlefont Answer 1}.\quad{}\(\left\{1,3,4,6,9,10\right\}\)%
\par\smallskip%
\noindent\textbf{\blocktitlefont Answer 2}.\quad{}\(\left\{1\right\}\)%
\par\smallskip%
\noindent\textbf{\blocktitlefont Answer 3}.\quad{}\(\left\{4,9\right\}\)%
\par\smallskip%
\noindent\textbf{\blocktitlefont Answer 4}.\quad{}\(\left\{3,6,10\right\}\)%
\par\smallskip%
\noindent\textbf{\blocktitlefont Solution 1}.\quad{}%
\begin{enumerate}[label=(\alph*)]
\item{}\(\{1, 3, 4, 6, 9, 10\}\text{.}\)%
\item{}\(\{1\}\text{.}\)%
\item{}\(\{4,9\}\text{.}\)%
\item{}\(\{3, 6, 10\}\text{.}\)%
\end{enumerate}
%
\par\smallskip%
\noindent\textbf{\blocktitlefont Solution 2}.\quad{}%
\begin{enumerate}[label=(\alph*)]
\item{}\(A \cup B = \{1,3,4,6,9,10\}\text{.}\) It includes everything that is in \(A\) or \(B\) or both.%
\item{}\(A \cap B = \{1\}\text{.}\) It contains everything that is in both \(A\) and \(B\text{.}\)%
\item{}\(A \setminus B = \{4, 9\}\text{.}\) It contains everything that is in \(A\) except anything that is also in \(B\text{.}\) We could also have written this set as \(A \cap \bar{B}\text{.}\)%
\item{}\(B \setminus A = \{3, 6, 10\}\text{.}\) It contains everything in \(B\) except anything that is also in \(A\text{.}\) Another way to write this is \(B \cap \bar{A}\text{.}\) Note that \(A \setminus B \ne B \setminus A\text{.}\)%
\end{enumerate}
%
\end{divisionsolution}%
\begin{divisionsolution}{0.3.4}{}{p:exercise:tpD}%
Find the least element of each of the following sets, if there is one.%
\begin{enumerate}[label=(\alph*)]
\item{}\(\{n \in \N \st n^2 - 3 \ge 2\}\text{.}\)%
\item{}\(\{n \in \N \st n^2 - 5 \in \N\}\text{.}\)%
\item{}\(\{n^2+1 \st n \in \N\}\text{.}\)%
\item{}\(\{n \in \N \st n = k^2 + 1 \text{ for some } k \in \N\}\text{.}\)%
\end{enumerate}
%
\par\smallskip%
\noindent\textbf{\blocktitlefont Answer 1}.\quad{}\(3\)%
\par\smallskip%
\noindent\textbf{\blocktitlefont Answer 2}.\quad{}\(3\)%
\par\smallskip%
\noindent\textbf{\blocktitlefont Answer 3}.\quad{}\(1\)%
\par\smallskip%
\noindent\textbf{\blocktitlefont Answer 4}.\quad{}\(1\)%
\par\smallskip%
\noindent\textbf{\blocktitlefont Solution}.\quad{}%
\begin{enumerate}[label=(\alph*)]
\item{}This is the set \(\{3, 4, 5, \ldots \}\) since we need each element to be a natural number whose square is at least three more than 2. Since \(3^2 - 3 = 6\) but \(2^2 - 3 = 1\) we see that the first such natural number is 3.%
\item{}We get the same set as we did in the previous part, and the smallest non-negative number for which \(n^2 - 5\) is a natural numbers is 3.%
\par
Note that if we didn't specify \(n \in \N\) then any integer less than \(-3\) would also be in the set, so there would not be a least element.%
\item{}This is the set \(\{1, 2, 5, 10, \ldots\}\text{,}\) namely the set of numbers that are the \emph{result} of squaring and adding 1 to a natural number. (\(0^2 + 1 = 1\text{,}\) \(1^2 + 1 = 2\text{,}\) \(2^2 + 1 = 5\) and so on.) Thus the least element of the set is 1.%
\item{}Now we are looking for natural numbers that are equal to taking some natural number, squaring it and adding 1. That is, \(\{1, 2, 5, 10, \ldots\}\text{,}\) the same set as the previous part. So again, the least element is 1.%
\end{enumerate}
%
\end{divisionsolution}%
\begin{divisionsolution}{0.3.5}{}{p:exercise:ZwM}%
Find the following cardinalities:%
\begin{enumerate}[label=(\alph*)]
\item{}\(|A|\) when \(A = \{4,5,6,\ldots,37\}\text{.}\)%
\item{}\(|A|\) when \(A = \{x \in \Z \st -2 \le x \le 100\}\text{.}\)%
\item{}\(|A \cap B|\) when \(A = \{x \in \N \st x \le 20\}\) and \(B = \{x \in \N \st x \mbox{ is prime} \}\text{.}\)%
\end{enumerate}
%
\par\smallskip%
\noindent\textbf{\blocktitlefont Answer 1}.\quad{}\(34\)%
\par\smallskip%
\noindent\textbf{\blocktitlefont Answer 2}.\quad{}\(103\)%
\par\smallskip%
\noindent\textbf{\blocktitlefont Answer 3}.\quad{}\(8\)%
\par\smallskip%
\noindent\textbf{\blocktitlefont Solution}.\quad{}%
\begin{enumerate}[label=(\alph*)]
\item{}34. Note that \(37-4 = 33\text{,}\) but this calculation would not include 4 itself.%
\item{}103. Again, you could compute this by \(100-(-2)+1\text{,}\) or simply count: 100 numbers from 1 through 100, plus -2, -1, and 0.%
\item{}8. There are 8 primes not greater than 20: \(\{2, 3, 5, 7, 11, 13, 17, 19\}\text{.}\)%
\end{enumerate}
%
\end{divisionsolution}%
\begin{divisionsolution}{0.3.6}{}{p:exercise:FDV}%
Find a set of largest possible size that is a subset of both \(\{1, 2, 3, 4, 5\}\) and \(\{2, 4, 6, 8,10\}\text{.}\)%
\par\smallskip%
\noindent\textbf{\blocktitlefont Answer}.\quad{}\(\left\{2,4\right\}\)%
\par\smallskip%
\noindent\textbf{\blocktitlefont Solution 1}.\quad{}\(\{2,4\}\text{.}\)%
\par\smallskip%
\noindent\textbf{\blocktitlefont Solution 2}.\quad{}If \(A\) is a subset of two sets, it must be a subset of their intersection. The largest such set will \emph{be} the intersection. In this case, that would be \(\{2,4\}\)%
\end{divisionsolution}%
\begin{divisionsolution}{0.3.7}{}{p:exercise:lLe}%
Find a set of smallest possible size that has both \(\{1,2,3,4,5\}\) and \(\{2,4,6,8,10\}\) as subsets.%
\par\smallskip%
\noindent\textbf{\blocktitlefont Answer}.\quad{}\(\left\{1,2,3,4,5,6,8,10\right\}\)%
\par\smallskip%
\noindent\textbf{\blocktitlefont Solution 1}.\quad{}\(\{1,2,3,4,5,6,8,10\}\)%
\par\smallskip%
\noindent\textbf{\blocktitlefont Solution 2}.\quad{}If two sets are both subsets of a set \(A\text{,}\) then the union of those two sets will also be a subset of \(A\text{.}\) The smallest possible size for \(A\) will then be the union itself. In this case, that is \(\{1,2,3,4,5,6,8,10\}\text{.}\)%
\end{divisionsolution}%
\begin{divisionsolution}{0.3.8}{}{p:exercise:RSn}%
Let \(A = \{n \in \N \st 20 \le n \lt 50\}\) and \(B = \{n \in \N \st 10 \lt n \le 30\}\text{.}\) Suppose \(C\) is a set such that \(C \subseteq A\) and \(C \subseteq B\text{.}\) What is the largest possible cardinality of \(C\text{?}\)%
\par\smallskip%
\noindent\textbf{\blocktitlefont Solution 1}.\quad{}11.%
\par\smallskip%
\noindent\textbf{\blocktitlefont Solution 2}.\quad{}For \(C \subseteq A\) and \(C \subseteq B\text{,}\) we must have that \(C \subseteq A \cap B\text{.}\) This means that the largest \(C\) could be would be to \emph{equal} \(A \cap B = \{n \in \N \st 20 \le n \le 30\}\text{,}\) which contains 11 elements.%
\end{divisionsolution}%
\begin{divisionsolution}{0.3.9}{}{p:exercise:xZw}%
Let \(A = \{1,2,3,4,5\}\) and \(B = \{2, 3, 4\}\text{.}\) How many sets \(C\) have the property that \(C \subseteq A\) and \(B \subseteq C\text{.}\)%
\par\smallskip%
\noindent\textbf{\blocktitlefont Hint}.\quad{}You should be able to write all of them out. Don't forget \(A\) and \(B\text{,}\) which are also candidates for \(C\text{.}\)%
\par\smallskip%
\noindent\textbf{\blocktitlefont Answer}.\quad{}\(4\)%
\par\smallskip%
\noindent\textbf{\blocktitlefont Solution}.\quad{}There will be exactly 4 such sets: \(\{2, 3, 4\}\text{,}\) \(\{1,2,3,4\}\text{,}\) \(\{2,3,4,5\}\) and \(\{1,2,3,4,5\}\text{.}\)%
\end{divisionsolution}%
\begin{divisionsolution}{0.3.10}{}{p:exercise:egF}%
Let \(A = \{x \in \N \st 3 \le x \le 13\}\), \(B = \{x \in \N \st x \mbox{ is even} \}\), and \(C = \{x \in \N \st x \mbox{ is odd} \}\).%
\begin{enumerate}[label=(\alph*)]
\item{}Find \(A \cap B\).%
\item{}Find \(A \cup B\).%
\item{}Find \(B \cap C\).%
\item{}Find \(B \cup C\).%
\end{enumerate}
%
\par\smallskip%
\noindent\textbf{\blocktitlefont Solution}.\quad{}%
\begin{enumerate}[label=(\alph*)]
\item{}\(A \cap B = \{4,6,8,10,12\} = \{x \in \N \st (3 \le x \le 13) \wedge x \mbox{ is even}\).%
\item{}\(A \cup B = \{3, 4, 5, \ldots, 12, 13\} = \{x \in \N \st (3 \le x \le 13) \vee x \mbox{ is even} \}\).%
\item{}\(B \cap C = \emptyset\).%
\item{}\(B \cup C = \N\).%
\end{enumerate}
%
\end{divisionsolution}%
\begin{divisionsolution}{0.3.11}{}{p:exercise:KnO}%
Find an example of sets \(A\) and \(B\) such that \(A\cap B = \{3, 5\}\) and \(A \cup B = \{2, 3, 5, 7, 8\}\).%
\par\smallskip%
\noindent\textbf{\blocktitlefont Solution}.\quad{}For example, \(A = \{2,3,5,7,8\}\) and \(B = \{3,5\}\).%
\end{divisionsolution}%
\begin{divisionsolution}{0.3.12}{}{p:exercise:quX}%
Find an example of sets \(A\) and \(B\) such that \(A \subseteq B\) and \(A \in B\).%
\par\smallskip%
\noindent\textbf{\blocktitlefont Solution}.\quad{}For example, \(A = \{1,2,3\}\) and \(B = \{1,2,3,4,5,\{1,2,3\}\}\)%
\end{divisionsolution}%
\begin{divisionsolution}{0.3.13}{}{p:exercise:WCg}%
Recall \(\Z = \{\ldots,-2,-1,0, 1,2,\ldots\}\) (the integers). Let \(\Z^+ = \{1, 2, 3, \ldots\}\) be the positive integers. Let \(2\Z\) be the even integers, \(3\Z\) be the multiples of 3, and so on.%
\begin{enumerate}[label=(\alph*)]
\item{}Is \(\Z^+ \subseteq 2\Z\)? Explain.%
\item{}Is \(2\Z \subseteq \Z^+\)? Explain.%
\item{}Find \(2\Z \cap 3\Z\). Describe the set in words, and using set notation.%
\item{}Express \(\{x \in \Z \st \exists y\in \Z (x = 2y \vee x = 3y)\}\) as a union or intersection of two sets already described in this problem.%
\end{enumerate}
%
\par\smallskip%
\noindent\textbf{\blocktitlefont Solution}.\quad{}%
\begin{enumerate}[label=(\alph*)]
\item{}No.%
\item{}No.%
\item{}\(2\Z \cap 3\Z\) is the set of all integers which are multiples of both 2 and 3 (so multiples of 6). Therefore \(2\Z \cap 3\Z = \{x \in \Z \st \exists y\in \Z(x = 6y)\}\).%
\item{}\(2\Z \cup 3\Z\).%
\end{enumerate}
%
\end{divisionsolution}%
\begin{divisionsolution}{0.3.14}{}{p:exercise:CJp}%
Let \(A_2\) be the set of all multiples of 2 except for \(2\). Let \(A_3\) be the set of all multiples of 3 except for 3. And so on, so that \(A_n\) is the set of all multiples of \(n\) except for \(n\), for any \(n \ge 2\). Describe (in words) the set \(\bar{A_2 \cup A_3 \cup A_4 \cup \cdots}\).%
\par\smallskip%
\noindent\textbf{\blocktitlefont Hint}.\quad{}It might help to think about what the union \(A_2 \cup A_3\) is first.  Then think about what numbers are \emph{not} in that union.  What will happen when you also include \(A_5\)?%
\par\smallskip%
\noindent\textbf{\blocktitlefont Solution}.\quad{}The set of primes.%
\end{divisionsolution}%
\begin{divisionsolution}{0.3.15}{}{p:exercise:iQy}%
Draw a Venn diagram to represent each of the following:%
\begin{enumerate}[label=(\alph*)]
\item{}\(\displaystyle A \cup \bar B\)%
\item{}\(\displaystyle \bar{(A \cup B)}\)%
\item{}\(\displaystyle A \cap (B \cup C)\)%
\item{}\(\displaystyle (A \cap B) \cup C\)%
\item{}\(\displaystyle \bar A \cap B \cap \bar C\)%
\item{}\(\displaystyle (A \cup B) \setminus C\)%
\end{enumerate}
%
\par\smallskip%
\noindent\textbf{\blocktitlefont Solution}.\quad{}%
\begin{multicols}{2}
\begin{enumerate}[label=(\alph*)]
\item{}\(A \cup \bar B\):%
\begin{sidebyside}{1}{0.4}{0.4}{0}%
\begin{sbspanel}{0.2}%
\resizebox{\linewidth}{!}{%
\begin{tikzpicture}[fill=gray!50, scale=.5]
    \fill \circleA;
      \begin{scope}
      \clip \circleB \twosetbox;
      \fill \twosetbox;
      \end{scope}
      \draw[thick] \circleA \circleAlabel \circleB \circleBlabel \twosetbox;
    \end{tikzpicture}
}%
\end{sbspanel}%
\end{sidebyside}%
\item{}\(\bar{(A \cup B)}\):%
\begin{sidebyside}{1}{0.4}{0.4}{0}%
\begin{sbspanel}{0.2}%
\resizebox{\linewidth}{!}{%
              \begin{tikzpicture}[scale=.5, fill=gray!50]
  \fill \twosetbox;
  \fill[white] \circleA \circleB;
  \draw[thick] \circleA \circleAlabel \circleB \circleBlabel \twosetbox;
\end{tikzpicture}
}%
\end{sbspanel}%
\end{sidebyside}%
\item{}\(A \cap (B \cup C)\):%
\begin{sidebyside}{1}{0.4}{0.4}{0}%
\begin{sbspanel}{0.2}%
\resizebox{\linewidth}{!}{%
              \begin{tikzpicture}[fill=gray!50, scale=.5]
\begin{scope}
  \clip \circleA;
  \fill \circleB \circleC;
\end{scope}
\draw[thick] \circleA \circleAlabel \circleB \circleBlabel \circleC \circleClabel \threesetbox;
\end{tikzpicture}
}%
\end{sbspanel}%
\end{sidebyside}%
\item{}\((A \cap B) \cup C\):%
\begin{sidebyside}{1}{0.4}{0.4}{0}%
\begin{sbspanel}{0.2}%
\resizebox{\linewidth}{!}{%
\begin{tikzpicture}[fill=gray!50, scale=.5]
  \begin{scope}
    \clip \circleA;
    \fill \circleB;
  \end{scope}
  \fill \circleC;
  \draw[thick] \circleA \circleAlabel \circleB \circleBlabel \circleC \circleClabel \threesetbox;
  \end{tikzpicture}
}%
\end{sbspanel}%
\end{sidebyside}%
\item{}\(\bar A \cap B \cap \bar C\):%
\begin{sidebyside}{1}{0.4}{0.4}{0}%
\begin{sbspanel}{0.2}%
\resizebox{\linewidth}{!}{%
\begin{tikzpicture}[fill=gray!50, scale=.5]
  \fill \circleB;
  \begin{scope}
    \clip \circleB;
    \fill[white] \circleA \circleC;
  \end{scope}

  \draw[thick] \circleA \circleAlabel \circleB \circleBlabel \circleC \circleClabel \threesetbox;
  \end{tikzpicture}
}%
\end{sbspanel}%
\end{sidebyside}%
\item{}\((A \cup B) \setminus C\):%
\begin{sidebyside}{1}{0.4}{0.4}{0}%
\begin{sbspanel}{0.2}%
\resizebox{\linewidth}{!}{%
\begin{tikzpicture}[fill=gray!50, scale=.5]
\fill \circleA;
\fill \circleB;
\fill[white] \circleC;
\draw[thick] \circleA \circleAlabel \circleB \circleBlabel \circleC \circleClabel \threesetbox;
\end{tikzpicture}
}%
\end{sbspanel}%
\end{sidebyside}%
\end{enumerate}
\end{multicols}
%
\end{divisionsolution}%
\begin{divisionsolution}{0.3.16}{}{p:exercise:OXH}%
Describe a set in terms of \(A\) and \(B\) (using set notation) which has the following Venn diagram:%
\begin{sidebyside}{1}{0.4}{0.4}{0}%
\begin{sbspanel}{0.2}%
\resizebox{\linewidth}{!}{%
\begin{tikzpicture}[fill=gray!50, scale=0.5]
\scope
\clip (-2,-2) rectangle (2,2)
    (1,0) circle (1);
\fill (0,0) circle (1);
\endscope
\scope
\clip (-2,-2) rectangle (2,2)
    (0,0) circle (1);
\fill (1,0) circle (1);
\endscope
\draw[thick] (0,0) circle (1) (-1,.7)  node [text=black,above] {\(A\)}
    (1,0) circle (1) (2,.7)  node [text=black,above] {\(B\)}
    (-1.5,-1.5) rectangle (2.5,1.5);
\end{tikzpicture}
}%
\end{sbspanel}%
\end{sidebyside}%
\par\smallskip%
\noindent\textbf{\blocktitlefont Solution}.\quad{}For example, \(A \cup B \cap \bar{(A \cap B)}\). Note that \(\bar{A \cap B}\) would almost work, but it also contains the area outside of both circles.%
\end{divisionsolution}%
\begin{divisionsolution}{0.3.17}{}{p:exercise:veQ}%
Let \(A = \{a, b, c, d\}\). Find \(\pow(A)\).%
\par\smallskip%
\noindent\textbf{\blocktitlefont Hint}.\quad{}We are looking for a set containing 16 sets.%
\par\smallskip%
\noindent\textbf{\blocktitlefont Solution}.\quad{}%
\begin{align*}
\pow(A) = \{\amp \emptyset, \{a\}, \{b\}, \{c\}, \{d\}, \{a,b\}, \{a,c\}, \{a,d\}, \{b,c\}, \{b,d\},\\
\amp \{c,d\} \{a,b,c\}, \{a,b,d\}, \{a,c,d\}, \{b,c,d\}, \{a,b,c,d\}\}\text{.}
\end{align*}
%
\end{divisionsolution}%
\begin{divisionsolution}{0.3.18}{}{p:exercise:blZ}%
Let \(A = \{1,2,3,4,5\}\text{,}\) \(B = \{3,4,5,6,7\}\text{,}\) and \(C = \{2,3,5\}\text{.}\)%
\begin{enumerate}[label=(\alph*)]
\item{}Find \(A \cap B\text{.}\)%
\item{}Find \(A \cup B\text{.}\)%
\item{}Find \(A \setminus B\text{.}\)%
\item{}Find \(A \cap \overline{(B \cup C)}\text{.}\)%
\end{enumerate}
%
\par\smallskip%
\noindent\textbf{\blocktitlefont Answer 1}.\quad{}\(\left\{3,4,5\right\}\)%
\par\smallskip%
\noindent\textbf{\blocktitlefont Answer 2}.\quad{}\(\left\{1,2,3,4,5,6,7\right\}\)%
\par\smallskip%
\noindent\textbf{\blocktitlefont Answer 3}.\quad{}\(\left\{1,2\right\}\)%
\par\smallskip%
\noindent\textbf{\blocktitlefont Answer 4}.\quad{}\(\left\{1\right\}\)%
\par\smallskip%
\noindent\textbf{\blocktitlefont Solution}.\quad{}%
\begin{enumerate}[label=(\alph*)]
\item{}\(A \cap B = \{3,4,5\}\text{.}\)%
\item{}\(A \cup B = \{1,2,3,4,5,6,7\}\text{.}\)%
\item{}\(A \setminus B = \{1,2\}\text{.}\)%
\item{}\(A \cap \overline{(B \cup C)} = \{1\}\text{.}\)%
\end{enumerate}
%
\end{divisionsolution}%
\begin{divisionsolution}{0.3.19}{}{p:exercise:Hti}%
Let \(A = \{1,2,3,4,5,6\}\). Find all sets \(B \in \pow(A)\) which have the property \(\{2,3,5\} \subseteq B\).%
\par\smallskip%
\noindent\textbf{\blocktitlefont Solution}.\quad{}\(\{2,3,5\}\), \(\{1,2,3,5\}\), \(\{2,3,4,5\}\), \(\{2,3,5,6\}\), \(\{1,2,3,4,5\}\), \(\{1,2,3,5,6\}\), \(\{2,3,4,5,6\}\), and \(\{1,2,3,4,5,6\}\).%
\end{divisionsolution}%
\begin{divisionsolution}{0.3.20}{}{p:exercise:nAr}%
Find an example of sets \(A\) and \(B\) such that \(|A| = 4\), \(|B| = 5\), and \(|A \cup B| = 9\).%
\par\smallskip%
\noindent\textbf{\blocktitlefont Solution}.\quad{}For example, \(A = \{1,2,3,4\}\) and \(B = \{5,6,7,8,9\}\) gives \(A \cup B = \{1,2,3,4,5,6,7,8,9\}\).%
\end{divisionsolution}%
\begin{divisionsolution}{0.3.21}{}{p:exercise:THA}%
Find an example of sets \(A\) and \(B\) such that \(|A| = 3\), \(|B| = 4\), and \(|A \cup B| = 5\).%
\par\smallskip%
\noindent\textbf{\blocktitlefont Solution}.\quad{}For example, \(A = \{1,2,3\}\) and \(B = \{2,3,4,5\}\) gives \(A\cup B = \{1,2,3,4,5\}\).%
\end{divisionsolution}%
\begin{divisionsolution}{0.3.22}{}{p:exercise:zOJ}%
Are there sets \(A\) and \(B\) such that \(|A| = |B|\), \(|A\cup B| = 10\), and \(|A\cap B| = 5\)? Explain.%
\par\smallskip%
\noindent\textbf{\blocktitlefont Solution}.\quad{}No. There must be 5 elements in common to both sets. Since there are 10 distinct elements all together in \(A\) and \(B\), this means that between \(A\) and \(B\), there must be 5 elements which they do not have in common (some in \(A\) but not in \(B\), some in \(B\) but not in \(A\)). But to have \(|A| = |B|\), we would need to exclude the same number of elements from both sets. Since 5 is odd, we would need to exclude 2.5 elements from each set making \(|A| = |B| = 7.5\) which is impossible.%
\end{divisionsolution}%
\begin{divisionsolution}{0.3.23}{}{p:exercise:fVS}%
Let \(A = \{x \in \N \st 4 \le x \lt 12\}\) and \(B = \{x \in \N \st x \text{ is even}\}\text{.}\)%
\begin{enumerate}[label=(\alph*)]
\item{}Find \(A \cap B\text{.}\)%
\item{}Find \(A \setminus B\text{.}\)%
\end{enumerate}
%
\par\smallskip%
\noindent\textbf{\blocktitlefont Answer 1}.\quad{}\(\left\{4,6,8,10\right\}\)%
\par\smallskip%
\noindent\textbf{\blocktitlefont Answer 2}.\quad{}\(\left\{5,7,9,11\right\}\)%
\par\smallskip%
\noindent\textbf{\blocktitlefont Solution}.\quad{}%
\begin{enumerate}[label=(\alph*)]
\item{}\(A \cap B\) will be the set of natural numbers that are both at least 4 and less than 12, and even. That is, \(A \cap B = \{x \in \N \st 4\le x \lt 12 \wedge x \text{ is even}\} = \{4, 6, 8, 10\}\text{.}\)%
\item{}\(A \setminus B\) is the set of all elements that are in \(A\) but not \(B\text{.}\) So this is \(\{x \in \N \st 4 \le x \lt 12 \wedge x \text{ is odd}\} = \{5,7,9,11\}\text{.}\)%
\par
Note this is the same set as \(A \cap \overline{B}\text{.}\)%
\end{enumerate}
%
\end{divisionsolution}%
\begin{divisionsolution}{0.3.24}{}{p:exercise:Mdb}%
Let \(X = \{n \in \N \st 10 \le n \lt 20\}\). Find examples of sets with the properties below and very briefly explain why your examples work.%
\begin{enumerate}[label=(\alph*)]
\item{}A set \(A \subseteq \N\) with \(|A| = 10\) such that \(X \setminus A = \{10, 12, 14\}\).%
\item{}A set \(B \in \pow(X)\) with \(|B| = 5\).%
\item{}A set \(C \subseteq \pow(X)\) with \(|C| = 5\).%
\item{}A set \(D \subseteq X \times X\) with \(|D| = 5\)%
\item{}A set \(E \subseteq X\) such that \(|E| \in E\).%
\end{enumerate}
%
\par\smallskip%
\noindent\textbf{\blocktitlefont Solution}.\quad{}%
\begin{enumerate}[label=(\alph*)]
\item{}For example, \(A = \{11, 13, 15, 16, 17, 18, 19, 20, 21, 22\}\). A correct example must not contain \(10, 12, 14\) and must contain \(11, 13, 15, 16, 17, 18, 19\) plus three other elements.%
\item{}For example, \(B = \{10, 11, 12, 13, 14\}\). Any 5 element subset of \(X\) will work.%
\item{}For example, \(C = \{\emptyset, \{10\}, \{10, 13\}, \{15\}, \{11, 12, 13, 14, 15\}\}\). We need \(C\) to be a set of 5 subsets of \(A\).%
\item{}For example, \(D = \{(10,11), (15,12), (13,13), (19,10), (10,19)\}\). Each of the five elements should be ordered pairs where each coordinate is an element of \(X\).%
\item{}We must have \(E = X\) here. The smallest number that can be in \(E\) is 10, which means that the smallest size \(E\) can have is 10. But the largest size \(E\) can have is also 10 because it must be a subset of a set of size 10. There is only one subset of \(X\) containing 10 elements, so \(E = X\).%
\end{enumerate}
%
\end{divisionsolution}%
\begin{divisionsolution}{0.3.25}{}{p:exercise:skk}%
Let \(A\), \(B\) and \(C\) be sets.%
\begin{enumerate}[label=(\alph*)]
\item{}Suppose that \(A \subseteq B\) and \(B \subseteq C\). Does this mean that \(A \subseteq C\)? Prove your answer. Hint: to prove that \(A \subseteq C\) you must prove the implication, ``for all \(x\), if \(x \in A\) then \(x \in C\).''%
\item{}Suppose that \(A \in B\) and \(B \in C\). Does this mean that \(A \in C\)? Give an example to prove that this does NOT always happen (and explain why your example works). You should be able to give an example where \(|A| = |B| = |C| = 2\).%
\end{enumerate}
%
\par\smallskip%
\noindent\textbf{\blocktitlefont Solution}.\quad{}%
\begin{enumerate}[label=(\alph*)]
\item{}Yes it does. We can see this using a Venn diagram -{}-{} the circle \(A\) is completely contained in the circle \(B\), and the circle \(B\) is completely contained in the circle \(C\). So the circle \(A\) is completely contained in the circle \(C\).%
\par
Here is a proof: Suppose \(A \subseteq B\) and \(B \subseteq C\). Then take any \(x \in A\). Since \(A \subseteq B\), we have that \(x \in B\) as well - that is, everything in \(A\) is also in \(B\), so this works for the arbitrary \(x\) we chose. Now that we know that \(x \in B\), we can conclude \(x \in C\), since \(B \subseteq C\) - everything in \(B\) is also in \(C\). Since \(x\) was an arbitrary element of \(A\), which we showed was also in \(C\), we have \(A \subseteq C\) - everything in \(A\) is also in \(C\).%
\item{}Let \(A = \{1,2\}\), \(B = \{0, \{1,2\}\} = \{0, A\}\) and \(C = \{5, \{0, A\}\} = \{5, B\}\). Note that \(A \in B\) and \(B \in C\), but neither of the elements of \(C\) is the set \(A\), so \(A \notin C\).%
\end{enumerate}
%
\end{divisionsolution}%
\begin{divisionsolution}{0.3.26}{}{p:exercise:Yrt}%
In a regular deck of playing cards there are 26 red cards and 12 face cards. Explain, using sets and what you have learned about cardinalities, why there are only 32 cards which are either red or a face card.%
\par\smallskip%
\noindent\textbf{\blocktitlefont Solution}.\quad{}If \(R\) is the set of red cards and \(F\) is the set of face cards, we want to find \(|R \cup F|\). This is not simply \(|R| + |F|\) because there are 6 cards which are both red and a face card; \(|R \cap F| = 6\). We find \(|R \cup F| = 32\).%
\end{divisionsolution}%
\begin{divisionsolution}{0.3.27}{}{p:exercise:EyC}%
Find an example of a set \(A\) with \(|A| = 3\) which contains only other sets and has the following property: for all sets \(B \in A\), we also have \(B \subseteq A\). Explain why your example works. (FYI: sets that have this property are called \terminology{transitive}.)%
\par\smallskip%
\noindent\textbf{\blocktitlefont Solution}.\quad{}Take \(A = \{\emptyset, \{\emptyset\}, \{\emptyset, \{\emptyset\}\}\}\). There are three things to check. First, the element \(\emptyset \in A\) is also a subset, since the empty set is a subset of every set. Second, \(\{\emptyset\}\in A\). Is \(\{\emptyset\} \subseteq A\)? Yes, because \(\{\emptyset\}\) contains one element, namely \(\emptyset\) which is also an element of \(A\). Finally, consider \(\{\emptyset, \{\emptyset\}\}\in A\). This two is a subset as both of its element are exactly the other two element of \(A\). Thus \(A\) is transitive. Notice also that each of the elements in \(A\) are also transitive sets. This happens to be the only set of size three with that property.%
\end{divisionsolution}%
\begin{divisionsolution}{0.3.28}{}{p:exercise:kFL}%
Consider the sets \(A\) and \(B\), where \(A = \{3, |B|\}\) and \(B = \{1, |A|, |B|\}\).  What are the sets?%
\par\smallskip%
\noindent\textbf{\blocktitlefont Solution}.\quad{}We need to be a little careful here.  If \(B\) contains 3 elements, then \(A\) contains just the number 3 (listed twice).  So that would make \(|A| = 1\), which would make \(B = \{1, 3\}\), which only has 2 elements.  Thus \(|B| \ne 3\).  This means that \(|A| = 2\), so \(B\) contains at least the elements 1 and 2.  Since \(\card{B} \ne 3\), we must have \(\card{B} = 2\), which agrees with the definition of \(B\).%
\par
Therefore it must be that \(A = \{2,3\}\) and \(B = \{1, 2\}\)%
\end{divisionsolution}%
\begin{divisionsolution}{0.3.29}{}{p:exercise:QMU}%
Explain why there is no set \(A\) which satisfies \(A = \{2, \card{A}\}\).%
\par\smallskip%
\noindent\textbf{\blocktitlefont Hint}.\quad{}It looks like you should be able to define the set \(A\) like this.  But consider the two possible values for \(\card{A}\).%
\par\smallskip%
\noindent\textbf{\blocktitlefont Solution}.\quad{}We have either \(\card{A} = 1\) or \(\card{A} = 2\).  In the first case, then \(A = \{1,2\}\), which has cardinality 2.  In the second case, then \(A = \{2\}\), which has cardinality 1.  Both cases are thus impossible.%
\end{divisionsolution}%
\begin{divisionsolution}{0.3.30}{}{p:exercise:wUd}%
Find all sets \(A\), \(B\), and \(C\) which satisfy the following.%
\begin{align*}
A = \amp \{1, \card{B}, \card{C}\}\\
B = \amp \{2, \card{A}, \card{C}\}\\
C = \amp \{1, 2, \card{A}, \card{B}\}\text{.}
\end{align*}
%
\par\smallskip%
\noindent\textbf{\blocktitlefont Solution}.\quad{}There are two possible solutions.%
\par
If \(\card{A} = 3\), then \(B = \{2, 3, \card{C}\}\), and \(C = \{1, 2, 3, \card{B}\}\).  Now \(|B|\) is either 2 or 3, but either way, we see that \(|C| = 3\), so in fact \(B = \{2,3\}\) making \(|B| = 2\).  This gives \(A = \{1, 2, 3\}\), consistent with our assumption.%
\par
What if \(\card{A} = 2\)?  Then \(B = \{2, \card{C}\}\) and \(C = \{1, 2, \card{B}\}\).  Since this means \(\card{B} \le 2\), we have \(\card{C} = 2\), which puts \(B = \{2\}\) so \(\card{B} = 1\), making \(A = \{1, 2\}\). We have \(B = \{2\}\) and \(C = \{1, 2\}\).  This is a solution, but are there any others?%
\par
No.  The only other possibility is \(\card{A} = 1\), but this would mean that \(\card{B} = \card{C} = 1\), and \(C\) already has two elements, so this is impossible.%
\end{divisionsolution}%
\section*{0.4 Functions}
\addcontentsline{toc}{section}{0.4 Functions}
\sectionmark{0.4 Functions}
\subsection*{Exercises}
\addcontentsline{toc}{subsection}{Exercises}
\begin{divisionsolution}{0.4.1}{}{p:exercise:blM}%
Let \(A = \{1,2,\ldots, 10\}\text{.}\) How many subsets of \(A\) contain exactly one element (i.e., how many singleton subsets are there)?%
\par
How many doubleton subsets (containing exactly two elements) are there?%
\par\smallskip%
\noindent\textbf{\blocktitlefont Hint}.\quad{}Write these out, or at least start to and look for a pattern.%
\par\smallskip%
\noindent\textbf{\blocktitlefont Answer 1}.\quad{}\(10\)%
\par\smallskip%
\noindent\textbf{\blocktitlefont Answer 2}.\quad{}\(45\)%
\par\smallskip%
\noindent\textbf{\blocktitlefont Solution}.\quad{}There are 10 singletons. There are 45 doubletons: nine that include 1, eight that include 2 (but not 1), 7 that include 3 (but not 1 or 2) and so on. \(9+8+7+\cdots+2+1 = 45\text{;}\) ).%
\end{divisionsolution}%
\begin{divisionsolution}{0.4.2}{}{p:exercise:HsV}%
Let \(A = \{2, 4, 6, 8\}\text{.}\) Suppose \(B\) is a set with \(|B| = 5\text{.}\)%
\begin{enumerate}[label=(\alph*)]
\item{}What are the smallest and largest possible values of \(|A \cup B|\text{?}\) Explain.%
\item{}What are the smallest and largest possible values of \(|A \cap B|\text{?}\) Explain.%
\item{}What are the smallest and largest possible values of \(|A \times B|\text{?}\) Explain.%
\end{enumerate}
%
\par\smallskip%
\noindent\textbf{\blocktitlefont Solution}.\quad{}%
\begin{enumerate}[label=(\alph*)]
\item{}\(5 \le |A\cup B| \le 9\text{.}\) This is because \(A \cup B\) contains everything that is either in \(A\) or in \(B\text{,}\) or in both (but counted just once). If there is no overlap between \(A\) and \(B\text{,}\) then all 5 elements in \(B\) are counted in addition to those in \(A\text{,}\) for a total of 9. On the other hand, if there is as much overlap as possible (i.e., \(A \subseteq B\) ) then there is only one more element in \(B\) that is not already in \(A\text{,}\) so the union will contain just the 5 elements in \(B\) (4 of which are also in \(A\) ).%
\item{}\(0 \le |A \cap B| \le 4\text{.}\) This is because \(A \cap B\) contains everything that is both in \(A\) and in \(B\text{.}\) There could be nothing in both sets, in which case the intersection would be the empty set, which has cardinality zero. The most overlap that could occur is if everything in \(A\) is also in \(B\text{,}\) in which case all 4 elements of \(A\) would be in \(B\) and thus in the intersection.%
\item{}\(|A \times B| = 20\) always. It doesn't matter what \(B\) is, just that it contains 5 elements. \(A \times B\) has all the pairs in which the first element in the pair comes from \(A\) and the second element comes from \(B\text{.}\) There will be 5 pairs with first element 2, another 5 pairs with first element 4, another 5 with first element 6, and another 5 with first element 8, for a total of 20 pairs.%
\end{enumerate}
%
\end{divisionsolution}%
\begin{divisionsolution}{0.4.3}{}{p:exercise:nAe}%
Consider the function \(f:\{1,2,3,4\} \to \{1,2,3,4\}\) given by%
\par
%
\begin{equation*}
f(n) = \twoline{1 \amp 2 \amp 3 \amp 4}{4 \amp 1 \amp 3 \amp 4}
\text{.}
\end{equation*}
%
\par
%
\begin{enumerate}[label=(\alph*)]
\item{}Find \(f(1)\text{.}\)%
\item{}Find an element \(n\) in the domain such that \(f(n) = 1\text{.}\)%
\item{}Find an element \(n\) of the domain such that \(f(n) = n\text{.}\)%
\item{}Find an element of the codomain that is not in the range.%
\end{enumerate}
%
\par\smallskip%
\noindent\textbf{\blocktitlefont Answer 1}.\quad{}\(4\)%
\par\smallskip%
\noindent\textbf{\blocktitlefont Answer 2}.\quad{}\(2\)%
\par\smallskip%
\noindent\textbf{\blocktitlefont Answer 3}.\quad{}\(4\)%
\par\smallskip%
\noindent\textbf{\blocktitlefont Answer 4}.\quad{}\(2\)%
\par\smallskip%
\noindent\textbf{\blocktitlefont Solution}.\quad{}%
\begin{enumerate}[label=(\alph*)]
\item{}\(f(1) = 4\text{,}\) since \(4\) is the number below 1 in the two-line notation.%
\item{}Such an \(n\) is \(n= 2\text{,}\) since \(f(2) = 1\text{.}\) Note that 2 is above a 1 in the notation.%
\item{}\(n = 4\) has this property. We say that 4 is a fixed point of \(f\text{.}\) Not all functions have such a point.%
\item{}Such an element is 2 (in fact, that is the only element in the codomain that is not in the range). In other words, 2 is not the image of any element under \(f\text{;}\) nothing is sent to 2.%
\end{enumerate}
%
\end{divisionsolution}%
\begin{divisionsolution}{0.4.4}{}{p:exercise:THn}%
The following functions all have \(\{1,2,3,4,5\}\) as both their domain and codomain. For each, determine whether it is (only) injective, (only) surjective, bijective, or neither injective nor surjective.%
\begin{enumerate}[label=(\alph*)]
\item{}\(f = \twoline{1 \amp 2 \amp 3 \amp 4 \amp 5}{3 \amp 3 \amp 3 \amp 3 \amp 3}\text{.}\) \quad(\begin{itemize*}[label=$\square$,leftmargin=3em,itemjoin=\hspace{1em}]
\item{}Injective%

\item{}Surjective%

\item{}Bijective%

\item{}Neither%

\end{itemize*})\quad
%
\item{}\(f = \twoline{1 \amp 2 \amp 3 \amp 4 \amp 5}{2 \amp 3 \amp 1 \amp 5 \amp 4}\text{.}\) \quad(\begin{itemize*}[label=$\square$,leftmargin=3em,itemjoin=\hspace{1em}]
\item{}Injective%

\item{}Surjective%

\item{}Bijective%

\item{}Neither%

\end{itemize*})\quad
%
\item{}\(f(x) = 6 - x\text{.}\) \quad(\begin{itemize*}[label=$\square$,leftmargin=3em,itemjoin=\hspace{1em}]
\item{}Injective%

\item{}Surjective%

\item{}Bijective%

\item{}Neither%

\end{itemize*})\quad
%
\item{}\(f(x) = \begin{cases} x/2 \amp \text{ if } x \text{ is even} \\ (x+1)/2 \amp \text{ if } x \text{ is odd}\end{cases}\text{.}\) \quad(\begin{itemize*}[label=$\square$,leftmargin=3em,itemjoin=\hspace{1em}]
\item{}Injective%

\item{}Surjective%

\item{}Bijective%

\item{}Neither%

\end{itemize*})\quad
%
\end{enumerate}
%
\par\smallskip%
\noindent\textbf{\blocktitlefont Answer 1}.\quad{}\(\text{Neither}\)%
\par\smallskip%
\noindent\textbf{\blocktitlefont Answer 2}.\quad{}\(\text{Bijective}\)%
\par\smallskip%
\noindent\textbf{\blocktitlefont Answer 3}.\quad{}\(\text{Bijective}\)%
\par\smallskip%
\noindent\textbf{\blocktitlefont Answer 4}.\quad{}\(\text{Neither}\)%
\par\smallskip%
\noindent\textbf{\blocktitlefont Solution}.\quad{}%
\begin{enumerate}[label=(\alph*)]
\item{}This is neither injective nor surjective. It is not injective because more than one element from the domain has 3 as its image. It is not surjective because there are elements of the codomain (1, 2, 4, and 5) that are not images of anything from the domain.%
\item{}This is a bijection. Every element in the codomain is the image of \emph{exactly} one element of the domain.%
\item{}This is a bijection. Note that we can write this function in two line notation as \(f = \twoline{1 \amp 2 \amp 3 \amp 4 \amp 5}{5 \amp 4 \amp 3 \amp 2 \amp 1}\text{.}\)%
\item{}In two line notation, this function is \(f = \twoline{1 \amp 2 \amp 3 \amp 4 \amp 5}{1 \amp 1 \amp 2 \amp 2 \amp 3}\text{.}\) From this we can quickly see it is neither injective (for example, 1 is the image of both 1 and 2) nor surjective (for example, 4 is not the image of anything).%
\end{enumerate}
%
\end{divisionsolution}%
\begin{divisionsolution}{0.4.5}{}{p:exercise:zOw}%
The following functions all have domain \(\{1,2,3,4,5\}\) and codomain \(\{1,2,3\}\text{.}\) For each, determine whether it is (only) injective, (only) surjective, bijective, or neither injective nor surjective.%
\begin{enumerate}[label=(\alph*)]
\item{}\(f = \twoline{1 \amp 2 \amp 3 \amp 4 \amp 5}{1 \amp 2 \amp 1 \amp 2 \amp 1}\text{.}\) \quad(\begin{itemize*}[label=$\square$,leftmargin=3em,itemjoin=\hspace{1em}]
\item{}Injective%

\item{}Surjective%

\item{}Bijective%

\item{}Neither%

\end{itemize*})\quad
%
\item{}\(f = \twoline{1 \amp 2 \amp 3 \amp 4 \amp 5}{1 \amp 2 \amp 3 \amp 1 \amp 2}\text{.}\) \quad(\begin{itemize*}[label=$\square$,leftmargin=3em,itemjoin=\hspace{1em}]
\item{}Injective%

\item{}Surjective%

\item{}Bijective%

\item{}Neither%

\end{itemize*})\quad
%
\item{}\(f(x) = \begin{cases} x \amp \text{ if } x \le 3 \\ x-3 \amp \text{ if } x \gt 3\end{cases}\text{.}\) \quad(\begin{itemize*}[label=$\square$,leftmargin=3em,itemjoin=\hspace{1em}]
\item{}Injective%

\item{}Surjective%

\item{}Bijective%

\item{}Neither%

\end{itemize*})\quad
%
\end{enumerate}
%
\par\smallskip%
\noindent\textbf{\blocktitlefont Answer 1}.\quad{}\(\text{Neither}\)%
\par\smallskip%
\noindent\textbf{\blocktitlefont Answer 2}.\quad{}\(\text{Surjective}\)%
\par\smallskip%
\noindent\textbf{\blocktitlefont Answer 3}.\quad{}\(\text{Surjective}\)%
\par\smallskip%
\noindent\textbf{\blocktitlefont Solution}.\quad{}%
\begin{enumerate}[label=(\alph*)]
\item{}This is neither injective nor surjective. It is not injective because more than one element from the domain has 1 as its image. It is not surjective because 3 is an element of the codomain but not the range.%
\item{}This is a surjective (but not injective). Every element of the domain is in the range, but 1 and 2 are images of more than one element.%
\item{}This is surjective (but not injective). Note that we can write this function in two line notation as \(f = \twoline{1 \amp 2 \amp 3 \amp 4 \amp 5}{1 \amp 2 \amp 3 \amp 1 \amp 2}\text{,}\) the same function as the previous part.%
\end{enumerate}
%
\end{divisionsolution}%
\begin{divisionsolution}{0.4.6}{}{p:exercise:fVF}%
The following functions all have domain \(\{1,2,3,4\}\) and codomain \(\{1,2,3,4,5\}\text{.}\) For each, determine whether it is (only) injective, (only) surjective, bijective, or neither injective nor surjective.%
\begin{enumerate}[label=(\alph*)]
\item{}\(f = \twoline{1 \amp 2 \amp 3 \amp 4}{1 \amp 2 \amp 5 \amp 4}\text{.}\) \quad(\begin{itemize*}[label=$\square$,leftmargin=3em,itemjoin=\hspace{1em}]
\item{}Injective%

\item{}Surjective%

\item{}Bijective%

\item{}Neither%

\end{itemize*})\quad
%
\item{}\(f = \twoline{1 \amp 2 \amp 3 \amp 4}{1 \amp 2 \amp 3 \amp 2}\text{.}\) \quad(\begin{itemize*}[label=$\square$,leftmargin=3em,itemjoin=\hspace{1em}]
\item{}Injective%

\item{}Surjective%

\item{}Bijective%

\item{}Neither%

\end{itemize*})\quad
%
\item{}\(f(x)\) gives the number of letters in the English word for the number \(x\text{.}\) For example, \(f(1) = 3\) since ``one'' contains three letters. \quad(\begin{itemize*}[label=$\square$,leftmargin=3em,itemjoin=\hspace{1em}]
\item{}Injective%

\item{}Surjective%

\item{}Bijective%

\item{}Neither%

\end{itemize*})\quad
%
\end{enumerate}
%
\par\smallskip%
\noindent\textbf{\blocktitlefont Answer 1}.\quad{}\(\text{Injective}\)%
\par\smallskip%
\noindent\textbf{\blocktitlefont Answer 2}.\quad{}\(\text{Neither}\)%
\par\smallskip%
\noindent\textbf{\blocktitlefont Answer 3}.\quad{}\(\text{Neither}\)%
\par\smallskip%
\noindent\textbf{\blocktitlefont Solution}.\quad{}%
\begin{enumerate}[label=(\alph*)]
\item{}This is injective (but not surjective). No element in the codomain is the image of more than one element from the domain (although 3 is the image of NO element in the domain).%
\item{}This is a neither injective nor surjective. The element 2 is the image of more than one element from the domain, and the element 5 is not the image of any element in the domain.%
\item{}This is neither injective nor surjective. We can write the function in two line notation as \(f = \twoline{1 \amp 2 \amp 3 \amp 4}{3 \amp 3 \amp 5 \amp 4}\text{.}\) No number is spelled with 1 letter, and ``one'' and ``two'' have three letters.%
\end{enumerate}
%
\end{divisionsolution}%
\begin{divisionsolution}{0.4.7}{}{p:exercise:McO}%
Consider the function \(f:\{1,2,3,4,5\} \to \{1,2,3,4\}\) given by the table below:%
\begin{sidebyside}{1}{0}{0}{0}%
\begin{sbspanel}{1}%
{\centering%
{\tabularfont%
\begin{tabular}{llllll}
\multicolumn{1}{cA}{\(x\)}&1&2&3&4&5\tabularnewline\hrulethin
\multicolumn{1}{cA}{\(f(x)\)}&3&2&4&1&2
\end{tabular}
}%
\par}
\end{sbspanel}%
\end{sidebyside}%
\par
%
\begin{enumerate}[label=(\alph*)]
\item{}Is \(f\) injective? Explain.%
\item{}Is \(f\) surjective? Explain.%
\item{}Write the function using two-line notation.%
\end{enumerate}
%
\par\smallskip%
\noindent\textbf{\blocktitlefont Solution}.\quad{}%
\begin{enumerate}[label=(\alph*)]
\item{}\(f\) is not injective, since \(f(2) = f(5)\); two different inputs have the same output.%
\item{}\(f\) is surjective, since every element of the codomain is an element of the range.%
\item{}\(f=\begin{pmatrix}1 \amp 2 \amp 3 \amp 4 \amp 5 \\ 3 \amp 2 \amp 4 \amp 1 \amp 2\end{pmatrix}\).%
\end{enumerate}
%
\end{divisionsolution}%
\begin{divisionsolution}{0.4.8}{}{p:exercise:sjX}%
Consider the function \(f:\{1,2,3,4\} \to \{1,2,3,4\}\) given by the graph below.%
\begin{sidebyside}{1}{0.3}{0.3}{0}%
\begin{sbspanel}{0.4}%
\resizebox{\linewidth}{!}{%
\begin{tikzpicture}[scale=1]
  \draw[thin, gray!50] (0,0) grid (4.5, 4.5);
  \draw[<->, thick] (0,4.5) node[left] {\(f(x)\)} -- (0,0) -- (4.5,0) node[below right] {\(x\)};
  \foreach \x in {1,2,3,4}
    \draw (\x,0) node[below] { \x} (0, \x) node[left] { \x};
  \fill (1,3) circle (.1) (2,4) circle (.1) (3,1) circle (.1) (4,3) circle (.1);
\end{tikzpicture}
}%
\end{sbspanel}%
\end{sidebyside}%
\par
%
\begin{enumerate}[label=(\alph*)]
\item{}Is \(f\) injective? Explain.%
\item{}Is \(f\) surjective? Explain.%
\item{}Write the function using two-line notation.%
\end{enumerate}
%
\par\smallskip%
\noindent\textbf{\blocktitlefont Solution}.\quad{}%
\begin{enumerate}[label=(\alph*)]
\item{}\(f\) is not injective, since \(f(1) = 3\) and \(f(4) = 3\).%
\item{}\(f\) is not surjective, since there is no input which gives 2 as an output.%
\item{}\(f=\begin{pmatrix} 1 \amp 2 \amp 3 \amp 4 \\ 3 \amp 4 \amp 1 \amp 3\end{pmatrix}\).%
\end{enumerate}
%
\end{divisionsolution}%
\begin{divisionsolution}{0.4.9}{}{p:exercise:Yrg}%
Write out all functions \(f: \{1,2,3\} \to \{a,b\}\) (using two-line notation).%
\par
How many functions are there?%
\par
How many are injective?%
\par
How many are surjective?%
\par
How many are bijective?%
\par\smallskip%
\noindent\textbf{\blocktitlefont Answer 1}.\quad{}\(8\)%
\par\smallskip%
\noindent\textbf{\blocktitlefont Answer 2}.\quad{}\(0\)%
\par\smallskip%
\noindent\textbf{\blocktitlefont Answer 3}.\quad{}\(6\)%
\par\smallskip%
\noindent\textbf{\blocktitlefont Answer 4}.\quad{}\(0\)%
\par\smallskip%
\noindent\textbf{\blocktitlefont Solution 1}.\quad{}There are 8 functions, including 6 surjective and zero injective funtions.%
\par\smallskip%
\noindent\textbf{\blocktitlefont Solution 2}.\quad{}There are 8 different functions. In two-line notation these are:%
\par
%
\begin{equation*}
f = \begin{pmatrix} 1 \amp 2 \amp 3 \\ a \amp a\amp a \end{pmatrix} \quad f = \begin{pmatrix} 1 \amp 2 \amp 3 \\ b \amp b \amp b \end{pmatrix}
\end{equation*}
%
\par
%
\begin{equation*}
f = \begin{pmatrix} 1 \amp 2 \amp 3 \\ a \amp a\amp b \end{pmatrix} \quad f = \begin{pmatrix} 1 \amp 2 \amp 3 \\ a \amp b \amp a \end{pmatrix} \quad f = \begin{pmatrix} 1 \amp 2 \amp 3 \\ b \amp a\amp a \end{pmatrix}
\end{equation*}
%
\par
%
\begin{equation*}
\quad f = \begin{pmatrix} 1 \amp 2 \amp 3 \\ b \amp b \amp a \end{pmatrix} \quad f = \begin{pmatrix} 1 \amp 2 \amp 3 \\ b \amp a\amp b \end{pmatrix} \quad f = \begin{pmatrix} 1 \amp 2 \amp 3 \\ a \amp b \amp b \end{pmatrix}
\end{equation*}
%
\par
None of the functions are injective. Exactly 6 of the functions are surjective. No functions are both (since no functions here are injective).%
\end{divisionsolution}%
\begin{divisionsolution}{0.4.10}{}{p:exercise:Eyp}%
Write out all functions \(f: \{1,2\} \to \{a,b,c\}\) (in two-line notation).%
\par
How many functions are there?%
\par
How many are injective?%
\par
How many are surjective?%
\par
How many are bijective?%
\par\smallskip%
\noindent\textbf{\blocktitlefont Answer 1}.\quad{}\(9\)%
\par\smallskip%
\noindent\textbf{\blocktitlefont Answer 2}.\quad{}\(6\)%
\par\smallskip%
\noindent\textbf{\blocktitlefont Answer 3}.\quad{}\(0\)%
\par\smallskip%
\noindent\textbf{\blocktitlefont Answer 4}.\quad{}\(0\)%
\par\smallskip%
\noindent\textbf{\blocktitlefont Solution}.\quad{}There are 9 functions: you have a choice of three outputs for \(f(1)\text{,}\) and for each, you have three choices for the output \(f(2)\text{.}\) Of these functions, 6 are injective, 0 are surjective, and 0 are both (i.e., bijective):%
\par
%
\begin{equation*}
f = \twoline{1 \amp 2}{a\amp a} \quad f = \twoline{1 \amp 2}{b \amp b} \quad f = \twoline{1 \amp 2}{c \amp c}
\end{equation*}
%
\par
%
\begin{equation*}
f = \twoline{1 \amp 2}{a\amp b} \quad f = \twoline{1 \amp 2}{a \amp c} \quad f = \twoline{1 \amp 2}{b \amp c}
\end{equation*}
%
\par
%
\begin{equation*}
f = \twoline{1 \amp 2}{b \amp a} \quad f = \twoline{1 \amp 2}{c \amp a} \quad f = \twoline{1 \amp 2}{c \amp b}
\end{equation*}
%
\end{divisionsolution}%
\begin{divisionsolution}{0.4.11}{}{p:exercise:kFy}%
Suppose \(f:\N \to \N\) satisfies the recurrence relation%
\begin{equation*}
f(n+1) = \begin{cases} \frac{f(n)}{2} \amp \text{ if } f(n) \text{ is even} \\ 3f(n) + 1 \amp \text{ if } f(n) \text{ is odd}\end{cases}\text{.}
\end{equation*}
Note that with the initial condition \(f(0) = 1\), the values of the function are: \(f(1) = 4\), \(f(2) = 2\), \(f(3) = 1\), \(f(4) = 4\), and so on, the images cycling through those three numbers. Thus \(f\) is NOT injective (and also certainly not surjective). Might it be under other initial conditions?\footnotemark{}%
\begin{enumerate}[label=(\alph*)]
\item{}If \(f\) satisfies the initial condition \(f(0) = 5\), is \(f\) injective? Explain why or give a specific example of two elements from the domain with the same image.%
\item{}If \(f\) satisfies the initial condition \(f(0) = 3\), is \(f\) injective? Explain why or give a specific example of two elements from the domain with the same image.%
\item{}If \(f\) satisfies the initial condition \(f(0) = 27\), then it turns out that \(f(105) = 10\) and no two numbers less than 105 have the same image. Could \(f\) be injective? Explain.%
\item{}Prove that no matter what initial condition you choose, the function cannot be surjective.%
\end{enumerate}
%
\par\smallskip%
\noindent\textbf{\blocktitlefont Solution}.\quad{}%
\begin{enumerate}[label=(\alph*)]
\item{}We would have \(f(1) = 16\), \(f(2) = 8\), \(f(3) = 4\), \(f(4) = 2\), \(f(5) = 1\) and then \(f(6) = 4\) again. So for example, \(f(3) = f(6)\), so the function is not injective.%
\item{}Here \(f(1) = 10\), \(f(2) = 5\), \(f(3) = 16\), and so on. But we see that the function has fallen into the same pattern as the previous version. Here we will have \(f(5) = f(8) = 4\), so again the function is not injective.%
\item{}Since \(f(105) = 10\), we can use what we have found above to say that for sure \(f(109) = f(112) = 4\), so again, the function is not injective.%
\item{}If a function were to be surjective, then every natural number would have to be an output for some natural number. But then there would be a first time that 1 was output. Say \(f(a) = 1\). From this point on, the only outputs you ever get are 4, 2, and 1 (and we would have already gotten these). Thus there would be only \(a\) different outputs. Thus in fact, \(f\) is not surjective.%
\end{enumerate}
%
\end{divisionsolution}%
\begin{divisionsolution}{0.4.12}{}{p:exercise:QMH}%
For each function given below, determine whether or not the function is injective and whether or not the function is surjective.%
\begin{enumerate}[label=(\alph*)]
\item{}\(f:\N \to \N\) given by \(f(n) = n+4\).%
\item{}\(f:\Z \to \Z\) given by \(f(n) = n+4\).%
\item{}\(f:\Z \to \Z\) given by \(f(n) = 5n - 8\).%
\item{}\(f:\Z \to \Z\) given by \(f(n) = \begin{cases}n/2 \amp \text{ if } n \text{ is even} \\ (n+1)/2 \amp \text{ if } n \text{ is odd} . \end{cases}\)%
\end{enumerate}
%
\par\smallskip%
\noindent\textbf{\blocktitlefont Solution}.\quad{}%
\begin{enumerate}[label=(\alph*)]
\item{}\(f\) is injective, but not surjective (since 0, for example, is never an output).%
\item{}\(f\) is injective and surjective. Unlike in the previous question, every integers is an output (of the integer 4 less than it).%
\item{}\(f\) is injective, but not surjective (10 is not 8 less than a multiple of 5, for example).%
\item{}\(f\) is not injective, but is surjective. Every integer is an output (of twice itself, for example) but some integers are outputs of more than one input: \(f(5) = 3 = f(6)\).%
\end{enumerate}
%
\end{divisionsolution}%
\begin{divisionsolution}{0.4.13}{}{p:exercise:wTQ}%
Let \(A = \{1,2,3,\ldots,10\}\). Consider the function \(f:\pow(A) \to \N\) given by \(f(B) = |B|\). That is, \(f\) takes a subset of \(A\) as an input and outputs the cardinality of that set.%
\begin{enumerate}[label=(\alph*)]
\item{}Is \(f\) injective? Prove your answer.%
\item{}Is \(f\) surjective? Prove your answer.%
\item{}Find \(f\inv(1)\).%
\item{}Find \(f\inv(0)\).%
\item{}Find \(f\inv(12)\).%
\end{enumerate}
%
\par\smallskip%
\noindent\textbf{\blocktitlefont Solution}.\quad{}%
\begin{enumerate}[label=(\alph*)]
\item{}\(f\) is not injective. To prove this, we must simply find two different elements of the domain which map to the same element of the codomain. Since \(f(\{1\}) = 1\) and \(f(\{2\}) = 1\), we see that \(f\) is not injective.%
\item{}\(f\) is not surjective. The largest subset of \(A\) is \(A\) itself, and \(|A| = 10\). So no natural number greater than 10 will ever be an output.%
\item{}\(f\inv(1) = \{\{1\}, \{2\}, \{3\}, \ldots \{10\}\}\) (the set of all the singleton subsets of \(A\)).%
\item{}\(f\inv(0) = \{\emptyset\}\). Note, it would be wrong to write \(f\inv(0) = \emptyset\) - that would claim that there is no input which has 0 as an output.%
\item{}\(f\inv(12) = \emptyset\), since there are no subsets of \(A\) with cardinality 12.%
\end{enumerate}
%
\end{divisionsolution}%
\begin{divisionsolution}{0.4.14}{}{p:exercise:daZ}%
Consider the function \(f:\N \to \N\) given \emph{recursively} by \(f(0) = 1\) and \(f(n+1) = 2\cdot f(n)\text{.}\) Find \(f(10)\text{.}\)%
\par\smallskip%
\noindent\textbf{\blocktitlefont Answer}.\quad{}\(1024\)%
\par\smallskip%
\noindent\textbf{\blocktitlefont Solution}.\quad{}\(f(10) = 1024\text{.}\) To find \(f(10)\text{,}\) we need to know \(f(9)\text{,}\) for which we need \(f(8)\text{,}\) and so on. So build up from \(f(0) = 1\text{.}\) Then \(f(1) = 2\text{,}\) \(f(2) = 4\text{,}\) \(f(3) = 8\text{,}\) .... In fact, it looks like a closed formula for \(f\) is \(f(n) = 2^n\text{.}\) Later we will see how to prove this is correct.%
\end{divisionsolution}%
\begin{divisionsolution}{0.4.15}{}{p:exercise:Jii}%
Consider the set \(\N^2 = \N \times \N\), the set of all ordered pairs \((a,b)\) where \(a\) and \(b\) are natural numbers. Consider a function \(f: \N^2 \to \N\) given by \(f((a,b)) =a+b\).%
\begin{enumerate}[label=(\alph*)]
\item{}Let \(A = \{(a,b) \in \N^2 \st a, b \le 10\}\). Find \(f(A)\).%
\item{}Find \(f\inv(3)\) and \(f\inv(\{0,1,2,3\})\).%
\item{}Give geometric descriptions of \(f\inv(n)\) and \(f\inv(\{0, 1, \ldots, n\})\) for any \(n \ge 1\).%
\item{}Find \(\card{f\inv(8)}\) and \(\card{f\inv(\{0,1, \ldots, 8\})}\).%
\end{enumerate}
%
\par\smallskip%
\noindent\textbf{\blocktitlefont Solution}.\quad{}%
\begin{enumerate}[label=(\alph*)]
\item{}\(f\inv(A) = \{0, 1, \ldots, 20\}\).%
\item{}\(f\inv(3) = \{(3,0), (2,1), (1,2), (0,3)\}\). \(f\inv(\{0,1,2,3\}) = \{(0,0), (1,0), (0,1), (2, 0), (1,1), (0, 2), (3,0), (2,1), (1,2), (0,3)\}\).%
\item{}The set \(f\inv(n)\) will be all the integer coordinate points that fall on the line \(x+y = n\) in the first quadrant (including \((n,0)\) and \((0,n)\)). The set \(f\inv(\{0, 1, \ldots,
n\})\) will include this line plus every integer coordinate point in the triangle formed by that line and the coordinate axes (including those on the axes).%
\item{}\(\card{f\inv(8)} = 9\). \(\card{f\inv(\{0, 1, \ldots, 8\})} = 45\).%
\end{enumerate}
%
\end{divisionsolution}%
\begin{divisionsolution}{0.4.16}{}{p:exercise:ppr}%
Let \(f:X \to Y\) be some function. Suppose \(3 \in Y\). What can you say about \(f\inv(3)\) if you know,%
\begin{enumerate}[label=(\alph*)]
\item{}\(f\) is injective? Explain.%
\item{}\(f\) is surjective? Explain.%
\item{}\(f\) is bijective? Explain.%
\end{enumerate}
%
\par\smallskip%
\noindent\textbf{\blocktitlefont Solution}.\quad{}%
\begin{enumerate}[label=(\alph*)]
\item{}\(|f\inv(3)| \le 1\). In other words, either \(f\inv(3)\) is the empty set or is a set containing exactly one element. Injective functions cannot have two elements from the domain both map to 3.%
\item{}\(|f\inv(3)| \ge 1\). In other words, \(f\inv(3)\) is a set containing at least one elements, possibly more. Surjective functions must have something map to 3.%
\item{}\(|f\inv(3)| = 1\). There is exactly one element from \(X\) which gets mapped to 3, so \(f\inv(3)\) is the set containing that one element.%
\end{enumerate}
%
\end{divisionsolution}%
\begin{divisionsolution}{0.4.17}{}{p:exercise:VwA}%
Find a set \(X\) and a function \(f:X \to \N\) so that \(f\inv(0) \cup f\inv(1) = X\).%
\par\smallskip%
\noindent\textbf{\blocktitlefont Solution}.\quad{}\(X\) can really be any set, as long as \(f(x) = 0\) or \(f(x) = 1\) for every \(x \in X\). For example, \(X = \N\) and \(f(n) = 0\) works.%
\end{divisionsolution}%
\begin{divisionsolution}{0.4.18}{}{p:exercise:BDJ}%
What can you deduce about the sets \(X\) and \(Y\) if you know,%
\begin{enumerate}[label=(\alph*)]
\item{}there is an injective function \(f:X \to Y\)? Explain.%
\item{}there is a surjective function \(f:X \to Y\)? Explain.%
\item{}there is a bijective function \(f:X \to Y\)? Explain.%
\end{enumerate}
%
\par\smallskip%
\noindent\textbf{\blocktitlefont Solution}.\quad{}%
\begin{enumerate}[label=(\alph*)]
\item{}\(|X| \le |Y|\). Otherwise two or more of the elements of \(X\) would need to map to the same element of \(Y\).%
\item{}\(|X| \ge |Y|\). Otherwise there would be one or more elements of \(Y\) which were never an output.%
\item{}\(|X| = |Y|\). This is the only way for both of the above to occur.%
\end{enumerate}
%
\end{divisionsolution}%
\begin{divisionsolution}{0.4.19}{}{p:exercise:hKS}%
Suppose \(f:X \to Y\) is a function. Which of the following are possible? Explain.%
\begin{enumerate}[label=(\alph*)]
\item{}\(f\) is injective but not surjective.%
\item{}\(f\) is surjective but not injective.%
\item{}\(|X| = |Y|\) and \(f\) is injective but not surjective.%
\item{}\(|X| = |Y|\) and \(f\) is surjective but not injective.%
\item{}\(|X| = |Y|\), \(X\) and \(Y\) are finite, and \(f\) is injective but not surjective.%
\item{}\(|X| = |Y|\), \(X\) and \(Y\) are finite, and \(f\) is surjective but not injective.%
\end{enumerate}
%
\par\smallskip%
\noindent\textbf{\blocktitlefont Solution}.\quad{}%
\begin{enumerate}[label=(\alph*)]
\item{}Possible. For example, let \(X=\{1,2\}\) and \(Y = \{1,2,3\}\) and consider \(f=\begin{pmatrix}1 \amp 2 \\ 1 \amp 3\end{pmatrix}\).%
\item{}Possible. For example, let \(X = \{1,2\}\) and \(Y = \{1\}\) with \(f(1) = f(2) = 1\).%
\item{}Possible. This is possible, but only if both \(X\) and \(Y\) are infinite. For example, if \(X = Y = \Z\), consider \(f(x) = 2x\).%
\item{}Possible. For example, again take \(X = Y = \N\) and consider \(f(x) = \begin{cases} 0 \amp \text{if } x = 0 \\ x-1 \amp \text{if } x \ge 1\end{cases}\)%
\item{}Not possible. If \(f\) is injective, then each input corresponds to a different output, so the range is the size of \(X\), which is the size of \(Y\), so the range is all of \(Y\).%
\item{}Not possible. If \(f\) is surjective, then every element of the codomain is an output. If more than one output corresponded to a single input, we would not have enough inputs to cover all the outputs.%
\end{enumerate}
%
\end{divisionsolution}%
\begin{divisionsolution}{0.4.20}{}{p:exercise:NSb}%
Let \(f:X \to Y\) and \(g:Y \to Z\) be functions. We can define the \terminology{composition} \index{composition} of \(f\) and \(g\) to be the function \(g\circ f:X \to Z\) for which the image of each \(x \in X\) is \(g(f(x))\). That is, plug \(x\) into \(f\), then plug the result into \(g\) (just like composition in algebra and calculus).%
\begin{enumerate}[label=(\alph*)]
\item{}If \(f\) and \(g\) are both injective, must \(g\circ f\) be injective? Explain.%
\item{}If \(f\) and \(g\) are both surjective, must \(g\circ f\) be surjective? Explain.%
\item{}Suppose \(g\circ f\) is injective. What, if anything, can you say about \(f\) and \(g\)? Explain.%
\item{}Suppose \(g\circ f\) is surjective. What, if anything, can you say about \(f\) and \(g\)? Explain.%
\end{enumerate}
%
\par\smallskip%
\noindent\textbf{\blocktitlefont Hint}.\quad{}Work with some examples. What if \(f = \twoline{1\amp 2 \amp 3}{a \amp a \amp b}\) and \(g = \twoline{a\amp b \amp c}{5 \amp 6 \amp 7}\)?%
\par\smallskip%
\noindent\textbf{\blocktitlefont Solution}.\quad{}%
\begin{enumerate}[label=(\alph*)]
\item{}Yes. If \(x_1\) and \(x_2\) are two different elements of the domain \(X\), then \(f(x_1)\) and \(f(x_2)\) must be different elements of \(Y\), since \(f\) is injective. Then \(g(f(x_1))\) must be different from \(g(f(x_2))\) since \(g\) is injective. So no two distinct elements from the domain will have the same image.%
\item{}Yes. We know that every element of \(Z\) is the image of at least one element from \(Y\) under \(g\). But every element from \(Y\) is the image of some element in \(X\) under \(f\). So every element from \(Z\) is the image of at least one element from \(X\) under \(g\circ f\).%
\item{}We know that \(f\) must be injective, and \(g\) could or could not be injective. Note that if \(f\) was not injective, then there would be two elements in \(X\) that went to the same element in \(Y\) and that element would need to go to one element in \(Z\), contradicting that \(g\circ f\) was injective. However, consider \(X = \{1,2\}\), \(Y = \{a, b, c\}\) and \(Z = \{5, 6\}\), together with \(f = \twoline{1 \amp 2}{a, b}\) and \(g = \twoline{a \amp b \amp c}{5, 6, 6}\). It is easy to see that \(g \circ f\) is still injective even though \(g\) is not.%
\item{}Now we know that \(g\) must be surjective, but \(f\) does not need to be (although it could be). If there is an element in \(Z\) that is not the image of anything in \(Y\) under \(g\), then that element certainly could not be the image of anything in \(X\) under \(g\circ f\). Note that the counterexample from the previous part also shows that \(f\) does not need to be surjective to make \(g\circ f\) surjective.%
\end{enumerate}
%
\end{divisionsolution}%
\begin{divisionsolution}{0.4.21}{}{p:exercise:tZk}%
Consider the function \(f:\Z \to \Z\) given by \(f(n) = \begin{cases}n+1 \amp \text{ if }n\text{ is even} \\ n-3 \amp \text{ if }n\text{ is odd} . \end{cases}\)%
\begin{enumerate}[label=(\alph*)]
\item{}Is \(f\) injective? Prove your answer.%
\item{}Is \(f\) surjective? Prove your answer.%
\end{enumerate}
%
\par\smallskip%
\noindent\textbf{\blocktitlefont Solution}.\quad{}%
\begin{enumerate}[label=(\alph*)]
\item{}\(f\) is injective.%
\begin{proof}{}{p:proof:Zba}
Let \(x\) and \(y\) be elements of the domain \(\Z\). Assume \(f(x) = f(y)\). If \(x\) and \(y\) are both even, then \(f(x) = x+1\) and \(f(y) = y+1\). Since \(f(x) = f(y)\), we have \(x + 1 = y + 1\) which implies that \(x = y\). Similarly, if \(x\) and \(y\) are both odd, then \(x - 3 = y-3\) so again \(x = y\). The only other possibility is that \(x\) is even an \(y\) is odd (or visa-versa). But then \(x + 1\) would be odd and \(y - 3\) would be even, so it cannot be that \(f(x) = f(y)\). Therefore if \(f(x) = f(y)\) we then have \(x = y\), which proves that \(f\) is injective.%
\end{proof}
\item{}\(f\) is surjective.%
\begin{proof}{}{p:proof:Fij}
Let \(y\) be an element of the codomain \(\Z\). We will show there is an element \(n\) of the domain (\(\Z\)) such that \(f(n) = y\). There are two cases: First, if \(y\) is even, then let \(n = y+3\). Since \(y\) is even, \(n\) is odd, so \(f(n) = n-3 = y+3-3 = y\) as desired. Second, if \(y\) is odd, then let \(n = y-1\). Since \(y\) is odd, \(n\) is even, so \(f(n) = n+1 = y-1+1 = y\) as needed. Therefore \(f\) is surjective.%
\end{proof}
\end{enumerate}
%
\end{divisionsolution}%
\begin{divisionsolution}{0.4.22}{}{p:exercise:agt}%
At the end of the semester a teacher assigns letter grades to each of her students. Is this a function? If so, what sets make up the domain and codomain, and is the function injective, surjective, bijective, or neither?%
\par\smallskip%
\noindent\textbf{\blocktitlefont Solution}.\quad{}Yes, this is a function, if you choose the domain and codomain correctly. The domain will be the set of students, and the codomain will be the set of possible grades. The function is almost certainly not injective, because it is likely that two students will get the same grade. The function might be surjective \textendash{} it will be if there is at least one student who gets each grade.%
\end{divisionsolution}%
\begin{divisionsolution}{0.4.23}{}{p:exercise:GnC}%
In the game of \emph{Hearts}, four players are each dealt 13 cards from a deck of 52. Is this a function? If so, what sets make up the domain and codomain, and is the function injective, surjective, bijective, or neither?%
\par\smallskip%
\noindent\textbf{\blocktitlefont Solution}.\quad{}Yes, as long as the set of cards is the domain and the set of players is the codomain. The function is not injective because multiple cards go to each player. It is surjective since all players get cards.%
\end{divisionsolution}%
\begin{divisionsolution}{0.4.24}{}{p:exercise:muL}%
Seven players are playing 5-card stud. Each player initially receives 5 cards from a deck of 52. Is this a function? If so, what sets make up the domain and codomain, and is the function injective, surjective, bijective, or neither?%
\par\smallskip%
\noindent\textbf{\blocktitlefont Solution 1}.\quad{}This is not a function.%
\par\smallskip%
\noindent\textbf{\blocktitlefont Solution 2}.\quad{}This cannot be a function. If the domain were the set of cards, then it is not a function because not every card gets dealt to a player. If the domain were the set of players, it would not be a function because a single player would get mapped to multiple cards. Since this is not a function, it doesn't make sense to say whether it is injective\slash{}surjective\slash{}bijective.%
\end{divisionsolution}%
\begin{divisionsolution}{0.4.25}{}{p:exercise:SBU}%
Consider the function \(f:\N \to \N\) that gives the number of handshakes that take place in a room of \(n\) people assuming everyone shakes hands with everyone else. Give a recursive definition for this function.%
\par\smallskip%
\noindent\textbf{\blocktitlefont Hint}.\quad{}To find the recurrence relation, consider how many \emph{new} handshakes occur when person \(n+1\) enters the room.%
\par\smallskip%
\noindent\textbf{\blocktitlefont Solution 1}.\quad{}The recurrence relation is \(f(n+1) = f(n) + n\).%
\par\smallskip%
\noindent\textbf{\blocktitlefont Solution 2}.\quad{}With zero people in the room, there are no handshakes. So the initial condition is \(f(0) = 0\). Now suppose there are \(n\) people in the room, and they have already shaken hands with each other. There have been \(f(n)\) handshakes already. Another person enters the room. They must shake hands with all \(n\) people, so there are \(n\) more handshakes.%
\par
Thus the recurrence relation is \(f(n+1) = f(n) + n\). A full recursive definition would be written:%
\begin{equation*}
f(0) = 0;~ f(n+1) = f(n) + n\text{.}
\end{equation*}
%
\end{divisionsolution}%
\begin{divisionsolution}{0.4.26}{}{p:exercise:yJd}%
Let \(f:X \to Y\) be a function and \(A \subseteq X\) be a finite subset of the domain. What can you say about the relationship between \(\card{A}\) and \(\card{f(A)}\)? Consider both the general case and what happens when you know \(f\) is injective, surjective, or bijective.%
\par\smallskip%
\noindent\textbf{\blocktitlefont Solution}.\quad{}In general, \(\card{A} \ge \card{f(A)}\), since you cannot get more outputs than you have inputs (each input goes to exactly one output), but you could have fewer outputs if the function is not injective. If the function is injective, then \(\card{A} = \card{f(A)}\), although you can have equality even if \(f\) is not injective (it must be injective \emph{restricted} to \(A\)).%
\end{divisionsolution}%
\begin{divisionsolution}{0.4.27}{}{p:exercise:eQm}%
Let \(f:X \to Y\) be a function and \(B \subseteq Y\) be a finite subset of the codomain. What can you say about the relationship between \(\card{B}\) and \(\card{f\inv(B)}\)? Consider both the general case and what happens when you know \(f\) is injective, surjective, or bijective.%
\par\smallskip%
\noindent\textbf{\blocktitlefont Solution}.\quad{}In general, there is no relationship between \(\card{B}\) and \(\card{f\inv(B)}\). This is because \(B\) might contain elements that are not in the range of \(f\), so we might even have \(f\inv(B) = \emptyset\). On the other hand, there might be lots of elements from the domain that all get sent to a few elements in \(B\), making \(f\inv(B)\) larger than \(B\).%
\par
More specifically, if \(f\) is injective, then \(\card{B} \ge \card{f\inv(B)}\) (since every element in \(B\) must come from at most one element from the domain). If \(f\) is surjective, then \(\card{B} \le \card{f\inv(B)}\) (since every element in \(B\) must come from at least one element of the domain). Thus if \(f\) is bijective then \(\card{B} = \card{f\inv(B)}\).%
\end{divisionsolution}%
\begin{divisionsolution}{0.4.28}{}{p:exercise:KXv}%
Let \(f:X \to Y\) be a function, \(A \subseteq X\) and \(B \subseteq Y\).%
\begin{enumerate}[label=(\alph*)]
\item{}Is \(f\inv\left(f(A)\right) = A\)? Always, sometimes, never? Explain.%
\item{}Is \(f\left(f\inv(B)\right) = B\)? Always, sometimes, never? Explain.%
\item{}If one or both of the above do not always hold, is there something else you can say? Will equality always hold for particular types of functions? Is there some other relationship other than equality that would always hold? Explore.%
\end{enumerate}
%
\par\smallskip%
\noindent\textbf{\blocktitlefont Solution}.\quad{}%
\begin{enumerate}[label=(\alph*)]
\item{}This sometimes happens, for example if \(X = Y = \{1,2,3\}\), \(A = \{1,2\}\) and \(f = \twoline{1 \amp 2 \amp 3}{2 \amp 3 \amp 1}\), since \(f(A) = \{2,3\}\) and \(f\inv(\{2,3\}) = \{1, 2\} = A\). However, not always: if instead \(f = \twoline{1\amp 2 \amp 3}{2 \amp 1 \amp 2}\), then \(f(A) = \{1,2\}\) and \(f\inv(\{1,2\}) = \{1,2,3\}\).%
\item{}Again, this only happens sometimes. An example of when it happens is to take \(X = Y = \{1,2,3\}\), \(B = \{2,3\}\) and \(f = \twoline{1\amp 2 \amp 3}{3\amp 1 \amp 2}\). Then \(f\inv(B) = \{1,3\}\) and \(f(\{1,3\}) = \{2, 3\} = B\). However, if instead we took \(f = \twoline{1 \amp 2 \amp 3}{1 \amp 1 \amp 1}\), then \(f\inv(B) = \emptyset\) and \(f(\emptyset) = \emptyset \ne B\)%
\item{}A reasonable thing to check would be whether equality held as long as \(f\) is a bijection (then perhaps equality would hold no matter what \(A\) and \(B\) are. Another direction to go would be to see whether you could replace the equality with subset. It doesn't look like \(f\inv(f(A)) \subseteq A\), but maybe \(f(f\inv(B)) \subseteq B\)? Maybe \(A \subseteq f\inv (f(A))\)?%
\end{enumerate}
%
\end{divisionsolution}%
\begin{divisionsolution}{0.4.29}{}{p:exercise:reE}%
Let \(f:X \to Y\) be a function and \(A, B \subseteq X\) be subsets of the domain.%
\begin{enumerate}[label=(\alph*)]
\item{}Is \(f(A \cup B) = f(A) \cup f(B)\)? Always, sometimes, or never? Explain.%
\item{}Is \(f(A \cap B) = f(A) \cap f(B)\)? Always, sometimes, or never? Explain.%
\end{enumerate}
%
\par\smallskip%
\noindent\textbf{\blocktitlefont Hint}.\quad{}One of these is not always true.  Try some examples!%
\par\smallskip%
\noindent\textbf{\blocktitlefont Solution}.\quad{}%
\begin{enumerate}[label=(\alph*)]
\item{}This is always true. To prove it, first consider an arbitrary element \(y \in f(A \cup B)\). Then \(y = f(x)\) for some \(x \in A \cup B\). If \(x \in A\), then \(y \in f(A)\). If \(x \in B\), then \(y \in f(B)\). So in either case, \(y \in f(A) \cup f(B)\). On the other hand, suppose \(y \in f(A) \cup f(B)\). Then \(y \in f(A)\) or \(y \in f(B)\). In the first, case, \(y = f(x)\) for some \(x \in A\). But then \(x \in A \cup B\) so \(y \in f(A \cup B)\). By the same argument, if \(y \in f(B)\) we also have \(y \in f(A \cup B)\).%
\item{}This is not always true (although it is sometimes). The issue is that \(A\) and \(B\) might not have much (or anything) in common, but their images might. For example, let \(f:\{1,2,3\} \to \{1,2,3\}\) be given by \(f= \twoline{1 \amp 2 \amp 3}{1 \amp 2 \amp 1}\). Let \(A = \{1, 2\}\) and \(B = \{2, 3\}\). Then \(A \cap B = \{2\}\) and \(f(A \cap B) = \{2\}\). However, both \(f(A) = \{1,2\}\) and \(f(B) = \{1,2\}\), so \(f(A) \cap f(B) = \{1,2\}\).%
\end{enumerate}
%
\end{divisionsolution}%
\begin{divisionsolution}{0.4.30}{}{p:exercise:XlN}%
Let \(f:X \to Y\) be a function and \(A, B \subseteq Y\) be subsets of the codomain.%
\begin{enumerate}[label=(\alph*)]
\item{}Is \(f\inv(A \cup B) = f\inv(A) \cup f\inv(B)\)? Always, sometimes, or never? Explain.%
\item{}Is \(f\inv(A \cap B) = f\inv(A) \cap f\inv(B)\)? Always, sometimes, or never? Explain.%
\end{enumerate}
%
\par\smallskip%
\noindent\textbf{\blocktitlefont Solution}.\quad{}%
\begin{enumerate}[label=(\alph*)]
\item{}This is always true. Suppose \(x \in f\inv(A \cup B)\). That means that \(f(x) \in A \cup B\). In other words, \(f(x) \in A\) or \(f(x) \in B\), which is to say \(x \in f\inv(A)\) or \(x\in f\inv(B)\). This is the same as \(x \in f\inv(A) \cup f\inv B\).%
\par
On the other hand, suppose \(x \in f\inv(A) \cup f\inv(B)\). So \(x \in f\inv(A)\) (meaning \(f(x) \in A\)) or \(x \in f\inv(B)\) (meaning \(f(x) \in B\)). This tells us that \(f(x) \in A \cup B\), which is to say \(x \in f\inv(A \cup B)\).%
\item{}This is also always true. Suppose first that \(x \in f\inv(A \cap B)\). Then \(f(x) \in A \cap B\) so \(f(x) \in A\) and \(f(x) \in B\). That is the same as saying \(x \in f\inv(A)\) and \(x \in f\inv(B)\), so \(x \in f\inv(A) \cap f\inv(B)\).%
\par
On the other hand, suppose \(x \in f\inv(A) \cap f\inv(B)\). This means that \(x \in f\inv(A)\) and \(x \in f\inv(B)\). In other words, \(f(x) \in A\) and \(f(x) \in B\), so \(f(x) \in A \cap B\). This is another way of saying \(x \in f\inv(A \cap B)\).%
\end{enumerate}
%
\end{divisionsolution}%
\chapter*{1 Counting}
\addcontentsline{toc}{chapter}{1 Counting}
\chaptermark{1 Counting}
\section*{1.1 Additive and Multiplicative Principles}
\addcontentsline{toc}{section}{1.1 Additive and Multiplicative Principles}
\sectionmark{1.1 Additive and Multiplicative Principles}
\subsection*{Exercises}
\addcontentsline{toc}{subsection}{Exercises}
\begin{divisionsolution}{1.1.1}{}{p:exercise:Abr}%
Suppose \(f:\N \to \N\) satisfies the recurrence \(f(n+1) = f(n) + 3\text{.}\) Note that this is not enough information to define the function, since we don't have an initial condition. For each of the initial conditions below, find the value of \(f(5)\text{.}\)%
\begin{enumerate}[label=(\alph*)]
\item{}\(f(0) = 0\text{.}\)%
\item{}\(f(0) = 1\text{.}\)%
\item{}\(f(0) = 2\text{.}\)%
\item{}\(f(0) = 100\text{.}\)%
\end{enumerate}
%
\par\smallskip%
\noindent\textbf{\blocktitlefont Solution}.\quad{}For each case, you must use the recurrence to find \(f(1)\text{,}\) \(f(2)\) ... \(f(5)\text{.}\) But notice each time you just add three to the previous. We do this 5 times.%
\begin{enumerate}[label=(\alph*)]
\item{}\(f(5) = 15\text{.}\)%
\item{}\(f(5) = 16\text{.}\)%
\item{}\(f(5) = 17\text{.}\)%
\item{}\(f(5) = 115\text{.}\)%
\end{enumerate}
%
\end{divisionsolution}%
\begin{divisionsolution}{1.1.2}{}{p:exercise:giA}%
Let \(X = \{n \in \N \st 0 \le n \le 999\}\) be the set of all numbers with three or fewer digits. Define the function \(f:X \to \N\) by \(f(abc) = a+b+c\text{,}\) where \(a\text{,}\) \(b\text{,}\) and \(c\) are the digits of the number in \(X\) (write numbers less than 100 with leading 0's to make them three digits). For example, \(f(253) = 2 + 5 + 3 = 10\text{.}\)%
\begin{enumerate}[label=(\alph*)]
\item{}Let \(A = \{n \in X \st 113 \le n \le 122\}\text{.}\) Find \(f(A)\text{.}\)%
\item{}Find \(f\inv(\{1,2\})\)%
\item{}Find \(f\inv(3)\text{.}\)%
\item{}Find \(f\inv(28)\text{.}\)%
\item{}Is \(f\) injective? Explain.%
\item{}Is \(f\) surjective? Explain.%
\end{enumerate}
%
\par\smallskip%
\noindent\textbf{\blocktitlefont Answer 1}.\quad{}\(\left\{3,4,5,6,7,8,9,10,11\right\}\)%
\par\smallskip%
\noindent\textbf{\blocktitlefont Answer 2}.\quad{}\(\left\{1,2,10,11,20,100,101,110,200\right\}\)%
\par\smallskip%
\noindent\textbf{\blocktitlefont Answer 3}.\quad{}\(\left\{3,12,21,30,102,111,120,201,210,300\right\}\)%
\par\smallskip%
\noindent\textbf{\blocktitlefont Answer 4}.\quad{}\(\left\{\right\}\)%
\par\smallskip%
\noindent\textbf{\blocktitlefont Solution}.\quad{}%
\begin{enumerate}[label=(\alph*)]
\item{}\(f(A) = \{3,4,5,6,7,8,9,10,11\}\text{.}\)%
\item{}\(f\inv(\{1,2\}) = \{001, 010, 100, 011, 101, 110, 002, 020, 200\}\text{.}\)%
\item{}\(f\inv(3) = \{003, 030, 300, 012, 021, 102, 201, 120, 210, 111\}\)%
\item{}\(f\inv(28) = \emptyset\) (since the largest sum of three digits is \(9+9+9 = 27\))%
\item{}Part (c) proves that \(f\) is not injective. The output 3 is assigned to 10 different inputs.%
\item{}Part (d) proves that \(f\) is not surjective. There is an element of the codomain (28) which is not assigned to any inputs.%
\end{enumerate}
%
\end{divisionsolution}%
\begin{divisionsolution}{1.1.3}{}{p:exercise:MpJ}%
Your Blu-ray collection consists of 9 comedies and 7 horror movies. Give an example of a question for which the answer is:%
\begin{enumerate}[label=(\alph*)]
\item{}16.%
\item{}63.%
\end{enumerate}
%
\par\smallskip%
\noindent\textbf{\blocktitlefont Solution}.\quad{}%
\begin{enumerate}[label=(\alph*)]
\item{}For example, 16 is the number of choices you have if you want to watch one movie, either a comedy or horror flick.%
\item{}For example, 63 is the number of choices you have if you will watch two movies, first a comedy and then a horror.%
\end{enumerate}
%
\end{divisionsolution}%
\begin{divisionsolution}{1.1.4}{}{p:exercise:swS}%
Your wardrobe consists of 5 shirts, 3 pairs of pants, and 17 bow ties. bow ties How many different outfits can you make?%
\par\smallskip%
\noindent\textbf{\blocktitlefont Solution}.\quad{}There are 255 outfits. Use the multiplicative principle.%
\end{divisionsolution}%
\begin{divisionsolution}{1.1.5}{}{p:exercise:YEb}%
For your college interview, you must wear a tie. You own 3 regular (boring) ties and 5 (cool) bow ties.%
\begin{enumerate}[label=(\alph*)]
\item{}How many choices do you have for your neck-wear?%
\item{}You realize that the interview is for clown college, so you should probably wear both a regular tie and a bow tie. How many choices do you have now?%
\item{}For the rest of your outfit, you have 5 shirts, 4 skirts, 3 pants, and 7 dresses. You want to select either a shirt to wear with a skirt or pants, or just a dress. How many outfits do you have to choose from?%
\end{enumerate}
%
\par\smallskip%
\noindent\textbf{\blocktitlefont Answer 1}.\quad{}\(8\)%
\par\smallskip%
\noindent\textbf{\blocktitlefont Answer 2}.\quad{}\(15\)%
\par\smallskip%
\noindent\textbf{\blocktitlefont Answer 3}.\quad{}\(42\)%
\par\smallskip%
\noindent\textbf{\blocktitlefont Solution}.\quad{}%
\begin{enumerate}[label=(\alph*)]
\item{}8 ties. Use the additive principle.%
\item{}15 ties. Use the multiplicative principle%
\item{}\(5\cdot (4+3) + 7 = 42\) outfits.%
\end{enumerate}
%
\end{divisionsolution}%
\begin{divisionsolution}{1.1.6}{}{p:exercise:ELk}%
We usually write numbers in decimal form (or base 10), meaning numbers are composed using 10 different ``digits'' \(\{0,1,\ldots, 9\}\text{.}\) Sometimes though it is useful to write numbers hexadecimal or base 16. Now there are 16 distinct digits that can be used to form numbers: \(\{0, 1, \ldots, 9, \mathrm{A, B, C, D, E, F}\}\text{.}\) So for example, a 3 digit hexadecimal number might be 2B8.%
\begin{enumerate}[label=(\alph*)]
\item{}How many 2-digit hexadecimals are there in which the first digit is E or F? Explain your answer in terms of the additive principle (using either events or sets).%
\item{}Explain why your answer to the previous part is correct in terms of the multiplicative principle (using either events or sets). Why do both the additive and multiplicative principles give you the same answer?%
\item{}How many 3-digit hexadecimals start with a letter (A-F) and end with a numeral (0-9)? Explain.%
\item{}How many 3-digit hexadecimals start with a letter (A-F) or end with a numeral (0-9) (or both)? Explain.%
\end{enumerate}
%
\par\smallskip%
\noindent\textbf{\blocktitlefont Answer 1}.\quad{}\(32\)%
\par\smallskip%
\noindent\textbf{\blocktitlefont Answer 2}.\quad{}\(960\)%
\par\smallskip%
\noindent\textbf{\blocktitlefont Answer 3}.\quad{}\(3136\)%
\par\smallskip%
\noindent\textbf{\blocktitlefont Solution}.\quad{}%
\begin{enumerate}[label=(\alph*)]
\item{}There are 16 hexadecimals in which the first digit is an E (one for each choice of second digit). Similarly, there are 16 hexadecimals in which the first digit is an F. We want the union of these two disjoint sets, so there are \(16 + 16 = 32\) two digits hexadecimals in which the first digit is either an E or an F.%
\item{}We can first select the first digit in 2 ways. We then select the second digit in 16 ways. The multiplicative principle says that the number of ways to accomplish both these tasks together is \(2 \cdot 16 = 32\text{.}\) Of course \(2 \cdot 16 = 16 + 16\) so we get the same answer as in part (a). There we divided the total number of outcomes into two groups of size 16, each group based on the choice we made for the first task (selecting the first digit).%
\item{}We can select the first digit in 6 ways, the second digit in 16 ways, and the third digit in 10 ways. Thus there are \(6\cdot 16 \cdot 10 = 960\) hexadecimals given these restrictions.%
\item{}The number of 3-digit hexadecimals that start with a letter is \(6 \cdot 16 \cdot 16 = 1536\text{.}\) The number of 3-hexadecimals that end with a numeral is \(16 \cdot 16 \cdot 10 = 2560\text{.}\) We want all the elements from both these sets. However, both sets include those 3-digit hexadecimals which both start with a letter and end with a numeral (found to be 960 in the previous part), so we must subtract these (once). Thus the number of 3-digit hexadecimals starting with a letter or ending with a numeral is:%
\par
%
\begin{equation*}
1536 + 2560 - 960 = 3136
\end{equation*}
%
\end{enumerate}
%
\end{divisionsolution}%
\begin{divisionsolution}{1.1.7}{}{p:exercise:kSt}%
Suppose you have sets \(A\) and \(B\) with \(\card{A} = 10\) and \(\card{B} = 15\text{.}\)%
\begin{enumerate}[label=(\alph*)]
\item{}What is the largest possible value for \(\card{A \cap B}\text{?}\)%
\item{}What is the smallest possible value for \(\card{A \cap B}\text{?}\)%
\item{}What are the possible values for \(\card{A \cup B}\text{?}\)%
\end{enumerate}
%
\par\smallskip%
\noindent\textbf{\blocktitlefont Answer 1}.\quad{}\(10\)%
\par\smallskip%
\noindent\textbf{\blocktitlefont Answer 2}.\quad{}\(0\)%
\par\smallskip%
\noindent\textbf{\blocktitlefont Solution}.\quad{}%
\begin{enumerate}[label=(\alph*)]
\item{}To maximize the number of elements in common between \(A\) and \(B\text{,}\) make \(A \subset B\text{.}\) This would give \(\card{A \cap B} = 10\text{.}\)%
\item{}\(A\) and \(B\) might have no elements in common, giving \(\card{A\cap B} = 0\text{.}\)%
\item{}\(15 \le \card{A \cup B} \le 25\text{.}\) In fact, when \(\card{A \cap B} = 0\) then \(\card{A \cup B} = 25\) and when \(\card{A \cap B} = 10\) then \(\card{A \cup B} = 15\text{.}\)%
\end{enumerate}
%
\end{divisionsolution}%
\begin{divisionsolution}{1.1.8}{}{p:exercise:QZC}%
If \(\card{A} = 8\) and \(\card{B} = 5\text{,}\) what is \(\card{A \cup B} + \card{A \cap B}\text{?}\)%
\par\smallskip%
\noindent\textbf{\blocktitlefont Answer}.\quad{}\(13\)%
\par\smallskip%
\noindent\textbf{\blocktitlefont Solution}.\quad{}\(\card{A \cup B} + \card{A \cap B} = 13\text{.}\) Use PIE: we know \(\card{A \cup B} = 8 + 5 - \card{A \cap B}\text{.}\)%
\end{divisionsolution}%
\begin{divisionsolution}{1.1.9}{}{p:exercise:xgL}%
A group of college students were asked about their TV watching habits. Of those surveyed, 28 students watch \emph{The Walking Dead}, 19 watch \emph{The Blacklist}, and 24 watch \emph{Game of Thrones}. Additionally, 16 watch \emph{The Walking Dead} and \emph{The Blacklist}, 14 watch \emph{The Walking Dead} and \emph{Game of Thrones}, and 10 watch \emph{The Blacklist} and \emph{Game of Thrones}. There are 8 students who watch all three shows. How many students surveyed watched at least one of the shows?%
\par\smallskip%
\noindent\textbf{\blocktitlefont Answer}.\quad{}\(39\)%
\par\smallskip%
\noindent\textbf{\blocktitlefont Solution}.\quad{}39 students. Use Venn diagram or PIE: \(28 + 19 + 24 - 16 - 14 - 10 + 8 = 39\text{.}\)%
\end{divisionsolution}%
\begin{divisionsolution}{1.1.10}{}{p:exercise:dnU}%
In a recent survey, 30 students reported whether they liked their potatoes Mashed, French-fried, or Twice-baked. 15 liked them mashed, 20 liked French fries, and 9 liked twice baked potatoes. Additionally, 12 students liked both mashed and fried potatoes, 5 liked French fries and twice baked potatoes, 6 liked mashed and baked, and 3 liked all three styles. How many students \emph{hate} potatoes? Explain why your answer is correct.%
\par\smallskip%
\noindent\textbf{\blocktitlefont Answer}.\quad{}\(6\)%
\par\smallskip%
\noindent\textbf{\blocktitlefont Solution 1}.\quad{}6 students \emph{don't} like potatoes.%
\par\smallskip%
\noindent\textbf{\blocktitlefont Solution 2}.\quad{}Using the principle of inclusion\slash{}exclusion, the number of students who like their potatoes in at least one of the ways described is%
\par
%
\begin{equation*}
15 + 20 + 9 - 12 - 5 - 6 + 3 = 24
\text{.}
\end{equation*}
%
\par
Therefore there are \(30-24 = 6\) students who do not like potatoes. You can also do this problem with a Venn diagram.%
\end{divisionsolution}%
\begin{divisionsolution}{1.1.11}{}{p:exercise:Jvd}%
Let \(A\), \(B\), and \(C\) be sets.%
\begin{enumerate}[label=(\alph*)]
\item{}Find \(\card{(A \cup C)\setminus B}\) provided \(\card{A} = 50\), \(\card{B} = 45\), \(\card{C} = 40\), \(\card{A\cap B} = 20\), \(\card{A \cap C} = 15\), \(\card{B \cap C} = 23\), and \(\card{A \cap B \cap C} = 12\).%
\item{}Describe a set in terms of \(A\), \(B\), and \(C\) with cardinality 26.%
\end{enumerate}
%
\par\smallskip%
\noindent\textbf{\blocktitlefont Hint}.\quad{}For part (a) you could use the formula for PIE, but for part (b) you might be better off drawing a Venn diagram.%
\par\smallskip%
\noindent\textbf{\blocktitlefont Solution}.\quad{}%
\begin{enumerate}[label=(\alph*)]
\item{}\(\card{(A \cup C)\setminus B} = 44\). Use PIE or a Venn diagram.%
\item{}One possibility: \((A \cup B) \cap C\). Here using a Venn diagram is quite a bit easier.%
\end{enumerate}
%
\end{divisionsolution}%
\begin{divisionsolution}{1.1.12}{}{p:exercise:pCm}%
For how many \(n \in \{1,2, \ldots, 500\}\) is \(n\) a multiple of one or more of 5, 6, or 7?%
\par\smallskip%
\noindent\textbf{\blocktitlefont Hint}.\quad{}To find out how many numbers are divisible by 6 and 7, for example, take \(500/42\) and round down.%
\par\smallskip%
\noindent\textbf{\blocktitlefont Answer}.\quad{}\(215\)%
\par\smallskip%
\noindent\textbf{\blocktitlefont Solution 1}.\quad{}215 values of \(n\text{.}\)%
\par\smallskip%
\noindent\textbf{\blocktitlefont Solution 2}.\quad{}215 values of \(n\text{.}\) Use PIE: \(100 + 83 + 71 - 16 - 14 -11 + 2 = 215\) or a Venn diagram.%
\end{divisionsolution}%
\begin{divisionsolution}{1.1.13}{}{p:exercise:VJv}%
For how many three digit numbers (100 to 999) is the \emph{sum of the digits} even? (For example, \(343\) has an even sum of digits: \(3+4+3 = 10\) which is even.) Find the answer and explain why it is correct in at least two \emph{different} ways.%
\par\smallskip%
\noindent\textbf{\blocktitlefont Hint}.\quad{}You could consider cases.  For example, any number of the form ODD-ODD-EVEN will have an even sum.  Alternatively, how many three digit numbers have the sum of their digits even if the first two digits are 54?  What if the first two digits are 19?%
\par\smallskip%
\noindent\textbf{\blocktitlefont Solution}.\quad{}There are multiple ways to do this.%
\begin{enumerate}[label=(\alph*)]
\item{}An even sum can occur in 4 ways: EEE, EOO, OEO, and OOE. There are \(4 \cdot 5 \cdot 5\) ways to build numbers of the first two types (there are only 4 choices for a starting even number - it cannot be 0) and \(5 \cdot 5 \cdot 5\) ways to build the second two types. This gives a total of 450 numbers.%
\item{}To build a 3 digit number with an even sum, you can choose any of 9 digits for the first digit, any of 10 digits for the second digit. Then the last digit must either be even (if the sum of the first two digits are even) or odd (if the sum of the first two digits are odd). Luckily there are the same number of even last digits and odd last digits - 5. So there are a total of \(9 \cdot 10 \cdot 5 = 450\) numbers with an even sum of digits.%
\item{}Start finding sums of digits from 3-digit numbers: \(100 \to\) odd, \(101 \to\) even, \(102 \to\)odd, \(103 \to\) even, and so on. So the numbers appear to alternate between even and odd sums. However, notice that 109 has an even sum while 110 does as well. But then 111 is odd, 112 is even, and so on. So we can conclude that half of the numbers 100 to 109 have even sum, half of the number 110 to 119 have even sum, half from 120 to 129, and so on. This means that overall half of the numbers will have even sum, so half of the 900 3-digit numbers will have even sum, namely 450 of them.%
\end{enumerate}
%
\end{divisionsolution}%
\begin{divisionsolution}{1.1.14}{}{p:exercise:BQE}%
The number 735000 factors as \(2^3 \cdot 3 \cdot 5^4 \cdot 7^2\). How many divisors does it have? Explain your answer using the multiplicative principle.%
\par\smallskip%
\noindent\textbf{\blocktitlefont Hint}.\quad{}For a simpler example, there are 4 divisors of \(6 = 2\cdot 3\).  They are \(1 = 2^0\cdot 3^0\), \(2 = 2^1\cdot 3^0\), \(3 = 2^0\cdot 3^1\) and \(6 = 2^1\cdot 3^1\).%
\par\smallskip%
\noindent\textbf{\blocktitlefont Solution}.\quad{}If you consider the factorization of any divisor of 735000 it must have at most three 2s, at most one 3, at most four 5s and at most two 7s, with no other prime factors. Thus to select a divisor, we just need to pick how many of these prime factors are present. There are 4 choices for how many 2s to include (between zero and four), 2 choices for how many 3s, 5 choices for how many 5s and 3 choices for how many 7s. Thus the number of divisors is:%
\begin{equation*}
4\cdot 2 \cdot 5 \cdot 3 = 120
\end{equation*}
%
\end{divisionsolution}%
\section*{1.2 Binomial Coefficients}
\addcontentsline{toc}{section}{1.2 Binomial Coefficients}
\sectionmark{1.2 Binomial Coefficients}
\subsection*{Exercises}
\addcontentsline{toc}{subsection}{Exercises}
\begin{divisionsolution}{1.2.1}{}{p:exercise:Ubz}%
How many positive integers less than 1000 are multiples of 3, 5, or 7? Explain your answer using the Principle of Inclusion\slash{}Exclusion.%
\par\smallskip%
\noindent\textbf{\blocktitlefont Answer}.\quad{}\(542\)%
\par\smallskip%
\noindent\textbf{\blocktitlefont Solution}.\quad{}Since \(1000/3 = 333.33\text{,}\) there are 333 multiples of 3 less than 1000. There are 199 multiples of 5 (strictly) less than 1000. There are 142 multiples of 7 less than 1000.%
\par
We also need the combinations of these. To be a multiple of 3 and 5 means you are a multiple of 15, and \(1000/15 = 66.67\) so there ar 66 multiples of 3 and 5. There will be 47 multiples of 3 and 7. There will be 28 multiples of 5 and 7. Finally, there will be 9 multiples of all three.%
\par
Using PIE, we get%
\par
%
\begin{equation*}
333+199 + 142 - 66 - 47 - 28 + 9 = 542
\end{equation*}
%
\par
multiples of 3, 5, or 7 less than 1000.%
\end{divisionsolution}%
\begin{divisionsolution}{1.2.2}{}{p:exercise:AiI}%
Consider all 5 letter ``words'' made from the letters \(a\) through \(h\text{.}\) (Recall, words are just strings of letters, not necessarily actual English words.)%
\begin{enumerate}[label=(\alph*)]
\item{}How many of these words are there total?%
\item{}How many of these words contain no repeated letters?%
\item{}How many of these words start with the sub-word ``aha''?%
\item{}How many of these words either start with ``aha'' or end with ``bah'' or both?%
\item{}How many of the words containing no repeats also do not contain the sub-word ``bad''?%
\end{enumerate}
%
\par\smallskip%
\noindent\textbf{\blocktitlefont Answer 1}.\quad{}\(32768\)%
\par\smallskip%
\noindent\textbf{\blocktitlefont Answer 2}.\quad{}\(6720\)%
\par\smallskip%
\noindent\textbf{\blocktitlefont Answer 3}.\quad{}\(64\)%
\par\smallskip%
\noindent\textbf{\blocktitlefont Answer 4}.\quad{}\(128\)%
\par\smallskip%
\noindent\textbf{\blocktitlefont Answer 5}.\quad{}\(6660\)%
\par\smallskip%
\noindent\textbf{\blocktitlefont Solution 1}.\quad{}%
\begin{enumerate}[label=(\alph*)]
\item{}\(8^5 = 32768\) words.%
\item{}\(8\cdot 7\cdot 6\cdot 5\cdot 4 = 6720\) words.%
\item{}\(8 \cdot 8 =64\) words.%
\item{}\(64 + 64 - 0 = 128\) words.%
\item{}\((8\cdot 7\cdot 6\cdot 5\cdot 4) - 3\cdot (5\cdot 4) = 6660\) words.%
\end{enumerate}
%
\par\smallskip%
\noindent\textbf{\blocktitlefont Solution 2}.\quad{}%
\begin{enumerate}[label=(\alph*)]
\item{}\(8^5 = 32768\) words, since you select from 8 letters 5 times.%
\item{}\(8\cdot 7\cdot 6\cdot 5\cdot 4 = 6720\) words. After selecting a letter, you have fewer letters to select for the next one.%
\item{}\(8 \cdot 8 =64\) words: you need to select the 4th and 5th letters.%
\item{}\(64 + 64 - 0 = 128\) words. There are 64 words which start with ``aha'' and another 64 words that end with ``bah.'' Perhaps we over counted the words that both start with ``aha'' and end with ``bah'', but since the words are only 5 letters long, there are no such words.%
\item{}\((8\cdot 7\cdot 6\cdot 5\cdot 4) - 3\cdot (5\cdot 4) = 6660\) words. All the words minus the bad ones. The taboo word can be in any of three positions (starting with letter 1, 2, or 3) and for each position we must choose the other two letters (from the remaining 5 letters).%
\end{enumerate}
%
\end{divisionsolution}%
\begin{divisionsolution}{1.2.3}{}{p:exercise:gpR}%
Let \(S = \{1, 2, 3, 4, 5, 6\}\)%
\begin{enumerate}[label=(\alph*)]
\item{}How many subsets are there total?%
\item{}How many subsets have \(\{2,3,5\}\) as a subset?%
\item{}How many subsets contain at least one odd number?%
\item{}How many subsets contain exactly one even number?%
\end{enumerate}
%
\par\smallskip%
\noindent\textbf{\blocktitlefont Answer 1}.\quad{}\(64\)%
\par\smallskip%
\noindent\textbf{\blocktitlefont Answer 2}.\quad{}\(8\)%
\par\smallskip%
\noindent\textbf{\blocktitlefont Answer 3}.\quad{}\(56\)%
\par\smallskip%
\noindent\textbf{\blocktitlefont Answer 4}.\quad{}\(24\)%
\par\smallskip%
\noindent\textbf{\blocktitlefont Solution}.\quad{}%
\begin{enumerate}[label=(\alph*)]
\item{}\(2^6 = 64\) subsets. We need to select yes\slash{}no for each of the six elements.%
\item{}\(2^3 = 8\) subsets. We need to select yes\slash{}no for each of the remaining three elements.%
\item{}\(2^6 - 2^3 = 56\) subsets. There are 8 subsets which do not contain any odd numbers (select yes\slash{}no for each even number).%
\item{}\(3\cdot 2^3 = 24\) subsets. First pick the even number. Then say yes or no to each of the odd numbers.%
\end{enumerate}
%
\end{divisionsolution}%
\begin{divisionsolution}{1.2.4}{}{p:exercise:Mxa}%
Let \(S = \{1, 2, 3, 4, 5, 6\}\)%
\begin{enumerate}[label=(\alph*)]
\item{}How many subsets are there of cardinality 4?%
\item{}How many subsets of cardinality 4 have \(\{2,3,5\}\) as a subset?%
\item{}How many subsets of cardinality 4 contain at least one odd number?%
\item{}How many subsets of cardinality 4 contain exactly one even number?%
\end{enumerate}
%
\par\smallskip%
\noindent\textbf{\blocktitlefont Answer 1}.\quad{}\(3\)%
\par\smallskip%
\noindent\textbf{\blocktitlefont Answer 2}.\quad{}\(15\)%
\par\smallskip%
\noindent\textbf{\blocktitlefont Answer 3}.\quad{}\(3\)%
\par\smallskip%
\noindent\textbf{\blocktitlefont Solution}.\quad{}%
\begin{enumerate}[label=(\alph*)]
\item{}\({6\choose 4} = 15\) subsets.%
\item{}\({3 \choose 1} = 3\) subsets. We need to select 1 of the 3 remaining elements to be in the subset.%
\item{}\({6 \choose 4} = 15\) subsets. All subsets of cardinality 4 must contain at least one odd number.%
\item{}\({3 \choose 1} = 3\) subsets. Select 1 of the 3 even numbers. The remaining three odd numbers of \(S\) must all be in the set.%
\end{enumerate}
%
\end{divisionsolution}%
\begin{divisionsolution}{1.2.5}{}{p:exercise:sEj}%
Let \(A = \{1,2,3,\ldots,9\}\text{.}\)%
\begin{enumerate}[label=(\alph*)]
\item{}How many subsets of \(A\) are there? That is, find \(|\pow(A)|\text{.}\) Explain.%
\item{}How many subsets of \(A\) contain exactly 5 elements? Explain.%
\item{}How many subsets of \(A\) contain only even numbers? Explain.%
\item{}How many subsets of \(A\) contain an even number of elements? Explain.%
\end{enumerate}
%
\par\smallskip%
\noindent\textbf{\blocktitlefont Answer 1}.\quad{}\(2^{9}\)%
\par\smallskip%
\noindent\textbf{\blocktitlefont Answer 2}.\quad{}\(\mathop{\rm C}\nolimits\!\left(9,5\right)\)%
\par\smallskip%
\noindent\textbf{\blocktitlefont Answer 3}.\quad{}\(2^{4}\)%
\par\smallskip%
\noindent\textbf{\blocktitlefont Answer 4}.\quad{}\(\mathop{\rm C}\nolimits\!\left(9,0\right)+\mathop{\rm C}\nolimits\!\left(9,2\right)+\mathop{\rm C}\nolimits\!\left(9,4\right)+\mathop{\rm C}\nolimits\!\left(9,6\right)+\mathop{\rm C}\nolimits\!\left(9,8\right)\)%
\par\smallskip%
\noindent\textbf{\blocktitlefont Solution 1}.\quad{}%
\begin{enumerate}[label=(\alph*)]
\item{}There are \(512\) subsets.%
\item{}\({9 \choose 5} = 126\text{.}\)%
\item{}\(2^4 = 16\text{.}\) (Note, if you wish to exclude the empty set - it does not contain odd numbers, but no evens either - then you could subtract 1).%
\item{}\(256\text{.}\)%
\end{enumerate}
%
\par\smallskip%
\noindent\textbf{\blocktitlefont Solution 2}.\quad{}%
\begin{enumerate}[label=(\alph*)]
\item{}There are \(512\) subsets. This is \(2^9\text{,}\) which makes sense because we are deciding yes or no on whether to include each element of \(A\) in the subset.%
\item{}Of the nine elements in \(A\text{,}\) we must choose five of them to be in the subset. So \({9 \choose 5} = 126\text{.}\)%
\item{}For each of the 9 elements from \(A\text{,}\) we must decide yes or no on whether to include them in the subset. However, for the odd numbers, we only have one choice: no. So there are only 4 elements we have two choices for, so the answer is \(2^4 = 16\text{.}\) (Note, if you wish to exclude the empty set - it does not contain odd numbers, but no evens either - then you could subtract 1).%
\item{}Count the number of subsets with each possible even cardinality:%
\par
%
\begin{equation*}
{9 \choose 0} + {9 \choose 2} + {9\choose 4} + {9 \choose 6} + {9 \choose 8} = 256
\end{equation*}
%
\end{enumerate}
%
\end{divisionsolution}%
\begin{divisionsolution}{1.2.6}{}{p:exercise:YLs}%
How many \(9\)-bit strings (that is, bit strings of length 9) are there which:%
\begin{enumerate}[label=(\alph*)]
\item{}Start with the sub-string 101? Explain.%
\item{}Have weight 5 (i.e., contain exactly five 1's) and start with the sub-string 101? Explain.%
\item{}Either start with \(101\) or end with \(11\) (or both)? Explain.%
\item{}Have weight 5 and either start with 101 or end with 11 (or both)? Explain.%
\end{enumerate}
%
\par\smallskip%
\noindent\textbf{\blocktitlefont Answer 1}.\quad{}\(64\)%
\par\smallskip%
\noindent\textbf{\blocktitlefont Answer 2}.\quad{}\(20\)%
\par\smallskip%
\noindent\textbf{\blocktitlefont Answer 3}.\quad{}\(176\)%
\par\smallskip%
\noindent\textbf{\blocktitlefont Answer 4}.\quad{}\(51\)%
\par\smallskip%
\noindent\textbf{\blocktitlefont Solution 1}.\quad{}%
\begin{enumerate}[label=(\alph*)]
\item{}\(2^6 = 64\text{.}\)%
\item{}\({6 \choose 3} = 20\text{.}\)%
\item{}176.%
\item{}51.%
\end{enumerate}
%
\par\smallskip%
\noindent\textbf{\blocktitlefont Solution 2}.\quad{}%
\begin{enumerate}[label=(\alph*)]
\item{}\(2^6 = 64\text{.}\) You have 2 choices for each of the remaining 6 bits.%
\item{}\({6 \choose 3} = 20\text{.}\) You need to choose 3 of the remaining 6 bits to be 1's.%
\item{}176. There are 64 strings that start with 101, and another 128 which end with 11 (we choose 1 or 0 for 7 bits, so \(2^7\)). However, we count the strings that start with 101 and end with 11 twice - there are \(16\) such strings (\(2^4\)). So using PIE, we have \(64 + 128 - 16 = 176\)%
\item{}51. There are \({6 \choose 3} = 20\) strings of weight 5 which start with 101, and another \({7 \choose 3} = 35\) which end with 11. We have over counted again - there are weight 5 strings which both start with 101 and end with 11, in fact \({4 \choose 1} = 4\) of them. So all together we have \(20 + 35 - 4 = 51\) strings.%
\end{enumerate}
%
\end{divisionsolution}%
\begin{divisionsolution}{1.2.7}{}{p:exercise:ESB}%
You break your piggy-bank to discover lots of pennies and nickels. You start arranging these in rows of 6 coins.%
\begin{enumerate}[label=(\alph*)]
\item{}You find yourself making rows containing an equal number of pennies and nickels. For fun, you decide to lay out every possible such row. How many coins will you need?%
\item{}How many coins would you need to make all possible rows of 6 coins (not necessarily with equal number of pennies and nickels)?%
\end{enumerate}
%
\par\smallskip%
\noindent\textbf{\blocktitlefont Hint}.\quad{}Pennies are sort of like 0's and nickels are sort of like 1's.%
\par\smallskip%
\noindent\textbf{\blocktitlefont Answer 1}.\quad{}\(120\)%
\par\smallskip%
\noindent\textbf{\blocktitlefont Answer 2}.\quad{}\(384\)%
\par\smallskip%
\noindent\textbf{\blocktitlefont Solution 1}.\quad{}%
\begin{enumerate}[label=(\alph*)]
\item{}We will need \(6\cdot 20 = 120\) coins (60 of each).%
\item{}We need \(6 \cdot 64 = 384\) coins (192 of each).%
\end{enumerate}
%
\par\smallskip%
\noindent\textbf{\blocktitlefont Solution 2}.\quad{}%
\begin{enumerate}[label=(\alph*)]
\item{}We can think of each row as a 6-bit string of weight 3 (since of the 6 coins, we require 3 to be pennies). Thus there are \({6 \choose 3} = 20\) rows possible. Each row requires 6 coins, so if we want to make all the rows at the same time, we will need 120 coins (60 of each).%
\item{}Now there are \(2^6 = 64\) rows possible, which is also \({6 \choose 0} + {6\choose 1} + {6 \choose 2} + {6 \choose 3} + {6 \choose 4} + {6 \choose 5} + {6 \choose 6}\text{,}\) if you break them up into rows containing 0, 1, 2, etc. pennies. Thus we need \(6 \cdot 64 = 384\) coins (192 of each).%
\end{enumerate}
%
\end{divisionsolution}%
\begin{divisionsolution}{1.2.8}{}{p:exercise:kZK}%
How many 10-bit strings contain 6 or more 1's?%
\par\smallskip%
\noindent\textbf{\blocktitlefont Answer}.\quad{}\(\mathop{\rm C}\nolimits\!\left(10,6\right)+\mathop{\rm C}\nolimits\!\left(10,7\right)+\mathop{\rm C}\nolimits\!\left(10,8\right)+\mathop{\rm C}\nolimits\!\left(10,9\right)+\mathop{\rm C}\nolimits\!\left(10,10\right)\)%
\par\smallskip%
\noindent\textbf{\blocktitlefont Solution}.\quad{}\({10 \choose 6} + {10\choose 7} + {10\choose 8} + {10 \choose 9} + {10\choose 10} = 386\) strings.%
\end{divisionsolution}%
\begin{divisionsolution}{1.2.9}{}{p:exercise:RgT}%
How many subsets of \(\{0,1,\ldots, 9\}\) have cardinality 6 or more?%
\par\smallskip%
\noindent\textbf{\blocktitlefont Hint}.\quad{}Break the question into five cases.%
\par\smallskip%
\noindent\textbf{\blocktitlefont Answer}.\quad{}\(386\)%
\par\smallskip%
\noindent\textbf{\blocktitlefont Solution}.\quad{}\({10 \choose 6} + {10\choose 7} + {10\choose 8} + {10 \choose 9} + {10\choose 10} = 386\) subsets. This is the same as the previous question, since we can think of each subset as a 10-bit string with a 1 representing that we include that element in the subset.%
\end{divisionsolution}%
\begin{divisionsolution}{1.2.10}{}{p:exercise:xoc}%
What is the coefficient of \(x^{12}\) in \((x+2)^{15}\text{?}\)%
\par\smallskip%
\noindent\textbf{\blocktitlefont Answer}.\quad{}\(\mathop{\rm C}\nolimits\!\left(15,12\right)\cdot 2^{3}\)%
\par\smallskip%
\noindent\textbf{\blocktitlefont Solution 1}.\quad{}\({15\choose 12}2^3 = 3640\text{.}\)%
\par\smallskip%
\noindent\textbf{\blocktitlefont Solution 2}.\quad{}To get an \(x^{12}\text{,}\) we must pick 12 of the 15 factors to contribute an \(x\text{,}\) leaving the other 3 to contribute a 2. There are \({15 \choose 12}\) ways to select these 12 factors. So the term containing an \(x^{12}\) will be \({15 \choose 12}x^{12}2^{3}\text{.}\) In other words, the coefficient of \(x^{12}\) is \({15\choose 12}2^3 = 3640\text{.}\)%
\end{divisionsolution}%
\begin{divisionsolution}{1.2.11}{}{p:exercise:dvl}%
Gridtown USA, besides having excellent donut shops, is known for its precisely laid out grid of streets and avenues. Streets run east-west, and avenues north-south, for the entire stretch of the town, never curving and never interrupted by parks or schools or the like.%
\par
Suppose you live on the corner of 3rd and 3rd and work on the corner of 12th and 12th. Thus you must travel 18 blocks to get to work as quickly as possible.%
\begin{enumerate}[label=(\alph*)]
\item{}How many different routes can you take to work, assuming you want to get there as quickly as possible? Explain.%
\item{}Now suppose you want to stop and get a donut on the way to work, from your favorite donut shop on the corner of 10th ave and 8th st. How many routes to work, stopping at the donut shop, can you take (again, ensuring the shortest possible route)? Explain.%
\item{}Disaster Strikes Gridtown: there is a pothole on 4th ave between 5th st and 6th st. How many routes to work can you take avoiding that unsightly (and dangerous) stretch of road? Explain.%
\item{}The pothole has been repaired (phew) and a new donut shop has opened on the corner of 4th ave and 5th st. How many routes to work drive by one or the other (or both) donut shops? Hint: the donut shops serve PIE.%
\end{enumerate}
%
\par\smallskip%
\noindent\textbf{\blocktitlefont Solution 1}.\quad{}%
\begin{enumerate}[label=(\alph*)]
\item{}\({18 \choose 9}\).%
\item{}\(\binom{12}{7}\binom{6}{2}\).%
\item{}\({18 \choose 9} - {3 \choose 1}{14 \choose 8}\)%
\item{}\({3\choose 1}{15 \choose 8} + {12 \choose 7}{6 \choose 2} - {3\choose 1}{9 \choose 6}{6 \choose 2}\)%
\end{enumerate}
%
\par\smallskip%
\noindent\textbf{\blocktitlefont Solution 2}.\quad{}%
\begin{enumerate}[label=(\alph*)]
\item{}\({18 \choose 9}\) since you must choose 9 of the 18 blocks to travel east.%
\item{}The donut shop is 12 blocks away, 5 one way, 7 the other. So to get from home to the donut shop, there are \({12 \choose 7}\) routes (or equivalently, \({12 \choose 5}\) ). Then from the donut shopd to work, you need to travel 6 more blocks, 2 on way and 4 the other. So there are \({6 \choose 2}\) (or \({6 \choose 4}\) ) routes from the donut shop to work.%
\par
For each of the ways to the donut shop, there are so many ways to work, so the multiplicative principle says the total number of ways from home to work via the donut shop is%
\begin{equation*}
\binom{12}{7}\binom{6}{2}
\end{equation*}
%
\item{}Routes to work that hit the pothole: \({3 \choose 1}1{14 \choose 8}\).%
\par
There for the number of routes to work which \emph{avoid} the pothole are%
\begin{equation*}
{18 \choose 9} - {3 \choose 1}{14 \choose 8}
\end{equation*}
%
\item{}The routes to work past the donut shop at (4,5): \({3\choose 1}{15 \choose 8}\). The routes to work past the donut shop at (10,8): \({12 \choose 7}{6 \choose 2}\). The routes to work past both: \({3\choose 1}{9 \choose 6}{6 \choose 2}\). So all together, using PIE:%
\begin{equation*}
{3\choose 1}{15 \choose 8} + {12 \choose 7}{6 \choose 2} - {3\choose 1}{9 \choose 6}{6 \choose 2}
\end{equation*}
%
\end{enumerate}
%
\end{divisionsolution}%
\begin{divisionsolution}{1.2.12}{}{p:exercise:JCu}%
What is the coefficient of \(x^9\) in the expansion of \((x+1)^{14} + x^3(x+2)^{15}\text{?}\)%
\par\smallskip%
\noindent\textbf{\blocktitlefont Answer}.\quad{}\(\mathop{\rm C}\nolimits\!\left(14,9\right)+\mathop{\rm C}\nolimits\!\left(15,6\right)\cdot 2^{9}\)%
\par\smallskip%
\noindent\textbf{\blocktitlefont Solution}.\quad{}\({14\choose 9} + {15 \choose 6}2^9\text{.}\)%
\end{divisionsolution}%
\begin{divisionsolution}{1.2.13}{}{p:exercise:pJD}%
Explain why the coefficient of \(x^5y^3\) the same as the coefficient of \(x^3y^5\) in the expansion of \((x+y)^8\)?%
\par\smallskip%
\noindent\textbf{\blocktitlefont Solution}.\quad{}The coefficient of \(x^5y^3\) is \({8\choose 5}\), since we must pick 5 of the 8 factors to contribute an \(x\). The coefficient of \(x^3y^5\) is \({8 \choose 3}\), since we pick 3 out of the 8 factors to contribute an \(x\). But \({8 \choose 5} = {8\choose 3}\), because we could just as easily have picked 5 out of the 8 factors to contribute a \(y\).%
\end{divisionsolution}%
\section*{1.3 Combinations and Permutations}
\addcontentsline{toc}{section}{1.3 Combinations and Permutations}
\sectionmark{1.3 Combinations and Permutations}
\subsection*{Exercises}
\addcontentsline{toc}{subsection}{Exercises}
\begin{divisionsolution}{1.3.1}{}{p:exercise:TyW}%
How many lattice paths start at (3,3) and%
\begin{enumerate}[label=(\alph*)]
\item{}end at (10,10)?%
\item{}end at (10,10) and pass through (5,7)?%
\item{}end at (10,10) and avoid (5,7)?%
\end{enumerate}
%
\par\smallskip%
\noindent\textbf{\blocktitlefont Answer 1}.\quad{}\(\mathop{\rm C}\nolimits\!\left(14,7\right)\)%
\par\smallskip%
\noindent\textbf{\blocktitlefont Answer 2}.\quad{}\(\mathop{\rm C}\nolimits\!\left(6,2\right)\mathop{\rm C}\nolimits\!\left(8,5\right)\)%
\par\smallskip%
\noindent\textbf{\blocktitlefont Answer 3}.\quad{}\(\mathop{\rm C}\nolimits\!\left(14,7\right)-\mathop{\rm C}\nolimits\!\left(6,2\right)\mathop{\rm C}\nolimits\!\left(8,5\right)\)%
\par\smallskip%
\noindent\textbf{\blocktitlefont Solution 1}.\quad{}%
\begin{enumerate}[label=(\alph*)]
\item{}\({14 \choose 7} = 3432\) paths.%
\item{}\({6 \choose 2}{8\choose 5} = 840\) paths.%
\item{}\({14 \choose 7} - {6\choose 2}{8 \choose 5}\) paths.%
\end{enumerate}
%
\par\smallskip%
\noindent\textbf{\blocktitlefont Solution 2}.\quad{}%
\begin{enumerate}[label=(\alph*)]
\item{}\({14 \choose 7} = 3432\) paths. The paths all have length 14 (7 steps up and 7 steps right), we just select which 7 of those 14 should be up.%
\item{}\({6 \choose 2}{8\choose 5} = 840\) paths. First travel to (5,7), and then continue on to (10,10).%
\item{}\({14 \choose 7} - {6\choose 2}{8 \choose 5}\) paths. Remove all the paths that you found in part (b).%
\end{enumerate}
%
\end{divisionsolution}%
\begin{divisionsolution}{1.3.2}{}{p:exercise:zGf}%
Suppose you are ordering a large pizza from \emph{D.P.~Dough}. You want 3 distinct toppings, chosen from their list of 11 vegetarian toppings.%
\begin{enumerate}[label=(\alph*)]
\item{}How many choices do you have for your pizza?%
\item{}How many choices do you have for your pizza if you refuse to have pineapple as one of your toppings?%
\item{}How many choices do you have for your pizza if you \emph{insist} on having pineapple as one of your toppings?%
\item{}How do the three questions above relate to each other? Explain.%
\end{enumerate}
%
\par\smallskip%
\noindent\textbf{\blocktitlefont Answer 1}.\quad{}\(\mathop{\rm C}\nolimits\!\left(11,3\right)\)%
\par\smallskip%
\noindent\textbf{\blocktitlefont Answer 2}.\quad{}\(\mathop{\rm C}\nolimits\!\left(10,3\right)\)%
\par\smallskip%
\noindent\textbf{\blocktitlefont Answer 3}.\quad{}\(\mathop{\rm C}\nolimits\!\left(10,2\right)\)%
\par\smallskip%
\noindent\textbf{\blocktitlefont Solution}.\quad{}%
\begin{enumerate}[label=(\alph*)]
\item{}\({11 \choose 3} = 165\) choices, since you have to select a 3-element subset of the set of 11 toppings.%
\item{}\({10 \choose 3} = 120\) choices, since you must select 3 of the 10 non-pineapple toppings.%
\item{}\({10 \choose 2} = 45\) choices, since you must select 2 of the remaining 10 non-pineapple toppings to have in addition to the pineapple.%
\item{}\(165 = 120 + 45\) choices, which makes sense because every 3-topping pizza either has pineapple or does not have pineapple as a topping.%
\end{enumerate}
%
\end{divisionsolution}%
\begin{divisionsolution}{1.3.3}{}{p:exercise:fNo}%
A pizza parlor offers 10 toppings.%
\begin{enumerate}[label=(\alph*)]
\item{}How many 3-topping pizzas could they put on their menu? Assume double toppings are not allowed.%
\item{}How many total pizzas are possible, with between zero and ten toppings (but not double toppings) allowed?%
\item{}The pizza parlor will list the 10 toppings in two equal-sized columns on their menu. How many ways can they arrange the toppings in the left column?%
\end{enumerate}
%
\par\smallskip%
\noindent\textbf{\blocktitlefont Answer 1}.\quad{}\(\mathop{\rm C}\nolimits\!\left(10,3\right)\)%
\par\smallskip%
\noindent\textbf{\blocktitlefont Answer 2}.\quad{}\(2^{10}\)%
\par\smallskip%
\noindent\textbf{\blocktitlefont Answer 3}.\quad{}\(\mathop{\rm P}\nolimits\!\left(10,5\right)\)%
\par\smallskip%
\noindent\textbf{\blocktitlefont Solution}.\quad{}%
\begin{enumerate}[label=(\alph*)]
\item{}\({10 \choose 3} = 120\) pizzas. We must choose (in no particular order) 3 out of the 10 toppings.%
\item{}\(2^{10} = 1024\) pizzas. Say yes or no to each topping.%
\item{}\(P(10,5) = 30240\) ways. Assign each of the 5 spots in the left column to a unique pizza topping.%
\end{enumerate}
%
\end{divisionsolution}%
\begin{divisionsolution}{1.3.4}{}{p:exercise:LUx}%
In an attempt to clean up your room, you have purchased a new floating shelf to put some of your 17 books you have stacked in a corner. These books are all by different authors. The new book shelf is large enough to hold 10 of the books.%
\begin{enumerate}[label=(\alph*)]
\item{}How many ways can you select and arrange 10 of the 17 books on the shelf? Notice that here we will allow the books to end up in any order. Explain.%
\item{}How many ways can you arrange 10 of the 17 books on the shelf if you insist they must be arranged alphabetically by author? Explain.%
\end{enumerate}
%
\par\smallskip%
\noindent\textbf{\blocktitlefont Hint}.\quad{}Which question should have the larger answer?  One of these is a combination, the other is a permutation.%
\end{divisionsolution}%
\begin{divisionsolution}{1.3.5}{}{p:exercise:sbG}%
Suppose you wanted to draw a quadrilateral using the dots below as vertices (corners). The dots are spaced one unit apart horizontally and two units apart vertically.%
\begin{sidebyside}{1}{0.23}{0.23}{0}%
\begin{sbspanel}{0.54}%
\resizebox{\linewidth}{!}{%
\begin{tikzpicture}[scale=.9]
  \foreach \x in {-3,...,3}
  \foreach \y in {-1,1}
  \fill (\x,\y) circle (3pt);
\end{tikzpicture}
}%
\end{sbspanel}%
\end{sidebyside}%
\par\medskip
A combination lock consists of a dial with 40 numbers on it. To open the lock, you turn the dial to the right until you reach a first number, then to the left until you get to second number, then to the right again to the third number. The numbers must be distinct. How many different combinations are possible?%
\par\smallskip%
\noindent\textbf{\blocktitlefont Answer}.\quad{}\(\mathop{\rm P}\nolimits\!\left(40,3\right)\)%
\par\smallskip%
\noindent\textbf{\blocktitlefont Solution}.\quad{}Despite its name, we are not looking for a combination here. The order in which the three numbers appears matters. There are \(P(40,3) = 40\cdot 39 \cdot 38\) different possibilities for the ``combination''. This is assuming you cannot repeat any of the numbers (if you could, the answer would be \(40^3\)).%
\end{divisionsolution}%
\begin{divisionsolution}{1.3.6}{}{p:exercise:YiP}%
How many triangles are there with vertices from the points shown below? Note, we are not allowing degenerate triangles - ones with all three vertices on the same line, but we do allow non-right triangles. Explain why your answer is correct.%
\begin{sidebyside}{1}{0.25}{0.25}{0}%
\begin{sbspanel}{0.5}%
\resizebox{\linewidth}{!}{%
\begin{tikzpicture}[scale=0.7]
  \foreach \i in {0,...,6} {
    \fill (\i,0) circle (2pt);
  }
  \foreach \i in {1,...,4} {
    \fill (0,\i) circle (2pt);
  }
\end{tikzpicture}
}%
\end{sbspanel}%
\end{sidebyside}%
\par\smallskip%
\noindent\textbf{\blocktitlefont Hint}.\quad{}If you pick any three points, you can get a triangle, unless those three points are all on the \(x\)-axis or on the \(y\)-axis.  There are other ways to start this as well, and any correct method should give the same answer.%
\par\smallskip%
\noindent\textbf{\blocktitlefont Solution 1}.\quad{}There are 120 triangles. Here are two ways (there are others as well) to get this:%
\begin{enumerate}[label=(\alph*)]
\item{}First count the triangles with the base on the \(x\)-axis. There are \({7 \choose 2}\) ways to pick the base. The third vertex of the triangle must be one of the 4 dots on the \(y\)-axis (not the origin) so there are a total of \({7 \choose 2}4\) of these triangles. The triangles with base on the \(y\) axis can be counted similarly: \({5 \choose 2}6\). However, we have counted all the right triangles twice - they have a base on the \(x\)-axis and also on the \(y\)-axis. There are \(4 \cdot 6\) right triangles. Thus the total number of triangles is:%
\begin{equation*}
{7 \choose 2}4 + {5 \choose 2}6 - 6\cdot 4 = 120
\end{equation*}
%
\item{}We must select 3 of the 11 dots. This can be done in \({11 \choose 3}\) ways. However, this will also give us degenerate triangles when all three vertices are on the \(x\)-axis or on the \(y\)-axis. There are \({7 \choose 3}\) ways we could have picked all three vertices on the \(x\)-axis. There are \({5 \choose 3}\) ways we could have picked all three vertices on the \(y\)-axis. Therefore the total number of triangles is%
\begin{equation*}
{11 \choose 3} - {7 \choose 3} - {5 \choose 3} = 120
\end{equation*}
%
\end{enumerate}
%
\par\smallskip%
\noindent\textbf{\blocktitlefont Solution 2}.\quad{}120.%
\end{divisionsolution}%
\begin{divisionsolution}{1.3.7}{}{p:exercise:EpY}%
Using the digits 2 through 8, find the number of different 5-digit numbers such that:%
\begin{enumerate}[label=(\alph*)]
\item{}Digits can be used more than once.%
\item{}Digits cannot be repeated, but can come in any order.%
\item{}Digits cannot be repeated and must be written in increasing order.%
\item{}Which of the above counting questions is a combination and which is a permutation? Explain why this makes sense.%
\end{enumerate}
%
\par\smallskip%
\noindent\textbf{\blocktitlefont Answer 1}.\quad{}\(7^{5}\)%
\par\smallskip%
\noindent\textbf{\blocktitlefont Answer 2}.\quad{}\(\mathop{\rm P}\nolimits\!\left(7,5\right)\)%
\par\smallskip%
\noindent\textbf{\blocktitlefont Answer 3}.\quad{}\(\mathop{\rm C}\nolimits\!\left(7,5\right)\)%
\par\smallskip%
\noindent\textbf{\blocktitlefont Solution}.\quad{}%
\begin{enumerate}[label=(\alph*)]
\item{}This is just the multiplicative principle. There are 7 digits which we can select for each of the 5 positions, so we have \(7^5 = 16807\) such numbers.%
\item{}Now we have 7 choices for the first number, 6 for the second, etc. So there are \(7 \cdot 6 \cdot 5 \cdot 4 \cdot 3 = P(7,5) = 2520\) such numbers.%
\item{}To build such a number we simply must select 5 different digits. After doing so, there will only be one way to arrange them into a 5-digit number. Thus there are \({7 \choose 5} = 21\) such numbers.%
\item{}The permutation is in part (b), while the combination is in part (c). At first this seems backwards, since usually we use combinations for when order does not matter. Here it looks like in part (c) that order does matter. The better way to distinguish between combinations and permutations is to ask whether we are counting different arrangements as different outcomes. In part (c), there is only one arrangement of any set of 5 digits, while in part (b) each set of 5 digits gives \(5!\) different outcomes.%
\end{enumerate}
%
\end{divisionsolution}%
\begin{divisionsolution}{1.3.8}{}{p:exercise:kxh}%
How many quadrilaterals are possible?%
\par
How many are squares?%
\par
How many are rectangles?%
\par
How many are parallelograms?%
\par
How many are trapezoids? (Here, as in calculus, a trapezoid is defined as a quadrilateral with \emph{at least} one pair of parallel sides. In particular, parallelograms are trapezoids.)%
\par
How many are trapezoids that are not parallelograms?%
\par\smallskip%
\noindent\textbf{\blocktitlefont Answer 1}.\quad{}\(\mathop{\rm C}\nolimits\!\left(7,2\right)\mathop{\rm C}\nolimits\!\left(7,2\right)\)%
\par\smallskip%
\noindent\textbf{\blocktitlefont Answer 2}.\quad{}\(5\)%
\par\smallskip%
\noindent\textbf{\blocktitlefont Answer 3}.\quad{}\(\mathop{\rm C}\nolimits\!\left(7,2\right)\)%
\par\smallskip%
\noindent\textbf{\blocktitlefont Answer 4}.\quad{}\(91\)%
\par\smallskip%
\noindent\textbf{\blocktitlefont Answer 5}.\quad{}\(\mathop{\rm C}\nolimits\!\left(7,2\right)\mathop{\rm C}\nolimits\!\left(7,2\right)\)%
\par\smallskip%
\noindent\textbf{\blocktitlefont Answer 6}.\quad{}\(441-91\)%
\par\smallskip%
\noindent\textbf{\blocktitlefont Solution 1}.\quad{}You can make \({7\choose 2}{7\choose 2} = 441\) quadrilaterals.%
\par
There are 5 squares.%
\par
There are \({7 \choose 2}\) rectangles.%
\par
There are \({7 \choose 2} + ({7 \choose 2}-1) + ({7 \choose 2} - 3) + ({7 \choose 2} - 6) + ({7 \choose 2} - 10) + ({7 \choose 2} - 15) = 91\) parallelograms.%
\par
All of the quadrilaterals are trapezoids. To count the non-parallelogram trapezoids, compute \({7\choose 2}{7\choose 2} - \left[ {7 \choose 2} + ({7 \choose 2}-1) + ({7 \choose 2} - 3) + ({7 \choose 2} - 6) + ({7 \choose 2} - 10) + ({7 \choose 2} - 15) \right]\text{.}\)%
\par\smallskip%
\noindent\textbf{\blocktitlefont Solution 2}.\quad{}\({7\choose 2}{7\choose 2} = 441\) quadrilaterals. We must pick two of the seven dots from the top row and two of the seven dots on the bottom row. However, it does not make a difference which of the two (on each row) we pick first because once these four dots are selected, there is exactly one quadrilateral that they determine.%
\par
There are 5 squares. You need to skip exactly one dot on the top and on the bottom to make the side lengths equal. Once you pick a dot on the top, the other three dots are determined.%
\par
There are \({7 \choose 2}\) rectangles. Once you select the two dots on the top, the bottom two are determined.%
\par
This is tricky since you need to worry about running out of space. One way to count: break into cases by the location of the top left corner. You get \({7 \choose 2} + ({7 \choose 2}-1) + ({7 \choose 2} - 3) + ({7 \choose 2} - 6) + ({7 \choose 2} - 10) + ({7 \choose 2} - 15) = 91\) parallelograms.%
\par
All of the quadralaterals are trapezoids. To count the non-parallelogram trapezoinds, compute \({7\choose 2}{7\choose 2} - \left[ {7 \choose 2} + ({7 \choose 2}-1) + ({7 \choose 2} - 3) + ({7 \choose 2} - 6) + ({7 \choose 2} - 10) + ({7 \choose 2} - 15) \right]\text{.}\)%
\end{divisionsolution}%
\begin{divisionsolution}{1.3.9}{}{p:exercise:QEq}%
An \emph{anagram} of a word is just a rearrangement of its letters. How many different anagrams of ``uncopyrightable'' are there? (This happens to be the longest common English word without any repeated letters.)%
\par\smallskip%
\noindent\textbf{\blocktitlefont Answer}.\quad{}\(15!\)%
\par\smallskip%
\noindent\textbf{\blocktitlefont Solution}.\quad{}Since there are 15 different letters, we have 15 choices for the first letter, 14 for the next, and so on. Thus there are \(15!\) anagrams.%
\end{divisionsolution}%
\begin{divisionsolution}{1.3.10}{}{p:exercise:wLz}%
How many anagrams are there of the word ``assesses'' that start with the letter ``a''?%
\par\smallskip%
\noindent\textbf{\blocktitlefont Hint}.\quad{}We just need a string of 7 letters: 4 of one type, 3 of the other.%
\par\smallskip%
\noindent\textbf{\blocktitlefont Answer}.\quad{}\(\mathop{\rm C}\nolimits\!\left(7,2\right)\)%
\par\smallskip%
\noindent\textbf{\blocktitlefont Solution 1}.\quad{}There are \({7 \choose 2} = 21\) anagrams starting with ``a''.%
\par\smallskip%
\noindent\textbf{\blocktitlefont Solution 2}.\quad{}After the first letter (a), we must rearrange the remaining 7 letters. There are only two letters (s and e), so this is really just a bit-string question (think of s as 1 and e as 0). Thus there \({7 \choose 2} = 21\) anagrams starting with ``a''.%
\end{divisionsolution}%
\begin{divisionsolution}{1.3.11}{}{p:exercise:cSI}%
How many anagrams are there of ``anagram''?%
\par\smallskip%
\noindent\textbf{\blocktitlefont Answer}.\quad{}\(\mathop{\rm C}\nolimits\!\left(7,3\right)\!\left(4!\right)\)%
\par\smallskip%
\noindent\textbf{\blocktitlefont Solution}.\quad{}First, decide where to put the ``a''s. There are 7 positions, and we must choose 3 of them to fill with an ``a''. This can be done in \({7 \choose 3}\) ways. The remaining 4 spots all get a different letter, so there are \(4!\) ways to finish off the anagram. This gives a total of \({7 \choose 3}\cdot 4!\) anagrams. Strangely enough, this is 840, which is also equal to \(P(7,4)\text{.}\) To get the answer that way, start by picking one of the 7 \emph{positions} to be filled by the ``n'', one of the remaining 6 positions to be filled by the ``g'', one of the remaining 5 positions to be filled by the ``r'', one of the remaining 4 positions to be filled by the ``m'' and then put ``a''s in the remaining 3 positions.%
\end{divisionsolution}%
\begin{divisionsolution}{1.3.12}{}{p:exercise:IZR}%
On a business retreat, your company of 20 executives go golfing.%
\begin{enumerate}[label=(\alph*)]
\item{}You need to divide up into foursomes (groups of 4 people): a first foursome, a second foursome, and so on. How many ways can you do this?%
\item{}After all your hard work, you realize that in fact, you want each foursome to include one of the five Board members. How many ways can you do this?%
\end{enumerate}
%
\par\smallskip%
\noindent\textbf{\blocktitlefont Answer 1}.\quad{}\(\mathop{\rm C}\nolimits\!\left(20,4\right)\mathop{\rm C}\nolimits\!\left(16,4\right)\mathop{\rm C}\nolimits\!\left(12,4\right)\mathop{\rm C}\nolimits\!\left(8,4\right)\mathop{\rm C}\nolimits\!\left(4,4\right)\)%
\par\smallskip%
\noindent\textbf{\blocktitlefont Answer 2}.\quad{}\(5!\mathop{\rm C}\nolimits\!\left(15,3\right)\mathop{\rm C}\nolimits\!\left(12,3\right)\mathop{\rm C}\nolimits\!\left(9,3\right)\mathop{\rm C}\nolimits\!\left(6,3\right)\mathop{\rm C}\nolimits\!\left(3,3\right)\)%
\par\smallskip%
\noindent\textbf{\blocktitlefont Solution 1}.\quad{}%
\begin{enumerate}[label=(\alph*)]
\item{}\({20 \choose 4}{16 \choose 4}{12 \choose 4}{8 \choose 4}{4 \choose 4}\) ways.%
\item{}\(5!{15 \choose 3}{12 \choose 3}{9 \choose 3}{6 \choose 3}{3 \choose 3}\) ways.%
\end{enumerate}
%
\par\smallskip%
\noindent\textbf{\blocktitlefont Solution 2}.\quad{}%
\begin{enumerate}[label=(\alph*)]
\item{}\({20 \choose 4}{16 \choose 4}{12 \choose 4}{8 \choose 4}{4 \choose 4}\) ways. Pick 4 out of 20 people to be in the first foursome, then 4 of the remaining 16 for the second foursome, and so on (use the multiplicative principle to combine).%
\item{}\(5!{15 \choose 3}{12 \choose 3}{9 \choose 3}{6 \choose 3}{3 \choose 3}\) ways. First determine the tee time of the 5 board members, then select 3 of the 15 non board members to golf with the first board member, then 3 of the remaining 12 to golf with the second, and so on.%
\end{enumerate}
%
\end{divisionsolution}%
\begin{divisionsolution}{1.3.13}{}{p:exercise:pha}%
How many different seating arrangements are possible for King Arthur and his 9 knights around their round table?%
\par\smallskip%
\noindent\textbf{\blocktitlefont Hint}.\quad{}There are 10 people seated around the table, but it does not matter where King Arthur sits, only who sits to his left, two seats to his left, and so on.  So the answer is not \(10!\text{.}\)%
\par\smallskip%
\noindent\textbf{\blocktitlefont Answer}.\quad{}\(\frac{10!}{10}\)%
\par\smallskip%
\noindent\textbf{\blocktitlefont Solution}.\quad{}\(9!\text{.}\)%
\end{divisionsolution}%
\begin{divisionsolution}{1.3.14}{}{p:exercise:Voj}%
We have seen that the formula for \(P(n,k)\) is \(\dfrac{n!}{(n-k)!}\). Your task here is to explain \emph{why} this is the right formula.%
\begin{enumerate}[label=(\alph*)]
\item{}Suppose you have 12 chips, each a different color. How many different stacks of 5 chips can you make? Explain your answer and why it is the same as using the formula for \(P(12,5)\). %
\item{}Using the scenario of the 12 chips again, what does \(12!\) count? What does \(7!\) count? Explain. %
\item{}Explain why it makes sense to divide \(12!\) by \(7!\) when computing \(P(12,5)\) (in terms of the chips).%
\item{}Does your explanation work for numbers other than 12 and 5? Explain the formula \(P(n,k) = \frac{n!}{(n-k)!}\) using the variables \(n\) and \(k\). %
\end{enumerate}
%
\end{divisionsolution}%
\section*{1.4 Combinatorial Proofs}
\addcontentsline{toc}{section}{1.4 Combinatorial Proofs}
\sectionmark{1.4 Combinatorial Proofs}
\subsection*{Exercises}
\addcontentsline{toc}{subsection}{Exercises}
\begin{divisionsolution}{1.4.1}{}{p:exercise:wEI}%
Give a combinatorial proof of the identity \(2+2+2 = 3\cdot 2\).%
\par\smallskip%
\noindent\textbf{\blocktitlefont Solution}.\quad{}\begin{solutionproof}
Question: How many 2-letter words start with \emph{a}, \emph{b}, or \emph{c} and end with either \emph{y} or \emph{z}?%
\par
Answer 1: There are two words that start with \emph{a}, two that start with \emph{b}, two that start with \emph{c}, for a total of \(2+2+2\).%
\par
Answer 2: There are three choices for the first letter and two choices for the second letter, for a total of \(3 \cdot 2\).%
\par
Since the two answers are both answers to the same question, they are equal. Thus \(2 + 2 + 2 = 3\cdot 2\).%
\end{solutionproof}
\end{divisionsolution}%
\begin{divisionsolution}{1.4.2}{}{p:exercise:cLR}%
Suppose you own \(x\) fezzes and \(y\) bow ties. Of course, \(x\) and \(y\) are both greater than 1.%
\begin{enumerate}[label=(\alph*)]
\item{}How many combinations of fez and bow tie can you make? You can wear only one fez and one bow tie at a time. Explain.%
\item{}Explain why the answer is \emph{also} \({x+y \choose 2} - {x \choose 2} - {y \choose 2}\). (If this is what you claimed the answer was in part (a), try it again.)%
\item{}Use your answers to parts (a) and (b) to give a combinatorial proof of the identity%
\begin{equation*}
{x+y \choose 2} - {x \choose 2} - {y \choose 2} = xy.\text{.}
\end{equation*}
%
\end{enumerate}
%
\par\smallskip%
\noindent\textbf{\blocktitlefont Solution}.\quad{}%
\begin{enumerate}[label=(\alph*)]
\item{}You have \(x\) choices for the fez, and for each choice of fez you have \(y\) choices for the bow tie. Thus you have \(x \cdot y\) choices for fez and bow tie combination.%
\item{}Line up all \(x+y\) quirky clothing items - the \(x\) fezzes and \(y\) bow ties. Now pick 2 of them. This can be done in \({x+y \choose 2}\) ways. However, we might have picked 2 fezzes, which is not allowed. There are \({x \choose 2}\) ways to pick 2 fezzes. Similarly, the \({x+y \choose 2}\) ways to pick two items includes \({y \choose 2}\) ways to select 2 bow ties, also not allowed. Thus the total number of ways to pick a fez and a bow ties is%
\begin{equation*}
{x+y \choose 2} - {x \choose 2} - {y \choose 2}
\end{equation*}
%
\item{}\begin{proof}{}{g:proof:idp140963556576}
The question is how many ways can you select one of \(x\) fezzes and one of \(y\) bow ties. We answer this question in two ways. First, the answer could be \(a\cdot b\). This is correct as described in part (a) above. Second, the answer could be \({x+y \choose 2} - {x \choose 2} - {y \choose 2}\). This is correct as described in part (b) above. Therefore%
\begin{equation*}
{x+y \choose 2} - {x \choose 2} - {y \choose 2} = xy\qedhere
\end{equation*}
%
\end{proof}
%
\end{enumerate}
%
\end{divisionsolution}%
\begin{divisionsolution}{1.4.3}{}{p:exercise:ITa}%
How many triangles can you draw using the dots below as vertices?%
\begin{sidebyside}{1}{0.3}{0.3}{0}%
\begin{sbspanel}{0.4}%
\resizebox{\linewidth}{!}{%
 \begin{tikzpicture}[scale=.75]
	\foreach \x in {-3,...,3}{
	\draw[fill] (\x,.5) circle (3pt);
	}
	\foreach \x in {-2,...,2}{
	\draw[fill] (90+\x*30:3.25) circle (3pt);
	}
\end{tikzpicture}
}%
\end{sbspanel}%
\end{sidebyside}%
\par
%
\begin{enumerate}[label=(\alph*)]
\item{}Find an expression for the answer which is the sum of three terms involving binomial coefficients.%
\item{}Find an expression for the answer which is the difference of two  binomial coefficients.%
\item{}Generalize the above to state and prove a binomial identity using a combinatorial proof.  Say you have \(x\) points on the horizontal axis and \(y\) points in the semi-circle.%
\end{enumerate}
%
\par\smallskip%
\noindent\textbf{\blocktitlefont Hint}.\quad{}There will be 185 triangles.  But to find them \textellipsis{}%
\begin{enumerate}[label=(\alph*)]
\item{}How many vertices of the triangle can be on the horizontal axis?%
\item{}Will \emph{any} three dots work as the vertices?%
\end{enumerate}
%
\end{divisionsolution}%
\begin{divisionsolution}{1.4.4}{}{p:exercise:paj}%
Consider all the triangles you can create using the points shown below as vertices. Note, we are not allowing degenerate triangles (ones with all three vertices on the same line) but we do allow non-right triangles.%
\begin{sidebyside}{1}{0.25}{0.25}{0}%
\begin{sbspanel}{0.5}%
\resizebox{\linewidth}{!}{%
\begin{tikzpicture}[scale=0.7]
  \foreach \i in {0,...,6} {
    \fill (\i,0) circle (2pt);
  }
  \foreach \i in {1,...,4} {
    \fill (0,\i) circle (2pt);
  }
\end{tikzpicture}
}%
\end{sbspanel}%
\end{sidebyside}%
\par
%
\begin{enumerate}[label=(\alph*)]
\item{}Find the number of triangles, and explain why your answer is correct.%
\item{}Find the number of triangles again, using a different method. Explain why your new method works.%
\item{}State a binomial identity that your two answers above establish (that is, give the binomial identity that your two answers a proof for). Then generalize this using \(m\)'s and \(n\)'s.%
\end{enumerate}
%
\par\smallskip%
\noindent\textbf{\blocktitlefont Hint}.\quad{}The answer is 120.%
\par\smallskip%
\noindent\textbf{\blocktitlefont Solution}.\quad{}There are 120 triangles. Here are a few ways (there are others as well) to get this:%
\begin{enumerate}[label=(\alph*)]
\item{}First count the triangles with the base on the \(x\) -axis. There are \({7 \choose 2}\) ways to pick the base. The third vertex of the triangle must be one of the 4 dots on the \(y\) -axis (not the origin) so there are a total of \({7 \choose 2}4\) of these triangles. The triangles with base on the \(y\) axis can be counted similarly: \({5 \choose 2}6\). However, we have counted all the right triangles twice - they have a base on the \(x\) -axis and also on the \(y\) -axis. There are \(4 \cdot 6\) right triangles. Thus the total number of triangles is:%
\begin{equation*}
{7 \choose 2}4 + {5 \choose 2}6 - 6\cdot 4 = 120
\end{equation*}
%
\item{}First count all the right triangles: \(6 \cdot 4\). Then count all the non-right triangles with two vertices on the \(x\) -axis: \(4 \cdot {6 \choose 2}\). Finally, the number of non-right triangles with two vertices on the \(y\) -axis: \(6 \cdot {4 \choose 2}\). All together then we have \({6 \choose 2}4 + {4 \choose 2}6 + 6 \cdot 4 = 120\).%
\item{}Another approach to count the non-right triangles is to first select one of the two sides not parallel to an axis. This can be done in \(6 \cdot 4\) ways. Then for each of these, select one of the remaining 8 points, which determines the triangle. However, this double counts the non-right triangles, so we take \(\frac{4\cdot 6 \cdot 8}{2}\), and then add on the \(6\cdot 4\) right triangles.%
\item{}We must select 3 of the 11 dots. This can be done in \({11 \choose 3}\) ways. However, this will also give us degenerate triangles when all three vertices are on the \(x\) -axis or on the \(y\) -axis. There are \({7 \choose 3}\) ways we could have picked all three vertices on the \(x\) -axis. There are \({5 \choose 3}\) ways we could have picked all three vertices on the \(y\) -axis. Therefore the total number of triangles is%
\begin{equation*}
{11 \choose 3} - {7 \choose 3} - {5 \choose 3} = 120
\end{equation*}
%
\end{enumerate}
%
\par
Picking two of these, we know the expressions must be equal (and not just because they are equal to 120). So we might write:%
\begin{equation*}
{11 \choose 3} - {7 \choose 3} - {5 \choose 3} = {7 \choose 2}4 + {5 \choose 2}6 - 6\cdot 4
\end{equation*}
%
\par
In general, suppose we have \(m\) dots on the \(x\) -axis (including the origin) and \(n\) dots on the \(y\) -axis (including the origin). Then counting the number of triangles in two different ways establishes:%
\begin{equation*}
{m+n \choose 3} - {m \choose 3} - {n \choose 3} = {m \choose 2}(n-1) + {n \choose 2}(m-1) - (m-1)(n-1)
\end{equation*}
%
\end{divisionsolution}%
\begin{divisionsolution}{1.4.5}{}{x:exercise:exc-bridesmaids}%
A woman is getting married. She has 15 best friends but can only select 6 of them to be her bridesmaids, one of which needs to be her maid of honor. How many ways can she do this?%
%
\begin{enumerate}[label=(\alph*)]
\item{}What if she first selects the 6 bridesmaids, and then selects one of them to be the maid of honor?%
\item{}What if she first selects her maid of honor, and then 5 other bridesmaids?%
\item{}Explain why \(6 {15 \choose 6} = 15 {14 \choose 5}\).%
\end{enumerate}
\par\smallskip%
\noindent\textbf{\blocktitlefont Solution}.\quad{}%
\begin{enumerate}[label=(\alph*)]
\item{}She has \({15 \choose 6}\) ways to select the 6 bridesmaids, and then for each way, has 6 choices for the maid of honor. Thus she has \({15 \choose 6}6\) choices.%
\item{}She has 15 choices for who will be her maid of honor. Then she needs to select 5 of the remaining 14 friends to be bridesmaids, which she can do in \({14 \choose 5}\) ways. Thus she has \(15 {14 \choose 5}\) choices.%
\item{}We have answered the question (how many wedding parties can the bride choose from) in two ways. The first way gives the left-hand side of the identity and the second way gives the right-hand side of the identity. Therefore the identity holds.%
\end{enumerate}
%
\end{divisionsolution}%
\begin{divisionsolution}{1.4.6}{}{p:exercise:BoB}%
Consider the identity:%
\begin{equation*}
k{n\choose k} = n{n-1 \choose k-1}\text{.}
\end{equation*}
%
\begin{enumerate}[label=(\alph*)]
\item{}Is this true? Try it for a few values of \(n\) and \(k\).%
\item{}Use the formula for \({n \choose k}\) to give an algebraic proof of the identity.%
\item{}Give a combinatorial proof of the identity.%
\end{enumerate}
%
\par\smallskip%
\noindent\textbf{\blocktitlefont Hint}.\quad{}Try Exercise~1.4.5%
\end{divisionsolution}%
\begin{divisionsolution}{1.4.7}{}{p:exercise:hvK}%
Give a combinatorial proof of the identity \({n \choose 2}{n-2 \choose k-2} = {n\choose k}{k \choose 2}\).%
\par\smallskip%
\noindent\textbf{\blocktitlefont Hint}.\quad{}What if you wanted a pair of co-maids-of-honor?%
\par\smallskip%
\noindent\textbf{\blocktitlefont Solution}.\quad{}\begin{solutionproof}
Question: You have a large container filled with ping-pong balls, all with a different number on them. You must select \(k\) of the balls, putting two of them in a jar and the others in a box. How many ways can you do this?%
\par
Answer 1: First select 2 of the \(n\) balls to put in the jar. Then select \(k-2\) of the remaining \(n-2\) balls to put in the box. The first task can be completed in \({n \choose 2}\) different ways, the second task in \({n-2 \choose k-2}\) ways. Thus there are \({n \choose 2}{n-2 \choose k-2}\) ways to select the balls.%
\par
Answer 2: First select \(k\) balls from the \(n\) in the container. Then pick 2 of the \(k\) balls you picked to put in the jar, placing the remaining \(k-2\) in the box. The first task can be completed in \({n \choose k}\) ways, the second task in \({k \choose 2}\) ways. Thus there are \({n \choose k}{k \choose 2}\) ways to select the balls.%
\par
Since both answers count the same thing, they must be equal and the identity is established.%
\end{solutionproof}
\end{divisionsolution}%
\begin{divisionsolution}{1.4.8}{}{p:exercise:NCT}%
Consider the binomial identity%
\begin{equation*}
\binom{n}{1} + 2 \binom{n}{2} + 3 \binom{n}{3} + \cdots + n\binom{n}{n} = n2^{n-1}\text{.}
\end{equation*}
%
\begin{enumerate}[label=(\alph*)]
\item{}Give a combinatorial proof of this identity.  Hint: What if some number of a group of \(n\) people wanted to go to an escape room, and among those going, one needed to be the team captain?%
\item{}Give an alternate proof by multiplying out \((1+x)^n\) and taking derivatives of both sides.%
\end{enumerate}
%
\par\smallskip%
\noindent\textbf{\blocktitlefont Hint}.\quad{}For the combinatorial proof: what if you don't yet know how many bridesmaids you will have?%
\end{divisionsolution}%
\begin{divisionsolution}{1.4.9}{}{p:exercise:tKc}%
Give a combinatorial proof for the identity \(1 + 2 + 3 + \cdots + n = {n+1 \choose 2}\).%
\par\smallskip%
\noindent\textbf{\blocktitlefont Hint}.\quad{}Count handshakes.%
\par\smallskip%
\noindent\textbf{\blocktitlefont Solution}.\quad{}\begin{solutionproof}
Question: How many subsets of \(A = {1,2,3, \ldots,
n+1}\) contain exactly two elements?%
\par
Answer 1: We must choose 2 elements from \(n+1\) choices, so there are \({n+1 \choose 2}\) subsets.%
\par
Answer 2: We break this question down into cases, based on what the larger of the two elements in the subset is. The larger element can't be 1, since we need at least one element smaller than it.%
\par
Larger element is 2: there is 1 choice for the smaller element.%
\par
Larger element is 3: there are 2 choices for the smaller element.%
\par
Larger element is 4: there are 3 choices for the smaller element.%
\par
And so on. When the larger element is \(n+1\), there are \(n\) choices for the smaller element. Since each two element subset must be in exactly one of these cases, the total number of two element subsets is \(1 + 2 + 3 + \cdots + n\).%
\par
Answer 1 and answer 2 are both correct answers to the same question, so they must be equal. Therefore,%
\begin{equation*}
1 + 2 + 3 + \cdots + n = {n+1 \choose 2}\qedhere
\end{equation*}
%
\end{solutionproof}
\end{divisionsolution}%
\begin{divisionsolution}{1.4.10}{}{p:exercise:ZRl}%
Consider the bit strings in \(\B^6_2\) (bit strings of length 6 and weight 2).%
%
\begin{enumerate}[label=(\alph*)]
\item{}How many of those bit strings start with 1?%
\item{}How many of those bit strings start with 01?%
\item{}How many of those bit strings start with 001?%
\item{}Are there any other strings we have not counted yet? Which ones, and how many are there?%
\item{}How many bit strings are there total in \(\B^6_2\)?%
\item{}What binomial identity have you just given a combinatorial proof for?%
\end{enumerate}
\par\smallskip%
\noindent\textbf{\blocktitlefont Solution}.\quad{}%
\begin{enumerate}[label=(\alph*)]
\item{}After the 1, we need to find a 5-bit string with one 1. There are \({5 \choose 1}\) ways to do this.%
\item{}\({4 \choose 1}\) strings (we need to pick 1 of the remaining 4 slots to be the second 1).%
\item{}\({3 \choose 1}\) strings.%
\item{}Yes. We still need strings starting with 0001 (there are \({2 \choose 1}\) of these) and strings starting 00001 (there is only \({1 \choose 1} = 1\) of these).%
\item{}\({6 \choose 2}\) strings%
\item{}An example of the Hockey Stick Theorem:%
\begin{equation*}
{1 \choose 1} + {2 \choose 1} + {3 \choose 1} + {4 \choose 1} + {5 \choose 1} = {6 \choose 2}
\end{equation*}
%
\end{enumerate}
%
\end{divisionsolution}%
\begin{divisionsolution}{1.4.11}{}{p:exercise:FYu}%
Let's count \terminology{ternary} digit strings, that is, strings in which each digit can be 0, 1, or 2.%
\begin{enumerate}[label=(\alph*)]
\item{}How many ternary digit strings contain exactly \(n\) digits?%
\item{}How many ternary digit strings contain exactly \(n\) digits and \(n\) 2's.%
\item{}How many ternary digit strings contain exactly \(n\) digits and \(n-1\) 2's. (Hint: where can you put the non-2 digit, and then what could it be?)%
\item{}How many ternary digit strings contain exactly \(n\) digits and \(n-2\) 2's. (Hint: see previous hint)%
\item{}How many ternary digit strings contain exactly \(n\) digits and \(n-k\) 2's.%
\item{}How many ternary digit strings contain exactly \(n\) digits and no 2's. (Hint: what kind of a string is this?)%
\item{}Use the above parts to give a combinatorial proof for the identity%
\begin{equation*}
{n \choose 0} + 2{n \choose 1} + 2^2{n \choose 2} + 2^3{n \choose 3} + \cdots + 2^n{n \choose n} = 3^n\text{.}
\end{equation*}
%
\end{enumerate}
%
\par\smallskip%
\noindent\textbf{\blocktitlefont Solution}.\quad{}%
\begin{enumerate}[label=(\alph*)]
\item{}\(3^n\) strings, since there are 3 choices for each of the \(n\) digits.%
\item{}\(1\) string, since all the digits need to be 2's. However, we might write this as \({n \choose 0}\) strings.%
\item{}There are \({n \choose 1}\) places to put the non-2 digit. That digit can be either a 0 or a 1, so there are \(2{n \choose 1}\) such strings.%
\item{}We must choose two slots to fill with 0's or 1's. There are \({n \choose 2}\) ways to do that. Once the slots are picked, we have two choices for the first slot (0 or 1) and two choices for the second slot (0 or 1). So there are a total of \(2^2{n \choose 2}\) such strings.%
\item{}There are \({n \choose k}\) ways to pick which slots don't have the 2's. Then those slots can be filled in \(2^k\) ways (0 or 1 for each slot). So there are \(2^k{n \choose k}\) such strings.%
\item{}These strings contain just 0's and 1's, so they are bit strings. There are \(2^n\) bit strings. But keeping with the pattern above, we might write this as \(2^n {n \choose n}\) strings.%
\item{}We answer the question of how many length \(n\) ternary digit strings there are in two ways. First, each digit can be one of three choices, so the total number of strings is \(3^n\). On the other hand, we could break the question down into cases by how many of the digits are 2's. If they are all 2's, then there are \({n \choose 0}\) strings. If all but one is a 2, then there are \(2{n \choose 1}\) strings. If all but 2 of the digits are 2's, then there are \(2^2{n \choose 2}\) strings. We choose 2 of the \(n\) digits to be non-2, and then there are 2 choices for each of those digits. And so on for every possible number of 2's in the string. Therefore \({n \choose 0} + 2{n \choose 1} + 2^2{n \choose 2} + 2^3{n \choose 3} + \cdots + 2^n{n \choose n} = 3^n\).%
\end{enumerate}
%
\end{divisionsolution}%
\begin{divisionsolution}{1.4.12}{}{p:exercise:mfD}%
How many ways are there to rearrange the letters in the word ``rearrange''? Answer this question in at least two different ways to establish a binomial identity.%
\par\smallskip%
\noindent\textbf{\blocktitlefont Solution}.\quad{}The word contains 9 letters: 3 ``r''s, 2 ``a''s and 2 ``e''s, along with an ``n'' and a ``g''. We could first select the positions for the ``r''s in \({9 \choose 3}\) ways, then the ``a''s in \({6 \choose 2}\) ways, the ``e''s in \({4 \choose 2}\) ways and then select one of the remaining two spots to put the ``n'' (placing the ``g'' in the last spot). This gives the answer%
\begin{equation*}
{9 \choose 3}{6 \choose 2}{4 \choose 2}{2\choose 1}{1\choose 1}\text{.}
\end{equation*}
%
\par
Alternatively, we could select the positions of the letters in the opposite order, which would give an answer%
\begin{equation*}
{9 \choose 1}{8\choose 1}{7 \choose 2}{5\choose 2}{3\choose 3}\text{.}
\end{equation*}
%
\par
(where the 3 ``r''s go in the remaining 3 spots). These two expressions are equal:%
\begin{equation*}
{9 \choose 3}{6 \choose 2}{4 \choose 2}{2\choose 1}{1\choose 1} = {9 \choose 1}{8\choose 1}{7 \choose 2}{5\choose 2}{3\choose 3}\text{.}
\end{equation*}
%
\end{divisionsolution}%
\begin{divisionsolution}{1.4.13}{}{p:exercise:SmM}%
Establish the identity below using a combinatorial proof.%
\begin{equation*}
{2 \choose 2}{n \choose 2} + {3 \choose 2}{n-1 \choose 2} + {4\choose 2}{n-2 \choose 2} + \cdots + {n\choose 2}{2\choose 2} = {n+3 \choose 5}\text{.}
\end{equation*}
%
\par\smallskip%
\noindent\textbf{\blocktitlefont Hint}.\quad{}This one might remind you of Example~1.4.6%
\par\smallskip%
\noindent\textbf{\blocktitlefont Solution}.\quad{}\begin{solutionproof}
Question: How many 5-element subsets are there of the set \(\{1,2,\ldots,
n+3\}\).%
\par
Answer 1: We choose 5 out of the \(n+3\) elements, so \({n+3 \choose 5}\) subsets.%
\par
Answer 2: Break this up into cases by what the ``middle'' (third smallest) element of the 5 element subset is. The smallest this could be is a 3. In that case, we have \({2 \choose 2}\) choices for the numbers below it, and \({n \choose 2}\) choices for the numbers above it. Alternatively, the middle number could be a 4. In this case there are \({3 \choose 2}\) choices for the bottom two numbers and \({n-1 \choose 2}\) choices for the top two numbers. If the middle number is 5, then there are \({4 \choose 2}\) choices for the bottom two numbers and \({n-2 \choose 2}\) choices for the top two numbers. An so on, all the way up to the largest the middle number could be, which is \(n+1\). In that case there are \({n \choose 2}\) choices for the bottom two numbers and \({2 \choose 2}\) choices for the top number. Thus the number of 5 element subsets is%
\begin{equation*}
{2 \choose 2}{n \choose 2} + {3 \choose 2}{n-1 \choose 2} + {4\choose 2}{n-2 \choose 2} + \cdots + {n\choose 2}{2\choose 2}\text{.}
\end{equation*}
%
\par
Since the two answers correctly answer the same question, we have%
\begin{equation*}
{2 \choose 2}{n \choose 2} + {3 \choose 2}{n-1 \choose 2} + {4\choose 2}{n-2 \choose 2} + \cdots + {n\choose 2}{2\choose 2} = {n+3 \choose 5}\text{.}\qedhere
\end{equation*}
%
\end{solutionproof}
\end{divisionsolution}%
\begin{divisionsolution}{1.4.14}{}{p:exercise:ytV}%
In Example~1.4.5 we established that the sum of any row in Pascal's triangle is a power of two.  Specifically,%
\begin{equation*}
{n\choose 0} + {n \choose 1} + {n\choose 2} + \cdots + {n \choose n} = 2^n\text{.}
\end{equation*}
The argument given there used the counting question, ``how many pizzas can you build using any number of \(n\) different toppings?''  To practice, give new proofs of this identity using different questions.%
\begin{enumerate}[label=(\alph*)]
\item{}Use a question about counting subsets.%
\item{}Use a question about counting bit strings.%
\item{}Use a question about counting lattice paths.%
\end{enumerate}
%
\par\smallskip%
\noindent\textbf{\blocktitlefont Hint}.\quad{}For the lattice paths, think about what sort of paths \(2^n\) would count.  Not all the paths will end at the same point, but you could describe the set of end points as a line.%
\end{divisionsolution}%
\section*{1.5 Stars and Bars}
\addcontentsline{toc}{section}{1.5 Stars and Bars}
\sectionmark{1.5 Stars and Bars}
\subsection*{Exercises}
\addcontentsline{toc}{subsection}{Exercises}
\begin{divisionsolution}{1.5.1}{}{p:exercise:SfI}%
Consider sets \(A\) and \(B\) with \(|A| = 10\) and \(|B| = 17\text{.}\)%
\begin{enumerate}[label=(\alph*)]
\item{}How many functions \(f: A \to B\) are there?%
\item{}How many functions \(f: A \to B\) are injective?%
\end{enumerate}
%
\par\smallskip%
\noindent\textbf{\blocktitlefont Answer 1}.\quad{}\(17^{10}\)%
\par\smallskip%
\noindent\textbf{\blocktitlefont Answer 2}.\quad{}\(\mathop{\rm P}\nolimits\!\left(17,10\right)\)%
\par\smallskip%
\noindent\textbf{\blocktitlefont Solution}.\quad{}%
\begin{enumerate}[label=(\alph*)]
\item{}\(17^{10}\) functions. There are 17 choices for the image of each element in the domain.%
\item{}\(P(17, 10)\) injective functions. There are 17 choices for image of the first element of the domain, then only 16 choices for the second, and so on.%
\end{enumerate}
%
\end{divisionsolution}%
\begin{divisionsolution}{1.5.2}{}{p:exercise:ymR}%
Consider functions \(f: \{1,2,3,4\} \to \{1,2,3,4,5,6\}\text{.}\)%
\begin{enumerate}[label=(\alph*)]
\item{}How many functions are there total?%
\item{}How many functions are injective?%
\item{}How many of the injective functions are \emph{increasing}? To be increasing means that if \(a \lt b\) then \(f(a) \lt f(b)\text{,}\) or in other words, the outputs get larger as the inputs get larger.%
\end{enumerate}
%
\par\smallskip%
\noindent\textbf{\blocktitlefont Answer 1}.\quad{}\(6^{4}\)%
\par\smallskip%
\noindent\textbf{\blocktitlefont Answer 2}.\quad{}\(\mathop{\rm P}\nolimits\!\left(6,4\right)\)%
\par\smallskip%
\noindent\textbf{\blocktitlefont Answer 3}.\quad{}\(\mathop{\rm C}\nolimits\!\left(6,4\right)\)%
\par\smallskip%
\noindent\textbf{\blocktitlefont Solution 1}.\quad{}%
\begin{enumerate}[label=(\alph*)]
\item{}\(6^4 = 1296\) functions.%
\item{}\(P(6, 4) = 6 \cdot 5 \cdot 4 \cdot 3 = 360\) functions.%
\item{}\({6 \choose 4} = 15\) functions.%
\end{enumerate}
%
\par\smallskip%
\noindent\textbf{\blocktitlefont Solution 2}.\quad{}%
\begin{enumerate}[label=(\alph*)]
\item{}\(6^4 = 1296\) functions, since there are six choices of where to send each of the 4 elements of the domain.%
\item{}\(P(6, 4) = 6 \cdot 5 \cdot 4 \cdot 3 = 360\) functions, since outputs cannot be repeated.%
\item{}\({6 \choose 4} = 15\) functions. Since the function must be injective and increasing, we just need to select the four distinct elements of the range from the six elements of the codomain. Once selected, we must put the smallest as the image of 1, the next smallest as the image of 2, and so on (doing this does not increase the number of functions, since there is one choice for how this event can occur).%
\end{enumerate}
%
\end{divisionsolution}%
\begin{divisionsolution}{1.5.3}{}{p:exercise:eua}%
Each of the counting problems below can be solved with stars and bars. For each, say what outcome the diagram%
\begin{equation*}
***|*||**|
\end{equation*}
represents, if there are the correct number of stars and bars for the problem. Otherwise, say why the diagram does not represent any outcome, and what a correct diagram would look like.%
\begin{enumerate}[label=(\alph*)]
\item{}How many ways are there to select a handful of 6 jellybeans from a jar that contains 5 different flavors?%
\item{}How many ways can you distribute 5 identical lollipops to 6 kids?%
\item{}How many 6-letter words can you make using the 5 vowels in alphabetical order?%
\item{}How many solutions are there to the equation \(x_1 + x_2 + x_3 + x_4 = 6\).%
\end{enumerate}
%
\par\smallskip%
\noindent\textbf{\blocktitlefont Solution}.\quad{}%
\begin{enumerate}[label=(\alph*)]
\item{}You take 3 strawberry, 1 lime, 0 licorice, 2 blueberry and 0 bubblegum.%
\item{}This is backwards. We don't want the stars to represent the kids because the kids are not identical, but the stars are. Instead we should use 5 stars (for the lollipops) and use 5 bars to switch between the 6 kids. For example,%
\begin{equation*}
**||***|||
\end{equation*}
would represent the outcome with the first kid getting 2 lollipops, the third kid getting 3, and the rest of the kids getting none.%
\item{}This is the word AAAEOO.%
\item{}This doesn't represent a solution. Each star should represent one of the 6 units that add up to 6, and the bars should \emph{switch} between the different variables. We have one too many bars. An example of a correct diagram would be%
\begin{equation*}
*|**||***\text{,}
\end{equation*}
representing that \(x_1 = 1\), \(x_2 = 2\), \(x_3 = 0\), and \(x_4 = 3\).%
\end{enumerate}
%
\end{divisionsolution}%
\begin{divisionsolution}{1.5.4}{}{p:exercise:KBj}%
A multiset is a collection of objects, just like a set, but can contain an object more than once (the order of the elements still doesn't matter). For example, \(\{1,1, 2, 5, 5, 7\}\) is a multiset of size 6.%
\begin{enumerate}[label=(\alph*)]
\item{}How many \emph{sets} of size 5 can be made using the 10 numeric digits 0 through 9?%
\item{}How many \emph{multi}sets of size 5 can be made using the 10 numeric digits 0 through 9?%
\end{enumerate}
%
\par\smallskip%
\noindent\textbf{\blocktitlefont Answer 1}.\quad{}\(\mathop{\rm C}\nolimits\!\left(10,5\right)\)%
\par\smallskip%
\noindent\textbf{\blocktitlefont Answer 2}.\quad{}\(\mathop{\rm C}\nolimits\!\left(14,5\right)\)%
\par\smallskip%
\noindent\textbf{\blocktitlefont Solution}.\quad{}%
\begin{enumerate}[label=(\alph*)]
\item{}\({10\choose 5}\) sets. We must select 5 of the 10 digits to put in the set.%
\item{}Use stars and bars: each star represents one of the 5 elements of the set, each bar represents a switch between digits. So there are 5 stars and 9 bars, giving us \({14 \choose 9}\) sets.%
\end{enumerate}
%
\end{divisionsolution}%
\begin{divisionsolution}{1.5.5}{}{p:exercise:qIs}%
Using the digits 2 through 8, find the number of different 5-digit numbers such that:%
\begin{enumerate}[label=(\alph*)]
\item{}Digits cannot be repeated and must be written in increasing order. For example, 23678 is okay, but 32678 is not.%
\item{}Digits \emph{can} be repeated and must be written in \emph{non-decreasing} order. For example, 24448 is okay, but 24484 is not.%
\end{enumerate}
%
\par\smallskip%
\noindent\textbf{\blocktitlefont Answer 1}.\quad{}\(\mathop{\rm C}\nolimits\!\left(7,5\right)\)%
\par\smallskip%
\noindent\textbf{\blocktitlefont Answer 2}.\quad{}\(\mathop{\rm C}\nolimits\!\left(11,6\right)\)%
\par\smallskip%
\noindent\textbf{\blocktitlefont Solution}.\quad{}%
\begin{enumerate}[label=(\alph*)]
\item{}There are \({7 \choose 5}\) numbers. We simply choose five of the seven digits and once chosen put them in increasing order.%
\item{}This requires stars and bars. Use a star to represent each of the 5 digits in the number, and use their position relative to the bars to say what numeral fills that spot. So we will have 5 stars and 6 bars, giving \({11 \choose 6}\) numbers.%
\end{enumerate}
%
\end{divisionsolution}%
\begin{divisionsolution}{1.5.6}{}{p:exercise:WPB}%
After gym class you are tasked with putting the 14 identical dodgeballs away into 5 bins.%
\begin{enumerate}[label=(\alph*)]
\item{}How many ways can you do this if there are no restrictions?%
\item{}How many ways can you do this if each bin must contain at least one dodgeball?%
\end{enumerate}
%
\par\smallskip%
\noindent\textbf{\blocktitlefont Answer 1}.\quad{}\(\mathop{\rm C}\nolimits\!\left(18,14\right)\)%
\par\smallskip%
\noindent\textbf{\blocktitlefont Answer 2}.\quad{}\(\mathop{\rm C}\nolimits\!\left(13,9\right)\)%
\par\smallskip%
\noindent\textbf{\blocktitlefont Solution}.\quad{}%
\begin{enumerate}[label=(\alph*)]
\item{}\({18 \choose 4}\) ways. Each outcome can be represented by a sequence of 14 stars and 4 bars.%
\item{}\({13 \choose 4}\) ways. First put one ball in each bin. This leaves 9 stars and 4 bars.%
\end{enumerate}
%
\end{divisionsolution}%
\begin{divisionsolution}{1.5.7}{}{p:exercise:CWK}%
How many integer solutions are there to the equation \(x + y + z = 8\) for which%
\begin{enumerate}[label=(\alph*)]
\item{}\(x\text{,}\) \(y\text{,}\) and \(z\) are all positive?%
\item{}\(x\text{,}\) \(y\text{,}\) and \(z\) are all non-negative?%
\item{}\(x\text{,}\) \(y\text{,}\) and \(z\) are all greater than or equal to \(-3\text{.}\)%
\end{enumerate}
%
\par\smallskip%
\noindent\textbf{\blocktitlefont Answer 1}.\quad{}\(\mathop{\rm C}\nolimits\!\left(7,2\right)\)%
\par\smallskip%
\noindent\textbf{\blocktitlefont Answer 2}.\quad{}\(\mathop{\rm C}\nolimits\!\left(10,2\right)\)%
\par\smallskip%
\noindent\textbf{\blocktitlefont Answer 3}.\quad{}\(\mathop{\rm C}\nolimits\!\left(19,2\right)\)%
\par\smallskip%
\noindent\textbf{\blocktitlefont Solution}.\quad{}%
\begin{enumerate}[label=(\alph*)]
\item{}\({7 \choose 2}\) solutions. After each variable gets 1 star for free, we are left with 5 stars and 2 bars.%
\item{}\({10 \choose 2}\) solutions. We have 8 stars and 2 bars.%
\item{}\({19 \choose 2}\) solutions. This problem is equivalent to finding the number of solutions to \(x' + y' + z' = 17\) where \(x'\text{,}\) \(y'\) and \(z'\) are non-negative. (In fact, we really just do a substitution. Let \(x = x'- 3\text{,}\) \(y = y' - 3\) and \(z = z' - 3\)).%
\end{enumerate}
%
\end{divisionsolution}%
\begin{divisionsolution}{1.5.8}{}{p:exercise:jdT}%
When playing Yahtzee, you roll five regular 6-sided dice. How many different outcomes are possible from a single roll? The order of the dice does not matter.%
\par\smallskip%
\noindent\textbf{\blocktitlefont Answer}.\quad{}\(\mathop{\rm C}\nolimits\!\left(10,5\right)\)%
\par\smallskip%
\noindent\textbf{\blocktitlefont Solution 1}.\quad{}\({10 \choose 5}\) outcomes.%
\par\smallskip%
\noindent\textbf{\blocktitlefont Solution 2}.\quad{}\({10 \choose 5}\) outcomes. We have 5 stars (the five dice) and 5 bars (the five switches between the numbers 1-6).%
\end{divisionsolution}%
\begin{divisionsolution}{1.5.9}{}{p:exercise:Plc}%
Solve the three counting problems below. Then say why it makes sense that they all have the same answer. That is, say how you can interpret them as each other.%
\begin{enumerate}[label=(\alph*)]
\item{}How many ways are there to distribute 8 cookies to 3 kids?%
\item{}How many solutions in non-negative integers are there to \(x+y+z = 8\)?%
\item{}How many different packs of 8 crayons can you make using crayons that come in red, blue and yellow?%
\end{enumerate}
%
\end{divisionsolution}%
\begin{divisionsolution}{1.5.10}{}{p:exercise:vsl}%
Your friend tells you she has 7 coins in her hand (just pennies, nickels, dimes and quarters). If you guess how many of each kind of coin she has, she will give them to you. If you guess randomly, what is the probability that you will be correct?%
\par\smallskip%
\noindent\textbf{\blocktitlefont Answer}.\quad{}\(\frac{1}{\mathop{\rm C}\nolimits\!\left(10,7\right)}\)%
\par\smallskip%
\noindent\textbf{\blocktitlefont Solution 1}.\quad{}There are \({10 \choose 3} = 120\) different combinations of coins possible. Thus you have a 1 in 120 chance of guessing correctly.%
\par\smallskip%
\noindent\textbf{\blocktitlefont Solution 2}.\quad{}We must figure out how many different combinations of 7 coins are possible. Let a star represent each coin, and a bar represent switching between type of coin. For example, \textasteriskcentered{}\textasteriskcentered{}\textbar{}\textasteriskcentered{}\textbar{}\textbar{}\textasteriskcentered{}\textasteriskcentered{}\textasteriskcentered{}\textasteriskcentered{} represents 2 pennies, one nickel, no dimes and 4 quarters. The number of such star and bar diagrams (with 7 stars and 3 bars) is \({10 \choose 7} = 120\text{.}\) Thus you have a 1 in 120 chance of guessing correctly.%
\end{divisionsolution}%
\begin{divisionsolution}{1.5.11}{}{p:exercise:bzu}%
How many integer solutions to \(x_1 + x_2 + x_3 + x_4 = 25\) are there for which \(x_1 \ge 1\text{,}\) \(x_2 \ge 2\text{,}\) \(x_3 \ge 3\) and \(x_4 \ge 4\text{?}\)%
\par\smallskip%
\noindent\textbf{\blocktitlefont Answer}.\quad{}\(\mathop{\rm C}\nolimits\!\left(18,15\right)\)%
\par\smallskip%
\noindent\textbf{\blocktitlefont Solution}.\quad{}\({18 \choose 15}\) solutions. Distribute 10 units to the variables before finding all solutions to \(x_1' + x_2' + x_3' + x_4' = 15\) in non-negative integers.%
\end{divisionsolution}%
\section*{1.6 Advanced Counting Using PIE}
\addcontentsline{toc}{section}{1.6 Advanced Counting Using PIE}
\sectionmark{1.6 Advanced Counting Using PIE}
\subsection*{Exercises}
\addcontentsline{toc}{subsection}{Exercises}
\begin{divisionsolution}{1.6.1}{}{p:exercise:COT}%
Consider functions \(f:\{1,2,3,4,5\} \to \{0,1,2,\ldots,9\}\text{.}\)%
\begin{enumerate}[label=(\alph*)]
\item{}How many of these functions are strictly increasing? Explain. (A function is strictly increasing provided if \(a \lt b\text{,}\) then \(f(a) \lt f(b)\text{.}\))%
\item{}How many of the functions are non-decreasing? Explain. (A function is non-decreasing provided if \(a \lt b\text{,}\) then \(f(a) \le f(b)\text{.}\))%
\end{enumerate}
%
\par\smallskip%
\noindent\textbf{\blocktitlefont Answer 1}.\quad{}\(\mathop{\rm C}\nolimits\!\left(10,5\right)\)%
\par\smallskip%
\noindent\textbf{\blocktitlefont Answer 2}.\quad{}\(\mathop{\rm C}\nolimits\!\left(14,5\right)\)%
\par\smallskip%
\noindent\textbf{\blocktitlefont Solution 1}.\quad{}%
\begin{enumerate}[label=(\alph*)]
\item{}\({10 \choose 5}\text{.}\) Note that a strictly increasing function is automatically injective.%
\item{}\({14 \choose 9}\text{.}\)%
\end{enumerate}
%
\par\smallskip%
\noindent\textbf{\blocktitlefont Solution 2}.\quad{}%
\begin{enumerate}[label=(\alph*)]
\item{}\({10 \choose 5}\text{.}\) Note that a strictly increasing function is automatically injective. So the five outputs must all be different. So let's first pick which five outputs we will use: there are \({10 \choose 5}\) ways to do this. Now how many ways are there to assign those outputs to the inputs \(1\) through 5? Only one way, since there is only one way to arrange numbers in increasing order.%
\item{}\({14 \choose 5}\text{.}\) This is in fact a stars and bars problem. The stars are the 5 inputs and the bars are the 9 separators between the 10 possible outputs. Think of it this way - we will specify \(f(1)\text{,}\) then \(f(2)\text{,}\) then \(f(3)\text{,}\) and so on in that order. Start with the possible output 0. We can use it as the output of \(f(1)\text{,}\) or we can switch to 1 as a potential output. Say we put \(f(1) = 1\text{.}\) Now we are at 1 (can't go back to 0). Should \(f(2) = 1\text{?}\) If yes, then we are putting down another star. If no, put down a bar and switch to 2. Maybe you switch to 3, then assign \(f(2) = 3\) and \(f(3) = 3\) (two more stars) before switching to 4 as a possible output. And so on.%
\end{enumerate}
%
\end{divisionsolution}%
\begin{divisionsolution}{1.6.2}{}{p:exercise:iWc}%
After a late night of math studying, you and your friends decide to go to your favorite tax-free fast food Mexican restaurant, \emph{Burrito Chime}. You decide to order off of the dollar menu, which has 7 items. Your group has \textdollar{}16 to spend (and will spend all of it).%
\begin{enumerate}[label=(\alph*)]
\item{}How many different orders are possible? Explain. (The \emph{order} in which the order is placed does not matter - just which and how many of each item that is ordered.)%
\item{}How many different orders are possible if you want to get at least one of each item? Explain.%
\item{}How many different orders are possible if you don't get more than 4 of any one item? Explain.%
\end{enumerate}
%
\par\smallskip%
\noindent\textbf{\blocktitlefont Solution}.\quad{}%
\begin{enumerate}[label=(\alph*)]
\item{}\(\d{22 \choose 16}\) - there are 16 stars and 6 bars.%
\item{}\(\d{15 \choose 11}\) - buy one of each item, using \textdollar{}7. This leaves you \textdollar{}11 to distribute to the 7 items, so 11 stars and 6 bars.%
\item{}%
\begin{equation*}
{22 \choose 16} - \left[{7 \choose 1}{17 \choose 11} - {7 \choose 2}{12 \choose 6} + {7 \choose 3}{7 \choose 1} \right]
\end{equation*}
%
\end{enumerate}
%
\end{divisionsolution}%
\begin{divisionsolution}{1.6.3}{}{p:exercise:Pdl}%
\emph{Conic}, your favorite math themed fast food drive-in offers 20 flavors which can be added to your soda. You have enough money to buy a large soda with 4 added flavors. How many different soda concoctions can you order if:%
\begin{enumerate}[label=(\alph*)]
\item{}You refuse to use any of the flavors more than once?%
\item{}You refuse repeats but care about the order the flavors are added?%
\item{}You allow yourself multiple shots of the same flavor?%
\item{}You allow yourself multiple shots, and care about the order the flavors are added?%
\end{enumerate}
%
\par\smallskip%
\noindent\textbf{\blocktitlefont Answer 1}.\quad{}\(\mathop{\rm C}\nolimits\!\left(20,4\right)\)%
\par\smallskip%
\noindent\textbf{\blocktitlefont Answer 2}.\quad{}\(\mathop{\rm P}\nolimits\!\left(20,4\right)\)%
\par\smallskip%
\noindent\textbf{\blocktitlefont Answer 3}.\quad{}\(\mathop{\rm C}\nolimits\!\left(23,4\right)\)%
\par\smallskip%
\noindent\textbf{\blocktitlefont Answer 4}.\quad{}\(20^{4}\)%
\par\smallskip%
\noindent\textbf{\blocktitlefont Solution}.\quad{}%
\begin{enumerate}[label=(\alph*)]
\item{}\({20 \choose 4}\) sodas (order does not matter and repeats are not allowed).%
\item{}\(P(20, 4) = 20\cdot 19\cdot 18 \cdot 17\) sodas (order matters and repeats are not allowed).%
\item{}\({23 \choose 4}\) sodas (order does not matter and repeats are allowed; 4 stars and 19 bars).%
\item{}\(20^4\) sodas (order matters and repeats are allowed; 20 choices 4 times).%
\end{enumerate}
%
\end{divisionsolution}%
\begin{divisionsolution}{1.6.4}{}{p:exercise:vku}%
The dollar menu at your favorite tax-free fast food restaurant has 7 items. You have \textdollar{}10 to spend. How many different meals can you buy if you spend all your money and:%
\begin{enumerate}[label=(\alph*)]
\item{}Purchase at least one of each item.%
\item{}Possibly skip some items.%
\item{}Don't get more than 2 of any particular item.%
\end{enumerate}
%
\par\smallskip%
\noindent\textbf{\blocktitlefont Answer 1}.\quad{}\(\mathop{\rm C}\nolimits\!\left(9,6\right)\)%
\par\smallskip%
\noindent\textbf{\blocktitlefont Answer 2}.\quad{}\(\mathop{\rm C}\nolimits\!\left(16,6\right)\)%
\par\smallskip%
\noindent\textbf{\blocktitlefont Answer 3}.\quad{}\(\mathop{\rm C}\nolimits\!\left(16,6\right)-\left(\mathop{\rm C}\nolimits\!\left(7,1\right)\mathop{\rm C}\nolimits\!\left(13,6\right)-\mathop{\rm C}\nolimits\!\left(7,2\right)\mathop{\rm C}\nolimits\!\left(10,6\right)+\mathop{\rm C}\nolimits\!\left(7,3\right)\mathop{\rm C}\nolimits\!\left(7,6\right)\right)\)%
\par\smallskip%
\noindent\textbf{\blocktitlefont Solution}.\quad{}%
\begin{enumerate}[label=(\alph*)]
\item{}\({9 \choose 3}\) meals. First spend \textdollar{}7 to buy one of each item, then use 3 stars to select items between 6 bars.%
\item{}\({16 \choose 6}\) meals. Here you have 10 stars and 6 bars (separating the 7 items).%
\item{}\({16 \choose 6} - \left[{7 \choose 1}{13 \choose 6} - {7 \choose 2}{10 \choose 6} + {7 \choose 3}{7 \choose 6}\right]\) meals. Use PIE to subtract all the meals in which you get 3 or more of a particular item.%
\end{enumerate}
%
\end{divisionsolution}%
\begin{divisionsolution}{1.6.5}{}{p:exercise:brD}%
After another gym class you are tasked with putting the 14 identical dodgeballs away into 5 bins. This time, no bin can hold more than 6 balls. How many ways can you clean up?%
\par\smallskip%
\noindent\textbf{\blocktitlefont Answer}.\quad{}\(\mathop{\rm C}\nolimits\!\left(18,4\right)-\left(\mathop{\rm C}\nolimits\!\left(5,1\right)\mathop{\rm C}\nolimits\!\left(11,4\right)-\mathop{\rm C}\nolimits\!\left(5,2\right)\mathop{\rm C}\nolimits\!\left(4,4\right)\right)\)%
\par\smallskip%
\noindent\textbf{\blocktitlefont Solution}.\quad{}\({18 \choose 4} - \left[ {5 \choose 1}{11 \choose 4} - {5 \choose 2}{4 \choose 4}\right]\text{.}\)%
\end{divisionsolution}%
\begin{divisionsolution}{1.6.6}{}{p:exercise:HyM}%
Based on the previous question, give a combinatorial proof for the identity:%
\begin{equation*}
{n \choose k} = {n+k-1 \choose k} - \sum_{j=1}^n (-1)^{j+1}{n \choose j}{n+k-(2j+1) \choose k - 2j}\text{.}
\end{equation*}
%
\par\smallskip%
\noindent\textbf{\blocktitlefont Solution}.\quad{}The question is, how many ways can you distribute \(k\) cookies to \(n\) kids so that each kid gets at most one cookie. On one hand, the answer is just \({n \choose k}\) since you must choose \(k\) kids to get a cookie. Alternatively, we can use stars and bars with PIE, which is how we get the right-hand side of the identity. Note that lots of the terms on the right-hand side will be zero, as soon as \(n+k-(2j+1)\) drops below \(k\).%
\end{divisionsolution}%
\begin{divisionsolution}{1.6.7}{}{p:exercise:nFV}%
Illustrate how the counting of derangements works by writing all permutations of \(\{1,2,3,4\}\) and the crossing out those which are not derangements. Keep track of the permutations you cross out more than once, using PIE.%
\par\smallskip%
\noindent\textbf{\blocktitlefont Solution}.\quad{}The 9 derangements are: 2143, 2341, 2413, 3142, 3412, 3421, 4123, 4312, 4321.%
\end{divisionsolution}%
\begin{divisionsolution}{1.6.8}{}{p:exercise:TNe}%
Consider the equation \(x_1 + x_2 + x_3 + x_4 = 15\text{.}\) How many solutions are there with \(2 \le x_i \le 5\) for all \(i \in \{1,2,3,4\}\text{?}\)%
\par\smallskip%
\noindent\textbf{\blocktitlefont Hint}.\quad{}Instead, count the solutions to \(y_1 + y_2 + y_3 + y_4 = 7\) with \(0 \le y_i \le 3\text{.}\)  Why is this equivalent?%
\par\smallskip%
\noindent\textbf{\blocktitlefont Answer}.\quad{}\(\mathop{\rm C}\nolimits\!\left(10,7\right)-\mathop{\rm C}\nolimits\!\left(4,1\right)\mathop{\rm C}\nolimits\!\left(6,3\right)\)%
\par\smallskip%
\noindent\textbf{\blocktitlefont Solution 1}.\quad{}There are \({10 \choose 7} - {4\choose 1} {6 \choose 3}\) solutions.%
\par\smallskip%
\noindent\textbf{\blocktitlefont Solution 2}.\quad{}The easiest way to solve this is to instead count the solutions to \(y_1 + y_2 + y_3 + y_4 = 7\) with \(0 \le y_i \le 3\text{.}\) By taking \(x_i = y_i+2\text{,}\) each solution to this new equation corresponds to exactly one solution to the original equation.%
\par
Now all the ways to distribute the 7 units to the four \(y_i\) variables can be found using stars and bars, specifically 7 stars and 3 bars, so \({10 \choose 7}\) ways. But this includes the ways that one or more \(y_i\) variables can be assigned more than 3 units. So subtract, using PIE. We get%
\par
%
\begin{equation*}
{10 \choose 7} - {4\choose 1} {6 \choose 3}
\text{.}
\end{equation*}
%
\par
The \({4 \choose 1}\) counts the number of ways to pick one variable to be over-assigned, the \({6 \choose 3}\) is the number of ways to assign the remaining 3 units to the 4 variables. Note that this is the final answer because it is not possible to have two variables both get 4 units.%
\end{divisionsolution}%
\begin{divisionsolution}{1.6.9}{}{p:exercise:zUn}%
Suppose you planned on giving 7 gold stars to some of the 13 star students in your class. Each student can receive at most one star. How many ways can you do this?%
\par
Use PIE. Then, find the numeric answer in Pascal's triangle and explain why that makes sense.%
\par\smallskip%
\noindent\textbf{\blocktitlefont Answer}.\quad{}\(\mathop{\rm C}\nolimits\!\left(13,7\right)\)%
\par\smallskip%
\noindent\textbf{\blocktitlefont Solution}.\quad{}Without any restriction, there would be \({19\choose 7}\) ways to distribute the stars. Now we must use PIE to eliminate all distributions in which one or more student gets more than one star:%
\par
%
\begin{equation*}
{19 \choose 7} - \left[{13 \choose 1}{17 \choose 5} - {13\choose 2}{15 \choose 3} + {13\choose 3}{13 \choose 1}\right] = 1716
\text{.}
\end{equation*}
%
\par
Interestingly enough, this number is also the value of \({13 \choose 7}\text{,}\) which makes sense: if each student can have at most one star, we must just pick the 7 out of 13 students to receive them.%
\end{divisionsolution}%
\begin{divisionsolution}{1.6.10}{}{p:exercise:gbw}%
The Grinch sneaks into a room with 6 Christmas presents to 6 different people. He proceeds to switch the name-labels on the presents. How many ways could he do this if:%
\begin{enumerate}[label=(\alph*)]
\item{}No present is allowed to end up with its original label? Explain what each term in your answer represents.%
\item{}Exactly 2 presents keep their original labels? Explain.%
\item{}Exactly 5 presents keep their original labels? Explain.%
\end{enumerate}
%
\par\smallskip%
\noindent\textbf{\blocktitlefont Solution}.\quad{}%
\begin{enumerate}[label=(\alph*)]
\item{}%
\begin{equation*}
6! - \left[{6 \choose 1}5! - {6 \choose 2}4! + {6 \choose 3}3! - {6 \choose 4}2! + {6 \choose 5}1! - {6 \choose 6}0!\right]
\end{equation*}
%
\item{}%
\begin{equation*}
{6 \choose 2}\left(4! - \left[{4\choose 1}3! - {4 \choose 2}2! + {4 \choose 3}1! - {4 \choose 4}0!\right]\right)
\end{equation*}
%
\item{}0. Once 5 presents have their original label, there is only one present left and only one label left, so the 6th present must get its own label.%
\end{enumerate}
%
\end{divisionsolution}%
\begin{divisionsolution}{1.6.11}{}{p:exercise:MiF}%
How many permutations of \(\{1,2,3,4,5\}\) leave exactly 1 element fixed?%
\par\smallskip%
\noindent\textbf{\blocktitlefont Answer}.\quad{}\(\mathop{\rm C}\nolimits\!\left(5,1\right)\!\left(4!-\left(\mathop{\rm C}\nolimits\!\left(4,1\right)\!\left(3!\right)-\mathop{\rm C}\nolimits\!\left(4,2\right)\!\left(2!\right)+\mathop{\rm C}\nolimits\!\left(4,3\right)\!\left(1!\right)-\mathop{\rm C}\nolimits\!\left(4,4\right)\!\left(0!\right)\right)\right)\)%
\par\smallskip%
\noindent\textbf{\blocktitlefont Solution}.\quad{}First pick one of the five elements to be fixed. For each such choice, derange the remaining four, using the standard advanced PIE formula. We get \({5 \choose 1}\left( 4! - \left[{4 \choose 1}3! - {4 \choose 2}2! + {4 \choose 3} 1! - {4 \choose 4} 0!\right] \right)\) permutations.%
\end{divisionsolution}%
\begin{divisionsolution}{1.6.12}{}{p:exercise:spO}%
Ten ladies of a certain age drop off their red hats at the hat check of a museum. As they are leaving, the hat check attendant gives the hats back randomly. In how many ways can exactly six of the ladies receive their own hat (and the other four not)? Explain.%
\par\smallskip%
\noindent\textbf{\blocktitlefont Answer}.\quad{}\(\mathop{\rm C}\nolimits\!\left(10,6\right)\!\left(4!-\left(\mathop{\rm C}\nolimits\!\left(4,1\right)\!\left(3!\right)-\mathop{\rm C}\nolimits\!\left(4,2\right)\!\left(2!\right)+\mathop{\rm C}\nolimits\!\left(4,3\right)\!\left(1!\right)-\mathop{\rm C}\nolimits\!\left(4,4\right)\!\left(0!\right)\right)\right)\)%
\par\smallskip%
\noindent\textbf{\blocktitlefont Solution}.\quad{}\({10 \choose 6}\left(4! - \left[{4 \choose 1} 3! - {4 \choose 2}2! + {4 \choose 3}1! - {4 \choose 4}0!\right]\right)\) ways. We choose 6 of the 10 ladies to get their own hat, and the multiply by the number of ways the remaining hats can be deranged.%
\end{divisionsolution}%
\begin{divisionsolution}{1.6.13}{}{p:exercise:YwX}%
Consider functions \(f: \{1,2,3,4\} \to \{a,b,c,d,e,f\}\text{.}\) How many functions have the property that \(f(1) \ne a\) or \(f(2) \ne b\text{,}\) or both?%
\par\smallskip%
\noindent\textbf{\blocktitlefont Answer}.\quad{}\(5\cdot 6^{3}+5\cdot 6^{3}-5^{2}\cdot 6^{2}\)%
\par\smallskip%
\noindent\textbf{\blocktitlefont Solution}.\quad{}There are \(5 \cdot 6^3\) functions for which \(f(1) \ne a\) and another \(5 \cdot 6^3\) functions for which \(f(2) \ne b\text{.}\) There are \(5^2 \cdot 6^2\) functions for which both \(f(1) \ne a\) and \(f(2) \ne b\text{.}\) So the total number of functions for which \(f(1) \ne a\) or \(f(2) \ne b\) or both is%
\par
%
\begin{equation*}
5 \cdot 6^3 + 5 \cdot 6^3 - 5^2 \cdot 6^2 = 1260
\text{.}
\end{equation*}
%
\end{divisionsolution}%
\begin{divisionsolution}{1.6.14}{}{p:exercise:EEg}%
Let \(d_n\) be the number of derangements of \(n\) objects. For example, using the techniques of this section, we find%
\begin{equation*}
d_3 = 3!-\left({3 \choose 1}2! - {3 \choose 2}1! + {3 \choose 3}0! \right)\text{.}
\end{equation*}
We can use the formula for \({n \choose k}\) to write this all in terms of factorials. After simplifying, for \(d_3\) we would get%
\begin{equation*}
d_3 = 3!\left(1 - \frac{1}{1} + \frac{1}{2} - \frac{1}{6} \right)\text{.}
\end{equation*}
Generalize this to find a nicer formula for \(d_n\). Bonus: For large \(n\), approximately what fraction of all permutations are derangements? Use your knowledge of Taylor series from calculus.%
\end{divisionsolution}%
\section*{1.7 Chapter Summary}
\addcontentsline{toc}{section}{1.7 Chapter Summary}
\sectionmark{1.7 Chapter Summary}
\subsection*{Chapter Review}
\addcontentsline{toc}{subsection}{Chapter Review}
\begin{divisionsolution}{1.7.1}{}{p:exercise:PrT}%
Consider sets \(A\) and \(B\) with \(|A| = 10\) and \(|B| = 5\text{.}\) How many functions \(f: A \to B\) are surjective?%
\par\smallskip%
\noindent\textbf{\blocktitlefont Answer}.\quad{}\(5^{10}-\left(\mathop{\rm C}\nolimits\!\left(5,1\right)\cdot 4^{10}-\mathop{\rm C}\nolimits\!\left(5,2\right)\cdot 3^{10}+\mathop{\rm C}\nolimits\!\left(5,3\right)\cdot 2^{10}-\mathop{\rm C}\nolimits\!\left(5,4\right)\cdot 1^{10}\right)\)%
\par\smallskip%
\noindent\textbf{\blocktitlefont Solution}.\quad{}\(5^{10} - \left[{5 \choose 1}4^{10} - {5 \choose 2}3^{10} + {5 \choose 3}2^{10} - {5 \choose 4}1^{10}\right]\) functions. The \(5^{10}\) is all the functions from \(A\) to \(B\text{.}\) We subtract those that aren't surjective. Pick one of the five elements in \(B\) to not have in the range (in \({5 \choose 1}\) ways) and count all those functions (\(4^{10}\)). But this overcounts the functions where two elements from \(B\) are excluded from the range, so subtract those. And so on, using PIE.%
\end{divisionsolution}%
\begin{divisionsolution}{1.7.2}{}{p:exercise:vzc}%
For each of the following counting problems, say whether the answer is \({10\choose 4}\), \(P(10,4)\), or neither. If you answer is ``neither,'' say what the answer should be instead.%
\begin{enumerate}[label=(\alph*)]
\item{}How many shortest lattice paths are there from \((0,0)\) to \((10,4)\)?%
\item{}If you have 10 bow ties, \index{bow ties} and you want to select 4 of them for next week, how many choices do you have?%
\item{}Suppose you have 10 bow ties and you will wear a different one on each of the next 4 days. How many choices do you have?%
\item{}If you want to wear 4 of your 10 bow ties next week (Monday through Sunday), how many ways can this be accomplished?%
\item{}Out of a group of 10 classmates, how many ways can you rank your top 4 friends?%
\item{}If 10 students come to their professor's office but only 4 can fit at a time, how different combinations of 4 students can see the prof first?%
\item{}How many 4 letter words can be made from the first 10 letters of the alphabet?%
\item{}How many ways can you make the word ``cake'' from the first 10 letters of the alphabet?%
\item{}How many ways are there to distribute 10 identical apples among 4 children?%
\item{}If you have 10 kids (and live in a shoe) and 4 types of cereal, how many ways can your kids eat breakfast?%
\item{}How many ways can you arrange exactly 4 ones in a string of 10 binary digits?%
\item{}You want to select 4 distinct, single-digit numbers as your lotto picks. How many choices do you have?%
\item{}10 kids want ice-cream. You have 4 varieties. How many ways are there to give the kids as much ice-cream as they want?%
\item{}How many 1-1 functions are there from \(\{1,2,\ldots, 10\}\) to \(\{a,b,c,d\}\)?%
\item{}How many surjective functions are there from \(\{1,2,\ldots, 10\}\) to \(\{a,b,c,d\}\)?%
\item{}Each of your 10 bow ties match 4 pairs of suspenders. How many outfits can you make?%
\item{}After the party, the 10 kids each choose one of 4 party-favors. How many outcomes?%
\item{}How many 6-elements subsets are there of the set \(\{1,2,\ldots, 10\}\)%
\item{}How many ways can you split up 11 kids into 5 named teams?%
\item{}How many solutions are there to \(x_1 + x_2 + \cdots + x_5 = 6\) where each \(x_i\) is a non-negative integer?%
\item{}Your band goes on tour. There are 10 cities within driving distance, but only enough time to play 4 of them. How many choices do you have for the cities on your tour?%
\item{}In how many different ways can you play the 4 cities you choose?%
\item{}Out of the 10 breakfast cereals available, you want to have 4 bowls. How many ways can you do this?%
\item{}There are 10 types of cookies available. You want to make a 4 cookie stack. How many different stacks can you make?%
\item{}From your home at (0,0) you want to go to either the donut shop at (5,4) or the one at (3,6). How many paths could you take?%
\item{}How many 10-digit numbers do not contain a sub-string of 4 repeated digits?%
\end{enumerate}
%
\par\smallskip%
\noindent\textbf{\blocktitlefont Solution}.\quad{}%
\begin{enumerate}[label=(\alph*)]
\item{}Neither. \({14 \choose 4}\) paths.%
\item{}\({10\choose 4}\) bow ties.%
\item{}\(P(10,4)\), since order is important.%
\item{}Neither. Assuming you will wear each of the 4 ties on just 4 of the 7 days, without repeats: \({10\choose 4}P(7,4)\).%
\item{}\(P(10,4)\).%
\item{}\({10\choose 4}\).%
\item{}Neither. Since you could repeat letters: \(10^4\). If no repeats are allowed, it would be \(P(10,4)\).%
\item{}Neither. Actually, ``k'' is the 11th letter of the alphabet, so the answer is 0. If ``k'' was among the first 10 letters, there would only be 1 way - write it down.%
\item{}Neither. Either \({9\choose 3}\) (if every kid gets an apple) or \({13 \choose 3}\) (if appleless kids are allowed).%
\item{}Neither. Note that this could not be \({10 \choose 4}\) since the 10 things and 4 things are from different groups. \(4^{10}\), assuming each kid eats one type of cerial.%
\item{}\({10 \choose 4}\) - don't be fooled by the ``arrange'' in there - you are picking 4 out of 10 \emph{spots} to put the 1's.%
\item{}\({10 \choose 4}\) (assuming order is irrelevant).%
\item{}Neither. \(16^{10}\) (each kid chooses yes or no to 4 varieties).%
\item{}Neither. 0.%
\item{}Neither. \(4^{10} - [{4\choose 1}3^{10} - {4\choose 2}2^{10} + {4 \choose 3}1^{10}]\).%
\item{}Neither. \(10\cdot 4\).%
\item{}Neither. \(4^{10}\).%
\item{}\({10 \choose 4}\) (which is the same as \({10 \choose 6}\)).%
\item{}Neither. If all the kids were identical, and you wanted no empty teams, it would be \({10 \choose 4}\). Instead, this will be the same as the number of surjective functions from a set of size 11 to a set of size 5.%
\item{}\({10 \choose 4}\).%
\item{}\({10 \choose 4}\).%
\item{}Neither. \(4!\).%
\item{}Neither. It's \({10 \choose 4}\) if you won't repeat any choices. If repetition is allowed, then this becomes \(x_1 + x_2 + \cdots +x_{10} = 4\), which has \({13 \choose 9}\) solutions in non-negative integers.%
\item{}Neither. Since repetition of cookie type is allowed, the answer is \(10^4\). Without repetition, you would have \(P(10,4)\).%
\item{}\({10 \choose 4}\) since that is equal to \({9 \choose 4} + {9 \choose 3}\).%
\item{}Neither. It will be a complicated (possibly PIE) counting problem.%
\end{enumerate}
%
\end{divisionsolution}%
\begin{divisionsolution}{1.7.3}{}{p:exercise:bGl}%
Let \(A = \{1,2,3,4,5\}\text{.}\) How many injective functions \(f:A \to A\) have the property that for each \(x \in A\text{,}\) \(f(x) \ne x\text{?}\)%
\par\smallskip%
\noindent\textbf{\blocktitlefont Hint}.\quad{}This is a sneaky way to ask for the number of derangements on 5 elements.  Do you see why?%
\par\smallskip%
\noindent\textbf{\blocktitlefont Answer}.\quad{}\(5!-\left(\mathop{\rm C}\nolimits\!\left(5,1\right)\!\left(4!\right)-\mathop{\rm C}\nolimits\!\left(5,2\right)\!\left(3!\right)+\mathop{\rm C}\nolimits\!\left(5,3\right)\!\left(2!\right)-\mathop{\rm C}\nolimits\!\left(5,4\right)\!\left(1!\right)+\mathop{\rm C}\nolimits\!\left(5,5\right)\!\left(0!\right)\right)\)%
\par\smallskip%
\noindent\textbf{\blocktitlefont Solution}.\quad{}\(5! - \left[{5 \choose 1}4! - {5 \choose 2}3! + {5 \choose 3}2! - {5 \choose 4}1! + {5 \choose 5}0!\right]\) functions.%
\end{divisionsolution}%
\begin{divisionsolution}{1.7.4}{}{p:exercise:HNu}%
Give a counting question where the answer is \(8\cdot 3 \cdot 3 \cdot 5\). Give another question where the answer is \(8 + 3 + 3 + 5\).%
\par\smallskip%
\noindent\textbf{\blocktitlefont Solution}.\quad{}You own 8 purple bow ties, \index{bow ties} 3 red bow ties, 3 blue bow ties and 5 green bow ties. How many ways can you select one of each color bow tie to take with you on a trip? \(8 \cdot 3 \cdot 3 \cdot 5\) ways. How many choices do you have for a single bow tie to wear tomorrow? \(8 + 3 + 3 + 5\) choices.%
\end{divisionsolution}%
\begin{divisionsolution}{1.7.5}{}{p:exercise:nUD}%
You have 9 presents to give to your 4 kids. How many ways can this be done if:%
\begin{enumerate}[label=(\alph*)]
\item{}The presents are identical, and each kid gets at least one present?%
\item{}The presents are identical, and some kids might get no presents?%
\item{}The presents are unique, and some kids might get no presents?%
\item{}The presents are unique and each kid gets at least one present?%
\end{enumerate}
%
\par\smallskip%
\noindent\textbf{\blocktitlefont Answer 1}.\quad{}\(\mathop{\rm C}\nolimits\!\left(8,5\right)\)%
\par\smallskip%
\noindent\textbf{\blocktitlefont Answer 2}.\quad{}\(\mathop{\rm C}\nolimits\!\left(12,9\right)\)%
\par\smallskip%
\noindent\textbf{\blocktitlefont Answer 3}.\quad{}\(4^{9}\)%
\par\smallskip%
\noindent\textbf{\blocktitlefont Answer 4}.\quad{}\(4^{9}-\left(\mathop{\rm C}\nolimits\!\left(4,1\right)\cdot 3^{9}-\mathop{\rm C}\nolimits\!\left(4,2\right)\cdot 2^{9}+\mathop{\rm C}\nolimits\!\left(4,3\right)\cdot 1^{9}\right)\)%
\par\smallskip%
\noindent\textbf{\blocktitlefont Solution}.\quad{}%
\begin{enumerate}[label=(\alph*)]
\item{}\({8 \choose 5}\) ways, after giving one present to each kid, you are left with 5 presents (stars) which need to be divide among the 4 kids (giving 3 bars).%
\item{}\({12 \choose 9}\) ways. You have 9 stars and 3 bars.%
\item{}\(4^9\text{.}\) You have 4 choices for whom to give each present. This is like making a function from the set of presents to the set of kids.%
\item{}\(4^9 - \left[{4 \choose 1}3^9 - {4\choose 2}2^9 + {4 \choose 3}1^9 \right]\) ways. Now the function from the set of presents to the set of kids must be surjective.%
\end{enumerate}
%
\end{divisionsolution}%
\begin{divisionsolution}{1.7.6}{}{p:exercise:UbM}%
bow tiesRecall, you own 3 regular ties and 5 bow ties. You realize that it would be okay to wear more than two ties to your clown college interview.%
\begin{enumerate}[label=(\alph*)]
\item{}You must select some of your ties to wear. Everything is okay, from no ties up to all ties. How many choices do you have?%
\item{}If you want to wear at least one regular tie and one bow tie, but are willing to wear up to all your ties, how many choices do you have for which ties to wear?%
\item{}How many choices of which ties to wear do you have if you wear exactly 2 of the 3 regular ties and 3 of the 5 bow ties?%
\item{}Once you have selected 2 regular and 3 bow ties, in how many orders could you put the ties on, assuming you must have one of the three bow ties on top?%
\end{enumerate}
%
\par\smallskip%
\noindent\textbf{\blocktitlefont Answer 1}.\quad{}\(2^{8}\)%
\par\smallskip%
\noindent\textbf{\blocktitlefont Answer 2}.\quad{}\(7\cdot 31\)%
\par\smallskip%
\noindent\textbf{\blocktitlefont Answer 3}.\quad{}\(\mathop{\rm C}\nolimits\!\left(3,2\right)\mathop{\rm C}\nolimits\!\left(5,3\right)\)%
\par\smallskip%
\noindent\textbf{\blocktitlefont Answer 4}.\quad{}\(3\cdot \left(4!\right)\)%
\par\smallskip%
\noindent\textbf{\blocktitlefont Solution}.\quad{}%
\begin{enumerate}[label=(\alph*)]
\item{}\(2^8 = 256\) choices. You have two choices for each tie: wear it or don't.%
\item{}You have 7 choices for regular ties (the 8 choices less the ``no regular tie'' option) and 31 choices for bow ties (32 total minus the ``no bow tie'' option). Thus total you have \(7 \cdot 31 = 217\) choices.%
\item{}\({3\choose 2}{5\choose 3} = 30\) choices.%
\item{}Select one of the 3 bow ties to go on top. There are then 4 choices for the next tie, 3 for the tie after that, and so on. Thus \(3\cdot 4! = 72\) choices.%
\end{enumerate}
%
\end{divisionsolution}%
\begin{divisionsolution}{1.7.7}{}{p:exercise:AiV}%
Consider five digit numbers \(\alpha = a_1a_2a_3a_4a_5\text{,}\) with each digit from the set \(\{1,2,3,4\}\text{.}\)%
\begin{enumerate}[label=(\alph*)]
\item{}How many such numbers are there?%
\item{}How many such numbers are there for which the \emph{sum} of the digits is even?%
\item{}How many such numbers contain more even digits than odd digits?%
\end{enumerate}
%
\par\smallskip%
\noindent\textbf{\blocktitlefont Answer 1}.\quad{}\(4^{5}\)%
\par\smallskip%
\noindent\textbf{\blocktitlefont Answer 2}.\quad{}\(4^{4}\cdot 2\)%
\par\smallskip%
\noindent\textbf{\blocktitlefont Answer 3}.\quad{}\(\mathop{\rm C}\nolimits\!\left(5,3\right)\cdot 2^{3}\cdot 2^{2}+\mathop{\rm C}\nolimits\!\left(5,4\right)\cdot 2^{4}\cdot 2+\mathop{\rm C}\nolimits\!\left(5,5\right)\cdot 2^{5}\)%
\par\smallskip%
\noindent\textbf{\blocktitlefont Solution}.\quad{}%
\begin{enumerate}[label=(\alph*)]
\item{}\(4^5\) numbers.%
\item{}\(4^4\cdot 2\) numbers (choose any digits for the first four digits - then pick either an even or an odd last digit to make the sum even).%
\item{}We need 3 or more even digits. 3 even digits: \({5 \choose 3}2^3 2^2\text{.}\) 4 even digits: \({5 \choose 4}2^4 2\text{.}\) 5 even digits: \({5 \choose 5}2^5\text{.}\) So all together: \({5 \choose 3}2^3 2^2 + {5 \choose 4}2^4 2 + {5 \choose 5}2^5\) numbers.%
\end{enumerate}
%
\end{divisionsolution}%
\begin{divisionsolution}{1.7.8}{}{p:exercise:gqe}%
In a recent small survey of airline passengers, 25 said they had flown American in the last year, 30 had flown Jet Blue, and 20 had flown Continental. Of those, 10 reported they had flown on American and Jet Blue, 12 had flown on Jet Blue and Continental, and 7 had flown on American and Continental. 5 passengers had flown on all three airlines.%
\par
How many passengers were surveyed? (Assume the results above make up the entire survey.)%
\par\smallskip%
\noindent\textbf{\blocktitlefont Answer}.\quad{}\(51\)%
\par\smallskip%
\noindent\textbf{\blocktitlefont Solution}.\quad{}51 passengers. We are asking for the size of the union of three non-disjoint sets. Using PIE, we have \(25+30+20-10 -12-7+5 = 51\text{.}\)%
\end{divisionsolution}%
\begin{divisionsolution}{1.7.9}{}{p:exercise:Mxn}%
Recall, by \(8\)-bit strings, we mean strings of binary digits, of length 8.%
\begin{enumerate}[label=(\alph*)]
\item{}How many \(8\)-bit strings are there total?%
\item{}How many \(8\)-bit strings have weight 5?%
\item{}How many subsets of the set \(\{a,b,c,d,e,f,g,h\}\) contain exactly 5 elements?%
\item{}Explain why your answers to parts (b) and (c) are the same. Why are these questions equivalent?%
\end{enumerate}
%
\par\smallskip%
\noindent\textbf{\blocktitlefont Answer 1}.\quad{}\(2^{8}\)%
\par\smallskip%
\noindent\textbf{\blocktitlefont Answer 2}.\quad{}\(\mathop{\rm C}\nolimits\!\left(8,5\right)\)%
\par\smallskip%
\noindent\textbf{\blocktitlefont Answer 3}.\quad{}\(\mathop{\rm C}\nolimits\!\left(8,5\right)\)%
\par\smallskip%
\noindent\textbf{\blocktitlefont Solution}.\quad{}%
\begin{enumerate}[label=(\alph*)]
\item{}\(2^8\) strings.%
\item{}\({8 \choose 5}\) strings.%
\item{}\({8 \choose 5}\) strings.%
\item{}There is a bijection between subsets and bit strings: a 1 means that element in is the subset, a 0 means that element is not in the subset. To get a subset of an 8 element set we have a 8-bit string. To make sure the subset contains exactly 5 elements, there must be 5 1's, so the weight must be 5.%
\end{enumerate}
%
\end{divisionsolution}%
\begin{divisionsolution}{1.7.10}{}{p:exercise:sEw}%
What is the coefficient of \(x^{10}\) in the expansion of \((x+1)^{13} + x^2(x+1)^{17}\text{?}\)%
\par\smallskip%
\noindent\textbf{\blocktitlefont Answer}.\quad{}\(\mathop{\rm C}\nolimits\!\left(13,10\right)+\mathop{\rm C}\nolimits\!\left(17,8\right)\)%
\par\smallskip%
\noindent\textbf{\blocktitlefont Solution}.\quad{}\({13 \choose 10} + {17 \choose 8}\text{.}\)%
\end{divisionsolution}%
\begin{divisionsolution}{1.7.11}{}{p:exercise:YLF}%
How many 8-letter words contain exactly 5 vowels? (One such word is ``aaioobtt''; don't consider ``y'' a vowel for this exercise.)%
\par
What if repeated letters were not allowed?%
\par\smallskip%
\noindent\textbf{\blocktitlefont Answer 1}.\quad{}\(\mathop{\rm C}\nolimits\!\left(8,5\cdot 5^{5}\cdot 21^{3}\right)\)%
\par\smallskip%
\noindent\textbf{\blocktitlefont Answer 2}.\quad{}\(\mathop{\rm C}\nolimits\!\left(8,5\right)\!\left(5!\right)\mathop{\rm P}\nolimits\!\left(21,3\right)\)%
\par\smallskip%
\noindent\textbf{\blocktitlefont Solution}.\quad{}With repeated letters allowed, we select which 5 of the 8 letters will be vowels, then pick one of the 5 vowels for each spot, and finally pick one of the other 21 letters for each of the remaining 3 spots. Thus, \({8 \choose 5}5^5 21^3\) words.%
\par
Without repeats, we still pick the positions of the vowels, but now each time we place one there, there is one fewer choice for the next one. Similarly, we cannot repeat the consonants. We get \({8 \choose 5}5!
P(21, 3)\) words.%
\end{divisionsolution}%
\begin{divisionsolution}{1.7.12}{}{p:exercise:ESO}%
For each of the following, find the number of shortest lattice paths from \((0,0)\) to \((8,8)\) which:%
\begin{enumerate}[label=(\alph*)]
\item{}pass through the point \((2,3)\text{.}\)%
\item{}avoid (do not pass through) the point \((7,5)\text{.}\)%
\item{}either pass through \((2,3)\) or \((5,7)\) (or both).%
\end{enumerate}
%
\par\smallskip%
\noindent\textbf{\blocktitlefont Answer 1}.\quad{}\(\mathop{\rm C}\nolimits\!\left(5,2\right)\mathop{\rm C}\nolimits\!\left(11,6\right)\)%
\par\smallskip%
\noindent\textbf{\blocktitlefont Answer 2}.\quad{}\(\mathop{\rm C}\nolimits\!\left(16,8\right)-\mathop{\rm C}\nolimits\!\left(12,7\right)\mathop{\rm C}\nolimits\!\left(4,1\right)\)%
\par\smallskip%
\noindent\textbf{\blocktitlefont Answer 3}.\quad{}\(\mathop{\rm C}\nolimits\!\left(5,2\right)\mathop{\rm C}\nolimits\!\left(11,6\right)+\mathop{\rm C}\nolimits\!\left(12,5\right)\mathop{\rm C}\nolimits\!\left(4,3\right)-\mathop{\rm C}\nolimits\!\left(5,2\right)\mathop{\rm C}\nolimits\!\left(7,3\right)\mathop{\rm C}\nolimits\!\left(4,3\right)\)%
\par\smallskip%
\noindent\textbf{\blocktitlefont Solution}.\quad{}%
\begin{enumerate}[label=(\alph*)]
\item{}\({5 \choose 2}{11 \choose 6}\) paths.%
\item{}\({16 \choose 8} - {12 \choose 7}{4 \choose 1}\) paths.%
\item{}\({5 \choose 2}{11 \choose 6} + {12 \choose 5}{4 \choose 3} - {5 \choose 2}{7 \choose 3}{4 \choose 3}\) paths.%
\end{enumerate}
%
\end{divisionsolution}%
\begin{divisionsolution}{1.7.13}{}{p:exercise:kZX}%
You live in Grid-Town on the corner of 2nd and 3rd, and work in a building on the corner of 10th and 13th. How many routes are there which take you from home to work and then back home, but by a different route?%
\par\smallskip%
\noindent\textbf{\blocktitlefont Answer}.\quad{}\(\mathop{\rm C}\nolimits\!\left(18,8\right)\!\left(\mathop{\rm C}\nolimits\!\left(18,8\right)-1\right)\)%
\par\smallskip%
\noindent\textbf{\blocktitlefont Solution}.\quad{}\({18 \choose 8}\left({18 \choose 8} - 1\right)\) routes. For each of the \(\binom{18}{8}\) routes to work there is exactly one fewer route back.%
\end{divisionsolution}%
\begin{divisionsolution}{1.7.14}{}{p:exercise:Rhg}%
How many 10-bit strings start with \(111\) or end with \(101\) or both?%
\par\smallskip%
\noindent\textbf{\blocktitlefont Answer}.\quad{}\(2^{7}+2^{7}-2^{4}\)%
\par\smallskip%
\noindent\textbf{\blocktitlefont Solution}.\quad{}\(2^7 + 2^7 - 2^4\) strings (using PIE).%
\end{divisionsolution}%
\begin{divisionsolution}{1.7.15}{}{p:exercise:xop}%
Explain using lattice paths why \(\sum_{k=0}^n {n \choose k} = 2^n\).%
\par\smallskip%
\noindent\textbf{\blocktitlefont Solution}.\quad{}\(2^n\) is the number of lattice paths which have length \(n\), since for each step you can go up or right. Such a path would end along the line \(x + y = n\). So you will end at \((0,n)\), or \((1,n-1)\) or \((2, n-2)\) or \textellipsis{} or \((n,0)\). Counting the paths to each of these points separately, give \({n \choose 0}\), \({n \choose 1}\), \({n \choose 2}\), \textellipsis{}, \({n \choose n}\) (each time choosing which of the \(n\) steps to be to the right). These two methods count the same quantity, so are equal.%
\end{divisionsolution}%
\begin{divisionsolution}{1.7.16}{}{p:exercise:dvy}%
How many 10-bit strings of weight 6 start with \(111\) or end with \(101\) or both?%
\par\smallskip%
\noindent\textbf{\blocktitlefont Answer}.\quad{}\(\mathop{\rm C}\nolimits\!\left(7,3\right)+\mathop{\rm C}\nolimits\!\left(7,4\right)-\mathop{\rm C}\nolimits\!\left(4,1\right)\)%
\par\smallskip%
\noindent\textbf{\blocktitlefont Solution}.\quad{}\({7 \choose 3} + {7 \choose 4} - {4 \choose 1}\) strings.%
\end{divisionsolution}%
\begin{divisionsolution}{1.7.17}{}{p:exercise:JCH}%
How many 6 letter words made from the letters \(a,b,c,d,e,f\) without repeats do not contain the sub-word ``bad'' in consecutive letters?%
\par
How many don't to contain the subword ``bad'' in not-necessarily consecutive letters (but in order)?%
\par\smallskip%
\noindent\textbf{\blocktitlefont Answer 1}.\quad{}\(6!-4\cdot \left(3!\right)\)%
\par\smallskip%
\noindent\textbf{\blocktitlefont Answer 2}.\quad{}\(6!-\mathop{\rm C}\nolimits\!\left(6,3\right)\!\left(3!\right)\)%
\par\smallskip%
\noindent\textbf{\blocktitlefont Solution}.\quad{}There are 4 spots to start the word, and then there are \(3!\) ways to arrange the other letters in the remaining three spots. Thus the number of words avoiding the sub-word ``bad'' in consecutive letters is \(6! - 4\cdot 3!\text{.}\)%
\par
If we now need to avoid words that put ``b'' before ``a'' before ``d'', we must choose which spots those letters go (in that order) and then arrange the remaining three letters. Thus, \(6! - {6 \choose 3}3!\) words.%
\end{divisionsolution}%
\begin{divisionsolution}{1.7.18}{}{p:exercise:pJQ}%
Suppose you have 20 one-dollar bills to give out as prizes to your top 5 discrete math students. How many ways can you do this if:%
\begin{enumerate}[label=(\alph*)]
\item{}Each of the 5 students gets at least 1 dollar?%
\item{}Some students might get nothing?%
\item{}Each student gets at least 1 dollar but no more than 7 dollars?%
\end{enumerate}
%
\par\smallskip%
\noindent\textbf{\blocktitlefont Hint}.\quad{}Stars and bars.%
\par\smallskip%
\noindent\textbf{\blocktitlefont Answer 1}.\quad{}\(\mathop{\rm C}\nolimits\!\left(19,15\right)\)%
\par\smallskip%
\noindent\textbf{\blocktitlefont Answer 2}.\quad{}\(\mathop{\rm C}\nolimits\!\left(24,20\right)\)%
\par\smallskip%
\noindent\textbf{\blocktitlefont Answer 3}.\quad{}\(\mathop{\rm C}\nolimits\!\left(19,15\right)-\left(\mathop{\rm C}\nolimits\!\left(5,1\right)\mathop{\rm C}\nolimits\!\left(12,8\right)-\mathop{\rm C}\nolimits\!\left(5,2\right)\mathop{\rm C}\nolimits\!\left(5,1\right)\right)\)%
\par\smallskip%
\noindent\textbf{\blocktitlefont Solution}.\quad{}%
\begin{enumerate}[label=(\alph*)]
\item{}\({19 \choose 15}\) ways.%
\item{}\({24 \choose 20}\) ways.%
\item{}\({19 \choose 15} - \left[{5 \choose 1}{12 \choose 8} - {5 \choose 2}{5 \choose 1} \right]\) ways.%
\end{enumerate}
%
\end{divisionsolution}%
\begin{divisionsolution}{1.7.19}{}{x:exercise:exr-cookie-counting}%
How many functions \(f: \{1,2,3,4,5\} \to \{a,b,c,d,e\}\) are there satisfying:%
\begin{enumerate}[label=(\alph*)]
\item{}\(f(1) = a\) or \(f(2) = b\) (or both)?%
\item{}\(f(1) \ne a\) or \(f(2) \ne b\) (or both)?%
\item{}\(f(1) \ne a\) \emph{and} \(f(2) \ne b\text{,}\) and \(f\) is injective?%
\item{}\(f\) is surjective, but \(f(1) \ne a\text{,}\) \(f(2) \ne b\text{,}\) \(f(3) \ne c\text{,}\) \(f(4) \ne d\) and \(f(5) \ne e\text{?}\)%
\end{enumerate}
%
\par\smallskip%
\noindent\textbf{\blocktitlefont Answer 1}.\quad{}\(5^{4}+5^{4}-5^{3}\)%
\par\smallskip%
\noindent\textbf{\blocktitlefont Answer 2}.\quad{}\(4\cdot 5^{4}+5\cdot 4\cdot 5^{3}-4\cdot 4\cdot 5^{3}\)%
\par\smallskip%
\noindent\textbf{\blocktitlefont Answer 3}.\quad{}\(5!-\left(4!+4!-3!\right)\)%
\par\smallskip%
\noindent\textbf{\blocktitlefont Answer 4}.\quad{}\(5!-\left(\mathop{\rm C}\nolimits\!\left(5,1\right)\!\left(4!\right)-\mathop{\rm C}\nolimits\!\left(5,2\right)\!\left(3!\right)+\mathop{\rm C}\nolimits\!\left(5,3\right)\!\left(2!\right)-\mathop{\rm C}\nolimits\!\left(5,4\right)\!\left(1!\right)+\mathop{\rm C}\nolimits\!\left(5,5\right)\!\left(0!\right)\right)\)%
\par\smallskip%
\noindent\textbf{\blocktitlefont Solution}.\quad{}%
\begin{enumerate}[label=(\alph*)]
\item{}\(5^4 + 5^4 - 5^3\) functions.%
\item{}\(4\cdot 5^4 + 5 \cdot 4 \cdot 5^3 - 4 \cdot 4 \cdot 5^3\) functions.%
\item{}\(5! - \left[ 4! + 4! - 3! \right]\) functions. Note we use factorials instead of powers because we are looking for injective functions.%
\item{}Note that being surjective here is the same as being injective, so we can start with all \(5!\) injective functions and subtract those which have one or more ``fixed point''. We get \(5! - \left[{5 \choose 1}4! - {5 \choose 2}3! + {5 \choose 3}2! - {5 \choose 4}1! + {5 \choose 5} 0!\right]\) functions.%
\end{enumerate}
%
\end{divisionsolution}%
\begin{divisionsolution}{1.7.20}{}{p:exercise:BYi}%
For which of the parts of the previous problem (Exercise~1.7.19) does it make sense to interpret the counting question as counting some number of functions? Say what the domain and codomain should be, and whether you are counting all functions, injections, surjections, or something else.%
\par\smallskip%
\noindent\textbf{\blocktitlefont Solution}.\quad{}%
\begin{enumerate}[label=(\alph*)]
\item{}You are giving your professor 4 types of cookies coming from 10 different types of cookies. This does not lend itself well to a function interpretation. We \emph{could} say that the domain contains the 4 types you will give your professor and the codomain contains the 10 you can choose from, but then counting injections would be too much (it doesn't matter if you pick type 3 first and type 2 second, or the other way around, just that you pick those two types).%
\item{}We want to consider injective functions from the set \(\{\)most, second most, second least, least\(\}\) to the set of 10 cookie types. We want injections because we cannot pick the same type of cookie to give most and least of (for example).%
\item{}This is not a good problem to interpret as a function. The problem is that the domain would have to be the 12 cookies you bake, but these elements are indistinguishable (there is not a first cookie, second cookie, etc.).%
\item{}The domain should be the 12 shapes, the codomain the 10 types of cookies. Since we can use the same type for different shapes, we are interested in counting all functions here.%
\item{}Here we insist that each type of cookie be given at least once, so now we are asking for the number of surjections of those functions counted in the previous part.%
\end{enumerate}
%
\end{divisionsolution}%
\chapter*{2 Sequences}
\addcontentsline{toc}{chapter}{2 Sequences}
\chaptermark{2 Sequences}
\section*{2.1 Describing Sequences}
\addcontentsline{toc}{section}{2.1 Describing Sequences}
\sectionmark{2.1 Describing Sequences}
\subsection*{Exercises}
\addcontentsline{toc}{subsection}{Exercises}
\begin{divisionsolution}{2.1.1}{}{p:exercise:QUY}%
How many functions map \(\{1,2,3,4,5,6\}\) \emph{onto} \(\{a,b,c,d\}\) (i.e., how many \emph{surjections} are there)?%
\par\smallskip%
\noindent\textbf{\blocktitlefont Answer}.\quad{}\(4^{6}-\left(\mathop{\rm C}\nolimits\!\left(4,1\right)\cdot 3^{6}-\mathop{\rm C}\nolimits\!\left(4,2\right)\cdot 2^{6}+\mathop{\rm C}\nolimits\!\left(4,3\right)\cdot 1^{6}\right)\)%
\par\smallskip%
\noindent\textbf{\blocktitlefont Solution}.\quad{}\(4^6 - \left[{4 \choose 1}3^6 - {4 \choose 2}2^6 + {4 \choose 3} 1^6 \right]\text{.}\)%
\end{divisionsolution}%
\begin{divisionsolution}{2.1.2}{}{p:exercise:xch}%
For each sequence given below, find a closed formula for \(a_n\), the \(n\)th term of the sequence (assume the first terms are \(a_0\)) by relating it to another sequence for which you already know the formula. In each case, briefly say how you got your answers.%
\begin{enumerate}[label=(\alph*)]
\item{}4, 5, 7, 11, 19, 35, \textellipsis{}%
\item{}0, 3, 8, 15, 24, 35, \textellipsis{}%
\item{}6, 12, 20, 30, 42, \textellipsis{}%
\item{}0, 2, 7, 15, 26, 40, 57, \textellipsis{} (Cryptic Hint: these might be called ``house numbers'')%
\end{enumerate}
%
\par\smallskip%
\noindent\textbf{\blocktitlefont Solution}.\quad{}%
\begin{enumerate}[label=(\alph*)]
\item{}If we subtract 3 from each term, we get \(1, 2, 4, 8, 16, 32, \ldots\), which are the powers of 2. So we must shift this sequence up by 3 to get our sequence. Thus%
\begin{equation*}
a_n = 2^n + 3
\end{equation*}
%
\item{}Add 1 to each term - we get \(1, 4, 9, 16, 25, 36, \ldots\), the square numbers. So maybe our sequence as formula \(a_n = n^2 - 1\). Does this work? \(a_3 = 8\). That is a term of our sequence, but it should be \(a_2\). So we want%
\begin{equation*}
a_n = (n+1)^2 - 1
\end{equation*}
%
\item{}Each term is even, so let's see what happens when we divide by 2: we get \(3, 6, 10, 15, 21,\ldots\). These are the triangle numbers, but starting at \(T_2\). We know \(T_n = \frac{n(n+1)}{2}\). We want \(a_0 = 2T_2\), \(a_1 = 2T_3\), and so on. In general, \(a_n = 2T_{n+2}\), so%
\begin{equation*}
a_n = (n+2)(n+3)
\end{equation*}
%
\par
(This is like a shift by 2 units to the left and a stretch by a factor of 2.)%
\item{}How far off from triangular number are these? The triangular numbers (starting with \(T_0\)) are \(0, 1, 3, 6, 10, 15, 21, \ldots\). The given sequence differs from this by \(0, 1, 4, 9, 16, 25, 36, \ldots\) the square numbers! Thus%
\begin{equation*}
a_n = \frac{n(n+1)}{2} + n^2
\end{equation*}
%
\end{enumerate}
%
\end{divisionsolution}%
\begin{divisionsolution}{2.1.3}{}{p:exercise:djq}%
Write out the first 5 terms (starting with \(a_0\) ) of each of the sequences described below. Then give either a closed formula or a recursive definition for the sequence (whichever is NOT given in the problem).%
\begin{enumerate}[label=(\alph*)]
\item{}\(a_n = \frac{1}{2}(n^2 + n)\).%
\item{}\(a_n = 2a_{n-1} - a_{n-2}\) with \(a_0 = 0\) and \(a_1 = 1\).%
\item{}\(a_n = na_{n-1}\) with \(a_0 = 1\).%
\end{enumerate}
%
\par\smallskip%
\noindent\textbf{\blocktitlefont Solution}.\quad{}%
\begin{enumerate}[label=(\alph*)]
\item{}\(a_0 = 0\), \(a_1 = 1\), \(a_2 = 3\), \(a_3 = 6\) \(a_4 = 10\). The sequence was described by a closed formula. These are the triangular numbers. A recursive definition is: \(a_n = a_{n-1} + n\) with \(a_0 = 0\).%
\item{}This is a recursive definition. We continue \(a_2 = 2\), \(a_3 = 3\), \(a_4 = 4\), \(a_5 = 5\), and so on. A closed formula is \(a_n = n\).%
\item{}We have \(a_0 = 1\), \(a_1 = 1\), \(a_2 = 2\), \(a_3 = 6\), \(a_4 = 24\), \(a_5 = 120\), and so on. The closed formula is \(a_n = n!\).%
\end{enumerate}
%
\end{divisionsolution}%
\begin{divisionsolution}{2.1.4}{}{p:exercise:Jqz}%
Consider the sequence \((a_n)_{n \ge 1}\) that starts \(1, 3, 5, 7, 9, \ldots\) (i.e, the odd numbers in order).%
\begin{enumerate}[label=(\alph*)]
\item{}Give a recursive definition and closed formula for the sequence.%
\item{}Write out the sequence \((b_n)_{n \ge 2}\) of partial sums of \((a_n)\).  Write down the recursive definition for \((b_n)\) and guess at the closed formula.%
\end{enumerate}
%
\par\smallskip%
\noindent\textbf{\blocktitlefont Solution}.\quad{}%
\begin{enumerate}[label=(\alph*)]
\item{}The recursive definition is \(a_n = a_{n-1} + 2\) with \(a_1 = 1\).  A closed formula is \(a_n = 2n-1\).%
\item{}The sequence of partial sums is \(1, 4, 9, 16, 25, 36, \ldots\).  A recursive definition is (as always) \(b_n = b_{n-1} + a_n\) which in this case is \(b_n = b_{n-1} + 2n-1\).  It appears that the closed formula is \(b_n = n^2\)%
\end{enumerate}
%
\end{divisionsolution}%
\begin{divisionsolution}{2.1.5}{}{p:exercise:pxI}%
\label{g:notation:idp140965978208}%
The Fibonacci sequence is \(0, 1, 1, 2, 3, 5, 8, 13, \ldots\) (where \(F_0 = 0\)). \index{Fibonacci sequence}%
\begin{enumerate}[label=(\alph*)]
\item{}Write out the first few terms of the sequence of partial sums: \(0\), \(0+1\), \(0+1+1\),\textellipsis{}%
\item{}Guess a formula for the sequence of partial sums expressed in terms of a single Fibonacci number.  For example, you might say \(F_0 + F_1 + \cdots + F_n = 3F_{n-1}^2 + n\), although that is definitely not correct.%
\end{enumerate}
%
\par\smallskip%
\noindent\textbf{\blocktitlefont Solution}.\quad{}%
\begin{enumerate}[label=(\alph*)]
\item{}\(0, 1, 2, 4, 7, 12, 20, \ldots\).%
\item{}\(F_0 + F_1 + \cdots + F_n = F_{n+2} - 1\).%
\end{enumerate}
%
\end{divisionsolution}%
\begin{divisionsolution}{2.1.6}{}{p:exercise:VER}%
Consider the three sequences below. For each, find a recursive definition. How are these sequences related?%
\begin{enumerate}[label=(\alph*)]
\item{}\(2, 4, 6, 10, 16, 26, 42, \ldots\).%
\item{}\(5, 6, 11, 17, 28, 45, 73, \ldots\).%
\item{}\(0, 0 , 0 , 0 , 0 , 0 , 0 ,\ldots\).%
\end{enumerate}
%
\par\smallskip%
\noindent\textbf{\blocktitlefont Solution}.\quad{}The sequences all have the same recurrence relation: \(a_n = a_{n-1} + a_{n-2}\) (the same as the Fibonacci numbers). The only difference is the initial conditions.%
\end{divisionsolution}%
\begin{divisionsolution}{2.1.7}{}{p:exercise:BMa}%
Write out the first few terms of the sequence given by \(a_1 = 3\); \(a_n = 2a_{n-1} + 4\). Then find a recursive definition for the sequence \(10, 24, 52, 108, \ldots\).%
\par\smallskip%
\noindent\textbf{\blocktitlefont Solution}.\quad{}\(3, 10, 24, 52, 108,\ldots\). The recursive definition for \(10, 24, 52, \ldots\) is \(a_n = 2a_{n-1} + 4\) with \(a_1 = 10\).%
\end{divisionsolution}%
\begin{divisionsolution}{2.1.8}{}{p:exercise:hTj}%
Write out the first few terms of the sequence given by \(a_n = n^2 - 3n + 1\). Then find a closed formula for the sequence (starting with \(a_1\)) \(0, 2, 6, 12, 20, \ldots\).%
\par\smallskip%
\noindent\textbf{\blocktitlefont Solution}.\quad{}\(-1, -1, 1, 5, 11, 19,\ldots\) Thus the sequence \(0, 2, 6, 12, 20,\ldots\) has closed formula \(a_n = (n+1)^2 - 3(n+1) + 2\).%
\end{divisionsolution}%
\begin{divisionsolution}{2.1.9}{}{p:exercise:Oas}%
Show that \(a_n = 3\cdot 2^n + 7\cdot 5^n\) is a solution to the recurrence relation \(a_n = 7a_{n-1} - 10a_{n-2}\). What would the initial conditions need to be for this to be the closed formula for the sequence?%
\par\smallskip%
\noindent\textbf{\blocktitlefont Solution}.\quad{}This closed formula would have \(a_{n-1} = 3\cdot 2^{n-1} + 7 \cdot 5^{n-1}\) and \(a_{n-2} = 3\cdot 2^{n-2} + 7 \cdot 5^{n-2}\). Then we would have%
\begin{align*}
7a_{n-1} - 10a_{n-2} = \amp 7(3\cdot 2^{n-1} + 7 \cdot 5^{n-1}) - 10(3\cdot 2^{n-2} + 7 \cdot 5^{n-2})\\
= \amp 21\cdot 2^{n-1} + 49 \cdot 5^{n-1} - 30\cdot 2^{n-2} - 70 \cdot 5^{n-2})\\
= \amp 21\cdot 2^{n-1} + 49 \cdot 5^{n-1} - 15\cdot 2^{n-1} - 14 \cdot 5^{n-1})\\
= \amp 6\cdot 2^{n-1} + 35 \cdot 5^{n-1}\\
= \amp 3\cdot 2^{n} + 7 \cdot 5^{n} = a_n\text{.}
\end{align*}
So the closed formula agrees with the recurrence relation. The closed formula has initial terms \(a_0 = 10\) and \(a_1 = 41\).%
\end{divisionsolution}%
\begin{divisionsolution}{2.1.10}{}{p:exercise:uhB}%
Show that \(a_n = 2^n - 5^n\) is also a solution to the recurrence relation \(a_n = 7a_{n-1} - 10a_{n-2}\). What would the initial conditions need to be for this to be the closed formula for the sequence?%
\end{divisionsolution}%
\begin{divisionsolution}{2.1.11}{}{p:exercise:aoK}%
Find a closed formula for the sequence with recursive definition \(a_n = 2a_{n-1} - a_{n-2}\) with \(a_1 = 1\) and \(a_2 = 2\).%
\par\smallskip%
\noindent\textbf{\blocktitlefont Hint}.\quad{}You will want to write out the sequence, guess a closed formula, and then verify that you are correct.%
\par\smallskip%
\noindent\textbf{\blocktitlefont Solution}.\quad{}Write out the first few terms of the sequence: \(1, 2, 3, 4, 5, 6,\ldots\). This is surprising at first, but note that we could write \(2a_{n-1} - a_{n-2} = a_{n-1} + (a_{n-1} -a_{n-2})\), and \(a_{n-1} - a_{n-2}\) is just the difference between the terms. Initially, the difference between terms is 1, so each time we are just adding one. So we see that \(a_n = n\) is the closed formula.%
\end{divisionsolution}%
\begin{divisionsolution}{2.1.12}{}{p:exercise:GvT}%
Give two different recursive definitions for the sequence with closed formula \(a_n = 3 + 2n\).  Prove you are correct. At least one of the recursive definitions should makes use of two previous terms and no constants.%
\par\smallskip%
\noindent\textbf{\blocktitlefont Hint}.\quad{}Write out the sequence, guess a recursive definition, and verify that the closed formula is a solution to that recursive definition.%
\par\smallskip%
\noindent\textbf{\blocktitlefont Solution}.\quad{}The sequence we get is \(3, 5, 7, 9, \ldots\). One recursive definition for this is \(a_n = a_{n-1} + 2\) with \(a_0 = 3\). Another option would be to take \(a_n = 2a_{n-1} - a_{n-2}\) with \(a_0 = 3\) and \(a_1 = 5\).%
\end{divisionsolution}%
\begin{divisionsolution}{2.1.13}{}{p:exercise:mDc}%
Use summation (\(\sum\)) or product (\(\prod\)) notation to rewrite the following.%
\begin{multicols}{2}
\begin{enumerate}[label=(\alph*)]
\item{}\(2 + 4 + 6 + 8 + \cdots + 2n\).%
\item{}\(1 + 5 + 9 + 13 + \cdots + 425\).%
\item{}\(1 + \frac{1}{2} + \frac{1}{3} + \frac{1}{4} + \cdots + \frac{1}{50}\).%
\item{}\(2 \cdot 4 \cdot 6 \cdot \cdots \cdot 2n\).%
\item{}\((\frac{1}{2})(\frac{2}{3})(\frac{3}{4})\cdots(\frac{100}{101})\).%
\end{enumerate}
\end{multicols}
%
\par\smallskip%
\noindent\textbf{\blocktitlefont Solution}.\quad{}%
\begin{multicols}{2}
\begin{enumerate}[label=(\alph*)]
\item{}\(\d\sum_{k=1}^n 2k\).%
\item{}\(\d\sum_{k=1}^{107} (1 + 4(k-1))\).%
\item{}\(\d\sum_{k=1}^{50} \frac{1}{k}\).%
\item{}\(\d\prod_{k=1}^n 2k\).%
\item{}\(\d\prod_{k=1}^{100} \frac{k}{k+1}\).%
\end{enumerate}
\end{multicols}
%
\end{divisionsolution}%
\begin{divisionsolution}{2.1.14}{}{p:exercise:SKl}%
Expand the following sums and products. That is, write them out the long way.%
\begin{multicols}{2}
\begin{enumerate}[label=(\alph*)]
\item{}\(\d\sum_{k=1}^{100} (3+4k)\).%
\item{}\(\d\sum_{k=0}^n 2^k\).%
\item{}\(\d\sum_{k=2}^{50}\frac{1}{(k^2 - 1)}\).%
\item{}\(\d\prod_{k=2}^{100}\frac{k^2}{(k^2-1)}\).%
\item{}\(\d\prod_{k=0}^n (2+3k)\).%
\end{enumerate}
\end{multicols}
%
\par\smallskip%
\noindent\textbf{\blocktitlefont Solution}.\quad{}%
\begin{enumerate}[label=(\alph*)]
\item{}\(\d\sum_{k=1}^{100} (3+4k) = 7 + 11 + 15 + \cdots + 403\).%
\item{}\(\d\sum_{k=0}^n 2^k = 1 + 2 + 4 + 8 + \cdots + 2^n\).%
\item{}\(\d\sum_{k=2}^{50}\frac{1}{(k^2 - 1)} = 1 + \frac{1}{3} + \frac{1}{8} + \frac{1}{15} + \cdots + \frac{1}{2499}\).%
\item{}\(\d\prod_{k=2}^{100}\frac{k^2}{(k^2-1)} = \frac{4}{3}\cdot\frac{9}{8}\cdot\frac{16}{15}\cdots\frac{10000}{9999}\).%
\item{}\(\d\prod_{k=0}^n (2+3k) = (2)(5)(8)(11)(14)\cdots(2+3n)\).%
\end{enumerate}
%
\end{divisionsolution}%
\begin{divisionsolution}{2.1.15}{}{p:exercise:yRu}%
Suppose you draw \(n\) lines in the plane so that every pair of lines cross (no lines are parallel) and no three lines cross at the same point.  This will create some number of regions in the plane, including some unbounded regions.  Call the number of regions \(R_n\).  Find a recursive formula for the number of regions created by \(n\) lines, and justify why your recursion is correct.%
\par\smallskip%
\noindent\textbf{\blocktitlefont Hint}.\quad{}Try an example: when you draw the 4th line, it will cross three other lines, so will be divided into four segments, two of which are infinite.  Each segment will divide a previous region into two.%
\end{divisionsolution}%
\begin{divisionsolution}{2.1.16}{}{p:exercise:eYD}%
A \terminology{ternary} string is a sequence of 0's, 1's and 2's.  Just like a bit string, but with three symbols.%
\par
Let's call a ternary string \emph{good} provided it never contains a 2 followed immediately by a 0.  Let \(G_n\) be the number of good strings of length \(n\).  For example, \(G_1 = 3\), and \(G_2 = 8\) (since of the 9 ternary strings of length 2, only one is not good).%
\par
Find, with justification, a recursive formula for \(G_n\), and use it to compute \(G_5\).%
\par\smallskip%
\noindent\textbf{\blocktitlefont Hint}.\quad{}Consider three cases: the last digit is a 0, a 1, or a 2.  Two of these should be easy to count, but strings ending in 0 cannot be proceeded by a 2, so require a little more work.%
\par\smallskip%
\noindent\textbf{\blocktitlefont Solution}.\quad{}Consider the set of good strings of length \(n\).  These can be partitioned into three sets based on their last digit.  The number of good length \(n\) strings that end in a 1 is \(G_{n-1}\), since any good string of length \(n-1\) can have a 1 appended to get such a length \(n\) string.  Similarly, there are \(G_{n-1}\) good length \(n\) strings that end with a 2.%
\par
What about the good length \(n\) strings that end with 0?  These can be formed from any good length \(n-1\) string except those that end with a 2.  The number of good length \(n-1\) strings that end with a 2 is \(G_{n-2}\).  Thus there are \(G_{n-1} - G_{n-2}\) good length \(n\) strings that end with 0.%
\par
All together then there are \(G_n = 3G_{n-1} - G_{n-2}\) good strings of length \(n\).  We can then extend the sequence: \(3, 8, 21, 55, 144,\ldots\).%
\end{divisionsolution}%
\begin{divisionsolution}{2.1.17}{}{p:exercise:LfM}%
Consider bit strings with length \(l\) and weight \(k\) (so strings of \(l\) 0's and 1's, including \(k\) 1's). We know how to count the number of these for a fixed \(l\) and \(k\). Now, we will count the number of strings for which the \emph{sum} of the length and the weight is fixed. For example, let's count all the bit strings for which \(l+k = 11\).%
\begin{enumerate}[label=(\alph*)]
\item{}Find examples of these strings of different lengths. What is the longest string possible? What is the shortest?%
\item{}How many strings are there of each of these lengths. Use this to count the total number of strings (with sum 11).%
\item{}The other approach: Let \(n = l+k\) vary. How many strings have sum \(n = 1\)? How many have sum \(n = 2\)? And so on. Find and explain a recurrence relation for the sequence \((a_n)\) which gives the number of strings with sum \(n\).%
\item{}Describe what you have found above in terms of Pascal's Triangle. What pattern have you discovered?%
\end{enumerate}
%
\par\smallskip%
\noindent\textbf{\blocktitlefont Solution}.\quad{}%
\begin{enumerate}[label=(\alph*)]
\item{}The longest string is 00000000000. If we insert a 1, we need to replace two 0's, which shortens the string. The more 1's, the shorter the string. We could have 111110, or 101111, or others like this (there will be 6 of them) which are shortest possible. So all strings range in length between 6 and 11.%
\item{}There is 1 string with length 11. To get a string with length 10, we need one of the 10 spots to be a 1. Thus there are 10 of these. To get a string of length 9, we need exactly 2 of the 9 spots to be 1's, so there are \({9 \choose 2} = 36\) such strings. There will be \({8 \choose 3} = 56\) strings with length 8, another \({7 \choose 4} = 35\) of length 7, and finally \({6 \choose 5} = 6\) of length 6. All together we have%
\begin{equation*}
1 + 10 + 36 + 56 + 35 + 6 = 144 \text{ strings}
\end{equation*}
%
\item{}There is 1 string with sum \(n = 1\) (just the string 0). There are two strings with sum \(n = 2\) (namely 00, 1). With sum \(n = 3\) we get 3 strings (000, 01, 10), and 5 with sum \(n = 4\). In fact, we recognize these as the Fibonacci numbers, which satisfy the recurrence \(f_n = f_{n-1} + f_{n-2}\) with initial condition here \(f_1 = 1\) and \(f_2 = 2\). The reason this is correct is that to get all the strings with sum \(n\), we can start with a 0 and finish with any string with sum \(n-1\), or start with a 1 and finish with any string with sum \(n - 2\). If we carry out this recurrence we find \(f_{11} = 144\).%
\item{}In this one specific case we have that%
\begin{equation*}
{11\choose 0} + {10 \choose 1} + {9 \choose 2} + {8 \choose 3} + {7 \choose 4} + {6 \choose 5} = f_{11}
\end{equation*}
On Pascal's triangle, we recognize that this is the sum of a diagonal line of cells (moving right, then right-up each time). Indeed, every such sum of diagonals gives a Fibonacci number.%
\end{enumerate}
%
\end{divisionsolution}%
\begin{divisionsolution}{2.1.18}{}{p:exercise:rmV}%
When bees play chess, they use a hexagonal board like the one shown below. The queen bee can move one space at a time either directly to the right or angled up-right or down-right (but can never move leftwards). How many different paths can the queen take from the top left hexagon to the bottom right hexagon? Explain your answer, and this relates to the previous question. (As an example, there are three paths to get to the second hexagon on the bottom row.)%
\begin{sidebyside}{1}{0.15}{0.15}{0}%
\begin{sbspanel}{0.7}%
\resizebox{\linewidth}{!}{%
\def\r{1}
\newcommand{\hexagon}[3]{
  \def\x{-cos{30}*\r*#1+cos{30}*#2*\r*2}
  \def\y{-\r*#1-sin{30}*\r*#1}
  \draw[thick] (\x,\y) +(90:\r) -- +(30:\r) -- +(-30:\r) -- +(-90:\r) -- +(-150:\r) -- +(150:\r) -- cycle;
  \draw (\x,\y) node{#3};
}
\begin{tikzpicture}[scale=.75]
\hexagon{1}{0}{\tiny start};
\hexagon{2}{2}{3}
\foreach \i in {1,...,5} {
  \foreach \j in {1,2} {
    \hexagon{\j}{\i}{};
  }
}
\hexagon{2}{6}{\tiny stop};
\end{tikzpicture}
}%
\end{sbspanel}%
\end{sidebyside}%
\par\smallskip%
\noindent\textbf{\blocktitlefont Hint}.\quad{}Think recursively, like you did in Pascal's triangle.%
\textbf{\blocktitlefont Solution}.\quad{}We can code each path through the hive using 0's and 1's as follows: A 0 corresponds to a move up-right or down-right, a 1 corresponds to a move directly to the right. Notice that a 1 move can be replaced by two consecutive 0 moves. This describes a bijection between strings of 0's and 1's with total sum of length and weight equal to 11 and paths through the hive, so there are 144 paths.%
\end{divisionsolution}%
\begin{divisionsolution}{2.1.19}{}{p:exercise:Xue}%
Let \(t_n\) denote the number of ways to tile a \(2\times n\) checkerboard using \(1\times 2\) dominoes.  Write out the first few terms of the sequence \((t_n)_{n \ge 1}\) and then give a recursive definition.  Explain why your recursive formula is correct.%
\par\smallskip%
\noindent\textbf{\blocktitlefont Hint}.\quad{}There is only one way to tile a \(2 \times 1\) board, and two ways to tile a \(2\times 2\) board (you can orient the dominoes in two ways).  In general, consider the two ways the domino covering the top left corner could be oriented.%
\end{divisionsolution}%
\section*{2.2 Arithmetic and Geometric Sequences}
\addcontentsline{toc}{section}{2.2 Arithmetic and Geometric Sequences}
\sectionmark{2.2 Arithmetic and Geometric Sequences}
\subsection*{Exercises}
\addcontentsline{toc}{subsection}{Exercises}
\begin{divisionsolution}{2.2.1}{}{p:exercise:qKf}%
Consider the sequence \(5, 9, 13, 17, 21, \ldots\) with \(a_1 = 5\)%
\begin{enumerate}[label=(\alph*)]
\item{}Give a recursive definition for the sequence.%
\item{}Give a closed formula for the \(n\)th term of the sequence.%
\item{}Is \(2013\) a term in the sequence? Explain.%
\item{}How many terms does the sequence \(5, 9, 13, 17, 21, \ldots, 533\) have?%
\item{}Find the sum: \(5 + 9 + 13 + 17 + 21 + \cdots + 533\). Show your work.%
\item{}Use what you found above to find \(b_n\), the \(n^{th}\) term of \(1, 6, 15, 28, 45, \ldots\), where \(b_0 = 1\)%
\end{enumerate}
%
\par\smallskip%
\noindent\textbf{\blocktitlefont Solution}.\quad{}%
\begin{enumerate}[label=(\alph*)]
\item{}\(a_n = a_{n-1} + 4\) with \(a_1 = 5\).%
\item{}\(a_n = 5 + 4(n-1)\).%
\item{}Yes, since \(2013 = 5 + 4(503-1)\) (so \(a_{503} = 2013\)).%
\item{}133%
\item{}\(\frac{538\cdot 133}{2} = 35777\).%
\item{}\(b_n = 1 + \frac{(4n+6)n}{2}\).%
\end{enumerate}
%
\end{divisionsolution}%
\begin{divisionsolution}{2.2.2}{}{p:exercise:WRo}%
To thank your math professor for doing such an amazing job all semester, you decide to bake Oscar cookies. You know how to make 10 different types of cookies.%
\begin{enumerate}[label=(\alph*)]
\item{}If you want to give your professor 4 different types of cookies, how many different combinations of cookie type can you select? Explain your answer.%
\item{}To keep things interesting, you decide to make a different number of each type of cookie. If again you want to select 4 cookie types, how many ways can you select the cookie types and decide for which there will be the most, second most, etc. Explain your answer.%
\item{}You change your mind again. This time you decide you will make a total of 12 cookies. Each cookie could be any one of the 10 types of cookies you know how to bake (and it's okay if you leave some types out). How many choices do you have? Explain.%
\item{}You realize that the previous plan did not account for presentation. This time, you once again want to make 12 cookies, each one could be any one of the 10 types of cookies. However, now you plan to shape the cookies into the numerals 1, 2, ..., 12 (and probably arrange them to make a giant clock, but you haven't decided on that yet). How many choices do you have for which types of cookies to bake into which numerals? Explain.%
\item{}The only flaw with the last plan is that your professor might not get to sample all 10 different varieties of cookies. How many choices do you have for which types of cookies to make into which numerals, given that each type of cookie should be present at least once? Explain.%
\end{enumerate}
%
\par\smallskip%
\noindent\textbf{\blocktitlefont Answer 1}.\quad{}\(\mathop{\rm C}\nolimits\!\left(12,4\right)\)%
\par\smallskip%
\noindent\textbf{\blocktitlefont Answer 2}.\quad{}\(\mathop{\rm P}\nolimits\!\left(12,4\right)\)%
\par\smallskip%
\noindent\textbf{\blocktitlefont Answer 3}.\quad{}\(\mathop{\rm C}\nolimits\!\left(21,12\right)\)%
\par\smallskip%
\noindent\textbf{\blocktitlefont Answer 4}.\quad{}\(10^{12}\)%
\par\smallskip%
\noindent\textbf{\blocktitlefont Answer 5}.\quad{}\(6.1871\times 10^{9}\)%
\par\smallskip%
\noindent\textbf{\blocktitlefont Solution}.\quad{}%
\begin{enumerate}[label=(\alph*)]
\item{}\({10 \choose 4}\) combinations. You need to choose 4 of the 10 cookie types. Order doesn't matter.%
\item{}\(P(10, 4) = 10 \cdot 9 \cdot 8 \cdot 7\) ways. You are choosing and arranging 4 out of 10 cookies. Order matters now.%
\item{}\({21 \choose 12}\) choices. You must switch between cookie type 9 times as you make your 12 cookies. The cookies are the stars, the switches between cookie types are the bars.%
\item{}\(10^{12}\) choices. You have 10 choices for the ``1'' cookie, 10 choices for the ``2'' cookie, and so on.%
\item{}\(10^{12} - \left[{10 \choose 1}9^{12} - {10 \choose 2}8^{12} + \cdots - {10 \choose 10}0^{12} \right]\) choices. We must use PIE to remove all the ways in which one or more cookie type is not selected.%
\end{enumerate}
%
\end{divisionsolution}%
\begin{divisionsolution}{2.2.3}{}{p:exercise:CYx}%
Find the closed formula for each of the following sequences by relating them to a well known sequence. Assume the first term given is \(a_1\text{.}\)%
\begin{enumerate}[label=(\alph*)]
\item{}\(2, 5, 10, 17, 26, \ldots\)%
\item{}\(0, 2, 5, 9, 14, 20, \ldots\)%
\item{}\(8, 12, 17, 23, 30,\ldots\)%
\item{}\(1, 5, 23, 119, 719,\ldots\)%
\end{enumerate}
%
\par\smallskip%
\noindent\textbf{\blocktitlefont Solution}.\quad{}%
\begin{enumerate}[label=(\alph*)]
\item{}Note that if we subtract 1 from each term, we get the square numbers. Thus \(a_n = n^2 + 1\text{.}\)%
\item{}These look like the triangular numbers, only shifted by 1. We get: \(a_n = \frac{n(n+1)}{2} - 1\text{.}\)%
\item{}If you subtract 2 from each term, you get triangular numbers, only starting with 6 instead of 1. So we must shift vertically and horizontally. \(a_n = \frac{(n+2)(n+3)}{2} + 2\text{.}\)%
\item{}These seem to grow very quickly. Further, if we add 1 to each term, we find the factorials, although starting with 2 instead of 1. This gives, \(a_n = (n+1)! - 1\) (where \(n! = 1 \cdot 2 \cdot 3 \cdots n\)).%
\end{enumerate}
%
\end{divisionsolution}%
\begin{divisionsolution}{2.2.4}{}{p:exercise:jfG}%
Consider the sequence \((a_n)_{n \ge 0}\) which starts \(8, 14, 20, 26, \ldots\text{.}\)%
\begin{enumerate}[label=(\alph*)]
\item{}What is the next term in the sequence?%
\item{}Find a formula for the \(n\)th term of this sequence.%
\item{}Find the sum of the first 100 terms of the sequence: \(\sum_{k=0}^{99}a_k\text{.}\)%
\end{enumerate}
%
\par\smallskip%
\noindent\textbf{\blocktitlefont Answer 1}.\quad{}\(32\)%
\par\smallskip%
\noindent\textbf{\blocktitlefont Answer 2}.\quad{}\(30500\)%
\par\smallskip%
\noindent\textbf{\blocktitlefont Solution}.\quad{}%
\begin{enumerate}[label=(\alph*)]
\item{}\(32\text{,}\) which is \(26+6\text{.}\)%
\item{}The sequence is arithmetic, with \(a_0 = 8\) and constant difference 6, so \(a_n = 8 + 6n\text{.}\)%
\item{}\(30500\text{.}\) We want \(8 + 14 + \cdots + 602\text{.}\) Reverse and add to get 100 sums of 610, a total of 61000, which is twice the sum we are looking for.%
\end{enumerate}
%
\end{divisionsolution}%
\begin{divisionsolution}{2.2.5}{}{p:exercise:PmP}%
Consider the sum \(4 + 11 + 18 + 25 + \cdots + 249\text{.}\)%
\begin{enumerate}[label=(\alph*)]
\item{}How many terms (summands) are in the sum?%
\item{}Compute the sum using a technique discussed in this section.%
\end{enumerate}
%
\par\smallskip%
\noindent\textbf{\blocktitlefont Answer}.\quad{}\(36\)%
\par\smallskip%
\noindent\textbf{\blocktitlefont Solution}.\quad{}%
\begin{enumerate}[label=(\alph*)]
\item{}36.%
\item{}\(\frac{253 \cdot 36}{2} = 4554\text{.}\)%
\end{enumerate}
%
\end{divisionsolution}%
\begin{divisionsolution}{2.2.6}{}{p:exercise:vtY}%
Consider the sequence \(1, 7, 13, 19, \ldots, 6n + 7\text{.}\)%
\begin{enumerate}[label=(\alph*)]
\item{}How many terms are there in the sequence? Your answer will be in terms of \(n\text{.}\)%
\item{}What is the second-to-last term?%
\item{}Find the sum of all the terms in the sequence, in terms of \(n\text{.}\)%
\end{enumerate}
%
\par\smallskip%
\noindent\textbf{\blocktitlefont Answer 1}.\quad{}\(n+2\)%
\par\smallskip%
\noindent\textbf{\blocktitlefont Answer 2}.\quad{}\(6n+1\)%
\par\smallskip%
\noindent\textbf{\blocktitlefont Answer 3}.\quad{}\(\frac{\left(6n+8\right)\!\left(n+2\right)}{2}\)%
\par\smallskip%
\noindent\textbf{\blocktitlefont Solution}.\quad{}%
\begin{enumerate}[label=(\alph*)]
\item{}\(n+2\) terms, since to get 1 using the formula \(6n+7\) we must use \(n=-1\text{.}\) Thus we have \(n\) terms, plus two, when \(n=0\) and \(n=-1\text{.}\)%
\item{}\(6n+1\text{,}\) which is 6 less than \(6n+7\) (or plug in \(n-1\) for \(n\)).%
\item{}\(\frac{(6n+8)(n+2)}{2}\text{.}\) Reverse and add. Each sum gives the constant \(6n+8\) and there are \(n+2\) terms.%
\end{enumerate}
%
\end{divisionsolution}%
\begin{divisionsolution}{2.2.7}{}{p:exercise:bBh}%
Find \(5 + 7 + 9 + 11+ \cdots + 521\) using a technique from this section.%
\par\smallskip%
\noindent\textbf{\blocktitlefont Answer}.\quad{}\(68117\)%
\par\smallskip%
\noindent\textbf{\blocktitlefont Solution 1}.\quad{}\(68117\text{.}\)%
\par\smallskip%
\noindent\textbf{\blocktitlefont Solution 2}.\quad{}\(68117\text{.}\) If we take \(a_0 = 5\text{,}\) the terms of the sum are an arithmetic sequence with closed formula \(a_n = 5+2n\text{.}\) Then \(521 = a_{258}\text{,}\) for a total of 259 terms in the sum. Reverse and add to get 259 identical 526 terms, which is twice the total we seek. \(526\cdot 259 = 68117\)%
\end{divisionsolution}%
\begin{divisionsolution}{2.2.8}{}{p:exercise:HIq}%
Find \(5 + 15 + 45 + \cdots + 5\cdot 3^{20}\text{.}\)%
\par\smallskip%
\noindent\textbf{\blocktitlefont Answer}.\quad{}\(\frac{5-5\cdot 3^{21}}{-2}\)%
\par\smallskip%
\noindent\textbf{\blocktitlefont Solution}.\quad{}\(\frac{5\cdot 3^{21}-5}{2}\text{.}\) Let the sum be \(S\text{,}\) and compute \(S - 3S = -2S\text{,}\) which causes terms except \(5\) and \(-5\cdot 3^{21}\) to cancel. Then solve for \(S\text{.}\)%
\end{divisionsolution}%
\begin{divisionsolution}{2.2.9}{}{p:exercise:nPz}%
Find \(1 - \frac{2}{3} + \frac{4}{9} - \cdots + \frac{2^{30}}{3^{30}}\text{.}\)%
\par\smallskip%
\noindent\textbf{\blocktitlefont Answer}.\quad{}\(\frac{1+\frac{2^{31}}{3^{31}}}{\frac{5}{3}}\)%
\par\smallskip%
\noindent\textbf{\blocktitlefont Solution}.\quad{}\(\frac{1 + \frac{2^{31}}{3^{31}}}{5/3}\text{.}\) This time compute \(S + \frac{2}{3}S\text{.}\)%
\end{divisionsolution}%
\begin{divisionsolution}{2.2.10}{}{p:exercise:TWI}%
Is there a pair of integers \((a,b)\) such that \(a,
x_1, y_1, b\) is part of an arithmetic sequences and \(a,
x_2, y_2, b\) is part of a geometric sequence with \(x_1, x_2, y_1, y_2\) all integers?%
\par\smallskip%
\noindent\textbf{\blocktitlefont Solution}.\quad{}A somewhat trivial class of solutions are when \(a = b\).  For example, \(1,1,1,1,\ldots\) is both an arithmetic sequence (\(d = 0\)) and a geometric sequence (\(r = 1\)).%
\par
Are there any integers \((a,b)\) with \(a\ne b\) that satisfy the requirements?  We would need \(b = a + 3d\) and \(b = a\cdot r^3\).  That would mean that \(b - a = 3d\) and \(\frac{b}{a} = r^3\).  We also need \(d\) and \(r\) to be integers.  This suggests \(a = 3\) and \(b = 81\), so that \(d = 26\) and \(r = 3\), giving the sequences \(3, 29, 54, 81\), and \(3, 9, 27, 81\).%
\end{divisionsolution}%
\begin{divisionsolution}{2.2.11}{}{p:exercise:AdR}%
Consider the sequence \(2, 7, 15, 26, 40, 57, \ldots\) (with \(a_0 = 2\)). By looking at the differences between terms, express the sequence as a sequence of partial sums. Then find a closed formula for the sequence by computing the \(n\)th partial sum.%
\par\smallskip%
\noindent\textbf{\blocktitlefont Solution}.\quad{}We have \(2 = 2\), \(7 = 2+5\), \(15 = 2 + 5 + 8\), \(26 = 2+5+8+11\), and so on. The terms in the sums are given by the arithmetic sequence \(b_n = 2+3n\). In other words, \(a_n = \sum_{k=0}^n (2+3k)\). To find the closed formula, we reverse and add. We get \(a_n = \frac{(4+3n)(n+1)}{2}\) (we have \(n+1\) there because there are \(n+1\) terms in the sum for \(a_n\)).%
\end{divisionsolution}%
\begin{divisionsolution}{2.2.12}{}{p:exercise:gla}%
Starting with any rectangle, we can create a new, larger rectangle by attaching a square to the longer side. For example, if we start with a \(2\times 5\) rectangle, we would glue on a \(5\times 5\) square, forming a \(5 \times 7\) rectangle:%
\begin{sidebyside}{1}{0.2}{0.2}{0}%
\begin{sbspanel}{0.6}%
\resizebox{\linewidth}{!}{%
 \begin{tikzpicture}[scale=.4]
	\draw[thick] (0,0) rectangle (2,5);
	\draw[thick] (2.2,0) rectangle (7.2,5);
	\draw (0,2.5) node[left]{ 5} (1,0) node[above]{ 2} (4.5,0) node[above]{ 5};
	\draw (9,2.5) node{ \(\rightsquigarrow\)};
	\draw[thick] (11,0) rectangle (18,5);
	\draw[dotted] (13,0) -- (13,5);
	\draw (11,2.5) node[left]{ 5}  (14.5,0) node[above]{ 7};
\end{tikzpicture}
}%
\end{sbspanel}%
\end{sidebyside}%
\par
The next rectangle would be formed by attaching a \(7 \times 7\) square to the top or bottom of the \(5\times 7\) rectangle. %
\begin{enumerate}[label=(\alph*)]
\item{}Create a sequence of rectangles using this rule starting with a \(1\times 2\) rectangle. Then write out the sequence of \emph{perimeters} for the rectangles (the first term of the sequence would be 6, since the perimeter of a \(1\times 2\) rectangle is 6 - the next term would be 10).%
\item{}Repeat the above part this time starting with a \(1 \times 3\) rectangle.%
\item{}Find recursive formulas for each of the sequences of perimeters you found in parts (a) and (b). Don't forget to give the initial conditions as well.%
\item{}Are the sequences arithmetic? Geometric? If not, are they \emph{close} to being either of these (i.e., are the differences or ratios \emph{almost} constant)? Explain.%
\end{enumerate}
%
\par\smallskip%
\noindent\textbf{\blocktitlefont Solution}.\quad{}%
\begin{enumerate}[label=(\alph*)]
\item{}The rectangles are \(1 \times 2\), \(2 \times 3\), \(3 \times 5\), \(5 \times 8\), \(8 \times 13\), and so on. The sequence of perimeters is%
\begin{equation*}
6, 10, 16, 26, 42, \ldots
\end{equation*}
%
\item{}For the sequence from (a), the recursive formula is              \(a_1 = 6\),              \(a_2 = 10\), and \(a_n = a_{n-1} + a_{n-2}\). For the sequence from (b), the recursive formula is \(a_1 = 8\),              \(a_2 = 14\), and \(a_n = a_{n-1} + a_{n-2}\). Notice that both sequence have the same rule for getting terms from the previous ones, it is just the initial conditions that are different.%
\item{}The sequences are not arithmetic because the differences between terms is not constant. Similarly the ratio between terms is not constant, so the sequences are not geometric either. However, look at the ratio between terms:              \(10/6 \approx 1.66\),              \(16/10 = 1.6\),              \(26/16 \approx 1.625\),              \(42/26 \approx 1.61\),              \textellipsis{}. In fact, the ratio between terms of the second sequence also floats around this same number. That number (and in fact, the limit of the ratios of either sequence as the terms increase) is \(\frac{1 + \sqrt{5}}{2} \approx 1.618\), also known as the golden ratio.%
\end{enumerate}
%
\end{divisionsolution}%
\begin{divisionsolution}{2.2.13}{}{p:exercise:Msj}%
If you have enough toothpicks, you can make a large triangular grid. Below, are the triangular grids of size 1 and of size 2. The size 1 grid requires 3 toothpicks, the size 2 grid requires 9 toothpicks.%
\begin{sidebyside}{2}{0.2}{0.1}{0.4}%
\begin{sbspanel}{0.1}[bottom]%
\resizebox{\linewidth}{!}{%
            \begin{tikzpicture}[scale=.8]
\draw[line width=1.8pt] (90:1) -- (-30:1) -- (210:1) -- (90:1);
\fill[color=white] (90:1) circle (3pt);
\fill[color=white] (-30:1) circle (3pt);
\fill[color=white] (210:1) circle (3pt);
\end{tikzpicture}
}%
\end{sbspanel}%
\begin{sbspanel}{0.2}[bottom]%
\resizebox{\linewidth}{!}{%
            \begin{tikzpicture}[scale=.8]
\draw[line width = 1.8pt] (-90:1) -- (210:2) -- (150:1) -- (-90:1) -- (-30:2) -- (30:1) -- (150:1) -- (90:2) -- (30:1) -- (-90:1);
\fill[color=white] (-90:1) circle (3pt);
\fill[color=white] (-30:2) circle (3pt);
\fill[color=white] (210:2) circle (3pt);
\fill[color=white] (90:2) circle (3pt);
\fill[color=white] (30:1) circle (3pt);
\fill[color=white] (150:1) circle (3pt);
\end{tikzpicture}
}%
\end{sbspanel}%
\end{sidebyside}%
\par
%
\begin{enumerate}[label=(\alph*)]
\item{}Let \(t_n\) be the number of toothpicks required to make a size \(n\) triangular grid. Write out the first 5 terms of the sequence \(t_1, t_2, \ldots\). %
\item{}Find a recursive definition for the sequence. Explain why you are correct. %
\item{}Is the sequence arithmetic or geometric? If not, is it the sequence of partial sums of an arithmetic or geometric sequence? Explain why your answer is correct. %
\item{}Use your results from part (c) to find a closed formula for the sequence. Show your work. %
\end{enumerate}
%
\end{divisionsolution}%
\begin{divisionsolution}{2.2.14}{}{p:exercise:szs}%
If you were to shade in a \(n\times n\) square on graph paper, you could do it the boring way (with sides parallel to the edge of the paper) or the interesting way, as illustrated below:%
\begin{sidebyside}{4}{0.025}{0.025}{0.05}%
\begin{sbspanel}{0.05}[center]%
\resizebox{\linewidth}{!}{%
\begin{tikzpicture}[scale=0.4]
\draw (0,0) rectangle (1,1);
\end{tikzpicture}
}%
\end{sbspanel}%
\begin{sbspanel}{0.15}[center]%
\resizebox{\linewidth}{!}{%
\begin{tikzpicture}[scale=0.4]
\draw (-1,0) rectangle (2,1) (0,-1) rectangle (1,2);
\end{tikzpicture}
}%
\end{sbspanel}%
\begin{sbspanel}{0.25}[center]%
\resizebox{\linewidth}{!}{%
\begin{tikzpicture}[scale=0.4]
\draw (-2,0) rectangle (3,1) (0,-2) rectangle (1,3) (-1,-1) rectangle (2,2);
\end{tikzpicture}
}%
\end{sbspanel}%
\begin{sbspanel}{0.35}[center]%
\resizebox{\linewidth}{!}{%
\begin{tikzpicture}[scale=0.4]
\draw (-3,0) rectangle (4,1) (-2,-1) rectangle (3,2) (-1,-2) rectangle (2,3) (0,-3) rectangle (1,4);
\end{tikzpicture}
}%
\end{sbspanel}%
\end{sidebyside}%
\par
The interesting thing here, is that a \(3\times 3\) square now has area 13. Our goal is the find a formula for the area of a \(n \times n\) (diagonal) square.%
\begin{enumerate}[label=(\alph*)]
\item{}Write out the first few terms of the sequence of areas (assume \(a_1 = 1\), \(a_2 = 5\), etc). Is the sequence arithmetic or geometric? If not, is it the sequence of partial sums of an arithmetic or geometric sequence? Explain why your answer is correct, referring to the diagonal squares.%
\item{}Use your results from part (a) to find a closed formula for the sequence. Show your work. Note, while there are lots of ways to find a closed formula here, you should use partial sums specifically.%
\item{}Find the closed formula in as many other interesting ways as you can.%
\end{enumerate}
%
\par\smallskip%
\noindent\textbf{\blocktitlefont Solution}.\quad{}%
\begin{enumerate}[label=(\alph*)]
\item{}The sequence is \(1, 5, 13, 25, 41, 61, \ldots\). This is not arithmetic (since \(5-1 = 4\) but \(13-5 = 8\) ). It is also not geometric (since \(5/1 = 5\) but \(13/5 = 2.6\) ). To recognize whether the sequence is a sequence of partial sums, we look at the differences. The sequence of differences is \(4, 8, 12, 16, 20, \ldots\). This \textbraceleft{}\textbackslash{}em is\textbraceright{} an arithmetic sequence. So we see that%
\begin{equation*}
a_1 = 1
\end{equation*}
%
\begin{equation*}
a_2 = 1+4
\end{equation*}
%
\begin{equation*}
a_3 = 1+4+8
\end{equation*}
%
\begin{equation*}
a_4 = 1+4+8+12
\end{equation*}
%
\begin{equation*}
a_n = 1 + \sum_{k = 1}^n 4(k-1)
\end{equation*}
%
\item{}We have \(a_n = 1 + 4 + 8 + \cdots + 4(n-1)\). If we reverse these and add corresponding terms (not including the 1) we get%
\begin{equation*}
2a_n = 2 + (4n) + (4n) + (4n) + \cdots + (4n)
\end{equation*}
On the right hand side of the equation we have the sum of \(n-1\) copies of \((4n)\) so we get%
\begin{equation*}
2a_n = 2 + (n-1)(4n)
\end{equation*}
or%
\begin{equation*}
a_n = 2n^2 - 2n + 1
\end{equation*}
%
\item{}Look along the diagonals. The \(n\) th figure will have \(n\) diagonals of \(n\) squares, and \(n-1\) diagonals of \(n-1\) squares. Thus all together, the number of squares is \(n^2 + (n-1)^2\).%
\par
Alternatively, if you count the number of squares in each row starting at the top, you will have \(1, 3, 5, \ldots, 2n-3, 2n-1, 2n-3, \ldots, 3, 1\). So we can think of the total number of squares as being the sum of the first \(n\) odd numbers (counting up) plus the sum of the first \(n-1\) odd numbers (counting down). But the sum of the first \(n\) odd numbers is \(n^2\), so again we find the total number of squares to be \(n^2 + (n-1)^2\).%
\par
Another: Enclose the figure in a larger square with dimensions \(2n-1 \times 2n-1\). Then look at what part of that square you don't want. You have 4 triangles you must remove. The \(n\) th figure requires removing a triangle with base \(n-1\) on each corner, so we must remove \(4T_{n-1}\), where \(T_n = \frac{n(n+1)}{2}\) is the \(n\) th triangular number. Thus the total number of squares in the figure is%
\begin{equation*}
(2n-1)^2 - 4\frac{(n-1)n}{2}\text{.}
\end{equation*}
%
\end{enumerate}
%
\end{divisionsolution}%
\begin{divisionsolution}{2.2.15}{}{p:exercise:YGB}%
Here is a surprising use of sequences to answer a counting question:  How many license plates consist of 6 symbols, using only the three numerals 1, 2, and 3 and the four letters a, b, c, and d, so that no numeral appears after any letter?  For example, ``31ddac'' and ``12321'' are acceptable license plates, but ``13ba2c'' is not.%
\begin{enumerate}[label=(\alph*)]
\item{}First answer this question by considering different cases: how many of the license plates contain no numerals?  How many contain one numeral, etc.%
\item{}Now use the techniques of this section to show why the answer is \(4^7 - 3^7\).%
\end{enumerate}
%
\end{divisionsolution}%
\section*{2.3 Polynomial Fitting}
\addcontentsline{toc}{section}{2.3 Polynomial Fitting}
\sectionmark{2.3 Polynomial Fitting}
\subsection*{Exercises}
\addcontentsline{toc}{subsection}{Exercises}
\begin{divisionsolution}{2.3.1}{}{p:exercise:kGY}%
Find \(x\) and \(y\) such that \(27,
x, y, 1\) is part of an arithmetic sequence.%
\par
Then find \(x\) and \(y\) so that the sequence is part of a geometric sequence.%
\par
(Warning: \(x\) and \(y\) might not be integers.)%
\par\smallskip%
\noindent\textbf{\blocktitlefont Solution 1}.\quad{}For arithmetic: \(x = 55/3\text{,}\) \(y = 29/3\text{.}\) For geometric: \(x = 9\) and \(y = 3\text{.}\)%
\par\smallskip%
\noindent\textbf{\blocktitlefont Solution 2}.\quad{}For arithmetic: \(x = 55/3\text{,}\) \(y = 29/3\text{.}\) For geometric: \(x = 9\) and \(y = 3\text{.}\) For the arithmetic sequence, we know \(27 + d = x\text{,}\) \(x + d = y\text{,}\) and \(y + d = 1\text{.}\) In other words, \(27 + 3d = 1\) so \(d = -26/3\text{.}\) Similarly we can find the common ratio for the geometric sequence by solving \(27\cdot r^3 = 1\) for \(r\text{.}\)%
\end{divisionsolution}%
\begin{divisionsolution}{2.3.2}{}{p:exercise:QOh}%
Find \(x\) and \(y\) such that \(5,
x, y, 32\) is part of an arithmetic sequence.%
\par
Then find \(x\) and \(y\) so that the sequence is part of a geometric sequence.%
\par
(Warning: \(x\) and \(y\) might not be integers.)%
\par\smallskip%
\noindent\textbf{\blocktitlefont Solution}.\quad{}For arithmetic: \(x = 14\text{,}\) \(y = 23\text{.}\) For geometric: \(x = 5*(32/5)^(1/3)\) and \(y = 5*(32/5)^(2/3)\text{.}\)%
\end{divisionsolution}%
\begin{divisionsolution}{2.3.3}{}{x:exercise:ex-cubic-seq}%
Use polynomial fitting to find the formula for the \(n\)th term of the sequence \((a_n)_{n \ge 0}\) which starts,%
\par
%
\begin{equation*}
0, 2, 6, 12, 20, \ldots 
\text{.}
\end{equation*}
%
\par\smallskip%
\noindent\textbf{\blocktitlefont Solution}.\quad{}\(a_n = n^2 + n\text{.}\) Here we know that we are looking for a quadratic because the second differences are constant. So \(a_n = an^2 + bn + c\text{.}\) Since \(a_0 = 0\text{,}\) we know \(c= 0\text{.}\) So just solve the system%
\par
%
\begin{equation*}
\begin{aligned}
2 \amp = a + b\\
6 \amp = 4a + 2b
\end{aligned}
\end{equation*}
%
\end{divisionsolution}%
\begin{divisionsolution}{2.3.4}{}{p:exercise:dcz}%
Use polynomial fitting to find the formula for the \(n\)th term of the sequence \((a_n)_{n \ge 0}\) which starts,%
\par
%
\begin{equation*}
1, 2, 4, 8, 15, 26 \ldots 
\text{.}
\end{equation*}
%
\par\smallskip%
\noindent\textbf{\blocktitlefont Solution}.\quad{}\(a_n = \frac{1}{6} (n^3 + 5n + 6)\text{.}\)%
\end{divisionsolution}%
\begin{divisionsolution}{2.3.5}{}{p:exercise:JjI}%
Make up sequences that have%
\begin{enumerate}[label=(\alph*)]
\item{}3, 3, 3, 3, \textellipsis{} as its second differences.%
\item{}1, 2, 3, 4, 5, \textellipsis{} as its third differences.%
\item{}1, 2, 4, 8, 16, \textellipsis{} as its 100th differences.%
\end{enumerate}
%
\end{divisionsolution}%
\begin{divisionsolution}{2.3.6}{}{p:exercise:pqR}%
Consider the sequence \(1, 3, 7, 13, 21, \ldots\). Explain how you know the closed formula for the sequence will be quadratic. Then ``guess'' the correct formula by comparing this sequence to the squares \(1, 4, 9, 16, \ldots\) (do not use polynomial fitting).%
\par\smallskip%
\noindent\textbf{\blocktitlefont Solution 1}.\quad{}\(a_n = n^2 - n + 1\).%
\par\smallskip%
\noindent\textbf{\blocktitlefont Solution 2}.\quad{}The first differences are \(2, 4, 6, 8, \ldots\), and the second differences are \(2, 2, 2, \ldots\). Thus the original sequence is \(\Delta^2\)-constant, so can be fit to a quadratic.%
\par
Call the original sequence \(a_n\). Consider \(a_n - n^2\). This gives \(0, -1, -2, -3, \ldots\). \emph{That} sequence has closed formula \(1-n\) (starting at \(n = 1\)) so we have \(a_n - n^2 = 1-n\) or equivalently \(a_n = n^2 - n + 1\).%
\end{divisionsolution}%
\begin{divisionsolution}{2.3.7}{}{p:exercise:Vya}%
Use a similar technique as in the previous exercise to find a closed formula for the sequence \(2, 11, 34, 77, 146, 247,\ldots\).%
\par\smallskip%
\noindent\textbf{\blocktitlefont Solution 1}.\quad{}\(a_n = n^3 + n^2 - n + 1\)%
\par\smallskip%
\noindent\textbf{\blocktitlefont Solution 2}.\quad{}This is a \(\Delta^3\)-constant sequence. If we subtract off \(n^3\), we are left with \(1, 3, 7, 13, 21, \ldots\), the sequence from the previous question. Thus here the closed formula is \(n^3 + n^2 - n + 1\).%
\end{divisionsolution}%
\begin{divisionsolution}{2.3.8}{}{x:exercise:ex-quad-diff}%
Use polynomial fitting to find the formula for the \(n\)th term of the sequence \((a_n)_{n \ge 0}\) which starts,%
\par
%
\begin{equation*}
2,5,11,21,36, \ldots 
\text{.}
\end{equation*}
%
\par\smallskip%
\noindent\textbf{\blocktitlefont Solution}.\quad{}\(a_n = \frac{1}{6} (n^3 + 6n^2 + 11n + 12)\text{.}\)%
\end{divisionsolution}%
\begin{divisionsolution}{2.3.9}{}{p:exercise:hMs}%
Generalize Exercise~2.3.8: Find a closed formula for the sequence of differences of \(a_n = an^2 + bn + c\). That is, prove that every quadratic sequence has arithmetic differences.%
\par\smallskip%
\noindent\textbf{\blocktitlefont Solution}.\quad{}\(a_{n-1} = a(n-1)^2 + b(n-1) + c = an^2 - 2an + a + bn - b + c\). Therefore \(a_n - a_{n-1} = 2an - a + b\), which is arithmetic. Notice that this is not quite the derivative of \(a_n\), which would be \(2an + b\), but it is close.%
\end{divisionsolution}%
\begin{divisionsolution}{2.3.10}{}{p:exercise:NTB}%
Can you use polynomial fitting to find the formula for the \(n\)th term of the sequence 4, 7, 11, 18, 29, 47, \textellipsis{}? Explain why or why not.%
\par\smallskip%
\noindent\textbf{\blocktitlefont Solution}.\quad{}No. The sequence of differences is the same as the original sequence so no differences will be constant.%
\end{divisionsolution}%
\begin{divisionsolution}{2.3.11}{}{p:exercise:uaK}%
Will the \(n\)th sequence of differences of \(2, 6, 18, 54, 162, \ldots\) ever be constant? Explain.%
\par\smallskip%
\noindent\textbf{\blocktitlefont Solution}.\quad{}No. The sequence is geometric, and in fact has closed formula \(2\cdot 3^n\). This is an exponential function, which is not equal to any polynomial of any degree. If the \(n\)th sequence of differences was constant, then the closed formula for the original sequence would be a degree \(n\) polynomial.%
\end{divisionsolution}%
\begin{divisionsolution}{2.3.12}{}{p:exercise:ahT}%
In their down time, ghost pirates enjoy stacking cannonballs in triangular based pyramids (aka, tetrahedrons), like those pictured here:%
\begin{sidebyside}{3}{0.08}{0.08}{0.16}%
\begin{sbspanel}{0.08}[bottom]%
\resizebox{\linewidth}{!}{%
\begin{tikzpicture}
  \draw (0,0) circle (10pt);
  \draw[very thin, color=brown!15] (0,0) circle (10pt);
  \shade[shading=axis,bottom color=black!70!brown, top color=black!15, shading angle=-40] (0,0) circle (10pt);
\end{tikzpicture}
}%
\end{sbspanel}%
\begin{sbspanel}{0.18}[bottom]%
\resizebox{\linewidth}{!}{%
\begin{tikzpicture}
  \foreach \pos in {(-.32, .2), (.32,.2), (0,.55), (0,0)}{
  \draw[very thin, color=brown!15] \pos circle (10pt);
  \shade[shading=axis,bottom color=black!70!brown, top color=black!15, shading angle=-40] \pos circle (10pt);
  }
\end{tikzpicture}
}%
\end{sbspanel}%
\begin{sbspanel}{0.26}[bottom]%
\resizebox{\linewidth}{!}{%
\begin{tikzpicture}
  \foreach \pos in {(-.64, .4), (.64,.4), (-.37, .75), (.37,.75), (0, 1.2), (-.32, .2), (.32,.2), (0,.6), (0,0)}{
  \draw[very thin, color=brown!15] \pos circle (10pt);
  \shade[shading=axis,bottom color=black!70!brown, top color=black!15, shading angle=-40] \pos circle (10pt);
  }
\end{tikzpicture}
}%
\end{sbspanel}%
\end{sidebyside}%
\par
Note, these are solid tetrahedrons, so there will be some cannonballs obscured from view (the picture on the right has one cannonball in the back not shown in the picture, for example)%
\par
The pirates wonder how many cannonballs would be required to build a pyramid 15 layers high (thus breaking the world cannonball stacking record). Can you help?%
\begin{enumerate}[label=(\alph*)]
\item{}Let \(P(n)\) denote the number of cannonballs needed to create a pyramid \(n\) layers high. So \(P(1) = 1\), \(P(2) = 4\), and so on. Calculate \(P(3)\), \(P(4)\) and \(P(5)\). %
\item{}Use polynomial fitting to find a closed formula for \(P(n)\). Show your work. %
\item{}Answer the pirate's question: how many cannonballs do they need to make a pyramid 15 layers high? %
\item{}Bonus: Locate this sequence in Pascal's triangle. Why does that make sense?%
\end{enumerate}
%
\end{divisionsolution}%
\section*{2.4 Solving Recurrence Relations}
\addcontentsline{toc}{section}{2.4 Solving Recurrence Relations}
\sectionmark{2.4 Solving Recurrence Relations}
\subsection*{Exercises}
\addcontentsline{toc}{subsection}{Exercises}
\begin{divisionsolution}{2.4.1}{}{p:exercise:maM}%
Find the next two terms in \((a_n)_{n\ge 0}\) beginning \(3, 5, 11, 21, 43, 85\ldots\). Then give a recursive definition for the sequence. Finally, use the characteristic root technique to find a closed formula for the sequence.%
\par\smallskip%
\noindent\textbf{\blocktitlefont Solution}.\quad{}171 and 341. \(a_n = a_{n-1} + 2a_{n-2}\) with \(a_0 = 3\) and \(a_1 = 5\). Closed formula: \(a_n = \frac{8}{3}2^n + \frac{1}{3}(-1)^n\). To find this solve the characteristic equation, \(x^2 - x - 2 = 0\), to get characteristic roots \(x = 2\) and \(x=-1\). Then solve the system%
\begin{align*}
3 \amp = a + b\\
5 \amp = 2a - b
\end{align*}
%
\end{divisionsolution}%
\begin{divisionsolution}{2.4.2}{}{p:exercise:ShV}%
Consider the sequences \(2, 5, 12, 29, 70, 169, 408,\ldots\) (with \(a_0 = 2\)).%
\begin{enumerate}[label=(\alph*)]
\item{}Describe the rate of growth of this sequence.%
\item{}Find a recursive definition for the sequence.%
\item{}Find a closed formula for the sequence.%
\item{}If you look at the sequence of differences between terms, and then the sequence of second differences, the sequence of third differences, and so on, will you ever get a constant sequence? Explain how you know.%
\end{enumerate}
%
\end{divisionsolution}%
\begin{divisionsolution}{2.4.3}{}{p:exercise:ype}%
Use polynomial fitting to find the formula for the \(n\)th term of the sequence \((a_n)_{n \ge 0}\) which starts,%
\par
%
\begin{equation*}
3, 6, 12, 22, 37, 58, \ldots 
\text{.}
\end{equation*}
%
\par\smallskip%
\noindent\textbf{\blocktitlefont Solution}.\quad{}\(a_n = \frac{1}{6} (n^3 + 6n^2 + 11n + 18)\text{.}\)%
\end{divisionsolution}%
\begin{divisionsolution}{2.4.4}{}{p:exercise:ewn}%
Show that \(4^n\) is a solution to the recurrence relation \(a_n = 3a_{n-1} + 4a_{n-2}\).%
\par\smallskip%
\noindent\textbf{\blocktitlefont Solution}.\quad{}We claim \(a_n = 4^n\) works. Plug it in: \(4^n = 3(4^{n-1}) + 4(4^{n-2})\). This works - just simplify the right-hand side.%
\end{divisionsolution}%
\begin{divisionsolution}{2.4.5}{}{p:exercise:KDw}%
Suppose \(a_n = n^2 + 3n + 4\text{.}\) Find a closed formula for the sequence of differences by computing \(a_n - a_{n-1}\text{.}\)%
\par\smallskip%
\noindent\textbf{\blocktitlefont Solution}.\quad{}\(a_{n-1} = (n-1)^2 + 3(n-1) + 4 = n^2 + n + 2\text{.}\) Thus \(a_n - a_{n-1} = 2n+2\text{.}\) Note that this is linear (arithmetic). We can check that we are correct. The sequence \(a_n\) is \(4, 8, 14, 22, 32, \ldots\) and the sequence of differences is thus \(4, 6, 8, 10,\ldots\) which agrees with \(2n+2\) (if we start at \(n = 1\)).%
\end{divisionsolution}%
\begin{divisionsolution}{2.4.6}{}{p:exercise:qKF}%
Solve the recurrence relation \(a_n = a_{n-1} + 2^n\) with \(a_0 = 5\text{.}\)%
\par\smallskip%
\noindent\textbf{\blocktitlefont Hint}.\quad{}Use telescoping or iteration.%
\par\smallskip%
\noindent\textbf{\blocktitlefont Solution}.\quad{}\(a_n = 3 + 2^{n+1}\text{.}\) We should use telescoping or iteration here. For example, telescoping gives%
\par
%
\begin{equation*}
\begin{aligned}
a_1 - a_0 \amp = 2^1\\
a_2 - a_1 \amp = 2^2\\
a_3 - a_2 \amp = 2^3\\
\vdots\amp \vdots\\
a_n - a_{n-1} \amp = 2^n
\end{aligned}
\end{equation*}
%
\par
which sums to \(a_n - a_0 = 2^{n+1} - 2\) (using the multiply-shift-subtract technique from Section~3.2 for the right-hand side). Substituting \(a_0 = 5\) and solving for \(a_n\) completes the solution.%
\end{divisionsolution}%
\begin{divisionsolution}{2.4.7}{}{p:exercise:WRO}%
Find the solution to the recurrence relation \(a_n = 3a_{n-1} + 4a_{n-2}\) with initial terms \(a_0 = 2\) and \(a_1 = 3\text{.}\)%
\par\smallskip%
\noindent\textbf{\blocktitlefont Solution}.\quad{}By the Characteristic Root Technique. \(a_n = 4^n + (-1)^n\text{.}\)%
\end{divisionsolution}%
\begin{divisionsolution}{2.4.8}{}{p:exercise:CYX}%
Suppose that \(r^n\) and \(q^n\) are both solutions to a recurrence relation of the form \(a_n = \alpha a_{n-1} + \beta a_{n-2}\). Prove that \(c\cdot r^n + d \cdot q^n\) is also a solution to the recurrence relation, for any constants \(c, d\).%
\end{divisionsolution}%
\begin{divisionsolution}{2.4.9}{}{p:exercise:jgg}%
Think back to the magical candy machine at your neighborhood grocery store. Suppose that the first time a quarter is put into the machine 1 Skittle comes out. The second time, 4 Skittles, the third time 16 Skittles, the fourth time 64 Skittles, etc.%
\begin{enumerate}[label=(\alph*)]
\item{}Find both a recursive and closed formula for how many Skittles the \emph{n}th customer gets.%
\item{}Check your solution for the closed formula by solving the recurrence relation using the Characteristic Root technique.%
\end{enumerate}
%
\end{divisionsolution}%
\begin{divisionsolution}{2.4.10}{}{p:exercise:Pnp}%
Let \(a_n\) be the number of \(1 \times n\) tile designs you can make using \(1 \times 1\) squares available in 4 colors and \(1 \times 2\) dominoes available in 5 colors.%
\begin{enumerate}[label=(\alph*)]
\item{}First, find a recurrence relation to describe the problem. Explain why the recurrence relation is correct (in the context of the problem).%
\item{}Write out the first 6 terms of the sequence \(a_1, a_2, \ldots\).%
\item{}Solve the recurrence relation. That is, find a closed formula for \(a_n\).%
\end{enumerate}
%
\par\smallskip%
\noindent\textbf{\blocktitlefont Solution 1}.\quad{}%
\begin{enumerate}[label=(\alph*)]
\item{}\(a_n = 4a_{n-1} + 5a_{n-2}\).%
\item{}4, 21, 104, 521, 2604, 13021%
\item{}\(a_n = \frac{5}{6} 5^n + \frac{1}{6}(-1)^n\).%
\end{enumerate}
%
\par\smallskip%
\noindent\textbf{\blocktitlefont Solution 2}.\quad{}%
\begin{enumerate}[label=(\alph*)]
\item{}\(a_n = 4a_{n-1} + 5a_{n-2}\). Each path of length \(n\) must either start with one of the 4 \(1\times 1\) tiles, in each case there are then \(a_{n-1}\) ways to finish the path, or start with one of the 5 \(1\times 2\) tiles, in each case there are then \(a_{n-2}\) ways to finish the path.%
\item{}4, 21, 104, 521, 2604, 13021%
\item{}The characteristic equation is \(x^2 - 4x - 5 = 0\) so the characteristic roots are \(x = 5\) and \(x = -1\). Therefore the general solution is%
\begin{equation*}
a_n = a 5^n + b (-1)^n
\end{equation*}
%
\par
We solve for \(a\) and \(b\) using the fact that \(a_1 = 4\) and \(a_2 = 21\). We get \(a = \frac{5}{6}\) and \(b = \frac{1}{6}\). Therefore the solution is%
\begin{equation*}
a_n = \frac{5}{6} 5^n + \frac{1}{6}(-1)^n
\end{equation*}
%
\end{enumerate}
%
\end{divisionsolution}%
\begin{divisionsolution}{2.4.11}{}{p:exercise:vuy}%
You have access to \(1 \times 1\) tiles which come in 2 different colors and \(1\times 2\) tiles which come in 3 different colors. We want to figure out how many different \(1 \times n\) path designs we can make out of these tiles.%
\begin{enumerate}[label=(\alph*)]
\item{}Find a recursive definition for the sequence \(a_n\) of paths of length \(n\).%
\item{}Solve the recurrence relation using the Characteristic Root technique.%
\end{enumerate}
%
\par\smallskip%
\noindent\textbf{\blocktitlefont Solution}.\quad{}%
\begin{enumerate}[label=(\alph*)]
\item{}A path of length \(n\) can start with a square (\(1\times 1\) tile) or a domino (\(1\times 2\) tile).  There are two choices for the square, followed by \(a_{n-1}\) choices for the rest of the path.  There are three choices for the domino, followed by \(a_{n-2}\) choices for the rest of the path.%
\par
Thus the recursive definition is \(a_n = 2a_{n-1} + 3 a_{n-2}\), with initial conditions \(a_1 = 2\) and \(a_2 = 7\).%
\item{}The characteristic polynomial is \(x^2 - 2x - 3\), which has roots \(x = 3\) and \(x = -1\).  Thus \(a_ = a\cdot 3^n + b (-1)^n\).  Using the initial conditions we find \(a = \frac{3}{4}\) and \(b = \frac{1}{4}\).%
\par
Therefore the closed formula is \(a_n = \frac{3}{4}3^n + \frac{1}{4}(-1)^n\).%
\end{enumerate}
%
\end{divisionsolution}%
\begin{divisionsolution}{2.4.12}{}{p:exercise:bBH}%
Solve the recurrence relation \(a_n = 2a_{n-1} - a_{n-2}\).%
\begin{enumerate}[label=(\alph*)]
\item{}What is the solution if the initial terms are \(a_0 = 1\) and \(a_1 = 2\)?%
\item{}What do the initial terms need to be in order for \(a_9 = 30\)?%
\item{}For which \(x\) are there initial terms which make \(a_9 = x\)?%
\end{enumerate}
%
\par\smallskip%
\noindent\textbf{\blocktitlefont Solution}.\quad{}We have characteristic polynomial \(x^2 - 2x + 1\), which has \(x = 1\) as the only repeated root. Thus using the characteristic root technique for repeated roots, the general solution is \(a_n = a + bn\) where \(a\) and \(b\) depend on the initial conditions.%
\begin{enumerate}[label=(\alph*)]
\item{}\(a_n = 1 + n\).%
\item{}For example, we could have \(a_0 = 21\) and \(a_1 = 22\).%
\item{}For every \(x\). Take \(a_0 = x-9\) and \(a_1 = x-8\).%
\end{enumerate}
%
\end{divisionsolution}%
\begin{divisionsolution}{2.4.13}{}{p:exercise:HIQ}%
Consider the recurrence relation \(a_n = 4a_{n-1} - 4a_{n-2}\).%
\begin{enumerate}[label=(\alph*)]
\item{}Find the general solution to the recurrence relation (beware the repeated root).%
\item{}Find the solution when \(a_0 = 1\) and \(a_1 = 2\). %
\item{}Find the solution when \(a_0 = 1\) and \(a_1 = 8\). %
\end{enumerate}
%
\end{divisionsolution}%
\section*{2.5 Induction}
\addcontentsline{toc}{section}{2.5 Induction}
\sectionmark{2.5 Induction}
\subsection*{Exercises}
\addcontentsline{toc}{subsection}{Exercises}
\begin{divisionsolution}{2.5.1}{}{p:exercise:Qlr}%
On the way to the market, you exchange your cow for some magic dark chocolate espresso beans. These beans have the property that every night at midnight, each bean splits into two, effectively doubling your collection. You decide to take advantage of this and each morning (around 8am) you eat 5 beans.%
\begin{enumerate}[label=(\alph*)]
\item{}Explain why it is true that \emph{if} at noon on day \(n\) you have a number of beans ending in a 5, then at noon on day \(n+1\) you will still have a number of beans ending in a 5.%
\item{}Why is the previous fact not enough to conclude that you will always have a number of beans ending in a 5? What additional fact would you need?%
\item{}Assuming you have the additional fact in part (b), and have successfully proved the fact in part (a), how do you know that you will always have a number of beans ending in a 5? Illustrate what is going on by carefully explaining how the two facts above prove that you will have a number of beans ending in a 5 on \emph{day 4} specifically. In other words, explain why induction works in this context.%
\end{enumerate}
%
\par\smallskip%
\noindent\textbf{\blocktitlefont Solution}.\quad{}%
\begin{enumerate}[label=(\alph*)]
\item{}If we have a number of beans ending in a 5 and we double it, we will get a number of beans ending in a 0 (since \(5\cdot 2 = 10\) ). Then if we subtract 5, we will once again get a number of beans ending in a 5. Thus if on any day we have a number ending in a 5, the next day will also have a number ending in a 5.%
\item{}If you don't \emph{start} with a number of beans ending in a 5 (on day 1), the above reasoning is still correct but not helpful. For example, if you start with a number ending in a 3, the next day you will have a number ending in a 1.%
\item{}Part (b) is the base case and part (a) is the inductive case. If on day 1 we have a number ending in a 5 (by part (b)), then on day 2 we will also have a number ending in a 5 (by part (a)). Then by part (a) again, we will have a number ending in a 5 on day 3. By part (a) again, this means we will have a number ending in a 5 on day 4%
\par
The proof by induction would say that on \emph{every} day we will have a number ending in a 5, and this works because we can start with the base case, then use the inductive case over and over until we get up to our desired \(n\).%
\end{enumerate}
%
\end{divisionsolution}%
\begin{divisionsolution}{2.5.2}{}{p:exercise:wsA}%
Use induction to prove for all \(n \in \N\) that \(\d\sum_{k=0}^n 2^k = 2^{n+1} - 1\).%
\par\smallskip%
\noindent\textbf{\blocktitlefont Solution}.\quad{}\begin{solutionproof}
We must prove that \(1 + 2 + 2^2 + 2^3 + \cdots +2^n = 2^{n+1} - 1\) for all \(n \in \N\). Thus let \(P(n)\) be the statement \(1 + 2 + 2^2 + \cdots + 2^n = 2^{n+1} - 1\). We will prove that \(P(n)\) is true for all \(n \in \N\). First we establish the base case, \(P(0)\), which claims that \(1 = 2^{0+1} -1\). Since \(2^1 - 1 = 2 - 1 = 1\), we see that \(P(0)\) is true. Now for the inductive case. Assume that \(P(k)\) is true for an arbitrary \(k \in \N\). That is, \(1 + 2 + 2^2 + \cdots + 2^k = 2^{k+1} - 1\). We must show that \(P(k+1)\) is true (i.e., that \(1 + 2 + 2^2 + \cdots + 2^{k+1} = 2^{k+2} - 1\)). To do this, we start with the left-hand side of \(P(k+1)\) and work to the right-hand side:%
\begin{align*}
1 + 2 + 2^2 + \cdots + 2^k + 2^{k+1} = \amp ~ 2^{k+1} - 1 + 2^{k+1} \amp \text{by inductive hypothesis}\\
= \amp ~2\cdot 2^{k+1} - 1 \amp\\
= \amp ~ 2^{k+2} - 1 \amp
\end{align*}
%
\par
Thus \(P(k+1)\) is true so by the principle of mathematical induction, \(P(n)\) is true for all \(n \in \N\).%
\end{solutionproof}
\end{divisionsolution}%
\begin{divisionsolution}{2.5.3}{}{p:exercise:czJ}%
Prove that \(7^n - 1\) is a multiple of 6 for all \(n \in \N\).%
\par\smallskip%
\noindent\textbf{\blocktitlefont Solution}.\quad{}\begin{solutionproof}
Let \(P(n)\) be the statement ``\(7^n - 1\) is a multiple of 6.'' We will show \(P(n)\) is true for all \(n \in \N\). First we establish the base case, \(P(0)\). Since \(7^0 - 1 = 0\), and \(0\) is a multiple of 6, \(P(0)\) is true. Now for the inductive case. Assume \(P(k)\) holds for an arbitrary \(k \in \N\). That is, \(7^k - 1\) is a multiple of 6, or in other words, \(7^k - 1 = 6j\) for some integer \(j\). Now consider \(7^{k+1} - 1\):%
\begin{align*}
7^{k+1} - 1 ~ \amp = 7^{k+1} - 7 + 6 \amp \text{by cleverness:} -1 = -7 + 6\\
\amp = 7(7^k - 1) + 6 \amp \text{factor out a 7 from the first two terms}\\
\amp = 7(6j) + 6 \amp \text{by the inductive hypothesis}\\
\amp = 6(7j + 1) \amp \text{factor out a 6}
\end{align*}
%
\par
Therefore \(7^{k+1} - 1\) is a multiple of 6, or in other words, \(P(k+1)\) is true. Therefore by the principle of mathematical induction, \(P(n)\) is true for all \(n \in \N\).%
\end{solutionproof}
\end{divisionsolution}%
\begin{divisionsolution}{2.5.4}{}{p:exercise:IGS}%
Prove that \(1 + 3 + 5 + \cdots + (2n-1) = n^2\) for all \(n \ge 1\).%
\par\smallskip%
\noindent\textbf{\blocktitlefont Solution}.\quad{}\begin{solutionproof}
Let \(P(n)\) be the statement \(1+3 +5 + \cdots + (2n-1) = n^2\). We will prove that \(P(n)\) is true for all \(n \ge 1\). First the base case, \(P(1)\). We have \(1 = 1^2\) which is true, so \(P(1)\) is established. Now the inductive case. Assume that \(P(k)\) is true for some fixed arbitrary \(k \ge 1\). That is, \(1 + 3 + 5 + \cdots + (2k-1) = k^2\). We will now prove that \(P(k+1)\) is also true (i.e., that \(1 + 3 + 5 + \cdots + (2k+1) = (k+1)^2\)). We start with the left-hand side of \(P(k+1)\) and work to the right-hand side:%
\begin{align*}
1 + 3 + 5 + \cdots + (2k-1) + (2k+1) ~ \amp = k^2 + (2k+1) \amp \text{by ind. hyp.}\\
\amp = (k+1)^2 \amp \text{by factoring}
\end{align*}
%
\par
Thus \(P(k+1)\) holds, so by the principle of mathematical induction, \(P(n)\) is true for all \(n \ge 1\).%
\end{solutionproof}
\end{divisionsolution}%
\begin{divisionsolution}{2.5.5}{}{p:exercise:oOb}%
Prove that \(F_0 + F_2 + F_4 + \cdots + F_{2n} = F_{2n+1} - 1\) where \(F_n\) is the \(n\)th Fibonacci number.%
\par\smallskip%
\noindent\textbf{\blocktitlefont Solution}.\quad{}\begin{solutionproof}
Let \(P(n)\) be the statement \(F_0 + F_2 + F_4 + \cdots + F_{2n} = F_{2n+1} - 1\). We will show that \(P(n)\) is true for all \(n \ge 0\). First the base case is easy because \(F_0 = 0\) and \(F_1 = 1\) so \(F_0 = F_1 - 1\). Now consider the inductive case. Assume \(P(k)\) is true, that is, assume \(F_0 + F_2 + F_4 + \cdots + F_{2k} = F_{2k+1} - 1\). To establish \(P(k+1)\) we work from left to right:%
\begin{align*}
F_0 + F_2 + \cdots + F_{2k} + F_{2k+2} ~ \amp = F_{2k+1} - 1 + F_{2k+2} \amp \text{by ind. hyp.}\\
\amp = F_{2k+1} + F_{2k+2} - 1 \amp\\
\amp = F_{2k+3} - 1 \amp \text{by recursive def.}
\end{align*}
%
\par
Therefore \(F_0 + F_2 + F_4 + \cdots + F_{2k+2} = F_{2k+3} - 1\), which is to say \(P(k+1)\) holds. Therefore by the principle of mathematical induction, \(P(n)\) is true for all \(n \ge 0\).%
\end{solutionproof}
\end{divisionsolution}%
\begin{divisionsolution}{2.5.6}{}{p:exercise:UVk}%
Prove that \(2^n \lt n!\) for all \(n \ge 4\). (Recall, \(n! = 1\cdot 2 \cdot 3 \cdot \cdots\cdot n\).)%
\par\smallskip%
\noindent\textbf{\blocktitlefont Solution}.\quad{}\begin{solutionproof}
Let \(P(n)\) be the statement \(2^n \lt  n!\). We will show \(P(n)\) is true for all \(n \ge 4\). First, we check the base case and see that yes, \(2^4 \lt  4!\) (as \(16 \lt  24\)) so \(P(4)\) is true. Now for the inductive case. Assume \(P(k)\) is true for an arbitrary \(k \ge 4\). That is, \(2^k \lt  k!\). Now consider \(P(k+1)\): \(2^{k+1} \lt  (k+1)!\). To prove this, we start with the left side and work to the right side.%
\begin{align*}
2^{k+1}~ \amp = 2\cdot 2^k \amp\\
\amp \lt 2\cdot k! \amp \text{by the inductive hypothesis}\\
\amp \lt (k+1) \cdot k! \amp \text{ since } k+1 \gt 2\\
\amp = (k+1)! \amp
\end{align*}
%
\par
Therefore \(2^{k+1} \lt (k+1)!\) so we have established \(P(k+1)\). Thus by the principle of mathematical induction \(P(n)\) is true for all \(n \ge 4\).%
\end{solutionproof}
\end{divisionsolution}%
\begin{divisionsolution}{2.5.7}{}{p:exercise:Bct}%
Prove, by mathematical induction, that \(F_0 + F_1 + F_2 + \cdots + F_{n} = F_{n+2} - 1\), where \(F_n\) is the \(n\)th Fibonacci number (\(F_0 = 0\), \(F_1 = 1\) and \(F_n = F_{n-1} + F_{n-2}\)).%
\par\smallskip%
\noindent\textbf{\blocktitlefont Solution}.\quad{}This is saying that if we add up the first \(n\) Fibonacci numbers, we will get another Fibonacci number (specifically, the \((n+2)\)th one). Induction is a good idea here because it will be easy to just add one more Fibonacci number to the sum we already have. If we already have \(F_{k+2}\) and we add \(F_{k+1}\) we can use the recurrence relation to simplify this, becoming \(F_{k+3}\).%
\begin{solutionproof}
Let \(P(n)\) be the statement \(F_0 + F_1 + F_2 + \cdots + F_n = F_{n+2} - 1\). We will prove that \(P(n)\) is true for all \(n \ge 0\).%
\par
Base case: \(P(0)\) states that \(F_0 = F_2 - 1\), which is true because \(F_0 = 0\) and \(F_2 = 1\).%
\par
Inductive case: Assume \(P(k)\) is true for an arbitrary fixed \(k \ge 0\). That is,%
\begin{equation*}
F_0 + F_1 + F_2 + \cdots + F_k = F_{k+2} - 1
\end{equation*}
%
\par
We must prove that \(P(k+1)\) is true as well (i.e. that \(F_0 + F_1 + \cdots +F_{k+1} = F_{k+3} - 1\)). Start with the left-hand side:%
\begin{align*}
F_0 + F_1 + F_2 + \cdots + F_k + F_{k+1} \amp = F_{k+2} - 1 + F_{k+1} \amp \mbox{ by the inductive hypothesis}\\
\amp = F_{k+3} - 1 \amp \mbox{ by the definition of the Fibonacci numbers}
\end{align*}
%
\par
Thus \(P(k+1)\) is true.%
\par
Therefore by the principle of mathematical induction, \(P(n)\) is true for all \(n \ge 0\).%
\end{solutionproof}
\end{divisionsolution}%
\begin{divisionsolution}{2.5.8}{}{p:exercise:hjC}%
Zombie Euler and Zombie Cauchy, two famous zombie mathematicians, have just signed up for Twitter accounts. After one day, Zombie Cauchy has more followers than Zombie Euler. Each day after that, the number of new followers of Zombie Cauchy is exactly the same as the number of new followers of Zombie Euler (and neither lose any followers). Explain how a proof by mathematical induction can show that on every day after the first day, Zombie Cauchy will have more followers than Zombie Euler. That is, explain what the base case and inductive case are, and why they together prove that Zombie Cauchy will have more followers on the 4th day.%
\par\smallskip%
\noindent\textbf{\blocktitlefont Solution}.\quad{}The idea here is that because we know Zombie Cauchy starts ahead, and each day increases by the same amount as Zombie Euler, he will always be ahead.%
\par
The base case is that Zombie Cauchy has more followers than Zombie Euler on day 1. We know this is true because it says so in the problem.%
\par
The inductive case is that \emph{if} Zombie Cauchy has more followers on day \(k\), then he will still have more followers on day \(k+1\). We know this is true because each day, the Zombies receive an equal number of new followers.%
\par
Together, the base case and inductive case show that on the 4th day, Zombie Cauchy will be ahead: he is ahead on day 1, and because on day 1 he is ahead, by the inductive case he will also be ahead on day 2. By the inductive case again, he will be ahead on day 3 since he is ahead on day 2, and since he is ahead on day 3, he will also be ahead on day 4. Of course we could keep doing this up to any day.%
\end{divisionsolution}%
\begin{divisionsolution}{2.5.9}{}{p:exercise:NqL}%
Find the largest number of points which a football team cannot get exactly using just 3-point field goals and 7-point touchdowns (ignore the possibilities of safeties, missed extra points, and two point conversions). Prove your answer is correct by mathematical induction.%
\par\smallskip%
\noindent\textbf{\blocktitlefont Hint}.\quad{}It is not possible to score exactly 11 points.  Can you prove that you can score \(n\) points for any \(n \ge 12\)?%
\par\smallskip%
\noindent\textbf{\blocktitlefont Solution}.\quad{}First note that it is impossible to make 11 points - if only field goals are made, the points must be a multiple of 3, if 1 touchdown is made, the possible point totals are 7, 10, 13, \textellipsis{} and two touchdowns are already too much.%
\par
We will prove that 11 is the largest number of points which cannot be made. In other words, any number of points greater than or equal to 12 can be made.%
\begin{solutionproof}
Let \(P(n)\) be the statement ``it is possible to make \(n\) points using touchdowns and field goals.'' We will prove \(P(n)\) is true for all \(n \ge 12\).%
\par
First the base case: You can make 12 points with 4 field goals, so \(P(12)\) is true.%
\par
Now the inductive case: Assume \(P(k)\) is true for some fixed \(k \ge 12\). That is, it is possible to make \(k\) points. Since \(k \ge 12\), we must have made the \(k\) points using either at least 2 field goals or at least 2 touchdowns, or both (because if we used just one of each we would have only 10 points). Now if the \(k\) points were accomplished with 2 (or more) field goals, then replace 2 field goals with 1 touchdown. This increases to point total by 1, giving \(k + 1\) points. On the other hand, if the \(k\) points were accomplished with \(2\) (or more) touchdowns, replace 2 touchdowns with 5 field goals, again increasing the point total by 1, giving \(k+1\) points. Using one of these two substitutions, we can make \(k+1\) points, so \(P(k+1)\) is true, establishing the inductive case.%
\par
Therefore by the principle of mathematical induction, \(P(n)\) is true for all \(n \ge 12\).%
\end{solutionproof}
\end{divisionsolution}%
\begin{divisionsolution}{2.5.10}{}{p:exercise:txU}%
Prove that the sum of \(n\) squares can be found as follows%
\begin{equation*}
1^2 +2^2 +3^2+...+n^2 = \frac{n(n+1)(2n+1)}{6}\text{.}
\end{equation*}
%
\par\smallskip%
\noindent\textbf{\blocktitlefont Solution}.\quad{}This question is asking us to show that the sum of squares for \(n\) numbers can be found using the formula \(\frac{n(n+1)(2n+1)}{6}\). We can definitely see this is true for the first few instances, but we are really taking it on faith that it is true for the first \(n\) squares. So, it seems like induction would be a good place to start.%
\begin{solutionproof}
First, let \(P(n)\) be the statement that is given.%
\par
Base case: We must now show that our base case \(P(1)\) is true%
\begin{equation*}
1^2 = \frac{1(1+1)(2(1)+1)}{6} = \frac{6}{6} =1
\end{equation*}
%
\par
Inductive case: Now, we assume that for some \(k\leq n\) that \(P(k)\) is true and show that \(P(k+1)\) is true. Namely, assume that \(1^2 +2^2 +3^2+...+k^2 = \frac{k(k+1)(2k+1)}{6}\) and show that%
\begin{equation*}
1^2 +2^2 +3^2+...+k^2+{(k+1)}^2 = \frac{(k+1)((k+1)+1)(2(k+1)+1)}{6}
\end{equation*}
%
\par
At this point it is advantageous to start on one side of the equality. So, choosing the left-hand side to start I shall manipulate it so that it looks like the right-hand side.%
\begin{align*}
1^2 +2^2 +3^2+...+k^2+{(k+1)}^2 =\\
= \amp \frac{k(k+1)(2k+1)}{6} +(k+1)^2 \mbox{ by our inductive hypothesis}\\
= \amp \frac{k(k+1)(2k+1)}{6} +\frac{6(k+1)^2}{6}\\
= \amp \frac{k(k+1)(2k+1)+6(k+1)^2}{6}\\
= \amp \frac{(k+1)[k(2k+1)+6(k+1)]}{6} \mbox{ factoring out  from each term}\\
= \amp \frac{(k+1)[2k^2+k+6k+6]}{6}\\
= \amp \frac{(k+1)[2k^2+7k+6]}{6}\\
= \amp \frac{(k+1)[(2k+3)(k+2)]}{6}\\
= \amp \frac{(k+1)(2(k+1)+1)((k+1)+1)}{6}
\end{align*}
%
\par
Thus, \(P(k+1)\) is true.%
\par
Therefore, by the principle of mathematical induction \(P(n)\) is true for all \(n \geq 1\)%
\end{solutionproof}
\end{divisionsolution}%
\begin{divisionsolution}{2.5.11}{}{p:exercise:ZFd}%
Prove that the sum of the interior angles of a convex \(n\)-gon is \((n-2)\cdot 180^\circ\). (A convex \(n\)-gon is a polygon with \(n\) sides for which each interior angle is less than \(180^\circ\).)%
\par\smallskip%
\noindent\textbf{\blocktitlefont Hint}.\quad{}Start with \((k+1)\)-gon and divide it up into a \(k\)-gon and a triangle.%
\par\smallskip%
\noindent\textbf{\blocktitlefont Solution}.\quad{}Every \(n\)-gon (for \(n \ge 4\) ) can be divided into a triangle and a \((n-1)\)-gon by drawing a line between two vertices one apart. The sum of the interior angles of the \(n\)-gon will then be the sum of the angles in the triangle and the angles in the \((n-1)\)-gon. This shows how we can go from one case to the next. Since we are adding a triangle, we are adding \(180^\circ\) to the sum, which is what incrementing \(n\) in \((n-2)\cdot 180^\circ\) does.%
\begin{solutionproof}
Let \(P(n)\) be the statement, ``the sum of the interior angles of a convex \(n\)-gon is \((n-2)\cdot 180^\circ\).'' We will prove \(P(n)\) is true for all \(n \ge 3\)%
\par
Base case: When \(n=3\), we have a triangle, and we know the sum of the interior angles of a triangle is \(180^\circ = (3-2)\cdot 180^\circ\). Thus \(P(3)\) is true%
\par
Inductive case: Assume \(P(k)\) is true for some arbitrary \(k \ge 3\). That is, the sum of the angles of any convex \(k\)-gon is \((k-2)\cdot 180^\circ\). Now consider an arbitrary convex \((k+1)\)-gon. Draw an edge between two vertices one apart (for example, between the first and third vertices counting clockwise). Since we have at least 4 vertices, this is a new edge, and divides the \(k+1\) -sided polygon into a convex \(k\)-gon and a triangle. The sum of the angles of the \((k+1)\)-gon will be exactly the sum of the angles in the \(k\)-gon plus the sum of the angles of the triangle. But the \(k\)-gon has sum of angles \((k-2)\cdot 180^\circ\) (by the inductive hypothesis) and the triangle has sum of angles \(180^\circ\), so together the sum of the interior angles of the \((k+1)\)-gon is%
\begin{equation*}
(k-2)\cdot 180^\circ + 180^\circ = ((k+1)-2)\cdot 180^\circ
\end{equation*}
Since we can do this for \emph{any} convex \((k+1)\)-gon, we see that \(P(k+1)\) is true%
\par
Therefore, by the principle of mathematical induction, \(P(n)\) is true for all \(n\ge 3\).%
\end{solutionproof}
\end{divisionsolution}%
\begin{divisionsolution}{2.5.12}{}{p:exercise:FMm}%
What is wrong with the following ``proof'' of the ``fact'' that \(n+3 = n+7\) for all values of \(n\) (besides of course that the thing it is claiming to prove is false)?%
\begin{proof}{}{p:proof:uhO}
Let \(P(n)\) be the statement that \(n + 3 = n + 7\). We will prove that \(P(n)\) is true for all \(n \in \N\). Assume, for induction that \(P(k)\) is true. That is, \(k+3 = k+7\). We must show that \(P(k+1)\) is true. Now since \(k + 3 = k + 7\), add 1 to both sides. This gives \(k + 3 + 1 = k + 7 + 1\). Regrouping \((k+1) + 3 = (k+1) + 7\). But this is simply \(P(k+1)\). Thus by the principle of mathematical induction \(P(n)\) is true for all \(n \in \N\).%
\end{proof}
\par\smallskip%
\noindent\textbf{\blocktitlefont Solution}.\quad{}The only problem is that we never established the base case. Of course, when \(n = 0\), \(0+3 \ne 0+7\).%
\end{divisionsolution}%
\begin{divisionsolution}{2.5.13}{}{p:exercise:lTv}%
The proof in the previous problem does not work. But if we modify the ``fact,'' we can get a working proof. Prove that \(n + 3 \lt n + 7\) for all values of \(n \in \N\). You can do this proof with algebra (without induction), but the goal of this exercise is to write out a valid induction proof.%
\par\smallskip%
\noindent\textbf{\blocktitlefont Solution}.\quad{}\begin{solutionproof}
Let \(P(n)\) be the statement that \(n + 3 \lt n + 7\). We will prove that \(P(n)\) is true for all \(n \in \N\). First, note that the base case holds: \(0+3 \lt 0+7\). Now assume for induction that \(P(k)\) is true. That is, \(k+3 \lt k+7\). We must show that \(P(k+1)\) is true. Now since \(k + 3 \lt k + 7\), add 1 to both sides. This gives \(k + 3 + 1 \lt k + 7 + 1\). Regrouping \((k+1) + 3 \lt (k+1) + 7\). But this is simply \(P(k+1)\). Thus by the principle of mathematical induction \(P(n)\) is true for all \(n \in \N\).%
\end{solutionproof}
\end{divisionsolution}%
\begin{divisionsolution}{2.5.14}{}{p:exercise:SaE}%
Find the flaw in the following ``proof'' of the ``fact'' that \(n \lt 100\) for every \(n \in \N\).%
\begin{proof}{}{p:proof:SKy}
Let \(P(n)\) be the statement \(n \lt 100\). We will prove \(P(n)\) is true for all \(n \in \N\). First we establish the base case: when \(n = 0\), \(P(n)\) is true, because \(0 \lt 100\). Now for the inductive step, assume \(P(k)\) is true. That is, \(k \lt 100\). Now if \(k \lt 100\), then \(k\) is some number, like 80. Of course \(80+1 = 81\) which is still less than 100. So \(k +1 \lt 100\) as well. But this is what \(P(k+1)\) claims, so we have shown that \(P(k) \imp P(k+1)\). Thus by the principle of mathematical induction, \(P(n)\) is true for all \(n \in \N\).%
\end{proof}
\par\smallskip%
\noindent\textbf{\blocktitlefont Solution}.\quad{}The problem here is that while \(P(0)\) is true, and while \(P(k) \imp P(k+1)\) for \emph{some} values of \(k\), there is at least one value of \(k\) (namely \(k = 99\)) when that implication fails. For a valid proof by induction, \(P(k) \imp P(k+1)\) must be true for all values of \(k\) greater than or equal to the base case.%
\end{divisionsolution}%
\begin{divisionsolution}{2.5.15}{}{x:exercise:exc-seq-lessthan-100}%
While the above proof does not work (it better not since the statement it is trying to prove is false!) we can prove something similar. Prove that there is a strictly increasing sequence \(a_1, a_2, a_3, \ldots\) of numbers (not necessarily integers) such that \(a_n \lt 100\) for all \(n \in \N\). (By \terminology{strictly increasing} we mean \(a_n \lt a_{n+1}\) for all \(n\). So each term must be larger than the last.)%
\par\smallskip%
\noindent\textbf{\blocktitlefont Hint}.\quad{}For the inductive step, you can assume you have a strictly increasing sequence up to \(a_k\) where \(a_k \lt 100\).  Now you just need to find the next term \(a_{k+1}\) so that \(a_{k} \lt a_{k+1} \lt 100\).  What should \(a_{k+1}\) be?%
\par\smallskip%
\noindent\textbf{\blocktitlefont Solution}.\quad{}\begin{solutionproof}
Let \(P(n)\) be the statement ``there is a strictly increasing sequence \(a_1, a_2, \ldots,
a_n\) with \(a_n \lt 100\).'' We will prove \(P(n)\) is true for all \(n \ge 1\). First we establish the base case: \(P(1)\) says there is a single number \(a_1\) with \(a_1 \lt 100\). This is true \textendash{} take \(a_1 = 0\). Now for the inductive step, assume \(P(k)\) is true. That is there exists a strictly increasing sequence \(a_1, a_2, a_3, \ldots,
a_k\) with \(a_k \lt 100\). Now consider this sequence, plus one more term, \(a_{k+1}\) which is greater than \(a_k\) but less than \(100\). Such a number exists, for example, the average between \(a_k\) and 100. So then \(P(k+1)\) is true, so we have shown that \(P(k) \imp P(k+1)\). Thus by the principle of mathematical induction, \(P(n)\) is true for all \(n \in \N\).%
\end{solutionproof}
\end{divisionsolution}%
\begin{divisionsolution}{2.5.16}{}{p:exercise:eoW}%
What is wrong with the following ``proof'' of the ``fact'' that for all \(n \in \N\), the number \(n^2 + n\) is odd?%
\begin{proof}{}{p:proof:LfZ}
Let \(P(n)\) be the statement ``\(n^2 + n\) is odd.'' We will prove that \(P(n)\) is true for all \(n \in \N\). Suppose for induction that \(P(k)\) is true, that is, that \(k^2 + k\) is odd. Now consider the statement \(P(k+1)\). Now \((k+1)^2 + (k+1) = k^2 + 2k + 1 + k + 1 = k^2 + k + 2k + 2\). By the inductive hypothesis, \(k^2 + k\) is odd, and of course \(2k + 2\) is even. An odd plus an even is always odd, so therefore \((k+1)^2 + (k+1)\) is odd. Therefore by the principle of mathematical induction, \(P(n)\) is true for all \(n \in \N\).%
\end{proof}
\par\smallskip%
\noindent\textbf{\blocktitlefont Solution}.\quad{}We once again failed to establish the base case: when \(n = 0\), \(n^2 + n = 0\) which is even, not odd.%
\end{divisionsolution}%
\begin{divisionsolution}{2.5.17}{}{p:exercise:Kwf}%
Now give a valid proof (by induction, even though you might be able to do so without using induction) of the statement, ``for all \(n \in \N\), the number \(n^2 + n\) is even.''%
\par\smallskip%
\noindent\textbf{\blocktitlefont Hint}.\quad{}For the inductive case, you will need to show that \((k+1)^2 + (k+1)\) is even.  Factor this out and locate the part of it that is \(k^2 + k\).  What have you assumed about that quantity?%
\par\smallskip%
\noindent\textbf{\blocktitlefont Solution}.\quad{}\begin{solutionproof}
Let \(P(n)\) be the statement ``\(n^2 + n\) is even.'' We will prove that \(P(n)\) is true for all \(n \in \N\). First the base case: when \(n = 0\), we have \(n^2 + n = 0\) which is even, so \(P(0)\) is true. Now suppose for induction that \(P(k)\) is true, that is, that \(k^2 + k\) is even. Now consider the statement \(P(k+1)\). Now \((k+1)^2 + (k+1) = k^2 + 2k + 1 + k + 1 = k^2 + k + 2k + 2\). By the inductive hypothesis, \(k^2 + k\) is even, and of course \(2k + 2\) is even. An even plus an even is always even, so therefore \((k+1)^2 + (k+1)\) is even. Therefore by the principle of mathematical induction, \(P(n)\) is true for all \(n \in \N\).%
\end{solutionproof}
\end{divisionsolution}%
\begin{divisionsolution}{2.5.18}{}{p:exercise:qDo}%
Prove that there is a sequence of positive real numbers \(a_0, a_1, a_2, \ldots\) such that the partial sum \(a_0 + a_1 + a_2 + \cdots + a_n\) is strictly less than \(2\) for all \(n \in \N\). Hint: think about how you could define what \(a_{k+1}\) is to make the induction argument work.%
\par\smallskip%
\noindent\textbf{\blocktitlefont Hint}.\quad{}This is similar to Exercise~2.5.15, although there you were showing that a sequence had all its terms less than some value, and here you are showing that the sum is less than som value.  But the partial sums forms a sequence, so this is actually very similar.%
\par\smallskip%
\noindent\textbf{\blocktitlefont Solution}.\quad{}The idea is to define the sequence so that \(a_n\) is less than the distance between the previous partial sum and 2. That way when you add it into the next partial sum, the partial sum is still less than 2. You could do this ahead of time, or use a clever \(P(n)\) in the induction proof.%
\begin{solutionproof}
Let \(P(n)\) be the statement, ``there is a sequence of positive real numbers \(a_0, a_1, a_2, \ldots,
a_n\) such that \(a_0 + a_1 + a_2 + \cdots + a_n \lt 2\).''%
\par
Base case: Pick any \(a_0 \lt 2\).%
\par
Inductive case: Assume that \(a_1 + a_2 + \cdots + a_k \lt 2\). Now let \(a_{k+1} = \frac{2- a_1 + a_2 + \cdots + a_k}{2}\). Then \(a_1 + a_2 + \cdots +a_k + a_{k+1} \lt 2\).%
\par
Therefore, by the principle of mathematical induction, \(P(n)\) is true for all \(n \in \N\)%
\end{solutionproof}
\end{divisionsolution}%
\begin{divisionsolution}{2.5.19}{}{p:exercise:WKx}%
Prove that every positive integer is either a power of 2, or can be written as the sum of distinct powers of 2.%
\par\smallskip%
\noindent\textbf{\blocktitlefont Solution}.\quad{}The proof will by by strong induction.%
\begin{solutionproof}
Let \(P(n)\) be the statement ``\(n\) is either a power of 2 or can be written as the sum of distinct powers of 2.'' We will show that \(P(n)\) is true for all \(n \ge 1\).%
\par
Base case: \(1 = 2^0\) is a power of 2, so \(P(1)\) is true.%
\par
Inductive case: Suppose \(P(k)\) is true for all \(k \lt n\). Now if \(n\) is a power of 2, we are done. If not, let \(2^x\) be the largest power of 2 strictly less than \(n\). Consider \(n - 2^x\), which is a smaller number, in fact smaller than both \(n\) and \(2^x\). Thus \(n-2^x\) is either a power of 2 or can be written as the sum of distinct powers of 2, but none of them are going to be \(2^x\), so the together with \(2^x\) we have written \(n\) as the sum of distinct powers of 2.%
\par
Therefore, by the principle of (strong) mathematical induction, \(P(n)\) is true for all \(n \ge 1\).%
\end{solutionproof}
\end{divisionsolution}%
\begin{divisionsolution}{2.5.20}{}{p:exercise:CRG}%
Prove, using strong induction, that every natural number is either a Fibonacci number or can be written as the \emph{sum} of \emph{distinct} Fibonacci numbers.%
\par\smallskip%
\noindent\textbf{\blocktitlefont Hint}.\quad{}As with the previous question, we will want to subtract something from \(n\) in the inductive step.  There we subtracted the largest power of 2 less than \(n\).  So what should you subtract here?%
\par
Note, you will still need to take care here that the sum you get from the inductive hypothesis, together with the number you subtracted will be a sum of \emph{distinct} Fibonacci numbers.  In fact, you could prove that the Fibonacci numbers in the sum are non-consecutive!%
\par\smallskip%
\noindent\textbf{\blocktitlefont Solution}.\quad{}\begin{solutionproof}
Let \(P(n)\) be the statement ``\(n\) is either a Fibonacci number or is the sum of distinct Fibonacci numbers.'' We will prove \(P(n)\) is true for all \(n \ge 0\). Note that \(P(0)\), \(P(1)\), \(P(2)\), and \(P(3)\) are all true since \(0, 1, 2, 3\) are Fibonacci numbers. Suppose that \(P(k)\) is true for all \(k \lt n\). Now if \(n\) is a Fibonacci number, then \(P(n)\) is true. Otherwise, let \(m\) be the largest Fibonacci number less than \(n\). Consider the number \(n - m\). This is a number smaller than \(n\), so \(P(n-m)\) is true. So wrote \(n-m\) either as a single Fibonacci number (if it is one) or as the sum of distinct Fibonacci numbers. Adding the Fibonacci number \(m\) will give \(n\) as the sum of Fibonacci numbers. All that is left is to explain why this sum does not have any repeats (that the numbers are distinct). We know that there are no repeats in the sum for \(n-m\) by the inductive hypothesis. So the only way we might get a repeat is if the sum for \(n-m\) contained an \(m\). But there must be a Fibonacci number between \(m\) and \(2m\) (since \(m\) plus the previous Fibonacci number is a Fibonacci number), and we picked \(m\) to be the largest Fibonacci number less than \(n\). This would mean that \(2m > n\), so there cannot be two \(m\)'s in the sum.%
\end{solutionproof}
\end{divisionsolution}%
\begin{divisionsolution}{2.5.21}{}{p:exercise:iYP}%
Use induction to prove that if \(n\) people all shake hands with each other, that the total number of handshakes is \(\frac{n(n-1)}{2}\).%
\par\smallskip%
\noindent\textbf{\blocktitlefont Hint}.\quad{}We have already proved this without using induction, but looking at it inductively sheds light onto the problem (and is fun).%
\par
The question you need to answer to complete the inductive step is, how many new handshakes take place when a person \(k+1\) enters the room.  Why does adding this give you the correct formula?%
\par\smallskip%
\noindent\textbf{\blocktitlefont Solution}.\quad{}Note, we have already proven this without using induction, but looking at it inductively sheds light onto the problem (and is fun).%
\begin{solutionproof}
Let \(P(n)\) be the statement ``when \(n\) people shake hands with each other, there are a total of \(\frac{n(n-1)}{2}\) handshakes.''%
\par
Base case: When \(n=2\), there will be one handshake, and \(\frac{2(2-1)}{2} = 1\). Thus \(P(2)\) is true.%
\par
Inductive case: Assume \(P(k)\) is true for arbitrary \(k\ge 2\) (that the number of handshakes among \(k\) people is \(\frac{k(k-1)}{2}\). What happens if a \(k+1\)st person shows up? How many \emph{new} handshakes take place? The new person must shake hands with everyone there, which is \(k\) new handshakes. So the total is now \(\frac{k(k-1)}{2} + k = \frac{(k+1)k}{2}\), as needed.%
\par
Therefore, by the principle of mathematical induction, \(P(n)\) is true for all \(n \ge 2\).%
\end{solutionproof}
\end{divisionsolution}%
\begin{divisionsolution}{2.5.22}{}{p:exercise:PfY}%
Suppose that a particular real number \(x\) has the property that \(x + \frac{1}{x}\) is an integer. Prove that \(x^n + \frac{1}{x^n}\) is an integer for all natural numbers \(n\).%
\par\smallskip%
\noindent\textbf{\blocktitlefont Hint}.\quad{}You will need to use strong induction. For the inductive case, try multiplying \(\left (x^k + \frac{1}{x^{k}}\right)\left(x+\frac{1}{x}\right)\) and collect which terms together are integers.%
\par\smallskip%
\noindent\textbf{\blocktitlefont Solution}.\quad{}When \(n = 0\), we get \(x^0 +\frac{1}{x^0} = 2\) and when \(n = 1\), \(x + \frac{1}{x}\) is an integer, so the base case holds. Now assume the result holds for all natural numbers \(n \lt  k\). In particular, we know that \(x^{k-1} + \frac{1}{x^{k-1}}\) and \(x + \frac{1}{x}\) are both integers. Thus their product is also an integer. But,%
\begin{align*}
\left(x^{k-1} + \frac{1}{x^{k-1}}\right)\left(x + \frac{1}{x}\right) \amp = x^k + \frac{x^{k-1}}{x} + \frac{x}{x^{k-1}} + \frac{1}{x^k}\\
\amp = x^k + \frac{1}{x^k} + x^{k-2} + \frac{1}{x^{k-2}}
\end{align*}
%
\par
Note also that \(x^{k-2} + \frac{1}{x^{k-2}}\) is an integer by the induction hypothesis, so we can conclude that \(x^k + \frac{1}{x^k}\) is an integer.%
\end{divisionsolution}%
\begin{divisionsolution}{2.5.23}{}{p:exercise:vnh}%
Use induction to prove that \(\d\sum_{k=0}^n {n \choose k} = 2^n\). That is, the sum of the \(n\)th row of Pascal's Triangle is \(2^n\).%
\par\smallskip%
\noindent\textbf{\blocktitlefont Hint}.\quad{}Here's the idea: since every entry in Pascal's Triangle is the sum of the two entries above it, we can get the \(k+1\)st row by adding up all the pairs of entry from the \(k\)th row. But doing this uses each entry on the \(k\)th row twice. Thus each time we drop to the next row, we double the total. Of course, row 0 has sum \(1 = 2^0\) (the base case). Now try to make this precise with a formal induction proof. You will use the fact that \({n \choose k} = {n-1 \choose k-1} + {n-1 \choose k}\) for the inductive case.%
\end{divisionsolution}%
\begin{divisionsolution}{2.5.24}{}{p:exercise:buq}%
Use induction to prove \({4 \choose 0} + {5 \choose 1} + {6 \choose 2} + \cdots + {4+n \choose n} = {5+n \choose n}\). (This is an example of the hockey stick theorem.)%
\par\smallskip%
\noindent\textbf{\blocktitlefont Hint}.\quad{}To see why this works, try it on a copy of Pascal's triangle. We are adding up the entries along a diagonal, starting with the 1 on the left-hand side of the 4th row. Suppose we add up the first 5 entries on this diagonal. The claim is that the sum is the entry below and to the left of the last of these 5 entries. Note that if this is true, and we instead add up the first 6 entries, we will need to add the entry one spot to the right of the previous sum. But these two together give the entry below them, which is below and left of the last of the 6 entries on the diagonal. If you follow that, you can see what is going on. But it is not a great proof. A formal induction proof is needed.%
\par\smallskip%
\noindent\textbf{\blocktitlefont Solution}.\quad{}\begin{solutionproof}
Let \(P(n)\) be the statement \({4 \choose 0} + {5 \choose 1} + {6 \choose 2} + \cdots + {4+n \choose n} = {5+n \choose n}\). For the base case, consider \(n = 0\). This says \({4 \choose 0} = {5 \choose 0}\). Since these are both 1, the base case is true. Now for the inductive case, suppose \(P(k)\) is true. That is, \({4 \choose 0} + {5 \choose 1} + {6 \choose 2} + \cdots + {4+k \choose k} = {5+k \choose k}\). If we add \({4+k+1 \choose k+1}\) to both sides, we get%
\begin{equation*}
{4 \choose 0} + {5 \choose 1} + {6 \choose 2} + \cdots + {4+k \choose k} + {5+k \choose k+1}= {5+k \choose k} + {5+k \choose k+1}
\end{equation*}
%
\par
But \({5+k \choose k} + {5+k \choose k+1} = {5+k+1 \choose k+1}\). In other words, we have%
\begin{equation*}
{4 \choose 0} + {5 \choose 1} + {6 \choose 2} + \cdots + {4+k \choose k} + {5+k \choose k+1} = {5+k+1 \choose k+1}
\end{equation*}
which is to say that \(P(k+1)\) is true. Therefore, by the principle of mathematical induction, \(P(n)\) is true for all \(n \ge 0\).%
\end{solutionproof}
\end{divisionsolution}%
\begin{divisionsolution}{2.5.25}{}{p:exercise:HBz}%
Use the product rule for logarithms (\(\log(ab) = \log(a) + \log(b)\)) to prove, by induction on \(n\), that \(\log(a^n) = n \log(a)\), for all natural numbers \(n \ge 2\).%
\par\smallskip%
\noindent\textbf{\blocktitlefont Solution}.\quad{}The idea here is that if we take the logarithm of \(a^n\), we can increase \(n\) by 1 if we multiply by another \(a\) (inside the logarithm). This results in adding 1 more \(\log(a)\) to the total.%
\begin{solutionproof}
Let \(P(n)\) be the statement \(\log(a^n) = n \log(a)\). The base case, \(P(2)\) is true, because \(\log(a^2) = \log(a\cdot a) = \log(a) + \log(a) = 2\log(a)\), by the product rule for logarithms. Now assume, for induction, that \(P(k)\) is true. That is, \(\log(a^k) = k\log(a)\). Consider \(\log(a^{k+1})\). We have%
\begin{equation*}
\log(a^{k+1}) = \log(a^k\cdot a) = \log(a^k) + \log(a) = k\log(a) + \log(a)\text{,}
\end{equation*}
with the last equality due to the inductive hypothesis. But this simplifies to \((k+1) \log(a)\), establishing \(P(k+1)\). Therefore by the principle of mathematical induction, \(P(n)\) is true for all \(n \ge 2\).%
\end{solutionproof}
\end{divisionsolution}%
\begin{divisionsolution}{2.5.26}{}{p:exercise:nII}%
Let \(f_1, f_2,\ldots, f_n\) be differentiable functions. Prove, using induction, that%
\begin{equation*}
(f_1 + f_2 + \cdots + f_n)' = f_1' + f_2' + \cdots + f_n'\text{.}
\end{equation*}
%
\par
You may assume \((f+g)' = f' + g'\) for any differentiable functions \(f\) and \(g\).%
\par\smallskip%
\noindent\textbf{\blocktitlefont Hint}.\quad{}You are allowed to assume the base case. For the inductive case, group all but the last function together as one sum of functions, then apply the usual sum of derivatives rule, and then the inductive hypothesis.%
\end{divisionsolution}%
\begin{divisionsolution}{2.5.27}{}{p:exercise:TPR}%
Suppose \(f_1, f_2, \ldots, f_n\) are differentiable functions. Use mathematical induction to prove the generalized product rule:%
\begin{equation*}
(f_1 f_2 f_3 \cdots f_n)' = f_1' f_2 f_3 \cdots f_n + f_1 f_2' f_3 \cdots f_n + f_1 f_2 f_3' \cdots f_n + \cdots + f_1 f_2 f_3 \cdots f_n'\text{.}
\end{equation*}
%
\par
You may assume the product rule for two functions is true.%
\par\smallskip%
\noindent\textbf{\blocktitlefont Hint}.\quad{}For the inductive step, we know by the product rule for two functions that%
\begin{equation*}
(f_1f_2f_3 \cdots f_k f_{k+1})' = (f_1f_2f_3\cdots f_k)'f_{k+1} + (f_1f_2f_3\cdots f_k)f_{k+1}'\text{.}
\end{equation*}
%
\par
Then use the inductive hypothesis on the first summand, and distribute.%
\end{divisionsolution}%
\begin{divisionsolution}{2.5.28}{}{p:exercise:zXa}%
You will prove that the Fibonacci numbers satisfy the identity \(F_n^2 + F_{n+1}^2 = F_{2n+1}\). One way to do this is to prove the more general identity,%
\begin{equation*}
F_mF_n + F_{m+1}F_{n+1} = F_{m+n+1}\text{,}
\end{equation*}
and realize that when \(m = n\) we get our desired result.%
\par
Note that we now have two variables, so we want to prove this for all \(m \ge 0\) \emph{and} all \(n \ge 0\) at the same time. For each such pair \((m,n)\), let \(P(m,n)\) be the statement \(F_mF_n + F_{m+1}F_{n+1} = F_{m+n+1}\)%
\begin{enumerate}[label=(\alph*)]
\item{}First fix \(m = 0\) and give a proof by mathematical induction that \(P(0,n)\) holds for all \(n \ge 0\). Note this proof will be very easy.%
\item{}Now fix an arbitrary \(n\) and give a proof by \emph{strong} mathematical induction that \(P(m,n)\) holds for all \(m \ge 0\).%
\item{}You can now conclude that \(P(m,n)\) holds for all \(m,n\ge 0\). Do you believe that? Explain why this sort of induction is valid. For example, why do your proofs above guarantee that \(P(2,3)\) is true?%
\end{enumerate}
%
\par\smallskip%
\noindent\textbf{\blocktitlefont Solution}.\quad{}%
\begin{enumerate}[label=(\alph*)]
\item{}The base case is when \(n = 0\), so we check \(P(0,0)\). This is just \(F_0F_0 + F_1F_1 = F_1\), which is true since \(0+1 = 1\).%
\par
For the inductive case, we assume \(P(0,k)\) is true and want to show that \(P(0,k+1)\) is true. But \(P(0,k+1)\) is the satement \(F_0F_{k+1} + F_{1}F_{k+2} = F_{k+2}\), which we see immediately is true since \(F_0 = 0\). (We didn't even need the inductive hypothesis here, although other double induction proofs might use it.)%
\par
Thus we see that \(P(0,n)\) is true for all \(n \ge 0\).%
\item{}Remember that \(n\) is fixed. So our base cases are for \(m = 0\) and \(m = 1\). That is, we need to verify that \(P(0,n)\) and \(P(1,n)\) are true. The first is already done above. The second is also clear: \(F_1F_n + F_2F_{n+1} = F_n + F_{n+1} = F_{n+2} = F_{1 + n + 1}\).%
\par
Now fix \(m \gt 1\) and assume \(P(k,n)\) is true for all \(k \lt m\). That is we know that \(F_kF_n + F_{k+1}F_{n+1} = F_{k+n+1}\). We must consider \(F_{m}F_n + F_{m+1}F_{n+1}\) (and show it equals \(F_{m+n+1}\)). We have,%
\begin{align*}
F_{m}F_n + F_{m+1}F_{n+1} \amp = (F_{m-2} + F_{m-1})F_n + (F_{m-1} + F_{m})F_{n+1}\\
\amp = F_{m-2}F_n + F_{m-1}F_n + F_{m-1}F_{n+1} + F_{m}F_{n+1}\\
\amp = (F_{m-2}F_n + F_{m-1}F_{n+1}) + (F_{m-1}F_n + F_{m}F_{n+1})\\
\amp = F_{m-2 + n + 1} + F_{m-1 + n + 1} = F_{m + n + 1}\text{.}
\end{align*}
The last line is the result of applying the inductive hypothesis twice, once for \(k = m-2\) and once of \(k = m-1\).%
\par
Therefore by the principle of strong mathematical induction, \(P(m,n)\) is true for all \(m \ge 0\) (and our fixed value of \(n\)).%
\item{}This is enough, since we now have a method for getting to \(P(m,n)\) for any \(m,n \ge 0\). For example, we know that \(P(2,3)\) is true because we know that \(P(0,3)\) is true (from the first part), and then we know we can get up to \(m = 2\) by doing induction with \(n = 3\) fixed.%
\end{enumerate}
%
\end{divisionsolution}%
\begin{divisionsolution}{2.5.29}{}{p:exercise:gej}%
Given a square, you can cut the square into smaller squares by cutting along lines parallel to the sides of the original square (these lines do not need to travel the entire side length of the original square). For example, by cutting along the lines below, you will divide a square into 6 smaller squares:%
\begin{sidebyside}{1}{0.325}{0.325}{0}%
\begin{sbspanel}{0.35}%
\resizebox{\linewidth}{!}{%
\begin{tikzpicture}[scale=.75]
\draw (0,0) rectangle (3,3) (0,0) rectangle (2,2) (1,2) -- (1,3) (2,2) -- (2,3) (2,2) -- (3,2) (2,1)-- (3,1);
\end{tikzpicture}
}%
\end{sbspanel}%
\end{sidebyside}%
\par
Prove, using strong induction, that it is possible to cut a square into \(n\) smaller squares for any \(n \ge 6\).%
\par\smallskip%
\noindent\textbf{\blocktitlefont Hint}.\quad{}You will need three base cases.  This is a very good hint actually, as it suggests that to prove \(P(n)\) is true, you would want to use the fact that \(P(n-3)\) is true.  So somehow you need to increase the number of squares by 3.%
\par\smallskip%
\noindent\textbf{\blocktitlefont Solution}.\quad{}\begin{solutionproof}
For each \(n\), let \(P(n)\) be the statement ``You can divide a square into \(n\) smaller squares.'' We will prove \(P(n)\) is true for all \(n \ge 6\) using strong induction.%
\par
First note that you can divide a square into \(6\), \(7\), or \(8\) squares as follows:%
\begin{sidebyside}{3}{0.0166666666666667}{0.0166666666666667}{0.0333333333333333}%
\begin{sbspanel}{0.3}%
\resizebox{\linewidth}{!}{%
\begin{tikzpicture}[scale=.75]
\draw (0,0) rectangle (3,3) (0,0) rectangle (2,2) (1,2) -- (1,3) (2,2) -- (2,3) (2,2) -- (3,2) (2,1)-- (3,1);
\end{tikzpicture}
}%
\end{sbspanel}%
\begin{sbspanel}{0.3}%
\resizebox{\linewidth}{!}{%
\begin{tikzpicture}[scale=.75]
\draw (0,0) rectangle (3,3) (0,1.5) -- (3,1.5) (1.5,0) -- (1.5,3) (0.75,0) -- (0.75, 1.5) (0, 0.75) -- (1.5, 0.75);
\end{tikzpicture}
}%
\end{sbspanel}%
\begin{sbspanel}{0.3}%
\resizebox{\linewidth}{!}{%
\begin{tikzpicture}[scale=.57]
\draw (0,0) rectangle (4,4) (3,0) rectangle (4,1) rectangle (3,2) rectangle (4,3) rectangle (3,4) rectangle (2,3) rectangle (1,4) rectangle (0,3);
\end{tikzpicture}
}%
\end{sbspanel}%
\end{sidebyside}%
\par
Now consider an arbitrary \(n \lt 8\) and suppose you can divide a square into \(k\) squares, for any \(6 \le k \lt n\). Then in particular, you can divide a square into \(n-3\) squares. Take any one of the small squares and divide it into 4. This replaces one of the squares with 4 new ones, increasing the total by 3, giving you \(n\) squares.%
\par
Therefore, by the principle of strong mathematical induction, you can divide a square into \(n\) squares for any \(n \ge 6\).%
\end{solutionproof}
\end{divisionsolution}%
\section*{2.6 Chapter Summary}
\addcontentsline{toc}{section}{2.6 Chapter Summary}
\sectionmark{2.6 Chapter Summary}
\subsection*{Chapter Review}
\addcontentsline{toc}{subsection}{Chapter Review}
\begin{divisionsolution}{2.6.1}{}{p:exercise:Iec}%
Find the solution to the recurrence relation \(a_n = 3a_{n-1} + 4a_{n-2}\) with initial terms \(a_0 = 5\) and \(a_1 = 8\text{.}\)%
\par\smallskip%
\noindent\textbf{\blocktitlefont Solution}.\quad{}\(a_n = \frac{13}{5} 4^n + \frac{12}{5} (-1)^n\text{.}\)%
\end{divisionsolution}%
\begin{divisionsolution}{2.6.2}{}{p:exercise:oll}%
Solve the recurrence relation \(a_n = 3a_{n-1} + 10a_{n-2}\) with initial terms \(a_0 = 4\) and \(a_1 = 1\text{.}\)%
\par\smallskip%
\noindent\textbf{\blocktitlefont Answer}.\quad{}\(2.71429\!\left(-2\right)^{n}+1.28571\cdot 5^{n}\)%
\par\smallskip%
\noindent\textbf{\blocktitlefont Solution}.\quad{}\(a_n = \frac{19}{7}(-2)^n + \frac{9}{7}5^n\text{.}\)%
\end{divisionsolution}%
\begin{divisionsolution}{2.6.3}{}{p:exercise:Usu}%
Find \(3 + 7 + 11+ \cdots + 427\text{.}\)%
\par\smallskip%
\noindent\textbf{\blocktitlefont Answer}.\quad{}\(23005\)%
\par\smallskip%
\noindent\textbf{\blocktitlefont Solution}.\quad{}\(\frac{430\cdot 107}{2} = 23005\text{.}\)%
\end{divisionsolution}%
\begin{divisionsolution}{2.6.4}{}{p:exercise:AzD}%
Consider the sequence \(5, 11, 19, 29, 41, 55,\ldots\). Assume \(a_1 = 5\).%
\begin{enumerate}[label=(\alph*)]
\item{}Find a closed formula for \(a_n\), the \(n\)th term of the sequence, by writing each term as a sum of a sequence. Hint: first find \(a_0\), but ignore it when collapsing the sum. %
\item{}Find a closed formula again, this time using either polynomial fitting or the characteristic root technique (whichever is appropriate). Show your work. %
\item{}Find a closed formula once again, this time by recognizing the sequence as a modification to some well known sequence(s). Explain. %
\end{enumerate}
%
\end{divisionsolution}%
\begin{divisionsolution}{2.6.5}{}{p:exercise:gGM}%
Consider the sequence \(2, 6, 10, 14, \ldots, 4n + 6\text{.}\)%
\begin{enumerate}[label=(\alph*)]
\item{}How many terms are there in the sequence?%
\item{}What is the second-to-last term?%
\item{}Find the sum of all the terms in the sequence.%
\end{enumerate}
%
\par\smallskip%
\noindent\textbf{\blocktitlefont Answer 1}.\quad{}\(n+2\)%
\par\smallskip%
\noindent\textbf{\blocktitlefont Answer 2}.\quad{}\(4n+2\)%
\par\smallskip%
\noindent\textbf{\blocktitlefont Answer 3}.\quad{}\(\frac{\left(4n+8\right)\!\left(n+2\right)}{2}\)%
\par\smallskip%
\noindent\textbf{\blocktitlefont Solution}.\quad{}%
\begin{enumerate}[label=(\alph*)]
\item{}\(n+2\) terms.%
\item{}\(4n+2\text{.}\)%
\item{}\(\frac{(4n+8)(n+2)}{2}\text{.}\)%
\end{enumerate}
%
\end{divisionsolution}%
\begin{divisionsolution}{2.6.6}{}{p:exercise:MNV}%
Suppose the closed formula for a particular sequence is a degree 3 polynomial. What can you say about the closed formula for:%
\begin{enumerate}[label=(\alph*)]
\item{}The sequence of partial sums.%
\item{}The sequence of second differences.%
\end{enumerate}
%
\par\smallskip%
\noindent\textbf{\blocktitlefont Solution}.\quad{}%
\begin{enumerate}[label=(\alph*)]
\item{}The sequence of partial sums will be a degree 4 polynomial (its sequence of differences will be the original sequence).%
\item{}The sequence of second differences will be a degree 1 polynomial - an arithmetic sequence.%
\end{enumerate}
%
\end{divisionsolution}%
\begin{divisionsolution}{2.6.7}{}{p:exercise:sVe}%
Consider the sequence given recursively by \(a_1 = 4\), \(a_2 = 6\) and \(a_n = a_{n-1} + a_{n-2}\).%
\begin{enumerate}[label=(\alph*)]
\item{}Write out the first 6 terms of the sequence.%
\item{}Could the closed formula for \(a_n\) be a polynomial? Explain.%
\end{enumerate}
%
\par\smallskip%
\noindent\textbf{\blocktitlefont Solution}.\quad{}%
\begin{enumerate}[label=(\alph*)]
\item{}\(4, 6, 10, 16, 26, 42, \ldots\).%
\item{}No, taking differences gives the original sequence back, so the differences will never be constant.%
\end{enumerate}
%
\end{divisionsolution}%
\begin{divisionsolution}{2.6.8}{}{p:exercise:Zcn}%
Consider the sequence given by \(a_n = 2\cdot 5^{n-1}\text{.}\)%
\begin{enumerate}[label=(\alph*)]
\item{}Find the first 4 terms of the sequence.%
\par
What sort of sequence is this?%
\par
\quad(\begin{itemize*}[label=$\square$,leftmargin=3em,itemjoin=\hspace{1em}]
\item{}arithmetic%

\item{}geometric%

\item{}neither%

\end{itemize*})\quad
%
\item{}Find the \emph{sum} of the first 25 terms. That is, compute \(\sum_{k=1}^{25}a_k\text{.}\)%
\end{enumerate}
%
\par\smallskip%
\noindent\textbf{\blocktitlefont Answer 1}.\quad{}\(\text{geometric}\)%
\par\smallskip%
\noindent\textbf{\blocktitlefont Answer 2}.\quad{}\(\frac{2-2\cdot 5^{25}}{-4}\)%
\par\smallskip%
\noindent\textbf{\blocktitlefont Solution}.\quad{}%
\begin{enumerate}[label=(\alph*)]
\item{}\(2, 10, 50, 250, \ldots\) The sequence is geometric.%
\item{}\(\frac{2 - 2\cdot 5^{25}}{-4}\text{.}\)%
\end{enumerate}
%
\end{divisionsolution}%
\begin{divisionsolution}{2.6.9}{}{p:exercise:Fjw}%
The in song \emph{The Twelve Days of Christmas}, my true love gave to me first 1 gift, then 2 gifts and 1 gift, then 3 gifts, 2 gifts and 1 gift, and so on. How many gifts did my true love give me all together during the twelve days?%
\end{divisionsolution}%
\begin{divisionsolution}{2.6.10}{}{p:exercise:lqF}%
Use polynomial fitting to find a closed formula for the sequence \((a_n)_{n\ge 1}\text{:}\)%
\par
%
\begin{equation*}
4, 11, 20, 31, 44, \ldots
\text{.}
\end{equation*}
%
\par\smallskip%
\noindent\textbf{\blocktitlefont Solution}.\quad{}\(a_n = n^2 + 4n - 1\text{.}\)%
\end{divisionsolution}%
\begin{divisionsolution}{2.6.11}{}{p:exercise:RxO}%
The sequence \((a_n)_{n \ge 1}\) starts \(-1, 0, 2, 5, 9, 14\ldots\) and has closed formula \(a_n = \dfrac{(n+1)(n-2)}{2}\text{.}\) Use this fact to find a closed formula for the sequence \((b_n)_{n \ge 1}\) which starts \(4, 10, 18, 28, 40, \ldots\text{.}\)%
\par\smallskip%
\noindent\textbf{\blocktitlefont Solution}.\quad{}\(b_n = (n+3)n\text{.}\)%
\end{divisionsolution}%
\begin{divisionsolution}{2.6.12}{}{p:exercise:xEX}%
Your magic chocolate bunnies \index{magic chocolate bunnies} reproduce like rabbits: every large bunny produces 2 new mini bunnies each day, and each day every mini bunny born the previous day grows into a large bunny. Assume you start with 2 mini bunnies and no bunny ever dies (or gets eaten).%
\begin{enumerate}[label=(\alph*)]
\item{}Write out the first few terms of the sequence.%
\item{}Give a recursive definition of the sequence and explain why it is correct.%
\item{}Find a closed formula for the \(n\)th term of the sequence.%
\end{enumerate}
%
\par\smallskip%
\noindent\textbf{\blocktitlefont Solution}.\quad{}%
\begin{enumerate}[label=(\alph*)]
\item{}On the first day, your 2 mini bunnies become 2 large bunnies. On day 2, your two large bunnies produce 4 mini bunnies. On day 3, you have 4 mini bunnies (produced by your 2 large bunnies) plus 6 large bunnies (your original 2 plus the 4 newly matured bunnies). On day 4, you will have \(12\) mini bunnies (2 for each of the 6 large bunnies) plus 10 large bunnies (your previous 6 plus the 4 newly matured). The sequence of total bunnies is \(2, 2, 6, 10, 22, 42\ldots\) starting with \(a_0 = 2\) and \(a_1 = 2\).%
\item{}\(a_n = a_{n-1} + 2a_{n-2}\). This is because the number of bunnies is equal to the number of bunnies you had the previous day (both mini and large) plus 2 times the number you had the day before that (since all bunnies you had 2 days ago are now large and producing 2 new bunnies each).%
\item{}Using the characteristic root technique, we find \(a_n = a2^n + b(-1)^n\), and we can find \(a\) and \(b\) to give \(a_n = \frac{4}{3}2^n + \frac{2}{3}(-1)^n\).%
\end{enumerate}
%
\end{divisionsolution}%
\begin{divisionsolution}{2.6.13}{}{p:exercise:dMg}%
Consider the sequence of partial sums of \emph{squares} of Fibonacci numbers: \(F_1^2\), \(F_1^2 + F_2^2\), \(F_1^2 + F_2^2 + F_3^2, \ldots\).  The sequences starts \(1, 2, 6, 15, 40,\ldots\)%
\begin{enumerate}[label=(\alph*)]
\item{}Guess a formula for the \(n\)th partial sum, in terms of Fibonacci numbers.  Hint: write each term as a product.%
\item{}Prove your formula is correct by mathematical induction.%
\item{}Explain what this problem has to do with the following picture:%
\begin{sidebyside}{1}{0.375}{0.375}{0}%
\begin{sbspanel}{0.25}%
\resizebox{\linewidth}{!}{%
\begin{tikzpicture}
  \draw (0,0) rectangle (13,8) rectangle (8,0) rectangle (13,5) rectangle (10,8) rectangle (8,6) rectangle (9,5);
\end{tikzpicture}
}%
\end{sbspanel}%
\end{sidebyside}%
\end{enumerate}
%
\end{divisionsolution}%
\begin{divisionsolution}{2.6.14}{}{p:exercise:JTp}%
Prove the following statements by mathematical induction:%
\begin{enumerate}[label=(\alph*)]
\item{}\(n! \lt n^n\) for \(n \ge 2\)%
\item{}\(\d\frac{1}{1\cdot 2} + \frac{1}{2\cdot 3} +\frac{1}{3\cdot 4}+\cdots + \frac{1}{n\cdot(n+1)} = \d\frac{n}{n+1}\) for all \(n \in \Z^+\).%
\item{}\(4^n - 1\) is a multiple of 3 for all \(n \in \N\).%
\item{}The \emph{greatest} amount of postage you \emph{cannot} make exactly using 4 and 9 cent stamps is 23 cents.%
\item{}Every even number squared is divisible by 4.%
\end{enumerate}
%
\par\smallskip%
\noindent\textbf{\blocktitlefont Hint}.\quad{}%
\begin{enumerate}[label=(\alph*)]
\item{}Hint: \((n+1)^{n+1} > (n+1) \cdot n^{n}\).%
\item{}Hint: This should be similar to the other sum proofs. The last bit comes down to adding fractions.%
\item{}Hint: Write \(4^{k+1} - 1 = 4\cdot 4^k - 4 + 3\).%
\item{}Hint: one 9-cent stamp is 1 more than two 4-cent stamps, and seven 4-cent stamps is 1 more than three 9-cent stamps.%
\item{}Careful to actually use induction here. The base case: \(2^2 = 4\). The inductive case: assume \((2n)^2\) is divisible by 4 and consider \((2n+2)^2 = (2n)^2 + 4n + 4\). This is divisible by 4 because \(4n +4\) clearly is, and by our inductive hypothesis, so is \((2n)^2\).%
\end{enumerate}
%
\end{divisionsolution}%
\begin{divisionsolution}{2.6.15}{}{p:exercise:qay}%
Prove \(1^3 + 2^3 + 3^3 + \cdots + n^3 = \left(\frac{n(n+1)}{2}\right)^2\) holds for all \(n \ge 1\), by mathematical induction.%
\par\smallskip%
\noindent\textbf{\blocktitlefont Hint}.\quad{}This is a straight forward induction proof. Note you will need to simplify \(\left(\frac{n(n+1)}{2}\right)^2 + (n+1)^3\) and get \(\left(\frac{(n+1)(n+2)}{2}\right)^2\).%
\end{divisionsolution}%
\begin{divisionsolution}{2.6.16}{}{p:exercise:WhH}%
Suppose \(a_0 = 1\), \(a_1 = 1\) and \(a_n = 3a_{n-1} - 2a_{n-1}\). Prove, using strong induction, that \(a_n = 1\) for all \(n\).%
\par\smallskip%
\noindent\textbf{\blocktitlefont Hint}.\quad{}There are two base cases \(P(0)\) and \(P(1)\). Then, for the inductive case, assume \(P(k)\) is true for all \(k \lt n\). This allows you to assume \(a_{n-1} = 1\) and \(a_{n-2} = 1\). Apply the recurrence relation.%
\end{divisionsolution}%
\begin{divisionsolution}{2.6.17}{}{p:exercise:CoQ}%
Prove using induction that every set containing \(n\) elements has \(2^n\) different subsets for any \(n \ge 1\).%
\par\smallskip%
\noindent\textbf{\blocktitlefont Solution}.\quad{}Let \(P(n)\) be the statement, ``every set containing \(n\) elements has \(2^n\) different subsets.'' We will show \(P(n)\) is true for all \(n \ge 1\). Base case: Any set with 1 element \(\{a\}\) has exactly 2 subsets: the empty set and the set itself. Thus the number of subsets is \(2= 2^1\). Thus \(P(1)\) is true. Inductive case: Suppose \(P(k)\) is true for some arbitrary \(k \ge 1\). Thus every set containing exactly \(k\) elements has \(2^k\) different subsets. Now consider a set containing \(k+1\) elements: \(A = \{a_1, a_2, \ldots, a_k, a_{k+1}\}\). Any subset of \(A\) must either contain \(a_{k+1}\) or not. In other words, a subset of \(A\) is just a subset of \(\{a_1, a_2,\ldots,
a_k\}\) with or without \(a_{k+1}\). Thus there are \(2^k\) subsets of \(A\) which contain \(a_{k+1}\) and another \(2^{k+1}\) subsets of \(A\) which do not contain \(a^{k+1}\). This gives a total of \(2^k + 2^k = 2\cdot 2^k = 2^{k+1}\) subsets of \(A\). But our choice of \(A\) was arbitrary, so this works for any subset containing \(k+1\) elements, so \(P(k+1)\) is true. Therefore, by the principle of mathematical induction, \(P(n)\) is true for all \(n \ge 1\).%
\end{divisionsolution}%
\chapter*{3 Symbolic Logic and Proofs}
\addcontentsline{toc}{chapter}{3 Symbolic Logic and Proofs}
\chaptermark{3 Symbolic Logic and Proofs}
\section*{3.1 Propositional Logic}
\addcontentsline{toc}{section}{3.1 Propositional Logic}
\sectionmark{3.1 Propositional Logic}
\subsection*{Exercises}
\addcontentsline{toc}{subsection}{Exercises}
\begin{divisionsolution}{3.1.1}{}{p:exercise:iCQ}%
Consider the statement about a party, ``If it's your birthday or there will be cake, then there will be cake.''%
\begin{enumerate}[label=(\alph*)]
\item{}Translate the above statement into symbols. Clearly state which statement is \(P\) and which is \(Q\).%
\item{}Make a truth table for the statement.%
\item{}Assuming the statement is true, what (if anything) can you conclude if there will be cake?%
\item{}Assuming the statement is true, what (if anything) can you conclude if there will not be cake?%
\item{}Suppose you found out that the statement was a lie. What can you conclude?%
\end{enumerate}
%
\par\smallskip%
\noindent\textbf{\blocktitlefont Solution}.\quad{}%
\begin{enumerate}[label=(\alph*)]
\item{}\(P\): it's your birthday; \(Q\): there will be cake. \((P \vee Q) \imp Q\)%
\item{}Hint: you should get three T's and one F.%
\item{}Only that there will be cake.%
\item{}It's NOT your birthday!%
\item{}It's your birthday, but the cake is a lie.%
\end{enumerate}
%
\end{divisionsolution}%
\begin{divisionsolution}{3.1.2}{}{p:exercise:OJZ}%
Make a truth table for the statement \((P \vee Q) \imp (P \wedge Q)\).%
\par\smallskip%
\noindent\textbf{\blocktitlefont Solution}.\quad{}\begin{sidebyside}{1}{0}{0}{0}%
\begin{sbspanel}{1}%
{\centering%
{\tabularfont%
\begin{tabular}{cAcBc}
\(P\)&\(Q\)&\((P \vee Q) \imp (P \wedge Q)\)\tabularnewline\hrulethin
T&T&T\tabularnewline[0pt]
T&F&F\tabularnewline[0pt]
F&T&F\tabularnewline[0pt]
F&F&T
\end{tabular}
}%
\par}
\end{sbspanel}%
\end{sidebyside}%
\end{divisionsolution}%
\begin{divisionsolution}{3.1.3}{}{p:exercise:uRi}%
Make a truth table for the statement \(\neg P \wedge (Q \imp P)\). What can you conclude about \(P\) and \(Q\) if you know the statement is true?%
\par\smallskip%
\noindent\textbf{\blocktitlefont Solution}.\quad{}\begin{sidebyside}{1}{0}{0}{0}%
\begin{sbspanel}{1}%
{\centering%
{\tabularfont%
\begin{tabular}{cAcBc}
\(P\)&\(Q\)&\(\neg P \wedge (Q \imp P)\)\tabularnewline\hrulethin
T&T&F\tabularnewline[0pt]
T&F&F\tabularnewline[0pt]
F&T&F\tabularnewline[0pt]
F&F&T
\end{tabular}
}%
\par}
\end{sbspanel}%
\end{sidebyside}%
\par
If the statement is true, then both \(P\) and \(Q\) are false.%
\end{divisionsolution}%
\begin{divisionsolution}{3.1.4}{}{p:exercise:aYr}%
Make a truth table for the statement \(\neg P \imp (Q \wedge R)\).%
\par\smallskip%
\noindent\textbf{\blocktitlefont Hint}.\quad{}Like above, only now you will need 8 rows instead of just 4.%
\end{divisionsolution}%
\begin{divisionsolution}{3.1.5}{}{p:exercise:HfA}%
Geoff Poshingten is out at a fancy pizza joint, and decides to order a calzone. When the waiter asks what he would like in it, he replies, ``I want either pepperoni or sausage. Also, if I have sausage, then I must also include quail. Oh, and if I have pepperoni or quail then I must also have ricotta cheese.''%
\begin{enumerate}[label=(\alph*)]
\item{}Translate Geoff's order into logical symbols.%
\item{}The waiter knows that Geoff is either a liar or a truth-teller (so either everything he says is false, or everything is true). Which is it?%
\item{}What, if anything, can the waiter conclude about the ingredients in Geoff's desired calzone?%
\end{enumerate}
%
\par\smallskip%
\noindent\textbf{\blocktitlefont Hint}.\quad{}You should write down three statements using the symbols \(P, Q, R, S\).  If Geoff is a truth-teller, then all three statements would be true.  If he was a liar, then all three statements would be false.  But in either case, we don't yet know whether the four atomic statements are true or false, since he hasn't said them by themselves.%
\par
A truth table might help, although is probably not entirely necessary.%
\end{divisionsolution}%
\begin{divisionsolution}{3.1.6}{}{p:exercise:nmJ}%
Determine whether the following two statements are logically equivalent: \(\neg(P \imp Q)\) and \(P \wedge \neg Q\). Explain how you know you are correct.%
\par\smallskip%
\noindent\textbf{\blocktitlefont Solution}.\quad{}Make a truth table for each and compare. The statements are logically equivalent.%
\end{divisionsolution}%
\begin{divisionsolution}{3.1.7}{}{p:exercise:TtS}%
Are the statements \(P \imp (Q\vee R)\) and \((P \imp Q) \vee (P \imp R)\) logically equivalent?%
\end{divisionsolution}%
\begin{divisionsolution}{3.1.8}{}{p:exercise:zBb}%
Simplify the following statements (so that negation only appears right before variables).%
\begin{enumerate}[label=(\alph*)]
\item{}\(\neg(P \imp \neg Q)\).%
\item{}\((\neg P \vee \neg Q) \imp \neg (\neg Q \wedge R)\).%
\item{}\(\neg((P \imp \neg Q) \vee \neg (R \wedge \neg R))\).%
\item{}It is false that if Sam is not a man then Chris is a woman, and that Chris is not a woman.%
\end{enumerate}
%
\par\smallskip%
\noindent\textbf{\blocktitlefont Solution}.\quad{}%
\begin{enumerate}[label=(\alph*)]
\item{}\(P \wedge Q\).%
\item{}\((\neg P \vee \neg R) \imp (Q \vee \neg R)\) or, replacing the implication with a disjunction first: \((P \wedge Q) \vee (Q \vee \neg R)\).%
\item{}\((P \wedge Q) \wedge (R \wedge \neg R)\). This is necessarily false, so it is also equivalent to \(P \wedge \neg P\).%
\item{}Either Sam is a woman and Chris is a man, or Chris is a woman.%
\end{enumerate}
%
\end{divisionsolution}%
\begin{divisionsolution}{3.1.9}{}{p:exercise:fIk}%
Use De Morgan's Laws, and any other logical equivalence facts you know to simplify the following statements. Show all your steps. Your final statements should have negations only appear directly next to the sentence variables or predicates (\(P\), \(Q\), \(E(x)\), etc.), and no double negations. It would be a good idea to use only conjunctions, disjunctions, and negations.%
\begin{enumerate}[label=(\alph*)]
\item{}\(\neg((\neg P \wedge Q) \vee \neg(R \vee \neg S))\).%
\item{}\(\neg((\neg P \imp \neg Q) \wedge (\neg Q \imp R))\) (careful with the implications).%
\item{}For both parts above, verify your answers are correct using truth tables. That is, use a truth table to check that the given statement and your proposed simplification are actually logically equivalent.%
\end{enumerate}
%
\par\smallskip%
\noindent\textbf{\blocktitlefont Solution}.\quad{}%
\begin{enumerate}[label=(\alph*)]
\item{}\(\neg((\neg P \wedge Q) \vee \neg(R \vee \neg S))\)%
\par
\(\neg(\neg P \wedge Q) \wedge \neg\neg(R \vee \neg S)\) by De Morgan's law.%
\par
\(\neg(\neg P \wedge Q) \wedge (R \vee \neg S)\) by double negation.%
\par
\((\neg\neg P \vee \neg Q) \wedge (R \vee \neg S)\) by De Morgan's law.%
\par
\((P \vee \neg Q) \wedge (R \vee \neg S)\) by double negation.%
\item{}We will need to convert the implications to disjunctions so we can apply De Morgan's law:%
\par
\(\neg((\neg P \imp \neg Q) \wedge (\neg Q \imp R))\)%
\par
\(\neg((\neg \neg P \vee \neg Q) \wedge (\neg\neg Q \vee R))\) by implication\slash{}disjunction equivalence.%
\par
\(\neg((P \vee \neg Q) \wedge (Q \vee R))\) by double negation.%
\par
\(\neg(P \vee \neg Q) \vee \neg (Q \vee R)\) by De Morgan's law.%
\par
\((\neg P \wedge \neg \neg Q) \vee (\neg Q \wedge \neg R)\) by De Morgan's law.%
\par
\((\neg P \wedge Q) \vee (\neg Q \wedge \neg R)\) by double negation.%
\item{}For each truth table, the columns for the original statement and the simplified statement must be identical. They are. For purposes of checking your work, here are the final columns for the two parts above:%
\begin{sidebyside}{1}{0.2}{0.2}{0}%
\begin{sbspanel}{0.6}%
{\centering%
{\tabularfont%
\begin{tabular}{ccccc}
\multicolumn{1}{cB}{\(P\)}&\multicolumn{1}{cB}{\(Q\)}&\multicolumn{1}{cB}{\(R\)}&\multicolumn{1}{cB}{(a)}&(b)\tabularnewline\hrulemedium
\multicolumn{1}{cB}{T}&\multicolumn{1}{cB}{T}&\multicolumn{1}{cB}{T}&\multicolumn{1}{cB}{T}&F\tabularnewline[0pt]
\multicolumn{1}{cB}{T}&\multicolumn{1}{cB}{T}&\multicolumn{1}{cB}{F}&\multicolumn{1}{cB}{F}&F\tabularnewline[0pt]
\multicolumn{1}{cB}{T}&\multicolumn{1}{cB}{F}&\multicolumn{1}{cB}{T}&\multicolumn{1}{cB}{T}&F\tabularnewline[0pt]
\multicolumn{1}{cB}{T}&\multicolumn{1}{cB}{F}&\multicolumn{1}{cB}{F}&\multicolumn{1}{cB}{T}&T\tabularnewline[0pt]
\multicolumn{1}{cB}{F}&\multicolumn{1}{cB}{T}&\multicolumn{1}{cB}{T}&\multicolumn{1}{cB}{F}&T\tabularnewline[0pt]
\multicolumn{1}{cB}{F}&\multicolumn{1}{cB}{T}&\multicolumn{1}{cB}{F}&\multicolumn{1}{cB}{F}&T\tabularnewline[0pt]
\multicolumn{1}{cB}{F}&\multicolumn{1}{cB}{F}&\multicolumn{1}{cB}{T}&\multicolumn{1}{cB}{T}&F\tabularnewline[0pt]
\multicolumn{1}{cB}{F}&\multicolumn{1}{cB}{F}&\multicolumn{1}{cB}{F}&\multicolumn{1}{cB}{T}&T
\end{tabular}
}%
\par}
\end{sbspanel}%
\end{sidebyside}%
\end{enumerate}
%
\end{divisionsolution}%
\begin{divisionsolution}{3.1.10}{}{p:exercise:LPt}%
Consider the statement, ``If a number is triangular or square, then it is not prime''%
\begin{enumerate}[label=(\alph*)]
\item{}Make a truth table for the statement \((T \vee S) \imp \neg P\).%
\item{}If you believed the statement was \emph{false}, what properties would a counterexample need to possess? Explain by referencing your truth table.%
\item{}If the statement were true, what could you conclude about the number 5657, which is definitely prime? Again, explain using the truth table.%
\end{enumerate}
%
\par\smallskip%
\noindent\textbf{\blocktitlefont Hint}.\quad{}%
\begin{enumerate}[label=(\alph*)]
\item{}There will be three rows in which the statement is false.%
\item{}Consider the three rows that evaluate to false and say what the truth values of \(T\), \(S\), and \(P\) are there.%
\item{}You are looking for a row in which \(P\) is true, and the whole statement is true.%
\end{enumerate}
%
\par\smallskip%
\noindent\textbf{\blocktitlefont Solution}.\quad{}%
\begin{enumerate}[label=(\alph*)]
\item{}\begin{sidebyside}{1}{0.25}{0.25}{0}%
\begin{sbspanel}{0.5}%
{\centering%
{\tabularfont%
\begin{tabular}{cccc}
\multicolumn{1}{cB}{\(T\)}&\multicolumn{1}{cB}{\(S\)}&\multicolumn{1}{cB}{\(P\)}&\((T\vee S) \imp \neg P\)\tabularnewline\hrulemedium
\multicolumn{1}{cB}{T}&\multicolumn{1}{cB}{T}&\multicolumn{1}{cB}{T}&F\tabularnewline[0pt]
\multicolumn{1}{cB}{T}&\multicolumn{1}{cB}{T}&\multicolumn{1}{cB}{F}&T\tabularnewline[0pt]
\multicolumn{1}{cB}{T}&\multicolumn{1}{cB}{F}&\multicolumn{1}{cB}{T}&F\tabularnewline[0pt]
\multicolumn{1}{cB}{T}&\multicolumn{1}{cB}{F}&\multicolumn{1}{cB}{F}&T\tabularnewline[0pt]
\multicolumn{1}{cB}{F}&\multicolumn{1}{cB}{T}&\multicolumn{1}{cB}{T}&F\tabularnewline[0pt]
\multicolumn{1}{cB}{F}&\multicolumn{1}{cB}{T}&\multicolumn{1}{cB}{F}&T\tabularnewline[0pt]
\multicolumn{1}{cB}{F}&\multicolumn{1}{cB}{F}&\multicolumn{1}{cB}{T}&T\tabularnewline[0pt]
\multicolumn{1}{cB}{F}&\multicolumn{1}{cB}{F}&\multicolumn{1}{cB}{F}&T
\end{tabular}
}%
\par}
\end{sbspanel}%
\end{sidebyside}%
%
\item{}There are three cases in which the statement is false: rows 1, 3 and 5. So one way to prove this was false would be to find a number that was triangular, square and prime (row 1). Or you could find a number that was triangular, not square, and prime (row 3) or one that is not triangular, is square and is not prime (row 5). In fact, since no square number is prime, we would need to do the second one (3 is both triangular and prime, but in fact this is the only one).%
\item{}Here we have a prime number, so we need to look at rows in which \(P\) is true. We also need to statement to be true. There is only one row with both these properties: row 7. And here we see that \(T\) and \(S\) are both false. So we would know that 5657 is neither triangular nor square.%
\end{enumerate}
%
\end{divisionsolution}%
\begin{divisionsolution}{3.1.11}{}{p:exercise:rWC}%
Tommy Flanagan was telling you what he ate yesterday afternoon. He tells you, ``I had either popcorn or raisins. Also, if I had cucumber sandwiches, then I had soda. But I didn't drink soda or tea.'' Of course you know that Tommy is the worlds worst liar, and everything he says is false. What did Tommy eat?%
\par
Justify your answer by writing all of Tommy's statements using sentence variables (\(P, Q, R, S, T\)), taking their negations, and using these to deduce what Tommy actually ate.%
\par\smallskip%
\noindent\textbf{\blocktitlefont Hint}.\quad{}Write down three statements, and then take the negation of each (since he is a liar).  You should find that Tommy ate one item and drank one item.  (\(Q\) is for cucumber sandwiches.)%
\par\smallskip%
\noindent\textbf{\blocktitlefont Solution}.\quad{}Let \(P\) be the statement, ``I had popcorn,'' \(Q\) be the statement, ``I had cucumber sandwiches,'' \(R\) be the statement, ``I had raisins,'' \(S\) be, ``I had soda,'' and \(T\) be, ``I had tea.'' Then the statements made by Tommy are:%
\begin{equation*}
P \vee R \qquad Q \imp S \qquad \neg(S \vee T)
\end{equation*}
%
\par
We need the negation of all of these. Thus what is true is:%
\begin{equation*}
\neg P \wedge \neg R \qquad Q \wedge \neg S \qquad S \vee T
\end{equation*}
%
\par
From the first two statements we can conclude that Tommy did not eat popcorn, did not eat raisins, did eat cucumber sandwiches and did not drink soda. From the last statement \(S \vee T\) and the fact that we know \(\neg S\) we can conclude \(T\), so Tommy did drink tea.%
\end{divisionsolution}%
\begin{divisionsolution}{3.1.12}{}{p:exercise:YdL}%
Determine if the following deduction rule is valid:%
\begin{sidebyside}{1}{0}{0}{0}%
\begin{sbspanel}{1}%
{\centering%
{\tabularfont%
\begin{tabular}{cc}
&\(P \vee Q\)\tabularnewline[0pt]
&\(\neg P\)\tabularnewline\hrulethin
\(\therefore\)&\(Q\)
\end{tabular}
}%
\par}
\end{sbspanel}%
\end{sidebyside}%
\par\smallskip%
\noindent\textbf{\blocktitlefont Solution}.\quad{}The deduction rule is valid. To see this, make a truth table which contains \(P \vee Q\) and \(\neg P\) (and \(P\) and \(Q\) of course). Look at the truth value of \(Q\) in each of the rows that have \(P \vee Q\) and \(\neg P\) true.%
\end{divisionsolution}%
\begin{divisionsolution}{3.1.13}{}{p:exercise:EkU}%
Determine if the following is a valid deduction rule:%
\begin{sidebyside}{1}{0}{0}{0}%
\begin{sbspanel}{1}%
{\centering%
{\tabularfont%
\begin{tabular}{cc}
&\(P \imp (Q \vee R)\)\tabularnewline[0pt]
&\(\neg(P \imp Q)\)\tabularnewline\hrulethin
\(\therefore\)&\(R\)
\end{tabular}
}%
\par}
\end{sbspanel}%
\end{sidebyside}%
\end{divisionsolution}%
\begin{divisionsolution}{3.1.14}{}{p:exercise:ksd}%
Determine if the following is a valid deduction rule:%
\begin{sidebyside}{1}{0}{0}{0}%
\begin{sbspanel}{1}%
{\centering%
{\tabularfont%
\begin{tabular}{cc}
&\((P \wedge Q) \imp R\)\tabularnewline[0pt]
&\(\neg P \vee \neg Q\)\tabularnewline\hrulethin
\(\therefore\)&\(\neg R\)
\end{tabular}
}%
\par}
\end{sbspanel}%
\end{sidebyside}%
\end{divisionsolution}%
\begin{divisionsolution}{3.1.15}{}{p:exercise:Qzm}%
Can you chain implications together? That is, if \(P \imp Q\) and \(Q \imp R\), does that means the \(P \imp R\)? Can you chain more implications together? Let's find out:%
\begin{enumerate}[label=(\alph*)]
\item{}Prove that the following is a valid deduction rule:%
\begin{sidebyside}{1}{0}{0}{0}%
\begin{sbspanel}{1}%
{\centering%
{\tabularfont%
\begin{tabular}{cc}
&\(P \imp Q\)\tabularnewline[0pt]
&\(Q \imp R\)\tabularnewline\hrulethin
\(\therefore\)&\(P \imp R\)
\end{tabular}
}%
\par}
\end{sbspanel}%
\end{sidebyside}%
\item{}Prove that the following is a valid deduction rule for any \(n \ge 2\):%
\begin{sidebyside}{1}{0}{0}{0}%
\begin{sbspanel}{1}%
{\centering%
{\tabularfont%
\begin{tabular}{cc}
&\(P_1 \imp P_2\)\tabularnewline[0pt]
&\(P_2 \imp P_3\)\tabularnewline[0pt]
&\(\vdots\)\tabularnewline[0pt]
&\(P_{n-1} \imp P_n\)\tabularnewline\hrulethin
\(\therefore\)&\(P_1 \imp P_n\).
\end{tabular}
}%
\par}
\end{sbspanel}%
\end{sidebyside}%
\par
I suggest you don't go through the trouble of writing out a \(2^n\) row truth table. Instead, you should use part (a) and mathematical induction. %
\end{enumerate}
%
\par\smallskip%
\noindent\textbf{\blocktitlefont Hint}.\quad{}For the second part, you can inductively assume that from the first \(n-2\) implications you can deduce \(P_1 \imp P_{n-1}\).  Then you are back in the case in part (a) again.%
\end{divisionsolution}%
\begin{divisionsolution}{3.1.16}{}{p:exercise:wGv}%
We can also simplify statements in predicate logic using our rules for passing negations over quantifiers, and then applying propositional logical equivalence to the ``inside'' propositional part. Simplify the statements below (so negation appears only directly next to predicates).%
\begin{enumerate}[label=(\alph*)]
\item{}\(\neg \exists x \forall y (\neg O(x) \vee E(y))\).%
\item{}\(\neg \forall x \neg \forall y \neg(x \lt y \wedge \exists z (x \lt z \vee y \lt z))\).%
\item{}There is a number \(n\) for which no other number is either less \(n\) than or equal to \(n\).%
\item{}It is false that for every number \(n\) there are two other numbers which \(n\) is between.%
\end{enumerate}
%
\par\smallskip%
\noindent\textbf{\blocktitlefont Solution}.\quad{}%
\begin{enumerate}[label=(\alph*)]
\item{}\(\forall x \exists y (O(x) \wedge \neg E(y))\).%
\item{}\(\exists x \forall y (x \ge y \vee \forall z (x \ge z \wedge y \ge z))\).%
\item{}There is a number \(n\) for which every other number is strictly greater than \(n\).%
\item{}There is a number \(n\) which is not between any other two numbers.%
\end{enumerate}
%
\end{divisionsolution}%
\begin{divisionsolution}{3.1.17}{}{p:exercise:cNE}%
Simplify the statements below to the point that negation symbols occur only directly next to predicates.%
\begin{enumerate}[label=(\alph*)]
\item{}\(\neg \forall x \forall y (x \lt y \vee y \lt x)\).%
\item{}\(\neg(\exists x P(x) \imp \forall y P(y))\).%
\end{enumerate}
%
\end{divisionsolution}%
\begin{divisionsolution}{3.1.18}{}{p:exercise:IUN}%
Simplifying negations will be especially useful in the next section when we try to prove a statement by considering what would happen if it were false.  For each statement below, write the \emph{negation} of the statement as simply as possible.  Don't just say, ``it is false that \textellipsis{}''.%
\begin{enumerate}[label=(\alph*)]
\item{}Every number is either even or odd.%
\item{}There is a sequence that is both arithmetic and geometric.%
\item{}For all numbers \(n\), if \(n\) is prime, then \(n+3\) is not prime.%
\end{enumerate}
%
\par\smallskip%
\noindent\textbf{\blocktitlefont Hint}.\quad{}It might help to translate the statements into symbols and then use the formulaic rules to simplify negations (i.e., rules for quantifiers and De Morgan's laws).  After simplifying, you should get \(\forall x(\neg E(x) \wedge \neg O(x))\), for the first one, for example.  Then translate this back into English.%
\end{divisionsolution}%
\begin{divisionsolution}{3.1.19}{}{p:exercise:pbW}%
Suppose \(P\) and \(Q\) are (possibly molecular) propositional statements. Prove that \(P\) and \(Q\) are logically equivalent if any only if \(P \iff Q\) is a tautology.%
\par\smallskip%
\noindent\textbf{\blocktitlefont Hint}.\quad{}What do these concepts mean in terms of truth tables?%
\end{divisionsolution}%
\begin{divisionsolution}{3.1.20}{}{p:exercise:Vjf}%
Suppose \(P_1, P_2, \ldots, P_n\) and \(Q\) are (possibly molecular) propositional statements. Suppose further that%
\begin{sidebyside}{1}{0}{0}{0}%
\begin{sbspanel}{1}%
{\centering%
{\tabularfont%
\begin{tabular}{ll}
&\(P_1\)\tabularnewline[0pt]
&\(P_2\)\tabularnewline[0pt]
&\(\vdots\)\tabularnewline[0pt]
&\(P_n\)\tabularnewline\hrulethin
\(\therefore\)&\(Q\)
\end{tabular}
}%
\par}
\end{sbspanel}%
\end{sidebyside}%
\par
is a valid deduction rule. Prove that the statement%
\begin{equation*}
(P_1 \wedge P_2 \wedge \cdots \wedge P_n) \imp Q
\end{equation*}
is a tautology.%
\end{divisionsolution}%
\section*{3.2 Proofs}
\addcontentsline{toc}{section}{3.2 Proofs}
\sectionmark{3.2 Proofs}
\subsection*{Exercises}
\addcontentsline{toc}{subsection}{Exercises}
\begin{divisionsolution}{3.2.1}{}{p:exercise:ENx}%
Consider the statement ``for all integers \(a\) and \(b\), if \(a + b\) is even, then \(a\) and \(b\) are even''%
\begin{enumerate}[label=(\alph*)]
\item{}Write the contrapositive of the statement.%
\item{}Write the converse of the statement.%
\item{}Write the negation of the statement.%
\item{}Is the original statement true or false? Prove your answer.%
\item{}Is the contrapositive of the original statement true or false? Prove your answer.%
\item{}Is the converse of the original statement true or false? Prove your answer.%
\item{}Is the negation of the original statement true or false? Prove your answer.%
\end{enumerate}
%
\par\smallskip%
\noindent\textbf{\blocktitlefont Solution}.\quad{}%
\begin{enumerate}[label=(\alph*)]
\item{}For all integers \(a\) and \(b\), if \(a\) or \(b\) is not even, then \(a+b\) is not even.%
\item{}For all integers \(a\) and \(b\), if \(a\) and \(b\) are even, then \(a+b\) is even.%
\item{}There are numbers \(a\) and \(b\) such that \(a+b\) is even but \(a\) and \(b\) are not both even.%
\item{}False. For example, \(a = 3\) and \(b = 5\). \(a+b = 8\), but neither \(a\) nor \(b\) are even.%
\item{}False, since it is equivalent to the original statement.%
\item{}True. Let \(a\) and \(b\) be integers. Assume both are even. Then \(a = 2k\) and \(b = 2j\) for some integers \(k\) and \(j\). But then \(a+b = 2k + 2j = 2(k+j)\) which is even.%
\item{}True, since the statement is false.%
\end{enumerate}
%
\end{divisionsolution}%
\begin{divisionsolution}{3.2.2}{}{p:exercise:kUG}%
For each of the statements below, say what method of proof you should use to prove them. Then say how the proof starts and how it ends. Bonus points for filling in the middle.%
\begin{enumerate}[label=(\alph*)]
\item{}There are no integers \(x\) and \(y\) such that \(x\) is a prime greater than 5 and \(x = 6y + 3\).%
\item{}For all integers \(n\), if \(n\) is a multiple of 3, then \(n\) can be written as the sum of consecutive integers.%
\item{}For all integers \(a\) and \(b\), if \(a^2 + b^2\) is odd, then \(a\) or \(b\) is odd.%
\end{enumerate}
%
\par\smallskip%
\noindent\textbf{\blocktitlefont Solution}.\quad{}%
\begin{enumerate}[label=(\alph*)]
\item{}Proof by contradiction. Start of proof: Assume, for the sake of contradiction, that there are integers \(x\) and \(y\) such that \(x\) is a prime greater than 5 and \(x = 6y + 3\). End of proof: \textellipsis{} this is a contradiction, so there are no such integers.%
\item{}Direct proof. Start of proof: Let \(n\) be an integer. Assume \(n\) is a multiple of 3. End of proof: Therefore \(n\) can be written as the sum of consecutive integers.%
\item{}Proof by contrapositive. Start of proof: Let \(a\) and \(b\) be integers. Assume that \(a\) and \(b\) are even. End of proof: Therefore \(a^2 + b^2\) is even.%
\end{enumerate}
%
\end{divisionsolution}%
\begin{divisionsolution}{3.2.3}{}{p:exercise:RbP}%
Consider the statement: for all integers \(n\), if \(n\) is even then \(8n\) is even.%
\begin{enumerate}[label=(\alph*)]
\item{}Prove the statement. What sort of proof are you using?%
\item{}Is the converse true? Prove or disprove.%
\end{enumerate}
%
\par\smallskip%
\noindent\textbf{\blocktitlefont Solution}.\quad{}%
\begin{enumerate}[label=(\alph*)]
\item{}Direct proof. \begin{proof}{}{p:proof:ckO}
Let \(n\) be an integer. Assume \(n\) is even. Then \(n = 2k\) for some integer \(k\). Thus \(8n = 16k = 2(8k)\). Therefore \(8n\) is even.%
\end{proof}
%
\item{}The converse is false. That is, there is an integer \(n\) such that \(8n\) is even but \(n\) is odd. For example, consider \(n = 3\). Then \(8n = 24\) which is even but \(n = 3\) is odd.%
\end{enumerate}
%
\end{divisionsolution}%
\begin{divisionsolution}{3.2.4}{}{p:exercise:xiY}%
The game TENZI comes with 40 six-sided dice (each numbered 1 to 6). Suppose you roll all 40 dice.%
\begin{enumerate}[label=(\alph*)]
\item{}Prove that there will be at least seven dice that land on the same number.%
\item{}How many dice would you have to roll before you were guaranteed that some four of them would all match or all be different? Prove your answer.%
\end{enumerate}
%
\par\smallskip%
\noindent\textbf{\blocktitlefont Solution}.\quad{}%
\begin{enumerate}[label=(\alph*)]
\item{}This is an example of the pigeonhole principle. We can prove it by contrapositive.%
\begin{proof}{}{p:proof:ozg}
Suppose that each number only came up 6 or fewer times. So there are at most six 1's, six 2's, and so on. That's a total of 36 dice, so you must not have rolled all 40 dice.%
\end{proof}
\item{}We can have 9 dice without any four matching or any four being all different: three 1's, three 2's, three 3's. We will prove that whenever you roll 10 dice, you will always get four matching or all being different.%
\begin{proof}{}{p:proof:UGp}
Suppose you roll 10 dice, but that there are NOT four matching rolls. This means at most, there are three of any given value. If we only had three different values, that would be only 9 dice, so there must be 4 different values, giving 4 dice that are all different.%
\end{proof}
\end{enumerate}
%
\end{divisionsolution}%
\begin{divisionsolution}{3.2.5}{}{p:exercise:dqh}%
Prove that for all integers \(n\), it is the case that \(n\) is  even if and only if \(3n\) is even.  That is, prove both implications: if \(n\) is even, then \(3n\) is even, and if \(3n\) is even, then \(n\) is even.%
\par\smallskip%
\noindent\textbf{\blocktitlefont Hint}.\quad{}One of the implications will be a direct proof, the other will be a proof by contrapositive.%
\end{divisionsolution}%
\begin{divisionsolution}{3.2.6}{}{p:exercise:Jxq}%
Prove that \(\sqrt 3\) is irrational.%
\par\smallskip%
\noindent\textbf{\blocktitlefont Hint}.\quad{}This is really an exercise in modifying the proof that \(\sqrt{2}\) is irrational.  There you proved things were even; here they will be multiples of 3.%
\par\smallskip%
\noindent\textbf{\blocktitlefont Solution}.\quad{}\begin{solutionproof}
Suppose \(\sqrt{3}\) were rational. Then \(\sqrt{3} = \frac{a}{b}\) for some integers \(a\) and \(b \ne 0\). Without loss of generality, assume \(\frac{a}{b}\) is reduced. Now%
\begin{equation*}
3 = \frac{a^2}{b^2}
\end{equation*}
%
\begin{equation*}
b^2 3 = a^2
\end{equation*}
%
\par
So \(a^2\) is a multiple of 3. This can only happen if \(a\) is a multiple of 3, so \(a = 3k\) for some integer \(k\). Then we have%
\begin{equation*}
b^2 3 = 9k^2
\end{equation*}
%
\begin{equation*}
b^2 = 3k^2
\end{equation*}
%
\par
So \(b^2\) is a multiple of 3, making \(b\) a multiple of 3 as well. But this contradicts our assumption that \(\frac{a}{b}\) is in lowest terms.%
\par
Therefore, \(\sqrt{3}\) is irrational.%
\end{solutionproof}
\end{divisionsolution}%
\begin{divisionsolution}{3.2.7}{}{p:exercise:pEz}%
Consider the statement: for all integers \(a\) and \(b\), if \(a\) is even and \(b\) is a multiple of 3, then \(ab\) is a multiple of 6.%
\begin{enumerate}[label=(\alph*)]
\item{}Prove the statement. What sort of proof are you using?%
\item{}State the converse. Is it true? Prove or disprove.%
\end{enumerate}
%
\par\smallskip%
\noindent\textbf{\blocktitlefont Hint}.\quad{}Part (a) should be a relatively easy direct proof.  Look for a counterexample for part (b).%
\par\smallskip%
\noindent\textbf{\blocktitlefont Solution}.\quad{}%
\begin{enumerate}[label=(\alph*)]
\item{}Direct proof. \begin{proof}{}{p:proof:IRn}
Let \(a\) and \(b\) be integers. Assume \(a\) is even and \(b\) is a multiple of 3. Then \(a = 2k\) and \(b = 3j\) for some integers \(k\) and \(j\). Now%
\begin{equation*}
ab = (2k)(3j) = 6(kj)
\end{equation*}
%
\par
Since \(kj\) is an integer, we have that \(ab\) is a multiple of 6.%
\end{proof}
%
\item{}The converse is: for all integers \(a\) and \(b\), if \(ab\) is a multiple of 6, then \(a\) is even and \(b\) is a multiple of 3. This is false. Consider \(a = 6\) and \(b = 5\). Then \(ab = 30\) which is a multiple of 6, but \(b\) is not divisible by 3 (nor is \(b\) even, so this is not just a matter of picking the right one to be \(b\)).%
\end{enumerate}
\end{divisionsolution}%
\begin{divisionsolution}{3.2.8}{}{p:exercise:VLI}%
Prove the statement: For all integers \(n\), if \(5n\) is odd, then \(n\) is odd. Clearly state the style of proof you are using.%
\par\smallskip%
\noindent\textbf{\blocktitlefont Solution}.\quad{}We will prove the contrapositive: if \(n\) is even, then \(5n\) is even.%
\begin{solutionproof}
Let \(n\) be an arbitrary integer, and suppose \(n\) is even. Then \(n = 2k\) for some integer \(k\). Thus \(5n = 5\cdot 2k = 10k = 2(5k)\). Since \(5k\) is an integer, we see that \(5n\) must be even. This completes the proof.%
\end{solutionproof}
\end{divisionsolution}%
\begin{divisionsolution}{3.2.9}{}{p:exercise:BSR}%
Prove the statement: For all integers \(a\), \(b\), and \(c\), if \(a^2 + b^2 = c^2\), then \(a\) or \(b\) is even.%
\par\smallskip%
\noindent\textbf{\blocktitlefont Hint}.\quad{}A proof by contradiction would be reasonable here, because then you get to assume that both \(a\) \emph{and} \(b\) are odd.  Deduce that \(c^2\) is even, and therefore a multiple of 4 (why? and why is that a contradiction?).%
\par\smallskip%
\noindent\textbf{\blocktitlefont Solution}.\quad{}\begin{solutionproof}
Suppose, contrary to stipulation, that there are integers \(a\), \(b\) and \(c\) such that \(a^2 + b^2 = c^2\) but \(a\) and \(b\) are both odd. Then \(a = 2k+1\) and \(b = 2j + 1\) for some integers \(k\) and \(j\). We then have%
\begin{equation*}
a^2 + b^2 = (2k+1)^2 + (2j+1)^2 = 4k^2 + 4k + 1 + 4j^2 + 4j + 1 = 4(k^2 + j^2 + k + j) + 2
\end{equation*}
%
\par
So \(c^2 = 4(k^2 + j^2 + k + j) + 2\). This means that \(c^2\) is even, which can only happen if \(c\) is even. But then \(c^2\) must be a multiple of 4. However, this is a contradiction because \(4(k^2 + j^2 + k + j) + 2\) is not a multiple of 4. Therefore, if \(a^2 + b^2 = c^2\), then \(a\) or \(b\) is even.%
\end{solutionproof}
\end{divisionsolution}%
\begin{divisionsolution}{3.2.10}{}{p:exercise:iaa}%
Suppose that you would like to prove the following implication:%
\begin{quote}%
For all numbers \(n\), if \(n\) is prime then \(n\) is solitary.%
\end{quote}
Write out the beginning and end of the argument if you were to prove the statement,%
\begin{enumerate}[label=(\alph*)]
\item{}Directly%
\item{}By contrapositive%
\item{}By contradiction%
\end{enumerate}
%
\par
You do not need to provide details for the proofs (since you do not know what solitary means). However, make sure that you provide the first few and last few lines of the proofs so that we can see that logical structure you would follow.%
\end{divisionsolution}%
\begin{divisionsolution}{3.2.11}{}{p:exercise:Ohj}%
Suppose you have a collection of 5-cent stamps and 8-cent stamps. We saw earlier that it is possible to make any amount of postage greater than 27 cents using combinations of both these types of stamps. But, let's ask some other questions:%
\begin{enumerate}[label=(\alph*)]
\item{}Prove that if you only use an even number of both types of stamps, the amount of postage you make must be even.%
\item{}Suppose you made an even amount of postage. Prove that you used an even number of at least one of the types of stamps.%
\item{}Suppose you made exactly 72 cents of postage. Prove that you used at least 6 of one type of stamp.%
\end{enumerate}
%
\par\smallskip%
\noindent\textbf{\blocktitlefont Hint}.\quad{}Use a different style of proof for each part.  The last part should remind you of the pigeonhole principle, so mimicking that proof might be helpful.%
\end{divisionsolution}%
\begin{divisionsolution}{3.2.12}{}{p:exercise:uos}%
Prove: \(x=y\) if and only if \(xy=\dfrac{(x+y)^2}{4}\). Note, you will need to prove two ``directions'' here: the ``if'' and the ``only if'' part.%
\end{divisionsolution}%
\begin{divisionsolution}{3.2.13}{}{p:exercise:avB}%
Prove that \(\log(7)\) is irrational.%
\par\smallskip%
\noindent\textbf{\blocktitlefont Hint}.\quad{}Note that if \(\log(7) = \frac{a}{b}\), then \(7 = 10^\frac{a}{b}\).  Can any power of 7 be the same as a power of 10?%
\par\smallskip%
\noindent\textbf{\blocktitlefont Solution}.\quad{}We give a proof by contradiction.%
\begin{solutionproof}
Suppose, contrary to stipulation that \(\log(7)\) is rational. Then \(\log(7) = \frac{a}{b}\) with \(a\) and \(b \ne 0\) integers. By properties of logarithms, this implies%
\begin{equation*}
7 = 10^{\frac{a}{b}}
\end{equation*}
%
\par
Equivalently,%
\begin{equation*}
7^b = 10^a
\end{equation*}
%
\par
But this is impossible as any power of 7 will be odd while any power of 10 will be even. Therefore, \(\log(7)\) is irrational.%
\end{solutionproof}
\end{divisionsolution}%
\begin{divisionsolution}{3.2.14}{}{p:exercise:GCK}%
Prove that there are no integer solutions to the equation \(x^2 = 4y + 3\).%
\par\smallskip%
\noindent\textbf{\blocktitlefont Hint}.\quad{}What if there were?  Deduce that \(x\) must be odd, and continue towards a contradiction.%
\par\smallskip%
\noindent\textbf{\blocktitlefont Solution}.\quad{}\begin{solutionproof}
Suppose there were integers \(x\) and \(y\) such that \(x^2 = 4y + 3\). Now \(x^2\) must be odd, since \(4y + 3\) is odd. Since \(x^2\) is odd, we know \(x\) must be odd as well. So \(x = 2k + 1\) for some integer \(k\). Then \(x^2 = 4k^2 + 4k + 1 = 4(k^2 + k) + 1\). Therefore we have%
\begin{equation*}
4(k^2 + k) + 1 = 4y + 3
\end{equation*}
which implies%
\begin{equation*}
4(k^2 + k) = 4y + 2
\end{equation*}
and therefore%
\begin{equation*}
2(k^2 + k) = 2y + 1\text{.}
\end{equation*}
%
\par
But this is a contradiction \textendash{} the left-hand side is even while the right-hand side is odd. Thus there are no integer solutions.%
\end{solutionproof}
\end{divisionsolution}%
\begin{divisionsolution}{3.2.15}{}{p:exercise:mJT}%
Prove that every prime number greater than 3 is either one more or one less than a multiple of 6.%
\par\smallskip%
\noindent\textbf{\blocktitlefont Hint}.\quad{}Prove the contrapositive by cases.  There will be 4 cases to consider.%
\par\smallskip%
\noindent\textbf{\blocktitlefont Solution}.\quad{}We must prove that for any integer greater than 3, if the integer is prime, then it is one more or one less than a multiple of 6. The contrapositive of this statement is that if a number greater than 3 is not one more or one less than a multiple of 6, then it is not prime.%
\begin{solutionproof}
Let \(n\) be an integer greater than 3, and assume \(n \ne 6k+1\) and \(n \ne 6k-1\) for any integer \(k\). Then for some integer \(k\), \(n = 6k\), \(n = 6k+2\), \(n = 6k+3\) or \(n = 6k+4\). Case 1: \(n = 6k\). Then \(n\) is a multiple of 6, so is not prime. Case 2: \(n = 6k+2\). Then \(n\) is even, so not prime. Case 3: \(n = 6k+3\). Then \(n\) is a multiple of 3, so not prime. Case 4: \(n = 6k+4\). Then \(n\) is even, so not prime. So in any case, \(n\) is not prime.%
\end{solutionproof}
\end{divisionsolution}%
\begin{divisionsolution}{3.2.16}{}{p:exercise:SRc}%
Your ``friend'' has shown you a ``proof'' he wrote to show that \(1 = 3\). Here is the proof:%
\begin{proof}{}{p:proof:Khk}
I claim that \(1 = 3\). Of course we can do anything to one side of an equation as long as we also do it to the other side. So subtract 2 from both sides. This gives \(-1 = 1\). Now square both sides, to get \(1 = 1\). And we all agree this is true.%
\end{proof}
What is going on here? Is your friend's argument valid? Is the argument a proof of the claim \(1=3\)? Carefully explain using what we know about logic.%
\par\smallskip%
\noindent\textbf{\blocktitlefont Hint}.\quad{}Your friend's proof a proof, but of what?  What implication follows from the given proof?  Is that helpful?%
\par\smallskip%
\noindent\textbf{\blocktitlefont Solution}.\quad{}In the proof we assume that \(1=3\) and conclude that \(1=1\). So we have proved the implication%
\begin{equation*}
1=3 \imp 1=1
\end{equation*}
%
\par
Note that we actually have a valid proof of this, and that the implication is true (for one thing, the ``if'' part is true, so the implication is automatically true). However, what we really want is to converse of this, that \(1=1 \imp 1=3\). But as we know, the converse is not implied by the original implication (it better not be, otherwise we could conclude that 1 actually was 3).%
\par
Another way to say this: we can never conclude anything about the ``if'' part of an implication, since the ``if'' part can be true or false even if the implication is true.%
\end{divisionsolution}%
\begin{divisionsolution}{3.2.17}{}{p:exercise:yYl}%
A standard deck of 52 cards consists of 4 suites (hearts, diamonds, spades and clubs) each containing 13 different values (Ace, 2, 3, \textellipsis{}, 10, J, Q, K). If you draw some number of cards at random you might or might not have a pair (two cards with the same value) or three cards all of the same suit. However, if you draw enough cards, you will be guaranteed to have these. For each of the following, find the smallest number of cards you would need to draw to be guaranteed having the specified cards. Prove your answers.%
\begin{enumerate}[label=(\alph*)]
\item{}Three of a kind (for example, three 7's).%
\item{}A flush of five cards (for example, five hearts).%
\item{}Three cards that are either all the same suit or all different suits. %
\end{enumerate}
%
\end{divisionsolution}%
\begin{divisionsolution}{3.2.18}{}{p:exercise:ffu}%
Suppose you are at a party with 19 of your closest friends (so including you, there are 20 people there). Explain why there must be least two people at the party who are friends with the same number of people at the party. Assume friendship is always reciprocated.%
\par\smallskip%
\noindent\textbf{\blocktitlefont Hint}.\quad{}Consider the set of \emph{numbers} of friends that everyone has.  If everyone had a different number of friends, this set must contain 20 elements.  Is that possible?  Why not?%
\par\smallskip%
\noindent\textbf{\blocktitlefont Solution}.\quad{}Suppose this was not the case. That is, suppose everyone at the party had a \emph{different} number of friends. What could these numbers be. The would have to be less than 20, so each number from 0 to 19 (that's 20 numbers) must be used exactly once. But this is impossible. If someone is friends with 19 people, then she is friends with everyone, including the person who is supposedly friends with 0 people.%
\par
What this says is that in any simple graph with 20 vertices, there must be at least two vertices which have the same degree.%
\end{divisionsolution}%
\begin{divisionsolution}{3.2.19}{}{p:exercise:LmD}%
Your friend has given you his list of 115 best Doctor Who episodes (in order of greatness). It turns out that you have seen 60 of them. Prove that there are at least two episodes you have seen that are exactly four episodes apart on your friend's list.%
\par\smallskip%
\noindent\textbf{\blocktitlefont Hint}.\quad{}This feels like the pigeonhole principle, although a bit more complicated.  At least, you could try to replicate the style of proof used by the pigeonhole principle.  How would the episodes need to be spaced out so that no two of your sixty were exactly 4 apart?%
\end{divisionsolution}%
\begin{divisionsolution}{3.2.20}{}{p:exercise:rtM}%
Suppose you have an \(n\times n\) chessboard but your dog has eaten one of the corner squares. Can you still cover the remaining squares with dominoes? What needs to be true about \(n\)? Give necessary and sufficient conditions (that is, say exactly which values of \(n\) work and which do not work). Prove your answers.%
\begin{sidebyside}{1}{0.4}{0.4}{0}%
\begin{sbspanel}{0.2}%
\resizebox{\linewidth}{!}{%
          \begin{tikzpicture}[scale=.25]
\foreach \x in {0,2,...,6}{
\foreach \y in {0,2,...,6}{
\draw[fill=white!85!black] (\x,\y) rectangle (\x+1, \y+1) rectangle (\x+2,\y+2);
}}
\draw (0,0) grid (8,8);
\draw[white, fill=white] (7,7) rectangle (8,8);
\draw[very thick] (0,0) -- (8,0) --(8,7) -- (7,7) -- (7,8) -- (0,8) -- (0,0);
\end{tikzpicture}
}%
\end{sbspanel}%
\end{sidebyside}%
\par\smallskip%
\noindent\textbf{\blocktitlefont Solution}.\quad{}Yes, we can in fact still cover the chessboard with dominoes but only if \(n\) is odd. So, my claim is that if \(n\) is odd, then I can cover the above chessboard. To prove this, I am going to first notice that if I take away the column and row attached to the missing square, I have created an \(n-1\times n-1\) chessboard where \(n-1\) is even. By the proof in class, we have shown that that can be completely covered with dominoes. So, now I just need to worry about the row and column I took away. Since \(n\) is odd, I know that the row and column are even and thus can be covered by dominoes lined up head to tail. Thus, I have shown that if \(n\) is odd then I can completely cover an \(n\times n\) chessboard if \(n\) is odd.%
\end{divisionsolution}%
\begin{divisionsolution}{3.2.21}{}{p:exercise:XAV}%
What if your \(n\times n\) chessboard is missing two opposite corners? Prove that no matter what \(n\) is, you will not be able to cover the remaining squares with dominoes.%
\begin{sidebyside}{1}{0.4}{0.4}{0}%
\begin{sbspanel}{0.2}%
\resizebox{\linewidth}{!}{%
            \begin{tikzpicture}[scale=.25]
\foreach \x in {0,2,...,6}{
\foreach \y in {0,2,...,6}{
\draw[fill=white!85!black] (\x,\y) rectangle (\x+1, \y+1) rectangle (\x+2,\y+2);
}}
\draw (0,0) grid (8,8);
\draw[white, fill=white] (7,7) rectangle (8,8) (0,0) rectangle (1,1);
\draw[very thick] (0,1) -- (1,1) -- (1,0) -- (8,0) --(8,7) -- (7,7) -- (7,8) -- (0,8) -- (0,1);
\end{tikzpicture}
}%
\end{sbspanel}%
\end{sidebyside}%
\par\smallskip%
\noindent\textbf{\blocktitlefont Solution}.\quad{}First notice that both removed squares have identical color. So their removal leaves us with \(\frac{n}{2}\) squares of one color and \(\frac{n-2}{2}\) squares of the other color. Since a domino always covers one black and one white squares, the remaining squares cannot be tiled.%
\end{divisionsolution}%
\section*{3.3 Chapter Summary}
\addcontentsline{toc}{section}{3.3 Chapter Summary}
\sectionmark{3.3 Chapter Summary}
\subsection*{Chapter Review}
\addcontentsline{toc}{subsection}{Chapter Review}
\begin{divisionsolution}{3.3.1}{}{x:exercise:tt}%
Complete a truth table for the statement \(\neg P \imp (Q \wedge R)\).%
\par\smallskip%
\noindent\textbf{\blocktitlefont Solution}.\quad{}\begin{sidebyside}{1}{0}{0}{0}%
\begin{sbspanel}{1}%
{\centering%
{\tabularfont%
\begin{tabular}{cAcAcBc}
\(P\)&\(Q\)&\(R\)&\(\neg P \imp (Q \wedge R)\)\tabularnewline\hrulethin
T&T&T&T\tabularnewline[0pt]
T&T&F&T\tabularnewline[0pt]
T&F&T&T\tabularnewline[0pt]
T&F&F&T\tabularnewline[0pt]
F&T&T&T\tabularnewline[0pt]
F&T&F&F\tabularnewline[0pt]
F&F&T&F\tabularnewline[0pt]
F&F&F&F
\end{tabular}
}%
\par}
\end{sbspanel}%
\end{sidebyside}%
\end{divisionsolution}%
\begin{divisionsolution}{3.3.2}{}{p:exercise:LPg}%
Suppose you know that the statement ``if Peter is not tall, then Quincy is fat and Robert is skinny'' is false. What, if anything, can you conclude about Peter and Robert if you know that Quincy is indeed fat? Explain (you may reference problem~3.3.1).%
\par\smallskip%
\noindent\textbf{\blocktitlefont Solution}.\quad{}Peter is not tall and Robert is not skinny. You must be in row 6 in the truth table above.%
\end{divisionsolution}%
\begin{divisionsolution}{3.3.3}{}{p:exercise:rWp}%
Are the statements \(P \imp (Q \vee R)\) and \((P \imp Q) \vee (P \imp R)\) logically equivalent? Explain your answer.%
\par\smallskip%
\noindent\textbf{\blocktitlefont Solution}.\quad{}Yes. To see this, make a truth table for each statement and compare.%
\end{divisionsolution}%
\begin{divisionsolution}{3.3.4}{}{p:exercise:Ydy}%
Is the following a valid deduction rule? Explain.%
\begin{sidebyside}{1}{0}{0}{0}%
\begin{sbspanel}{1}%
{\centering%
{\tabularfont%
\begin{tabular}{cc}
&\(P \imp Q\)\tabularnewline[0pt]
&\(P\imp R\)\tabularnewline\hrulethin
\(\therefore\)&\(P \imp (Q \wedge R)\).
\end{tabular}
}%
\par}
\end{sbspanel}%
\end{sidebyside}%
\par\smallskip%
\noindent\textbf{\blocktitlefont Solution}.\quad{}Make a truth table that includes all three statements in the argument:%
\begin{sidebyside}{1}{0}{0}{0}%
\begin{sbspanel}{1}%
{\centering%
{\tabularfont%
\begin{tabular}{cAcAcBcAcAc}
\(P\)&\(Q\)&\(R\)&\(P \imp Q\)&\(P \imp R\)&\(P \imp (Q \wedge R)\)\tabularnewline\hrulethin
T&T&T&T&T&T\tabularnewline[0pt]
T&T&F&T&F&F\tabularnewline[0pt]
T&F&T&F&T&F\tabularnewline[0pt]
T&F&F&F&F&F\tabularnewline[0pt]
F&T&T&T&T&T\tabularnewline[0pt]
F&T&F&T&T&T\tabularnewline[0pt]
F&F&T&T&T&T\tabularnewline[0pt]
F&F&F&T&T&T
\end{tabular}
}%
\par}
\end{sbspanel}%
\end{sidebyside}%
\par
Notice that in every row for which both \(P \imp Q\) and \(P \imp R\) is true, so is \(P \imp (Q \wedge R)\). Therefore, whenever the premises of the argument are true, so is the conclusion. In other words, the deduction rule is valid.%
\end{divisionsolution}%
\begin{divisionsolution}{3.3.5}{}{p:exercise:EkH}%
Write the negation, converse and contrapositive for each of the statements below.%
\begin{enumerate}[label=(\alph*)]
\item{}If the power goes off, then the food will spoil.%
\item{}If the door is closed, then the light is off.%
\item{}\(\forall x (x \lt 1 \imp x^2 \lt 1)\).%
\item{}For all natural numbers \(n\), if \(n\) is prime, then \(n\) is solitary.%
\item{}For all functions \(f\), if \(f\) is differentiable, then \(f\) is continuous.%
\item{}For all integers \(a\) and \(b\), if \(a\cdot b\) is even, then \(a\) and \(b\) are even.%
\item{}For every integer \(x\) and every integer \(y\) there is an integer \(n\) such that if \(x > 0\) then \(nx > y\).%
\item{}For all real numbers \(x\) and \(y\), if \(xy = 0\) then \(x = 0\) or \(y = 0\).%
\item{}For every student in Math 228, if they do not understand implications, then they will fail the exam.%
\end{enumerate}
%
\par\smallskip%
\noindent\textbf{\blocktitlefont Solution}.\quad{}%
\begin{enumerate}[label=(\alph*)]
\item{}Negation: The power goes off and the food does not spoil.%
\par
Converse: If the food spoils, then the power went off.%
\par
Contrapositive: If the food does not spoil, then the power did not go off.%
\item{}Negation: The door is closed and the light is on.%
\par
Converse: If the light is off then the door is closed.%
\par
Contrapositive: If the light is on then the door is open.%
\item{}Negation: \(\exists x (x \lt 1 \wedge x^2 \ge 1)\)%
\par
Converse: \(\forall x( x^2 \lt 1 \imp x \lt 1)\)%
\par
Contrapositive: \(\forall x (x^2 \ge 1 \imp x \ge 1)\).%
\item{}Negation: There is a natural number \(n\) which is prime but not solitary.%
\par
Converse: For all natural numbers \(n\), if \(n\) is solitary, then \(n\) is prime.%
\par
Contrapositive: For all natural numbers \(n\), if \(n\) is not solitary then \(n\) is not prime.%
\item{}Negation: There is a function which is differentiable and not continuous.%
\par
Converse: For all functions \(f\), if \(f\) is continuous then \(f\) is differentiable.%
\par
Contrapositive: For all functions \(f\), if \(f\) is not continuous then \(f\) is not differentiable.%
\item{}Negation: There are integers \(a\) and \(b\) for which \(a\cdot b\) is even but \(a\) or \(b\) is odd.%
\par
Converse: For all integers \(a\) and \(b\), if \(a\) and \(b\) are even then \(ab\) is even.%
\par
Contrapositive: For all integers \(a\) and \(b\), if \(a\) or \(b\) is odd, then \(ab\) is odd.%
\item{}Negation: There are integers \(x\) and \(y\) such that for every integer \(n\), \(x \gt 0\) and \(nx \le y\).%
\par
Converse: For every integer \(x\) and every integer \(y\) there is an integer \(n\) such that if \(nx >
y\) then \(x > 0\).%
\par
Contrapositive: For every integer \(x\) and every integer \(y\) there is an integer \(n\) such that if \(nx \le y\) then \(x \le 0\).%
\item{}Negation: There are real numbers \(x\) and \(y\) such that \(xy = 0\) but \(x \ne 0\) and \(y \ne 0\).%
\par
Converse: For all real numbers \(x\) and \(y\), if \(x = 0\) or \(y = 0\) then \(xy = 0\)%
\par
Contrapositive: For all real numbers \(x\) and \(y\), if \(x \ne 0\) and \(y \ne 0\) then \(xy \ne 0\).%
\item{}Negation: There is at least one student in Math 228 who does not understand implications but will still pass the exam.%
\par
Converse: For every student in Math 228, if they fail the exam, then they did not understand implications.%
\par
Contrapositive: For every student in Math 228, if they pass the exam, then they understood implications.%
\end{enumerate}
%
\end{divisionsolution}%
\begin{divisionsolution}{3.3.6}{}{p:exercise:krQ}%
Consider the statement: for all integers \(n\), if \(n\) is even and \(n \le 7\) then \(n\) is negative or \(n \in \{0,2,4,6\}\).%
\begin{enumerate}[label=(\alph*)]
\item{}Is the statement true? Explain why.%
\item{}Write the negation of the statement. Is it true? Explain.%
\item{}State the contrapositive of the statement. Is it true? Explain.%
\item{}State the converse of the statement. Is it true? Explain.%
\end{enumerate}
%
\par\smallskip%
\noindent\textbf{\blocktitlefont Solution}.\quad{}%
\begin{enumerate}[label=(\alph*)]
\item{}The statement is true. If \(n\) is an even integer less than or equal to 7, then the only way it could not be negative is if \(n\) was equal to 0, 2, 4, or 6.%
\item{}There is an integer \(n\) such that \(n\) is even and \(n \le 7\) but \(n\) is not negative and \(n \not\in \{0,2,4,6\}\). This is false, since the original statement is true.%
\item{}For all integers \(n\), if \(n\) is not negative and \(n \not\in\{0,2,4,6\}\) then \(n\) is odd or \(n > 7\). This is true, since the contrapositive is equivalent to the original statement (which is true).%
\item{}For all integers \(n\), if \(n\) is negative or \(n \in \{0,2,4,6\}\) then \(n\) is even and \(n \le 7\). This is false. \(n = -3\) is a counterexample.%
\end{enumerate}
%
\end{divisionsolution}%
\begin{divisionsolution}{3.3.7}{}{p:exercise:QyZ}%
Consider the statement: \(\forall x (\forall y (x + y = y) \imp \forall z (x\cdot z = 0))\).%
\begin{enumerate}[label=(\alph*)]
\item{}Explain what the statement says in words. Is this statement true? Be sure to state what you are taking the universe of discourse to be.%
\item{}Write the converse of the statement, both in words and in symbols. Is the converse true?%
\item{}Write the contrapositive of the statement, both in words and in symbols. Is the contrapositive true?%
\item{}Write the negation of the statement, both in words and in symbols. Is the negation true?%
\end{enumerate}
%
\par\smallskip%
\noindent\textbf{\blocktitlefont Solution}.\quad{}%
\begin{enumerate}[label=(\alph*)]
\item{}For any number \(x\), if it is the case that adding any number to \(x\) gives that number back, then multiplying any number by \(x\) will give 0. This is true (of the integers or the reals). The ``if'' part only holds if \(x = 0\), and in that case, anything times \(x\) will be 0.%
\item{}The converse in words is this: for any number \(x\), if everything times \(x\) is zero, then everything added to \(x\) gives itself. Or in symbols: \(\forall x (\forall z (x \cdot z = 0) \imp \forall y (x + y = y))\). The converse is true: the only number which when multiplied by any other number gives 0 is \(x = 0\). And if \(x = 0\), then \(x + y = y\).%
\item{}The contrapositive in words is: for any number \(x\), if there is some number which when multiplied by \(x\) does not give zero, then there is some number which when added to \(x\) does not give that number. In symbols: \(\forall x (\exists z (x\cdot z \ne 0) \imp \exists y (x + y \ne y))\). We know the contrapositive must be true because the original implication is true.%
\item{}The negation: there is a number \(x\) such that any number added to \(x\) gives the number back again, but there is a number you can multiply \(x\) by and not get 0. In symbols: \(\exists x (\forall y (x + y = y) \wedge \exists z (x \cdot z \ne 0))\). Of course since the original implication is true, the negation is false.%
\end{enumerate}
%
\end{divisionsolution}%
\begin{divisionsolution}{3.3.8}{}{p:exercise:wGi}%
Simplify the following.%
\begin{enumerate}[label=(\alph*)]
\item{}\(\neg (\neg (P \wedge \neg Q) \imp \neg(\neg R \vee \neg(P \imp R)))\).%
\item{}\(\neg \exists x \neg \forall y \neg \exists z (z = x + y \imp \exists w (x - y = w))\).%
\end{enumerate}
%
\par\smallskip%
\noindent\textbf{\blocktitlefont Solution}.\quad{}%
\begin{enumerate}[label=(\alph*)]
\item{}\((\neg P \vee Q) \wedge (\neg R \vee (P \wedge \neg R))\).%
\item{}\(\forall x \forall y \forall z (z = x+y \wedge \forall w (x-y \ne w))\).%
\end{enumerate}
%
\end{divisionsolution}%
\begin{divisionsolution}{3.3.9}{}{p:exercise:cNr}%
Consider the statement: for all integers \(n\), if \(n\) is odd, then \(7n\) is odd.%
\begin{enumerate}[label=(\alph*)]
\item{}Prove the statement. What sort of proof are you using?%
\item{}Prove the converse. What sort of proof are you using?%
\end{enumerate}
%
\par\smallskip%
\noindent\textbf{\blocktitlefont Solution}.\quad{}%
\begin{enumerate}[label=(\alph*)]
\item{}Direct proof. \begin{proof}{}{p:proof:yvv}
Let \(n\) be an integer. Assume \(n\) is odd. So \(n = 2k+1\) for some integer \(k\). Then%
\begin{equation*}
7n = 7(2k+1) = 14k + 7 = 2(7k +3) + 1\text{.}
\end{equation*}
%
\par
Since \(7k + 3\) is an integer, we see that \(7n\) is odd.%
\end{proof}
%
\item{}The converse is: for all integers \(n\), if \(7n\) is odd, then \(n\) is odd. We will prove this by contrapositive. \begin{proof}{}{p:proof:eCE}
Let \(n\) be an integer. Assume \(n\) is not odd. Then \(n = 2k\) for some integer \(k\). So \(7n = 14k = 2(7k)\) which is to say \(7n\) is even. Therefore \(7n\) is not odd.%
\end{proof}
%
\end{enumerate}
%
\end{divisionsolution}%
\begin{divisionsolution}{3.3.10}{}{p:exercise:IUA}%
Suppose you break your piggy bank and scoop up a handful of 22 coins (pennies, nickels, dimes and quarters).%
\begin{enumerate}[label=(\alph*)]
\item{}Prove that you must have at least 6 coins of a single denomination.%
\item{}Suppose you have an odd number of pennies. Prove that you must have an odd number of at least one of the other types of coins.%
\item{}How many coins would you need to scoop up to be sure that you either had 4 coins that were all the same or 4 coins that were all different? Prove your answer.%
\end{enumerate}
%
\par\smallskip%
\noindent\textbf{\blocktitlefont Solution}.\quad{}%
\begin{enumerate}[label=(\alph*)]
\item{}Suppose you only had 5 coins of each denomination. This means you have 5 pennies, 5 nickels, 5 dimes and 5 quarters. This is a total of 20 coins. But you have more than 20 coins, so you must have more than 5 of at least one type.%
\item{}Suppose you have 22 coins, including \(2k\) nickels, \(2j\) dimes, and \(2l\) quarters (so an even number of each of these three types of coins). The number of pennies you have will then be%
\begin{equation*}
22 - 2k - 2j - 2l = 2(11-k-j-l)\text{.}
\end{equation*}
But this says that the number of pennies is also even (it is 2 times an integer). Thus we have established the contrapositive of the statement, ``If you have an odd number of pennies then you have an odd number of at least one other coin type.''%
\item{}You need 10 coins. You could have 3 pennies, 3 nickels, and 3 dimes. The 10th coin must either be a quarter, giving you 4 coins that are all different, or else a 4th penny, nickel or dime. To prove this, assume you don't have 4 coins that are all the same or all different. In particular, this says that you only have 3 coin types, and each of those types can only contain 3 coins, for a total of 9 coins, which is less than 10.%
\end{enumerate}
%
\end{divisionsolution}%
\begin{divisionsolution}{3.3.11}{}{p:exercise:pbJ}%
You come across four trolls playing bridge. They declare:%
\begin{quote}%
Troll 1: All trolls here see at least one knave.%
\par
Troll 2: I see at least one troll that sees only knaves.%
\par
Troll 3: Some trolls are scared of goats.%
\par
Troll 4: All trolls are scared of goats.%
\end{quote}
Are there any trolls that are not scared of goats? Recall, of course, that all trolls are either knights (who always tell the truth) or knaves (who always lie).%
\end{divisionsolution}%
\chapter*{4 Graph Theory}
\addcontentsline{toc}{chapter}{4 Graph Theory}
\chaptermark{4 Graph Theory}
\section*{4.1 Definitions}
\addcontentsline{toc}{section}{4.1 Definitions}
\sectionmark{4.1 Definitions}
\subsection*{Exercises}
\addcontentsline{toc}{subsection}{Exercises}
\begin{divisionsolution}{4.1.1}{}{p:exercise:nGV}%
If 10 people each shake hands with each other, how many handshakes took place? What does this question have to do with graph theory?%
\par\smallskip%
\noindent\textbf{\blocktitlefont Solution}.\quad{}This is asking for the number of edges in \(K_{10}\). Each vertex (person) has degree (shook hands with) 9 (people). So the sum of the degrees is \(90\). However, the degrees count each edge (handshake) twice, so there are 45 edges in the graph. That is how many handshakes took place.%
\end{divisionsolution}%
\begin{divisionsolution}{4.1.2}{}{p:exercise:TOe}%
Among a group of 5 people, is it possible for everyone to be friends with exactly 2 of the people in the group? What about 3 of the people in the group?%
\par\smallskip%
\noindent\textbf{\blocktitlefont Solution}.\quad{}It is possible for everyone to be friends with exactly 2 people. You could arrange the 5 people in a circle and say that everyone is friends with the two people on either side of them (so you get the graph \(C_5\)). However, it is not possible for everyone to be friends with 3 people. That would lead to a graph with an odd number of odd degree vertices which is impossible since the sum of the degrees must be even.%
\end{divisionsolution}%
\begin{divisionsolution}{4.1.3}{}{p:exercise:zVn}%
Is it possible for two \emph{different} (non-isomorphic) graphs to have the same number of vertices and the same number of edges? What if the degrees of the vertices in the two graphs are the same (so both graphs have vertices with degrees 1, 2, 2, 3, and 4, for example)? Draw two such graphs or explain why not.%
\par\smallskip%
\noindent\textbf{\blocktitlefont Hint}.\quad{}Both situations are possible.  Go find some examples.%
\par\smallskip%
\noindent\textbf{\blocktitlefont Solution}.\quad{}Yes. For example, both graphs below contain 6 vertices, 7 edges, and have degrees (2,2,2,2,3,3).%
\begin{sidebyside}{2}{0.1}{0.1}{0.2}%
\begin{sbspanel}{0.4}[bottom]%
\resizebox{\linewidth}{!}{%
		\begin{tikzpicture}
 \draw[thick] (-2,0) \v -- (-1,0) \v -- (-1.5,1) \v -- (-2,0) (-1.5,1) -- (1.5, 1) \v -- (1,0) \v -- (2,0) \v -- (1.5,1);
\end{tikzpicture}
}%
\end{sbspanel}%
\begin{sbspanel}{0.2}[bottom]%
\resizebox{\linewidth}{!}{%
		\begin{tikzpicture}
\foreach \x in {0,...,5}
  \draw[thick] (\x*60:1) \v -- (\x*60 + 60:1);
  \draw[thick] (0:1) -- (180:1);
\end{tikzpicture}
}%
\end{sbspanel}%
\end{sidebyside}%
\end{divisionsolution}%
\begin{divisionsolution}{4.1.4}{}{p:exercise:gcw}%
Are the two graphs below equal? Are they isomorphic? If they are isomorphic, give the isomorphism. If not, explain.%
\par
Graph 1: \(V = \{a,b,c,d,e\}\), \(E = \{\{a,b\}, \{a,c\}, \{a,e\}, \{b,d\}, \{b,e\}, \{c,d\}\}\). %
\par
Graph 2:%
\begin{sidebyside}{1}{0.375}{0.375}{0}%
\begin{sbspanel}{0.25}%
\resizebox{\linewidth}{!}{%
					\begin{tikzpicture}
\foreach \x in {0,...,4} {
\coordinate (v\x) at (90-72*\x:.75);}
\draw (v3) \vl{\footnotesize \(d\)} -- (v0) \vr{\footnotesize \(a\)} -- (v2) \vr{\footnotesize \(c\)} -- (v1) \vr{\footnotesize \(b\)} -- (v4) \vl{\footnotesize \(e\)} -- (v3) -- (v2);
\end{tikzpicture}
}%
\end{sbspanel}%
\end{sidebyside}%
\par\smallskip%
\noindent\textbf{\blocktitlefont Solution}.\quad{}The graphs are not equal. For example, graph 1 has an edge \(\{a,b\}\) but graph 2 does not have that edge. They are isomorphic. One possible isomorphism is \(f:G_1 \to G_2\) defined by \(f(a) = d\), \(f(b) = c\), \(f(c) = e\), \(f(d) = b\), \(f(e) = a\).%
\end{divisionsolution}%
\begin{divisionsolution}{4.1.5}{}{p:exercise:MjF}%
Consider the following two graphs:%
\begin{description}
\item[{\(G_1\)}]\(V_1=\{a,b,c,d,e,f,g\}\)%
\par
\(E_1=\{\{a,b\},\{a,d\},\{b,c\},\{b,d\},\{b,e\},\{b,f\},\{c,g\},\{d,e\}\),%
\par
\(\{e,f\},\{f,g\}\}\).%
\item[{\(G_2\)}]\(V_2=\{v_1,v_2,v_3,v_4,v_5,v_6,v_7\}\),%
\par
\(E_2=\{\{v_1,v_4\},\{v_1,v_5\},\{v_1,v_7\},\{v_2,v_3\},\{v_2,v_6\}\),%
\par
\(\{v_3,v_5\},\{v_3,v_7\},\{v_4,v_5\},\{v_5,v_6\},\{v_5,v_7\}\}\)%
\end{description}
%
\begin{enumerate}[label=(\alph*)]
\item{}Let \(f:G_1 \rightarrow G_2\) be a function that takes the vertices of Graph 1 to vertices of Graph 2. The function is given by the following table:%
\begin{sidebyside}{1}{0}{0}{0}%
\begin{sbspanel}{1}%
{\centering%
{\tabularfont%
\begin{tabular}{llllllll}
\multicolumn{1}{lA}{\(x\)}&\(a\)&\(b\)&\(c\)&\(d\)&\(e\)&\(f\)&\(g\)\tabularnewline\hrulethin
\multicolumn{1}{lA}{\(f(x)\)}&\(v_4\)&\(v_5\)&\(v_1\)&\(v_6\)&\(v_2\)&\(v_3\)&\(v_7\)
\end{tabular}
}%
\par}
\end{sbspanel}%
\end{sidebyside}%
\par
Does \(f\) define an isomorphism between Graph 1 and Graph 2?%
\item{}Define a new function \(g\) (with \(g \ne f\)) that defines an isomorphism between Graph 1 and Graph 2. %
\item{}Is the graph pictured below isomorphic to Graph 1 and Graph 2? Explain.%
\begin{sidebyside}{1}{0.4}{0.4}{0}%
\begin{sbspanel}{0.2}%
\resizebox{\linewidth}{!}{%
\begin{tikzpicture}
	\draw (-1, 0) coordinate (v1) -- (0,0) coordinate (v2) -- (1,0) coordinate (v3) -- (1,1) coordinate (v4) -- (0,1) coordinate (v5) -- (-1,1) coordinate (v6) -- (v1) --(0,.5) coordinate (v7) -- (v2) (v7) -- (v3) (v7) -- (v5);
	\foreach \i in {1,...,7}{
		\fill (v\i) \v;
	}
	\end{tikzpicture}
}%
\end{sbspanel}%
\end{sidebyside}%
\end{enumerate}
%
\end{divisionsolution}%
\begin{divisionsolution}{4.1.6}{}{p:exercise:sqO}%
What is the largest number of edges possible in a graph with 10 vertices?  What is the largest number of edges possible in a \emph{bipartite} graph with 10 vertices?  What is the largest number of edges possible in a \emph{tree} with 10 vertices?%
\par\smallskip%
\noindent\textbf{\blocktitlefont Hint}.\quad{}The bipartite graph is a little tricky.  You will definitely want a complete bipartite graph, but it could be \(K_{5,5}\) or maybe \(K_{1,9}\), or \textellipsis{}%
\end{divisionsolution}%
\begin{divisionsolution}{4.1.7}{}{p:exercise:YxX}%
Which of the graphs below are bipartite? Justify your answers.%
\begin{sidebyside}{4}{0.025}{0.025}{0.05}%
\begin{sbspanel}{0.2}[center]%
\resizebox{\linewidth}{!}{%
					\begin{tikzpicture}
  \draw (-1,1) \v -- (0,2) \v -- (1,1) \v -- (0,0) \v -- (-1,1) -- (0,1) \v -- (1,1);
\end{tikzpicture}
}%
\end{sbspanel}%
\begin{sbspanel}{0.2}[center]%
\resizebox{\linewidth}{!}{%
					\begin{tikzpicture}
  \draw (0:1) \v -- (120:1) \v -- (60:1) \v -- (300:1) \v -- (180:1) \v -- (240:1) \v -- cycle;
\end{tikzpicture}
}%
\end{sbspanel}%
\begin{sbspanel}{0.2}[center]%
\resizebox{\linewidth}{!}{%
					\begin{tikzpicture}
  \draw (360/7:1) \v -- (2*360/7:1) \v -- (3*360/7:1) \v -- (4*360/7:1) \v -- (5*360/7:1) \v -- (6*360/7:1) \v -- (0:1) \v -- cycle;
\end{tikzpicture}
}%
\end{sbspanel}%
\begin{sbspanel}{0.2}[center]%
\resizebox{\linewidth}{!}{%
					\begin{tikzpicture}
  \draw (0,0) \v;
  \foreach \x in {0,...,7}
  \draw (0,0) -- (\x*360/8+22.5:1) \v;
\end{tikzpicture}
}%
\end{sbspanel}%
\end{sidebyside}%
\par\smallskip%
\noindent\textbf{\blocktitlefont Hint}.\quad{}The first graph is bipartite, which can be seen by labeling it as follows.%
\begin{sidebyside}{1}{0.4}{0.4}{0}%
\begin{sbspanel}{0.2}%
\resizebox{\linewidth}{!}{%
          \begin{tikzpicture}
  \draw (-1,1) \vb{\footnotesize $A$} -- (0,2) \vb{\footnotesize $B$} -- (1,1) \vb{\footnotesize $A$} -- (0,0) \vb{\footnotesize $B$} -- (-1,1) -- (0,1) \vb{\footnotesize $B$} -- (1,1);
\end{tikzpicture}
}%
\end{sbspanel}%
\end{sidebyside}%
\par
Two of the remaining three are also bipartite.%
\par\smallskip%
\noindent\textbf{\blocktitlefont Solution}.\quad{}Three of the graphs are bipartite. The one which is not is \(C_7\) (second from the right). To see that the three graphs are bipartite, we can just give the bipartition into two sets \(A\) and \(B\), as labeled below:%
\begin{sidebyside}{3}{0.0616666666666667}{0.0616666666666667}{0.123333333333333}%
\begin{sbspanel}{0.2}[center]%
\resizebox{\linewidth}{!}{%
					\begin{tikzpicture}
	\draw (-1,1) \vb{\footnotesize $A$} -- (0,2) \vb{\footnotesize $B$} -- (1,1) \vb{\footnotesize $A$} -- (0,0) \vb{\footnotesize $B$} -- (-1,1) -- (0,1) \vb{\footnotesize $B$} -- (1,1);
\end{tikzpicture}
}%
\end{sbspanel}%
\begin{sbspanel}{0.23}[center]%
\resizebox{\linewidth}{!}{%
					\begin{tikzpicture}
	\draw (0:1) \vr{\footnotesize $B$} -- (120:1) \vl{\footnotesize $A$} -- (60:1) \vr{\footnotesize $B$} -- (300:1) \vr{\footnotesize $A$} -- (180:1) \vl{\footnotesize $B$} -- (240:1) \vl{\footnotesize $A$} -- cycle;
\end{tikzpicture}
}%
\end{sbspanel}%
\begin{sbspanel}{0.2}[center]%
\resizebox{\linewidth}{!}{%
					\begin{tikzpicture}
	\draw (0,0) \va{\footnotesize $B$};
	\foreach \x in {0,...,7}
	\draw (0,0) -- (\x*360/8+22.5:1) \va{\footnotesize $A$};
\end{tikzpicture}
}%
\end{sbspanel}%
\end{sidebyside}%
\par
The graph \(C_7\) is not bipartite because it is an \emph{odd} cycle. You would want to put every other vertex into the set \(A\), but if you travel clockwise in this fashion, the last vertex will also be put into the set \(A\), leaving two \(A\) vertices adjacent (which makes it not a bipartition).%
\end{divisionsolution}%
\begin{divisionsolution}{4.1.8}{}{p:exercise:EFg}%
For which \(n \ge 3\) is the graph \(C_n\) bipartite?%
\par\smallskip%
\noindent\textbf{\blocktitlefont Hint}.\quad{}\(C_4\) is bipartite; \(C_5\) is not.  What about all the other values of \(n\)?%
\par\smallskip%
\noindent\textbf{\blocktitlefont Solution}.\quad{}\(C_n\) is bipartite if and only if \(n\) or is even. Put every other vertex into the first part, the others into the second.%
\end{divisionsolution}%
\begin{divisionsolution}{4.1.9}{}{p:exercise:kMp}%
For each of the following, try to give two \emph{different} unlabeled graphs with the given properties, or explain why doing so is impossible.%
\begin{enumerate}[label=(\alph*)]
\item{}Two different trees with the same number of vertices and the same number of edges. A tree is a connected graph with no cycles.%
\item{}Two different graphs with 8 vertices all of degree 2.%
\item{}Two different graphs with 5 vertices all of degree 4.%
\item{}Two different graphs with 5 vertices all of degree 3.%
\end{enumerate}
%
\par\smallskip%
\noindent\textbf{\blocktitlefont Solution}.\quad{}%
\begin{enumerate}[label=(\alph*)]
\item{}For example:%
\begin{sidebyside}{2}{0.15}{0.15}{0.3}%
\begin{sbspanel}{0.2}[bottom]%
\resizebox{\linewidth}{!}{%
\begin{tikzpicture}
\draw (0,0) \v -- (-1,1) \v (0,0) -- (0,1) \v (0,0) -- (1,1) \v;
\end{tikzpicture}
}%
\end{sbspanel}%
\begin{sbspanel}{0.2}[bottom]%
\resizebox{\linewidth}{!}{%
\begin{tikzpicture}
\draw (0,0) \v -- (-1,1) \v (0,0) -- (.5,.5) \v -- (1,1) \v;
\end{tikzpicture}
}%
\end{sbspanel}%
\end{sidebyside}%
\item{}This is not possible if we require the graphs to be connected. If not, we could take \(C_8\) as one graph and two copies of \(C_4\) as the other.%
\item{}Not possible. If you have a graph with 5 vertices all of degree 4, then every vertex must be adjacent to every other vertex. This is the graph \(K_5\).%
\item{}This is not possible. In fact, there is not even one graph with this property (such a graph would have \(5\cdot 3/2 = 7.5\) edges).%
\end{enumerate}
%
\end{divisionsolution}%
\begin{divisionsolution}{4.1.10}{}{p:exercise:QTy}%
Decide whether the statements below about subgraphs are true or false.  For those that are true, briefly explain why (1 or 2 sentences).  For any that are false, give a counterexample.%
\begin{enumerate}[label=(\alph*)]
\item{}Any subgraph of a complete graph is also complete.%
\item{}Any \emph{induced} subgraph of a complete graph is also complete.%
\item{}Any subgraph of a bipartite graph is bipartite.%
\item{}Any subgraph of a tree is a tree.%
\end{enumerate}
%
\par\smallskip%
\noindent\textbf{\blocktitlefont Solution}.\quad{}%
\begin{multicols}{4}
\begin{enumerate}[label=(\alph*)]
\item{}False.%
\item{}True.%
\item{}True.%
\item{}False.%
\end{enumerate}
\end{multicols}
%
\end{divisionsolution}%
\begin{divisionsolution}{4.1.11}{}{p:exercise:xaH}%
Let \(k_1, k_2, \ldots,
k_j\) be a list of positive integers that sum to \(n\) (i.e, \(\sum_{i=1}^j k_i = n\)). Use two graphs containing \(n\) vertices to explain why%
\begin{equation*}
\sum_{i = 1}^j \binom{k_i}{2} \le \binom{n}{2}\text{.}
\end{equation*}
%
\par\smallskip%
\noindent\textbf{\blocktitlefont Hint}.\quad{}How many edges does \(K_n\) have?  One of the two graphs will not be connected (unless \(j=1\)).%
\par\smallskip%
\noindent\textbf{\blocktitlefont Solution}.\quad{}First consider the graph \(K_n\), which has \(\binom{n}{2}\) vertices. Compare this to a graph with \(n\) vertices consisting of \(j\) components, disconnected from each other, each a complete graph on its vertices. In fact, the first component will be \(K_{k_1}\), the second \(K_{k_2}\), and so on. The total number of edges in this graph will be \(\sum_{i=1}^j\binom{k_j}{2}\). Of course this disconnected graph will have fewer edges than \(K_n\).%
\end{divisionsolution}%
\begin{divisionsolution}{4.1.12}{}{p:exercise:dhQ}%
We often define graph theory concepts using set theory.  For example, given a graph \(G = (V, E)\) and a vertex \(v \in V\), we define%
\begin{equation*}
N(v) = \{u \in V \st \{v,u\} \in E\}\text{.}
\end{equation*}
We define \(N[v] = N(v) \cup \{v\}\).  The goal of this problem is to figure out what all this means.%
\begin{enumerate}[label=(\alph*)]
\item{}Let \(G\) be the graph with \(V = \{a,b,c,d,e,f\}\) and \(E = \{\{a,b\}, \{a,e\},\{b, c\}, \{b,e\}, \{c,d\}, \{c, f\}, \{d, f\}, \{e,f\}\}\).  Find \(N(a)\), \(N[a]\), \(N(c)\), and \(N[c]\).%
\item{}What is the largest and smallest possible values for \(|N(v)|\) and \(|N[v]|\) for the graph in part (a)?  Explain.%
\item{}Give an example of a graph \(G = (V, E)\) (probably different than the one above) for which \(N[v] = V\) for some vertex \(v \in V\).  Is there a graph for which \(N[v] = V\) for \emph{all} \(v \in V\)?  Explain.%
\item{}Give an example of a graph \(G = (V,E)\) for which \(N(v) = \emptyset\) for some \(v \in V\).  Is there an example of such a graph for which \(N[u] = V\) for some other \(u \in V\) as well?  Explain.%
\item{}Describe in words what \(N(v)\) and \(N[v]\) mean in general.%
\end{enumerate}
%
\par\smallskip%
\noindent\textbf{\blocktitlefont Hint}.\quad{}You should be able to deduce everything directly from the definition.  However, perhaps it would be helpful to know that the \(N\) stands for \emph{neighborhood}.%
\end{divisionsolution}%
\begin{divisionsolution}{4.1.13}{}{p:exercise:JoZ}%
A graph is a way of representing the relationships between elements in a set: an edge between the vertices \(x\) and \(y\) tells us that \(x\) is related to \(y\) (which we can write as \(x \sim y\)).  Not all sorts of relationships can be represented by a graph though.  For each relationship described below, either draw the graph or explain why the relationship cannot be represented by a graph.%
\begin{enumerate}[label=(\alph*)]
\item{}The set \(V = \{1,2, \ldots, 9\}\) and the relationship \(x \sim y\) when \(x-y\) is a non-zero multiple of 3.%
\item{}The set \(V = \{1,2, \ldots, 9\}\) and the relationship \(x \sim y\) when \(y\) is a multiple of \(x\).%
\item{}The set \(V = \{1,2,\ldots, 9\}\) and the relationship \(x \sim y\) when \(0 \lt |x-y| \lt 3\).%
\end{enumerate}
%
\par\smallskip%
\noindent\textbf{\blocktitlefont Hint}.\quad{}Be careful to make sure the edges are not ``directed.''  In a graph, if \(a\) is adjacent to \(b\), then \(b\) is adjacent to \(a\).  In the language of relations, we say that the edge relation is \terminology{symmetric}.%
\end{divisionsolution}%
\begin{divisionsolution}{4.1.14}{}{p:exercise:pwi}%
Consider graphs with \(n\) vertices.  Remember, graphs do not need to be \emph{connected}.%
\begin{enumerate}[label=(\alph*)]
\item{}How many edges must the graph have to guarantee at least one vertex has degree two or more?  Prove your answer.%
\item{}How many edges must the graph have to guarantee all vertices have degree two or more?  Prove your answer.%
\end{enumerate}
%
\par\smallskip%
\noindent\textbf{\blocktitlefont Hint}.\quad{}You might want to answer the questions for some specific values of \(n\) to get a feel for them, but your final answers should be in terms of \(n\).%
\end{divisionsolution}%
\begin{divisionsolution}{4.1.15}{}{p:exercise:VDr}%
Prove that any graph with at least two vertices must have two vertices of the same degree.%
\par\smallskip%
\noindent\textbf{\blocktitlefont Hint}.\quad{}Try a small example first: any graph with 8 vertices must have two vertices of the same degree.  If not, what would the degree sequence be?%
\end{divisionsolution}%
\begin{divisionsolution}{4.1.16}{}{x:exercise:exr-degree1handshake}%
Suppose \(G\) is a connected graph with \(n > 1\) vertices and \(n-1\) edges.  Prove that \(G\) has a vertex of degree 1.%
\par\smallskip%
\noindent\textbf{\blocktitlefont Hint}.\quad{}Use the handshake lemma~4.1.5.  What would happen if all the vertices had degree 2?%
\end{divisionsolution}%
\section*{4.2 Trees}
\addcontentsline{toc}{section}{4.2 Trees}
\sectionmark{4.2 Trees}
\subsection*{Exercises}
\addcontentsline{toc}{subsection}{Exercises}
\begin{divisionsolution}{4.2.1}{}{p:exercise:HzZ}%
Which of the following graphs are trees?%
\begin{enumerate}[label=(\alph*)]
\item{}\(G = (V, E)\) with \(V = \{a, b, c, d, e\}\) and \(E = \{\{a, b\}, \{a,e\}, \{b, c\}, \{c,d\}, \{d,e\} \}\)%
\item{}\(G = (V, E)\) with \(V = \{a, b, c, d, e\}\) and \(E = \{\{a, b\}, \{b, c\}, \{c,d\}, \{d,e\}\}\)%
\item{}\(G = (V, E)\) with \(V = \{a, b, c, d, e\}\) and \(E = \{\{a, b\}, \{a, c\}, \{a,d\}, \{a,e\}\}\)%
\item{}\(G = (V, E)\) with \(V = \{a, b, c, d, e\}\) and \(E = \{\{a, b\}, \{a, c\}, \{d,e\}\}\)%
\end{enumerate}
%
\par\smallskip%
\noindent\textbf{\blocktitlefont Solution}.\quad{}%
\begin{enumerate}[label=(\alph*)]
\item{}This is not a tree since it contains a cycle. Note also that there are too many edges to be a tree, since we know that all trees with \(v\) vertices have \(v-1\) edges.%
\item{}This is a tree since it is connected and contains no cycles (which you can see by drawing the graph). All paths are trees.%
\item{}This is a tree since it is connected and contains no cycles (draw the graph). All stars are trees.%
\item{}This is a not a tree since it is not connected. Note that there are not enough edges to be a tree.%
\end{enumerate}
%
\end{divisionsolution}%
\begin{divisionsolution}{4.2.2}{}{x:exercise:ex-tree-deg-seq}%
For each degree sequence below, decide whether it must always, must never, or could possibly be a degree sequence for a tree. Remember, a degree sequence lists out the degrees (number of edges incident to the vertex) of all the vertices in a graph in non-increasing order.%
\begin{enumerate}[label=(\alph*)]
\item{}\((4,1,1,1,1)\)%
\item{}\((3,3,2,1,1)\)%
\item{}\((2,2,2,1,1)\)%
\item{}\((4, 4, 3, 3, 3, 2, 2, 1, 1, 1, 1, 1, 1, 1)\)%
\end{enumerate}
%
\par\smallskip%
\noindent\textbf{\blocktitlefont Solution}.\quad{}%
\begin{enumerate}[label=(\alph*)]
\item{}This must be the degree sequence for a tree. This is because the vertex of degree 4 must be adjacent to the four vertices of degree 1 (there are no other vertices for it to be adjacent to), and thus we get a star.%
\item{}This cannot be a tree. Each degree 3 vertex is adjacent to all but one of the vertices in the graph. Thus each must be adjacent to one of the degree 1 vertices (and not the other). That means both degree 3 vertices are adjacent to the degree 2 vertex, and to each other, so that means there is a cycle.%
\par
Alternatively, count how many edges there are!%
\item{}This might or might not be a tree. The length 4 path has this degree sequence (this is a tree), but so does the union of a 3-cycle and a length 1 path (which is not connected, so not a tree).%
\item{}This cannot be a tree. The sum of the degrees is 28, so there are 14 edges. But there are 14 vertices as well, so we don't have \(v = e+1\), meaning this cannot be a tree.%
\end{enumerate}
%
\end{divisionsolution}%
\begin{divisionsolution}{4.2.3}{}{p:exercise:TOr}%
For each degree sequence below, decide whether it must always, must never, or could possibly be a degree sequence for a tree.  Justify your answers.%
\begin{enumerate}[label=(\alph*)]
\item{}\((3, 3, 2, 2, 2)\)%
\item{}\((3, 2, 2, 1, 1, 1)\)%
\item{}\((3, 3, 3, 1, 1, 1)\)%
\item{}\((4, 4, 1, 1, 1, 1, 1, 1)\)%
\end{enumerate}
%
\par\smallskip%
\noindent\textbf{\blocktitlefont Hint}.\quad{}Careful: the graphs might not be connected.%
\end{divisionsolution}%
\begin{divisionsolution}{4.2.4}{}{p:exercise:zVA}%
Suppose you have a graph with \(v\) vertices and \(e\) edges that satisfies \(v = e+1\). Must the graph be a tree? Prove your answer.%
\par\smallskip%
\noindent\textbf{\blocktitlefont Hint}.\quad{}Try Exercise~4.2.2.%
\par\smallskip%
\noindent\textbf{\blocktitlefont Solution}.\quad{}No, such a graph need not be a tree. If the graph has both a cycle and is not connected, it could still have \(v=e+1\). For example, a graph that consists of a 3-cycle together with a length 1 path will have 5 vertices and 4 edges, but is not a tree.%
\end{divisionsolution}%
\begin{divisionsolution}{4.2.5}{}{p:exercise:gcJ}%
Prove that any graph (not necessarily a tree) with \(v\) vertices and \(e\) edges that satisfies \(v \gt e+1\) will NOT be connected.%
\par\smallskip%
\noindent\textbf{\blocktitlefont Hint}.\quad{}Try a proof by contradiction and consider a spanning tree of the graph.%
\par\smallskip%
\noindent\textbf{\blocktitlefont Solution}.\quad{}Suppose there was a connected graph \(G\) satisfying \(v \gt e+1\). Let \(T\) be a spanning tree of \(G\), and let \(e'\) be the number of edges in \(T\) (of course the number of vertices in \(T\) is \(v\), the same as in \(G\)). Since \(T\) is a tree, we have that \(v = e'+1\). This means that \(e' \gt e\), but that is impossible, since a spanning tree is a subgraph.%
\end{divisionsolution}%
\begin{divisionsolution}{4.2.6}{}{p:exercise:MjS}%
If a graph \(G\) with \(v\) vertices and \(e\) edges is connected and has \(v \lt e+1\), must it contain a cycle? Prove your answer.%
\par\smallskip%
\noindent\textbf{\blocktitlefont Solution}.\quad{}Yes. We will prove the contrapositive. Assume \(G\) does not contain a cycle. Then \(G\) is a tree, so would have \(v = e+1\), contrary to stipulation.%
\end{divisionsolution}%
\begin{divisionsolution}{4.2.7}{}{x:exercise:ex-forest}%
We define a \terminology{forest} to be a graph with no cycles.%
\begin{enumerate}[label=(\alph*)]
\item{}Explain why this is a good name. That is, explain why a forest is a union of trees.%
\item{}Suppose \(F\) is a forest consisting of \(m\) trees and \(v\) vertices. How many edges does \(F\) have? Explain.%
\item{}Prove that any graph \(G\) with \(v\) vertices and \(e\) edges that satisfies \(v \lt e+1\) must contain a cycle (i.e., not be a forest).%
\end{enumerate}
%
\par\smallskip%
\noindent\textbf{\blocktitlefont Hint}.\quad{}For part (b), trying some simple examples should give you the formula.  Then you just need to prove it is correct.%
\par\smallskip%
\noindent\textbf{\blocktitlefont Solution}.\quad{}%
\begin{enumerate}[label=(\alph*)]
\item{}If a forest is connected, then because it has no cycles, it is a tree (and as such a union of trees). On the other hand, if the forest is not connected, then it is the union of some number of connected components. Each of these are connected graphs that contain no cycles, so each is a tree. Thus the forest is a union of trees.%
\item{}The total number of edges will be \(e = v - m\). The reason is that \(F\) is the union of trees, each of which consist of some number of vertices and some number of edges. In fact, each tree in the forest will have one fewer edge than it has vertices. If we add up all the vertices, we will get \(v\). If we add up all the edges, we will get \(v\) less one edge for each tree, of which there are \(m\).%
\item{}Suppose that graph \(G\) has \(m\) connected components. Assume that \(G\) has no cycles. Then \(G\) is a forest. Thus we have \(v = e+m\), contradicting the assumption that \(v \lt e+1\). Thus \(G\) cannot be a forest.%
\end{enumerate}
%
\end{divisionsolution}%
\begin{divisionsolution}{4.2.8}{}{p:exercise:Yyk}%
Give a careful proof of Corollary~4.2.2: A graph is a forest if and only if there is at most one path between any pair of vertices.  Use proof by contrapositive (and not a proof by contradiction) for both directions.%
\par\smallskip%
\noindent\textbf{\blocktitlefont Hint}.\quad{}Examining the proof of Proposition~4.2.1 gives you most of what you need, but make sure to just give the relevant parts, and take care to not use proof by contradiction.%
\end{divisionsolution}%
\begin{divisionsolution}{4.2.9}{}{p:exercise:EFt}%
Give a careful proof by induction on the number of vertices, that every tree is bipartite.%
\par\smallskip%
\noindent\textbf{\blocktitlefont Hint}.\quad{}You will need to remove a vertex of degree one, apply the inductive hypothesis to the result, and then say which set the degree one vertex to.%
\end{divisionsolution}%
\begin{divisionsolution}{4.2.10}{}{p:exercise:kMC}%
Consider the tree drawn below.%
\begin{sidebyside}{1}{0.35}{0.35}{0}%
\begin{sbspanel}{0.3}%
\resizebox{\linewidth}{!}{%
\begin{tikzpicture}
  {\footnotesize
    \coordinate (a) at (0,0);
    \coordinate (b) at (1,0);
    \coordinate (c) at (.5,1);
    \coordinate (d) at (1.5,1);
    \coordinate (e) at (2,0);
    \coordinate (f) at (3,0);
    \coordinate (g) at (2.5,1);
    \coordinate (h) at (3.5,1);
    \coordinate (i) at (4,0);
    \draw (a) -- (b) -- (c) (b) -- (d) (b) -- (e) -- (f) -- (g) (f) -- (h) (f) -- (i);
    \foreach \x in {a,...,i}{
    \draw (\x) \vb{\footnotesize $\x$};
    }
  }
\end{tikzpicture}
}%
\end{sbspanel}%
\end{sidebyside}%
\par
%
\begin{enumerate}[label=(\alph*)]
\item{}Suppose we designate vertex \(e\) as the root.  List the children, parents and siblings of each vertex.  Does any vertex other than \(e\) have grandchildren?%
\item{}Suppose \(e\) is \emph{not} chosen as the root.  Does our choice of root vertex change the \emph{number} of children \(e\) has?  The number of grandchildren?  How many are there of each?%
\item{}In fact, pick any vertex in the tree and suppose it is not the root.  Explain why the number of children of that vertex does not depend on which other vertex is the root.%
\item{}Does the previous part work for other trees?  Give an example of a different tree for which it holds.  Then either prove that it always holds or give an example of a tree for which it doesn't.%
\end{enumerate}
%
\par\smallskip%
\noindent\textbf{\blocktitlefont Hint}.\quad{}If \(e\) is the root, then \(b\) will have three children (\(a\), \(c\), and \(d\)), all of which will be siblings, and have \(b\) as their parent.  \(a\) will not have any children.%
\par
In general, how can you determine the number of children a vertex will have, if it is not a root?%
\end{divisionsolution}%
\begin{divisionsolution}{4.2.11}{}{p:exercise:QTL}%
Let \(T\) be a rooted tree that contains vertices \(u\), \(v\), and \(w\) (among possibly others). Prove that if \(w\) is a descendant of both \(u\) and \(v\), then \(u\) is a descendant of \(v\) or \(v\) is a descendant of \(u\).%
\end{divisionsolution}%
\begin{divisionsolution}{4.2.12}{}{p:exercise:xaU}%
Unless it is already a tree, a given graph \(G\) will have multiple spanning trees. How similar or different must these be?%
\begin{enumerate}[label=(\alph*)]
\item{}Must all spanning trees of a given graph be isomorphic to each other? Explain why or give a counterexample.%
\item{}Must all spanning trees of a given graph have the same number of edges? Explain why or give a counterexample.%
\item{}Must all spanning trees of a graph have the same number of leaves (vertices of degree 1)? Explain why or give a counterexample.%
\end{enumerate}
%
\par\smallskip%
\noindent\textbf{\blocktitlefont Solution}.\quad{}%
\begin{enumerate}[label=(\alph*)]
\item{}No, although there are graphs for which this is true. For example, \(K_4\) has a spanning tree that is a path (of three edges) and also a spanning tree that is a star (with center vertex of degree 3).%
\item{}Yes. For a fixed graph, we have a fixed number \(v\) of vertices. Any spanning tree of the graph will also have \(v\) vertices, and since it is a tree, must have \(v-1\) edges.%
\item{}No, although there are graph for which this is true (note that if all spanning trees are isomorphic, then all spanning trees will have the same number of leaves). Again, \(K_4\) is a counterexample. One spanning tree is a path, with only two leaves, another spanning tree is a star with 3 leaves.%
\end{enumerate}
%
\end{divisionsolution}%
\begin{divisionsolution}{4.2.13}{}{p:exercise:did}%
Find all spanning trees of the graph below. How many different spanning trees are there? How many different spanning trees are there \emph{up to isomorphism} (that is, if you grouped all the spanning trees by which are isomorphic, how many groups would you have)?%
\begin{sidebyside}{1}{0.35}{0.35}{0}%
\begin{sbspanel}{0.3}%
\resizebox{\linewidth}{!}{%
\begin{tikzpicture}
  \footnotesize{
  \draw (0,0) -- (-1,.75) \vl{$a$} -- (-1,-.75) \vl{$b$} -- (0,0) \vb{$c$} -- (1,1) \vb{$d$} -- (2,0) \vr{$e$} -- (1,-1) \va{$f$} -- (0,0) -- (2,0);
  }
\end{tikzpicture}
}%
\end{sbspanel}%
\end{sidebyside}%
\end{divisionsolution}%
\begin{divisionsolution}{4.2.14}{}{p:exercise:Jpm}%
Give an example of a graph that has exactly 7 different spanning trees. Note, it acceptable for some or all of these spanning trees to be isomorphic.%
\par\smallskip%
\noindent\textbf{\blocktitlefont Hint}.\quad{}There is an example with 7 edges.%
\par\smallskip%
\noindent\textbf{\blocktitlefont Solution}.\quad{}There are more complicated ways of doing this, but one example is to start with the cycle \(C_7\). Any edge that you remove will result in a tree (in fact a path), and there are 7 edges to choose from. Thus there are exactly 7 spanning trees of this graph (although they are all isomorphic).%
\end{divisionsolution}%
\begin{divisionsolution}{4.2.15}{}{p:exercise:pwv}%
Prove that every connected graph which is not itself a tree must have at last three different (although possibly isomorphic) spanning trees.%
\par\smallskip%
\noindent\textbf{\blocktitlefont Hint}.\quad{}The previous exercise will be helpful.%
\par\smallskip%
\noindent\textbf{\blocktitlefont Solution}.\quad{}If a connected graph is not a tree, then it has a cycle. That cycle must consist of at least three edges, call them \(e_1\), \(e_2\), and \(e_3\). Removing any one of these three edges cannot disconnect the graph, so we can start to form our spanning tree in at least three ways, and then be sure to include both of the other two edges. For example, say we start with the graph \(G\). First, remove \(e_1\), to get a graph \(G'\) that is still connected. Now build a spanning tree that includes \(e_2\) and \(e_3\) by putting those two edges into the spanning tree first (putting those two edges in cannot create a cycle). Then keep adding edges until you get a spanning tree.%
\end{divisionsolution}%
\begin{divisionsolution}{4.2.16}{}{p:exercise:VDE}%
Consider edges that must be in every spanning tree of a graph. Must every graph have such an edge? Give an example of a graph that has exactly one such edge.%
\par\smallskip%
\noindent\textbf{\blocktitlefont Hint}.\quad{}Note that such an edge, if removed, would disconnect the graph. We call graphs that have an edge like this \terminology{1-connected}.%
\par\smallskip%
\noindent\textbf{\blocktitlefont Solution}.\quad{}Note that such an edge, if removed, would disconnect the graph. A cycle does not have an edge like this. In fact, we call graphs that have an edge like this \terminology{1-connected}.%
\par
There are lots of examples of graphs with only one edge like this. In fact, we could take just a single edge as our graph. Or something like this:%
\begin{sidebyside}{1}{0.35}{0.35}{0}%
\begin{sbspanel}{0.3}%
\resizebox{\linewidth}{!}{%
\begin{tikzpicture}
  \draw (-1,0) \v -- (-2,1) \v -- (-2,-1) \v -- (-1,0) -- (1,0) \v -- (2,1) \v -- (2,-1) \v -- (1,0);
\end{tikzpicture}
}%
\end{sbspanel}%
\end{sidebyside}%
\end{divisionsolution}%
\section*{4.3 Planar Graphs}
\addcontentsline{toc}{section}{4.3 Planar Graphs}
\sectionmark{4.3 Planar Graphs}
\subsection*{Exercises}
\addcontentsline{toc}{subsection}{Exercises}
\begin{divisionsolution}{4.3.1}{}{p:exercise:Vpj}%
Is it possible for a planar graph to have 6 vertices, 10 edges and 5 faces? Explain.%
\par\smallskip%
\noindent\textbf{\blocktitlefont Solution}.\quad{}No. A (connected) planar graph must satisfy Euler's formula: \(v - e + f = 2\). Here \(v - e + f = 6 - 10 + 5 = 1\).%
\end{divisionsolution}%
\begin{divisionsolution}{4.3.2}{}{p:exercise:Bws}%
The graph \(G\) has 6 vertices with degrees \(2, 2, 3, 4, 4, 5\). How many edges does \(G\) have? Could \(G\) be planar? If so, how many faces would it have. If not, explain.%
\par\smallskip%
\noindent\textbf{\blocktitlefont Solution}.\quad{}\(G\) has 10 edges, since \(10 = \frac{2+2+3+4+4+5}{2}\). It could be planar, and then it would have 6 faces, using Euler's formula: \(6-10+f = 2\) means \(f = 6\). To make sure that it is actually planar though, we would need to draw a graph with those vertex degrees without edges crossing. This can be done by trial and error (and is possible).%
\end{divisionsolution}%
\begin{divisionsolution}{4.3.3}{}{p:exercise:hDB}%
Is it possible for a connected graph with 7 vertices and 10 edges to be drawn so that no edges cross and create 4 faces?  Explain.%
\par\smallskip%
\noindent\textbf{\blocktitlefont Hint}.\quad{}What would Euler's formula tell you?%
\end{divisionsolution}%
\begin{divisionsolution}{4.3.4}{}{p:exercise:NKK}%
Is it possible for a graph with 10 vertices and edges to be a connected planar graph?  Explain.%
\end{divisionsolution}%
\begin{divisionsolution}{4.3.5}{}{p:exercise:tRT}%
Is there a connected planar graph with an odd number of faces where every vertex has degree 6?  Prove your answer.%
\par\smallskip%
\noindent\textbf{\blocktitlefont Hint}.\quad{}You can use the handshake lemma to find the number of edges, in terms of \(v\), the number of vertices.%
\end{divisionsolution}%
\begin{divisionsolution}{4.3.6}{}{p:exercise:ZZc}%
I'm thinking of a polyhedron containing 12 faces. Seven are triangles and four are quadralaterals. The polyhedron has 11 vertices including those around the mystery face. How many sides does the last face have?%
\par\smallskip%
\noindent\textbf{\blocktitlefont Solution}.\quad{}Say the last polyhedron has \(n\) edges, and also \(n\) vertices. The total number of edges the polyhedron has then is \((7 \cdot 3 + 4 \cdot 4 + n)/2 = (37 + n)/2\). In particular, we know the last face must have an odd number of edges. We also have that \(v = 11 \). By Euler's formula, we have \(11 - (37+n)/2 + 12 = 2\), and solving for \(n\) we get \(n = 5\), so the last face is a pentagon.%
\end{divisionsolution}%
\begin{divisionsolution}{4.3.7}{}{p:exercise:Ggl}%
Consider some classic polyhedrons.%
\begin{enumerate}[label=(\alph*)]
\item{}An \emph{octahedron} is a regular polyhedron made up of 8 equilateral triangles (it sort of looks like two pyramids with their bases glued together). Draw a planar graph representation of an octahedron. How many vertices, edges and faces does an octahedron (and your graph) have? %
\item{}The traditional design of a soccer ball is in fact a (spherical projection of a) truncated icosahedron. This consists of 12 regular pentagons and 20 regular hexagons. No two pentagons are adjacent (so the edges of each pentagon are shared only by hexagons). How many vertices, edges, and faces does a truncated icosahedron have? Explain how you arrived at your answers. Bonus: draw the planar graph representation of the truncated icosahedron. %
\item{}Your ``friend'' claims that he has constructed a convex polyhedron out of 2 triangles, 2 squares, 6 pentagons and 5 octagons. Prove that your friend is lying. Hint: each vertex of a convex polyhedron must border at least three faces. %
\end{enumerate}
%
\end{divisionsolution}%
\begin{divisionsolution}{4.3.8}{}{p:exercise:mnu}%
Prove Euler's formula using induction on the number of edges in the graph.%
\par\smallskip%
\noindent\textbf{\blocktitlefont Solution}.\quad{}\begin{solutionproof}
Let \(P(n)\) be the statement, ``every connected planar graph containing \(n\) edges satisfies \(v - n + f = 2\).'' We will show \(P(n)\) is true for all \(n \ge 0\).%
\par
Base case: there is only one graph with zero edges, namely a single isolated vertex. In this case \(v = 1\), \(f = 1\) and \(e = 0\), so Euler's formula holds.%
\par
Inductive case: Suppose \(P(k)\) is true for some arbitrary \(k \ge 0\). Now consider an arbitrary graph containing \(k+1\) edges (and \(v\) vertices and \(f\) faces). No matter what this graph looks like, we can remove a single edge to get a graph with \(k\) edges which we can apply the inductive hypothesis to.%
\par
There are two cases: either the graph contains a cycle or it does not. If the graph contains a cycle, then pick an edge that is part of this cycle, and remove it.  This will not disconnect the graph, and will decrease the number of faces by 1 (since the edge was bordering two distinct faces).  So by the inductive hypothesis we will have \(v - k + f-1 = 2\). Adding the edge back will give \(v - (k+1) + f = 2\) as needed.%
\par
If the graph does not contain a cycle, then it is a tree, so has a vertex of degree 1.  Then we can pick the edge to remove to be incident to such a degree 1 vertex. In this case, also remove that vertex. The smaller graph will now satisfy \(v-1 - k + f = 2\) by the induction hypothesis (removing the edge and vertex did not reduce the number of faces). Adding the edge and vertex back gives \(v - (k+1) + f = 2\), as required.%
\par
Therefore, by the principle of mathematical induction, Euler's formula holds for all planar graphs.%
\end{solutionproof}
\end{divisionsolution}%
\begin{divisionsolution}{4.3.9}{}{p:exercise:SuD}%
Prove Euler's formula using induction on the number of \emph{vertices} in the graph.%
\end{divisionsolution}%
\begin{divisionsolution}{4.3.10}{}{p:exercise:yBM}%
Euler's formula (\(v - e + f = 2\)) holds for all \emph{connected} planar graphs. What if a graph is not connected? Suppose a planar graph has two components. What is the value of \(v - e + f\) now? What if it has \(k\) components?%
\par\smallskip%
\noindent\textbf{\blocktitlefont Solution}.\quad{}Say the first component has \(v_1\) vertices, \(e_1\) edges and \(f_1\) faces. The second graph has \(v_2\) vertices, \(e_2\) edges and \(f_2\) faces. Thinking of each of these separately, we have%
\begin{equation*}
v_1 - e_1 + f_1 = 2\text{,}
\end{equation*}
%
%
\begin{equation*}
v_2 - e_2 + f_2 = 2\text{.}
\end{equation*}
Adding these two equations gives%
\begin{equation*}
v - e + f = 4
\end{equation*}
%
\par
(since the graph has \(v = v_1 + v_2\) vertices, etc). However, the two components have one common face (the outside of one of them must be contained in one of the faces of the other) so in fact we get%
\begin{equation*}
v - e + f = 3\text{.}
\end{equation*}
%
\par
In general, a planar graph with \(k\) components will satisfy \(v - e + f = 1 + k\).%
\end{divisionsolution}%
\begin{divisionsolution}{4.3.11}{}{p:exercise:eIV}%
Prove that the \emph{Petersen graph} \index{Petersen graph} (below) is not planar.%
\begin{sidebyside}{1}{0.36}{0.36}{0}%
\begin{sbspanel}{0.28}%
\resizebox{\linewidth}{!}{%
          \begin{tikzpicture}[scale=.7]
  \draw[thick] (18:2) -- (90:2) -- (162:2)  -- (234:2) -- (306:2) -- cycle;
  \draw[thick] (18:1) --  (162:1)  -- (306:1) -- (90:1) -- (234:1) --cycle;
  \foreach \x in {18, 90, 162, 234, 306}
  \draw[thick] (\x:1) \v -- (\x:2) \v;
\end{tikzpicture}
}%
\end{sbspanel}%
\end{sidebyside}%
\par\smallskip%
\noindent\textbf{\blocktitlefont Hint}.\quad{}What is the length of the shortest cycle? (This quantity is usually called the \terminology{girth} of the graph.)%
\par\smallskip%
\noindent\textbf{\blocktitlefont Solution}.\quad{}\begin{solutionproof}
Suppose, for contradiction, that the Petersen graph were planar. Then it would satisfy Euler's formula: \(V - E + F = 2\). Since the graph has 10 vertices and 15 edges, this says that there must be \(7\) faces.%
\par
Now let \(B\) be the total number of boundaries around all faces when the graph is drawn in a planar way. Since each edge is used in two boundaries we have \(B = 2E\). On the other hand, each face is surrounded by \emph{at least} 5 boundaries, since the shortest cycle (circuit) in the graph contains 5 edges. Thus \(F \le \frac{B}{5}\). Putting these two facts together we get%
\begin{equation*}
F \le \frac{2E}{5}
\end{equation*}
%
\par
This is a contradiction, since \(7 \not\le \frac{2\cdot 30}{6}\). Alternatively, the above relationship says that \(F \le 6\), but we said \(F = 7\) above.%
\par
Therefore the Petersen graph is not planar.%
\end{solutionproof}
\end{divisionsolution}%
\begin{divisionsolution}{4.3.12}{}{p:exercise:KQe}%
Prove that any planar graph with \(v\) vertices and \(e\) edges satisfies \(e \le 3v - 6\).%
\par\smallskip%
\noindent\textbf{\blocktitlefont Solution}.\quad{}\begin{solutionproof}
We know in any planar graph the number of faces \(f\) satisfies \(3f \le 2e\) since each face is bounded by at least three edges, but each edge borders two faces. Combine this with Euler's formula:%
\begin{equation*}
v - e + f = 2
\end{equation*}
%
\begin{equation*}
v - e + \frac{2e}{3} \ge 2
\end{equation*}
%
\begin{equation*}
3v - e \ge 6
\end{equation*}
%
\begin{equation*}
3v - 6 \ge e\text{.}\qedhere
\end{equation*}
%
\end{solutionproof}
\end{divisionsolution}%
\begin{divisionsolution}{4.3.13}{}{p:exercise:qXn}%
Prove that any planar graph must have a vertex of degree 5 or less.%
\end{divisionsolution}%
\begin{divisionsolution}{4.3.14}{}{x:exercise:exr-grotzsch-nonplanar}%
Give a careful proof that the graph below is not planar.%
\begin{sidebyside}{1}{0.35}{0.35}{0}%
\begin{sbspanel}{0.3}%
\resizebox{\linewidth}{!}{%
\begin{tikzpicture}[scale=0.8]
  \foreach \x in {0,...,4}{
\draw (0,0) \v -- (90+\x*72:1) \v -- (162+\x*72:2) \v -- (90+\x*72:2) -- (162+\x*72:1);
}
\end{tikzpicture}
}%
\end{sbspanel}%
\end{sidebyside}%
\par\smallskip%
\noindent\textbf{\blocktitlefont Hint}.\quad{}The girth of the graph is 4.%
\end{divisionsolution}%
\begin{divisionsolution}{4.3.15}{}{p:exercise:DlF}%
Explain why we cannot use the same sort of proof we did in Exercise~4.3.14 to prove that the graph below is not planar.  Then explain how you know the graph is not planar anyway.%
\begin{sidebyside}{1}{0.35}{0.35}{0}%
\begin{sbspanel}{0.3}%
\resizebox{\linewidth}{!}{%
\begin{tikzpicture}[scale=0.8]
  \foreach \x in {0,...,4}{
  \draw (0,0) \v -- (90+\x*72:1) \v -- (162+\x*72:2) \v -- (90+\x*72:2) -- (162+\x*72:1);
  \draw[black!80!blue] (90+\x*72:1) -- (162+\x*72:1);
  }
\end{tikzpicture}
}%
\end{sbspanel}%
\end{sidebyside}%
\par\smallskip%
\noindent\textbf{\blocktitlefont Hint}.\quad{}What has happened to the girth?  Careful: we have a different number of edges as well.  Better check Euler's formula.%
\end{divisionsolution}%
\section*{4.4 Coloring}
\addcontentsline{toc}{section}{4.4 Coloring}
\sectionmark{4.4 Coloring}
\subsection*{Exercises}
\addcontentsline{toc}{subsection}{Exercises}
\begin{divisionsolution}{4.4.1}{}{p:exercise:dhD}%
What is the smallest number of colors you need to properly color the vertices of \(K_{4,5}\)? That is, find the chromatic number of the graph.%
\par\smallskip%
\noindent\textbf{\blocktitlefont Solution}.\quad{}2, since the graph is bipartite. One color for the top set of vertices, another color for the bottom set of vertices.%
\end{divisionsolution}%
\begin{divisionsolution}{4.4.2}{}{p:exercise:JoM}%
Draw a graph with chromatic number 6 (i.e., which requires 6 colors to properly color the vertices). Could your graph be planar? Explain.%
\par\smallskip%
\noindent\textbf{\blocktitlefont Solution}.\quad{}For example, \(K_6\). If the chromatic number is 6, then the graph is not planar; the 4-color theorem states that all planar graphs can be colored with 4 or fewer colors.%
\end{divisionsolution}%
\begin{divisionsolution}{4.4.3}{}{p:exercise:pvV}%
Find the chromatic number of each of the following graphs.%
\begin{sidebyside}{3}{0.0916666666666667}{0.0916666666666667}{0.183333333333333}%
\begin{sbspanel}{0.15}[bottom]%
\resizebox{\linewidth}{!}{%
\begin{tikzpicture}
	    \draw  (-1,1) \v -- (0,2) \v -- (1,1) \v -- (0,0) \v -- (-1,1) -- (0,1) \v -- (1,1);
	  \end{tikzpicture}
}%
\end{sbspanel}%
\begin{sbspanel}{0.15}[bottom]%
\resizebox{\linewidth}{!}{%
\begin{tikzpicture}
	    \draw  (360/7:1) \v -- (2*360/7:1) \v -- (3*360/7:1) \v -- (4*360/7:1) \v -- (5*360/7:1) \v -- (6*360/7:1) \v -- (0:1) \v -- cycle;
	  \end{tikzpicture}
}%
\end{sbspanel}%
\begin{sbspanel}{0.15}[bottom]%
\resizebox{\linewidth}{!}{%
\begin{tikzpicture}
	    \draw (0,0) \v;
	    \foreach \x in {0,...,4}
	    \draw  (0,0) -- (\x*360/5:1) \v -- (\x*360/5+72:1);
	  \end{tikzpicture}
}%
\end{sbspanel}%
\end{sidebyside}%
\begin{sidebyside}{2}{0.15}{0.15}{0.3}%
\begin{sbspanel}{0.2}%
\resizebox{\linewidth}{!}{%
\begin{tikzpicture}
	    \foreach \x in {0,...,4}
	    \draw  (\x*72+18:1) \v -- (\x*72+90:1) -- (\x*72-54:1);
	  \end{tikzpicture}
}%
\end{sbspanel}%
\begin{sbspanel}{0.2}%
\resizebox{\linewidth}{!}{%
\begin{tikzpicture}[scale=.5]
	    \draw  (18:2) -- (90:2) -- (162:2)  -- (234:2) -- (306:2) -- cycle;
	    \draw  (18:1) --  (162:1)  -- (306:1) -- (90:1) -- (234:1) --cycle;
	    \foreach \x in {18, 90, 162, 234, 306}
	    \draw  (\x:1) \v -- (\x:2) \v;
	  \end{tikzpicture}
}%
\end{sbspanel}%
\end{sidebyside}%
\par\smallskip%
\noindent\textbf{\blocktitlefont Solution}.\quad{}The chromatic numbers are 2, 3, 4, 5, and 3 respectively from left to right.%
\end{divisionsolution}%
\begin{divisionsolution}{4.4.4}{}{p:exercise:VDe}%
A group of 10 friends decides to head up to a cabin in the woods (where nothing could possibly go wrong). Unfortunately, a number of these friends have dated each other in the past, and things are still a little awkward. To get to the cabin, they need to divide up into some number of cars, and no two people who dated should be in the same car.%
\begin{enumerate}[label=(\alph*)]
\item{}What is the smallest number of cars you need if all the relationships were strictly heterosexual? Represent an example of such a situation with a graph. What kind of graph do you get? %
\item{}Because a number of these friends dated there are also conflicts between friends of the same gender, listed below. Now what is the smallest number of conflict-free cars they could take to the cabin?%
\begin{sidebyside}{1}{0.05}{0.05}{0}%
\begin{sbspanel}{0.9}%
{\centering%
{\tabularfont%
\begin{tabular}{lllllllllll}
\multicolumn{1}{lA}{Friend}&A&B&C&D&E&F&G&H&I&J\tabularnewline\hrulethin
\multicolumn{1}{lA}{Conflicts}&CFJ&J&AEF&H&CFG&ACEGI&EFI&D&AFG&B
\end{tabular}
}%
\par}
\end{sbspanel}%
\end{sidebyside}%
\end{enumerate}
%
\end{divisionsolution}%
\begin{divisionsolution}{4.4.5}{}{p:exercise:BKn}%
What is the smallest number of colors that can be used to color the vertices of a cube so that no two adjacent vertices are colored identically?%
\par\smallskip%
\noindent\textbf{\blocktitlefont Solution}.\quad{}The cube can be represented as a planar graph and colored with two colors as follows:%
\begin{sidebyside}{1}{0.4}{0.4}{0}%
\begin{sbspanel}{0.2}%
\resizebox{\linewidth}{!}{%
\begin{tikzpicture}
			\foreach \ang in {45, 135, 225, 315} {
			\draw (\ang:.4) \v -- (\ang:1) \v -- (\ang+90:1) (\ang:.4) -- (\ang+90:.4);
			}
			\draw (45:.4) node[right]{\footnotesize R} (135:.4) node[left]{\footnotesize B} (225:.4) node[left]{\footnotesize R} (315:.4) node[right]{\footnotesize B} (45:1) node[right]{\footnotesize B} (135:1) node[left]{\footnotesize R} (225:1) node[left]{\footnotesize B} (315:1) node[right]{\footnotesize R};
			\end{tikzpicture}
}%
\end{sbspanel}%
\end{sidebyside}%
\par
Since it would be impossible to color the vertices with a single color, we see that the cube has chromatic number 2 (it is bipartite).%
\end{divisionsolution}%
\begin{divisionsolution}{4.4.6}{}{p:exercise:hRw}%
Prove the chromatic number of any tree is two. Recall, a tree is a connected graph with no cycles.%
\begin{enumerate}[label=(\alph*)]
\item{}Describe a procedure to color the tree below.%
\begin{sidebyside}{1}{0.325}{0.325}{0}%
\begin{sbspanel}{0.35}%
\resizebox{\linewidth}{!}{%
\begin{tikzpicture}[scale=.7]
\draw (-1,1) \v -- (-1.5,1.5) \v (-1,1) -- (-.9,2)\v (-1,1) -- (-.2,.5) \v (-1,1) -- (-.5,.2)\v;
\draw (-1,1) -- (1,1) \v -- (.5,1.3) \v (1,1) -- (1,2)\v (1,1) -- (1.5,1.3) \v (1,1) -- (1.5,1)\v;
\draw (1,1) -- (0,0) \v -- (-.2,-1) \v (0,0) -- (.2,-1) \v (0,0) -- (2,0) \v -- (2,1) \v -- (2.5,1.5) \v (2,1) -- (2.4,1.1) \v;
\draw (2,0) -- (2,-1) \v -- (2.4,-.5) \v (2,-1) -- (2.4, -1.4) \v (2,-1) -- (2,-1.7) \v (2,-1) -- (1.6,-1.4) \v (2,-1) -- (1.6,-.6) \v;
\draw (2,0) -- (3,1) \v -- (3,1.5) \v -- (3.5,2)\v (3,1) -- (3,.5) \v -- (3.5, 0) \v;
\end{tikzpicture}
}%
\end{sbspanel}%
\end{sidebyside}%
\item{}The chromatic number of \(C_n\) is two when \(n\) is even. What goes wrong when \(n\) is odd?%
\item{}Prove that your procedure from part (a) always works for any tree. %
\item{}Now, prove using induction that every tree has chromatic number 2. %
\end{enumerate}
%
\end{divisionsolution}%
\begin{divisionsolution}{4.4.7}{}{p:exercise:NYF}%
The two problems below can be solved using graph coloring. For each problem, represent the situation with a graph, say whether you should be coloring vertices or edges and why, and use the coloring to solve the problem.%
\begin{enumerate}[label=(\alph*)]
\item{}Your Quidditch league has 5 teams. You will play a tournament next week in which every team will play every other team once. Each team can play at most one match each day, but there is plenty of time in the day for multiple matches. What is the fewest number of days over which the tournament can take place?%
\item{}Ten members of Math Club are driving to a math conference in a neighboring state. However, some of these students have dated in the past, and things are still a little awkward. Each student lists which other students they refuse to share a car with; these conflicts are recored in the table below. What is the fewest number of cars the club needs to make the trip? Do not worry about running out of seats, just avoid the conflicts.%
\begin{sidebyside}{1}{0.1}{0.1}{0}%
\begin{sbspanel}{0.8}%
{\centering%
{\tabularfont%
\begin{tabular}{ccccccccccc}
\multicolumn{1}{lB}{Student:}&A&B&C&D&E&F&G&H&I&J\tabularnewline\hrulemedium
\multicolumn{1}{lB}{Conflicts:}&BEJ&ADG&HJ&BF&AI&DJ&B&CI&EHJ&ACFI
\end{tabular}
}%
\par}
\end{sbspanel}%
\end{sidebyside}%
\end{enumerate}
%
\par\smallskip%
\noindent\textbf{\blocktitlefont Hint}.\quad{}For (a), you will want the teams to be vertices and games to be edges.  Which does it make sense to color?%
\par\smallskip%
\noindent\textbf{\blocktitlefont Solution}.\quad{}%
\begin{enumerate}[label=(\alph*)]
\item{}The graph to represent this question is \(K_5\), since each vertex (team) is adjacent to (plays) each other vertex (team). The edges are thus the games that are played. We cannot have a team play more than one game per day, so we color the edges observing the rule that two edges incident to the same vertex must be colored differently. Edges that are colored the same \emph{can} be played on the same day, so we are looking for the smallest number of colors needed to color the edges in this way. You obviously need at least 4 colors, but this does not work. In fact, there is a coloring using 5 colors, so you need 5 days for the tournament.%
\item{}Here we color the vertices. The chromatic number of this graph is 3, so three cars are needed.%
\end{enumerate}
%
\end{divisionsolution}%
\begin{divisionsolution}{4.4.8}{}{p:exercise:ufO}%
Prove the 6-color theorem: every planar graph has chromatic number 6 or less. Do not assume the 4-color theorem (whose proof is MUCH harder), but you may assume the fact that every planar graph contains a vertex of degree at most 5.%
\end{divisionsolution}%
\begin{divisionsolution}{4.4.9}{}{p:exercise:amX}%
Not all graphs are perfect. Give an example of a graph with chromatic number 4 that does not contain a copy of \(K_4\). That is, there should be no 4 vertices all pairwise adjacent.%
\par\smallskip%
\noindent\textbf{\blocktitlefont Solution}.\quad{}The wheel graph below has this property. The outside of the wheel forms an odd cycle, so requires 3 colors, the center of the wheel must be different than all the outside vertices.%
\begin{sidebyside}{1}{0.4}{0.4}{0}%
\begin{sbspanel}{0.2}%
\resizebox{\linewidth}{!}{%
\begin{tikzpicture}

			\foreach \ang in {18, 90, ..., 306}{
			\draw (0,0) -- (\ang:1) \v -- (\ang+72:1);
			}
			\draw (0,0) \v;
			\end{tikzpicture}
}%
\end{sbspanel}%
\end{sidebyside}%
\end{divisionsolution}%
\begin{divisionsolution}{4.4.10}{}{p:exercise:Gug}%
Find the chromatic number of the graph below and prove you are correct.%
\begin{sidebyside}{1}{0.35}{0.35}{0}%
\begin{sbspanel}{0.3}%
\resizebox{\linewidth}{!}{%
\begin{tikzpicture}[scale=0.8]
  \foreach \x in {0,...,4}{
\draw (0,0) \v -- (90+\x*72:1) \v -- (162+\x*72:2) \v -- (90+\x*72:2) -- (162+\x*72:1);
}
\end{tikzpicture}
}%
\end{sbspanel}%
\end{sidebyside}%
\par\smallskip%
\noindent\textbf{\blocktitlefont Hint}.\quad{}The chromatic number is 4.  Now prove this!%
\par
Note that you cannot use the 4-color theorem, or Brooke's theorem, or the clique number here.  In fact, this graph, called the \emph{Grötzsch graph} is the smallest graph with chromatic number 4 that does not contain any triangles.%
\end{divisionsolution}%
\begin{divisionsolution}{4.4.11}{}{p:exercise:mBp}%
Prove by induction on vertices that any graph \(G\) which contains at least one vertex of degree less than \(\Delta(G)\) (the maximal degree of all vertices in \(G\)) has chromatic number at most \(\Delta(G)\).%
\end{divisionsolution}%
\begin{divisionsolution}{4.4.12}{}{p:exercise:SIy}%
You have a set of magnetic alphabet letters (one of each of the 26 letters in the alphabet) that you need to put into boxes. For obvious reasons, you don't want to put two consecutive letters in the same box. What is the fewest number of boxes you need (assuming the boxes are able to hold as many letters as they need to)?%
\par\smallskip%
\noindent\textbf{\blocktitlefont Solution}.\quad{}If we drew a graph with each letter representing a vertex, and each edge connecting two letters that were consecutive in the alphabet, we would have a graph containing two vertices of degree 1 (A and Z) and the remaining 24 vertices all of degree 2 (for example, \(D\) would be adjacent to both \(C\) and \(E\)). By Brooks' theorem, this graph has chromatic number at most 2, as that is the maximal degree in the graph and the graph is not a complete graph or odd cycle. Thus only two boxes are needed.%
\end{divisionsolution}%
\begin{divisionsolution}{4.4.13}{}{p:exercise:yPH}%
Suppose you colored edges of a graph either red or blue (not requiring that adjacent edges be colored differently).  What must be true of the graph to guarantee some vertex is incident to three edges of the same color?  Prove your answer.%
\par\smallskip%
\noindent\textbf{\blocktitlefont Hint}.\quad{}You can color \(K_5\) in such a way that every vertex is adjacent to exactly two blue edges and two red edges.  However, there is a graph with only 5 edges that will result in a vertex incident to three edges of the same color no matter how they are colored.  What is it, and how can you generalize?%
\par\smallskip%
\noindent\textbf{\blocktitlefont Solution}.\quad{}%
\end{divisionsolution}%
\begin{divisionsolution}{4.4.14}{}{p:exercise:eWQ}%
Prove that if you color every edge of \(K_6\) either red or blue, you are guaranteed a monochromatic triangle (that is, an all red or an all blue triangle).%
\par\smallskip%
\noindent\textbf{\blocktitlefont Hint}.\quad{}The previous exercise is useful as a starting point.%
\par\smallskip%
\noindent\textbf{\blocktitlefont Solution}.\quad{}Fix a vertex \(v\) of \(K_6\) and consider its five incident edges.  At least three of these must be colored identically (either all red or all blue).  Without loss of generality, say three are colored blue.  Let \(v_1, v_2, v_3\) be the vertices also incident to these blue edges.  If any edge between any pair of these three vertices is blue, then we have a blue triangle.  If not, then all three of these edges are red, and we have a red triangle.%
\end{divisionsolution}%
\section*{4.5 Euler Paths and Circuits}
\addcontentsline{toc}{section}{4.5 Euler Paths and Circuits}
\sectionmark{4.5 Euler Paths and Circuits}
\subsection*{Exercises}
\addcontentsline{toc}{subsection}{Exercises}
\begin{divisionsolution}{4.5.1}{}{p:exercise:Nwp}%
You and your friends want to tour the southwest by car. You will visit the nine states below, with the following rather odd rule: you must cross each border between neighboring states exactly once (so, for example, you must cross the Colorado-Utah border exactly once). Can you do it? If so, does it matter where you start your road trip? What fact about graph theory solves this problem?%
\begin{sidebyside}{1}{0.3}{0.3}{0}%
\begin{sbspanel}{0.4}%
\resizebox{\linewidth}{!}{%
            \begin{tikzpicture}[scale=.25]
\USA[every state={draw=white, line width = .7pt, fill=black!10}, CA={fill=gray}, NV={fill=gray},NM={fill=gray},AZ={fill=gray},UT={fill=gray},CO={fill=gray},TX={fill=gray},KS={fill=gray},OK={fill=gray}];
\end{tikzpicture}
}%
\end{sbspanel}%
\end{sidebyside}%
\par\smallskip%
\noindent\textbf{\blocktitlefont Solution}.\quad{}This is a question about finding Euler paths. Draw a graph with a vertex in each state, and connect vertices if their states share a border. Exactly two vertices will have odd degree: the vertices for Nevada and Utah. Thus you must start your road trip at in one of those states and end it in the other.%
\end{divisionsolution}%
\begin{divisionsolution}{4.5.2}{}{p:exercise:tDy}%
Which of the following graphs contain an Euler path? Which contain an Euler circuit?%
\begin{multicols}{6}
\begin{enumerate}[label=(\alph*)]
\item{}\(\displaystyle K_4\)%
\item{}\(K_5\).%
\item{}\(\displaystyle K_{5,7}\)%
\item{}\(\displaystyle K_{2,7}\)%
\item{}\(\displaystyle C_7\)%
\item{}\(\displaystyle P_7\)%
\end{enumerate}
\end{multicols}
%
\par\smallskip%
\noindent\textbf{\blocktitlefont Solution}.\quad{}%
\begin{enumerate}[label=(\alph*)]
\item{}\(K_4\) does not have an Euler path or circuit.%
\item{}\(K_5\) has an Euler circuit (so also an Euler path).%
\item{}\(K_{5,7}\) does not have an Euler path or circuit.%
\item{}\(K_{2,7}\) has an Euler path but not an Euler circuit.%
\item{}\(C_7\) has an Euler circuit (it is a circuit graph!)%
\item{}\(P_7\) has an Euler path but no Euler circuit.%
\end{enumerate}
%
\end{divisionsolution}%
\begin{divisionsolution}{4.5.3}{}{p:exercise:ZKH}%
Edward A. Mouse has just finished his brand new house. The floor plan is shown below:%
\begin{sidebyside}{1}{0.26}{0.26}{0}%
\begin{sbspanel}{0.48}%
\resizebox{\linewidth}{!}{%
\begin{tikzpicture}[scale=.8]
\draw[very thick] (-3,0) rectangle (3,3);
\draw[very thick] (-3,1.8) --(-2.7,1.8) (-2.3,1.8) -- (-1.5, 1.8) (-1.5, 1.6) -- (-1,1.6) (-.6, 1.6) -- (.3,1.6) (.7,1.6) -- (1, 1.6) (1, .8) -- (1.5, .8) (1.9,.8) -- (3,.8);
\draw[very thick] (-1.5,0) -- (-1.5, .8) (-1.5, 1.2) -- (-1.5,2.1) (-1.5,2.5) -- (-1.5,3);
\draw[very thick] (0,0) -- (0,.6) (0,1) -- (0,1.6);
\draw[very thick] (1,0) -- (1,.2) (1,.6) -- (1,1) (1,1.4) -- (1,2.1) (1,2.5) -- (1,3);
\end{tikzpicture}
}%
\end{sbspanel}%
\end{sidebyside}%
\par
%
\begin{enumerate}[label=(\alph*)]
\item{}Edward wants to give a tour of his new pad to a lady-mouse-friend. Is it possible for them to walk through every doorway exactly once? If so, in which rooms must they begin and end the tour? Explain. %
\item{}Is it possible to tour the house visiting each room exactly once (not necessarily using every doorway)? Explain. %
\item{}After a few mouse-years, Edward decides to remodel. He would like to add some new doors between the rooms he has. Of course, he cannot add any doors to the exterior of the house. Is it possible for each room to have an odd number of doors? Explain. %
\end{enumerate}
%
\end{divisionsolution}%
\begin{divisionsolution}{4.5.4}{}{p:exercise:FRQ}%
For which \(n\) does the graph \(K_n\) contain an Euler circuit? Explain.%
\par\smallskip%
\noindent\textbf{\blocktitlefont Solution}.\quad{}When \(n\) is odd, \(K_n\) contains an Euler circuit. This is because every vertex has degree \(n-1\), so an odd \(n\) results in all degrees being even.%
\end{divisionsolution}%
\begin{divisionsolution}{4.5.5}{}{p:exercise:lYZ}%
For which \(m\) and \(n\) does the graph \(K_{m,n}\) contain an Euler path? An Euler circuit? Explain.%
\par\smallskip%
\noindent\textbf{\blocktitlefont Solution}.\quad{}If both \(m\) and \(n\) are even, then \(K_{m,n}\) has an Euler circuit. When both are odd, there is no Euler path or circuit. If one is 2 and the other is odd, then there is an Euler path but not an Euler circuit.%
\end{divisionsolution}%
\begin{divisionsolution}{4.5.6}{}{p:exercise:Sgi}%
For which \(n\) does \(K_n\) contain a Hamilton path? A Hamilton cycle? Explain.%
\par\smallskip%
\noindent\textbf{\blocktitlefont Solution}.\quad{}All values of \(n\). In particular, \(K_n\) contains \(C_n\) as a subgroup, which is a cycle that includes every vertex.%
\end{divisionsolution}%
\begin{divisionsolution}{4.5.7}{}{p:exercise:ynr}%
For which \(m\) and \(n\) does the graph \(K_{m,n}\) contain a Hamilton path? A Hamilton cycle? Explain.%
\par\smallskip%
\noindent\textbf{\blocktitlefont Hint}.\quad{}This is harder than the previous three questions.  Think about which ``side'' of the graph the Hamilton path would need to be on every other step.%
\par\smallskip%
\noindent\textbf{\blocktitlefont Solution}.\quad{}As long as \(|m-n| \le 1\), the graph \(K_{m,n}\) will have a Hamilton path. To have a Hamilton cycle, we must have \(m=n\).%
\end{divisionsolution}%
\begin{divisionsolution}{4.5.8}{}{p:exercise:euA}%
A bridge builder has come to Königsberg and would like to add bridges so that it \emph{is} possible to travel over every bridge exactly once. How many bridges must be built?%
\par\smallskip%
\noindent\textbf{\blocktitlefont Solution}.\quad{}If we build one bridge, we can have an Euler path. Two bridges must be built for an Euler circuit.%
\begin{sidebyside}{1}{0.4}{0.4}{0}%
\begin{sbspanel}{0.2}%
\resizebox{\linewidth}{!}{%
       \begin{tikzpicture}[scale=1, yscale=.5]
\draw (-1,-2) \v to [out=120, in=240] (-1,0) \v to [out=120, in=240] (-1,2) \v to [out=300, in=60] (-1,0) to [out=300, in=60] (-1,-2);
 \draw (1,0) \v -- (-1,2) (-1,0) -- (1,0) -- (-1,-2);
 \draw[dashed] (-1,-2) -- (-1,0);
 \draw[dashed] (1,0) to [out=120, in=0] (-1,2);
 \end{tikzpicture}
}%
\end{sbspanel}%
\end{sidebyside}%
\end{divisionsolution}%
\begin{divisionsolution}{4.5.9}{}{p:exercise:KBJ}%
Below is a graph representing friendships between a group of students (each vertex is a student and each edge is a friendship). Is it possible for the students to sit around a round table in such a way that every student sits between two friends? What does this question have to do with paths?%
\begin{sidebyside}{1}{0.35}{0.35}{0}%
\begin{sbspanel}{0.3}%
\resizebox{\linewidth}{!}{%
          \begin{tikzpicture}
	\foreach \x in {1,...,9}{
	\coordinate (v\x) at (90-\x*360/9:1.5);
	\draw (v\x) \v;
	}
	\draw (v1) -- (v6) -- (v3) -- (v8) -- (v4) -- (v7) -- (v2) -- (v5) -- (v9) -- (v1);
	\draw (v1) -- (v3) -- (v5) (v4) -- (v5) (v4) -- (v7) -- (v6) -- (v9) (v3) -- (v7) (v9) -- (v3);
\end{tikzpicture}
}%
\end{sbspanel}%
\end{sidebyside}%
\par\smallskip%
\noindent\textbf{\blocktitlefont Hint}.\quad{}If you read off the names of the students in order, you would need to read each student's name exactly once, and the last name would need to be of a student who was friends with the first.  What sort of a cycle is this?%
\par\smallskip%
\noindent\textbf{\blocktitlefont Solution}.\quad{}We are looking for a Hamiltonian cycle, and this graph does have one:%
\begin{sidebyside}{1}{0.35}{0.35}{0}%
\begin{sbspanel}{0.3}%
\resizebox{\linewidth}{!}{%
        \begin{tikzpicture}
	\foreach \x in {1,...,9}{
	\coordinate (v\x) at (90-\x*360/9:1.5);
	}
	\draw[color=gray] (v1) -- (v3) -- (v5) (v4) -- (v5) (v4) -- (v7) -- (v6) -- (v9) (v3) -- (v7) (v9) -- (v3);
	\draw[line width=1.2pt, color=blue] (v1) -- (v6) -- (v3) -- (v8) (v8) -- (v4) (v4) -- (v7) -- (v2) -- (v5) -- (v9) -- (v1);
	\foreach \x in {1,...,9}{
	\draw (v\x) \v;
	}
\end{tikzpicture}
}%
\end{sbspanel}%
\end{sidebyside}%
\end{divisionsolution}%
\begin{divisionsolution}{4.5.10}{}{p:exercise:qIS}%
On the table rest 8 dominoes, as shown below. If you were to line them up in a single row, so that any two sides touching had matching numbers, what would the sum of the two end numbers be?%
\begin{sidebyside}{8}{0.0125}{0.0125}{0.025}%
\begin{sbspanel}{0.1}%
\resizebox{\linewidth}{!}{%
\begin{tikzpicture}
  \dominoborder
  \draw (0,0) \fourdots (0,1) \twodots;;
\end{tikzpicture}
}%
\end{sbspanel}%
\begin{sbspanel}{0.1}%
\resizebox{\linewidth}{!}{%
\begin{tikzpicture}
  \dominoborder
  \draw (0,0) \twodots (0,1) \sixdots;
\end{tikzpicture}
}%
\end{sbspanel}%
\begin{sbspanel}{0.1}%
\resizebox{\linewidth}{!}{%
\begin{tikzpicture}
 \dominoborder
  \draw (0,0) \threedots (0,1) \onedot;
\end{tikzpicture}
}%
\end{sbspanel}%
\begin{sbspanel}{0.1}%
\resizebox{\linewidth}{!}{%
\begin{tikzpicture}
 \dominoborder
  \draw (0,0) \sixdots (0,1) \fourdots;
\end{tikzpicture}
}%
\end{sbspanel}%
\begin{sbspanel}{0.1}%
\resizebox{\linewidth}{!}{%
\begin{tikzpicture}
 \dominoborder
  \draw (0,0) \threedots (0,1) \fivedots;
\end{tikzpicture}
}%
\end{sbspanel}%
\begin{sbspanel}{0.1}%
\resizebox{\linewidth}{!}{%
\begin{tikzpicture}
 \dominoborder
  \draw (0,0) \threedots (0,1) \fourdots;
\end{tikzpicture}
}%
\end{sbspanel}%
\begin{sbspanel}{0.1}%
\resizebox{\linewidth}{!}{%
\begin{tikzpicture}
 \dominoborder
  \draw (0,0) \fivedots (0,1) \sixdots;
\end{tikzpicture}
}%
\end{sbspanel}%
\begin{sbspanel}{0.1}%
\resizebox{\linewidth}{!}{%
\begin{tikzpicture}
 \dominoborder
  \draw (0,0) \sixdots (0,1) \threedots;
\end{tikzpicture}
}%
\end{sbspanel}%
\end{sidebyside}%
\par\smallskip%
\noindent\textbf{\blocktitlefont Hint}.\quad{}Draw a graph with 6 vertices and 8 edges.  What sort of path would be appropriate?%
\end{divisionsolution}%
\begin{divisionsolution}{4.5.11}{}{p:exercise:WQb}%
Is there anything we can say about whether a graph has a Hamilton path based on the degrees of its vertices?%
\begin{enumerate}[label=(\alph*)]
\item{}Suppose a graph has a Hamilton path. What is the maximum number of vertices of degree one the graph can have? Explain why your answer is correct. %
\item{}Find a graph which does not have a Hamilton path even though no vertex has degree one. Explain why your example works.%
\end{enumerate}
%
\end{divisionsolution}%
\begin{divisionsolution}{4.5.12}{}{p:exercise:CXk}%
Consider the following graph:%
\begin{sidebyside}{1}{0.36}{0.36}{0}%
\begin{sbspanel}{0.28}%
\resizebox{\linewidth}{!}{%
\begin{tikzpicture}[scale=.7]
\foreach \x in {0, 45, ..., 315}
  \draw  (\x:2) \v -- (\x+45:2);
\draw (0,0) \v -- (45:2) (0,0) -- (135:2) (0,0) -- (225:2) (0,0) -- (315:2);
\draw (-1,0) \v -- (90:2) (-1,0) -- (180:2) (-1,0) -- (270:2);
\draw (1,0) \v -- (90:2) (1,0) -- (0:2) (1,0) -- (270:2);
\end{tikzpicture}
}%
\end{sbspanel}%
\end{sidebyside}%
\par
%
\begin{enumerate}[label=(\alph*)]
\item{}Find a Hamilton path. Can your path be extended to a Hamilton cycle?%
\item{}Is the graph bipartite? If so, how many vertices are in each ``part''?%
\item{}Use your answer to part (b) to prove that the graph has no Hamilton cycle.%
\item{}Suppose you have a bipartite graph \(G\) in which one part has at least two more vertices than the other. Prove that \(G\) does not have a Hamilton path.%
\end{enumerate}
%
\end{divisionsolution}%
\section*{4.6 Matching in Bipartite Graphs}
\addcontentsline{toc}{section}{4.6 Matching in Bipartite Graphs}
\sectionmark{4.6 Matching in Bipartite Graphs}
\subsection*{Exercises}
\addcontentsline{toc}{subsection}{Exercises}
\begin{divisionsolution}{4.6.1}{}{p:exercise:hKF}%
Find a matching of the bipartite graphs below or explain why no matching exists.%
\begin{sidebyside}{3}{0.0166666666666667}{0.0166666666666667}{0.0333333333333333}%
\begin{sbspanel}{0.2}[bottom]%
\resizebox{\linewidth}{!}{%
\begin{tikzpicture}
\coordinate (a) at (0,0);
\coordinate (A) at (0,1);
\coordinate (b) at (1,0);
\coordinate (B) at (1,1);
\coordinate (c) at (2,0);
\coordinate (C) at (2,1);
\draw (a) \v -- (B) \v -- (c) \v -- (C) \v -- (a) \v -- (A)\v -- (b) \v;
\end{tikzpicture}
}%
\end{sbspanel}%
\begin{sbspanel}{0.3}[bottom]%
\resizebox{\linewidth}{!}{%
\begin{tikzpicture}
  \coordinate (a) at (0,0);
  \coordinate (A) at (0,1);
  \coordinate (b) at (1,0);
  \coordinate (B) at (1,1);
  \coordinate (c) at (2,0);
  \coordinate (C) at (2,1);
  \coordinate (d) at (3,0);
  \coordinate (D) at (3,1);
  \draw (a) \v -- (A) \v (b) \v -- (B) \v (c) \v -- (C) \v (d) \v (D)\v;
  \draw (a) -- (C) -- (b) -- (D) (A) -- (c) (A) -- (d) -- (C);
\end{tikzpicture}
}%
\end{sbspanel}%
\begin{sbspanel}{0.4}[bottom]%
\resizebox{\linewidth}{!}{%
\begin{tikzpicture}
  \coordinate (a) at (0,0);
  \coordinate (A) at (0,1);
  \coordinate (b) at (1,0);
  \coordinate (B) at (1,1);
  \coordinate (c) at (2,0);
  \coordinate (C) at (2,1);
  \coordinate (d) at (3,0);
  \coordinate (D) at (3,1);
  \coordinate (e) at (4,0);
  \coordinate (E) at (4,1);
  \draw (a) \v (A) \v (b) \v (B) \v (c) \v (C) \v (d) \v (D)\v (e)\v (E) \v;
  \draw (a) -- (A) (a) -- (B) (A) -- (b) (A) -- (c) (b) -- (C) (B) -- (c) -- (D) (c) -- (E) (C) -- (d) -- (E) (D) -- (e) -- (E);
\end{tikzpicture}
}%
\end{sbspanel}%
\end{sidebyside}%
\par\smallskip%
\noindent\textbf{\blocktitlefont Solution}.\quad{}The first and third graphs have a matching, shown in bold (there are other matchings as well). The middle graph does not have a matching. If you look at the three circled vertices, you see that they only have two neighbors, which violates the matching condition \(\card{N(S)} \ge \card{S}\) (the three circled vertices form the set \(S\)).%
\begin{sidebyside}{3}{0.0166666666666667}{0.0166666666666667}{0.0333333333333333}%
\begin{sbspanel}{0.2}[bottom]%
\resizebox{\linewidth}{!}{%
\begin{tikzpicture} \coordinate (a) at (0,0); \coordinate (A) at (0,1); \coordinate (b) at (1,0); \coordinate (B) at (1,1); \coordinate (c) at (2,0); \coordinate (C) at (2,1); \draw (a) \v -- (B) \v -- (c) \v -- (C) \v -- (a) \v -- (A)\v -- (b) \v; \draw[very thick] (a) -- (C) (A) -- (b) (c) -- (B); \end{tikzpicture}
}%
\end{sbspanel}%
\begin{sbspanel}{0.3}[bottom]%
\resizebox{\linewidth}{!}{%
\begin{tikzpicture} \coordinate (a) at (0,0); \coordinate (A) at (0,1); \coordinate (b) at (1,0); \coordinate (B) at (1,1); \coordinate (c) at (2,0); \coordinate (C) at (2,1); \coordinate (d) at (3,0); \coordinate (D) at (3,1); \draw (a) \v -- (A) \v (b) \v -- (B) \v (c) \v -- (C) \v (d) \v (D)\v; \draw (a) -- (C) -- (b) -- (D) (A) -- (c) (A) -- (d) -- (C); \draw[dashed] (a) circle (7pt) (c) circle (7pt) (d) circle (7pt); \end{tikzpicture}
}%
\end{sbspanel}%
\begin{sbspanel}{0.4}[bottom]%
\resizebox{\linewidth}{!}{%
\begin{tikzpicture} \coordinate (a) at (0,0); \coordinate (A) at (0,1); \coordinate (b) at (1,0); \coordinate (B) at (1,1); \coordinate (c) at (2,0); \coordinate (C) at (2,1); \coordinate (d) at (3,0); \coordinate (D) at (3,1); \coordinate (e) at (4,0); \coordinate (E) at (4,1); \draw (a) \v (A) \v (b) \v (B) \v (c) \v (C) \v (d) \v (D)\v (e)\v (E) \v; \draw (a) -- (A) (a) -- (B) (A) -- (b) (A) -- (c) (b) -- (C) (B) -- (c) -- (D) (c) -- (E) (C) -- (d) -- (E) (D) -- (e) -- (E); \draw[very thick] (a) -- (A) (b) -- (C) (c) -- (B) (d) -- (E) (e) -- (D); \end{tikzpicture}
}%
\end{sbspanel}%
\end{sidebyside}%
\end{divisionsolution}%
\begin{divisionsolution}{4.6.2}{}{p:exercise:NRO}%
A bipartite graph that doesn't have a matching might still have a \terminology{partial matching}. By this we mean a set of \emph{edges} for which no vertex belongs to more than one edge (but possibly belongs to none). Every bipartite graph (with at least one edge) has a partial matching, so we can look for the largest partial matching in a graph.%
\par
Your ``friend'' claims that she has found the largest partial matching for the graph below (her matching is in bold). She explains that no other edge can be added, because all the edges not used in her partial matching are connected to matched vertices. Is she correct?%
\begin{sidebyside}{1}{0.3}{0.3}{0}%
\begin{sbspanel}{0.4}%
\resizebox{\linewidth}{!}{%
\begin{tikzpicture}
\foreach \x in {0,...,4} {
 \coordinate (a\x) at (\x,0);
 \coordinate (b\x) at (\x,2);
 \draw (a\x) \v (b\x) \v;
 }
 \draw[line width=2pt] (a0) -- (b0) (a1) -- (b1) (a3) -- (b2) (a4) -- (b4);
 \draw[very thin] (a0) -- (b1) (a1) -- (b2) (a2)--(b0)  (a0)--(b2) (a3) -- (b4) (a4) -- (b3);
\end{tikzpicture}
}%
\end{sbspanel}%
\end{sidebyside}%
\end{divisionsolution}%
\begin{divisionsolution}{4.6.3}{}{p:exercise:tYX}%
One way you might check to see whether a partial matching is maximal is to construct an \terminology{alternating path}. This is a sequence of adjacent edges, which alternate between edges in the matching and edges not in the matching (no edge can be used more than once). If an alternating path starts and stops with an edge \emph{not} in the matching, then it is called an \terminology{augmenting path}.%
\begin{enumerate}[label=(\alph*)]
\item{}Find the largest possible alternating path for the partial matching of your friend's graph. Is it an augmenting path? How would this help you find a larger matching?%
\begin{sidebyside}{1}{0.3}{0.3}{0}%
\begin{sbspanel}{0.4}%
\resizebox{\linewidth}{!}{%
\begin{tikzpicture}
\foreach \x in {0,...,4} {
 \coordinate (a\x) at (\x,0);
 \coordinate (b\x) at (\x,2);
 \draw (a\x) \v (b\x) \v;
 }
 \draw[line width=2pt] (a0) -- (b0) (a1) -- (b1) (a3) -- (b2) (a4) -- (b4);
 \draw[very thin] (a0) -- (b1) (a1) -- (b2) (a2)--(b0) (a0)--(b2) (a3) -- (b4) (a4) -- (b3);
\end{tikzpicture}
}%
\end{sbspanel}%
\end{sidebyside}%
\item{}Find the largest possible alternating path for the partial matching below. Are there any augmenting paths? Is the partial matching the largest one that exists in the graph?%
\begin{sidebyside}{1}{0.25}{0.25}{0}%
\begin{sbspanel}{0.5}%
\resizebox{\linewidth}{!}{%
\begin{tikzpicture}
\foreach \x in {0,...,5} {
 \coordinate (a\x) at (\x,0);
 \coordinate (b\x) at (\x,2);
 \draw (a\x) \v (b\x) \v;
 }
 \draw[line width=2pt] (a0) -- (b1) (a1) -- (b2) (a2) -- (b0) (a3) -- (b4) (a4) -- (b5);
 \draw[very thin] (a0) -- (b2) (a1) -- (b0) (a2)--(b1) (a2) -- (b3) (a2) -- (b4) (a3) -- (b2) (a4) -- (b2) (a4)-- (b3) (a4) -- (b4) (a5) -- (b4) (a5)--(b5);
\end{tikzpicture}
}%
\end{sbspanel}%
\end{sidebyside}%
\end{enumerate}
%
\end{divisionsolution}%
\begin{divisionsolution}{4.6.4}{}{p:exercise:agg}%
The two richest families in Westeros have decided to enter into an alliance by marriage. The first family has 10 sons, the second has 10 girls. The ages of the kids in the two families match up. To avoid impropriety, the families insist that each child must marry someone either their own age, or someone one position younger or older. In fact, the graph representing agreeable marriages looks like this:%
\begin{sidebyside}{1}{0.05}{0.05}{0}%
\begin{sbspanel}{0.9}%
\resizebox{\linewidth}{!}{%
          \begin{tikzpicture}
\foreach \x in {0,...,9} {
 \coordinate (a\x) at (\x,0);
 \coordinate (b\x) at (\x,2);
 \draw (a\x) \v -- (b\x) \v;
 }
\draw (a0) -- (b1) -- (a2) -- (b3) -- (a4) -- (b5) -- (a6) -- (b7) -- (a8) -- (b9);
\draw (b0) -- (a1) -- (b2) -- (a3) -- (b4) -- (a5) -- (b6) -- (a7) -- (b8) -- (a9);
\end{tikzpicture}
}%
\end{sbspanel}%
\end{sidebyside}%
\par
The question: how many different acceptable marriage arrangements which marry off all 20 children are possible?%
\begin{enumerate}[label=(\alph*)]
\item{}How many marriage arrangements are possible if we insist that there are exactly 6 boys marry girls not their own age?%
\item{}Could you generalize the previous answer to arrive at the total number of marriage arrangements?%
\item{}How do you know you are correct? Try counting in a different way. Look at smaller family sizes and get a sequence.%
\item{}Can you give a recurrence relation that fits the problem?%
\end{enumerate}
%
\end{divisionsolution}%
\begin{divisionsolution}{4.6.5}{}{p:exercise:Gnp}%
We say that a set of vertices \(A \subseteq V\) is a \terminology{vertex cover} if every edge of the graph is incident to a vertex in the cover (so a vertex cover covers the \emph{edges}). Since \(V\) itself is a vertex cover, every graph has a vertex cover. The interesting question is about finding a \terminology{minimal} vertex cover, one that uses the fewest possible number of vertices.%
\begin{enumerate}[label=(\alph*)]
\item{}Suppose you had a matching of a graph. How can you use that to get a minimal vertex cover? Will your method always work?%
\item{}Suppose you had a minimal vertex cover for a graph. How can you use that to get a partial matching? Will your method always work?%
\item{}What is the relationship between the size of the minimal vertex cover and the size of the maximal partial matching in a graph?%
\end{enumerate}
%
\end{divisionsolution}%
\begin{divisionsolution}{4.6.6}{}{p:exercise:muy}%
For many applications of matchings, it makes sense to use bipartite graphs. You might wonder, however, whether there is a way to find matchings in graphs in general.%
\begin{enumerate}[label=(\alph*)]
\item{}For which \(n\) does the complete graph \(K_n\) have a matching?%
\item{}Prove that if a graph has a matching, then \(\card{V}\) is even.%
\item{}Is the converse true? That is, do all graphs with \(\card{V}\) even have a matching?%
\item{}What if we also require the matching condition? Prove or disprove: If a graph with an even number of vertices satisfies \(\card{N(S)} \ge \card{S}\) for all \(S \subseteq V\), then the graph has a matching.%
\end{enumerate}
%
\end{divisionsolution}%
\section*{4.7 Chapter Summary}
\addcontentsline{toc}{section}{4.7 Chapter Summary}
\sectionmark{4.7 Chapter Summary}
\subsection*{Chapter Review}
\addcontentsline{toc}{subsection}{Chapter Review}
\begin{divisionsolution}{4.7.1}{}{p:exercise:RhG}%
Which (if any) of the graphs below are the same? Which are different? Explain.%
\begin{sidebyside}{3}{0.04}{0.04}{0.08}%
\begin{sbspanel}{0.28}[bottom]%
\resizebox{\linewidth}{!}{%
            \begin{tikzpicture}[yscale=.7]
  \draw (-2,0) \v -- (0,0) \v -- (2,0) \v -- (-1,2) \v -- (1,2) \v -- (0,0) -- (-1,2) (1,2) -- (-2,0);
\end{tikzpicture}
}%
\end{sbspanel}%
\begin{sbspanel}{0.28}[bottom]%
\resizebox{\linewidth}{!}{%
            \begin{tikzpicture}[yscale=.7]
  \draw (-2,0) \v -- (0,0) \v -- (2,0) \v -- (0,1) \v -- (-2,0) -- (0,2) \v -- (2,0) (0,2) -- (0,1);
\end{tikzpicture}
}%
\end{sbspanel}%
\begin{sbspanel}{0.2}[bottom]%
\resizebox{\linewidth}{!}{%
            \begin{tikzpicture}[yscale=1]
  \draw (-1, 0) \v -- (-1,1) \v -- (0,1) \v -- (1,1) \v -- (1,0) \v -- (0,1) -- (-1,0);
  \draw (-1,1) to [out=60, in=120] (1,1);
\end{tikzpicture}
}%
\end{sbspanel}%
\end{sidebyside}%
\par\smallskip%
\noindent\textbf{\blocktitlefont Solution}.\quad{}The first and the third graphs are the same (try dragging vertices around to make the pictures match up), but the middle graph is different (which you can see, for example, by noting that the middle graph has only one vertex of degree 2, while the others have two such vertices).%
\end{divisionsolution}%
\begin{divisionsolution}{4.7.2}{}{p:exercise:xoP}%
Which of the graphs in the previous question contain Euler paths or circuits? Which of the graphs are planar?%
\par\smallskip%
\noindent\textbf{\blocktitlefont Solution}.\quad{}The first (and third) graphs contain an Euler path. All the graphs are planar.%
\end{divisionsolution}%
\begin{divisionsolution}{4.7.3}{}{p:exercise:dvY}%
Draw a graph which has an Euler circuit but is not planar.%
\par\smallskip%
\noindent\textbf{\blocktitlefont Solution}.\quad{}For example, \(K_5\).%
\end{divisionsolution}%
\begin{divisionsolution}{4.7.4}{}{p:exercise:JDh}%
Draw a graph which does not have an Euler path and is also not planar.%
\par\smallskip%
\noindent\textbf{\blocktitlefont Solution}.\quad{}For example, \(K_{3,3}\).%
\end{divisionsolution}%
\begin{divisionsolution}{4.7.5}{}{p:exercise:pKq}%
Consider the graph \(G = (V, E)\) with \(V = \{a,b,c,d,e,f,g\}\) and \(E = \{ab, ac, af, bg,
cd, ce\}\) (here we are using the shorthand for edges: \(ab\) really means \(\{a,b\}\), for example).%
\begin{enumerate}[label=(\alph*)]
\item{}Is the graph \(G\) isomorphic to \(G' = (V', E')\) with \(V' = \{t, u, v, w, x, y, z\}\) and \(E' = \{tz, uv,
uy, uz, vw,
vx\}\)? If so, give the isomoprhism. If not, explain how you know.%
\item{}Find a graph \(G''\) with 7 vertices and 6 edges which is NOT isomorphic to \(G\), or explain why this is not possible.%
\item{}Write down the \emph{degree sequence} for \(G\). That is, write down the degrees of all the vertices, in decreasing order.%
\item{}Find a connected graph \(G'''\) with the same degree sequence of \(G\) which is NOT isomorphic to \(G\), or explain why this is not possible.%
\item{}What kind of graph is \(G\)? Is \(G\) complete? Bipartite? A tree? A cycle? A path? A wheel?%
\item{}Is \(G\) planar?%
\item{}What is the chromatic number of \(G\)? Explain.%
\item{}Does \(G\) have an Euler path or circuit? Explain.%
\end{enumerate}
%
\par\smallskip%
\noindent\textbf{\blocktitlefont Solution}.\quad{}%
\begin{enumerate}[label=(\alph*)]
\item{}Yes, the graphs are isomorphic, which you can see by drawing them. One isomorphism is:%
\begin{equation*}
f = \begin{pmatrix}
a \amp b \amp c \amp d \amp e \amp f \amp g \\
u \amp z \amp v \amp x \amp w \amp y \amp t
\end{pmatrix}\text{.}
\end{equation*}
%
\item{}This is easy to do if you draw the picture. Here is such a graph:%
\begin{sidebyside}{1}{0.4}{0.4}{0}%
\begin{sbspanel}{0.2}%
\resizebox{\linewidth}{!}{%
\begin{tikzpicture}
\draw (0,0) \v -- (-1, 1) \v -- (-1,2) \v (0,0) -- (0,1) \v -- (0,2) \v (0,0) -- (1,1) \v -- (1,2) \v;
\end{tikzpicture}
}%
\end{sbspanel}%
\end{sidebyside}%
\par
Any labeling of this graph will be not isomorphic to \(G\). For example, we could take \(V'' = \{a,b,c,d,e,f,g\}\) and \(E'' = \{ab,
ac, ad, be, cf, dg\} \).%
\item{}The degree sequence for \(G\) is \((3, 3, 2, 1, 1, 1, 1)\).%
\item{}In general this should be possible: the degree sequence does not determine the graph's isomorphism class. However, in this case, I was almost certain this was not possible. That is, until I stumbled up this:%
\begin{sidebyside}{1}{0.35}{0.35}{0}%
\begin{sbspanel}{0.3}%
\resizebox{\linewidth}{!}{%
\begin{tikzpicture}
\draw (0,0) \v -- (-1, 1) \v -- (-1.5,2) \v  (-1,1) -- (-.5,2) \v (0,0) -- (1,1) \v -- (1.5,2) \v (1,1) -- (0.5, 2) \v;
\end{tikzpicture}
}%
\end{sbspanel}%
\end{sidebyside}%
\item{}\(G\) is a tree (there are no cycles) and as such also bipartite.%
\item{}Yes, all trees are planar. You can draw them in the plane without edges crossing.%
\item{}The chromatic number of \(G\) is 2. It shouldn't be hard to give a 2-coloring (for example, color \(a, d, e, g\) red and \(b, c, f\) blue), but we know that all bipartite graphs have chromatic number 2.%
\item{}It is clear from the drawing that there is no Euler path, let alone an Euler circuit. Also, since there are more than 2 vertices of odd degree, we know for sure there is no Euler path.%
\end{enumerate}
%
\end{divisionsolution}%
\begin{divisionsolution}{4.7.6}{}{p:exercise:VRz}%
If a graph has 10 vertices and 10 edges and contains an Euler circuit, must it be planar? How many faces would it have?%
\par\smallskip%
\noindent\textbf{\blocktitlefont Solution}.\quad{}Yes. According to Euler's formula it would have 2 faces. It does. The only such graph is \(C_{10}\).%
\end{divisionsolution}%
\begin{divisionsolution}{4.7.7}{}{p:exercise:BYI}%
Suppose \(G\) is a graph with \(n\) vertices, each having degree 5.%
\begin{enumerate}[label=(\alph*)]
\item{}For which values of \(n\) does this make sense?%
\item{}For which values of \(n\) does the graph have an Euler path?%
\item{}What is the smallest value of \(n\) for which the graph might be planar? (tricky)%
\end{enumerate}
%
\par\smallskip%
\noindent\textbf{\blocktitlefont Solution}.\quad{}%
\begin{enumerate}[label=(\alph*)]
\item{}Only if \(n \ge 6\) and is even.%
\item{}None.%
\item{}12. Such a graph would have \(\frac{5n}{2}\) edges. If the graph is planar, then \(n - \frac{5n}{2} + f = 2\) so there would be \(\frac{4+3n}{2}\) faces. Also, we must have \(3f \le 2e\), since the graph is simple. So we must have \(3\left(\frac{4 + 3n}{2}\right) \le 5n\). Solving for \(n\) gives \(n \ge 12\).%
\end{enumerate}
%
\end{divisionsolution}%
\begin{divisionsolution}{4.7.8}{}{p:exercise:ifR}%
At a school dance, 6 girls and 4 boys take turns dancing (as couples) with each other.%
\begin{enumerate}[label=(\alph*)]
\item{}How many couples danced if every girl dances with every boy?%
\item{}How many couples danced if everyone danced with everyone else (regardless of gender)?%
\item{}Explain what graphs can be used to represent these situations.%
\end{enumerate}
%
\par\smallskip%
\noindent\textbf{\blocktitlefont Solution}.\quad{}%
\begin{enumerate}[label=(\alph*)]
\item{}There were 24 couples: 6 choices for the girl and 4 choices for the boy.%
\item{}There were 45 couples: \({10 \choose 2}\) since we must choose two of the 10 people to dance together.%
\item{}For part (a), we are counting the number of edges in \(K_{4,6}\). In part (b) we count the edges of \(K_{10}\).%
\end{enumerate}
%
\end{divisionsolution}%
\begin{divisionsolution}{4.7.9}{}{p:exercise:Ona}%
Among a group of \(n\) people, is it possible for everyone to be friends with an odd number of people in the group? If so, what can you say about \(n\)?%
\par\smallskip%
\noindent\textbf{\blocktitlefont Solution}.\quad{}Yes, as long as \(n\) is even. If \(n\) were odd, then corresponding graph would have an odd number of odd degree vertices, which is impossible.%
\end{divisionsolution}%
\begin{divisionsolution}{4.7.10}{}{p:exercise:uuj}%
Your friend has challenged you to create a convex polyhedron containing 9 triangles and 6 pentagons.%
\begin{enumerate}[label=(\alph*)]
\item{}Is it possible to build such a polyhedron using \emph{only} these shapes? Explain.%
\item{}You decide to also include one heptagon (seven-sided polygon). How many vertices does your new convex polyhedron contain?%
\item{}Assuming you are successful in building your new 16-faced polyhedron, could every vertex be the joining of the same number of faces? Could each vertex join either 3 or 4 faces? If so, how many of each type of vertex would there be?%
\end{enumerate}
%
\par\smallskip%
\noindent\textbf{\blocktitlefont Solution}.\quad{}%
\begin{enumerate}[label=(\alph*)]
\item{}No. The 9 triangles each contribute 3 edges, and the 6 pentagons contribute 5 edges. This gives a total of 57, which is exactly twice the number of edges, since each edge borders exactly 2 faces. But 57 is odd, so this is impossible.%
\item{}Now adding up all the edges of all the 16 polygons gives a total of 64, meaning there would be 32 edges in the polyhedron. We can then use Euler's formula \(v - e + f = 2\) to deduce that there must be 18 vertices.%
\item{}If you add up all the vertices from each polygon separately, we get a total of 64. This is not divisible by 3, so it cannot be that each vertex belongs to exactly 3 faces. Could they all belong to 4 faces? That would mean there were \(64/4 = 16\) vertices, but we know from Euler's formula that there must be 18 vertices. We can write \(64 = 3x + 4y\) and solve for \(x\) and \(y\) (as integers). We get that there must be 10 vertices with degree 4 and 8 with degree 3. (Note the number of faces joined at a vertex is equal to its degree in graph theoretic terms.)%
\end{enumerate}
%
\end{divisionsolution}%
\begin{divisionsolution}{4.7.11}{}{p:exercise:aBs}%
Is there a convex polyhedron which requires 5 colors to properly color the vertices of the polyhedron? Explain.%
\par\smallskip%
\noindent\textbf{\blocktitlefont Solution}.\quad{}No. Every polyhedron can be represented as a planar graph, and the Four Color Theorem says that every planar graph has chromatic number at most 4.%
\end{divisionsolution}%
\begin{divisionsolution}{4.7.12}{}{p:exercise:GIB}%
How many edges does the graph \(K_{n,n}\) have? For which values of \(n\) does the graph contain an Euler circuit? For which values of \(n\) is the graph planar?%
\par\smallskip%
\noindent\textbf{\blocktitlefont Solution}.\quad{}\(K_{n,n}\) has \(n^2\) edges. The graph will have an Euler circuit when \(n\) is even. The graph will be planar only when \(n \lt 3\).%
\end{divisionsolution}%
\begin{divisionsolution}{4.7.13}{}{p:exercise:mPK}%
The graph \(G\) has 6 vertices with degrees \(1, 2, 2, 3, 3, 5\). How many edges does \(G\) have? If \(G\) was planar how many faces would it have? Does \(G\) have an Euler path?%
\par\smallskip%
\noindent\textbf{\blocktitlefont Solution}.\quad{}\(G\) has 8 edges (since the sum of the degrees is 16). If \(G\) is planar, then it will have 4 faces (since \(6 - 8 + 4 = 2\)). \(G\) does not have an Euler path since there are more than 2 vertices of odd degree.%
\end{divisionsolution}%
\begin{divisionsolution}{4.7.14}{}{p:exercise:SWT}%
What is the smallest number of colors you need to properly color the vertices of \(K_{7}\). Can you say whether \(K_7\) is planar based on your answer?%
\par\smallskip%
\noindent\textbf{\blocktitlefont Solution}.\quad{}\(7\) colors. Thus \(K_7\) is not planar (by the contrapositive of the Four Color Theorem).%
\end{divisionsolution}%
\begin{divisionsolution}{4.7.15}{}{p:exercise:zec}%
What is the smallest number of colors you need to properly color the vertices of \(K_{3,4}\)? Can you say whether \(K_{3,4}\) is planar based on your answer?%
\par\smallskip%
\noindent\textbf{\blocktitlefont Solution}.\quad{}The chromatic number of \(K_{3,4}\) is 2, since the graph is bipartite. You cannot say whether the graph is planar based on this coloring (the converse of the Four Color Theorem is not true). In fact, the graph is \emph{not} planar, since it contains \(K_{3,3}\) as a subgraph.%
\end{divisionsolution}%
\begin{divisionsolution}{4.7.16}{}{p:exercise:fll}%
Prove that \(K_{3,4}\) is not planar. Do this using Euler's formula, not just by appealing to the fact that \(K_{3,3}\) is not planar.%
\par\smallskip%
\noindent\textbf{\blocktitlefont Solution}.\quad{}We have that \(K_{3,4}\) has 7 vertices and 12 edges (each vertex in the group of 3 has degree 4). Then by Euler's formula we have that \(7 - 12 + f = 2\) so if the graph were planar, it would have \(f = 7\) faces. However, since the girth of the graph is 4 (there are no cycles of length 3) we get that \(4f \le 2e\). But this would mean that \(28 \le 24\), a contradiction.%
\end{divisionsolution}%
\begin{divisionsolution}{4.7.17}{}{p:exercise:Lsu}%
A dodecahedron is a regular convex polyhedron made up of 12 regular pentagons.%
\begin{enumerate}[label=(\alph*)]
\item{}Suppose you color each pentagon with one of three colors. Prove that there must be two adjacent pentagons colored identically.%
\item{}What if you use four colors?%
\item{}What if instead of a dodecahedron you colored the faces of a cube?%
\end{enumerate}
%
\par\smallskip%
\noindent\textbf{\blocktitlefont Solution}.\quad{}For all these questions, we are really coloring the vertices of a graph. You get the graph by first drawing a planar representation of the polyhedron and then taking its planar dual: put a vertex in the center of each face (including the outside) and connect two vertices if their faces share an edge.%
\begin{enumerate}[label=(\alph*)]
\item{}Since the planar dual of a dodecahedron contains a 5-wheel, it's chromatic number is at least 4. Alternatively, suppose you could color the faces using 3 colors without any two adjacent faces colored the same. Take any face and color it blue. The 5 pentagons bordering this blue pentagon cannot be colored blue. Color the first one red. Its two neighbors (adjacent to the blue pentagon) get colored green. The remaining 2 cannot be blue or green, but also cannot both be red since they are adjacent to each other. Thus a 4th color is needed.%
\item{}The planar dual of the dodecahedron is itself a planar graph. Thus by the 4-color theorem, it can be colored using only 4 colors without two adjacent vertices (corresponding to the faces of the polyhedron) being colored identically.%
\item{}The cube can be properly 3-colored. Color the ``top'' and ``bottom'' red, the ``front'' and ``back'' blue, and the ``left'' and ``right'' green.%
\end{enumerate}
%
\end{divisionsolution}%
\begin{divisionsolution}{4.7.18}{}{p:exercise:rzD}%
Decide whether the following statements are true or false. Prove your answers.%
\begin{enumerate}[label=(\alph*)]
\item{}If two graph \(G_1\) and \(G_2\) have the same chromatic number, then they are isomorphic.%
\item{}If two graphs \(G_1\) and \(G_2\) have the same number of vertices and edges and have the same chromatic number, then they are isomorphic.%
\item{}If two graphs are isomorphic, then they have the same chromatic number.%
\end{enumerate}
%
\par\smallskip%
\noindent\textbf{\blocktitlefont Solution}.\quad{}%
\begin{enumerate}[label=(\alph*)]
\item{}False. To prove this, we can give an example of a pair of graphs with the same chromatic number that are not isomorphic. For example, \(K_{3,3}\) and \(K_{3,4}\) both have chromatic number 2, but are not isomorphic.%
\item{}False. The previous example does not work, but you can easily draw two trees that have the same number of vertices and edges but are not isomorphic. Since all trees have chromatic number 2, this is a counterexample.%
\item{}True. If there is an isomorphism from \(G_1\) to \(G_2\), then we have a bijection that tells us how to match up vertices between the graph. Any proper vertex coloring of \(G_1\) will tell us how to properly color \(G_2\), simply by coloring \(f(v_i)\) the same color as \(v_i\), for each vertex \(v_i \in V\). That is, color the vertices in \(G_2\) exactly how you color the corresponding vertices in \(G_1\). Similarly, any proper vertex coloring of \(G_2\) corresponds to a proper vertex coloring of \(G_1\). Thus the smallest number of colors needed to properly color \(G_1\) cannot be smaller than the smallest number of colors needed to properly color \(G_2\), and vice-versa, so the chromatic numbers must be equal.%
\end{enumerate}
%
\end{divisionsolution}%
\begin{divisionsolution}{4.7.19}{}{p:exercise:XGM}%
If a planar graph \(G\) with \(7\) vertices divides the plane into 8 regions, how many edges must \(G\) have?%
\par\smallskip%
\noindent\textbf{\blocktitlefont Solution}.\quad{}\(G\) has \(13\) edges, since we need \(7 - e + 8 = 2\).%
\end{divisionsolution}%
\begin{divisionsolution}{4.7.20}{}{p:exercise:DNV}%
Consider the graph below:%
\begin{sidebyside}{1}{0.375}{0.375}{0}%
\begin{sbspanel}{0.25}%
\resizebox{\linewidth}{!}{%
\begin{tikzpicture}[scale=.8]
  \draw (0,0) \v -- (-1.5, .5) \v -- (0,1.5) \v -- (1.5,.5) \v -- (0,0) -- (-1,2) \v -- (0,1.5) -- (1,2) \v -- (0,0) -- (0, 1.5);
\end{tikzpicture}
}%
\end{sbspanel}%
\end{sidebyside}%
\par
%
\begin{enumerate}[label=(\alph*)]
\item{}Does the graph have an Euler path or circuit? Explain.%
\item{}Is the graph planar? Explain.%
\item{}Is the graph bipartite? Complete? Complete bipartite?%
\item{}What is the chromatic number of the graph.%
\end{enumerate}
%
\par\smallskip%
\noindent\textbf{\blocktitlefont Solution}.\quad{}%
\begin{enumerate}[label=(\alph*)]
\item{}The graph does have an Euler path, but not an Euler circuit. There are exactly two vertices with odd degree. The path starts at one and ends at the other.%
\item{}The graph is planar. Even though as it is drawn edges cross, it is easy to redraw it without edges crossing.%
\item{}The graph is not bipartite (there is an odd cycle), nor complete.%
\item{}The chromatic number of the graph is 3.%
\end{enumerate}
%
\end{divisionsolution}%
\begin{divisionsolution}{4.7.21}{}{p:exercise:jVe}%
For each part below, say whether the statement is true or false. Explain why the true statements are true, and give counterexamples for the false statements.%
\begin{enumerate}[label=(\alph*)]
\item{}Every bipartite graph is planar.%
\item{}Every bipartite graph has chromatic number 2.%
\item{}Every bipartite graph has an Euler path.%
\item{}Every vertex of a bipartite graph has even degree.%
\item{}A graph is bipartite if and only if the sum of the degrees of all the vertices is even.%
\end{enumerate}
%
\par\smallskip%
\noindent\textbf{\blocktitlefont Solution}.\quad{}%
\begin{enumerate}[label=(\alph*)]
\item{}False. For example, \(K_{3,3}\) is not planar.%
\item{}True. The graph is bipartite so it is possible to divide the vertices into two groups with no edges between vertices in the same group. Thus we can color all the vertices of one group red and the other group blue.%
\item{}False. \(K_{3,3}\) has 6 vertices with degree 3, so contains no Euler path.%
\item{}False. \(K_{3,3}\) again.%
\item{}False. The sum of the degrees of all vertices is even for \emph{all} graphs so this property does not imply that the graph is bipartite.%
\end{enumerate}
%
\end{divisionsolution}%
\begin{divisionsolution}{4.7.22}{}{p:exercise:Qcn}%
Consider the statement ``If a graph is planar, then it has an Euler path.''%
\begin{enumerate}[label=(\alph*)]
\item{}Write the converse of the statement.%
\item{}Write the contrapositive of the statement.%
\item{}Write the negation of the statement.%
\item{}Is it possible for the contrapositive to be false? If it was, what would that tell you?%
\item{}Is the original statement true or false? Prove your answer.%
\item{}Is the converse of the statement true or false? Prove your answer.%
\end{enumerate}
%
\par\smallskip%
\noindent\textbf{\blocktitlefont Solution}.\quad{}%
\begin{enumerate}[label=(\alph*)]
\item{}If a graph has an Euler path, then it is planar.%
\item{}If a graph does not have an Euler path, then it is not planar.%
\item{}There is a graph which is planar and does not have an Euler path.%
\item{}Yes. In fact, in this case it is because the original statement is false.%
\item{}False. \(K_4\) is planar but does not have an Euler path.%
\item{}False. \(K_5\) has an Euler path but is not planar.%
\end{enumerate}
%
\end{divisionsolution}%
\begin{divisionsolution}{4.7.23}{}{p:exercise:wjw}%
Let \(G\) be a connected graph with \(v\) vertices and \(e\) edges.  Use mathematical induction to prove that if \(G\) contains exactly one cycle (among other edges and vertices), then \(v = e\).%
\par
Note: this is asking you to prove a special case of Euler's formula for planar graphs, so do not use that formula in your proof.%
\par\smallskip%
\noindent\textbf{\blocktitlefont Hint}.\quad{}You might want to give the proof in two parts.  First prove by induction that the cycle \(C_n\) has \(v=e\).  Then consider what happens if the graph is more than just the cycle.%
\end{divisionsolution}%
\chapter*{5 Additional Topics}
\addcontentsline{toc}{chapter}{5 Additional Topics}
\chaptermark{5 Additional Topics}
\section*{5.1 Generating Functions}
\addcontentsline{toc}{section}{5.1 Generating Functions}
\sectionmark{5.1 Generating Functions}
\subsection*{Exercises}
\addcontentsline{toc}{subsection}{Exercises}
\begin{divisionsolution}{5.1.1}{}{p:exercise:Gtg}%
Find the generating function for each of the following sequences by relating them back to a sequence with known generating function.%
\begin{enumerate}[label=(\alph*)]
\item{}\(4,4,4,4,4,\ldots\).%
\item{}\(2, 4, 6, 8, 10, \ldots\).%
\item{}\(0,0,0,2,4,6,8,10,\ldots\).%
\item{}\(1, 5, 25, 125, \ldots\).%
\item{}\(1, -3, 9, -27, 81, \ldots\).%
\item{}\(1, 0, 5, 0, 25, 0, 125, 0, \ldots\).%
\item{}\(0, 1, 0, 0, 2, 0, 0, 3, 0, 0, 4, 0, 0, 5, \ldots\).%
\end{enumerate}
%
\par\smallskip%
\noindent\textbf{\blocktitlefont Solution}.\quad{}%
\begin{multicols}{3}
\begin{enumerate}[label=(\alph*)]
\item{}\(\dfrac{4}{1-x}\).%
\item{}\(\dfrac{2}{(1-x)^2}\).%
\item{}\(\dfrac{2x^3}{(1-x)^2}\).%
\item{}\(\dfrac{1}{1-5x}\).%
\item{}\(\dfrac{1}{1+3x}\).%
\item{}\(\dfrac{1}{1-5x^2}\).%
\item{}\(\dfrac{x}{(1-x^3)^2}\).%
\end{enumerate}
\end{multicols}
%
\end{divisionsolution}%
\begin{divisionsolution}{5.1.2}{}{p:exercise:mAp}%
Find the sequence generated by the following generating functions:%
\begin{enumerate}[label=(\alph*)]
\item{}\(\dfrac{4x}{1-x}\).%
\item{}\(\dfrac{1}{1-4x}\).%
\item{}\(\dfrac{x}{1+x}\).%
\item{}\(\dfrac{3x}{(1+x)^2}\).%
\item{}\(\dfrac{1+x+x^2}{(1-x)^2}\) (Hint: multiplication).%
\end{enumerate}
%
\par\smallskip%
\noindent\textbf{\blocktitlefont Solution}.\quad{}%
\begin{enumerate}[label=(\alph*)]
\item{}\(0, 4, 4, 4, 4, 4, \ldots\).%
\item{}\(1, 4, 16, 64, 256, \ldots\).%
\item{}\(0, 1, -1, 1, -1, 1, -1, \ldots\).%
\item{}\(0, 3, -6, 9, -12, 15, -18, \ldots\).%
\item{}\(1, 3, 6, 9, 12, 15, \ldots\).%
\end{enumerate}
%
\end{divisionsolution}%
\begin{divisionsolution}{5.1.3}{}{p:exercise:SHy}%
Show how you can get the generating function for the triangular numbers in three different ways:%
\begin{enumerate}[label=(\alph*)]
\item{}Take two derivatives of the generating function for \(1,1,1,1,1, \ldots\)%
\item{}Use differencing.%
\item{}Multiply two known generating functions.%
\end{enumerate}
%
\par\smallskip%
\noindent\textbf{\blocktitlefont Solution}.\quad{}%
\begin{enumerate}[label=(\alph*)]
\item{}The second derivative of \(\dfrac{1}{1-x}\) is \(\dfrac{2}{(1-x)^3}\) which expands to \(2 + 6x + 12x^2 + 20x^3 + 30x^4 + \cdots\). Dividing by 2 gives the generating function for the triangular numbers.%
\item{}Compute \(A - xA\) and you get \(1 + 2x + 3x^2 + 4x^3 + \cdots\) which can be written as \(\dfrac{1}{(1-x)^2}\). Solving for \(A\) gives the correct generating function.%
\item{}The triangular numbers are the sum of the first \(n\) numbers \(1,2,3,4, \ldots\). To get the sequence of partial sums, we multiply by \(\frac{1}{1-x}\). So this gives the correct generating function again.%
\end{enumerate}
%
\end{divisionsolution}%
\begin{divisionsolution}{5.1.4}{}{p:exercise:yOH}%
Use differencing to find the generating function for \(4, 5, 7, 10, 14, 19, 25, \ldots\).%
\par\smallskip%
\noindent\textbf{\blocktitlefont Solution}.\quad{}Call the generating function \(A\). Compute \(A - xA = 4 + x + 2x^2 + 3x^3 + 4x^4 + \cdots\). Thus \(A - xA = 4 + \dfrac{x}{(1-x)^2}\). Solving for \(A\) gives \(\d\frac{4}{1-x} + \frac{x}{(1-x)^3}\).%
\end{divisionsolution}%
\begin{divisionsolution}{5.1.5}{}{p:exercise:eVQ}%
Find a generating function for the sequence with recurrence relation \(a_n = 3a_{n-1} - a_{n-2}\) with initial terms \(a_0 = 1\) and \(a_1 = 5\).%
\par\smallskip%
\noindent\textbf{\blocktitlefont Solution}.\quad{}\(\dfrac{1+2x}{1-3x + x^2}\).%
\end{divisionsolution}%
\begin{divisionsolution}{5.1.6}{}{p:exercise:LcZ}%
Use the recurrence relation for the Fibonacci numbers to find the generating function for the Fibonacci sequence.%
\par\smallskip%
\noindent\textbf{\blocktitlefont Solution}.\quad{}Compute \(A - xA - x^2A\) and the solve for \(A\). The generating function will be \(\dfrac{x}{1-x-x^2}\).%
\end{divisionsolution}%
\begin{divisionsolution}{5.1.7}{}{p:exercise:rki}%
Use multiplication to find the generating function for the sequence of partial sums of Fibonacci numbers, \(S_0, S_1, S_2, \ldots\) where \(S_0 = F_0\), \(S_1 = F_0 + F_1\), \(S_2 = F_0 + F_1 + F_2\), \(S_3 = F_0 + F_1 + F_2 + F_3\) and so on.%
\par\smallskip%
\noindent\textbf{\blocktitlefont Solution}.\quad{}\(\dfrac{x}{(1-x)(1-x-x^2)}\).%
\end{divisionsolution}%
\begin{divisionsolution}{5.1.8}{}{p:exercise:Xrr}%
Find the generating function for the sequence with closed formula \(a_n = 2(5^n) + 7(-3)^n\).%
\par\smallskip%
\noindent\textbf{\blocktitlefont Solution}.\quad{}\(\dfrac{2}{1-5x} + \dfrac{7}{1+3x}\).%
\end{divisionsolution}%
\begin{divisionsolution}{5.1.9}{}{p:exercise:DyA}%
Find a closed formula for the \(n\)th term of the sequence with generating function \(\dfrac{3x}{1-4x} + \dfrac{1}{1-x}\).%
\par\smallskip%
\noindent\textbf{\blocktitlefont Solution}.\quad{}\(a_n = 3\cdot 4^{n-1} + 1\).%
\end{divisionsolution}%
\begin{divisionsolution}{5.1.10}{}{p:exercise:jFJ}%
Find \(a_7\) for the sequence with generating function \(\dfrac{2}{(1-x)^2}\cdot\dfrac{x}{1-x-x^2}\).%
\par\smallskip%
\noindent\textbf{\blocktitlefont Hint}.\quad{}You should ``multiply'' the two sequences.%
\par\smallskip%
\noindent\textbf{\blocktitlefont Solution}.\quad{}We will have \(a_7 = 158\).  The sequence for \(\frac{2}{(1-x)^2}\) is \(2, 4, 6, 8, \ldots\).  The sequence for \(frac{x}{1-x-x^2}\) is \(0, 1, 1, 2, 3, 5, 8,\ldots\).  To get the 7th term of the product, we compute \(2\cdot 13 + 4 \cdot 8 + 6 \cdot 5 + \cdots + 14 \cdot 0\).%
\end{divisionsolution}%
\begin{divisionsolution}{5.1.11}{}{p:exercise:vUb}%
Explain how we know that \(\dfrac{1}{(1-x)^2}\) is the generating function for \(1, 2, 3, 4, \ldots\).%
\par\smallskip%
\noindent\textbf{\blocktitlefont Solution}.\quad{}Starting with \(\frac{1}{1-x} = 1 + x + x^2 + x^3 +\cdots\), we can take derivatives of both sides, given \(\frac{1}{(1-x)^2} = 1 + 2x + 3x^2 + \cdots\). By the definition of generating functions, this says that \(\frac{1}{(1-x)^2}\) generates the \emph{sequence} 1, 2, 3\textellipsis{}. You can also find this using differencing or by multiplying.%
\end{divisionsolution}%
\begin{divisionsolution}{5.1.12}{}{p:exercise:cbk}%
Starting with the generating function for \(1,2,3,4, \ldots\), find a generating function for each of the following sequences.%
\begin{enumerate}[label=(\alph*)]
\item{}\(1, 0, 2, 0, 3, 0, 4,\ldots\).%
\item{}\(1, -2, 3, -4, 5, -6, \ldots\).%
\item{}\(0, 3, 6, 9, 12, 15, 18, \ldots\).%
\item{}\(0, 3, 9, 18, 30, 45, 63,\ldots\). (Hint: relate this sequence to the previous one.)%
\end{enumerate}
%
\par\smallskip%
\noindent\textbf{\blocktitlefont Solution}.\quad{}%
\begin{enumerate}[label=(\alph*)]
\item{}\(\frac{1}{(1-x^2)^2}\).%
\item{}\(\frac{1}{(1+x)^2}\).%
\item{}\(\frac{3x}{(1-x)^2}\).%
\item{}\(\frac{3x}{(1-x)^3}\). (partial sums).%
\end{enumerate}
%
\end{divisionsolution}%
\begin{divisionsolution}{5.1.13}{}{p:exercise:Iit}%
You may assume that \(1, 1, 2, 3, 5, 8,\ldots\) has generating function \(\dfrac{1}{1-x-x^2}\) (because it does). Use this fact to find the sequence generated by each of the following generating functions.%
\begin{enumerate}[label=(\alph*)]
\item{}\(\frac{x^2}{1-x-x^2}\).%
\item{}\(\frac{1}{1-x^2-x^4}\).%
\item{}\(\frac{1}{1-3x-9x^2}\).%
\item{}\(\frac{1}{(1-x-x^2)(1-x)}\).%
\end{enumerate}
%
\par\smallskip%
\noindent\textbf{\blocktitlefont Solution}.\quad{}%
\begin{enumerate}[label=(\alph*)]
\item{}\(0,0,1,1,2,3,5,8, \ldots\).%
\item{}\(1, 0, 1, 0, 2, 0, 3, 0, 5, 0, 8, 0, \ldots\).%
\item{}\(1, 3, 18, 81, 405, \ldots\).%
\item{}\(1, 2, 4, 7, 12, 20, \ldots\).%
\end{enumerate}
%
\end{divisionsolution}%
\begin{divisionsolution}{5.1.14}{}{p:exercise:opC}%
Find the generating function for the sequence \(1, -2, 4, -8, 16, \ldots\).%
\par\smallskip%
\noindent\textbf{\blocktitlefont Solution}.\quad{}\(\frac{1}{1+2x}\).%
\end{divisionsolution}%
\begin{divisionsolution}{5.1.15}{}{p:exercise:UwL}%
Find the generating function for the sequence \(1, 1, 1, 2, 3, 4, 5, 6, \ldots\).%
\par\smallskip%
\noindent\textbf{\blocktitlefont Solution}.\quad{}\(\frac{x^3}{(1-x)^2} + \frac{1}{1-x}\).%
\end{divisionsolution}%
\begin{divisionsolution}{5.1.16}{}{p:exercise:ADU}%
Suppose \(A\) is the generating function for the sequence \(3, 5, 9, 15, 23, 33, \ldots\).%
\begin{enumerate}[label=(\alph*)]
\item{}Find a generating function (in terms of \(A\)) for the sequence of differences between terms.%
\item{}Write the sequence of differences between terms and find a generating function for it (without referencing \(A\)).%
\item{}Use your answers to parts (a) and (b) to find the generating function for the original sequence.%
\end{enumerate}
%
\par\smallskip%
\noindent\textbf{\blocktitlefont Solution}.\quad{}%
\begin{enumerate}[label=(\alph*)]
\item{}\((1-x)A = 3 + 2x + 4x^2 + 6x^3 + \cdots\) which is almost right. We can fix it like this: \(2 + 4x + 6x^2 + \cdots = \frac{(1-x)A - 3}{x}\).%
\item{}We know \(2 + 4x + 6x^3 + \cdots = \frac{2}{(1-x)^2}\).%
\item{}\(A = \frac{2x}{(1-x)^3} + \frac{3}{1-x} = \frac{3 -4x + 3x^2}{(1-x)^3}\).%
\end{enumerate}
%
\end{divisionsolution}%
\section*{5.2 Introduction to Number Theory}
\addcontentsline{toc}{section}{5.2 Introduction to Number Theory}
\sectionmark{5.2 Introduction to Number Theory}
\subsection*{Exercises}
\addcontentsline{toc}{subsection}{Exercises}
\begin{divisionsolution}{5.2.1}{}{p:exercise:OVU}%
Suppose \(a\), \(b\), and \(c\) are integers. Prove that if \(a \mid b\), then \(a \mid bc\).%
\par\smallskip%
\noindent\textbf{\blocktitlefont Solution}.\quad{}\begin{solutionproof}
Suppose \(a \mid b\). Then \(b\) is a multiple of \(a\), or in other words, \(b = ak\) for some \(k\). But then \(bc = akc\), and since \(kc\) is an integer, this says \(bc\) is a multiple of \(a\). In other words, \(a \mid bc\).%
\end{solutionproof}
\end{divisionsolution}%
\begin{divisionsolution}{5.2.2}{}{p:exercise:vdd}%
Suppose \(a\), \(b\), and \(c\) are integers. Prove that if \(a \mid b\) and \(a \mid c\) then \(a \mid b+c\) and \(a \mid b-c\).%
\par\smallskip%
\noindent\textbf{\blocktitlefont Solution}.\quad{}\begin{solutionproof}
Assume \(a \mid b\) and \(a \mid c\). This means that \(b\) and \(c\) are both multiples of \(a\), so \(b = am\) and \(c = an\) for integers \(m\) and \(n\). Then \(b+c = am+an = a(m+n)\), so \(b+c\) is a multiple of \(a\), or equivalently, \(a \mid b+c\). Similarly, \(b-c = am-an = a(m-n)\), so \(b-c\) is a multiple of \(a\), which is to say \(a \mid b-c\).%
\end{solutionproof}
\end{divisionsolution}%
\begin{divisionsolution}{5.2.3}{}{p:exercise:bkm}%
Write out the remainder classes for \(n = 4\).%
\par\smallskip%
\noindent\textbf{\blocktitlefont Solution}.\quad{}\(\{\ldots, -8, -4, 0, 4, 8, 12, \ldots\}\), \(\{\ldots, -7, -3, 1, 5, 9, 13, \ldots\}\),%
\par
\(\{\ldots, -6, -2, 2, 6, 10, 14, \ldots\}\), and \(\{\ldots, -5, -1, 3, 7, 11, 15, \ldots\}\).%
\end{divisionsolution}%
\begin{divisionsolution}{5.2.4}{}{p:exercise:Hrv}%
What is the largest \(n\) such that \(16\) and \(25\) are in the same remainder class modulo \(n\)?  Write out the remainder class they both belong to and give an example of a number more than 100 in that class.%
\par\smallskip%
\noindent\textbf{\blocktitlefont Solution}.\quad{}If \(16\) and \(25\) are in the same remainder class modulo \(n\), then we would have \(16 \equiv 25 \pmod{n}\), which means \(25 - 16 = 9\) is a multiple of \(n\).  The largest such \(n\) is therefore \(n = 9\).  A number that is in the same remainder class mod 9 is \(106\).%
\end{divisionsolution}%
\begin{divisionsolution}{5.2.5}{}{p:exercise:nyE}%
Let \(a\), \(b\), \(c\), and \(n\) be integers. Prove that if \(a \equiv b \pmod{n}\) and \(c \equiv d \pmod{n}\), then \(a-c \equiv b-d \pmod{n}\).%
\par\smallskip%
\noindent\textbf{\blocktitlefont Solution}.\quad{}\begin{solutionproof}
Assume \(a \equiv b \pmod n\) and \(c \equiv d \pmod n\). This means \(a = b + kn\) and \(c = d + jn\) for some integers \(k\) and \(j\). Consider \(a-c\). We have:%
\begin{equation*}
a-c = b+kn - (d+jn) = b-d + (k-j)n\text{.}
\end{equation*}
%
\par
In other words, \(a-c\) is \(b-d\) more than some multiple of \(n\), so \(a-c \equiv b-d \pmod n\).%
\end{solutionproof}
\end{divisionsolution}%
\begin{divisionsolution}{5.2.6}{}{p:exercise:TFN}%
Find the remainder of \(3^{456}\) when divided by%
\begin{multicols}{4}
\begin{enumerate}[label=(\alph*)]
\item{}2.%
\item{}5.%
\item{}7.%
\item{}9.%
\end{enumerate}
\end{multicols}
%
\par\smallskip%
\noindent\textbf{\blocktitlefont Solution}.\quad{}%
\begin{enumerate}[label=(\alph*)]
\item{}\(3^{456} \equiv 1^{456} = 1 \pmod 2\).%
\item{}\(3^{456} = 9^{228} \equiv (-1)^{228} = 1 \pmod{5}\).%
\item{}\(3^{456} = 9^{228} \equiv 2^{228} = 8^{76} \equiv 1^{76} = 1 \pmod 7\).%
\item{}\(3^{456} = 9^{228} \equiv 0^{228} = 0 \pmod{9}\).%
\end{enumerate}
%
\end{divisionsolution}%
\begin{divisionsolution}{5.2.7}{}{p:exercise:zMW}%
Repeat the previous exercise, this time dividing \(2^{2019}\).%
\end{divisionsolution}%
\begin{divisionsolution}{5.2.8}{}{p:exercise:fUf}%
Determine which of the following congruences have solutions, and find any solutions (between 0 and the modulus) by trial and error.%
\begin{enumerate}[label=(\alph*)]
\item{}\(4x \equiv 5 \pmod 6\).%
\item{}\(6x \equiv 3 \pmod 9\).%
\item{}\(x^2 \equiv 2 \pmod 4\).%
\end{enumerate}
%
\par\smallskip%
\noindent\textbf{\blocktitlefont Solution}.\quad{}For all of these, just plug in all integers between 0 and the modulus to see which, if any, work.%
\begin{enumerate}[label=(\alph*)]
\item{}No solutions.%
\item{}\(x = 2\), \(x = 5\), \(x = 8\).%
\item{}No solutions.%
\end{enumerate}
%
\end{divisionsolution}%
\begin{divisionsolution}{5.2.9}{}{p:exercise:Mbo}%
Determine which of the following congruences have solutions, and find any solutions (between 0 and the modulus) by trial and error.%
\begin{enumerate}[label=(\alph*)]
\item{}\(4x \equiv 5 \pmod 7\).%
\item{}\(6x \equiv 4 \pmod 9\).%
\item{}\(x^2 \equiv 2 \pmod 7\).%
\end{enumerate}
%
\par\smallskip%
\noindent\textbf{\blocktitlefont Solution}.\quad{}For all of these, just plug in all integers between 0 and the modulus to see which, if any, work.%
\begin{enumerate}[label=(\alph*)]
\item{}\(x = 3\).%
\item{}No solutions.%
\item{}\(x = 3\).%
\end{enumerate}
%
\end{divisionsolution}%
\begin{divisionsolution}{5.2.10}{}{p:exercise:six}%
Solve the following congruence \(5x + 8 \equiv 11 \pmod{22}\).  That is, describe the general solution.%
\par\smallskip%
\noindent\textbf{\blocktitlefont Solution}.\quad{}\(x = 5+22k\) for \(k \in \Z\).%
\end{divisionsolution}%
\begin{divisionsolution}{5.2.11}{}{p:exercise:YpG}%
Solve the congruence: \(6x \equiv 4 \pmod{10}\).%
\par\smallskip%
\noindent\textbf{\blocktitlefont Solution}.\quad{}\(x = 4 + 5k\) for \(k \in \Z\).%
\end{divisionsolution}%
\begin{divisionsolution}{5.2.12}{}{p:exercise:EwP}%
Solve the congruence: \(4x \equiv 24 \pmod{30}\).%
\par\smallskip%
\noindent\textbf{\blocktitlefont Solution}.\quad{}\(x = 6 + 15k\) for \(k \in \Z\).%
\end{divisionsolution}%
\begin{divisionsolution}{5.2.13}{}{p:exercise:kDY}%
Solve the congruence: \(341x \equiv 2941 \pmod{9}\).%
\par\smallskip%
\noindent\textbf{\blocktitlefont Hint}.\quad{}First reduce each number modulo 9, which can be done by adding up the digits of the numbers.%
\par\smallskip%
\noindent\textbf{\blocktitlefont Solution}.\quad{}\(x = 2 + 9k\) for \(k \in \Z\).%
\end{divisionsolution}%
\begin{divisionsolution}{5.2.14}{}{p:exercise:QLh}%
I'm thinking of a number. If you multiply my number by 7, add 5, and divide the result by 11, you will be left with a remainder of 2. What remainder would you get if you divided my original number by 11?%
\par\smallskip%
\noindent\textbf{\blocktitlefont Solution}.\quad{}We must solve \(7x + 5 \equiv 2 \pmod{11}\). This gives \(x \equiv 9 \pmod{11}\). In general, \(x = 9 + 11k\), but when you divide any such \(x\) by 11, the remainder will be 9.%
\end{divisionsolution}%
\begin{divisionsolution}{5.2.15}{}{p:exercise:wSq}%
Solve the following linear Diophantine equation, using modular arithmetic (describe the general solutions).%
\begin{equation*}
6x + 10y = 32\text{.}
\end{equation*}
%
\par\smallskip%
\noindent\textbf{\blocktitlefont Solution}.\quad{}Divide through by 2: \(3x + 5y = 16\). Convert to a congruence, modulo 3: \(5y \equiv 16 \pmod 3\), which reduces to \(2y \equiv 1 \pmod 3\). So \(y \equiv 2 \pmod 3\) or \(y = 2 + 3k\). Plug this back into \(3x + 5y = 16\) and solve for \(x\), to get \(x = 2-5k\). So the general solution is \(x = 2-5k\) and \(y = 2+3k\) for \(k \in \Z\).%
\end{divisionsolution}%
\begin{divisionsolution}{5.2.16}{}{p:exercise:cZz}%
Solve the following linear Diophantine equation, using modular arithmetic (describe the general solutions).%
\begin{equation*}
17x + 8y = 31\text{.}
\end{equation*}
%
\par\smallskip%
\noindent\textbf{\blocktitlefont Solution}.\quad{}\(x = 7+8k\) and \(y = -11 - 17k\) for \(k \in \Z\).%
\end{divisionsolution}%
\begin{divisionsolution}{5.2.17}{}{p:exercise:JgI}%
Solve the following linear Diophantine equation, using modular arithmetic (describe the general solutions).%
\begin{equation*}
35x + 47y = 1\text{.}
\end{equation*}
%
\par\smallskip%
\noindent\textbf{\blocktitlefont Solution}.\quad{}\(x = -4-47k\) and \(y = 3 + 35k\) for \(k \in \Z\).%
\end{divisionsolution}%
\begin{divisionsolution}{5.2.18}{}{p:exercise:pnR}%
You have a 13 oz. bottle and a 20 oz. bottle, with which you wish to measure exactly 2 oz. However, you have a limited supply of water. If any water enters either bottle and then gets dumped out, it is gone forever. What is the least amount of water you can start with and still complete the task?%
\par\smallskip%
\noindent\textbf{\blocktitlefont Hint}.\quad{}Solve the Diophantine equation \(13x + 20 y = 2\) (why?).  Then consider which value of \(k\) (the parameter in the solution) is optimal.%
\par\smallskip%
\noindent\textbf{\blocktitlefont Solution}.\quad{}First, solve the Diophantine equation \(13x + 20 y = 2\). The general solution is \(x = -6 - 20k\) and \(y = 4+13k\). Now if \(k = 0\), this correspond to filling the 20 oz. bottle 4 times, and emptying the 13 oz. bottle 6 times, which would require 80 oz. of water. Increasing \(k\) would require considerably more water. Perhaps \(k = -1\) would be better? Then we would have \(x = -6+20 = 14\) and \(y = 4-13 = -11\), which describes the solution where we fill the 13 oz. bottle 14 times, and empty the 20 oz. bottle 11 times. This would require 182 oz. of water. Thus the most efficient procedure is to repeatedly fill the 20 oz bottle, emptying it into the 13 oz bottle, and discarding full 13 oz. bottles, which requires 80 oz. of water.%
\end{divisionsolution}%
%
\backmatter
%
%
%% A lineskip in table of contents as transition to appendices, backmatter
\addtocontents{toc}{\vspace{\normalbaselineskip}}
%
\cleardoublepage
\pagestyle{empty}
\vspace*{\stretch{1}}
\begin{backcolophon}{g:colophon:idp140972671120}%
This book was authored in PreTeXt.%
\end{backcolophon}%
\vspace*{\stretch{2}}
\end{document}