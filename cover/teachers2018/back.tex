\documentclass[10pt]{article}

\usepackage{url}
\usepackage{xcolor}
\usepackage{tikz}
\usepackage{mdframed}
\usepackage[papersize={8.5in,11in}, hmargin={1.75in, 1.75in}, top=.75in]{geometry}

\definecolor{bg}{HTML}{FFBB00}
\pagecolor{bg}
\definecolor{tc}{HTML}{001133}
\definecolor{orange}{HTML}{FF5511}

\usepackage[T1]{fontenc}
\usepackage{newpxtext}
\usepackage[vvarbb,cmintegrals,cmbraces,bigdelims]{newpxmath}
\usepackage[scr=rsfso]{mathalfa}% \mathscr is fancier than \mathcal
\linespread{1.04}         % adds more leading (space between lines)
% quantifiers look strange, so change those back to normal:
	\DeclareSymbolFont{mysymbols}{OMS}{cmsy}{b}{n} %note we make the figures bold to better match newpx.  Replace the ``b'' with an ``m'' to undo this.
	%\SetSymbolFont{mysymbols}  {bold}{OMS}{cmsy}{b}{n}
	%\DeclareSymbolFont{myoperators}   {OT1}{cmr} {m}{n}
	%\SetSymbolFont{myoperators}{bold}{OT1}{cmr} {bx}{n}
	\DeclareMathSymbol{\forall}{\mathord}{mysymbols}{"38}
	\DeclareMathSymbol{\exists}{\mathord}{mysymbols}{"39}
	%\DeclareMathSymbol{\pm}{\mathbin}{mysymbols}{"06}
	%\DeclareMathSymbol{+}{\mathbin}{myoperators}{"2B}
	%\DeclareMathSymbol{-}{\mathbin}{mysymbols}{"00}
	%\DeclareMathSymbol{=}{\mathrel}{myoperators}{"3D}


\begin{document}

\pagestyle{empty}
~
\begin{center}
\begin{tikzpicture}[yshift=-.75in, scale=.9, remember picture, overlay, color=bg!50!orange]
\def\r{.55}
\newcommand{\hexbox}[3]{
  \def\x{-cos{30}*\r*#1+cos{30}*#2*\r*2}
  \def\y{-\r*#1-sin{30}*\r*#1}
  \draw (\x,\y) +(90:\r) -- +(30:\r) -- +(-30:\r) -- +(-90:\r) -- +(-150:\r) -- +(150:\r) -- cycle;
  \draw (\x,\y) node{#3};
}


% Pascal's triangle
%put row of 1's down left side:
  \foreach \row in {0,...,16} {
    \hexbox{\row}{0}{\large 1}
  }
%fill in the rest of the triangle:
  \foreach \row in {1,...,16} {
    \pgfmathsetmacro{\entry}{1};
    \foreach \col in {1,...,\row} {
      % iterative formula : val = precval * (row-col+1)/col
      % (+ 0.5 to bypass rounding errors)
      \pgfmathtruncatemacro{\entry}{\entry*((\row-\col+1)/\col)+0.5};
      \global\let\entry=\entry
      \ifnum \entry<100
	\hexbox{\row}{\col}{\large \entry}
      \else \ifnum \entry<1000
	\hexbox{\row}{\col}{\entry}
      \else \ifnum \entry<10000
	\hexbox{\row}{\col}{\footnotesize \entry}
	\else
	\hexbox{\row}{\col}{\scriptsize \entry}
	\fi
      \fi
      \fi
    }
  }
  % \foreach \row in{17,...,20} {
  % \foreach \col in {1,...,\row} {
  % \hexbox{\row}{\col}{}
  % }
  % }
\end{tikzpicture}
\end{center}
\hspace{-3em}{\color{orange} Mathematics}

{\color{tc}
\vskip 1.5in
\noindent This gentle introduction to discrete mathematics is written future elementary and middle school teachers specializing in mathematics.  The text is a on \emph{Discrete Mathematics: an Open Introduction}, which has been used at the University of Northern Colorado and elsewhere as the primary textbook in the Discrete Mathematics course for math majors.  The \emph{for Teachers} version selects and arranges topics to better suit students in a course designed for elementery education majors concentrating on mathematics.

\vskip 2em

\noindent Four main topics are covered: sequences, counting, logic, and graph theory.  Along the way proofs are introduced, including proofs by contradiction and proofs by induction.

\vskip 2em

\noindent While there are many fine discrete math textbooks available, this text has the following advantages:
\begin{itemize}
\item[--] It is written to be used in an inquiry rich course.
\item[--] It is written to be used in a course for future math teachers.
\item[--] It is open source, with low cost print editions and free electronic editions.
\end{itemize}

\vskip 5em
\begin{center}To download the current version, or for information on obtaining the PreTeXt source, visit: \\\url{http://discretetext.oscarlevin.com/teachers/}.
\end{center}


}
\clearpage

\end{document}



