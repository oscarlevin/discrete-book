 \protect \noindent {\protect \bf  Solutions for Section 0.2} \protect \heading {}{}{Section 0.2\ Solutions} \protect \begin {itemize} 
\begin{ans}{0.2.1.}
    \begin{parts}
	\part $A \cap B = \{3,4,5\}$.  %Find $A \cap B$.
	\part $A \cup B = \{1,2,3,4,5,6,7\}$. %Find $A \cup B$.
	\part $A \setminus B = \{1,2\}$. %Find $A \setminus B$.
	\part $A \times B = \{(1,2), (1,3), (1,5), (2,2), (2,3), (2,5), (3,2), (3,3), (3,5), (4,2), (4,3), (4,5), (5,2), (5,3), (5,5)\}$.
	\part Yes.  %Is $C \subseteq A$?
	\part No. %Is $C \subseteq B$?
    \end{parts}
  
\end{ans}
\begin{ans}{0.2.2.}
    \begin{parts}
  %Find $A \cap B$
	\part $A \cap B = \{4,6,8,10,12\}$
  % Find $A \cup B$.
	\part $A \cup B = \{x \in \N \st (3 \le x \le 13) \vee x \mbox{ is even}\}.$ (the set of all natural numbers which are either even or between 3 and 13 inclusive).
  %Find $B \cap C$.
	\part $B \cap C = \emptyset$.
  %Find $B \cup C$.
	\part $B \cup C = \N$.
      \end{parts}
  
\end{ans}
\begin{ans}{0.2.3.}
    For example, $A = \{2,3,5,7,8\}$ and $B = \{3,5\}$.
  
\end{ans}
\begin{ans}{0.2.4.}
    Let $A = \{1,2,3\}$ and $B = \{1,2,3,4,5,\{1,2,3\}\}$
  
\end{ans}
\begin{ans}{0.2.5.}
    \begin{parts}
	\part No.
	\part No.
	\part $2\Z \cap 3\Z$ is the set of all integers which are multiples of both 2 and 3 (so multiples of 6).  Therefore $2\Z \cap 3\Z = \{x \in \Z \st \exists y\in \Z(x = 6y)\}$.
	\part $2\Z \cup 3\Z$.
 \end{parts}
  
\end{ans}
\begin{ans}{0.2.6.}
    The set of primes.
  
\end{ans}
\begin{ans}{0.2.7.}
%  \begin{multicols}{3}
      \begin{parts}
	  \def\circleA{(-.5,0) circle (1)}
	  \def\circleAlabel{(-1.5,.6) node[above]{$A$}}
	  \def\circleB{(.5,0) circle (1)}
	  \def\circleBlabel{(1.5,.6) node[above]{$B$}}
	  \def\circleC{(0,-1) circle (1)}
	  \def\circleClabel{(.5,-2) node[right]{$C$}}
	  \def\twosetbox{(-2,-1.5) rectangle (2,1.5)}
	  \def\threesetbox{(-2,-2.5) rectangle (2,1.5)}



	    \part  $A \cup \bar B$:

	    \begin{tikzpicture}[fill=gray!50]
	  %Fill A:
	  \fill \circleA;
	  %Fill \bar B:
	    \begin{scope}
	    \clip \circleB \twosetbox; %This defines the scope to everything in the twosetbox which is not in circleB.
	    \fill \twosetbox;
	    \end{scope}
	    \draw[thick] \circleA \circleAlabel \circleB \circleBlabel \twosetbox;
	  \end{tikzpicture}


	  %
	  \part $\bar{(A \cup B)}$:

	  \begin{tikzpicture}[fill=gray!50]
	    \fill \twosetbox;
	    \fill[white] \circleA \circleB;
	    \draw[thick] \circleA \circleAlabel \circleB \circleBlabel \twosetbox;
	  \end{tikzpicture}

%	  \columnbreak

	  %
	  \part $A \cap (B \cup C)$:

	  \begin{tikzpicture}[fill=gray!50]
	  \begin{scope}
	    \clip \circleA;
	    \fill \circleB \circleC;
	  \end{scope}
	  \draw[thick] \circleA \circleAlabel \circleB \circleBlabel \circleC \circleClabel \threesetbox;
	  \end{tikzpicture}

	  %
	  \part $(A \cap B) \cup C$:

	  \begin{tikzpicture}[fill=gray!50]
	  \begin{scope}
	    \clip \circleA;
	    \fill \circleB;
	  \end{scope}
	  \fill \circleC;
	  \draw[thick] \circleA \circleAlabel \circleB \circleBlabel \circleC \circleClabel \threesetbox;
	  \end{tikzpicture}


%	  \columnbreak
	  %
	  \part $\bar A \cap B \cap \bar C$:

	  \begin{tikzpicture}[fill=gray!50]
	  \fill \circleB;
	  \begin{scope}
	    \clip \circleB;
	    \fill[white] \circleA \circleC;
	  \end{scope}

	  \draw[thick] \circleA \circleAlabel \circleB \circleBlabel \circleC \circleClabel \threesetbox;
	  \end{tikzpicture}

	  %
	  \part $(A \cup B) \setminus C$:

	  \begin{tikzpicture}[fill=gray!50]
	  \fill \circleA;
	  \fill \circleB;
	  \fill[white] \circleC;
	  \draw[thick] \circleA \circleAlabel \circleB \circleBlabel \circleC \circleClabel \threesetbox;
	  \end{tikzpicture}

	  \end{parts}
%	   \end{multicols}
  
\end{ans}
\begin{ans}{0.2.8.}
    For example, $A \cup B \cap \bar{(A \cap B)}$.  Note that $\bar{A \cap B}$ would almost work, but also contain the area outside of both circles.
  
\end{ans}
\begin{ans}{0.2.9.}
      \begin{parts}
	  \part 34.
	  \part 103.
	  \part 8.
      \end{parts}
  
\end{ans}
\begin{ans}{0.2.10.}
    $\pow(A) = \{\emptyset, \{a\}, \{b\}, \{c\}, \{a,b\}, \{a,c\}, \{b,c\}, \{a,b,c\}\}$.
  
\end{ans}
\begin{ans}{0.2.11.}
      There are 10 singletons.  There are 45 doubletons (because $45 = 9+8+7+\cdots+2+1$).
  
\end{ans}
\begin{ans}{0.2.12.}
      $\{2,3,5\}, \{1,2,3,5\}, \{2,3,4,5\}, \{2,3,5,6\}, \{1,2,3,4,5\}, \{1,2,3,5,6\}, \{2,3,4,5,6\}$, and $\{1,2,3,4,5,6\}$.
  
\end{ans}
\begin{ans}{0.2.13.}
   For example $A = \{1,2,3,4\}$ and $B = \{5,6,7,8,9\}$.
  
\end{ans}
\begin{ans}{0.2.14.}
    For example, $A = \{1,2,3\}$ and $B = \{2,3,4,5\}$.
  
\end{ans}
\begin{ans}{0.2.15.}
	 No.  There must be 5 elements in common to both sets.  Since there are 10 distinct elements all together in $A$ and $B$, this means that between $A$ and $B$, there must be 5 elements which they do not have in common (some in $A$ but not in $B$, some in $B$ but not in $A$).  But 5 is odd, so to have $|A| = |B|$, we would need 7.5 elements in each set, which is impossible.
	
\end{ans}
\begin{ans}{0.2.16.}
      If $R$ is the set of red cards and $F$ is the set of face cards, we want to find $|R \cup F|$.  This is not simply $|R| + |F|$ because there are 6 cards which are both red and a face card; $|R \cap F| = 6$.  We find $|R \cup F| = 32$.
  
\end{ans}
\protect \end {itemize}
 \protect \noindent {\protect \bf  Solutions for Section 0.3} \protect \heading {}{}{Section 0.3\ Solutions} \protect \begin {itemize} 
\begin{ans}{0.3.1.}
	There are 8 different functions.  For example, $f(1) = a$, $f(2) = a$, $f(3) = a$; or $f(1) = a$, $f(2) = b$, $f(3) = a$, and so on.  None of the functions are injective.  Exactly 6 of the functions are surjective.  No functions are both (since no functions here are injective).
	
\end{ans}
\begin{ans}{0.3.2.}
	There are nine functions - you have a choice of three outputs for $f(1)$, and for each, you have three choices for the output $f(2)$.  Of these functions, 6 are injective, 0 are surjective, and 0 are both.
	
\end{ans}
\begin{ans}{0.3.3.}
		\begin{parts}
		%Is $f$ injective?  Explain.
		\part $f$ is not injective, since $f(2) = f(5)$ - two different inputs have the same output.
		% Is $f$ surjective?  Explain.
		\part $f$ is surjective, since every element of the codomain is an element of the range.
		\end{parts}
	
\end{ans}
\begin{ans}{0.3.4.}
		\begin{parts}
		%Is $f$ injective?  Explain.
		  \part $f$ is not injective, since $f(1) = 3$ and $f(4) = 3$.
		%   Is $f$ surjective?  Explain.
		  \part $f$ is not surjective, since there is no input which gives 2 as an output.
		\end{parts}
	
\end{ans}
\begin{ans}{0.3.5.}
		\begin{parts}
		% $f:\N \to \N$ given by $f(n) = n+4$.
		  \part $f$ is injective, but not surjective.
		%   $f:\Z \to \Z$ given by $f(n) = n+4$.
		  \part $f$ is injective and surjective.
		  %$f:\Z \to \Z$ given by $f(n) = 5n - 8$.
		  \part $f$ is injective, but not surjective.
		%   $f:\Z \to \Z$ given by $f(n) = \begin{cases}
		%                                          n/2 & \mbox{ if $n$ is even}\\
		%                                          (n+1)/2 & \mbox{ if $n$ is odd}.
		%                                        \end{cases}$
		 \part $f$ is not injective, but is surjective.
		\end{parts}
	
\end{ans}
\begin{ans}{0.3.6.}
		\begin{parts}
		% Is $f$ injective?  Prove your answer.
		  \part $f$ is not injective.  To prove this, we must simply find two different elements of the domain which map to the same element of the codomain.  Since $f(\{1\}) = 1$ and $f(\{2\}) = 1$, we see that $f$ is not injective.
		%   Is $f$ surjective?  Prove your answer.
		  \part $f$ is not surjective.  The largest subset of $A$ is $A$ itself, and $|A| = 10$.  So no natural number greater than 10 will ever be an output.
		%   Find $f\inv(1)$.
		  \part $f\inv(1) = \{\{1\}, \{2\}, \{3\}, \ldots \{10\}\}$ (the set of all the singleton subsets of $A$).
		%   Find $f\inv(0)$.
		  \part $f\inv(0) = \{\emptyset\}$.  Note, it would be wrong to write $f\inv(0) = \emptyset$ - that would claim that there is no input which has 0 as an output.
		   % Find $f\inv(12)$.
		  \part $f\inv(12) = \emptyset$, since there are no subsets of $A$ with cardinality 12.
		\end{parts}
	
\end{ans}
\begin{ans}{0.3.7.}
		\begin{parts}
		% Find $f\inv(3)$.
		  \part $f\inv(3) = \{003, 030, 300, 012, 021, 102, 201, 120, 210, 111\}$
		%   Find $f\inv(28)$.
		  \part $f\inv(28) = \emptyset$ (since the largest sum of three digits is $9+9+9 = 27$)
		%   Use one of the parts above to prove that $f$ is not injective.
		  \part Part (a) proves that $f$ is not injective - the output 3 is assigned to 10 different inputs.
		%   Use one of the parts above to prove that $f$ is not surjective.
		  \part Part (b) proves that $f$ is not surjective - there is an element of the codomain (28) which is assigned to no inputs.
		\end{parts}
	
\end{ans}
\begin{ans}{0.3.8.}
		\begin{parts}
			\part $|f\inv(3)| \le 1$.  In other words, either $f\inv(3)$ is the emptyset or is a set containing exactly one element.  Injective functions cannot have two elements from the domain both map to 3. %$f$ is injective? Explain.
			\part $|f\inv(3)| \ge 1$.  In other words, $f\inv(3)$ is a set containing at least one elements, possibly more.  Surjective functions cannot have nothing mapping to 3.%$f$ is surjective? Explain.
			\part $|f\inv(3)| = 1$.  There is exactly one element from $X$ which gets mapped to 3, so $f\inv(3)$ is the set containing that one element. %$f$ is bijective? Explain.
		\end{parts}
	
\end{ans}
\begin{ans}{0.3.9.}
		$X$ can really be any set, as long as $f(x) = 0$ or $f(x) = 1$ for every $x \in X$.  For example, $X = \N$ and $f(n) = 0$ works.
	
\end{ans}
\begin{ans}{0.3.10.}
		\begin{parts}
		% there is a injective function $f:X \to Y$.  Explain.
		  \part $|X| \le |Y|$ - otherwise two or more of the elements of $X$ would need to map to the same element of $Y$.
		%   there is a surjective function $f:X \to Y$.  Explain.
		  \part $|X| \ge |Y|$ - otherwise there would be one or more elements of $Y$ which were never an output.
		%   there is a bijection $f:X \to Y$.  Explain.
		  \part $|X| = |Y|$.  This is the only way for both of the above to occur.
		\end{parts}
	
\end{ans}
\begin{ans}{0.3.11.}
		\begin{parts}
		% $f$ is injective but not surjective.
		  \part Yes. (Can you give an example?)
		  \part Yes. %$f$ is surjective but not injective.
		  \part Yes. %$|X| = |Y|$ and $f$ is injective but not surjective.
		  \part Yes. %$|X| = |Y|$ and $f$ is surjective but not injective.
		  \part No. %$|X| = |Y|$, $X$ and $Y$ are finite, and $f$ is injective but not surjective.
		  \part No. %$|X| = |Y|$, $X$ and $Y$ are finite, and $f$ is surjective but not injective.
		\end{parts}
	
\end{ans}
\begin{ans}{0.3.12.}
		\begin{parts}
		% Is $f$ injective?  Prove your answer.
		  \part $f$ is injective.
		  \begin{proof}
		   Let $x$ and $y$ be elements of the domain $\Z$.  Assume $f(x) = f(y)$.  If $x$ and $y$ are both even, then $f(x) = x+1$ and $f(y) = y+1$.  Since $f(x) = f(y)$, we have $x + 1 = y + 1$ which implies that $x = y$.  Similarly, if $x$ and $y$ are both odd, then $x - 3 = y-3$ so again $x = y$.  The only other possibility is that $x$ is even an $y$ is odd (or visa-versa).  But then $x + 1$ would be odd and $y - 3$ would be even, so it cannot be that $f(x) = f(y)$.  Therefore if $f(x) = f(y)$ we then have $x = y$, which proves that $f$ is injective.
		  \end{proof}
		% Is $f$ surjective?  Prove your answer.
		  \part $f$ is surjective.
		  \begin{proof}
		   Let $y$ be an element of the codomain $\Z$.  We will show there is an element $n$ of the domain ($\Z$) such that $f(n) = y$.  There are two cases.  First, if $y$ is even, then let $n = y+3$.  Since $y$ is even, $n$ is odd, so $f(n) = n-3 = y+3-3 = y$ as desired.  Second, if $y$ is odd, then let $n = y-1$.  Since $y$ is odd, $n$ is even, so $f(n) = n+1 = y-1+1 = y$ as needed.  Therefore $f$ is surjective.
		  \end{proof}

		\end{parts}
	
\end{ans}
\begin{ans}{0.3.13.}
	   Yes, this is a function, if you choose the domain and codomain correctly.  The domain will be the set of students, and the codomain will be the set of possible grades.  The function is almost certainly not injective, because it is likely that two students will get the same grade.  The function might be surjective - it will be if there is at least one student who gets each grade.
	
\end{ans}
\begin{ans}{0.3.14.}
		Yes, as long as the set of cards is the domain and the set of players is the codomain.  The function is not injective because multiple cards go to each player.  It is surjective since all players get cards.
	
\end{ans}
\begin{ans}{0.3.15.}
	  This cannot be a function.  If the domain were the set of cards, then it is not a function because not every card gets dealt to a player.  If the domain were the set of players, it would not be a function because a single player would get mapped to multiple cards.  Since this is not a function, it doesn't make sense to say whether it is injective/surjective/bijective.
	
\end{ans}
\protect \end {itemize}
 \protect \noindent {\protect \bf  Solutions for Section 1.1} \protect \heading {}{}{Section 1.1\ Solutions} \protect \begin {itemize} 
\begin{ans}{1.1.1.}
    255.
  
\end{ans}
\begin{ans}{1.1.2.}
    8.
  
\end{ans}
\begin{ans}{1.1.3.}
    15.
  
\end{ans}
\begin{ans}{1.1.4.}
    \begin{parts}
      \part $2^8 = 256$.  You have two choices for each tie - wear it or don't. %You must select some of your ties to wear - everything is okay, from no ties up to all ties.  How many choices do you have?
      \part You have 7 choices for regular ties (the 8 choices less the ``no regular tie'' option) and 31 choices for bow ties (32 total minus the ``no bow tie'' option).  Thus total you have $7 \cdot 31 = 217$.  %If you want to wear at least one regular tie and one bow tie, but are willing to wear up to all your ties, how many choices do you have for which ties to wear?
      \part ${3\choose 2}{5\choose 3} = 30$  %How many choices do you have if you wear exactly 2 of the 3 regular ties and 3 of the 5 bow ties?
      \part $5! = 120$  %Once you have selected 2 regular and 3 bow ties, in how many orders could you put the ties on, assuming you must have one of the three bow ties on top?
    \end{parts}
  
\end{ans}
\begin{ans}{1.1.5.}
    \begin{parts}
      \part 16 is the number of choices you have if you want to watch one movie, either a comedy or horror flick.
      \part 63 is the number of choices you have if you will watch two movies, first a comedy and then a horror.
    \end{parts}
  
\end{ans}
\begin{ans}{1.1.6.}
    $0 \le |A \cap B| \le 10$ and $15 \le |A \cup B| \le 25$.
  
\end{ans}
\begin{ans}{1.1.7.}
      $|A \cup B| + |A \cap B| = 13$
  
\end{ans}
\begin{ans}{1.1.8.}
    39.
  
\end{ans}
\begin{ans}{1.1.9.}
      $|(A \cup C)\cap \bar B| = 44$.  Use a Venn diagram.
    
\end{ans}
\begin{ans}{1.1.10.}
	One possibility: $(A \cup B) \cap C$.
    
\end{ans}
\begin{ans}{1.1.11.}
    \begin{parts}
      \part $8^5$, since you select from 8 letters 5 times.  %How many of these words are there total?
      \part $8\cdot 7\cdot 6\cdot 5\cdot 4$.  After selecting a letter, you have fewer letters to select for the next one.  %How many of these words contain no repeated letters?
      \part 64 - you need to select the 4th and 5th letters. %How many of these words (repetitions allowed) start with the sub-word ``aha''?
      \part $64 + 64 - 0 = 128$.  There are 64 words which start with ``aha'' and another 64 words that end with ``bah.''  Perhaps we over counted the words that both start with ``aha'' and end with ``bah'' but since the words are only 5 letters long, there are no such words.  %How many of these words (repetitions allowed) either start with ``aha'' or end with ``bah'' or both?
      \part $(8\cdot 7\cdot 6\cdot 5\cdot 4) - 3\cdot (5\cdot 4) = 6660$ - all the words minus the bad ones.  The taboo word can be in any of three positions (starting with letter 1, 2, or 3) and for each position we must choose the other two letters (from the remaining 5 letters) %How many of the words containing no repeats also do not contain the sub-word ``bad'' (in consecutive letters)?
    \end{parts}
  
\end{ans}
\protect \end {itemize}
 \protect \noindent {\protect \bf  Solutions for Section 1.2} \protect \heading {}{}{Section 1.2\ Solutions} \protect \begin {itemize} 
\begin{ans}{1.2.1.}
    ${10 \choose 6} + {10\choose 7} + {10\choose 8} + {10 \choose 9} + {10\choose 10} = 386$
  
\end{ans}
\begin{ans}{1.2.2.}
    Use the binomial theorem.  ${14\choose 9} + {15 \choose 6}2^9$.
  
\end{ans}
\begin{ans}{1.2.3.}
    \begin{parts}
      \part $2^6 = 64$  %How many subsets are there total?
      \part $2^3 = 8$.  We need to select yes/no for each of the remaining three elements.  %How many subsets contain $\{2,3,5\}$ as a subset?
      \part $2^3 = 8$.  We need to decide yes/no for the three non-prime elements.  %How many subsets of $S$ contain no prime numbers?
      \part $2^6 - 2^3 = 56$.  There are 8 subsets which do not contain any odd numbers. %How many subsets contain at least one odd number?
      \part 9.  We need to select one odd (3 choices) and one even (3 choices).  %How many doubletons (i.e., subsets of two elements) contain exactly one even number?
    \end{parts}
  
\end{ans}
\begin{ans}{1.2.4.}
    \begin{parts}
      \part ${14 \choose 7}$ %end at (10,10)?
      \part ${6 \choose 2}{8\choose 5}$ %end at (10,10) and pass through (5,7)?
      \part ${14 \choose 7} - {6\choose 2}{8 \choose 5}$ %end at (10,10) and avoid (5,7)?
    \end{parts}
  
\end{ans}
\protect \end {itemize}
 \protect \noindent {\protect \bf  Solutions for Section 1.3} \protect \heading {}{}{Section 1.3\ Solutions} \protect \begin {itemize} 
\begin{ans}{1.3.1.}
    \begin{parts}
      \part ${10 \choose 3}$ %How many 3-topping pizzas could they put on their menu?  Assume double toppings are not allowed.
      \part $2^{10}$ %How many total pizzas are possible, with between zero and ten toppings (but not double toppings) allowed?
      \part $P(10,5)$  %The pizza parlor will list the 10 toppings in two columns on their menu.  How many ways can they arrange the toppings in the left column?
    \end{parts}
  
\end{ans}
\begin{ans}{1.3.2.}
    ${7\choose 2}{7\choose 2}$
  
\end{ans}
\begin{ans}{1.3.3.}
    \begin{parts}
      \part 5 (you need to skip one dot the top and the bottom). %Squares?
      \part ${7 \choose 2}$ - once you select the two dots on the top, the bottom two are determined. % Rectangles?
      \part This is tricky - you need to worry about running out of space.  One way to count: break into cases by the location of the top left corner.  You get ${7 \choose 2} + ({7 \choose 2}-1) + ({7 \choose 2} - 3) + ({7 \choose 2} - 6) + ({7 \choose 2} - 10) + ({7 \choose 2} - 15)$ %Parallelograms?
      \part All of them %Trapezoids?
    \end{parts}
  
\end{ans}
\begin{ans}{1.3.4.}
    \begin{parts}
      \part ${20 \choose 4}{16 \choose 4}{12 \choose 4}{8 \choose 4}{4 \choose 4}$ %You need to divide up into foursomes (groups of 4 people): a first foursome, a second foursome, and so on.  How many ways can you do this?
      \part $5!{15 \choose 3}{12 \choose 3}{9 \choose 3}{6 \choose 3}{3 \choose 3}$ %After all your hard work, you realize that in fact, you want each foursome to include one of the five CEO's.  How many ways can you do this?
    \end{parts}
  
\end{ans}
\begin{ans}{1.3.5.}
     $9!$ (there are 10 people seated around the table, but it does not matter where King Arthur sits, only who sits to his left, two seats to his left, and so on).
  
\end{ans}
\protect \end {itemize}
 \protect \noindent {\protect \bf  Solutions for Section 1.4} \protect \heading {}{}{Section 1.4\ Solutions} \protect \begin {itemize} 
\begin{ans}{1.4.1.}
	\begin{proof}
        \underline{Question}: How many subsets of $A = {1,2,3, \ldots, n+1}$ contain exactly two elements?

        \underline{Answer 1}: We must choose 2 elements from $n+1$ choices, so there are ${n+1 \choose 2}$ subsets.

        \underline{Answer 2}: We break this question down into cases, based on what the larger of the two elements in the subset is. The larger element can't be 1, since we need at least one element smaller than it.

        Larger element is 2: there is 1 choice for the smaller element.

        Larger element is 3: there are 2 choices for the smaller element.

        Larger element is 4: there are 3 choices for the smaller element.

        And so on.  When the larger element is $n+1$, there are $n$ choices for the smaller element.  Since each two element subset must be in exactly one of these cases, the total number of two element subsets is $1 + 2 + 3 + \cdots + n$.

        Answer 1 and answer 2 are both correct, so they must be equal.  Therefore
        \[1 + 2 + 3 + \cdots + n = {n+1 \choose 2}\]
       \end{proof}
	
\end{ans}
\begin{ans}{1.4.2.}
		\begin{parts}
		 \part She has ${15 \choose 6}$ ways to select the 6 bride's maids, and then for each way, has 6 choices for the maid of honor.  Thus she has ${15 \choose 6}6$ choices.  %What if she first selects the 6 bride's maids, and then selects one of them to be the maid of honor?
		 \part She has 15 choices for who will be her maid of honor.  Then she needs to select 5 of the remaining 14 friends to be bride's maids, which she can do in ${14 \choose 5}$ ways.  Thus she has $15 {14 \choose 5}$ choices.  %What if she first selects her maid of honor, and then 5 other bride's maids?
		 \part We have answered the question (how many wedding parties can the bride choose from) in two ways.  The first way gives the left hand side of the identity and the second way gives the right hand side of the identity.  Therefore the identity holds. %Explain why $6 {15 \choose 6} = 15 {14 \choose 5}$.
		\end{parts}
	
\end{ans}
\begin{ans}{1.4.3.}
		\begin{parts}
		 \part After the 1, we need to find a 5-bit string with one 1.  There are ${5 \choose 1}$ ways to do this. %How many of those bit strings start with 1?
		 \part ${4 \choose 1}$ (we need to pick 1 of the remaining 4 slots to be the second 1). %How many of those bit strings start with 01?
		 \part ${3 \choose 1}$ %How many of those bit strings start with 001?
		 \part Yes.  We still need strings starting with 0001 (there are ${2 \choose 1}$ of these) and strings starting 00001 (there is only ${1 \choose 1} = 1$ of these).  %Are there any other strings we have not counted yet?  Which ones, and how many are there?
		 \part ${6 \choose 2}$ %How many bit strings are there total in $\b B^6_2$?
		 \part An example of the Hockey Stick Theorem:  %What binomial identity have you just given a combinatorial proof for?
		 \[{1 \choose 1} + {2 \choose 1} + {3 \choose 1} + {4 \choose 1} + {5 \choose 1} = {6 \choose 2}\]
		\end{parts}
	
\end{ans}
\begin{ans}{1.4.4.}
		\begin{parts}
		 \part $3^n$, since there are 3 choices for each of the $n$ digits.  %How many ternary digit strings contain exactly $n$ digits?
		 \part $1$, since all the digits need to be 2's.  However, we might write this as ${n \choose 0}$.  %How many ternary digit strings contain exactly $n$ digits and $n$ 2's.
		 \part There are ${n \choose 1}$ places to put the non-2 digit.  That digit can be either a 0 or a 1, so there are $2{n \choose 1}$ such strings.  %How many ternary digit strings contain exactly $n$ digits and $n-1$ 2's.  (Hint: where can you put the non-2 digit, and then what could it be?)
		 \part We must choose two slots to fill with 0's or 1's.  There are ${n \choose 2}$ ways to do that.  Once the slots are picked, we have two choices for the first slot (0 or 1) and two choices for the second slot (0 or 1).  So there are a total of $2^2{n \choose 2}$ such strings. %How many ternary digit strings contain exactly $n$ digits and $n-2$ 2's.  (Hint: see previous hint)
		 \part There are ${n \choose k}$ ways to pick which slots don't have the 2's.  Then those slots can be filled in $2^k$ ways (0 or 1 for each slot).  So there are $2^k{n \choose k}$ such strings. %How many ternary digit strings contain exactly $n$ digits and $n-k$ 2's.
		 \part These strings contain just 0's and 1's - so they are bit strings.  There are $2^n$ bit strings.  But keeping with the pattern above, we might write this as $2^n {n \choose n}$. %How many ternary digit strings contain exactly $n$ digits and no 2's. (Hint: what kind of a string is this?)
		 \part We answer the question of how many length $n$ ternary digit strings there are in two ways.  First, each digit can be one of three choices, so the total number of strings is $3^n$.  On the other hand, we could break the question down into cases by how many of the digits are 2's.  If they are all 2's, then there are ${n \choose 0}$ strings.  If all but one is a 2, then there are $2{n \choose 1}$ strings.  If all but 2 of the digits are 2's, then there are $2^2{n \choose 2}$ strings - we choose 2 of the $n$ digits to be non-2, and then there are 2 choices for each of those digits.  And so on for every possible number of 2's in the string.  %Use the above parts to give a combinatorial proof for the identity
		 %\[{n \choose 0} + 2{n \choose 1} + 2^2{n \choose 2} + 2^3{n \choose 3} + \cdots + 2^n{n \choose n} = 3^n\]
		\end{parts}
	
\end{ans}
\begin{ans}{1.4.5.}
		\begin{proof}
         \underline{Question}: How many $k$-letter words can you make using $n$ different letters without repeating any letter?

         \underline{Answer 1}: There are $n$ choices for the first letter, $n-1$ choices for the second letter, $n-2$ choices for the third letter, and so on until $n - (k-1)$ choices for the $k$th letter (since $k-1$ letters have already been assigned at that point).  The product of these numbers can be written $\frac{n!}{(n-k)!}$ which is $P(n,k)$.

         \underline{Answer 2}: First pick $k$ letters to be in the word from the $n$ choices.  This can be done in ${n \choose k}$ ways.  Now arrange those letters into a word - there are $k$ choices for the first letter, $k-1$ choices for the second, and so on, for a total of $k!$ arrangements of the $k$ letters.  Thus the total number of words is ${n \choose k}k!$.
        \end{proof}
	
\end{ans}
\protect \end {itemize}
 \protect \noindent {\protect \bf  Solutions for Section 1.5} \protect \heading {}{}{Section 1.5\ Solutions} \protect \begin {itemize} 
\begin{ans}{1.5.1.}
	 \begin{parts}
	   \part ${18 \choose 4}$.  Each outcome can be represented by a sequence of 14 stars and 4 bars. %How many ways can you do this if there are no restrictions?
	   \part ${13 \choose 4}$.  First put one ball in each bin.  This leaves 9 stars and 4 bars.%How many ways can you do this if each bin must contain at least one dodge-ball?
	   \part ${18 \choose 4} - \left[ {5 \choose 1}{11 \choose 4} - {5 \choose 2}{4 \choose 4}\right]$.  Subtract all the distributions for which one or more bins contain 7 or more balls.  %How many ways can you do this if no bin can hold more than 6 balls?
	 \end{parts}
	
\end{ans}
\begin{ans}{1.5.2.}
	\begin{parts}
	  \part ${7 \choose 2}$.  After each variable gets 1 star for free, we are left with 5 stars and 2 bars.  %$x$, $y$, and $z$ are all positive?
	  \part ${10 \choose 2}$.  We have 8 stars and 2 bars.  %$x$, $y$, and $z$ are all non-negative?
	  \part ${19 \choose 2}$.  This problem is equivalent to finding the number of solutions to $x' + y' + z' = 17$ where $x'$, $y'$ and $z'$ are non-negative.  (In fact, we really just do a substitution.  Let $x = x'- 3$, $y = y' - 3$ and $z = z' - 3$).  %$x$, $y$, and $z$ are all greater than $-3$.
	\end{parts}
	
\end{ans}
\begin{ans}{1.5.3.}
	${10 \choose 5}$.  We have 5 stars (the five dice) and 5 bars (the five switches between the number 1-6).
	
\end{ans}
\begin{ans}{1.5.4.}
	${18 \choose 3}$.  Distribute 10 units to the variables before finding all solutions to $x_1' + x_2' + x_3' + x_4' = 15$ in non-negative integers.
	
\end{ans}
\protect \end {itemize}
 \protect \noindent {\protect \bf  Solutions for Section 1.6} \protect \heading {}{}{Section 1.6\ Solutions} \protect \begin {itemize} 
\begin{ans}{1.6.1.}
	There are 8 different functions.  For example, $f(1) = a$, $f(2) = a$, $f(3) = a$; or $f(1) = a$, $f(2) = b$, $f(3) = a$, and so on.  None of the functions are injective.  Exactly 6 of the functions are surjective.  No functions are both (since no functions here are injective).
	
\end{ans}
\begin{ans}{1.6.2.}
	There are nine functions - you have a choice of three outputs for $f(1)$, and for each, you have three choices for the output $f(2)$.  Of these functions, 6 are injective, 0 are surjective, and 0 are both.
	
\end{ans}
\begin{ans}{1.6.3.}
	\begin{parts}
	  \part $6^4 = 1296$, since there are six choices of where to send each of the 4 elements of the domain. %How many functions are there total?
	  \part $P(6, 4) = 6 \cdot 5 \cdot 4 \cdot 3 = 360$, since outputs cannot be repeated.  %How many functions are injective?
	  \part None. %How many functions are surjective?
	  \part There are $5 \cdot 6^3$ functions for which $f(1) \ne a$ and another $5 \cdot 6^3$ functions for which $f(2) \ne b$.  There are $5^2 \cdot 6^2$ functions for which both $f(1) \ne a$ and $f(2) \ne b$.  So the total number of functions for which $f(1) \ne a$ or $f(2) \ne b$ or both is
	  \[5 \cdot 6^3 + 5 \cdot 6^3 - 5^2 \cdot 6^2 = 1260\] %How many functions have the property that $f(1) \ne a$ or $f(2) \ne b$, or both?
	\end{parts}
	
\end{ans}
\begin{ans}{1.6.4.}
	\begin{parts}
	  \part $17^{10}$ %How many functions $f: A \to B$ are there?
	  \part $P(17, 10)$  %How many functions $f: A \to B$ are injective?
	\end{parts}
	
\end{ans}
\protect \end {itemize}
 \protect \noindent {\protect \bf  Solutions for Section 1.7} \protect \heading {}{}{Section 1.7\ Solutions} \protect \begin {itemize} 
\begin{ans}{1.7.1.}
	$5^{10} - \left[{5 \choose 1}4^{10} - {5 \choose 2}3^{10} + {5 \choose 3}2^{10} - {5 \choose 4}1^{10}\right]$ %Consider sets $A$ and $B$ with $|A| = 10$ and $|B| = 5$.  How many functions $f: A \to B$ are surjective?
	
\end{ans}
\begin{ans}{1.7.2.}
	$5! - \left[{5 \choose 1}4! - {5 \choose 2}3! + {5 \choose 3}2! - {5 \choose 4}1! + {5 \choose 5}0!\right]$.  This is a sneaky way to as for the number of derangements on 5 elements. %Let $A = \{1,2,3,4,5\}$.  How many injective functions $f:A \to A$ have the property that for each $x \in A$, $f(x) \ne x$?
	
\end{ans}
\begin{ans}{1.7.3.}
	${10 \choose 6}\left(4! - \left[{4 \choose 1} 3! - {4 \choose 2}2! + {4 \choose 3}1! - {4 \choose 4}0!\right]\right)$.  We choose 6 of the 10 ladies to get their own hat, and the multiply by the number of ways the remaining hats can be deranged.
	
\end{ans}
\protect \end {itemize}
 \protect \noindent {\protect \bf  Solutions for Section 2.1} \protect \heading {}{}{Section 2.1\ Solutions} \protect \begin {itemize} 
\begin{ans}{2.1.1.}
		\begin{parts}
		%$2, 5, 10, 17, 26, \ldots$
		  \part $a_n = n^2 + 1$
		%   $0, 2, 5, 9, 14, 20, \ldots$
		  \part $a_n = \frac{n(n+1)}{2} - 1$
		%   $8, 12, 17, 23, 30,\ldots$
		  \part $a_n = \frac{(n+2)(n+3)}{2} + 2$
		%   $1, 5, 23, 119, 719,\ldots$
		  \part $a_n = (n+1)! - 1$ (where $n! = 1 \cdot 2 \cdot 3 \cdots n$)
		\end{parts}
	
\end{ans}
\begin{ans}{2.1.2.}
		\begin{parts}
		%Give the recursive definition for the sequence.
		  \part $F_n = F_{n-1} + F_{n-2}$ with $F_0 = 0$ and $F_1 = 1$.
		%   Write out the first few terms of the sequence of partial sums.
		  \part  $0, 1, 2, 4, 7, 12, 20, \ldots$
		  %Give a closed formula for the sequence of partial sums in terms of $F_n$  (for example, you might say $F_0 + F_1 + \cdots + F_n = 3F_{n-1}^2 + n$, although that is definitely not correct).
		  \part $F_0 + F_1 + \cdots + F_n = F_{n+2} - 1$
		\end{parts}
	
\end{ans}
\begin{ans}{2.1.3.}
		$3, 10, 24, 52, 108,\ldots$.  The recursive definition for $10, 24, 52, \ldots$ is $a_n = 2a_{n-1} + 4$ with $a_1 = 10$.
	
\end{ans}
\begin{ans}{2.1.4.}
		$-1, -1, 1, 5, 11, 19,\ldots$  Thus the sequence $0, 2, 6, 12, 20,\ldots$ has closed formula $a_n = (n+1)^2 - 3(n+1) + 2$.
	
\end{ans}
\protect \end {itemize}
 \protect \noindent {\protect \bf  Solutions for Section 2.2} \protect \heading {}{}{Section 2.2\ Solutions} \protect \begin {itemize} 
\begin{ans}{2.2.1.}
		\begin{parts}
		% What is the next term in the sequence?
		\part 32.
		% Find a formula for the $n$th term of this sequence, assuming $a_1 = 8$.
		\part $a_n = 8 + 6(n-1)$
		% Find the sum of the first 100 terms of the sequence: $\sum_{k=1}^{100}a_k$.
		\part $30500$.
		\end{parts}
	
\end{ans}
\begin{ans}{2.2.2.}
		\begin{parts}
		% How many terms are there in the sequence?
		\part $n+2$ terms.
		\part $6n+1$. %second to last term
		%Find the sum of all the terms in the sequence.
		\part $\frac{(6n+8)(n+2)}{2}$
		\end{parts}
	
\end{ans}
\begin{ans}{2.2.3.}
		68117
	
\end{ans}
\begin{ans}{2.2.4.}
		$\frac{5-5\cdot 3^{21}}{-2}$
	
\end{ans}
\begin{ans}{2.2.5.}
		$\frac{1 + \frac{2^{31}}{3^{31}}}{5/3}$
	
\end{ans}
\begin{ans}{2.2.6.}
		For arithmetic: $x = 55/3$, $y = 29/3$.  For geometric: $x = 9$ and $y = 3$.
	
\end{ans}
\begin{ans}{2.2.7.}
		\begin{parts}
		  \part $\d\sum_{k=1}^n 2k$		%$2 + 4 + 6 + 8 + \cdots + 2n$
		  \part $\d\sum_{k=1}^{107} (1 + 4(k-1))$		%$1 + 5 + 9 + 13 + \cdots + 425$
		  \part $\d\sum_{k=1}^{50} \frac{1}{k}$		%$1 + \frac{1}{2} + \frac{1}{3} + \frac{1}{4} + \cdots + \frac{1}{50}$
		  \part $\d\prod_{k=1}^n 2k$		%$2 \cdot 4 \cdot 6 \cdot \cdots \cdot 2n$
		  \part $\d\prod_{k=1}^{100} \frac{k}{k+1}$	%$(\frac{1}{2})(\frac{2}{3})(\frac{3}{4})\cdots(\frac{100}{101})$
		\end{parts}
	
\end{ans}
\begin{ans}{2.2.8.}
		\begin{parts}
		  \part $\d\sum_{k=1}^{100} (3+4k) = 7 + 11 + 15 + \cdots + 403$
		  \part $\d\sum_{k=0}^n 2^k = 1 + 2 + 4 + 8 + \cdots + 2^n$
		  \part $\d\sum_{k=2}^{50}\frac{1}{(k^2 - 1)} = 1 + \frac{1}{3} + \frac{1}{8} + \frac{1}{15} + \cdots + \frac{1}{2499}$
		  \part $\d\prod_{k=2}^{100}\frac{k^2}{(k^2-1} = \frac{4}{3}\cdot\frac{9}{8}\cdot\frac{16}{15}\cdots\frac{10000}{9999}$
		  \part $\d\prod_{k=0}^n (2+3k) = (2)(5)(8)(11)(14)\cdots(2+3n)$
		\end{parts}
	
\end{ans}
\protect \end {itemize}
 \protect \noindent {\protect \bf  Solutions for Section 2.3} \protect \heading {}{}{Section 2.3\ Solutions} \protect \begin {itemize} 
\begin{ans}{2.3.1.}
		\begin{parts}
		\part Hint: third differences are constant, so $a_n = an^3 + bn^2 + cn + d$.  Use the terms of the sequence to solve for $a, b, c,$ and $d$.
		\part $a_n = n^2 - n$
		\end{parts}
	
\end{ans}
\begin{ans}{2.3.2.}
		No.  The sequence of differences is the same as the original sequence so no differences will be constant.
	
\end{ans}
\protect \end {itemize}
 \protect \noindent {\protect \bf  Solutions for Section 2.4} \protect \heading {}{}{Section 2.4\ Solutions} \protect \begin {itemize} 
\begin{ans}{2.4.1.}
		171 and 341.  $a_n = a_{n-1} + 2a_{n-2}$ with $a_0 = 3$ and $a_1 = 5$.  Closed formula: $a_n = \frac{8}{3}2^n + \frac{1}{3}(-1)^n$
	
\end{ans}
\begin{ans}{2.4.2.}
		By telescoping or iteration.  $a_n = 3 + 2^{n+1}$
	
\end{ans}
\begin{ans}{2.4.3.}
		We claim $a_n = 4^n$ works.  Plug it in: $4^n = 3(4^{n-1}) + 4(4^{n-2})$.  This works - just simplify the right hand side.
	
\end{ans}
\begin{ans}{2.4.4.}
		By the Characteristic Root Technique.  $a_n = 4^n + (-1)^n$.
	
\end{ans}
\begin{ans}{2.4.5.}
		$a_n = \frac{13}{5} 4^n + \frac{12}{5} (-1)^n$
	
\end{ans}
\begin{ans}{2.4.6.}
		The general solution is $a_n = a + bn$ where $a$ and $b$ depend on the initial conditions.  %Solve the recurrence relation $a_n = 2a_{n-1} - a_{n-2}$.
		\begin{parts}
		  \part $a_n = 1 + n$
		  %What is the solution if the initial terms are $a_0 = 1$ and $a_1 = 2$?
		  \part For example, we could have $a_0 = 21$ and $a_1 = 22$.  %What do the initial terms need to be in order for $a_9 = 30$?
		  \part For every $x$ - take $a_0 = x-9$ and $a_1 = x-8$.  %For which $x$ are there initial terms which make $a_9 = x$?
		\end{parts}
	
\end{ans}
\begin{ans}{2.4.7.}
		$a_n = \frac{19}{7}(-2)^n + \frac{9}{7}5^n$
		%Solve the recurrence relation $a_n = 3a_{n-1} + 10a_{n-2}$ with initial terms $a_0 = 4$ and $a_1 = 1$.
	
\end{ans}
\protect \end {itemize}
 \protect \noindent {\protect \bf  Solutions for Section 2.5} \protect \heading {}{}{Section 2.5\ Solutions} \protect \begin {itemize} 
\begin{ans}{2.5.1.}
		\begin{proof}
		 We must prove that $1 + 2 + 2^2 + 2^3 + \cdots +2^n = 2^{n+1} - 1$ for all $n \in \N$.  Thus let $P(n)$ be the statement $1 + 2 + 2^2 + \cdots + 2^n = 2^{n+1} - 1$.  We will prove that $P(n)$ is true for all $n \in \N$.

		 First we establish the base case, $P(0)$, which claims that $1 = 2^{0+1} -1$.  Since $2^1 - 1 = 2 - 1 = 1$, we see that $P(0)$ is true.

		 Now for the inductive case.  Assume that $P(k)$ is true for an arbitrary $k \in \N$.  That is, $1 + 2 + 2^2 + \cdots + 2^k = 2^{k+1} - 1$.  We must show that $P(k+1)$ is true (i.e., that $1 + 2 + 2^2 + \cdots + 2^{k+1} = 2^{k+2} - 1$).  To do this, we start with the left hand side of $P(k+1)$ and work to the right hand side:
		 \begin{align*}
		  1 + 2 + 2^2 + \cdots + 2^k + 2^{k+1} = &~ 2^{k+1} - 1 + 2^{k+1} & \mbox{ \footnotesize by the inductive hypothesis}\\
		   = & ~2\cdot 2^{k+1} - 1 & \\
		   = &~ 2^{k+2} - 1 &
		 \end{align*}
		Thus $P(k+1)$ is true so by the principle of mathematical induction, $P(n)$ is true for all $n \in \N$.
		\end{proof}
	
\end{ans}
\begin{ans}{2.5.2.}
		\begin{proof}
		 Let $P(n)$ be the statement ``$7^n - 1$ is a multiple of 6.''  We will show $P(n)$ is true for all $n \in \N$.

		 First we establish the base case, $P(0)$.  Since $7^0 - 1 = 0$, and $0$ is a multiple of 6, $P(0)$ is true.

		 Now for the inductive case.  Assume $P(k)$ holds for an arbitrary $k \in \N$.  That is, $7^k - 1$ is a multiple of 6, or in other words, $7^k - 1 = 6j$ for some integer $j$.  Now consider $7^{k+1} - 1$:
		 \begin{align*}
		  7^{k+1} - 1 ~ & = 7^{k+1} - 7 + 6 & \mbox{ \footnotesize by cleverness: $-1 = -7 + 6$}\\
		  & = 7(7^k - 1) + 6 & \mbox{ \footnotesize factor out a 7 from the first two terms}\\
		  & = 7(6j) + 6 & \mbox{ \footnotesize by the inductive hypothesis}\\
		  & = 6(7j + 1) & \mbox{ \footnotesize factor out a 6}
		 \end{align*}
		Therefore $7^{k+1} - 1$ is a multiple of 6, or in other words, $P(k+1)$ is true.  Therefore by the principle of mathematical induction, $P(n)$ is true for all $n \in \N$.
		\end{proof}
	
\end{ans}
\begin{ans}{2.5.3.}
		\begin{proof}
		 Let $P(n)$ be the statement $1+3 +5 + \cdots + (2n-1) = n^2$.  We will prove that $P(n)$ is true for all $n \ge 1$.

		 First the base case, $P(1)$.  We have $ 1 = 1^2$ which is true, so $P(1)$ is established.

		 Now the inductive case.  Assume that $P(k)$ is true for some fixed arbitrary $k \ge 1$.  That is, $1 + 3 + 5 + \cdots + (2k-1) = k^2$.  We will now prove that $P(k+1)$ is also true (i.e., that $1 + 3 + 5 + \cdots + (2k+1) = (k+1)^2$).  We start with the left hand side of $P(k+1)$ and work to the right hand side:
		 \begin{align*}
		  1 + 3 + 5 + \cdots + (2k-1) + (2k+1) ~ & = k^2 + (2k+1) & \mbox{ \footnotesize by the induction hypothesis}\\
		  & = (k+1)^2 & \mbox{ \footnotesize by factoring}
		 \end{align*}
		Thus $P(k+1)$ holds, so by the principle of mathematical induction, $P(n)$ is true for all $n \ge 1$.
		\end{proof}
	
\end{ans}
\begin{ans}{2.5.4.}
		\begin{proof}
		 Let $P(n)$ be the statement $F_0 + F_2 + F_4 + \cdots + F_{2n} = F_{2n+1} - 1$.  We will show that $P(n)$ is true for all $n \ge 0$.  First the base case is easy because $F_0 = 0$ and $F_1 = 1$ so $F_0 = F_1 - 1$.  Now consider the inductive case.  Assume $P(k)$ is true, that is, assume $F_0 + F_2 + F_4 + \cdots + F_{2k} = F_{2k+1} - 1$.  To establish $P(k+1)$ we work from left to right:
		 \begin{align*}
		  F_0 + F_2 + F_4 + \cdots + F_{2k} + F_{2k+2} ~ & = F_{2k+1} - 1 + F_{2k+2} & \mbox{\footnotesize by the inductive hypothesis}\\
		  & = F_{2k+1} + F_{2k+2} - 1 & \\
		  & = F_{2k+3} - 1 & \mbox{\footnotesize by the recursive definition of the Fibonacci numbers}
		 \end{align*}
		Therefore $F_0 + F_2 + F_4 + \cdots + F_{2k+2} = F_{2k+3} - 1$, which is to say $P(k+1)$ holds.  Therefore by the principle of mathematical induction, $P(n)$ is true for all $n \ge 0$.
		\end{proof}
	
\end{ans}
\begin{ans}{2.5.5.}
		\begin{proof}
		 Let $P(n)$ be the statement $2^n < n!$.  We will show $P(n)$ is true for all $n \ge 4$.  First, we check the base case and see that yes, $2^4 < 4!$ (as $16 < 24$) so $P(4)$ is true.  Now for the inductive case.  Assume $P(k)$ is true for an arbitrary $k \ge 4$.  That is, $2^k < k!$.  Now consider $P(k+1)$: $2^{k+1} < (k+1)!$.  To prove this, we start with the left side and work to the right side.
		 \begin{align*}
		  2^{k+1}~ & = 2\cdot 2^k & \\
		  & < 2\cdot k! & \mbox{ \footnotesize by the inductive hypothesis}\\
		  & < (k+1) \cdot k! & \mbox{ \footnotesize since $k+1 > 2$}\\
		  & = (k+1)! &
		 \end{align*}
		Therefore $2^{k+1} < (k+1)!$ so we have established $P(k+1)$.  Thus by the principle of mathematical induction $P(n)$ is true for all $n \ge 4$.
		\end{proof}
	
\end{ans}
\begin{ans}{2.5.6.}
  		The only problem is that we never established the base case.  Of course, when $n = 0$, $0+3 \ne 0+7$.
  	
\end{ans}
\begin{ans}{2.5.7.}
		\begin{proof}
		    Let $P(n)$ be the statement that $n + 3 < n + 7$.  We will prove that $P(n)$ is true for all $n \in \N$.  First, note that the base case holds: $0+3 < 0+7$.  Now assume for induction that $P(k)$ is true.  That is, $k+3 < k+7$.  We must show that $P(k+1)$ is true.  Now since $k + 3 < k + 7$, add 1 to both sides.  This gives $k + 3 + 1 < k + 7 + 1$.  Regrouping $(k+1) + 3 < (k+1) + 7$.  But this is simply $P(k+1)$.  Thus by the principle of mathematical induction $P(n)$ is true for all $n \in \N$.
		\end{proof}
	
\end{ans}
\begin{ans}{2.5.8.}
 		The problem here is that while $P(0)$ is true, and while $P(k) \imp P(k+1)$ for {\em some} values of $k$, there is at least one value of $k$ (namely $k = 99$) when that implication fails.  For a valid proof by induction, $P(k) \imp P(k+1)$ must be true for all values of $k$ greater than or equal to the base case.
 	
\end{ans}
\begin{ans}{2.5.9.}
		\begin{proof}
		 Let $P(n)$ be the statement ``there is a strictly increasing sequence $a_1, a_2, a_3, \ldots, a_n$ with $a_n < 100$.''  We will prove $P(n)$ is true for all $n \ge 1$. First we establish the base case: $P(1)$ says there is a single number $a_1$ with $a_1 < 100$.  This is true - take $a_1 = 0$.  Now for the inductive step, assume $P(k)$ is true.  That is there exists a strictly increasing sequence $a_1, a_2, a_3, \ldots, a_k$ with $a_k < 100$.  Now consider this sequence, plus one more term, $a_{k+1}$ which is greater than $a_k$ but less than $100$.  Such a number exists - for example, the average between $a_k$ and 100.  So then $P(k+1)$ is true, so we have shown that $P(k) \imp P(k+1)$.  Thus by the principle of mathematical induction, $P(n)$ is true for all $n \in \N$.
		\end{proof}

	
\end{ans}
\begin{ans}{2.5.10.}
  		We once again failed to establish the base case: when $n = 0$, $n^2 + n = 0$ which is even, not odd.
  	
\end{ans}
\begin{ans}{2.5.11.}
		  \begin{proof}
		    Let $P(n)$ be the statement ``$n^2 + n$ is even.''  We will prove that $P(n)$ is true for all $n \in \N$.  First the base case: when $n = 0$, we have $n^2 + n = 0$ which is even, so $P(0)$ is true.  Now suppose for induction that $P(k)$ is true, that is, that $k^2 + k$ is even.  Now consider the statement $P(k+1)$.  Now $(k+1)^2 + (k+1) = k^2 + 2k + 1 + k + 1 = k^2 + k + 2k + 2$.  By the inductive hypothesis, $k^2 + k$ is even, and of course $2k + 2$ is even.  An even plus an even is always even, so therefore $(k+1)^2 + (k+1)$ is even.  Therefore by the principle of mathematical induction, $P(n)$ is true for all $n \in \N$.
		  \end{proof}
	
\end{ans}
\begin{ans}{2.5.12.}
		 Further hint: the idea is to define the sequence so that $a_n$ is less than the distance between the previous partial sum and 2.  That way when you add it into the next partial sum, the partial sum is still less than 2.  You could do this ahead of time, or use a clever $P(n)$ in the induction proof.  Let $P(n)$ be the statement, ``there is a sequence of positive real numbers $a_1, a_2, a_3, \ldots, a_n$ such that $a_1 + a_2 + a_3 + \cdots + a_n < 2$.''  The base case should be easy (just pick $a_1 < 2$).  For the inductive case, you know that $a_1 + a_2 + \cdots + a_k < 2$ so you just need to argue that you can find some $a_{k+1}$ small enough to have $a_1 + a_2 + \cdots +a_k + a_{k+1} < 2$.
	
\end{ans}
\begin{ans}{2.5.13.}
		The base case should be easy - 0 is a power of 2.  For the inductive case, you actually want to use strong induction.  Suppose $k$ is either a power of 2 or can be written as the sum of distinct powers of 2, for any $k < n$.  Now if $n$ is a power of 2, we are done.  If not, subtract the largest power of 2 from $n$ possible.  You get $n - 2^x$, which is a smaller number, in fact smaller than both $n$ and $2^x$.  Thus $n-2^x$ is either a power of 2 or can be written as the sum of distinct powers of 2, but none of them are going to be $2^x$, so the together with $2^x$ we have written $n$ as the sum of distinct powers of 2.
	
\end{ans}
\begin{ans}{2.5.14.}
	  If $n = 2$, this should work out (so their's your base case).  If we assume it works for $k$ people (that the number of handshakes is $\frac{k(k-1)}{2}$, what happens if a $k+1$st person shows up.  How many {\em new} handshakes take place?  Now make this into a formal induction argument.

	  Note, we have already proven this without using induction, but this is fun too.
	
\end{ans}
\begin{ans}{2.5.15.}
		When $n = 0$, we get $x^0 +\frac{1}{x^0} = 2$ and when $n = 1$, $x + \frac{1}{x}$ is an integer, so the base case holds.  Now assume the result holds for all natural numbers $n < k$.  In particular, we know that $x^{k-1} + \frac{1}{x^{k-1}}$ and $x + \frac{1}{x}$ are both integers.  Thus their product is also an integer.  But,
		\begin{align*}
		\left(x^{k-1} + \frac{1}{x^{k-1}}\right)\left(x + \frac{1}{x}\right) & = x^k + \frac{x^{k-1}}{x} + \frac{x}{x^{k-1}} + \frac{1}{x^k}\\
		& = x^k + \frac{1}{x^k} + x^{k-2} + \frac{1}{x^{k-2}}
		\end{align*}
		Note also that $x^{k-2} + \frac{1}{x^{k-2}}$ is an integer by the induction hypothesis, so we can conclude that $x^k + \frac{1}{x^k}$ is an integer.


	
\end{ans}
\begin{ans}{2.5.16.}
		Here's the idea: since every entry in Pascal's Triangle is the sum of the two entries above it, we can get the $k+1$st row by adding up all the pairs of entry from the $k$th row.  But doing this uses each entry on the $k$th row twice.  Thus each time we drop to the next row, we double the total.  Of course, row 0 has sum $1 = 2^0$ (the base case).  Now try to make this precise with a formal induction proof.  You will use the fact that ${n \choose k} = {n-1 \choose k-1} + {n-1 \choose k}$ for the inductive case.
	
\end{ans}
\begin{ans}{2.5.17.}
		To see why this works, try it on a copy of Pascal's triangle.  We are adding up the entries along a diagonal, starting with the 1 on the left hand side of the 4th row.  Suppose we add up the first 5 entries on this diagonal.  The claim is that the sum is the entry below and to the left of the last of these 5 entries.  Note that if this is true, and we instead add up the first 6 entries, we will need to add the entry one spot to the right of the previous sum.  But these two together give the entry below them, which is below and left of the last of the 6 entries on the diagonal.

		If you follow that, you can see what is going on.  But it is not a great proof.  A formal induction proof is needed:

		\begin{proof}
			Let $P(n)$ be the statement ${4 \choose 0} + {5 \choose 1} + {6 \choose 2} + \cdots + {4+n \choose n} = {5+n \choose n}$.  For the base case, consider $n = 0$.  This says ${4 \choose 0} = {5 \choose 0}$.  Since these are both 1, the base case is true.  Now for the inductive case, suppose $P(k)$ is true.  That is, ${4 \choose 0} + {5 \choose 1} + {6 \choose 2} + \cdots + {4+k \choose k} = {5+k \choose k}$.  If we add ${4+k+1 \choose k+1}$ to both sides, we get \[{4 \choose 0} + {5 \choose 1} + {6 \choose 2} + \cdots + {4+k \choose k} + {5+k \choose k+1}= {5+k \choose k} + {5+k \choose k+1}\]
			But ${5+k \choose k} + {5+k \choose k+1} = {5+k+1 \choose k+1}$.  In other words, we have
			\[{4 \choose 0} + {5 \choose 1} + {6 \choose 2} + \cdots + {4+k \choose k} + {5+k \choose k+1} = {5+k+1 \choose k+1}\]
			which is to say that $P(k+1)$ is true.

			Therefore, by the principle of mathematical induction, $P(n)$ is true for all $n \ge 0$.
		\end{proof}
	
\end{ans}
\begin{ans}{2.5.18.}
		The idea here is that if we take the logarithm of $a^n$, we can increase $n$ by 1 if we multiply by another $a$ (inside the logarithm).  This results in adding 1 more $\log(a)$ to the total.

		\begin{proof}
			Let $P(n)$ be the statement $\log(a^n) = n \log(a)$.  The base case, $P(2)$ is true, because $\log(a^2) = \log(a\cdot a) = \log(a) + \log(a) = 2\log(a)$, by the product rule for logarithms.

			Now assume, for induction, that $P(k)$ is true.  That is, $\log(a^k) = k\log(a)$.  Consider $\log(a^{k+1})$.  We have
			\[\log(a^{k+1}) = \log(a^k\cdot a) = \log(a^k) + \log(a) = k\log(a) + \log(a)\]
			with the last equality due to the inductive hypothesis.  But this simplifies to $(k+1) \log(a)$, establishing $P(k+1)$.

			Therefore by the principle of mathematical induction, $P(n)$ is true for all $n \ge 2$.
		\end{proof}
	
\end{ans}
\begin{ans}{2.5.19.}
		Hint: You are allowed to assume the base case.  For the inductive case, group all but the last function together as one sum of functions, then apply the usual sum of derivatives rule, and then the inductive hypothesis.
	
\end{ans}
\begin{ans}{2.5.20.}
		Hint: for the inductive step, we know by the product rule for two functions that \[(f_1f_2f_3 \cdots f_k f_{k+1})' = (f_1f_2f_3\cdots f_k)'f_{k+1} + (f_1f_2f_3\cdots f_k)f_{k+1}'\]
		Then use the inductive hypothesis on the first summand, and distribute.
	
\end{ans}
\protect \end {itemize}
 \protect \noindent {\protect \bf  Solutions for Section 3.1} \protect \heading {}{}{Section 3.1\ Solutions} \protect \begin {itemize} 
\begin{ans}{3.1.1.}
    \begin{parts}
      \part $P$: it's your birthday; $Q$: there will be cake.  $(P \vee Q) \imp Q$
      \part Hint: you should get three T's and one F.
      \part Only that there will be cake.
      \part It's NOT your birthday!
      \part It's your birthday, but the cake is a lie.
    \end{parts}
  
\end{ans}
\begin{ans}{3.1.2.}
    \begin{parts}
      \part $P \wedge Q$
      \part $P \imp \neg Q$
      \part Jack passed math or Jill passed math (or both).
      \part If Jack and Jill did not both pass math, then Jill did.
      \part
	\begin{subparts}
	  \subpart Nothing else.
	  \subpart  Jack did not pass math either.
	\end{subparts}
    \end{parts}
  
\end{ans}
\begin{ans}{3.1.3.}
    \begin{parts}
	\part Three statements: $P \vee S$, $S \imp Q$, $(P \vee Q) \imp R$.  You could also connect the first two with a $\wedge$.
	\part He cannot be lying about all three sentences, so he is telling the truth.
	\part No matter what, Geoff wants ricotta.  If he doesn't have quail, then he must have pepperoni but not sausage.
    \end{parts}
  
\end{ans}
\begin{ans}{3.1.4.}
 \begin{tabular}{c|c|c}
             $P$ & $Q$ & $(P \vee Q) \imp (P \wedge Q)$\\ \hline
             T & T & T \\
             T & F & F \\
             F & T & F \\
             F & F & T
          \end{tabular}
\end{ans}
\begin{ans}{3.1.5.}
      \begin{tabular}{c|c|c}
             $P$ & $Q$ & $\neg P \wedge (Q \imp P)$\\ \hline
             T & T & F \\
             T & F & F \\
             F & T & F \\
             F & F & T
          \end{tabular}
	If the statement is true, then both $P$ and $Q$ are false.
    
\end{ans}
\begin{ans}{3.1.6.}
    Hint: Like above, only now you will need 8 rows instead of just 4.
  
\end{ans}
\begin{ans}{3.1.7.}
    The argument is valid.  To see this, make a truth table which contains $P \vee Q$ and $\neg P$ (and $P$ and $Q$ of course).  Look at the truth value of $Q$ in each of the rows that have $P \vee Q$ and $\neg P$ true.
  
\end{ans}
\begin{ans}{3.1.8.}
    The argument form is valid.  Again, make a truth table containing the premises and conclusion - look at the rows for which the premises are true.
  
\end{ans}
\begin{ans}{3.1.9.}
    The argument is NOT valid.  If you make a truth table containing the premises and conclusion, there will be a row with both premises true but the conclusion false.  For example, if $P$ and $Q$ are false and $R$ is true, then $P \wedge Q$ is false, so $(P \wedge Q) \imp R$ is true.  Also $\neg P$ is true, so $\neg P \vee \neg Q$ is true.  However, $\neg R$ is false.
  
\end{ans}
\protect \end {itemize}
 \protect \noindent {\protect \bf  Solutions for Section 3.2} \protect \heading {}{}{Section 3.2\ Solutions} \protect \begin {itemize} 
\begin{ans}{3.2.1.}
    Make a truth table for each and compare.  The statements are logically equivalent.
  
\end{ans}
\begin{ans}{3.2.2.}
    Again, make two truth tables.  The statements are logically equivalent.
  
\end{ans}
\begin{ans}{3.2.3.}
    \begin{parts}
      \part If Oscar drinks milk, then he eats Chinese food.
      \part If Oscar does not drink milk, then he does not eat Chinese food.
      \part Yes.  The original statement would be false too.
      \part Nothing. The converse need not be true.
      \part He does not eat Chinese food. The contrapositive would be true.
    \end{parts}
  
\end{ans}
\begin{ans}{3.2.4.}
    \begin{parts}
      \part $P \wedge Q$
      \part $(P \vee Q) \vee (Q \wedge \neg R)$
      \part F.  Or $(P \wedge Q) \wedge (R \wedge \neg R)$
      \part Either Sam is a woman and Chris is a man, or Chris is a woman.
    \end{parts}
  
\end{ans}
\begin{ans}{3.2.5.}
 The statements are equivalent to the\ldots
    \begin{parts}
      \part converse.
      \part implication.
      \part neither.
      \part implication.
      \part converse.
      \part converse.

      \part implication.
      \part converse.
      \part converse.
      \part converse (in fact, this IS the converse).
      \part implication (the statement is the contrapositive of the implication).
      \part neither.
    \end{parts}
  
\end{ans}
\begin{ans}{3.2.6.}
    Hint: of course there are many answers.  It helps to assume that the statement is true and the converse is NOT true.  Think about what that means in the real world and then start saying it in different ways.  Some ideas: use necessary and sufficient language, use ``only if,'' consider negations, use ``or else'' language.
  
\end{ans}
\protect \end {itemize}
 \protect \noindent {\protect \bf  Solutions for Section 3.3} \protect \heading {}{}{Section 3.3\ Solutions} \protect \begin {itemize} 
\begin{ans}{3.3.1.}
     \begin{parts}
	\part $\neg \exists x (E(x) \wedge O(x))$
	\part $\forall x (E(x) \imp O(x+1))$
	\part $\exists x(P(x) \wedge E(x))$ (where $P(x)$ means ``$x$ is prime'')
	\part $\forall x \forall y \exists z(x < z < y \vee y < z < x)$
	\part $\forall x \neg \exists y (x < y < x+1)$
    \end{parts}
  
\end{ans}
\begin{ans}{3.3.2.}
    \begin{parts}
	\part Any even number plus 2 is an even number.
	\part For any $x$ there is a $y$ such that $\sin(x) = y$.  In other words, every number $x$ is in the domain of sine.
	\part For every $y$ there is an $x$ such that $\sin(x) = y$.  In other words, every number $y$ is in the range of sine (which is false).
	\part For any numbers, if the cubes of two numbers are equal, then the numbers are equal.
      \end{parts}
  
\end{ans}
\begin{ans}{3.3.3.}
    \begin{parts}
	\part $\forall x \exists y (O(x) \wedge \neg E(y))$
	\part $\exists x \forall y (x \ge y \vee \forall z (x \ge z \wedge y \ge z))$
	\part There is a number $n$ for which every other number is strictly greater than $n$.
	\part There is a number $n$ which is not between any other two numbers.
      \end{parts}
  
\end{ans}
\protect \end {itemize}
 \protect \noindent {\protect \bf  Solutions for Section 3.4} \protect \heading {}{}{Section 3.4\ Solutions} \protect \begin {itemize} 
\begin{ans}{3.4.1.}
     \begin{parts}
 	\part For all integers $a$ and $b$, if $a$ or $b$ are not even, then $a+b$ is not even.
 	\part For all integers $a$ and $b$, if $a$ and $b$ are even, then $a+b$ is even.
 	\part There are numbers $a$ and $b$ such that $a+b$ is even but $a$ and $b$ are not both even.
 	\part False.  For example, $a = 3$ and $b = 5$.  $a+b = 8$, but neither $a$ nor $b$ are even.
 	\part False, since it is equivalent to the original statement.
 	\part True.  Let $a$ and $b$ be integers.  Assume both are even.  Then $a = 2k$ and $b = 2j$ for some integers $k$ and $j$.  But then $a+b = 2k + 2j = 2(k+j)$ which is even.
 	\part True, since the statement is false.
       \end{parts}
   
\end{ans}
\begin{ans}{3.4.2.}
     \begin{proof}
      Suppose $\sqrt{3}$ were rational.  Then $\sqrt{3} = \frac{a}{b}$ for some integers $a$ and $b \ne 0$.  Without loss of generality, assume $\frac{a}{b}$ is reduced.  Now
 \[3 = \frac{a^2}{b^2}\]
 \[b^2 3 = a^2\]
 So $a^2$ is a multiple of 3.  This can only happen if $a$ is a multiple of 3, so $a = 3k$ for some integer $k$.  Then we have
 \[b^2 3 = 9k^2\]
 \[b^2 = 3k^2\]
 So $b^2$ is a multiple of 3, making $b$ a multiple of 3 as well.  But this contradicts our assumption that $\frac{a}{b}$ is in lowest terms.
     \end{proof}
   
\end{ans}
\protect \end {itemize}
 \protect \noindent {\protect \bf  Solutions for Section 4.1} \protect \heading {}{}{Section 4.1\ Solutions} \protect \begin {itemize} 
\begin{ans}{4.1.1.}
		This is asking for the number of edges in $K_{10}$.  Each vertex (person) has degree (shook hands with) 9 (people).  So the sum of the degrees is $90$.  However, the degrees count each edge (handshake) twice, so there are 45 edges in the graph.  That is how many handshakes took place.%If 10 people each shake hands with each other, how many handshakes took place?  What does this question have to do with graph theory?
	
\end{ans}
\begin{ans}{4.1.2.}
		It is possible for everyone to be friends with exactly 2 people - you could arrange the 5 people in a circle and say that everyone is friends with the two people on either side of them (so you get the graph $C_5$).  However, it is not possible for everyone to be friends with 3 people - that would lead to a graph with an odd number of odd degree vertices which is impossible - the sum of the degrees must be even.  %Among a group of 5 people, is it possible for everyone to be friends with exactly 2 of the people in the group?  What about 3 of the people in the group?
	
\end{ans}
\begin{ans}{4.1.3.}
		Yes.  For example, both graphs below contain 6 vertices, 7 edges, and have degrees (2,2,2,2,3,3).
		\begin{center}
		  \hfill
		  \begin{tikzpicture}
		   \draw[thick] (-2,0) \v -- (-1,0) \v -- (-1.5,1) \v -- (-2,0) (-1.5,1) -- (1.5, 1) \v -- (1,0) \v -- (2,0) \v -- (1.5,1);
		  \end{tikzpicture}
		  \hfill
		  \begin{tikzpicture}
		  \foreach \x in {0,...,5}
		    \draw[thick] (\x*60:1) \v -- (\x*60 + 60:1);
		    \draw[thick] (0:1) -- (180:1);
		  \end{tikzpicture}
		  \hfill ~
		\end{center}
	
\end{ans}
\begin{ans}{4.1.4.}
		Three of the graphs are bipartite.  The one which is not is $C_7$ (second from the right).
	
\end{ans}
\begin{ans}{4.1.5.}
		$C_n$ is bipartite if and only if $n = 1$ or is even.
	
\end{ans}
\protect \end {itemize}
 \protect \noindent {\protect \bf  Solutions for Section 4.2} \protect \heading {}{}{Section 4.2\ Solutions} \protect \begin {itemize} 
\begin{ans}{4.2.1.}
		No.  A (connected) planar graph must satisfy Euler's formula: $V - E + F = 2$.  Here $V - E + F = 6 - 10 + 5 = 1$. %Is it possible for a planar graph to have 6 vertices, 10 edges and 5 faces?  Explain.
	
\end{ans}
\begin{ans}{4.2.2.}
		$G$ has 10 edges.  It could be planar, and then it would have 6 faces. %The graph $G$ has 6 vertices with degrees $2, 2, 3, 4, 4, 5$.  How many edges does $G$ have?  Could $G$ be planar?  If so, how many faces would it have.
	
\end{ans}
\begin{ans}{4.2.3.}
		Yes.  According to Euler's formula it would have 2 faces.  It does.  The only such graph is $C_{10}$. %If a graph has 10 vertices and 10 edges and contains an Euler circuit, must it be planar?  How many faces would it have?
	
\end{ans}
\protect \end {itemize}
 \protect \noindent {\protect \bf  Solutions for Section 4.3} \protect \heading {}{}{Section 4.3\ Solutions} \protect \begin {itemize} 
\begin{ans}{4.3.1.}
		2, since the graph is bipartite.  One color for the top set of vertices, another color for the bottom set of vertices.  %What is the smallest number of colors you need to properly color the vertices of $K_{4,5}$.  That is, find the chromatic number of the graph.
	
\end{ans}
\begin{ans}{4.3.2.}
		For example, $K_6$.  If the chromatic number is 6, then the graph is not planar - the 4-color theorem states that all planar graphs can be colored with 4 or fewer colors. %Draw a graph with chromatic number 6 (i.e., which requires 6 colors to properly color the vertices).  Could your graph be planar?  Explain.
	
\end{ans}
\begin{ans}{4.3.3.}
		The chromatic numbers are 2, 3, 4, 5, and 3 respectively from left to right. %Find the chromatic number of each of the following graphs.
	
\end{ans}
\protect \end {itemize}
 \protect \noindent {\protect \bf  Solutions for Section 4.4} \protect \heading {}{}{Section 4.4\ Solutions} \protect \begin {itemize} 
\begin{ans}{4.4.1.}
		This is a question about finding Euler paths.  Draw a graph with a vertex in each state, and connect vertices if their states share a border.  Exactly two vertices will have odd degree - the vertices for Nevada and Utah.  Thus you must start your road trip at in one of those states and end it in the other. %You and your friends want to tour the southwest by car.  You will visit the nine states below, with the following rather odd rule: you must cross each border between neighboring states exactly once (so, for example, you must cross the Colorado-Utah border exactly once).  Can you do it?  If so, does it matter where you start your road trip?  What fact about graph theory solves this problem?
	
\end{ans}
\begin{ans}{4.4.2.}
		\begin{parts}
		  \part $K_4$ does not have an Euler path or circuit.
		  \part $K_5$ has an Euler circuit (so also an Euler path).
		  \part $K_{5,7}$ does not have an Euler path or circuit.
		  \part $K_{2,7}$ has an Euler path but not an Euler circuit.
		  \part $C_7$ has an Euler circuit (it is a circuit graph!)
		  \part $P_7$ has an Euler path but no Euler circuit.
		\end{parts}
	
\end{ans}
\begin{ans}{4.4.3.}
		When $n$ is odd, $K_n$ contains an Euler circuit.  This is because every vertex has degree $n-1$, so an odd $n$ results in all degrees being even.%For which $n$ does the graph $K_n$ contain an Euler circuit?  Explain.
	
\end{ans}
\begin{ans}{4.4.4.}
		If both $m$ and $n$ are even, then $K_{m,n}$ has an Euler circuit.  When both are odd, there is no Euler path or circuit.  If one is 2 and the other is odd, then there is an Euler path but not an Euler circuit. %For which $m$ and $n$ does the graph $K_{m,n}$ contain an Euler path?  An Euler circuit?  Explain
	
\end{ans}
\protect \end {itemize}
 \protect \noindent {\protect \bf  Solutions for Section A.1} \protect \heading {}{}{Section A.1\ Solutions} \protect \begin {itemize} 
\begin{ans}{A.1.1.}
		\begin{parts}
		  \part $\dfrac{4}{1-x}$  %$4,4,4,4,4,\ldots$
		  \part $\dfrac{2}{(1-x)^2}$  %$2, 4, 6, 8, 10, \ldots$
		  \part $\dfrac{2x^3}{(1-x}^2$  %$0,0,0,2,4,6,8,10,\ldots$
		  \part $\dfrac{1}{1-5x}$  %$1, 5, 25, 125, \ldots$
		  \part $\dfrac{1}{1+3x}$  %$1, -3, 9, -27, 81, \ldots$
		  \part $\dfrac{1}{1-5x^2}$  %$1, 0, 5, 0, 25, 0, 125, 0, \ldots$
		  \part $\dfrac{x}{(1-x^3)^2}$  %$0, 1, 0, 0, 2, 0, 0, 3, 0, 0, 4, 0, 0, 5, \ldots$
		\end{parts}
	
\end{ans}
\begin{ans}{A.1.2.}
		\begin{parts}
		  \part $0, 4, 4, 4, 4, 4, \ldots$  %$\dfrac{4x}{1-x}$
		  \part $1, 4, 16, 64, 256, \ldots$  %$\dfrac{1}{1-4x}$
		  \part $0, 1, -1, 1, -1, 1, -1, \ldots$  %$\dfrac{x}{1+x}$
		  \part $0, 3, -6, 9, -12, 15, -18, \ldots $  %$\dfrac{3x}{(1+x)^2}$
		  \part $1, 3, 6, 9, 12, 15, \ldots$  %$\dfrac{1+x+x^2}{(1-x)^2}$ (Hint: multiplication)
		\end{parts}
	
\end{ans}
\begin{ans}{A.1.3.}
		\begin{parts}
		  \part The second derivative of $\dfrac{1}{1-x}$ is $\dfrac{2}{(1-x)^3}$ which expands to $2 + 6x + 12x^2 + 20x^3 + 30x^4 + \cdots$.  Dividing by 2 gives the generating function for the triangular numbers. %Take two derivatives of the generating function for $1,1,1,1,1, \ldots$
		  \part Compute $A - xA$ and you get $1 + 2x + 3x^2 + 4x^3 + \cdots$ which can be written as $\dfrac{1}{(1-x)^2}$.  Solving for $A$ gives the correct generating function. %Use differencing.
		  \part The triangular numbers are the sum of the first $n$ numbers $1,2,3,4, \ldots$.  To get the sequence of partial sums, we multiply by $\frac{1}{1-x}$. so this gives the correct generating function again. %Multiply two known generating functions.
		\end{parts}
	
\end{ans}
\begin{ans}{A.1.4.}
		Call the generating function $A$.  Compute $A - xA = 4 + x + 2x^2 + 3x^3 + 4x^4 + \cdots$.  Thus $A - xA = 4 + \dfrac{x}{(1-x)^2}$.  Solving for $A$ gives $\d\frac{4}{1-x} + \frac{x}{(1-x)^3}$.  %Use differencing to find the generating function for $4, 5, 7, 10, 14, 19, 25, \ldots$
	
\end{ans}
\begin{ans}{A.1.5.}
		$\dfrac{1+2x}{1-3x + x^2}$  %Find a generating function for the sequence with recurrence relation $a_n = 3a_{n-1} - a_{n-2}$ with initial terms $a_0 = 1$ and $a_1 = 5$.
	
\end{ans}
\begin{ans}{A.1.6.}
		Compute $A - xA - x^2A$ and the solve for $A$.  The generating function will be $\dfrac{x}{1-x-x^2}$.  %Use the recurrence relation for the Fibonacci numbers to find the generating function for the Fibonacci sequence.
	
\end{ans}
\begin{ans}{A.1.7.}
		$\dfrac{x}{(1-x)(1-x-x^2)}$  %Use multiplication to find the generating function for the sequence of partial sums of Fibonacci numbers, $S_0, S_1, S_2, \ldots$ where $S_0 = F_0$, $S_1 = F_0 + F_1$, $S_2 = F_0 + F_1 + F_2$, $S_3 = F_0 + F_1 + F_2 + F_3$ and so on.
	
\end{ans}
\begin{ans}{A.1.8.}
		$\dfrac{2}{1-5x} + \dfrac{7}{1+3x}$.  %Find the generating function for the sequence with closed formula $a_n = 2(5^n) + 7(-3)^n$.
	
\end{ans}
\begin{ans}{A.1.9.}
		$a_n = 3\cdot 4^{n-1} + 1$  %Find a closed formula for the $n$th term of the sequence with generating function $\dfrac{3x}{1-4x} + \dfrac{1}{1-x}$.
	
\end{ans}
\begin{ans}{A.1.10.}
		Hint: you should ``multiply'' the two sequences.  Answer: 158.  %Find $a_7$ for the sequence with generating function $\dfrac{2}{(1-x)^2}\cdot\dfrac{x}{1-x-x^2}$
	
\end{ans}
\protect \end {itemize}
 \protect \noindent {\protect \bf  Solutions for Section A.2} \protect \heading {}{}{Section A.2\ Solutions} \protect \begin {itemize} 
\begin{ans}{A.2.1.}
		\begin{proof}
			Suppose $a \mid b$.  Then $b$ is a multiple of $a$, or in other words, $b = ak$ for some $k$.  But then $bc = akc$, and since $kc$ is an integer, this says $bc$ is a multiple of $a$.  In other words, $a \mid bc$.
		\end{proof}
	
\end{ans}
\begin{ans}{A.2.2.}
		\begin{proof}
			Assume $a \mid b$ and $a \mid c$.  This means that $b$ and $c$ are both multiples of $a$, so $b = am$ and $c = an$ for integers $m$ and $n$.  Then $b+c = am+an = a(m+n)$, so $b+c$ is a multiple of $a$, or equivalently, $a \mid b+c$.  Similarly, $b-c = am-an = a(m-n)$, so $b-c$ is a multiple of $a$, which is to say $a \mid b-c$.
		\end{proof}
	
\end{ans}
\begin{ans}{A.2.3.}
		$\{\ldots, -8, -4, 0, 4, 8, 12, \ldots\}$, $\{\ldots, -7, -3, 1, 5, 9, 13, \ldots\}$, $\{\ldots, -6, -2, 2, 6, 10, 14, \ldots\}$, and $\{\ldots, -5, -1, 3, 7, 11, 15, \ldots\}$.
	
\end{ans}
\begin{ans}{A.2.4.}
		\begin{proof}
			Assume $a \equiv b \pmod n$ and $c \equiv d \pmod n$.  This means $a = b + kn$ and $c = d + jn$ for some integers $k$ and $j$.  Consider $a-c$.  We have:
			\[a-c = b+kn - (d+jn) = b-d + (k-j)n\]
			In other words, $a-c$ is $b-d$ more than some multiple of $n$, so $a-c \equiv b-d \pmod n$.
		\end{proof}
	
\end{ans}
\begin{ans}{A.2.5.}
		\begin{parts}
			\part $3^{456} \equiv 1^{456} = 1 \pmod 2$.
			\part $3^{456} = 9^{228} \equiv (-1)^{228} = 1 \pmod{5}$
			\part $3^{456} = 9^{228} \equiv 2^{228} = 8^{76} \equiv 1^{76} = 1 \pmod 7$
			\part $3^{456} = 9^{228} \equiv 0^{228} = 0 \pmod{9}$
		\end{parts}
	
\end{ans}
\begin{ans}{A.2.6.}
		For all of these, just plug in all integers between 0 and the modulus to see which, if any, work.
		\begin{parts}
			\part No solutions.
			\part $x = 3$.
			\part $x = 2$, $x = 5$, $x = 8$.
			\part No solutions.
			\part No solutions.
			\part $x = 3$.
		\end{parts}
	
\end{ans}
\begin{ans}{A.2.7.}
		\begin{parts}
			\part $x = 5+22k$ for $k \in \Z$.
			\part $x = 4 + 5k$ for $k \in \Z$.
			\part $x = 6 + 15k$ for $k \in \Z$.
			\part Hint: first reduce each number modulo 9, which can be done by adding up the digits of the numbers.  Answer: $x = 2 + 9k$ for $k \in \Z$.
		\end{parts}
	
\end{ans}
\begin{ans}{A.2.8.}
		We must solve $7x + 5 \equiv 2 \pmod{11}$.  This gives $x \equiv 9 \pmod{11}$.  In general, $x = 9 + 11k$, but when you divide any such $x$ by 11, the remainder will be 9.
	
\end{ans}
\begin{ans}{A.2.9.}
  		\begin{parts}
	  		\part Divide through by 2: $3x + 5y = 16$.  Convert to a congruence, modulo 3: $5y \equiv 16 \pmod 3$, which reduces to $2y \equiv 1 \pmod 3$.  So $y \equiv 2 \pmod 3$ or $y = 2 + 3k$.  Plug this back into $3x + 5y = 16$ and solve for $x$, to get $x = 2-5k$.  So the general solution is $x = 2-5k$ and $y = 2+3k$ for $k \in \Z$.
	  		\part $x = 7+8k$ and $y = -11 - 17k$ for $k \in \Z$.
	  		\part $x = -4-47k$ and $y = 3 + 35k$ for $k \in \Z$.
	  	\end{parts}
  	
\end{ans}
\begin{ans}{A.2.10.}
		First, solve the Diophantine equation $13x + 20 y = 2$.  The general solution is $x = -6 - 20k$ and $y = 4+13k$.  Now if $k = 0$, this correspond to filling the 20 oz. bottle 4 times, and emptying the 13 oz. bottle 6 times, which would require 80 oz. of water.  Increasing $k$ would require considerably more water.  Perhaps $k = -1$ would be better?  Then we would have $x = -6+20 = 14$ and $y = 4-13 = -11$, which describes the solution where we fill the 13 oz. bottle 14 times, and empty the 20 oz. bottle 11 times.  This would require 182 oz. of water.  Thus the most efficient procedure is to repeatedly fill the 20 oz bottle, emptying it into the 13 oz bottle, and discarding full 13 oz. bottles, which requires 80 oz. of water.
	
\end{ans}
\protect \end {itemize}
