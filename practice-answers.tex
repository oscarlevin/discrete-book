 \protect \noindent {\protect \textbf  {Solutions for Section 0.2}} \protect \begin {itemize} 
\begin{ans}{0.2.1.}
    \begin{parts}
	\part $A \cap B = \{3,4,5\}$.  %Find $A \cap B$.
	\part $A \cup B = \{1,2,3,4,5,6,7\}$. %Find $A \cup B$.
	\part $A \setminus B = \{1,2\}$. %Find $A \setminus B$.
	 \part $A \cap \bar{(B \cup C)} = \{1\}$.
	\part $A \times B = \{(1,2), (1,3), (1,5), (2,2), (2,3), (2,5), (3,2), (3,3), (3,5), (4,2), (4,3), (4,5), (5,2), (5,3), (5,5)\}$.
	\part Yes.  %Is $C \subseteq A$?
	\part No. %Is $C \subseteq B$?
    \end{parts}
  
\end{ans}
\begin{ans}{0.2.2.}
    \begin{parts}
  %Find $A \cap B$
	\part $A \cap B = \{4,6,8,10,12\}$
  % Find $A \cup B$.
	\part $A \cup B = \{x \in \N \st (3 \le x \le 13) \vee x \mbox{ is even}\}.$ (the set of all natural numbers which are either even or between 3 and 13 inclusive).
  %Find $B \cap C$.
	\part $B \cap C = \emptyset$.
  %Find $B \cup C$.
	\part $B \cup C = \N$.
      \end{parts}
  
\end{ans}
\begin{ans}{0.2.3.}
    For example, $A = \{2,3,5,7,8\}$ and $B = \{3,5\}$.
  
\end{ans}
\begin{ans}{0.2.4.}
    Let $A = \{1,2,3\}$ and $B = \{1,2,3,4,5,\{1,2,3\}\}$
  
\end{ans}
\begin{ans}{0.2.5.}
    \begin{parts}
	\part No.
	\part No.
	\part $2\Z \cap 3\Z$ is the set of all integers which are multiples of both 2 and 3 (so multiples of 6).  Therefore $2\Z \cap 3\Z = \{x \in \Z \st \exists y\in \Z(x = 6y)\}$.
	\part $2\Z \cup 3\Z$.
 \end{parts}
  
\end{ans}
\begin{ans}{0.2.6.}
    The set of primes.
  
\end{ans}
\begin{ans}{0.2.7.}
%  \begin{multicols}{3}
      \begin{parts}
	  \def\circleA{(-.5,0) circle (1)}
	  \def\circleAlabel{(-1.5,.6) node[above]{$A$}}
	  \def\circleB{(.5,0) circle (1)}
	  \def\circleBlabel{(1.5,.6) node[above]{$B$}}
	  \def\circleC{(0,-1) circle (1)}
	  \def\circleClabel{(.5,-2) node[right]{$C$}}
	  \def\twosetbox{(-2,-1.5) rectangle (2,1.5)}
	  \def\threesetbox{(-2,-2.5) rectangle (2,1.5)}



	    \part  $A \cup \bar B$:

	    \begin{tikzpicture}[fill=gray!50]
	  %Fill A:
	  \fill \circleA;
	  %Fill \bar B:
	    \begin{scope}
	    \clip \circleB \twosetbox; %This defines the scope to everything in the twosetbox which is not in circleB.
	    \fill \twosetbox;
	    \end{scope}
	    \draw[thick] \circleA \circleAlabel \circleB \circleBlabel \twosetbox;
	  \end{tikzpicture}


	  %
	  \part $\bar{(A \cup B)}$:

	  \begin{tikzpicture}[fill=gray!50]
	    \fill \twosetbox;
	    \fill[white] \circleA \circleB;
	    \draw[thick] \circleA \circleAlabel \circleB \circleBlabel \twosetbox;
	  \end{tikzpicture}

%	  \columnbreak

	  %
	  \part $A \cap (B \cup C)$:

	  \begin{tikzpicture}[fill=gray!50]
	  \begin{scope}
	    \clip \circleA;
	    \fill \circleB \circleC;
	  \end{scope}
	  \draw[thick] \circleA \circleAlabel \circleB \circleBlabel \circleC \circleClabel \threesetbox;
	  \end{tikzpicture}

	  %
	  \part $(A \cap B) \cup C$:

	  \begin{tikzpicture}[fill=gray!50]
	  \begin{scope}
	    \clip \circleA;
	    \fill \circleB;
	  \end{scope}
	  \fill \circleC;
	  \draw[thick] \circleA \circleAlabel \circleB \circleBlabel \circleC \circleClabel \threesetbox;
	  \end{tikzpicture}


%	  \columnbreak
	  %
	  \part $\bar A \cap B \cap \bar C$:

	  \begin{tikzpicture}[fill=gray!50]
	  \fill \circleB;
	  \begin{scope}
	    \clip \circleB;
	    \fill[white] \circleA \circleC;
	  \end{scope}

	  \draw[thick] \circleA \circleAlabel \circleB \circleBlabel \circleC \circleClabel \threesetbox;
	  \end{tikzpicture}

	  %
	  \part $(A \cup B) \setminus C$:

	  \begin{tikzpicture}[fill=gray!50]
	  \fill \circleA;
	  \fill \circleB;
	  \fill[white] \circleC;
	  \draw[thick] \circleA \circleAlabel \circleB \circleBlabel \circleC \circleClabel \threesetbox;
	  \end{tikzpicture}

	  \end{parts}
%	   \end{multicols}
  
\end{ans}
\begin{ans}{0.2.8.}
    For example, $A \cup B \cap \bar{(A \cap B)}$.  Note that $\bar{A \cap B}$ would almost work, but also contain the area outside of both circles.
  
\end{ans}
\begin{ans}{0.2.9.}
      \begin{parts}
	  \part 34.
	  \part 103.
	  \part 8.
      \end{parts}
  
\end{ans}
\begin{ans}{0.2.10.}
    $\pow(A) = \{\emptyset, \{a\}, \{b\}, \{c\}, \{a,b\}, \{a,c\}, \{b,c\}, \{a,b,c\}\}$.
  
\end{ans}
\begin{ans}{0.2.11.}
      There are 10 singletons.  There are 45 doubletons (because $45 = 9+8+7+\cdots+2+1$).
  
\end{ans}
\begin{ans}{0.2.12.}
      $\{2,3,5\}, \{1,2,3,5\}, \{2,3,4,5\}, \{2,3,5,6\}, \{1,2,3,4,5\}, \{1,2,3,5,6\}, \{2,3,4,5,6\}$, and $\{1,2,3,4,5,6\}$.
  
\end{ans}
\begin{ans}{0.2.13.}
   For example $A = \{1,2,3,4\}$ and $B = \{5,6,7,8,9\}$.
  
\end{ans}
\begin{ans}{0.2.14.}
    For example, $A = \{1,2,3\}$ and $B = \{2,3,4,5\}$.
  
\end{ans}
\begin{ans}{0.2.15.}
	 No.  There must be 5 elements in common to both sets.  Since there are 10 distinct elements all together in $A$ and $B$, this means that between $A$ and $B$, there must be 5 elements which they do not have in common (some in $A$ but not in $B$, some in $B$ but not in $A$).  But 5 is odd, so to have $|A| = |B|$, we would need 7.5 elements in each set, which is impossible.
	
\end{ans}
\begin{ans}{0.2.16.}
      If $R$ is the set of red cards and $F$ is the set of face cards, we want to find $|R \cup F|$.  This is not simply $|R| + |F|$ because there are 6 cards which are both red and a face card; $|R \cap F| = 6$.  We find $|R \cup F| = 32$.
  
\end{ans}
\protect \end {itemize}
 \protect \noindent {\protect \textbf  {Solutions for Section 0.3}} \protect \begin {itemize} 
\begin{ans}{0.3.1.}
	There are 8 different functions.  For example, $f(1) = a$, $f(2) = a$, $f(3) = a$; or $f(1) = a$, $f(2) = b$, $f(3) = a$, and so on.  None of the functions are injective.  Exactly 6 of the functions are surjective.  No functions are both (since no functions here are injective).
	
\end{ans}
\begin{ans}{0.3.2.}
	There are nine functions - you have a choice of three outputs for $f(1)$, and for each, you have three choices for the output $f(2)$.  Of these functions, 6 are injective, 0 are surjective, and 0 are both.
	
\end{ans}
\begin{ans}{0.3.3.}
		\begin{parts}
		%Is $f$ injective?  Explain.
		\part $f$ is not injective, since $f(2) = f(5)$ - two different inputs have the same output.
		% Is $f$ surjective?  Explain.
		\part $f$ is surjective, since every element of the codomain is an element of the range.
		\end{parts}
	
\end{ans}
\begin{ans}{0.3.4.}
		\begin{parts}
		%Is $f$ injective?  Explain.
		  \part $f$ is not injective, since $f(1) = 3$ and $f(4) = 3$.
		%   Is $f$ surjective?  Explain.
		  \part $f$ is not surjective, since there is no input which gives 2 as an output.
		\end{parts}
	
\end{ans}
\begin{ans}{0.3.5.}
		\begin{parts}
		% $f:\N \to \N$ given by $f(n) = n+4$.
		  \part $f$ is injective, but not surjective.
		%   $f:\Z \to \Z$ given by $f(n) = n+4$.
		  \part $f$ is injective and surjective.
		  %$f:\Z \to \Z$ given by $f(n) = 5n - 8$.
		  \part $f$ is injective, but not surjective.
		%   $f:\Z \to \Z$ given by $f(n) = \begin{cases}
		%                                          n/2 & \mbox{ if $n$ is even}\\
		%                                          (n+1)/2 & \mbox{ if $n$ is odd}.
		%                                        \end{cases}$
		 \part $f$ is not injective, but is surjective.
		\end{parts}
	
\end{ans}
\begin{ans}{0.3.6.}
		\begin{parts}
		% Is $f$ injective?  Prove your answer.
		  \part $f$ is not injective.  To prove this, we must simply find two different elements of the domain which map to the same element of the codomain.  Since $f(\{1\}) = 1$ and $f(\{2\}) = 1$, we see that $f$ is not injective.
		%   Is $f$ surjective?  Prove your answer.
		  \part $f$ is not surjective.  The largest subset of $A$ is $A$ itself, and $|A| = 10$.  So no natural number greater than 10 will ever be an output.
		%   Find $f\inv(1)$.
		  \part $f\inv(1) = \{\{1\}, \{2\}, \{3\}, \ldots \{10\}\}$ (the set of all the singleton subsets of $A$).
		%   Find $f\inv(0)$.
		  \part $f\inv(0) = \{\emptyset\}$.  Note, it would be wrong to write $f\inv(0) = \emptyset$ - that would claim that there is no input which has 0 as an output.
		   % Find $f\inv(12)$.
		  \part $f\inv(12) = \emptyset$, since there are no subsets of $A$ with cardinality 12.
		\end{parts}
	
\end{ans}
\begin{ans}{0.3.7.}
		\begin{parts}
		% Find $f\inv(3)$.
		  \part $f\inv(3) = \{003, 030, 300, 012, 021, 102, 201, 120, 210, 111\}$
		%   Find $f\inv(28)$.
		  \part $f\inv(28) = \emptyset$ (since the largest sum of three digits is $9+9+9 = 27$)
		%   Use one of the parts above to prove that $f$ is not injective.
		  \part Part (a) proves that $f$ is not injective - the output 3 is assigned to 10 different inputs.
		%   Use one of the parts above to prove that $f$ is not surjective.
		  \part Part (b) proves that $f$ is not surjective - there is an element of the codomain (28) which is assigned to no inputs.
		\end{parts}
	
\end{ans}
\begin{ans}{0.3.8.}
		\begin{parts}
			\part $|f\inv(3)| \le 1$.  In other words, either $f\inv(3)$ is the emptyset or is a set containing exactly one element.  Injective functions cannot have two elements from the domain both map to 3. %$f$ is injective? Explain.
			\part $|f\inv(3)| \ge 1$.  In other words, $f\inv(3)$ is a set containing at least one elements, possibly more.  Surjective functions cannot have nothing mapping to 3.%$f$ is surjective? Explain.
			\part $|f\inv(3)| = 1$.  There is exactly one element from $X$ which gets mapped to 3, so $f\inv(3)$ is the set containing that one element. %$f$ is bijective? Explain.
		\end{parts}
	
\end{ans}
\begin{ans}{0.3.9.}
		$X$ can really be any set, as long as $f(x) = 0$ or $f(x) = 1$ for every $x \in X$.  For example, $X = \N$ and $f(n) = 0$ works.
	
\end{ans}
\begin{ans}{0.3.10.}
		\begin{parts}
		% there is a injective function $f:X \to Y$.  Explain.
		  \part $|X| \le |Y|$ - otherwise two or more of the elements of $X$ would need to map to the same element of $Y$.
		%   there is a surjective function $f:X \to Y$.  Explain.
		  \part $|X| \ge |Y|$ - otherwise there would be one or more elements of $Y$ which were never an output.
		%   there is a bijection $f:X \to Y$.  Explain.
		  \part $|X| = |Y|$.  This is the only way for both of the above to occur.
		\end{parts}
	
\end{ans}
\begin{ans}{0.3.11.}
		\begin{parts}
		% $f$ is injective but not surjective.
		  \part Yes. (Can you give an example?)
		  \part Yes. %$f$ is surjective but not injective.
		  \part Yes. %$|X| = |Y|$ and $f$ is injective but not surjective.
		  \part Yes. %$|X| = |Y|$ and $f$ is surjective but not injective.
		  \part No. %$|X| = |Y|$, $X$ and $Y$ are finite, and $f$ is injective but not surjective.
		  \part No. %$|X| = |Y|$, $X$ and $Y$ are finite, and $f$ is surjective but not injective.
		\end{parts}
	
\end{ans}
\begin{ans}{0.3.12.}
		\begin{parts}
		% Is $f$ injective?  Prove your answer.
		  \part $f$ is injective.
		  \begin{proof}
		   Let $x$ and $y$ be elements of the domain $\Z$.  Assume $f(x) = f(y)$.  If $x$ and $y$ are both even, then $f(x) = x+1$ and $f(y) = y+1$.  Since $f(x) = f(y)$, we have $x + 1 = y + 1$ which implies that $x = y$.  Similarly, if $x$ and $y$ are both odd, then $x - 3 = y-3$ so again $x = y$.  The only other possibility is that $x$ is even an $y$ is odd (or visa-versa).  But then $x + 1$ would be odd and $y - 3$ would be even, so it cannot be that $f(x) = f(y)$.  Therefore if $f(x) = f(y)$ we then have $x = y$, which proves that $f$ is injective.
		  \end{proof}
		% Is $f$ surjective?  Prove your answer.
		  \part $f$ is surjective.
		  \begin{proof}
		   Let $y$ be an element of the codomain $\Z$.  We will show there is an element $n$ of the domain ($\Z$) such that $f(n) = y$.  There are two cases.  First, if $y$ is even, then let $n = y+3$.  Since $y$ is even, $n$ is odd, so $f(n) = n-3 = y+3-3 = y$ as desired.  Second, if $y$ is odd, then let $n = y-1$.  Since $y$ is odd, $n$ is even, so $f(n) = n+1 = y-1+1 = y$ as needed.  Therefore $f$ is surjective.
		  \end{proof}

		\end{parts}
	
\end{ans}
\begin{ans}{0.3.13.}
	   Yes, this is a function, if you choose the domain and codomain correctly.  The domain will be the set of students, and the codomain will be the set of possible grades.  The function is almost certainly not injective, because it is likely that two students will get the same grade.  The function might be surjective - it will be if there is at least one student who gets each grade.
	
\end{ans}
\begin{ans}{0.3.14.}
		Yes, as long as the set of cards is the domain and the set of players is the codomain.  The function is not injective because multiple cards go to each player.  It is surjective since all players get cards.
	
\end{ans}
\begin{ans}{0.3.15.}
	  This cannot be a function.  If the domain were the set of cards, then it is not a function because not every card gets dealt to a player.  If the domain were the set of players, it would not be a function because a single player would get mapped to multiple cards.  Since this is not a function, it doesn't make sense to say whether it is injective/surjective/bijective.
	
\end{ans}
\protect \end {itemize}
 \protect \noindent {\protect \textbf  {Solutions for Section 1.1}} \protect \begin {itemize} 
\begin{ans}{1.1.1.}
    255.
  
\end{ans}
\begin{ans}{1.1.2.}
    8.
  
\end{ans}
\begin{ans}{1.1.3.}
    15.
  
\end{ans}
\begin{ans}{1.1.4.}
		$7 + 3\cdot (4+3) = 28$.
	
\end{ans}
\begin{ans}{1.1.5.}
    \begin{parts}
      \part 16 is the number of choices you have if you want to watch one movie, either a comedy or horror flick.
      \part 63 is the number of choices you have if you will watch two movies, first a comedy and then a horror.
    \end{parts}
  
\end{ans}
\begin{ans}{1.1.6.}
    $0 \le |A \cap B| \le 10$ and $15 \le |A \cup B| \le 25$.
  
\end{ans}
\begin{ans}{1.1.7.}
      $|A \cup B| + |A \cap B| = 13$
  
\end{ans}
\begin{ans}{1.1.8.}
    39.
  
\end{ans}
\begin{ans}{1.1.9.}
      $|(A \cup C)\setminus B| = 44$.  Use a Venn diagram.
    
\end{ans}
\begin{ans}{1.1.10.}
	One possibility: $(A \cup B) \cap C$.
    
\end{ans}
\begin{ans}{1.1.11.}
    \begin{parts}
      \part $8^5$, since you select from 8 letters 5 times.  %How many of these words are there total?
      \part $8\cdot 7\cdot 6\cdot 5\cdot 4$.  After selecting a letter, you have fewer letters to select for the next one.  %How many of these words contain no repeated letters?
      \part 64 - you need to select the 4th and 5th letters. %How many of these words (repetitions allowed) start with the sub-word ``aha''?
      \part $64 + 64 - 0 = 128$.  There are 64 words which start with ``aha'' and another 64 words that end with ``bah.''  Perhaps we over counted the words that both start with ``aha'' and end with ``bah'' but since the words are only 5 letters long, there are no such words.  %How many of these words (repetitions allowed) either start with ``aha'' or end with ``bah'' or both?
      \part $(8\cdot 7\cdot 6\cdot 5\cdot 4) - 3\cdot (5\cdot 4) = 6660$. All the words minus the bad ones.  The taboo word can be in any of three positions (starting with letter 1, 2, or 3) and for each position we must choose the other two letters (from the remaining 5 letters) %How many of the words containing no repeats also do not contain the sub-word ``bad'' (in consecutive letters)?
    \end{parts}
  
\end{ans}
\protect \end {itemize}
 \protect \noindent {\protect \textbf  {Solutions for Section 1.2}} \protect \begin {itemize} 
\begin{ans}{1.2.1.}
    \begin{parts}
      \part $2^6 = 64$  %How many subsets are there total?
      \part $2^3 = 8$.  We need to select yes/no for each of the remaining three elements.  %How many subsets contain $\{2,3,5\}$ as a subset?
%      \part $2^3 = 8$.  We need to decide yes/no for the three non-prime elements.  %How many subsets of $S$ contain no prime numbers?
      \part $2^6 - 2^3 = 56$.  There are 8 subsets which do not contain any odd numbers. %How many subsets contain at least one odd number?
      \part $3\cdot 2^3 = 24$.  First pick the even number.  Then say yes or no to each of the odd numbers.
    \end{parts}
  
\end{ans}
\begin{ans}{1.2.2.}
    \begin{parts}
      \part ${6\choose 4} = 15$
      \part ${3 \choose 1} = 3$.  We need to select 1 of the 3 remaining elements to be in the subset.

      \part ${6 \choose 4} = 15$.  All subsets of cardinality 4 must contain at least one odd number.
      \part ${3 \choose 1} = 3$.  Select 1 of the 3 even numbers.  The remaining three odd numbers of $S$ must all be in the set.
    \end{parts}
  
\end{ans}
\begin{ans}{1.2.3.}
	\part We can think of each row as a 6-bit string of weight 3 (since of the 6 coins, we require 3 to be pennies).  Thus there are ${6 \choose 3} = 20$ rows possible.  Each row requires 6 coins, so if we want to make all the rows at the same time, we will need 120 coins (60 of each).

	\part Now there are $2^6 = 64$ rows possible, which is also ${6 \choose 0} + {6\choose 1} + {6 \choose 2} + {6 \choose 3} + {6 \choose 4} + {6 \choose 5} + {6 \choose 6}$.  Thus we need $6 \cdot 64 = 384$ coins (192 of each).
	
\end{ans}
\begin{ans}{1.2.4.}
    ${10 \choose 6} + {10\choose 7} + {10\choose 8} + {10 \choose 9} + {10\choose 10} = 386$
  
\end{ans}
\begin{ans}{1.2.5.}
	${10 \choose 6} + {10\choose 7} + {10\choose 8} + {10 \choose 9} + {10\choose 10} = 386$.  This is the same as the previous question, since we can think of each subset as a 10-bit string with a 1 representing that we include that element in the subset.
	
\end{ans}
\begin{ans}{1.2.6.}
		To get an $x^{12}$, we must pick 12 of the 15 factors to contribute an $x$, leaving the other 3 to contribute a 2.  There are ${15 \choose 12}$ ways to select these 12 factors.  So the term containing an $x^{12}$ will be ${15 \choose 12}x^{12}2^{3}$.  In other words the coefficient of $x^{12}$ is ${15\choose 12}2^3$.
	
\end{ans}
\begin{ans}{1.2.7.}
    Use the binomial theorem.  ${14\choose 9} + {15 \choose 6}2^9$.
  
\end{ans}
\begin{ans}{1.2.8.}
    \begin{parts}
      \part ${14 \choose 7}$ %end at (10,10)?
      \part ${6 \choose 2}{8\choose 5}$ %end at (10,10) and pass through (5,7)?
      \part ${14 \choose 7} - {6\choose 2}{8 \choose 5}$ %end at (10,10) and avoid (5,7)?
    \end{parts}
  
\end{ans}
\begin{ans}{1.2.9.}
	 \part ${11 \choose 3} = 165$, since you have to select a 3-element subset of the set of 11 toppings.
	 \part ${10 \choose 3} = 120$, since you must select 3 of the 10 non-pineapple toppings.
	 \part ${10 \choose 2} = 45$, since you must select 2 of the remaining 10 non-pineapple toppings to have in addition to the pineapple.
	 \part $165  = 120 + 45$, which makes sense because every 3-topping pizza either has pineapple or does not have pineapple as a topping.
	
\end{ans}
\begin{ans}{1.2.10.}
		The coefficient of $x^5y^3$ is ${8\choose 5}$, since we must pick 5 of the 8 factors to contribute an $x$.  The coefficient of $x^3y^5$ is ${8 \choose 3}$, since we pick 3 out of the 8 factors to contribute an $x$.  But ${8 \choose 5} = {8\choose 3}$, because we could just as easily have picked 5 out of the 8 factors to contribute a $y$.
	
\end{ans}
\protect \end {itemize}
 \protect \noindent {\protect \textbf  {Solutions for Section 1.3}} \protect \begin {itemize} 
\begin{ans}{1.3.1.}
    \begin{parts}
      \part ${10 \choose 3}$ %How many 3-topping pizzas could they put on their menu?  Assume double toppings are not allowed.
      \part $2^{10}$ %How many total pizzas are possible, with between zero and ten toppings (but not double toppings) allowed?
      \part $P(10,5)$  %The pizza parlor will list the 10 toppings in two columns on their menu.  How many ways can they arrange the toppings in the left column?
    \end{parts}
  
\end{ans}
\begin{ans}{1.3.2.}
	Despite its name, we are not looking for a combination here.  The order in which the three numbers appears matters.  There are $P(40,3) = 40\cdot 39 \cdot 38$ different possibilities for the ``combination''.
\end{ans}
\begin{ans}{1.3.3.}
	\begin{parts}
		\part This is just the multiplicative principle.  There are 7 digits which we can select for each of the 5 positions, so we have $7^5$ such numbers.
		\part Now we have 7 choices for the first number, 6 for the second, etc.  So there are $7 \cdot 6 \cdot 5 \cdot 4 \cdot 3 = P(7,5)$ such numbers.
		\part To build such a number we simply must select 5 different digits.  After doing so, there will only be one way to arrange them into a 5-digit number.  Thus there are ${7 \choose 5}$ such numbers.
		\part The permutation is in part (b), while the combination is in part (c).  At first this seems backwards, since usually we use combinations for when order does not matter.  Here it looks like in part (c) that order does matter.  The better way to distinguish between combinations and permutations is to ask whether we are counting different arrangements as different outcomes.  In part (c), there is only one arrangement of any set of 5 digits, while in part (b) each set of 5 digits gives $5!$ different outcomes.
	\end{parts}
	
\end{ans}
\begin{ans}{1.3.4.}
    ${7\choose 2}{7\choose 2}$
  
\end{ans}
\begin{ans}{1.3.5.}
    \begin{parts}
      \part 5 (you need to skip one dot the top and the bottom). %Squares?
      \part ${7 \choose 2}$ - once you select the two dots on the top, the bottom two are determined. % Rectangles?
      \part This is tricky - you need to worry about running out of space.  One way to count: break into cases by the location of the top left corner.  You get ${7 \choose 2} + ({7 \choose 2}-1) + ({7 \choose 2} - 3) + ({7 \choose 2} - 6) + ({7 \choose 2} - 10) + ({7 \choose 2} - 15)$ %Parallelograms?
      \part All of them %Trapezoids?
    \end{parts}
  
\end{ans}
\begin{ans}{1.3.6.}
		Since there are 15 different letters, we have 15 choices for the first letter, 14 for the next, and so on.  Thus there are $15!$ anagrams.
	
\end{ans}
\begin{ans}{1.3.7.}
	 After the first letter, we must rearrange the remaining 7 letters.  There are only two letters, so this is really just a bit-string question.  Thus there ${7 \choose 2}$ anagrams starting with ``a''.
	
\end{ans}
\begin{ans}{1.3.8.}
		First, decide where to put the ``a''s.  There are 7 positions, and we must choose 3 of them to fill with an ``a''.  This can be done in ${7 \choose 3}$ ways.  The remaining 4 spots all get a different letter, so there are $4!$ ways to finish off the anagram.  This gives a total of ${7 \choose 3}\cdot 4!$ anagrams.  Strangely enough, this is 840, which is also equal to $P(7,4)$.  To get the answer that way, start by picking one of the 7 \emph{positions} to be filled by the ``n'', one of the remaining 6 positions to be filled by the ``g'', one of the remaining 5 positions to be filled by the ``r'', one of the remaining 4 positions to be filled by the ``m'' and then put ``a''s in the remaining 3 positions.
	
\end{ans}
\begin{ans}{1.3.9.}
    \begin{parts}
      \part ${20 \choose 4}{16 \choose 4}{12 \choose 4}{8 \choose 4}{4 \choose 4}$ %You need to divide up into foursomes (groups of 4 people): a first foursome, a second foursome, and so on.  How many ways can you do this?
      \part $5!{15 \choose 3}{12 \choose 3}{9 \choose 3}{6 \choose 3}{3 \choose 3}$ %After all your hard work, you realize that in fact, you want each foursome to include one of the five CEO's.  How many ways can you do this?
    \end{parts}
  
\end{ans}
\begin{ans}{1.3.10.}
     $9!$ (there are 10 people seated around the table, but it does not matter where King Arthur sits, only who sits to his left, two seats to his left, and so on).
  
\end{ans}
\protect \end {itemize}
 \protect \noindent {\protect \textbf  {Solutions for Section 1.4}} \protect \begin {itemize} 
\begin{ans}{1.4.1.}
		\begin{proof}
			\underline{Question}: How many subsets of size $k$ are there of the set $\{1,2,\ldots, n\}$?

			\underline{Answer 1}: You must choose $k$ out of $n$ elements to put in the set, which can be done in ${n \choose k}$ ways.

			\underline{Answer 2}: First count the number of $k$-element subsets of $\{1,2,\ldots, n\}$ which contain the number $n$.  We must choose $k-1$ of the $n-1$ other element to include in this set.  Thus there are ${n-1\choose k-1}$ such subsets.  We have not yet counted all the $k$-element subsets of $\{1,2,\ldots, n\}$ though.  In fact, we have missed exactly those subsets which do NOT contain $n$.  To form one of these subsets, we need to choose $k$ of the other $n-1$ elements, so this can be done in ${n-1 \choose k}$ ways.    Thus the answer to the question is ${n-1 \choose k-1} + {n-1 \choose k}$.
		\end{proof}
	
\end{ans}
\begin{ans}{1.4.2.}
		\begin{proof}
		\underline{Question}: How many 2-letter words start with \textit{a}, \textit{b}, or \textit{c} and end with either \textit{y} or \textit{z}?

		\underline{Answer 1}: There are two words that start with \textit{a}, two that start with \textit{b}, two that start with \textit{c}, for a total of $2+2+2$.

		\underline{Answer 2}:  There are three choices for the first letter and two choices for the second letter, for a total of $3 \cdot 2$.

		Since the two answer are both answers to the same question, they are equal.  Thus $2 + 2 + 2 = 3\cdot 2$.
		\end{proof}
	
\end{ans}
\begin{ans}{1.4.3.}
	\begin{proof}
        \underline{Question}: How many subsets of $A = {1,2,3, \ldots, n+1}$ contain exactly two elements?

        \underline{Answer 1}: We must choose 2 elements from $n+1$ choices, so there are ${n+1 \choose 2}$ subsets.

        \underline{Answer 2}: We break this question down into cases, based on what the larger of the two elements in the subset is. The larger element can't be 1, since we need at least one element smaller than it.

        Larger element is 2: there is 1 choice for the smaller element.

        Larger element is 3: there are 2 choices for the smaller element.

        Larger element is 4: there are 3 choices for the smaller element.

        And so on.  When the larger element is $n+1$, there are $n$ choices for the smaller element.  Since each two element subset must be in exactly one of these cases, the total number of two element subsets is $1 + 2 + 3 + \cdots + n$.

        Answer 1 and answer 2 are both correct, so they must be equal.  Therefore
        \[1 + 2 + 3 + \cdots + n = {n+1 \choose 2}\]
       \end{proof}
	
\end{ans}
\begin{ans}{1.4.4.}
		\begin{parts}
		 \part She has ${15 \choose 6}$ ways to select the 6 bride's maids, and then for each way, has 6 choices for the maid of honor.  Thus she has ${15 \choose 6}6$ choices.  %What if she first selects the 6 bride's maids, and then selects one of them to be the maid of honor?
		 \part She has 15 choices for who will be her maid of honor.  Then she needs to select 5 of the remaining 14 friends to be bride's maids, which she can do in ${14 \choose 5}$ ways.  Thus she has $15 {14 \choose 5}$ choices.  %What if she first selects her maid of honor, and then 5 other bride's maids?
		 \part We have answered the question (how many wedding parties can the bride choose from) in two ways.  The first way gives the left hand side of the identity and the second way gives the right hand side of the identity.  Therefore the identity holds. %Explain why $6 {15 \choose 6} = 15 {14 \choose 5}$.
		\end{parts}
	
\end{ans}
\begin{ans}{1.4.5.}
		\begin{proof}
		\underline{Question}: You have a large container filled with ping-pong balls, all with a different number of them.  You must select $k$ of the balls, putting two of them in a jar and the others in a box.  How many ways can you do this?

		\underline{Answer 1}: First select 2 of the $n$ balls to put in the jar, then select $k-2$ of the remaining $n-2$ balls to put in the box.  The first task can be completed in ${n \choose 2}$ different ways, the second task in ${n-2 \choose k-2}$ ways.  Thus there are ${n \choose 2}{n-2 \choose k-2}$ ways to select the balls.

		\underline{Answer 2}:  First select $k$ balls from the $n$ in the container.  Then pick 2 of the $k$ balls you picked to put in the jar, placing the remaining $k-2$ in the box.  The first task can be completed in ${n \choose k}$ ways, the second task in ${k \choose 2}$ ways.  Thus there are ${n \choose k}{k \choose 2}$ ways to select the balls.

		Since both answers count the same thing, they must be equal.
		\end{proof}
	
\end{ans}
\begin{ans}{1.4.6.}
		\begin{parts}
		 \part After the 1, we need to find a 5-bit string with one 1.  There are ${5 \choose 1}$ ways to do this. %How many of those bit strings start with 1?
		 \part ${4 \choose 1}$ (we need to pick 1 of the remaining 4 slots to be the second 1). %How many of those bit strings start with 01?
		 \part ${3 \choose 1}$ %How many of those bit strings start with 001?
		 \part Yes.  We still need strings starting with 0001 (there are ${2 \choose 1}$ of these) and strings starting 00001 (there is only ${1 \choose 1} = 1$ of these).  %Are there any other strings we have not counted yet?  Which ones, and how many are there?
		 \part ${6 \choose 2}$ %How many bit strings are there total in $\b B^6_2$?
		 \part An example of the Hockey Stick Theorem:  %What binomial identity have you just given a combinatorial proof for?
		 \[{1 \choose 1} + {2 \choose 1} + {3 \choose 1} + {4 \choose 1} + {5 \choose 1} = {6 \choose 2}\]
		\end{parts}
	
\end{ans}
\begin{ans}{1.4.7.}
		\begin{parts}
		 \part $3^n$, since there are 3 choices for each of the $n$ digits.  %How many ternary digit strings contain exactly $n$ digits?
		 \part $1$, since all the digits need to be 2's.  However, we might write this as ${n \choose 0}$.  %How many ternary digit strings contain exactly $n$ digits and $n$ 2's.
		 \part There are ${n \choose 1}$ places to put the non-2 digit.  That digit can be either a 0 or a 1, so there are $2{n \choose 1}$ such strings.  %How many ternary digit strings contain exactly $n$ digits and $n-1$ 2's.  (Hint: where can you put the non-2 digit, and then what could it be?)
		 \part We must choose two slots to fill with 0's or 1's.  There are ${n \choose 2}$ ways to do that.  Once the slots are picked, we have two choices for the first slot (0 or 1) and two choices for the second slot (0 or 1).  So there are a total of $2^2{n \choose 2}$ such strings. %How many ternary digit strings contain exactly $n$ digits and $n-2$ 2's.  (Hint: see previous hint)
		 \part There are ${n \choose k}$ ways to pick which slots don't have the 2's.  Then those slots can be filled in $2^k$ ways (0 or 1 for each slot).  So there are $2^k{n \choose k}$ such strings. %How many ternary digit strings contain exactly $n$ digits and $n-k$ 2's.
		 \part These strings contain just 0's and 1's - so they are bit strings.  There are $2^n$ bit strings.  But keeping with the pattern above, we might write this as $2^n {n \choose n}$. %How many ternary digit strings contain exactly $n$ digits and no 2's. (Hint: what kind of a string is this?)
		 \part We answer the question of how many length $n$ ternary digit strings there are in two ways.  First, each digit can be one of three choices, so the total number of strings is $3^n$.  On the other hand, we could break the question down into cases by how many of the digits are 2's.  If they are all 2's, then there are ${n \choose 0}$ strings.  If all but one is a 2, then there are $2{n \choose 1}$ strings.  If all but 2 of the digits are 2's, then there are $2^2{n \choose 2}$ strings - we choose 2 of the $n$ digits to be non-2, and then there are 2 choices for each of those digits.  And so on for every possible number of 2's in the string.  %Use the above parts to give a combinatorial proof for the identity
		 %\[{n \choose 0} + 2{n \choose 1} + 2^2{n \choose 2} + 2^3{n \choose 3} + \cdots + 2^n{n \choose n} = 3^n\]
		\end{parts}
	
\end{ans}
\begin{ans}{1.4.8.}
		The word contains 9 letters: 3 ``r''s, 2 ``a''s and 2 ``e''s, along with an ``n'' and a ``g''.  We could first select the positions for the ``r''s in ${9 \choose 3}$ ways, then the ``a''s in ${6 \choose 2}$ ways, the ``e''s in ${4 \choose 2}$ ways and then select one of the remaining two spots to put the ``n'' (placing the ``g'' in the last spot).  This gives the answer
		\[{9 \choose 3}{6 \choose 2}{4 \choose 2}{2\choose 1}{1\choose 1}\]
		Alternatively, we could select the positions of the letters in the opposite order, which would give an answer
		\[{9 \choose 1}{8\choose 1}{7 \choose 2}{5\choose 2}{3\choose 3}\]
		(where the 3 ``r''s go in the remaining 3 spots).  These two expressions are equal:
		\[{9 \choose 3}{6 \choose 2}{4 \choose 2}{2\choose 1}{1\choose 1} = {9 \choose 1}{8\choose 1}{7 \choose 2}{5\choose 2}{3\choose 3}\]
	
\end{ans}
\begin{ans}{1.4.9.}
		\begin{proof}
         \underline{Question}: How many $k$-letter words can you make using $n$ different letters without repeating any letter?

         \underline{Answer 1}: There are $n$ choices for the first letter, $n-1$ choices for the second letter, $n-2$ choices for the third letter, and so on until $n - (k-1)$ choices for the $k$th letter (since $k-1$ letters have already been assigned at that point).  The product of these numbers can be written $\frac{n!}{(n-k)!}$ which is $P(n,k)$.

         \underline{Answer 2}: First pick $k$ letters to be in the word from the $n$ choices.  This can be done in ${n \choose k}$ ways.  Now arrange those letters into a word - there are $k$ choices for the first letter, $k-1$ choices for the second, and so on, for a total of $k!$ arrangements of the $k$ letters.  Thus the total number of words is ${n \choose k}k!$.
        \end{proof}
	
\end{ans}
\begin{ans}{1.4.10.}
		\begin{proof}
		\underline{Question}: How many 5-element subsets are there of the set $\{1,2,\ldots, n+3\}$.

		\underline{Answer 1}: We choose 5 out of the $n+3$ elements, so ${n+3 \choose 5}$.

		\underline{Answer 2}: Break this up into cases by what the ``middle'' (third smallest) element of the 5 element subset is.  The smallest this could be is a 3.  In that case, we have ${2 \choose 2}$ choices for the numbers below it, and ${n \choose 2}$ choices for the numbers above it.  Alternatively, the middle number could be a 4.  In this case there are ${3 \choose 2}$ choices for the bottom two numbers and ${n-1 \choose 2}$ choices for the top two numbers.  If the middle number is 5, then there are ${4 \choose 2}$ choices for the bottom two numbers and ${n-2 \choose 2}$ choices for the top two numbers.  An so on, all the way up to the largest the middle number could be, which is $n+1$.  In that case there are ${n \choose 2}$ choices for the bottom two numbers and ${2 \choose 2}$ choices for the top number.  Thus the number of 5 element subsets is.
		\[{2 \choose 2}{n \choose 2} + {3 \choose 2}{n-1 \choose 2} + {4\choose 2}{n-2 \choose 2} + \cdots + {n\choose 2}{2\choose 2}\]
		\end{proof}
	
\end{ans}
\protect \end {itemize}
 \protect \noindent {\protect \textbf  {Solutions for Section 1.5}} \protect \begin {itemize} 
\begin{ans}{1.5.1.}
		\begin{parts}
			\part ${10\choose 5}$.  We must select 5 of the 10 digits to put in the set.
			\part Use stars and bars: each star represents one of the 5 elements of the set, each bar represents a switch between digits.  So there are 5 stars and 9 bars, giving us ${14 \choose 9}$.
		\end{parts}
	
\end{ans}
\begin{ans}{1.5.2.}
		\begin{parts}
			\part You take 3 strawberry, 1 lime, 0 licorice, 2 blueberry and 0 bubblegum.
			\part This is backwards.  We don't want the stars to represent the kids because the kids are not identical, but the stars are.  Instead we should use 5 stars (for the lollipops) and use 5 bars to switch between the 6 kids.  For example, **||***||| would represent the outcome with the first kid getting 2 lollipops, the third kid getting 3, and the rest of the kids getting none.
			\part This is the word AAAEOO.
			\part This doesn't represent a solution.  Each star should represent one of the 6 units that add up to 6, and the bars should \emph{switch} between the different variables.  We have one too many bars.
		\end{parts}
	
\end{ans}
\begin{ans}{1.5.3.}
	 \begin{parts}
	   \part ${18 \choose 4}$.  Each outcome can be represented by a sequence of 14 stars and 4 bars. %How many ways can you do this if there are no restrictions?
	   \part ${13 \choose 4}$.  First put one ball in each bin.  This leaves 9 stars and 4 bars.%How many ways can you do this if each bin must contain at least one dodge-ball?
%	   \part ${18 \choose 4} - \left[ {5 \choose 1}{11 \choose 4} - {5 \choose 2}{4 \choose 4}\right]$.  Subtract all the distributions for which one or more bins contain 7 or more balls.  %How many ways can you do this if no bin can hold more than 6 balls?
	 \end{parts}
	
\end{ans}
\begin{ans}{1.5.4.}
	\begin{parts}
	  \part ${7 \choose 2}$.  After each variable gets 1 star for free, we are left with 5 stars and 2 bars.  %$x$, $y$, and $z$ are all positive?
	  \part ${10 \choose 2}$.  We have 8 stars and 2 bars.  %$x$, $y$, and $z$ are all non-negative?
	  \part ${19 \choose 2}$.  This problem is equivalent to finding the number of solutions to $x' + y' + z' = 17$ where $x'$, $y'$ and $z'$ are non-negative.  (In fact, we really just do a substitution.  Let $x = x'- 3$, $y = y' - 3$ and $z = z' - 3$).  %$x$, $y$, and $z$ are all greater than $-3$.
	\end{parts}
	
\end{ans}
\begin{ans}{1.5.5.}
	\begin{parts}
		\part This is ${7 \choose 5}$.  We simply choose five of the seven digits and once chosen put them in increasing order.
		\part This requires stars and bars.  Use a star to represent each of the 5 digits in the number, and use their position relative to the bars to say what numeral fills that spot.  So we will have 5 stars and 6 bars, giving ${11 \choose 6}$.
	\end{parts}
	
\end{ans}
\begin{ans}{1.5.6.}
	${10 \choose 5}$.  We have 5 stars (the five dice) and 5 bars (the five switches between the number 1-6).
	
\end{ans}
\begin{ans}{1.5.7.}
	 We must figure out how many different combinations of 7 coins are possible.  Let a star represent each coin, and a bar represent switching between type of coin.  So for example **|*||**** represents 2 pennies, one nickel, no dimes and 4 quarters.  The number of such star and bar diagrams (with 7 stars and 3 bars) is ${10 \choose 3} = 120$.  Thus you have a 1 in 120 chance of guessing correctly.
	
\end{ans}
\begin{ans}{1.5.8.}
	${18 \choose 3}$.  Distribute 10 units to the variables before finding all solutions to $x_1' + x_2' + x_3' + x_4' = 15$ in non-negative integers.
	
\end{ans}
\begin{ans}{1.5.9.}
	  The answer to each of the counting questions is ${10 \choose 2}$, using 8 stars and 2 bars.  The bars separate the kids, the variables, and the colors.  To see why these are really the same, notice that we can think of the 3 kids as named $x$, $y$, and $z$.  Or think of $x$ as red, $y$ as blue and $z$ as yellow.  Notice that in each case, the three things are distinguishable, while the 8 things (cookies, units, crayons) are all identical.
	
\end{ans}
\begin{ans}{1.5.10.}
   \begin{parts}
    \part ${20 \choose 4}$ (order does not matter and repeats are not allowed)
    \part $P(20, 4) = 20\cdot 19\cdot 18 \cdot 17$ (order matters and repeats are not allowed)
    \part ${23 \choose 19}$ (order does not matter and repeats are allowed; stars and bars)
    \part $20^4$ (order matters and repeats are allowed; 20 choices 4 times)
   \end{parts}
  
\end{ans}
\protect \end {itemize}
 \protect \noindent {\protect \textbf  {Solutions for Section 1.6}} \protect \begin {itemize} 
\begin{ans}{1.6.1.}
  Hint: Stars and bars.
   \begin{parts}
    \part ${9 \choose 6}$
    \part ${16 \choose 6}$
    \part ${16 \choose 6} - \left[{7 \choose 1}{13 \choose 6} - {7 \choose 2}{10 \choose 6} + {7 \choose 3}{7 \choose 6}\right]$
   \end{parts}
  
\end{ans}
\begin{ans}{1.6.2.}
 ${18 \choose 4} - \left[ {5 \choose 1}{11 \choose 4} - {5 \choose 2}{4 \choose 4}\right]$.  Subtract all the distributions for which one or more bins contain 7 or more balls.  %How many ways can you do this if no bin can hold more than 6 balls?
	
\end{ans}
\begin{ans}{1.6.3.}
	The easiest way to solve this is to instead count the solutions to $y_1 + y_2 + y_3 + y_4 = 7$ with $0 \le y_i \le 3$.  By taking $x_i = y_i+2$, each solution to this new equation corresponds to exactly one solution to the original equation.

	Now all the ways to distribute the 7 units to the four $y_i$ variables can be found using stars and bars, specifically 7 stars and 3 bars, so ${10 \choose 3}$.  But this includes the ways that one or more $y_i$ variables can be assigned more than 3 units.  So subtract, using PIE.  We get
	\[{10 \choose 3} - {4\choose 1} {6 \choose 3}.\]
	Note that this is the final answer because it is not possible to have two variables both get 4 units.
	
\end{ans}
\begin{ans}{1.6.4.}
		Without any restriction, there would be ${19\choose 12}$ ways to distribute the stars.  Now we must use PIE to eliminate all distributions in which one or more student gets more than one star:
		\[{19 \choose 12} - \left[{13 \choose 1}{17 \choose 12} - {13\choose 2}{15 \choose 12} + {13\choose 3}{13 \choose 12}\right] = 1716\]
		Interestingly enough, this number is also the value of ${13 \choose 7}$, which makes sense: if each student can have at most one star, we must just pick the 7 out of 13 students to receive them.
	
\end{ans}
\begin{ans}{1.6.5.}
	The question is, how many ways can you distribute $k$ cookies to $n$ kids so that each kid gets at most one cookie.  On one hand, the answer is just ${n \choose k}$ since you must choose $k$ kids to get a cookie.  Alternatively, we can use stars and bars with PIE, which is how we get the right hand side of the identity.  Note that lots of the terms on the right hand side will be zero, as soon as $n+k-(2j+1)$ drops below $k$.
	
\end{ans}
\begin{ans}{1.6.6.}
		The 9 derangements are: 2143, 2341, 2413, 3142, 3412, 3421, 4123, 4312, 4321.
	
\end{ans}
\begin{ans}{1.6.7.}
	${10 \choose 6}\left(4! - \left[{4 \choose 1} 3! - {4 \choose 2}2! + {4 \choose 3}1! - {4 \choose 4}0!\right]\right)$.  We choose 6 of the 10 ladies to get their own hat, and the multiply by the number of ways the remaining hats can be deranged.
	
\end{ans}
\protect \end {itemize}
 \protect \noindent {\protect \textbf  {Solutions for Section 1.7}} \protect \begin {itemize} 
\begin{ans}{1.7.1.}
	There are 8 different functions.  For example, $f(1) = a$, $f(2) = a$, $f(3) = a$; or $f(1) = a$, $f(2) = b$, $f(3) = a$, and so on.  None of the functions are injective.  Exactly 6 of the functions are surjective.  No functions are both (since no functions here are injective).
	
\end{ans}
\begin{ans}{1.7.2.}
	There are nine functions - you have a choice of three outputs for $f(1)$, and for each, you have three choices for the output $f(2)$.  Of these functions, 6 are injective, 0 are surjective, and 0 are both.
	
\end{ans}
\begin{ans}{1.7.3.}
	\begin{parts}
	  \part $6^4 = 1296$, since there are six choices of where to send each of the 4 elements of the domain. %How many functions are there total?
	  \part $P(6, 4) = 6 \cdot 5 \cdot 4 \cdot 3 = 360$, since outputs cannot be repeated.  %How many functions are injective?
	  \part None. %How many functions are surjective?
	  \part There are $5 \cdot 6^3$ functions for which $f(1) \ne a$ and another $5 \cdot 6^3$ functions for which $f(2) \ne b$.  There are $5^2 \cdot 6^2$ functions for which both $f(1) \ne a$ and $f(2) \ne b$.  So the total number of functions for which $f(1) \ne a$ or $f(2) \ne b$ or both is
	  \[5 \cdot 6^3 + 5 \cdot 6^3 - 5^2 \cdot 6^2 = 1260\] %How many functions have the property that $f(1) \ne a$ or $f(2) \ne b$, or both?
	\end{parts}
	
\end{ans}
\begin{ans}{1.7.4.}
	\begin{parts}
	  \part $17^{10}$ %How many functions $f: A \to B$ are there?
	  \part $P(17, 10)$  %How many functions $f: A \to B$ are injective?
	\end{parts}
	
\end{ans}
\begin{ans}{1.7.5.}
	$5^{10} - \left[{5 \choose 1}4^{10} - {5 \choose 2}3^{10} + {5 \choose 3}2^{10} - {5 \choose 4}1^{10}\right]$ %Consider sets $A$ and $B$ with $|A| = 10$ and $|B| = 5$.  How many functions $f: A \to B$ are surjective?
	
\end{ans}
\begin{ans}{1.7.6.}
	$5! - \left[{5 \choose 1}4! - {5 \choose 2}3! + {5 \choose 3}2! - {5 \choose 4}1! + {5 \choose 5}0!\right]$.  This is a sneaky way to as for the number of derangements on 5 elements. %Let $A = \{1,2,3,4,5\}$.  How many injective functions $f:A \to A$ have the property that for each $x \in A$, $f(x) \ne x$?
	
\end{ans}
\protect \end {itemize}
 \protect \noindent {\protect \textbf  {Solutions for Section 1.8}} \protect \begin {itemize} 
\begin{ans}{1.8.1.}
	\begin{parts}
	  \part ${8 \choose 3}$, after giving one present to each kid, you are left with 5 presents (stars) which need to be divide among the 4 kids (giving 3 bars). %The presents are identical, and each kid gets at least one present?
	  \part ${12 \choose 3}$.  You have 9 stars and 3 bars.  %The presents are identical, and some kids might get no presents?
	  \part $4^9$.  You have 4 choices for whom to give each present.  This is like making a function from the set of present to the set of kids. %The presents are unique, and some kids might get no presents?
	  \part $4^9 - \left[{4 \choose 1}3^9 - {4\choose 2}2^9 + {4 \choose 3}1^9 \right]$.  Now the function from the set of present to the set of kids must be surjective. %the presents are unique and each kid gets at least one present?
	\end{parts}
	
\end{ans}
\begin{ans}{1.8.2.}
	\begin{parts}
	    \part Neither.  ${14 \choose 4}$.
	  \part ${10\choose 4}$
	  \part $P(10,4)$, since order is important.
	  \part Neither. Assuming you will wear each of the 4 ties on just 4 of the 7 days, without repeats: ${10\choose 4}P(7,4)$.
	  \part $P(10,4)$
	  \part ${10\choose 4}$
	  \part Neither. Since you could repeat letters: $10^4$. If no repeats are allowed, it would be $P(10,4)$.
	  \part Neither.  Actually, ``k'' is the 11th letter of the alphabet, so the answer is 0.  If ``k'' was among the first 10 letters, there would only be 1 way - write it down.
	  \part Neither.  Either ${9\choose 3}$ (if every kid gets an apple) or ${13 \choose 3}$ (if appleless kids are allowed).
	  \part Neither.  Note that this could not be ${10 \choose 4}$ since the 10 things and 4 things are from different groups.  $4^{10}$
	  \part ${10 \choose 4}$ - don't be fooled by the ``arrange'' in there - you are picking 4 out of 10 {\em spots} to put the 1's.
	  \part ${10 \choose 4}$ (assuming order is irrelevant).
	  \part Neither.  $16^{10}$ (each kid chooses yes or no to 4 varieties).
	  \part Neither.  0.
	  \part Neither.  $4^{10} - [{4\choose 1}3^{10} - {4\choose 2}2^{10} + {4 \choose 3}1^{10}]$
	  \part Neither.  $10\cdot 4$.
	  \part Neither. $4^{10}$.
	  \part ${10 \choose 4}$ (which is the same as ${10 \choose 6}$).
	  \part Neither.  If all the kids were identical, and you wanted no empty teams, it would be ${10 \choose 4}$.  Instead, this will be the same as the number of surjective functions from a set of size 11 to a set of size 5.
	  \part ${10 \choose 4}$
	  \part ${10 \choose 4}$
	  \part Neither.  $4!$.
	  \part Neither.  It's ${10 \choose 4}$ if you won't repeat any choices.  If repetition is allowed, then this becomes $x_1 + x_2 + \cdots +x_{10} = 4$, which has ${13 \choose 9}$ solutions in non-negative integers.
	  \part Neither.  Since repetition of cookie type is allowed, the answer is $10^4$.  Without repetition, you would have $P(10,4)$.
	  \part ${10 \choose 4}$ since that is equal to ${9 \choose 4} + {9 \choose 3}$.
	  \part Neither.  It will be a complicated (possibly PIE) counting problem.
	\end{parts}
	
\end{ans}
\begin{ans}{1.8.3.}
    \begin{parts}
      \part $2^8 = 256$.  You have two choices for each tie - wear it or don't. %You must select some of your ties to wear - everything is okay, from no ties up to all ties.  How many choices do you have?
      \part You have 7 choices for regular ties (the 8 choices less the ``no regular tie'' option) and 31 choices for bow ties (32 total minus the ``no bow tie'' option).  Thus total you have $7 \cdot 31 = 217$.  %If you want to wear at least one regular tie and one bow tie, but are willing to wear up to all your ties, how many choices do you have for which ties to wear?
      \part ${3\choose 2}{5\choose 3} = 30$  %How many choices do you have if you wear exactly 2 of the 3 regular ties and 3 of the 5 bow ties?
      \part $5! = 120$  %Once you have selected 2 regular and 3 bow ties, in how many orders could you put the ties on, assuming you must have one of the three bow ties on top?
    \end{parts}
  
\end{ans}
\begin{ans}{1.8.4.}
		You own 8 purple bow ties, 3 red bow ties, 3 blue bow ties and 5 green bow ties.  How many ways can you select one of each color bow tie to take with you on a trip?  $8 \cdot 3 \cdot 3 \cdot 5$.  How many choices do you have for a single bow tie to wear tomorrow?  $8 + 3 + 3 + 5$.
	
\end{ans}
\begin{ans}{1.8.5.}
		\begin{parts}
		\part $4^5$. %How many such numbers are there?
		\part $4^4\cdot 2$ (choose any digits for the first four digits - then pick either an even or an odd last digit to make the sum even). %How many such numbers are there for which the {\em sum} of the digits is even?
		\part We need 3 or more even digits.  3 even digits: ${5 \choose 3}2^3 2^2$.  4 even digits: ${5 \choose 4}2^4 2$.  5 even digits: ${5 \choose 5}2^5$.  So all together: ${5 \choose 3}2^3 2^2 + {5 \choose 4}2^4 2 + {5 \choose 5}2^5$.  %  How many such numbers contain more even digits than odd digits?
		\end{parts}
	
\end{ans}
\begin{ans}{1.8.6.}
		$|A \cup B|$ is the number of things that are in $A$ or in $B$ or in both.  If you count up everything in each set independently, then anything which is in both sets (in $A \cap B$) is counted twice.
	
\end{ans}
\begin{ans}{1.8.7.}
		215.  Use PIE: $100 + 83 + 71 - 16 - 14 -11 + 2 = 215$ or a Venn diagram.  To find out how many numbers are divisible by 6 and 7, for example, take $500/42$ and round down.
	
\end{ans}
\begin{ans}{1.8.8.}
		51.
	
\end{ans}
\begin{ans}{1.8.9.}
		\begin{parts}
		  \part $2^8$. %How many $8$-bit strings are there total?
		  \part ${8 \choose 5}$  %How many $8$-bit strings have weight 5?
		  \part ${8 \choose 5}$ %How many subsets of the set $\{a,b,c,d,e,f,g,h\}$ contain exactly 5 elements?
		  \part There is a bijection between subsets and bit strings: a 1 means that element in is the subset, a 0 means that element is not in the subset.  To get a subset of an 8 element set we have a 8-bit string.  To make sure the subset contains exactly 5 elements, there must be 5 1's, so the weight must be 5. %Explain why your answers to parts (b) and (c) are the same.  Why are these questions equivalent?
		\end{parts}
	
\end{ans}
\begin{ans}{1.8.10.}
		${13 \choose 10} + {17 \choose 8}$
	
\end{ans}
\begin{ans}{1.8.11.}
		 With repeated letters allowed: ${8 \choose 5}5^5 21^3$.  Without repeats: ${8 \choose 5}5! P(21, 3)$.
	
\end{ans}
\begin{ans}{1.8.12.}
		\begin{parts}
		  \part ${5 \choose 2}{11 \choose 6}$ %pass through the point $(2,3)$.
		  \part ${16 \choose 8} - {12 \choose 7}{4 \choose 1}$   %avoid (do not pass through) the point $(7,5)$.
		  \part ${5 \choose 2}{11 \choose 6} + {12 \choose 5}{4 \choose 3} - {5 \choose 2}{7 \choose 3}{4 \choose 3}$ %either pass through $(2,3)$ or $(5,7)$ (or both).
		\end{parts}
	
\end{ans}
\begin{ans}{1.8.13.}
		 ${18 \choose 8}\left({18 \choose 8} - 1\right)$
	
\end{ans}
\begin{ans}{1.8.14.}
		 A test had $n$ questions on it, of which you must answer any $k$ questions.  How many choices do you have as to what order you answer the questions on the test?  $P(n,k)$.  When grading the test, how many different combinations of question might the professor see?  ${n \choose k}$.
	
\end{ans}
\begin{ans}{1.8.15.}
		 $2^7 + 2^7 - 2^4$.
	
\end{ans}
\begin{ans}{1.8.16.}
		${7 \choose 3} + {7 \choose 4} - {4 \choose 1}$.
	
\end{ans}
\begin{ans}{1.8.17.}
		(a) $6! - 4\cdot 3!$.  (b) $6! - {6 \choose 3}3!$.
	
\end{ans}
\begin{ans}{1.8.18.}
		$2^n$ is the number of lattice paths which have length $n$, since for each step you can go up or right.  Such a path would end along the line $x + y = n$.  So you will end at $(0,n)$, or $(1,n-1)$ or $(2, n-2)$ or \ldots or $(n,0)$.  Counting the paths to each of these points separately, give ${n \choose 0}$, ${n \choose 1}$, ${n \choose 2}$, \ldots, ${n \choose n}$ (each time choosing which of the $n$ steps to be to the right).
	
\end{ans}
\begin{ans}{1.8.19.}
		Hint: give a combinatorial proof for the identity $P(n,k) = {n \choose k} k!$.
	
\end{ans}
\begin{ans}{1.8.20.}
		Of your $n$ bow ties, you decide to give $k$ away to charity.  How many ways can you do this?  On one hand, you can choose $k$ of the $n$ bow ties to give away in ${n \choose k}$ ways.  Alternatively, you can choose which bow ties to keep.  You must keep $n -k$ of them if you will give $k$ away, so you can choose the bow ties to keep in ${n \choose n-k}$ ways.  This gives a combinatorial proof for the identity.
	
\end{ans}
\begin{ans}{1.8.21.}
		Hint: stars and bars%Suppose you have 20 one-dollar bills to give out as prizes to your top 5 discrete math students.  How many ways can you do this if:
		\begin{parts}
		  \part ${19 \choose 4}$ %each of the 5 students gets at least 1 dollar?
		  \part ${24 \choose 4}$ %some students might get nothing?
		  \part ${19 \choose 4} - \left[{5 \choose 1}{12 \choose 4} - {5 \choose 2}{5 \choose 4}  \right]$ %each student gets at least 1 dollar but no more than 7 dollars?
		\end{parts}
	
\end{ans}
\begin{ans}{1.8.22.}
		\begin{parts}
		  \part $5^4 + 5^4 - 5^3$ %$f(1) = a$ or $f(2) = b$ (or both)?
		  \part $4\cdot 5^4 + 5 \cdot 4 \cdot 5^3 - 4 \cdot 4 \cdot 5^3$ %$f(1) \ne a$ or $f(2) \ne b$ (or both)?
		  \part $5! - \left[ 4! + 4! - 3! \right]$ %$f(1) \ne a$ {\em and} $f(2) \ne b$, and are also one-to-one?
		  \part $5! - \left[{5 \choose 1}4! - {5 \choose 2}3! + {5 \choose 3}2! - {5 \choose 4}1! + {5 \choose 5} 0!\right]$ %are onto but have $f(1) \ne a$, $f(2) \ne b$, $f(3) \ne c$, $f(4) \ne d$ and $f(5) \ne e$?
		\end{parts}
	
\end{ans}
\begin{ans}{1.8.23.}
		 ${5 \choose 1}\left( 4! - \left[{4 \choose 1}3! - {4 \choose 2}2! + {4 \choose 3} 1! - {4 \choose 4} 0!\right] \right)$
	
\end{ans}
\begin{ans}{1.8.24.}
		$4^6 - \left[{4 \choose 1}3^6 - {4 \choose 2}2^6 + {4 \choose 3} 1^6 \right]$
	
\end{ans}
\begin{ans}{1.8.25.}
		\begin{parts}
		 \part ${10 \choose 4}$.  You need to choose 4 of the 10 cookie types.  Order doesn't matter. %If you want to give your professor 4 different types of cookies, how many different combinations of cookie type can you select?  Explain your answer.
		 \part $P(10, 4) = 10 \cdot 9 \cdot 8 \cdot 7$.  You are choosing and arranging 4 out of 10 cookies.  Order matters now.  %To keep things interesting, you decide to make a different number of each type of cookie.  If again you want to select 4 cookie types, how many ways can you select the cookie types and decide for which there will be the most, second most, etc.  Explain your answer.
		 \part ${21 \choose 9}$.  You must switch between cookie type 9 times as you make your 12 cookies.  The cookies are the stars, the switches between cookie types are the bars. %You change your mind again.  This time you decide you will make a total of 12 cookies.  Each cookie could be any one of the 10 types of cookies you know how to bake (and it's okay if you leave some types out).  How many choices do you have?  Explain.
		 \part $10^{12}$.  You have 10 choices for the ``1'' cookie, 10 choices for the ``2'' cookie, and so on. %You realize that the previous plan did not account for presentation.  This time, you once again want to make 12 cookies, each one could be any one of the 10 types of cookies.  However, now you plan to shape the cookies into the numerals 1, 2, \ldots, 12 (and probably arrange them to make a giant clock - but you haven't decided on that yet).  How many choices do you have for which types of cookies to bake into which numerals?  Explain.
		 \part $10^{12} - \left[{10 \choose 1}9^{12} - {10 \choose 2}8^{12} + \cdots - {10 \choose 10}0^{12}   \right]$.  We must use PIE to remove all the ways in which one or more cookie type is not selected.
		\end{parts}
	
\end{ans}
\begin{ans}{1.8.26.}
		\begin{parts}
			\part You are giving your professor 4 types of cookies coming from 10 different types of cookies.  This does not lend itself well to a function interpretation.  We {\em could} say that the domain contains the 4 types you will give your professor and the codomain contains the 10 you can choose from, but then counting injections would be too much (it doesn't matter if you pick type 3 first and type 2 second, or the other way around, just that you pick those two types).
			\part We want to consider injective functions from the set $\{\mbox{most, second most, second least, least}\}$ to the set of 10 cookie types.  We want injections because we cannot pick the same type of cookie to give most and least of (for example).
			\part This is not a good problem to interpret as a function.  The problem is that the domain would have to be the 12 cookies you bake, but these elements are indistinguishable (there is not a first cookie, second cookie, etc.).
			\part The domain should be the 12 shapes, the codomain the 10 types of cookies.  Since we can use the same type for different shapes, we are interested in counting all functions here.
			\part Here we insist that each type of cookie be given at least once, so now we are asking for the number of surjections of those functions counted in the previous part.
		\end{parts}
	
\end{ans}
\protect \end {itemize}
 \protect \noindent {\protect \textbf  {Solutions for Section 2.1}} \protect \begin {itemize} 
\begin{ans}{2.1.1.}
		\begin{parts}
		%$2, 5, 10, 17, 26, \ldots$
		  \part $a_n = n^2 + 1$
		%   $0, 2, 5, 9, 14, 20, \ldots$
		  \part $a_n = \frac{n(n+1)}{2} - 1$
		%   $8, 12, 17, 23, 30,\ldots$
		  \part $a_n = \frac{(n+2)(n+3)}{2} + 2$
		%   $1, 5, 23, 119, 719,\ldots$
		  \part $a_n = (n+1)! - 1$ (where $n! = 1 \cdot 2 \cdot 3 \cdots n$)
		\end{parts}
	
\end{ans}
\begin{ans}{2.1.2.}
		\begin{parts}
		%Give the recursive definition for the sequence.
		  \part $F_n = F_{n-1} + F_{n-2}$ with $F_0 = 0$ and $F_1 = 1$.
		%   Write out the first few terms of the sequence of partial sums.
		  \part  $0, 1, 2, 4, 7, 12, 20, \ldots.$
		  %Give a closed formula for the sequence of partial sums in terms of $F_n$  (for example, you might say $F_0 + F_1 + \cdots + F_n = 3F_{n-1}^2 + n$, although that is definitely not correct).
		  \part $F_0 + F_1 + \cdots + F_n = F_{n+2} - 1.$
		\end{parts}
	
\end{ans}
\begin{ans}{2.1.3.}
		The sequences all have the same recurrence relation: $a_n = a_{n-1} + a_{n-2}$ (the same as the Fibonacci numbers).  The only difference is the initial conditions.
	
\end{ans}
\begin{ans}{2.1.4.}
		$3, 10, 24, 52, 108,\ldots$.  The recursive definition for $10, 24, 52, \ldots$ is $a_n = 2a_{n-1} + 4$ with $a_1 = 10$.
	
\end{ans}
\begin{ans}{2.1.5.}
		$-1, -1, 1, 5, 11, 19,\ldots$  Thus the sequence $0, 2, 6, 12, 20,\ldots$ has closed formula $a_n = (n+1)^2 - 3(n+1) + 2$.
	
\end{ans}
\begin{ans}{2.1.6.}
		Write out the first few terms of the sequence: $1, 2, 3, 4, 5, 6,\ldots$.  This is surprising at first, but note that we could write $2a_{n-1} - a_{n-2} = a_{n-1} + (a_{n-1} -a_{n-2})$, and $a_{n-1} - a_{n-2}$ is just the difference between the terms.  Initially, the difference between terms is 1, so each time we are just adding one.  So we see that $a_n = n$ is the closed formula.
	
\end{ans}
\begin{ans}{2.1.7.}
		The sequence we get is $3, 5, 7, 9, \ldots$.  One recursive definition for this is $a_n = a_{n-1} + 2$ with $a_0 = 3$.  Another option would be to take $a_n = 2a_{n-1} - a_{n-2}$ with $a_0 = 3$ and $a_1 = 5$.
	
\end{ans}
\protect \end {itemize}
 \protect \noindent {\protect \textbf  {Solutions for Section 2.2}} \protect \begin {itemize} 
\begin{ans}{2.2.1.}
		\begin{parts}
		% What is the next term in the sequence?
		\part 32.
		% Find a formula for the $n$th term of this sequence, assuming $a_1 = 8$.
		\part $a_n = 8 + 6(n-1)$
		% Find the sum of the first 100 terms of the sequence: $\sum_{k=1}^{100}a_k$.
		\part $30500$.
		\end{parts}
	
\end{ans}
\begin{ans}{2.2.2.}
		\begin{parts}
		% How many terms are there in the sequence?
		\part $n+2$ terms.
		\part $6n+1$. %second to last term
		%Find the sum of all the terms in the sequence.
		\part $\frac{(6n+8)(n+2)}{2}$
		\end{parts}
	
\end{ans}
\begin{ans}{2.2.3.}
		68117
	
\end{ans}
\begin{ans}{2.2.4.}
		$\frac{5-5\cdot 3^{21}}{-2}$
	
\end{ans}
\begin{ans}{2.2.5.}
		$\frac{1 + \frac{2^{31}}{3^{31}}}{5/3}$
	
\end{ans}
\begin{ans}{2.2.6.}
		For arithmetic: $x = 55/3$, $y = 29/3$.  For geometric: $x = 9$ and $y = 3$.
	
\end{ans}
\begin{ans}{2.2.7.}
		We have $2 = 2$, $7 = 2+5$, $15 = 2 + 5 + 8$, $26 = 2+5+8+11$, and so on.  The terms in the sums are given by the arithmetic sequence $b_n = 2+3n$.  In other words, $a_n = \sum_{k=0}^n 2+3k$.  To find this, we reverse and add.  We get $a_n = \frac{(4+3n)(n+1)}{2}$ (we have $n+1$ there because there are $n+1$ terms in the sum for $a_n$).
	
\end{ans}
\begin{ans}{2.2.8.}
		\begin{parts}
		  \part $\d\sum_{k=1}^n 2k$		%$2 + 4 + 6 + 8 + \cdots + 2n$
		  \part $\d\sum_{k=1}^{107} (1 + 4(k-1))$		%$1 + 5 + 9 + 13 + \cdots + 425$
		  \part $\d\sum_{k=1}^{50} \frac{1}{k}$		%$1 + \frac{1}{2} + \frac{1}{3} + \frac{1}{4} + \cdots + \frac{1}{50}$
		  \part $\d\prod_{k=1}^n 2k$		%$2 \cdot 4 \cdot 6 \cdot \cdots \cdot 2n$
		  \part $\d\prod_{k=1}^{100} \frac{k}{k+1}$	%$(\frac{1}{2})(\frac{2}{3})(\frac{3}{4})\cdots(\frac{100}{101})$
		\end{parts}
	
\end{ans}
\begin{ans}{2.2.9.}
		\begin{parts}
		  \part $\d\sum_{k=1}^{100} (3+4k) = 7 + 11 + 15 + \cdots + 403$
		  \part $\d\sum_{k=0}^n 2^k = 1 + 2 + 4 + 8 + \cdots + 2^n$
		  \part $\d\sum_{k=2}^{50}\frac{1}{(k^2 - 1)} = 1 + \frac{1}{3} + \frac{1}{8} + \frac{1}{15} + \cdots + \frac{1}{2499}$
		  \part $\d\prod_{k=2}^{100}\frac{k^2}{(k^2-1} = \frac{4}{3}\cdot\frac{9}{8}\cdot\frac{16}{15}\cdots\frac{10000}{9999}$
		  \part $\d\prod_{k=0}^n (2+3k) = (2)(5)(8)(11)(14)\cdots(2+3n)$
		\end{parts}
	
\end{ans}
\protect \end {itemize}
 \protect \noindent {\protect \textbf  {Solutions for Section 2.3}} \protect \begin {itemize} 
\begin{ans}{2.3.1.}
		\begin{parts}
		\part Hint: third differences are constant, so $a_n = an^3 + bn^2 + cn + d$.  Use the terms of the sequence to solve for $a, b, c,$ and $d$.  You should get $a_n = 1/6 (12+11 n+6 n^2+n^3)$,
		\part $a_n = n^2 - n$
		\end{parts}
	
\end{ans}
\begin{ans}{2.3.2.}
		The first differences are $2, 4, 6, 8, \ldots$, and the second differences are $2, 2, 2, \ldots$.  Thus the original sequence is $\Delta^2$-constant, so can be fit to a quadratic.

		Call the original sequence $a_n$.  Consider $a_n - n^2$. This gives $0, -1, -2, -3, \ldots$.  \emph{That} sequence has closed formula $1-n$ (starting at $n = 1$) so we have $a_n - n^2 = 1-n$ or equivalently $a_n = n^2 - n + 1$.
	
\end{ans}
\begin{ans}{2.3.3.}
	 This is a $\Delta^3$-constant sequence.  If we subtract off $n^3$, we are left with $1, 3, 7, 13, 21, \ldots$, the sequence from the previous question.  Thus here the closed formula is $n^3 + n^2 - n + 1$.
	
\end{ans}
\begin{ans}{2.3.4.}
		$a_{n-1} = (n-1)^2 + 3(n-1) + 4 = n^2 + n + 2$.  Thus $a_n - a_{n-1} = 2n+2$.  Note that this is linear (arithmetic).  We can check that we are correct.  The sequence $a_n$ is $4, 8, 14, 22, 32, \ldots$ and the sequence of differences is thus $4, 6, 8, 10,\ldots$ which agrees with $2n+2$ (if we start at $n = 1$).
	
\end{ans}
\begin{ans}{2.3.5.}
		$a_{n-1} = a(n-1)^2 + b(n-1) + c = an^2 - 2an + a + bn - b + c$.  Therefore $a_n - a_{n-1} = 2an - a + b$, which is arithmetic.  Notice that this is not quite the derivative of $a_n$, which would be $2an + b$, but it is close.
	
\end{ans}
\begin{ans}{2.3.6.}
		No.  The sequence of differences is the same as the original sequence so no differences will be constant.
	
\end{ans}
\begin{ans}{2.3.7.}
		No.  The sequence is geometric, and in fact has closed formula $2\cdot 3^n$.  This is an exponential function, which is not equal to any polynomial of any degree.  If the $n$th sequence of differences was constant, then the closed formula for the original sequence would be a degree $n$ polynomial.
	
\end{ans}
\protect \end {itemize}
 \protect \noindent {\protect \textbf  {Solutions for Section 2.4}} \protect \begin {itemize} 
\begin{ans}{2.4.1.}
		171 and 341.  $a_n = a_{n-1} + 2a_{n-2}$ with $a_0 = 3$ and $a_1 = 5$.  Closed formula: $a_n = \frac{8}{3}2^n + \frac{1}{3}(-1)^n$
	
\end{ans}
\begin{ans}{2.4.2.}
		By telescoping or iteration.  $a_n = 3 + 2^{n+1}$
	
\end{ans}
\begin{ans}{2.4.3.}
		We claim $a_n = 4^n$ works.  Plug it in: $4^n = 3(4^{n-1}) + 4(4^{n-2})$.  This works - just simplify the right hand side.
	
\end{ans}
\begin{ans}{2.4.4.}
		By the Characteristic Root Technique.  $a_n = 4^n + (-1)^n$.
	
\end{ans}
\begin{ans}{2.4.5.}
		$a_n = \frac{13}{5} 4^n + \frac{12}{5} (-1)^n$
	
\end{ans}
\begin{ans}{2.4.6.}
		The general solution is $a_n = a + bn$ where $a$ and $b$ depend on the initial conditions.  %Solve the recurrence relation $a_n = 2a_{n-1} - a_{n-2}$.
		\begin{parts}
		  \part $a_n = 1 + n$
		  %What is the solution if the initial terms are $a_0 = 1$ and $a_1 = 2$?
		  \part For example, we could have $a_0 = 21$ and $a_1 = 22$.  %What do the initial terms need to be in order for $a_9 = 30$?
		  \part For every $x$ - take $a_0 = x-9$ and $a_1 = x-8$.  %For which $x$ are there initial terms which make $a_9 = x$?
		\end{parts}
	
\end{ans}
\begin{ans}{2.4.7.}
		$a_n = \frac{19}{7}(-2)^n + \frac{9}{7}5^n$
		%Solve the recurrence relation $a_n = 3a_{n-1} + 10a_{n-2}$ with initial terms $a_0 = 4$ and $a_1 = 1$.
	
\end{ans}
\protect \end {itemize}
 \protect \noindent {\protect \textbf  {Solutions for Section 2.5}} \protect \begin {itemize} 
\begin{ans}{2.5.1.}
		\begin{proof}
		 We must prove that $1 + 2 + 2^2 + 2^3 + \cdots +2^n = 2^{n+1} - 1$ for all $n \in \N$.  Thus let $P(n)$ be the statement $1 + 2 + 2^2 + \cdots + 2^n = 2^{n+1} - 1$.  We will prove that $P(n)$ is true for all $n \in \N$.

		 First we establish the base case, $P(0)$, which claims that $1 = 2^{0+1} -1$.  Since $2^1 - 1 = 2 - 1 = 1$, we see that $P(0)$ is true.

		 Now for the inductive case.  Assume that $P(k)$ is true for an arbitrary $k \in \N$.  That is, $1 + 2 + 2^2 + \cdots + 2^k = 2^{k+1} - 1$.  We must show that $P(k+1)$ is true (i.e., that $1 + 2 + 2^2 + \cdots + 2^{k+1} = 2^{k+2} - 1$).  To do this, we start with the left hand side of $P(k+1)$ and work to the right hand side:
		 \begin{align*}
		  1 + 2 + 2^2 + \cdots + 2^k + 2^{k+1} = &~ 2^{k+1} - 1 + 2^{k+1} & \mbox{ \footnotesize by the inductive hypothesis}\\
		   = & ~2\cdot 2^{k+1} - 1 & \\
		   = &~ 2^{k+2} - 1 &
		 \end{align*}
		Thus $P(k+1)$ is true so by the principle of mathematical induction, $P(n)$ is true for all $n \in \N$.
		\end{proof}
	
\end{ans}
\begin{ans}{2.5.2.}
		\begin{proof}
		 Let $P(n)$ be the statement ``$7^n - 1$ is a multiple of 6.''  We will show $P(n)$ is true for all $n \in \N$.

		 First we establish the base case, $P(0)$.  Since $7^0 - 1 = 0$, and $0$ is a multiple of 6, $P(0)$ is true.

		 Now for the inductive case.  Assume $P(k)$ holds for an arbitrary $k \in \N$.  That is, $7^k - 1$ is a multiple of 6, or in other words, $7^k - 1 = 6j$ for some integer $j$.  Now consider $7^{k+1} - 1$:
		 \begin{align*}
		  7^{k+1} - 1 ~ & = 7^{k+1} - 7 + 6 & \mbox{ \footnotesize by cleverness: $-1 = -7 + 6$}\\
		  & = 7(7^k - 1) + 6 & \mbox{ \footnotesize factor out a 7 from the first two terms}\\
		  & = 7(6j) + 6 & \mbox{ \footnotesize by the inductive hypothesis}\\
		  & = 6(7j + 1) & \mbox{ \footnotesize factor out a 6}
		 \end{align*}
		Therefore $7^{k+1} - 1$ is a multiple of 6, or in other words, $P(k+1)$ is true.  Therefore by the principle of mathematical induction, $P(n)$ is true for all $n \in \N$.
		\end{proof}
	
\end{ans}
\begin{ans}{2.5.3.}
		\begin{proof}
		 Let $P(n)$ be the statement $1+3 +5 + \cdots + (2n-1) = n^2$.  We will prove that $P(n)$ is true for all $n \ge 1$.

		 First the base case, $P(1)$.  We have $ 1 = 1^2$ which is true, so $P(1)$ is established.

		 Now the inductive case.  Assume that $P(k)$ is true for some fixed arbitrary $k \ge 1$.  That is, $1 + 3 + 5 + \cdots + (2k-1) = k^2$.  We will now prove that $P(k+1)$ is also true (i.e., that $1 + 3 + 5 + \cdots + (2k+1) = (k+1)^2$).  We start with the left hand side of $P(k+1)$ and work to the right hand side:
		 \begin{align*}
		  1 + 3 + 5 + \cdots + (2k-1) + (2k+1) ~ & = k^2 + (2k+1) & \mbox{ \footnotesize by the induction hypothesis}\\
		  & = (k+1)^2 & \mbox{ \footnotesize by factoring}
		 \end{align*}
		Thus $P(k+1)$ holds, so by the principle of mathematical induction, $P(n)$ is true for all $n \ge 1$.
		\end{proof}
	
\end{ans}
\begin{ans}{2.5.4.}
		\begin{proof}
		 Let $P(n)$ be the statement $F_0 + F_2 + F_4 + \cdots + F_{2n} = F_{2n+1} - 1$.  We will show that $P(n)$ is true for all $n \ge 0$.  First the base case is easy because $F_0 = 0$ and $F_1 = 1$ so $F_0 = F_1 - 1$.  Now consider the inductive case.  Assume $P(k)$ is true, that is, assume $F_0 + F_2 + F_4 + \cdots + F_{2k} = F_{2k+1} - 1$.  To establish $P(k+1)$ we work from left to right:
		 \begin{align*}
		  F_0 + F_2 + F_4 + \cdots + F_{2k} + F_{2k+2} ~ & = F_{2k+1} - 1 + F_{2k+2} & \mbox{\footnotesize by the inductive hypothesis}\\
		  & = F_{2k+1} + F_{2k+2} - 1 & \\
		  & = F_{2k+3} - 1 & \mbox{\footnotesize by the recursive definition of the Fibonacci numbers}
		 \end{align*}
		Therefore $F_0 + F_2 + F_4 + \cdots + F_{2k+2} = F_{2k+3} - 1$, which is to say $P(k+1)$ holds.  Therefore by the principle of mathematical induction, $P(n)$ is true for all $n \ge 0$.
		\end{proof}
	
\end{ans}
\begin{ans}{2.5.5.}
		\begin{proof}
		 Let $P(n)$ be the statement $2^n < n!$.  We will show $P(n)$ is true for all $n \ge 4$.  First, we check the base case and see that yes, $2^4 < 4!$ (as $16 < 24$) so $P(4)$ is true.  Now for the inductive case.  Assume $P(k)$ is true for an arbitrary $k \ge 4$.  That is, $2^k < k!$.  Now consider $P(k+1)$: $2^{k+1} < (k+1)!$.  To prove this, we start with the left side and work to the right side.
		 \begin{align*}
		  2^{k+1}~ & = 2\cdot 2^k & \\
		  & < 2\cdot k! & \mbox{ \footnotesize by the inductive hypothesis}\\
		  & < (k+1) \cdot k! & \mbox{ \footnotesize since $k+1 > 2$}\\
		  & = (k+1)! &
		 \end{align*}
		Therefore $2^{k+1} < (k+1)!$ so we have established $P(k+1)$.  Thus by the principle of mathematical induction $P(n)$ is true for all $n \ge 4$.
		\end{proof}
	
\end{ans}
\begin{ans}{2.5.6.}
  		The only problem is that we never established the base case.  Of course, when $n = 0$, $0+3 \ne 0+7$.
  	
\end{ans}
\begin{ans}{2.5.7.}
		\begin{proof}
		    Let $P(n)$ be the statement that $n + 3 < n + 7$.  We will prove that $P(n)$ is true for all $n \in \N$.  First, note that the base case holds: $0+3 < 0+7$.  Now assume for induction that $P(k)$ is true.  That is, $k+3 < k+7$.  We must show that $P(k+1)$ is true.  Now since $k + 3 < k + 7$, add 1 to both sides.  This gives $k + 3 + 1 < k + 7 + 1$.  Regrouping $(k+1) + 3 < (k+1) + 7$.  But this is simply $P(k+1)$.  Thus by the principle of mathematical induction $P(n)$ is true for all $n \in \N$.
		\end{proof}
	
\end{ans}
\begin{ans}{2.5.8.}
 		The problem here is that while $P(0)$ is true, and while $P(k) \imp P(k+1)$ for {\em some} values of $k$, there is at least one value of $k$ (namely $k = 99$) when that implication fails.  For a valid proof by induction, $P(k) \imp P(k+1)$ must be true for all values of $k$ greater than or equal to the base case.
 	
\end{ans}
\begin{ans}{2.5.9.}
		\begin{proof}
		 Let $P(n)$ be the statement ``there is a strictly increasing sequence $a_1, a_2, a_3, \ldots, a_n$ with $a_n < 100$.''  We will prove $P(n)$ is true for all $n \ge 1$. First we establish the base case: $P(1)$ says there is a single number $a_1$ with $a_1 < 100$.  This is true - take $a_1 = 0$.  Now for the inductive step, assume $P(k)$ is true.  That is there exists a strictly increasing sequence $a_1, a_2, a_3, \ldots, a_k$ with $a_k < 100$.  Now consider this sequence, plus one more term, $a_{k+1}$ which is greater than $a_k$ but less than $100$.  Such a number exists - for example, the average between $a_k$ and 100.  So then $P(k+1)$ is true, so we have shown that $P(k) \imp P(k+1)$.  Thus by the principle of mathematical induction, $P(n)$ is true for all $n \in \N$.
		\end{proof}

	
\end{ans}
\begin{ans}{2.5.10.}
  		We once again failed to establish the base case: when $n = 0$, $n^2 + n = 0$ which is even, not odd.
  	
\end{ans}
\begin{ans}{2.5.11.}
		  \begin{proof}
		    Let $P(n)$ be the statement ``$n^2 + n$ is even.''  We will prove that $P(n)$ is true for all $n \in \N$.  First the base case: when $n = 0$, we have $n^2 + n = 0$ which is even, so $P(0)$ is true.  Now suppose for induction that $P(k)$ is true, that is, that $k^2 + k$ is even.  Now consider the statement $P(k+1)$.  Now $(k+1)^2 + (k+1) = k^2 + 2k + 1 + k + 1 = k^2 + k + 2k + 2$.  By the inductive hypothesis, $k^2 + k$ is even, and of course $2k + 2$ is even.  An even plus an even is always even, so therefore $(k+1)^2 + (k+1)$ is even.  Therefore by the principle of mathematical induction, $P(n)$ is true for all $n \in \N$.
		  \end{proof}
	
\end{ans}
\begin{ans}{2.5.12.}
		 Further hint: the idea is to define the sequence so that $a_n$ is less than the distance between the previous partial sum and 2.  That way when you add it into the next partial sum, the partial sum is still less than 2.  You could do this ahead of time, or use a clever $P(n)$ in the induction proof.  Let $P(n)$ be the statement, ``there is a sequence of positive real numbers $a_1, a_2, a_3, \ldots, a_n$ such that $a_1 + a_2 + a_3 + \cdots + a_n < 2$.''  The base case should be easy (just pick $a_1 < 2$).  For the inductive case, you know that $a_1 + a_2 + \cdots + a_k < 2$ so you just need to argue that you can find some $a_{k+1}$ small enough to have $a_1 + a_2 + \cdots +a_k + a_{k+1} < 2$.
	
\end{ans}
\begin{ans}{2.5.13.}
		The base case should be easy - 0 is a power of 2.  For the inductive case, you actually want to use strong induction.  Suppose $k$ is either a power of 2 or can be written as the sum of distinct powers of 2, for any $k < n$.  Now if $n$ is a power of 2, we are done.  If not, subtract the largest power of 2 from $n$ possible.  You get $n - 2^x$, which is a smaller number, in fact smaller than both $n$ and $2^x$.  Thus $n-2^x$ is either a power of 2 or can be written as the sum of distinct powers of 2, but none of them are going to be $2^x$, so the together with $2^x$ we have written $n$ as the sum of distinct powers of 2.
	
\end{ans}
\begin{ans}{2.5.14.}
	  If $n = 2$, this should work out (so their's your base case).  If we assume it works for $k$ people (that the number of handshakes is $\frac{k(k-1)}{2}$, what happens if a $k+1$st person shows up.  How many {\em new} handshakes take place?  Now make this into a formal induction argument.

	  Note, we have already proven this without using induction, but this is fun too.
	
\end{ans}
\begin{ans}{2.5.15.}
		When $n = 0$, we get $x^0 +\frac{1}{x^0} = 2$ and when $n = 1$, $x + \frac{1}{x}$ is an integer, so the base case holds.  Now assume the result holds for all natural numbers $n < k$.  In particular, we know that $x^{k-1} + \frac{1}{x^{k-1}}$ and $x + \frac{1}{x}$ are both integers.  Thus their product is also an integer.  But,
		\begin{align*}
		\left(x^{k-1} + \frac{1}{x^{k-1}}\right)\left(x + \frac{1}{x}\right) & = x^k + \frac{x^{k-1}}{x} + \frac{x}{x^{k-1}} + \frac{1}{x^k}\\
		& = x^k + \frac{1}{x^k} + x^{k-2} + \frac{1}{x^{k-2}}
		\end{align*}
		Note also that $x^{k-2} + \frac{1}{x^{k-2}}$ is an integer by the induction hypothesis, so we can conclude that $x^k + \frac{1}{x^k}$ is an integer.


	
\end{ans}
\begin{ans}{2.5.16.}
		Here's the idea: since every entry in Pascal's Triangle is the sum of the two entries above it, we can get the $k+1$st row by adding up all the pairs of entry from the $k$th row.  But doing this uses each entry on the $k$th row twice.  Thus each time we drop to the next row, we double the total.  Of course, row 0 has sum $1 = 2^0$ (the base case).  Now try to make this precise with a formal induction proof.  You will use the fact that ${n \choose k} = {n-1 \choose k-1} + {n-1 \choose k}$ for the inductive case.
	
\end{ans}
\begin{ans}{2.5.17.}
		To see why this works, try it on a copy of Pascal's triangle.  We are adding up the entries along a diagonal, starting with the 1 on the left hand side of the 4th row.  Suppose we add up the first 5 entries on this diagonal.  The claim is that the sum is the entry below and to the left of the last of these 5 entries.  Note that if this is true, and we instead add up the first 6 entries, we will need to add the entry one spot to the right of the previous sum.  But these two together give the entry below them, which is below and left of the last of the 6 entries on the diagonal.

		If you follow that, you can see what is going on.  But it is not a great proof.  A formal induction proof is needed:

		\begin{proof}
			Let $P(n)$ be the statement ${4 \choose 0} + {5 \choose 1} + {6 \choose 2} + \cdots + {4+n \choose n} = {5+n \choose n}$.  For the base case, consider $n = 0$.  This says ${4 \choose 0} = {5 \choose 0}$.  Since these are both 1, the base case is true.  Now for the inductive case, suppose $P(k)$ is true.  That is, ${4 \choose 0} + {5 \choose 1} + {6 \choose 2} + \cdots + {4+k \choose k} = {5+k \choose k}$.  If we add ${4+k+1 \choose k+1}$ to both sides, we get \[{4 \choose 0} + {5 \choose 1} + {6 \choose 2} + \cdots + {4+k \choose k} + {5+k \choose k+1}= {5+k \choose k} + {5+k \choose k+1}\]
			But ${5+k \choose k} + {5+k \choose k+1} = {5+k+1 \choose k+1}$.  In other words, we have
			\[{4 \choose 0} + {5 \choose 1} + {6 \choose 2} + \cdots + {4+k \choose k} + {5+k \choose k+1} = {5+k+1 \choose k+1}\]
			which is to say that $P(k+1)$ is true.

			Therefore, by the principle of mathematical induction, $P(n)$ is true for all $n \ge 0$.
		\end{proof}
	
\end{ans}
\begin{ans}{2.5.18.}
		The idea here is that if we take the logarithm of $a^n$, we can increase $n$ by 1 if we multiply by another $a$ (inside the logarithm).  This results in adding 1 more $\log(a)$ to the total.

		\begin{proof}
			Let $P(n)$ be the statement $\log(a^n) = n \log(a)$.  The base case, $P(2)$ is true, because $\log(a^2) = \log(a\cdot a) = \log(a) + \log(a) = 2\log(a)$, by the product rule for logarithms.

			Now assume, for induction, that $P(k)$ is true.  That is, $\log(a^k) = k\log(a)$.  Consider $\log(a^{k+1})$.  We have
			\[\log(a^{k+1}) = \log(a^k\cdot a) = \log(a^k) + \log(a) = k\log(a) + \log(a)\]
			with the last equality due to the inductive hypothesis.  But this simplifies to $(k+1) \log(a)$, establishing $P(k+1)$.

			Therefore by the principle of mathematical induction, $P(n)$ is true for all $n \ge 2$.
		\end{proof}
	
\end{ans}
\begin{ans}{2.5.19.}
		Hint: You are allowed to assume the base case.  For the inductive case, group all but the last function together as one sum of functions, then apply the usual sum of derivatives rule, and then the inductive hypothesis.
	
\end{ans}
\begin{ans}{2.5.20.}
		Hint: for the inductive step, we know by the product rule for two functions that \[(f_1f_2f_3 \cdots f_k f_{k+1})' = (f_1f_2f_3\cdots f_k)'f_{k+1} + (f_1f_2f_3\cdots f_k)f_{k+1}'\]
		Then use the inductive hypothesis on the first summand, and distribute.
	
\end{ans}
\protect \end {itemize}
 \protect \noindent {\protect \textbf  {Solutions for Section 2.6}} \protect \begin {itemize} 
\begin{ans}{2.6.1.}
   $\frac{430\cdot 107}{2} = 23005$
  
\end{ans}
\begin{ans}{2.6.2.}
		\begin{parts}
		\part 36.  %How many terms (summands) are in the sum?
		\part $\frac{253 \cdot 36}{2} = 4554$.  %Compute the sum.  Remember to show all your work.
		\end{parts}
	
\end{ans}
\begin{ans}{2.6.3.}
	 \begin{parts}
	 \part $n+2$ terms
	 \part $4n+2$
	 \part $\frac{(4n+8)(n+2)}{2}$
	 \end{parts}
  
\end{ans}
\begin{ans}{2.6.4.}
		\begin{parts}
		  \part $a_n = a_{n-1} + 4$ with $a_1 = 5$.  %Give a recursive definition for the sequence.
		  \part $a_n = 5 + 4(n-1)$  %Give a closed formula for the $n$th term of the sequence.
		  \part Yes, since $2013 = 5 + 4(503-1)$ (so $a_{503} = 2013$).
		  \part 133 %How many terms does the sequence $5, 9, 13, 17, 21, \ldots, 533$ have?
		  \part $\frac{538\cdot 133}{2} = 35777$  %Find the sum: $5 + 9 + 13 + 17 + 21 + \cdots + 533$.  Show your work.
		  \part $b_n = 1 + \frac{(4n+1)n}{2}$.
		\end{parts}
	
\end{ans}
\begin{ans}{2.6.5.}
		\begin{parts}
		\part $2, 10, 50, 250, \ldots$  The sequence is geometric. %Find the first 4 terms of the sequence.  What sort of sequence is this?
		\part $\frac{2 - 2\cdot 5^{25}}{-4}$.  %Find the {\em sum} of the first 25 terms.  That is, compute $\d\sum_{k=1}^{25}a_k$.
		\end{parts}
	
\end{ans}
\begin{ans}{2.6.6.}
		$a_n = n^2 + 4n - 1$
	
\end{ans}
\begin{ans}{2.6.7.}
	 \begin{parts}
	  \part The sequence of partial sums will be a degree 4 polynomial (its sequence of differences will be the original sequence).
	  \part The sequence of second differences will be a degree 1 polynomial - an arithmetic sequence.
	 \end{parts}
  
\end{ans}
\begin{ans}{2.6.8.}
	 	\begin{parts}
	 	\part $4, 6, 10, 16, 26, 42, \ldots$  %Write out the first 6 terms of the sequence.
	 	\part No, taking differences gives the original sequence back, so the differences will never be constant.  %Could the closed formula for $a_n$ be a polynomial?  Explain.
	 	\end{parts}
	
\end{ans}
\begin{ans}{2.6.9.}
		 $b_n = (n+3)n$
	
\end{ans}
\begin{ans}{2.6.10.}
		\begin{parts}
		 \part $1, 2, 16,68, 364, \ldots$  %Write out the first 5 terms of the sequence defined by this recurrence relation.
		 \part $a_n = \frac{3}{7}(-2)^n + \frac{4}{7}5^n$  %Solve the recurrence relation.
		\end{parts}
	
\end{ans}
\begin{ans}{2.6.11.}
		\begin{parts}
		  \part $a_2 = 14$.  $a_3 = 52$  %Find the next two terms of the sequence ($a_2$ and $a_3$).
		  \part $a_n = \frac{1}{6}(-2)^n + \frac{5}{6}4^n$  %Solve the recurrence relation.   That is, find a closed formula for the $n$th term of the sequence.
%		  \part $\frac{1+x}{1-2x-8x^2}$  %Find the generating function for the sequence.  Hint: use the recurrence relation.
		\end{parts}
	
\end{ans}
\begin{ans}{2.6.12.}
	 \begin{parts}
	 	\part On the first day, your 2 mini bunnies become 2 large bunnies.  On day 2, your two large bunnies produce 4 mini bunnies.  On day 3, you have 4 mini bunnies (produced by your 2 large bunnies) plus 6 large bunnies (your original 2 plus the 4 newly matured bunnies).  On day 4, you will have $12$ mini bunnies (2 for each of the 6 large bunnies) plus 10 large bunnies (your previous 6 plus the 4 newly matured).  The sequence of total bunnies is $2, 2, 6, 10, 22, 42\ldots$ starting with $a_0 = 2$ and $a_1 = 2$.
	 	\part $a_n = a_{n-1} + 2a_{n-2}$.  This is because the number of bunnies is equal to the number of bunnies you had the previous day (both mini and large) plus 2 times the number you had the day before that (since all bunnies you had 2 days ago are now large and producing 2 new bunnies each).
	 	\part Using the characteristic root technique, we find $a_n = a2^n + b(-1)^n$, and we can find $a$ and $b$ to give $a_n = \frac{4}{3}2^n + \frac{2}{3}(-1)^n$.
	 \end{parts}
	
\end{ans}
\begin{ans}{2.6.13.}
		\begin{parts}
		 \part Hint: $(n+1)^{n+1} > (n+1) \cdot n^{n}$.
		 \part Hint: This should be similar to the other sum proofs.  The last bit comes down to adding fractions.
		 \part Hint: Write $4^{k+1} - 1 = 4\cdot 4^k - 4 + 3$.
%		 \part Hint: Use the fact $F_{2n} + F_{2n+1} = F_{2n+2}$
		 \part Hint: one 9-cent stamp is 1 more than two 4-cent stamps, and seven 4-cent stamps is 1 more than three 9-cent stamps.
		 \part Careful to actually use induction here.  The base case: $2^2 = 4$.  The inductive case: assume $(2n)^2$ is divisible by 4 and consider $(2n+2)^2 = (2n)^2 + 4n + 4$.  This is divisible by 4 because $4n +4$ clearly is, and by our inductive hypothesis, so is $(2n)^2$.
		\end{parts}
	
\end{ans}
\begin{ans}{2.6.14.}
		Hint: This is a straight forward induction proof.  Note you will need to simplify $\left(\frac{n(n+1)}{2}\right)^2 + (n+1)^3$ and get $\left(\frac{(n+1)(n+2)}{2}\right)^2$.
	
\end{ans}
\begin{ans}{2.6.15.}
		Hint: there are two base cases $P(0)$ and $P(1)$.  Then, for the inductive case, assume $P(k)$ is true for all $k < n$.  This allows you to assume $a_{n-1} = 1$ and $a_{n-2} = 1$.  Apply the recurrence relation.
	
\end{ans}
\begin{ans}{2.6.16.}
		Note that $1 = 2^0$ - this is your base case.  Now suppose $k$ can be written as the sum of distinct powers of 2 for all $1\le k \le n$.  We can then write $n$ as the sum of distinct powers of 2 as follows: subtract the largest power of 2 less than $n$ from $n$.  That is, write $n = 2^j + k$ for the largest possible $j$.  But $k$ is now less than $n$, and also less than $2^j$, so write $k$ as the sum of distinct powers of 2 (we can do so by the inductive hypothesis).  Thus $n$ can be written as the sum of distinct powers of 2 for all $n \ge 1$.
	
\end{ans}
\begin{ans}{2.6.17.}
		Let $P(n)$ be the statement, ``every set containing $n$ elements has $2^n$ different subsets.''  We will show $P(n)$ is true for all $n \ge 1$.

		\underline{Base case}: Any set with 1 element $\{a\}$ has exactly 2 subsets: the empty set and the set itself.  Thus the number of subsets is $2= 2^1$.  Thus $P(1)$ is true.

		\underline{Inductive case}: Suppose $P(k)$ is true for some arbitrary $k \ge 1$.  Thus every set containing exactly $k$ elements has $2^k$ different subsets.  Now consider a set containing $k+1$ elements: $A = \{a_1, a_2, \ldots, a_k, a_{k+1}\}$.  Any subset of $A$ must either contain $a_{k+1}$ or not.  In other words, a subset of $A$ is just a subset of $\{a_1, a_2,\ldots, a_k\}$ with or without $a_{k+1}$.  Thus there are $2^k$ subsets of $A$ which contain $a_{k+1}$ and another $2^{k+1}$ subsets of $A$ which do not contain $a^{k+1}$.  This gives a total of $2^k + 2^k = 2\cdot 2^k = 2^{k+1}$ subsets of $A$.  But our choice of $A$ was arbitrary, so this works for any subset containing $k+1$ elements, so $P(k+1)$ is true.

		Therefore, by the principle of mathematical induction, $P(n)$ is true for all $n \ge 1$.
	
\end{ans}
\protect \end {itemize}
 \protect \noindent {\protect \textbf  {Solutions for Section 3.1}} \protect \begin {itemize} 
\begin{ans}{3.1.1.}
    \begin{parts}
      \part $P$: it's your birthday; $Q$: there will be cake.  $(P \vee Q) \imp Q$
      \part Hint: you should get three T's and one F.
      \part Only that there will be cake.
      \part It's NOT your birthday!
      \part It's your birthday, but the cake is a lie.
    \end{parts}
  
\end{ans}
\begin{ans}{3.1.2.}
    \begin{parts}
      \part $P \wedge Q$
      \part $P \imp \neg Q$
      \part Jack passed math or Jill passed math (or both).
      \part If Jack and Jill did not both pass math, then Jill did.
      \part
	\begin{subparts}
	  \subpart Nothing else.
	  \subpart  Jack did not pass math either.
	\end{subparts}
    \end{parts}
  
\end{ans}
\begin{ans}{3.1.3.}
    \begin{parts}
	\part Three statements: $P \vee S$, $S \imp Q$, $(P \vee Q) \imp R$.  You could also connect the first two with a $\wedge$.
	\part He cannot be lying about all three sentences, so he is telling the truth.
	\part No matter what, Geoff wants ricotta.  If he doesn't have quail, then he must have pepperoni but not sausage.
    \end{parts}
  
\end{ans}
\begin{ans}{3.1.4.}
 \begin{tabular}{c|c|c}
             $P$ & $Q$ & $(P \vee Q) \imp (P \wedge Q)$\\ \hline
             T & T & T \\
             T & F & F \\
             F & T & F \\
             F & F & T
          \end{tabular}
\end{ans}
\begin{ans}{3.1.5.}
      \begin{tabular}{c|c|c}
             $P$ & $Q$ & $\neg P \wedge (Q \imp P)$\\ \hline
             T & T & F \\
             T & F & F \\
             F & T & F \\
             F & F & T
          \end{tabular}
	If the statement is true, then both $P$ and $Q$ are false.
    
\end{ans}
\begin{ans}{3.1.6.}
    Hint: Like above, only now you will need 8 rows instead of just 4.
  
\end{ans}
\begin{ans}{3.1.7.}
    The argument is valid.  To see this, make a truth table which contains $P \vee Q$ and $\neg P$ (and $P$ and $Q$ of course).  Look at the truth value of $Q$ in each of the rows that have $P \vee Q$ and $\neg P$ true.
  
\end{ans}
\begin{ans}{3.1.8.}
    The argument form is valid.  Again, make a truth table containing the premises and conclusion - look at the rows for which the premises are true.
  
\end{ans}
\begin{ans}{3.1.9.}
    The argument is NOT valid.  If you make a truth table containing the premises and conclusion, there will be a row with both premises true but the conclusion false.  For example, if $P$ and $Q$ are false and $R$ is true, then $P \wedge Q$ is false, so $(P \wedge Q) \imp R$ is true.  Also $\neg P$ is true, so $\neg P \vee \neg Q$ is true.  However, $\neg R$ is false.
  
\end{ans}
\protect \end {itemize}
 \protect \noindent {\protect \textbf  {Solutions for Section 3.2}} \protect \begin {itemize} 
\begin{ans}{3.2.1.}
    Make a truth table for each and compare.  The statements are logically equivalent.
  
\end{ans}
\begin{ans}{3.2.2.}
    Again, make two truth tables.  The statements are logically equivalent.
  
\end{ans}
\begin{ans}{3.2.3.}
    \begin{parts}
      \part If Oscar drinks milk, then he eats Chinese food.
      \part If Oscar does not drink milk, then he does not eat Chinese food.
      \part Yes.  The original statement would be false too.
      \part Nothing. The converse need not be true.
      \part He does not eat Chinese food. The contrapositive would be true.
    \end{parts}
  
\end{ans}
\begin{ans}{3.2.4.}
    \begin{parts}
      \part $P \wedge Q$
      \part $(P \vee Q) \vee (Q \wedge \neg R)$
      \part F.  Or $(P \wedge Q) \wedge (R \wedge \neg R)$
      \part Either Sam is a woman and Chris is a man, or Chris is a woman.
    \end{parts}
  
\end{ans}
\begin{ans}{3.2.5.}
 The statements are equivalent to the\ldots
    \begin{parts}
      \part converse.
      \part implication.
      \part neither.
      \part implication.
      \part converse.
      \part converse.

      \part implication.
      \part converse.
      \part converse.
      \part converse (in fact, this IS the converse).
      \part implication (the statement is the contrapositive of the implication).
      \part neither.
    \end{parts}
  
\end{ans}
\begin{ans}{3.2.6.}
    Hint: of course there are many answers.  It helps to assume that the statement is true and the converse is NOT true.  Think about what that means in the real world and then start saying it in different ways.  Some ideas: use necessary and sufficient language, use ``only if,'' consider negations, use ``or else'' language.
  
\end{ans}
\protect \end {itemize}
 \protect \noindent {\protect \textbf  {Solutions for Section 3.3}} \protect \begin {itemize} 
\begin{ans}{3.3.1.}
     \begin{parts}
	\part $\neg \exists x (E(x) \wedge O(x))$
	\part $\forall x (E(x) \imp O(x+1))$
	\part $\exists x(P(x) \wedge E(x))$ (where $P(x)$ means ``$x$ is prime'')
	\part $\forall x \forall y \exists z(x < z < y \vee y < z < x)$
	\part $\forall x \neg \exists y (x < y < x+1)$
    \end{parts}
  
\end{ans}
\begin{ans}{3.3.2.}
    \begin{parts}
	\part Any even number plus 2 is an even number.
	\part For any $x$ there is a $y$ such that $\sin(x) = y$.  In other words, every number $x$ is in the domain of sine.
	\part For every $y$ there is an $x$ such that $\sin(x) = y$.  In other words, every number $y$ is in the range of sine (which is false).
	\part For any numbers, if the cubes of two numbers are equal, then the numbers are equal.
      \end{parts}
  
\end{ans}
\begin{ans}{3.3.3.}
    \begin{parts}
	\part $\forall x \exists y (O(x) \wedge \neg E(y))$
	\part $\exists x \forall y (x \ge y \vee \forall z (x \ge z \wedge y \ge z))$
	\part There is a number $n$ for which every other number is strictly greater than $n$.
	\part There is a number $n$ which is not between any other two numbers.
      \end{parts}
  
\end{ans}
\begin{ans}{3.3.4.}
	  If $P(x)$ is true of every $x$, then in particular it is true of $x = 0$ (or any fixed element of the universe).  So then there is definitely some $x$ (namely 0) for which $P(x)$ holds.  Thus $\forall x P(x) \imp \exists x P(x)$ is always true.

	  The converse is not always true though.  Consider the predicate $x = 5$.  So $P(x)$ is true if and only if $x = 5$.  Certainly it is true that $\exists x P(x)$ (since we can take $x = 5$), but false that $\forall x P(x)$.
	
\end{ans}
\begin{ans}{3.3.5.}
		\begin{parts}
			\part This says that everything has a square root (every element is the square of something).  This is true of the positive real numbers, and also of the complex numbers.  It is false of the natural numbers though, as for $x = 2$ there is no natural number $y$ such that $y^2 = 2$.
			\part This asserts that between every pair of numbers there is some number strictly between them.  This is true of the rationals (and reals) but false of the integers.  If $x = 1$ and $y = 2$, then there is nothing we can take for $z$.
			\part Here we are saying that something is between every pair of numbers.  For almost every universe, this is false.  In fact, if the universe contains $\{1,2,3, 4\}$, then no matter what we take $x$ to be, there will be a pair that $x$ is NOT between.  However, the set $\{1,2,3\}$ as our universe makes the statement true.  Let $x = 2$.  Then no matter what $y$ and $z$ we pick, if $y < z$, then 2 is between them.
		\end{parts}
	
\end{ans}
\begin{ans}{3.3.6.}
		Let $P(x,y)$ be the predicate $x < y$.  It is true that for all $x$ there is some $y$ greater than it (since there are infinitely many numbers).  However, there is not a natural number $y$ which is greater than every number $x$.

		We cannot do the reverse of this though.  If there is some $y$ for which every $x$ satisfies $P(x,y)$, then certainly for every $x$ there is some $y$ which satisfies $P(x,y)$.  The first is saying we can find one $y$ that works for every $x$.  The second allows different $y$'s to work for different $x$'s, but there is nothing preventing us from using the same $y$ that work for every $x$.
	
\end{ans}
\protect \end {itemize}
 \protect \noindent {\protect \textbf  {Solutions for Section 3.4}} \protect \begin {itemize} 
\begin{ans}{3.4.1.}
     \begin{parts}
 	\part For all integers $a$ and $b$, if $a$ or $b$ are not even, then $a+b$ is not even.
 	\part For all integers $a$ and $b$, if $a$ and $b$ are even, then $a+b$ is even.
 	\part There are numbers $a$ and $b$ such that $a+b$ is even but $a$ and $b$ are not both even.
 	\part False.  For example, $a = 3$ and $b = 5$.  $a+b = 8$, but neither $a$ nor $b$ are even.
 	\part False, since it is equivalent to the original statement.
 	\part True.  Let $a$ and $b$ be integers.  Assume both are even.  Then $a = 2k$ and $b = 2j$ for some integers $k$ and $j$.  But then $a+b = 2k + 2j = 2(k+j)$ which is even.
 	\part True, since the statement is false.
       \end{parts}
   
\end{ans}
\begin{ans}{3.4.2.}
	\begin{parts}
	  \part Direct proof.
	  \begin{proof}
	    Let $n$ be an integer.  Assume $n$ is even.  Then $n = 2k$ for some integer $k$.  Thus $8n = 16k = 2(8k)$.  Therefore $8n$ is even.
	  \end{proof}

	  \part The converse is false.  That is, there is an integer $n$ such that $8n$ is even but $n$ is odd.  For example, consider $n = 3$.  Then $8n = 24$ which is even but $n = 3$ is odd.
	\end{parts}
	
\end{ans}
\begin{ans}{3.4.3.}
     \begin{proof}
      Suppose $\sqrt{3}$ were rational.  Then $\sqrt{3} = \frac{a}{b}$ for some integers $a$ and $b \ne 0$.  Without loss of generality, assume $\frac{a}{b}$ is reduced.  Now
 \[3 = \frac{a^2}{b^2}\]
 \[b^2 3 = a^2\]
 So $a^2$ is a multiple of 3.  This can only happen if $a$ is a multiple of 3, so $a = 3k$ for some integer $k$.  Then we have
 \[b^2 3 = 9k^2\]
 \[b^2 = 3k^2\]
 So $b^2$ is a multiple of 3, making $b$ a multiple of 3 as well.  But this contradicts our assumption that $\frac{a}{b}$ is in lowest terms.
     \end{proof}
   
\end{ans}
\begin{ans}{3.4.4.}
	\begin{parts}
	  \part Direct proof.
	  \begin{proof}
	    Let $a$ and $b$ be integers.  Assume $a$ is even and $b$ is a multiple of 3.  Then $a = 2k$ and $b = 3j$ for some integers $k$ and $j$.  Now
	    \[ab = (2k)(3j) = 6(kj)\]
	    Since $kj$ is an integer, we have that $ab$ is a multiple of 6.
	  \end{proof}

	  \part The converse is: for all integers $a$ and $b$, if $ab$ is a multiple of 6, then $a$ is even and $b$ is a multiple of 3.  This is false.  Consider $a = 3$ and $b = 10$.  Then $ab = 30$ which is a multiple of 6, but $a$ is not even and $b$ is not divisible by 3.
	\end{parts}
	
\end{ans}
\begin{ans}{3.4.5.}
	We will prove the contrapositive: if $n$ is even, then $5n$ is even.
	  \begin{proof}
	    Let $n$ be an arbitrary integer, and suppose $n$ is even.  Then $n = 2k$ for some integer $k$.  Thus $5n = 5\cdot 2k = 10k = 2(5k)$.  Since $5k$ is an integer, we see that $5n$ must be even.  This completes the proof.
	  \end{proof}

	
\end{ans}
\begin{ans}{3.4.6.}
	  \begin{proof}
	    Suppose, contrary to stipulation, that there are integers $a$, $b$ and $c$ such that $a^2 + b^2 = c^2$ but $a$ and $b$ are both odd.  Then $a = 2k+1$ and $b = 2j + 1$ for some integers $k$ and $j$.  We then have
	    \[a^2 + b^2 = (2k+1)^2 + (2j+1)^2 = 4k^2 + 4k + 1 + 4j^2 + 4j + 1 = 4(k^2 + j^2 + k + j) + 2\]
	    So $c^2 = 4(k^2 + j^2 + k + j) + 2$.  This means that $c^2$ is even, which can only happen if $c$ is even.  But then $c^2$ must be a multiple of 4.  However, this is a contradiction because $4(k^2 + j^2 + k + j) + 2$ is not a multiple of 4.
	  \end{proof}

	
\end{ans}
\begin{ans}{3.4.7.}
	This is an example of the pigeonhole principle.  We can prove it by contrapositive.

	\begin{proof}
	Suppose that each number only came up 6 or fewer times.  So there are at most six 1's, six 2's, and so on.  That's a total of 36 dice, so you must not have rolled all 40 dice.
	\end{proof}
	
\end{ans}
\begin{ans}{3.4.8.}
		We can have 9 dice without any four matching or all being different: three 1's, three 2's, three 3's.  We will prove that whenever you roll 10 dice, you will always get four matching or all being different.
		\begin{proof}
			Suppose you roll 10 dice, but that there are NOT four matching rolls.  This means at most, there are three of any given value.  If we only had three different values, that would be only 9 dice, so there must be 4 different values, giving 4 dice that are all different.
		\end{proof}
	
\end{ans}
\begin{ans}{3.4.9.}
	 We give a proof by contradiction.
	\begin{proof}
	  Suppose, contrary to stipulation that $\log(7)$ is rational.  Then $\log(7) = \frac{a}{b}$ with $a$ and $b \ne 0$ integers.  By properties of logarithms, this implies
	  \[7 = 10^{\frac{a}{b}}\]
	  Equivalently,
	  \[7^b = 10^a\]
	  But this is impossible as any power of 7 will be odd while any power of 10 will be even.
	\end{proof}
	
\end{ans}
\begin{ans}{3.4.10.}
		\begin{proof}
		  Suppose there were integers $x$ and $y$ such that $x^2 = 4y + 3$.  Now $x^2$ must be odd, since $4y + 3$ is odd.  Since $x^2$ is odd, we know $x$ must be odd as well.  So $x = 2k + 1$ for some integer $k$.  Then $x^2 = 4k^2 + 4k + 1 = 4(k^2 + k) + 1$.  Therefore we have,
		  \[4(k^2 + k) + 1 = 4y + 3\]
		  which implies
		  \[4(k^2 + k) = 4y + 2\]
		  and therefore
		  \[2(k^2 + k) = 2y + 1.\]
		  But this is a contradiction -- the left hand side is even while the right hand side is odd.
		\end{proof}

	
\end{ans}
\begin{ans}{3.4.11.}
		\begin{parts}
		% There are no integers $x$ and $y$ such that $x$ is a prime greater than 5 and $x = 6y + 3$.
		 \part Proof by contradiction.  Start of proof: Assume, for the sake of contradiction, that there are integers $x$ and $y$ such that $x$ is a prime greater than 5 and $x = 6y + 3$.  End of proof: \ldots this is a contradiction, so there are no such integers.
		%  For all integers $n$, if $n$ is a multiple of 3, then $n$ can be written as the sum of consecutive integers.
		 \part Direct proof.  Start of proof: Let $n$ be an integers.  Assume $n$ is a multiple of 3.  End of proof: Therefore $n$ can be written as the sum of consecutive integers.
		%  For all integers $a$ and $b$, if $a^2 + b^2$ is odd, then $a$ or $b$ is odd.
		 \part Proof by contrapositive.  Start of proof: Let $a$ and $b$ be integers.  Assume that $a$ and $b$ are even.  End of proof: Therefore $a^2 + b^2$ is even.
		\end{parts}
	
\end{ans}
\protect \end {itemize}
 \protect \noindent {\protect \textbf  {Solutions for Section 3.5}} \protect \begin {itemize} 
\begin{ans}{3.5.1.}
    \begin{tabular}{c|c|c||c}
                     $P$&$Q$&$R$& $\neg P \imp (Q \wedge R)$ \\ \hline
                     T & T & T & T\\
                     T & T & F & T\\
                     T & F & T & T\\
                     T & F & F & T \\
                     F & T & T & T\\
                     F & T & F & F\\
                     F & F & T & F\\
                     F & F & F & F
                    \end{tabular}
  
\end{ans}
\begin{ans}{3.5.2.}
    Peter is not tall and Robert is not skinny.  You must be in row 6 in the truth table above.
  
\end{ans}
\begin{ans}{3.5.3.}
    Yes.  To see this, make a truth table for each statement and compare.
  
\end{ans}
\begin{ans}{3.5.4.}
    Make a truth table that includes all three statements in the argument:

    \begin{tabular}{c|c|c||c|c|c}
     $P$ & $Q$ & $R$ & $P \imp Q$ & $P \imp R$ & $P \imp (Q \wedge R)$ \\ \hline
      T  &  T  &  T  &      T     &      T     &   T \\
      T  &  T  &  F  &      T     &      F     &   F \\
      T  &  F  &  T  &      F     &      T     &   F \\
      T  &  F  &  F  &      F     &      F     &   F \\
      F  &  T  &  T  &      T     &      T     &   T \\
      F  &  T  &  F  &      T     &      T     &   T \\
      F  &  F  &  T  &      T     &      T     &   T \\
      F  &  F  &  F  &      T     &      T     &   T
    \end{tabular}

  Notice that in every row for which both $P \imp Q$ and $P \imp R$ is true, so is $P \imp (Q \wedge R)$.  Therefore, whenever the premises of the argument are true, so is the conclusion.  In other words, the deduction rule is valid.
  
\end{ans}
\begin{ans}{3.5.5.}
	\begin{parts}
	  \part  Negation: The power goes off and the food does not spoil.\\
	  Converse: If the food spoils, then the power went off.\\
	  Contrapositive: If the food does not spoil, then the power did not go off.

	  \part   Negation: The door is closed and the light is on.\\
	  Converse: If the light is off then the door is closed.\\
	  Contrapositive: If the light is on then the door is open.
	  \part
	    Negation: $\exists x (x < 1 \wedge x^2 \ge 1)$\\
	  Converse: $\forall x( x^2 < 1 \imp x < 1)$\\
	  Contrapositive: $\forall x (x^2 \ge 1 \imp x \ge 1)$.
	  \part Negation: There is a natural number $n$ which is prime but not solitary.\\
	  Converse: For all natural numbers $n$, if $n$ is solitary, then $n$ is prime.\\
	  Contrapositive: For all natural numbers $n$, if $n$ is not solitary then $n$ is not prime.

	  \part Negation: There is a function which is differentiable and not continuous.\\
	  Converse: For all functions $f$, if $f$ is continuous then $f$ is differentiable. \\
	  Contrapositive: For all functions $f$, if $f$ is not continuous then $f$ is not differentiable.

	  \part Negation: There are integers $a$ and $b$ for which $a\cdot b$ is even but $a$ or $b$ is odd.\\
	  Converse: For all integers $a$ and $b$, if $a$ and $b$ are even then $ab$ is even.\\
	  Contrapositive: For all integers $a$ and $b$, if $a$ or $b$ is odd, then $ab$ is odd.

	  \part Negation: There are integers $x$ and $y$ such that for every integer $n$, $x \le 0$ and $nx \le y$. \\
	  Converse: For every integer $x$ and every integer $y$ there is an integer $n$ such that if $nx > y$ then $x > 0$.\\
	  Contrapositive: For every integer $x$ and every integer $y$ there is an integer $n$ such that if $nx \le y$ then $x \le 0$.

	  \part  Negation: There are real numbers $x$ and $y$ such that $xy = 0$ but $x \ne 0$ and $y \ne 0$.\\
	  Converse: For all real numbers $x$ and $y$, if $x = 0$ or $y = 0$ then $xy = 0$\\
	  Contrapositive: For all real numbers $x$ and $y$, if $x \ne 0$ and $y \ne 0$ then $xy \ne 0$.

	  \part Negation: There is at least one student in Math 228 who does not understand implications but will still pass the exam.\\
	  Converse: For every student in Math 228, if they fail the exam, then they did not understand implications.\\
	  Contrapositive: For every student in Math 228, if they pass the exam, then they understood implications.

	\end{parts}
	
\end{ans}
\begin{ans}{3.5.6.}
      \begin{parts}
	% Is the statement true?  Explain why.
	\part The statement is true.  If $n$ is an even integer less than or equal to 8, then the only way it could not be negative is if $n$ was equal to 0, 2, 4, or 6.
	% Write the negation of the statement.  Is it true?  Explain.
	\part There is an integer $n$ such that $n$ is even and $n \le 7$ but $n$ is not negative and $n \not\in \{0,2,4,6\}$.  This is false, since the original statement is true.
	%  State the contrapositive of the statement.  Is it true?  Explain.
	\part For all integers $n$, if $n$ is not negative and $n \not\in\{0,2,4,6\}$ then $n$ is odd or $n > 7$.  This is true, since the contrapositive is equivalent to the original statement (which is true).
	%  State the converse of the statement.  Is it true?  Explain.
	\part For all integers $n$, if $n$ is negative or $n \in \{0,2,4,6\}$ then $n$ is even and $n \le 7$.  This is false.  $n = -3$ is a counter-example.
      \end{parts}
  
\end{ans}
\begin{ans}{3.5.7.}
      \begin{parts}
	\part For any number $x$, if it is the case that adding any number to $x$ gives that number back, then multiplying any number by $x$ will give 0.  This is true (of the integers or the reals) - the ``if'' part only holds if $x = 0$, and in that case, anything times $x$ will be 0.
	\part The converse in words is this: for any number $x$, if everything times $x$ is zero, then everything added to $x$ gives itself.  Or in symbols: $\forall x (\forall z (x \cdot z = 0) \imp \forall y (x + y = y))$.  The converse is true - the only number which when multiplied by any other number gives 0 is $x = 0$.  And if $x = 0$, then $x + y = y$.
	\part The contrapositive in words is: for any number $x$, if there is some number which when multiplied by $x$ does not give zero, then there is some number which when added to $x$ does not give that number.  In symbols: $\forall x (\exists z (x\cdot z \ne 0) \imp \exists y (x + y \ne y))$.  We know the contrapositive must be true because the original implication is true.
	\part The negation: there is a number $x$ such that any number added to $x$ gives the number back again, but there is a number you can multiply $x$ by and not get 0.  In symbols: $\exists x (\forall y (x + y = y) \wedge \exists z (x \cdot z \ne 0))$.  Of course since the original implication is true, the negation is false.
      \end{parts}
  
\end{ans}
\begin{ans}{3.5.8.}
      \begin{parts}
	  \part If you have lost weight, then you exercised.
	  \part If you exercise, then you will lose weight.
	  \part If you are American, then you are patriotic.
	  \part If you are patriotic, then you are American.
	  \part If a number is rational, then it is real.
	  \part If a number is not even, then it is prime.  (Or the contrapositive: if a number is not prime, then it is even.)
	  \part If the Broncos don't win the Super Bowl, then they didn't play in the Super Bowl.  Alternatively, if the Broncos play in the Super Bowl, then they will win the Super Bowl.
      \end{parts}
  
\end{ans}
\begin{ans}{3.5.9.}
      \begin{parts}
	    % $\neg (\neg (P \wedge \neg Q) \imp \neg(\neg R \vee \neg(P \imp R)))$
	    \part $(\neg P \vee Q) \wedge (\neg R \vee (P \wedge \neg R))$
	    %  $\neg \exists x \neg \forall y \neg \exists z (z = x + y \imp \exists w (x - y = w))$
	    \part $\forall x \forall y \forall z (z = x+y \wedge \forall w (x-y \ne w))$
      \end{parts}
  
\end{ans}
\begin{ans}{3.5.10.}
	\begin{parts}
	  \part Direct proof.
	  \begin{proof}
	    Let $n$ be an integer.  Assume $n$ is odd.  So $n = 2k+1$ for some integer $k$.  Then
	    \[7n = 7(2k+1) = 14k + 7 = 2(7k +3) + 1\]
	    Since $7k + 3$ is an integer, we see that $7n$ is odd.
	  \end{proof}

	  \part The converse is: for all integers $n$ if $7n$ is odd, then $n$ is odd.  We will prove this by contrapositive.
	  \begin{proof}
	    Let $n$ be an integer.  Assume $n$ is not odd.  Then $n = 2k$ for some integer $k$.  So $7n = 14k = 2(7k)$ which is to say $7n$ is even.  Therefore $7n$ is not odd.
	  \end{proof}

	\end{parts}
	
\end{ans}
\begin{ans}{3.5.11.}
		\begin{parts}
		\part Suppose you only had 5 coins of each denomination.  This means you have 5 pennies, 5 nickels, 5 dimes and 5 quarters.  This is a total of 20 coins.  But you have more than 20 coins, so you must have more than 5 of at least one type.
		\part Suppose you have 22 coins, including $2k$ nickels, $2j$ dimes, and $2l$ quarters (so an even number of each of these three types of coins).  The number of pennies you have will then be
		\[22 - 2k - 2j - 2l = 2(11-k-j-l)\]
		But this says that the number of pennies is also even (it is 2 times an integer).  Thus we have established the contrapositive of the statement, ``If you have an odd number of pennies then you have an odd number of at least one other coin type.''
		\part You need 10 coins.  You could have 3 pennies, 3 nickels, and 3 dimes.  The 10th coin must either be a quarter, giving you 4 coins that are all different, or else a 4th penny, nickel or dime.  To prove this, assume you don't have 4 coins that are all the same or all different.  In particular, this says that you only have 3 coin types, and each of those types can only contain 3 coins, for a total of 9 coins, which is less than 10.
		\end{parts}
	
\end{ans}
\protect \end {itemize}
 \protect \noindent {\protect \textbf  {Solutions for Section 4.1}} \protect \begin {itemize} 
\begin{ans}{4.1.1.}
		This is asking for the number of edges in $K_{10}$.  Each vertex (person) has degree (shook hands with) 9 (people).  So the sum of the degrees is $90$.  However, the degrees count each edge (handshake) twice, so there are 45 edges in the graph.  That is how many handshakes took place.%If 10 people each shake hands with each other, how many handshakes took place?  What does this question have to do with graph theory?
	
\end{ans}
\begin{ans}{4.1.2.}
		It is possible for everyone to be friends with exactly 2 people - you could arrange the 5 people in a circle and say that everyone is friends with the two people on either side of them (so you get the graph $C_5$).  However, it is not possible for everyone to be friends with 3 people - that would lead to a graph with an odd number of odd degree vertices which is impossible - the sum of the degrees must be even.  %Among a group of 5 people, is it possible for everyone to be friends with exactly 2 of the people in the group?  What about 3 of the people in the group?
	
\end{ans}
\begin{ans}{4.1.3.}
		Yes.  For example, both graphs below contain 6 vertices, 7 edges, and have degrees (2,2,2,2,3,3).
		\begin{center}
		  \hfill
		  \begin{tikzpicture}
		   \draw[thick] (-2,0) \v -- (-1,0) \v -- (-1.5,1) \v -- (-2,0) (-1.5,1) -- (1.5, 1) \v -- (1,0) \v -- (2,0) \v -- (1.5,1);
		  \end{tikzpicture}
		  \hfill
		  \begin{tikzpicture}
		  \foreach \x in {0,...,5}
		    \draw[thick] (\x*60:1) \v -- (\x*60 + 60:1);
		    \draw[thick] (0:1) -- (180:1);
		  \end{tikzpicture}
		  \hfill ~
		\end{center}
	
\end{ans}
\begin{ans}{4.1.4.}
		The graphs are not equal.  For example, graph 1 has an edge $\{a,b\}$ but graph 2 does not have that edge.  They are isomorphic.  One possible isomorphism is $f:G_1 \to G_2$ defined by $f(a) = d$, $f(b) = c$, $f(c) = e$, $f(d) = b$, $f(e) = a$.
	
\end{ans}
\begin{ans}{4.1.5.}
		Three of the graphs are bipartite.  The one which is not is $C_7$ (second from the right).
	
\end{ans}
\begin{ans}{4.1.6.}
		$C_n$ is bipartite if and only if $n = 1$ or is even.
	
\end{ans}
\begin{ans}{4.1.7.}
		\begin{parts}
		\part For example:

		\tikz{
			\draw (0,0) \v -- (-1,1) \v (0,0) -- (0,1) \v (0,0) -- (1,1) \v;
		} \qquad
		\tikz{
			\draw (0,0) \v -- (-1,1) \v (0,0) -- (.5,.5) \v -- (1,1) \v;
		}

		\part This is not possible if we require the graphs to be connected.  If not, we could take $C_8$ as one graph and two copies of $C_4$ as the other.

		\part Not possible.  If you have a graph with 5 vertices all of degree 4, then every vertex must be adjacent to every other vertex.  This is the graph $K_5$.
		\part This is not possible.  In fact, there is not even one graph with this property (such a graph would have $5\cdot 3/2 = 7.5$ edges).
		\end{parts}
	
\end{ans}
\protect \end {itemize}
 \protect \noindent {\protect \textbf  {Solutions for Section 4.2}} \protect \begin {itemize} 
\begin{ans}{4.2.1.}
		No.  A (connected) planar graph must satisfy Euler's formula: $v - e + f = 2$.  Here $v - e + f = 6 - 10 + 5 = 1$. %Is it possible for a planar graph to have 6 vertices, 10 edges and 5 faces?  Explain.
	
\end{ans}
\begin{ans}{4.2.2.}
		$G$ has 10 edges.  It could be planar, and then it would have 6 faces. %The graph $G$ has 6 vertices with degrees $2, 2, 3, 4, 4, 5$.  How many edges does $G$ have?  Could $G$ be planar?  If so, how many faces would it have.
	
\end{ans}
\begin{ans}{4.2.3.}
		Yes.  According to Euler's formula it would have 2 faces.  It does.  The only such graph is $C_{10}$. %If a graph has 10 vertices and 10 edges and contains an Euler circuit, must it be planar?  How many faces would it have?
	
\end{ans}
\begin{ans}{4.2.4.}
	Say the last polyhedron has $n$ edges, and also $n$ vertices.  The total number of edges the polyhedron has then is $(7 \cdot 3 + 4 \cdot 4 + n)/2 = (37 + n)/2$.  In particular, we know the last face must have an odd number of edges.  By Euler's formula, we have $v - (37+n)/2 + 12 = 2$, so $v = (17 + n)/2$.  But we also know that $v = 11 + n$.  Putting these together we get $n = 5$, so the last face is a pentagon.
	
\end{ans}
\begin{ans}{4.2.5.}
		\begin{proof}
			Let $P(n)$ be the statement, ``every planar graph containing $n$ edges satisfies $v - n + f = 2$.''  We will show $P(n)$ is true for all $n \ge 0$.

			Base case: there is only one graph with zero edges, namely a single isolated vertex.  In this case $v = 1$, $f = 1$ and $e = 0$, so Euler's formula holds.

			Inductive case:  Suppose $P(k)$ is true for some arbitrary $k \ge 0$.  Now consider an arbitrary graph containing $k+1$ edges (and $v$ vertices and $f$ faces).  No matter what this graph looks like, we can remove a single edge to get a graph with $k$ edges which we can apply the inductive hypothesis to.  There are two possibilities.  First, the edge we remove might be incident to a degree 1 vertex.  In this case, also remove that vertex.  The smaller graph will now satisfy $v-1 - k + f = 2$ by the induction hypothesis (removing the edge and vertex did not reduce the number of faces).  Adding the edge and vertex back gives $v - (k+1) + f = 2$, as required.  The second case is that the edge we remove is incident to vertices of degree greater than one.  In this case, removing the edge will keep the number of vertices the same but reduce the number of faces by one.  So by the inductive hypothesis we will have $v - k + f-1 = 2$.  Adding the edge back will give $v - (k+1) + f = 2$ as needed.

			Therefore, by the principle of mathematical induction, Euler's formula holds for all planar graphs.
		\end{proof}
	
\end{ans}
\begin{ans}{4.2.6.}
		Say the first component has $v_1$ vertices, $e_1$ edges and $f_1$ faces.  The second graph has $v_2$ vertices, $e_2$ edges and $f_2$ faces.  Thinking of each of these separately, we have
		\[v_1 - e_1 + f_1 = 2,\]
		\[v_2 - e_2 + f_2 = 2.\]
		Adding these two equations gives
		\[v - e + f = 4\]
		(since the graph has $v = v_1 + v_2$ vertices, etc).  However, the two components have one common face (the outside of one of them must be contained in one of the faces of the other) so in fact we get
		\[v - e + f = 3.\]
		In general, a planar graph with $k$ components will satisfy $v - e + f = 1 + k$.
	
\end{ans}
\begin{ans}{4.2.7.}
		\begin{proof}
		We know in any planar graph the number of faces $f$ satisfies $3f \le 2e$ since each face is bounded by at least three edges, but each edge borders two faces.  Combine this with Euler's formula:
				\[v - e + f = 2\]
				\[v - e + \frac{2e}{3} \ge 2\]
				\[3v - e \ge 6\]
				\[3v - 6 \ge e\]
		\end{proof}

	
\end{ans}
\begin{ans}{4.2.8.}
		\begin{proof}
		 Suppose this were not the case.  Then there would be a graph with $v$ vertices, each with degree 6 or more.  At a minimum then, there would be $6v/2 = 3v$ edges, so $e \ge 3v$.  By the previous exercise, we also have that $e \le 3v - 6$.  But these two facts are contradictory.
		 \end{proof}
	
\end{ans}
\protect \end {itemize}
 \protect \noindent {\protect \textbf  {Solutions for Section 4.3}} \protect \begin {itemize} 
\begin{ans}{4.3.1.}
		2, since the graph is bipartite.  One color for the top set of vertices, another color for the bottom set of vertices.  %What is the smallest number of colors you need to properly color the vertices of $K_{4,5}$.  That is, find the chromatic number of the graph.
	
\end{ans}
\begin{ans}{4.3.2.}
		For example, $K_6$.  If the chromatic number is 6, then the graph is not planar - the 4-color theorem states that all planar graphs can be colored with 4 or fewer colors. %Draw a graph with chromatic number 6 (i.e., which requires 6 colors to properly color the vertices).  Could your graph be planar?  Explain.
	
\end{ans}
\begin{ans}{4.3.3.}
		The chromatic numbers are 2, 3, 4, 5, and 3 respectively from left to right. %Find the chromatic number of each of the following graphs.
	
\end{ans}
\begin{ans}{4.3.4.}
		The cube can be represented as a planar graph and colored with two colors as follows:

		\begin{center}
		\begin{tikzpicture}
		\foreach \ang in {45, 135, 225, 315} {
		\draw (\ang:.4) \v -- (\ang:1) \v -- (\ang+90:1) (\ang:.4) -- (\ang+90:.4);
		}
		\draw (45:.4) node[right]{\tiny R} (135:.4) node[left]{\tiny B} (225:.4) node[left]{\tiny R} (315:.4) node[right]{\tiny B} (45:1) node[right]{\tiny B} (135:1) node[left]{\tiny R} (225:1) node[left]{\tiny B} (315:1) node[right]{\tiny R};
		\end{tikzpicture}
		\end{center}

		Since it would be impossible to color the vertices with a single color, we see that the cube has chromatic number 2 (it is bipartite).
	
\end{ans}
\begin{ans}{4.3.5.}
		The wheel graph below has this property.  The outside of the wheel forms an odd cycle, so requires 3 colors, the center of the wheel must be different than all the outside vertices.

		\begin{center}
		\begin{tikzpicture}

		\foreach \ang in {18, 90, ..., 306}{
		\draw (0,0) -- (\ang:1) \v -- (\ang+72:1);
		}
		\draw (0,0) \v;
		\end{tikzpicture}
		\end{center}
	
\end{ans}
\begin{ans}{4.3.6.}
		\begin{proof}
		Let $G$ be a graph with $n$ vertices, maximal degree $\Delta(G)$ and at least one vertex of degree less than $\Delta(G)$.  Assume for the sake of induction that all graphs $G'$ with fewer than $n$ vertices and a vertex of degree less than $\Delta(G')$ have chromatic number less than $\Delta(G')$.

		Find a vertex of $G$ with degree less than $\Delta(G)$ and remove it.  This forms a subgraph $G'$ which has $n-1$ vertices.  Also, since we removed edges, we know that $\Delta(G') \le \Delta(G)$.  If these maximal degrees are equal, then $G'$ must also have a vertex of degree less than $\Delta(G')$, since at least one of its vertices had one more edge in $G$.  In this case, we can apply our inductive hypothesis to produce a coloring of the vertices of $G'$ using just $\Delta(G)$ colors.  If $\Delta(G') < \Delta(G)$, then we also can easily find a proper coloring of the vertices of $G'$ using just $\Delta(G)$ colors by starting with any vertex and coloring it and all of its neighbors differently, and then fanning out.  Thus $G'$ has a proper vertex coloring using just $\Delta(G)$-many colors.

		Now move back to $G$.  Put the removed vertex back into the graph.  Since it is adjacent to at most $\Delta(G) - 1$ other vertices, there will be one of the $\Delta(G)$ colors that is not present among its neighbors, which we could use to color the newly inserted vertex.
		\end{proof}
	
\end{ans}
\begin{ans}{4.3.7.}
		If we drew a graph with each letter representing a vertex, and each edge connecting two letters that were consecutive in the alphabet, we would have a graph containing two vertices of degree 1 (A and Z) and the remaining 24 vertices all of degree 2 (for example, $D$ would be adjacent to both $C$ and $E$).  By Brooks' theorem, this graph has chromatic number at most 2, as that is the maximal degree in the graph and the graph is not a complete graph or odd cycle.  Thus only two boxes are needed.
	
\end{ans}
\begin{ans}{4.3.8.}
		\begin{proof}
		Start with a single vertex of $K_6$, call it $v_0$.  There are five edges incident to $v_0$, and by the pigeonhole principle, three of these must be colored identically.  Without loss of generality, say these edges are $(v_0, v_1)$, $(v_0, v_2)$ and $(v_0,v_3)$, and are colored red.  Consider the edges $(v_1,v_2)$, $(v_2,v_3)$, and $(v_3, v_1)$.  If any of these are colored red, we would have a monochromatic red triangle (it plus the two edges incident to $v_0$).  If they are all colored blue, then we have a monochromatic blue triangle (those three edges).  Either way, we are guaranteed to monochromatic triangle.

		\end{proof}
	
\end{ans}
\protect \end {itemize}
 \protect \noindent {\protect \textbf  {Solutions for Section 4.4}} \protect \begin {itemize} 
\begin{ans}{4.4.1.}
		This is a question about finding Euler paths.  Draw a graph with a vertex in each state, and connect vertices if their states share a border.  Exactly two vertices will have odd degree - the vertices for Nevada and Utah.  Thus you must start your road trip at in one of those states and end it in the other. %You and your friends want to tour the southwest by car.  You will visit the nine states below, with the following rather odd rule: you must cross each border between neighboring states exactly once (so, for example, you must cross the Colorado-Utah border exactly once).  Can you do it?  If so, does it matter where you start your road trip?  What fact about graph theory solves this problem?
	
\end{ans}
\begin{ans}{4.4.2.}
		\begin{parts}
		  \part $K_4$ does not have an Euler path or circuit.
		  \part $K_5$ has an Euler circuit (so also an Euler path).
		  \part $K_{5,7}$ does not have an Euler path or circuit.
		  \part $K_{2,7}$ has an Euler path but not an Euler circuit.
		  \part $C_7$ has an Euler circuit (it is a circuit graph!)
		  \part $P_7$ has an Euler path but no Euler circuit.
		\end{parts}
	
\end{ans}
\begin{ans}{4.4.3.}
		When $n$ is odd, $K_n$ contains an Euler circuit.  This is because every vertex has degree $n-1$, so an odd $n$ results in all degrees being even.%For which $n$ does the graph $K_n$ contain an Euler circuit?  Explain.
	
\end{ans}
\begin{ans}{4.4.4.}
		If both $m$ and $n$ are even, then $K_{m,n}$ has an Euler circuit.  When both are odd, there is no Euler path or circuit.  If one is 2 and the other is odd, then there is an Euler path but not an Euler circuit. %For which $m$ and $n$ does the graph $K_{m,n}$ contain an Euler path?  An Euler circuit?  Explain
	
\end{ans}
\begin{ans}{4.4.5.}
		If we build one bridge, we can have an Euler path.  Two bridges must be built for an Euler circuit.
		\begin{center}
		\begin{tikzpicture}[scale=1, yscale=.5]
		 \draw (-1,-2) \v to [out=120, in=240] (-1,0) \v to [out=120, in=240] (-1,2) \v to [out=300, in=60] (-1,0) to [out=300, in=60] (-1,-2);
		  \draw (1,0) \v -- (-1,2) (-1,0) -- (1,0) -- (-1,-2);
		  \draw[dashed] (-1,-2) -- (-1,0);
		  \draw[dashed] (1,0) to [out=120, in=0] (-1,2);
		  \end{tikzpicture}
		\end{center}
	
\end{ans}
\begin{ans}{4.4.6.}
		We are looking for a Hamiltonian cycle, and this graph does have one:

		\begin{center}
		\tikz{
			\foreach \x in {1,...,9}{
			\coordinate (v\x) at (90-\x*360/9:1.5);
			}
			\draw[color=gray] (v1) -- (v3) -- (v5) (v4) -- (v5) (v4) -- (v7) -- (v6) -- (v9) (v3) -- (v7) (v9) -- (v3);
			\draw[line width=1.2pt, color=blue] (v1) -- (v6) -- (v3) -- (v8) (v8) -- (v4) (v4) -- (v7) -- (v2) -- (v5) -- (v9) -- (v1);
			\foreach \x in {1,...,9}{
			\draw (v\x) \v;
			}
		}
		\end{center}
	
\end{ans}
\protect \end {itemize}
 \protect \noindent {\protect \textbf  {Solutions for Section 4.6}} \protect \begin {itemize} 
\begin{ans}{4.6.1.}
		The first and the third graphs are the same, but the middle graph is different.
	
\end{ans}
\begin{ans}{4.6.2.}
		The first (and third) graphs contain an Euler path.  All the graphs are planar.
	
\end{ans}
\begin{ans}{4.6.3.}
		For example, $K_5$.
	
\end{ans}
\begin{ans}{4.6.4.}
		For example, $K_{3,3}$.
	
\end{ans}
\begin{ans}{4.6.5.}
		\begin{parts}
		  \part Only if $n \ge 6$ and is even.%For which values of $n$ does this make sense?
		  \part None. %For which values of $n$ does the graph have an Euler path?
		  \part 12. Such a graph would have $\frac{5n}{2}$ edges.  If the graph is planar, then $n - \frac{5n}{2} + f = 2$ so there would be $\frac{4+3n}{2}$ faces.  Also, we must have $3f \le 2e$, since the graph is simple.  So we must have $3\frac{4 + 3n}{2} \le 5n$.  Solving for $n$ gives $n \ge 12$.%What is the smallest value of $n$ for which the graph might be planar? (tricky)
		\end{parts}
	
\end{ans}
\begin{ans}{4.6.6.}
  \begin{parts}
	 \part There were 24 couples: 6 choices for the girl and 4 choices for the boy.
	 \part There were 45 couples: ${10 \choose 2}$ since we must choose two of the 10 people to dance together.
	 \part For part (a), we are counting the number of edges in $K_{4,6}$.  In part (b) we count the edges of $K_{10}$.
  \end{parts}
  
\end{ans}
\begin{ans}{4.6.7.}
  Yes, as long as $n$ is even.  If $n$ were odd, then corresponding graph would have an odd number of odd degree vertices, which is impossible.
  
\end{ans}
\begin{ans}{4.6.8.}
	  \begin{parts}
	  \part No.  The 9 triangles each contribute 3 edges, and the 6 pentagons contribute 5 edges.  This gives a total of 57, which is exactly twice the number of edges, since each edge borders exactly 2 faces.  But 57 is odd, so this is impossible.
	  \part Now adding up all the edges of all the 16 polygons gives a total of 64, meaning there would be 32 edges in the polyhedron.  We can then use Euler's formula $v - e + f = 2$ to deduce that there must be 18 vertices.
	  \part If you add up all the vertices from each polygon separately, we get a total of 64.  This is not divisible by 3, so it cannot be that each vertex belongs to exactly 3 faces.  Could they all belong to 4 faces?  That would mean there were $64/4 = 16$ vertices, but we know from Euler's formula that there must be 18 vertices.  We can write $64 = 3x + 4y$ and solve for $x$ and $y$ (as integers).  We get that there must be 10 vertices with degree 4 and 8 with degree 3. (Note the number of faces joined at a vertex is equal to its degree in graph theoretic terms.)
	  \end{parts}
  
\end{ans}
\begin{ans}{4.6.9.}
		 No.  Every polyhedron can be represented as a planar graph, and the Four Color Theorem says that every planar graph has chromatic number at most 4.
	 
\end{ans}
\begin{ans}{4.6.10.}
  $K_{n,n}$ has $n^2$ edges.  The graph will have an Euler circuit when $n$ is even.  The graph will be planar only when $n < 3$.
  
\end{ans}
\begin{ans}{4.6.11.}
  $G$ has 8 edges (since the sum of the degrees is 16).  If $G$ is planar, then it will have 4 faces (since $6 - 8 + 4 = 2$).  $G$ does not have an Euler path since there are more than 2 vertices of odd degree.
  
\end{ans}
\begin{ans}{4.6.12.}
  $7$ colors.  Thus $K_7$ is not planar (by the contrapositive of the Four Color Theorem).
  
\end{ans}
\begin{ans}{4.6.13.}
  The chromatic number of $K_{3,4}$ is 2, since the graph is bipartite.  You cannot say whether the graph is planar based on this coloring (the converse of the Four Color Theorem is not true).  In fact, the graph is {\em not} planar, since it contains $K_{3,3}$ as a subgraph.
  
\end{ans}
\begin{ans}{4.6.14.}
		For all these questions, we are really coloring the vertices of a graph.  You get the graph by first drawing a planar representation of the polyhedron and then taking its planar dual: put a vertex in the center of each face (including the outside) and connect two vertices if their faces share an edge.
		\begin{parts}
			\part Since the planar dual of a dodecahedron contains a 5-wheel, it's chromatic number is at least 4.  Alternatively, suppose you could color the faces using 3 colors without any two adjacent faces colored the same.  Take any face and color it blue.  The 5 pentagons bordering this blue pentagon cannot be colored blue.  Color the first one red.  Its two neighbors (adjacent to the blue pentagon) get colored green.  The remaining 2 cannot be blue or green, but also cannot both be red since they are adjacent to each other.  Thus a 4th color is needed.
			\part The planar dual of the dodecahedron is itself a planar graph.  Thus be the 4-color theorem, it can be colored using only 4 colors without two adjacent vertices (corresponding to the faces of the polyhedron) being colored identically.
			\part The cube can be properly 3-colored.  Color the ``top'' and ``bottom'' red, the ``front'' and ``back'' blue, and the ``left'' and ``right'' green.
		\end{parts}
	
\end{ans}
\begin{ans}{4.6.15.}
  $G$ has $13$ edges, since we need $7 - e + 8 = 2$.
  
\end{ans}
\begin{ans}{4.6.16.}
  \begin{parts}
	 \part The graph does have an Euler path, but not an Euler circuit.  There are exactly two vertices with odd degree.  The path starts at one and ends at the other.
	 \part The graph is planar.  Even though as it is drawn edges cross, it is easy to redraw it without edges crossing.
	 \part The graph is not bipartite (there is an odd cycle), nor complete.
	 \part The chromatic number of the graph is 3.
  \end{parts}
  
\end{ans}
\begin{ans}{4.6.17.}
  \begin{parts}
	 \part False.  For example, $K_{3,3}$ is not planar.
	 \part True.  The graph is bipartite so it is possible to divide the vertices into two groups with no edges between vertices in the same group.  Thus we can color all the vertices of one group red and the other group blue.
	 \part False.  $K_{3,3}$ has 6 vertices with degree 3, so contains no Euler path.
	 \part False.  $K_{3,3}$ again.
	 \part False.  The sum of the degrees of all vertices is even for {\em all} graphs so this property does not imply that the graph is bipartite.
  \end{parts}
  
\end{ans}
\begin{ans}{4.6.18.}
	The first and third graphs have a matching, shown in bold (there are other matchings as well).  The middle graph does not have a matching.  If you look at the three circled vertices, you see that they only have two neighbors, which violates the matching condition $|N(S)| \ge S$ (the three circled vertices form the set $S$).

	 \begin{tikzpicture}
	 \coordinate (a) at (0,0);
	 \coordinate (A) at (0,1);
	 \coordinate (b) at (1,0);
	 \coordinate (B) at (1,1);
	 \coordinate (c) at (2,0);
	 \coordinate (C) at (2,1);
	 \draw (a) \v -- (B) \v -- (c) \v -- (C) \v -- (a) \v -- (A)\v -- (b) \v;
	 \draw[very thick] (a) -- (C) (A) -- (b) (c) -- (B);
	 \end{tikzpicture}
	 \hfill
	 \begin{tikzpicture}
	 \coordinate (a) at (0,0);
	 \coordinate (A) at (0,1);
	 \coordinate (b) at (1,0);
	 \coordinate (B) at (1,1);
	 \coordinate (c) at (2,0);
	 \coordinate (C) at (2,1);
	 \coordinate (d) at (3,0);
	 \coordinate (D) at (3,1);
	 \draw (a) \v -- (A) \v (b) \v -- (B) \v (c) \v -- (C) \v (d) \v  (D)\v;
	 \draw (a) -- (C) -- (b) -- (D) (A) -- (c) (A) -- (d) -- (C);
	 \draw[dashed] (a) circle (7pt) (c) circle (7pt) (d) circle (7pt);
	 \end{tikzpicture}
	 \hfill
	 \begin{tikzpicture}
	 \coordinate (a) at (0,0);
	 \coordinate (A) at (0,1);
	 \coordinate (b) at (1,0);
	 \coordinate (B) at (1,1);
	 \coordinate (c) at (2,0);
	 \coordinate (C) at (2,1);
	 \coordinate (d) at (3,0);
	 \coordinate (D) at (3,1);
	 \coordinate (e) at (4,0);
	 \coordinate (E) at (4,1);
	 \draw (a) \v (A) \v (b) \v (B) \v (c) \v  (C) \v (d) \v  (D)\v (e)\v (E) \v;
	 \draw (a) -- (A) (a) -- (B) (A) -- (b) (A) -- (c) (b) -- (C) (B) -- (c) -- (D) (c) -- (E) (C) -- (d) -- (E) (D) -- (e) -- (E);
	 \draw[very thick] (a) -- (A) (b) -- (C) (c) -- (B) (d) -- (E) (e) -- (D);
	 \end{tikzpicture}

	
\end{ans}
\begin{ans}{4.6.19.}
  \begin{parts}
  \part If a graph has an Euler path, then it is planar.
  \part If a graph does not have an Euler path, then it is not planar.
  \part There is a graph which is planar and does not have an Euler path.
  \part Yes.  In fact, in this case it is because the original statement is false.
  \part False.  $K_4$ is planar but does not have an Euler path.
  \part False.  $K_5$ has an Euler path but is not planar.
  \end{parts}
  
\end{ans}
\protect \end {itemize}
 \protect \noindent {\protect \textbf  {Solutions for Section A.1}} \protect \begin {itemize} 
\begin{ans}{A.1.1.}
		\begin{parts}
		  \part $\dfrac{4}{1-x}$  %$4,4,4,4,4,\ldots$
		  \part $\dfrac{2}{(1-x)^2}$  %$2, 4, 6, 8, 10, \ldots$
		  \part $\dfrac{2x^3}{(1-x}^2$  %$0,0,0,2,4,6,8,10,\ldots$
		  \part $\dfrac{1}{1-5x}$  %$1, 5, 25, 125, \ldots$
		  \part $\dfrac{1}{1+3x}$  %$1, -3, 9, -27, 81, \ldots$
		  \part $\dfrac{1}{1-5x^2}$  %$1, 0, 5, 0, 25, 0, 125, 0, \ldots$
		  \part $\dfrac{x}{(1-x^3)^2}$  %$0, 1, 0, 0, 2, 0, 0, 3, 0, 0, 4, 0, 0, 5, \ldots$
		\end{parts}
	
\end{ans}
\begin{ans}{A.1.2.}
		\begin{parts}
		  \part $0, 4, 4, 4, 4, 4, \ldots$  %$\dfrac{4x}{1-x}$
		  \part $1, 4, 16, 64, 256, \ldots$  %$\dfrac{1}{1-4x}$
		  \part $0, 1, -1, 1, -1, 1, -1, \ldots$  %$\dfrac{x}{1+x}$
		  \part $0, 3, -6, 9, -12, 15, -18, \ldots $  %$\dfrac{3x}{(1+x)^2}$
		  \part $1, 3, 6, 9, 12, 15, \ldots$  %$\dfrac{1+x+x^2}{(1-x)^2}$ (Hint: multiplication)
		\end{parts}
	
\end{ans}
\begin{ans}{A.1.3.}
		\begin{parts}
		  \part The second derivative of $\dfrac{1}{1-x}$ is $\dfrac{2}{(1-x)^3}$ which expands to $2 + 6x + 12x^2 + 20x^3 + 30x^4 + \cdots$.  Dividing by 2 gives the generating function for the triangular numbers. %Take two derivatives of the generating function for $1,1,1,1,1, \ldots$
		  \part Compute $A - xA$ and you get $1 + 2x + 3x^2 + 4x^3 + \cdots$ which can be written as $\dfrac{1}{(1-x)^2}$.  Solving for $A$ gives the correct generating function. %Use differencing.
		  \part The triangular numbers are the sum of the first $n$ numbers $1,2,3,4, \ldots$.  To get the sequence of partial sums, we multiply by $\frac{1}{1-x}$. so this gives the correct generating function again. %Multiply two known generating functions.
		\end{parts}
	
\end{ans}
\begin{ans}{A.1.4.}
		Call the generating function $A$.  Compute $A - xA = 4 + x + 2x^2 + 3x^3 + 4x^4 + \cdots$.  Thus $A - xA = 4 + \dfrac{x}{(1-x)^2}$.  Solving for $A$ gives $\d\frac{4}{1-x} + \frac{x}{(1-x)^3}$.  %Use differencing to find the generating function for $4, 5, 7, 10, 14, 19, 25, \ldots$
	
\end{ans}
\begin{ans}{A.1.5.}
		$\dfrac{1+2x}{1-3x + x^2}$  %Find a generating function for the sequence with recurrence relation $a_n = 3a_{n-1} - a_{n-2}$ with initial terms $a_0 = 1$ and $a_1 = 5$.
	
\end{ans}
\begin{ans}{A.1.6.}
		Compute $A - xA - x^2A$ and the solve for $A$.  The generating function will be $\dfrac{x}{1-x-x^2}$.  %Use the recurrence relation for the Fibonacci numbers to find the generating function for the Fibonacci sequence.
	
\end{ans}
\begin{ans}{A.1.7.}
		$\dfrac{x}{(1-x)(1-x-x^2)}$  %Use multiplication to find the generating function for the sequence of partial sums of Fibonacci numbers, $S_0, S_1, S_2, \ldots$ where $S_0 = F_0$, $S_1 = F_0 + F_1$, $S_2 = F_0 + F_1 + F_2$, $S_3 = F_0 + F_1 + F_2 + F_3$ and so on.
	
\end{ans}
\begin{ans}{A.1.8.}
		$\dfrac{2}{1-5x} + \dfrac{7}{1+3x}$.  %Find the generating function for the sequence with closed formula $a_n = 2(5^n) + 7(-3)^n$.
	
\end{ans}
\begin{ans}{A.1.9.}
		$a_n = 3\cdot 4^{n-1} + 1$  %Find a closed formula for the $n$th term of the sequence with generating function $\dfrac{3x}{1-4x} + \dfrac{1}{1-x}$.
	
\end{ans}
\begin{ans}{A.1.10.}
		Hint: you should ``multiply'' the two sequences.  Answer: 158.  %Find $a_7$ for the sequence with generating function $\dfrac{2}{(1-x)^2}\cdot\dfrac{x}{1-x-x^2}$
	
\end{ans}
\begin{ans}{A.1.11.}
		Starting with $\frac{1}{1-x} = 1 + x + x^2 + x^3 +\cdots$, we can take derivatives of both sides, given $\frac{1}{(1-x)^2} = 1 + 2x + 3x^2 + \cdots$.  By the definition of generating functions, this says that $\frac{1}{(1-x)^2}$ generates the {\em sequence} 1, 2, 3\ldots.  You can also find this using differencing or by multiplying.
	
\end{ans}
\begin{ans}{A.1.12.}
		\begin{parts}
		 \part $\frac{1}{(1-x^2)^2}$  %$1, 0, 2, 0, 3, 0, 4,\ldots$
		 \part $\frac{1}{(1+x)^2}$  %$1, -2, 3, -4, 5, -6, \ldots$
		 \part $\frac{3x}{(1-x)^2}$  %$0, 3, 6, 9, 12, 15, 18, \ldots$
		 \part $\frac{3x}{(1-x)^3}$  (partial sums)  %$0, 3, 9, 18, 30, 45, 63,\ldots$ (Hint: relate this sequence to the previous one.)
		 \end{parts}
	
\end{ans}
\begin{ans}{A.1.13.}
		\begin{parts}
		  \part $0,0,1,1,2,3,5,8, \ldots$   %$\frac{x^2}{1-x-x^2}$
		  \part $1, 0, 1, 0, 2, 0, 3, 0, 5, 0, 8, 0, \ldots$  %$\frac{1}{1-x^2-x^4}$
		  \part $1, 3, 18, 81, 405, \ldots$  %$\frac{1}{1-3x-9x^2}$
		  \part $1, 2, 4, 7, 12, 20$  %$\frac{1}{(1-x-x^2)(1-x)}$
		\end{parts}
	
\end{ans}
\begin{ans}{A.1.14.}
		$\frac{1}{1+2x}$
	
\end{ans}
\begin{ans}{A.1.15.}
		$\frac{x^3}{(1-x)^2} + \frac{1}{1-x}$
	
\end{ans}
\begin{ans}{A.1.16.}
		\begin{parts}
		 \part $(1-x)A = 3 + 2x + 4x^2 + 6x^3 + \cdots$ which is almost right.  We can fix it like this:
		 $2 + 4x + 6x^2 + \cdots = \frac{(1-x)A - 3}{x}$
		 \part We know $2 + 4x + 6x^3 + \cdots = \frac{2}{(1-x)^2}$
		 \part $A = \frac{2x}{(1-x)^3} + \frac{3}{1-x} = \frac{3 -4x + 3x^2}{(1-x)^3}$
		\end{parts}
	
\end{ans}
\protect \end {itemize}
 \protect \noindent {\protect \textbf  {Solutions for Section A.2}} \protect \begin {itemize} 
\begin{ans}{A.2.1.}
		\begin{proof}
			Suppose $a \mid b$.  Then $b$ is a multiple of $a$, or in other words, $b = ak$ for some $k$.  But then $bc = akc$, and since $kc$ is an integer, this says $bc$ is a multiple of $a$.  In other words, $a \mid bc$.
		\end{proof}
	
\end{ans}
\begin{ans}{A.2.2.}
		\begin{proof}
			Assume $a \mid b$ and $a \mid c$.  This means that $b$ and $c$ are both multiples of $a$, so $b = am$ and $c = an$ for integers $m$ and $n$.  Then $b+c = am+an = a(m+n)$, so $b+c$ is a multiple of $a$, or equivalently, $a \mid b+c$.  Similarly, $b-c = am-an = a(m-n)$, so $b-c$ is a multiple of $a$, which is to say $a \mid b-c$.
		\end{proof}
	
\end{ans}
\begin{ans}{A.2.3.}
		$\{\ldots, -8, -4, 0, 4, 8, 12, \ldots\}$, $\{\ldots, -7, -3, 1, 5, 9, 13, \ldots\}$, $\{\ldots, -6, -2, 2, 6, 10, 14, \ldots\}$,\\ and $\{\ldots, -5, -1, 3, 7, 11, 15, \ldots\}$.
	
\end{ans}
\begin{ans}{A.2.4.}
		\begin{proof}
			Assume $a \equiv b \pmod n$ and $c \equiv d \pmod n$.  This means $a = b + kn$ and $c = d + jn$ for some integers $k$ and $j$.  Consider $a-c$.  We have:
			\[a-c = b+kn - (d+jn) = b-d + (k-j)n\]
			In other words, $a-c$ is $b-d$ more than some multiple of $n$, so $a-c \equiv b-d \pmod n$.
		\end{proof}
	
\end{ans}
\begin{ans}{A.2.5.}
		\begin{parts}
			\part $3^{456} \equiv 1^{456} = 1 \pmod 2$.
			\part $3^{456} = 9^{228} \equiv (-1)^{228} = 1 \pmod{5}$
			\part $3^{456} = 9^{228} \equiv 2^{228} = 8^{76} \equiv 1^{76} = 1 \pmod 7$
			\part $3^{456} = 9^{228} \equiv 0^{228} = 0 \pmod{9}$
		\end{parts}
	
\end{ans}
\begin{ans}{A.2.6.}
		For all of these, just plug in all integers between 0 and the modulus to see which, if any, work.
		\begin{parts}
			\part No solutions.
			\part $x = 3$.
			\part $x = 2$, $x = 5$, $x = 8$.
			\part No solutions.
			\part No solutions.
			\part $x = 3$.
		\end{parts}
	
\end{ans}
\begin{ans}{A.2.7.}
		\begin{parts}
			\part $x = 5+22k$ for $k \in \Z$.
			\part $x = 4 + 5k$ for $k \in \Z$.
			\part $x = 6 + 15k$ for $k \in \Z$.
			\part Hint: first reduce each number modulo 9, which can be done by adding up the digits of the numbers.  Answer: $x = 2 + 9k$ for $k \in \Z$.
		\end{parts}
	
\end{ans}
\begin{ans}{A.2.8.}
		We must solve $7x + 5 \equiv 2 \pmod{11}$.  This gives $x \equiv 9 \pmod{11}$.  In general, $x = 9 + 11k$, but when you divide any such $x$ by 11, the remainder will be 9.
	
\end{ans}
\begin{ans}{A.2.9.}
  		\begin{parts}
	  		\part Divide through by 2: $3x + 5y = 16$.  Convert to a congruence, modulo 3: $5y \equiv 16 \pmod 3$, which reduces to $2y \equiv 1 \pmod 3$.  So $y \equiv 2 \pmod 3$ or $y = 2 + 3k$.  Plug this back into $3x + 5y = 16$ and solve for $x$, to get $x = 2-5k$.  So the general solution is $x = 2-5k$ and $y = 2+3k$ for $k \in \Z$.
	  		\part $x = 7+8k$ and $y = -11 - 17k$ for $k \in \Z$.
	  		\part $x = -4-47k$ and $y = 3 + 35k$ for $k \in \Z$.
	  	\end{parts}
  	
\end{ans}
\begin{ans}{A.2.10.}
		First, solve the Diophantine equation $13x + 20 y = 2$.  The general solution is $x = -6 - 20k$ and $y = 4+13k$.  Now if $k = 0$, this correspond to filling the 20 oz. bottle 4 times, and emptying the 13 oz. bottle 6 times, which would require 80 oz. of water.  Increasing $k$ would require considerably more water.  Perhaps $k = -1$ would be better?  Then we would have $x = -6+20 = 14$ and $y = 4-13 = -11$, which describes the solution where we fill the 13 oz. bottle 14 times, and empty the 20 oz. bottle 11 times.  This would require 182 oz. of water.  Thus the most efficient procedure is to repeatedly fill the 20 oz bottle, emptying it into the 13 oz bottle, and discarding full 13 oz. bottles, which requires 80 oz. of water.
	
\end{ans}
\protect \end {itemize}
