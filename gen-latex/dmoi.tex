%**************************************%
%* Generated from MathBook XML source *%
%*    on 2016-07-22T12:51:08-06:00    *%
%*                                    *%
%*   http://mathbook.pugetsound.edu   *%
%*                                    *%
%**************************************%
\documentclass[10pt,]{book}
%% Load geometry package to allow page margin adjustments
\usepackage{geometry}
\geometry{letterpaper,total={5.0in,9.0in}}
%% Custom Preamble Entries, early (use latex.preamble.early)
%% Inline math delimiters, \(, \), need to be robust
%% 2016-01-31:  latexrelease.sty  supersedes  fixltx2e.sty
%% If  latexrelease.sty  exists, bugfix is in kernel
%% If not, bugfix is in  fixltx2e.sty
%% See:  https://tug.org/TUGboat/tb36-3/tb114ltnews22.pdf
%% and read "Fewer fragile commands" in distribution's  latexchanges.pdf
\IfFileExists{latexrelease.sty}{}{\usepackage{fixltx2e}}
%% Page Layout Adjustments (latex.geometry)
%% This LaTeX file may be compiled with pdflatex or xelatex
%% The following provides engine-specific capabilities
%% Generally, xelatex will do better languages other than US English
%% You can pick from the conditional if you will only ever use one engine
\usepackage{ifthen}
\usepackage{ifxetex}
\ifthenelse{\boolean{xetex}}{%
%% begin: xelatex-specific configuration
%% fontspec package will make Latin Modern (lmodern) the default font
\usepackage{xltxtra}
\usepackage{fontspec}
%% end: xelatex-specific configuration
}{%
%% begin: pdflatex-specific configuration
%% translate common Unicode to their LaTeX equivalents
%% Also, fontenc with T1 makes CM-Super the default font
%% (\input{ix-utf8enc.dfu} from the "inputenx" package is possible addition (broken?)
\usepackage[T1]{fontenc}
\usepackage[utf8]{inputenc}
%% end: pdflatex-specific configuration
}
%% Monospace font: Inconsolata (zi4)
%% Sponsored by TUG: http://levien.com/type/myfonts/inconsolata.html
%% See package documentation for excellent instructions
%% One caveat, seem to need full file name to locate OTF files
%% Loads the "upquote" package as needed, so we don't have to
%% Upright quotes might come from the  textcomp  package, which we also use
%% We employ the shapely \ell to match Google Font version
%% pdflatex: "varqu" option produces best upright quotes
%% xelatex: add StylisticSet 1 for shapely \ell
%% xelatex: add StylisticSet 2 for plain zero
%% xelatex: we add StylisticSet 3 for upright quotes
%% 
\ifthenelse{\boolean{xetex}}{%
%% begin: xelatex-specific monospace font
\usepackage{zi4}
\setmonofont[BoldFont=Inconsolatazi4-Bold.otf,StylisticSet={1,3}]{Inconsolatazi4-Regular.otf}
%% end: xelatex-specific monospace font
}{%
%% begin: pdflatex-specific monospace font
\usepackage[varqu]{zi4}
%% end: pdflatex-specific monospace font
}
%% Symbols, align environment, bracket-matrix
\usepackage{amsmath}
\usepackage{amssymb}
%% allow more columns to a matrix
%% can make this even bigger by overriding with  latex.preamble.late  processing option
\setcounter{MaxMatrixCols}{30}
%% Semantic Macros
%% To preserve meaning in a LaTeX file
%% Only defined here if required in this document
%% Used for inline definitions of terms
\newcommand{\terminology}[1]{\textbf{#1}}
%% Subdivision Numbering, Chapters, Sections, Subsections, etc
%% Subdivision numbers may be turned off at some level ("depth")
%% A section *always* has depth 1, contrary to us counting from the document root
%% The latex default is 3.  If a larger number is present here, then
%% removing this command may make some cross-references ambiguous
%% The precursor variable $numbering-maxlevel is checked for consistency in the common XSL file
\setcounter{secnumdepth}{3}
%% Environments with amsthm package
%% Theorem-like environments in "plain" style, with or without proof
\usepackage{amsthm}
\theoremstyle{plain}
%% Numbering for Theorems, Conjectures, Examples, Figures, etc
%% Controlled by  numbering.theorems.level  processing parameter
%% Always need a theorem environment to set base numbering scheme
%% even if document has no theorems (but has other environments)
\newtheorem{theorem}{Theorem}[section]
%% Only variants actually used in document appear here
%% Style is like a theorem, and for statements without proofs
%% Numbering: all theorem-like numbered consecutively
%% i.e. Corollary 4.3 follows Theorem 4.2
%% Example-like environments, normal text
%% Numbering is in sync with theorems, etc
\theoremstyle{definition}
\newtheorem{example}[theorem]{Example}
%% Numbering for Projects (independent of others)
%% Controlled by  numbering.projects.level  processing parameter
%% Always need a project environment to set base numbering scheme
%% even if document has no projectss (but has other blocks)
\newtheorem{project}{Project}[section]
%% Project-like environments, normal text
\theoremstyle{definition}
\newtheorem{activity}[project]{\emph{Investigate!}}
%% assemblage: minimally structured content, high visibility presentation
%% Package for breakable highlight boxes
\usepackage{mdframed}
%% assemblage style
\mdfdefinestyle{assemblage}{framemethod=default,linewidth=2pt,roundcorner=16pt,backgroundcolor=black!05}
%% Miscellaneous environments, normal text
%% Numbering for inline exercises and lists is in sync with theorems, etc
\theoremstyle{definition}
\newtheorem{exercise}[theorem]{Exercise}
%% Localize LaTeX supplied names (possibly none)
\renewcommand*{\appendixname}{Appendix}
\renewcommand*{\chaptername}{Chapter}
%% For improved tables
\usepackage{array}
%% Some extra height on each row is desirable, especially with horizontal rules
%% Increment determined experimentally
\setlength{\extrarowheight}{0.2ex}
%% Define variable thickness horizontal rules, full and partial
%% Thicknesses are 0.03, 0.05, 0.08 in the  booktabs  package
\makeatletter
\newcommand{\hrulethin}  {\noalign{\hrule height 0.04em}}
\newcommand{\hrulemedium}{\noalign{\hrule height 0.07em}}
\newcommand{\hrulethick} {\noalign{\hrule height 0.11em}}
%% We preserve a copy of the \setlength package before other
%% packages (extpfeil) get a chance to load packages that redefine it
\let\oldsetlength\setlength
\newlength{\Oldarrayrulewidth}
\newcommand{\crulethin}[1]%
{\noalign{\global\oldsetlength{\Oldarrayrulewidth}{\arrayrulewidth}}%
\noalign{\global\oldsetlength{\arrayrulewidth}{0.04em}}\cline{#1}%
\noalign{\global\oldsetlength{\arrayrulewidth}{\Oldarrayrulewidth}}}%
\newcommand{\crulemedium}[1]%
{\noalign{\global\oldsetlength{\Oldarrayrulewidth}{\arrayrulewidth}}%
\noalign{\global\oldsetlength{\arrayrulewidth}{0.07em}}\cline{#1}%
\noalign{\global\oldsetlength{\arrayrulewidth}{\Oldarrayrulewidth}}}
\newcommand{\crulethick}[1]%
{\noalign{\global\oldsetlength{\Oldarrayrulewidth}{\arrayrulewidth}}%
\noalign{\global\oldsetlength{\arrayrulewidth}{0.11em}}\cline{#1}%
\noalign{\global\oldsetlength{\arrayrulewidth}{\Oldarrayrulewidth}}}
%% Single letter column specifiers defined via array package
\newcolumntype{A}{!{\vrule width 0.04em}}
\newcolumntype{B}{!{\vrule width 0.07em}}
\newcolumntype{C}{!{\vrule width 0.11em}}
\makeatother
%% Figures, Tables, Listings, Floats
%% The [H]ere option of the float package fixes floats in-place,
%% in deference to web usage, where floats are totally irrelevant
%% We re/define the figure, table and listing environments, if used
%%   1) New mbxfigure and/or mbxtable environments are defined with float package
%%   2) Standard LaTeX environments redefined to use new environments
%%   3) Standard LaTeX environments redefined to step theorem counter
%%   4) Counter for new environments is set to the theorem counter before caption
%% You can remove all this figure/table setup, to restore standard LaTeX behavior
%% HOWEVER, numbering of figures/tables AND theorems/examples/remarks, etc
%% WILL ALL de-synchronize with the numbering in the HTML version
%% You can remove the [H] argument of the \newfloat command, to allow flotation and 
%% preserve numbering, BUT the numbering may then appear "out-of-order"
\usepackage{float}
\usepackage[bf]{caption} % http://tex.stackexchange.com/questions/95631/defining-a-new-type-of-floating-environment 
\usepackage{newfloat}
% Figure environment setup so that it no longer floats
\SetupFloatingEnvironment{figure}{fileext=lof,placement={H},within=section,name=Figure}
% figures have the same number as theorems: http://tex.stackexchange.com/questions/16195/how-to-make-equations-figures-and-theorems-use-the-same-numbering-scheme 
\makeatletter
\let\c@figure\c@theorem
\makeatother
%% Raster graphics inclusion, wrapped figures in paragraphs
\usepackage{graphicx}
%% Colors for Sage boxes, author tools (red hilites), red/green edits
\usepackage[usenames,dvipsnames,svgnames,table]{xcolor}
%% More flexible list management, esp. for references and exercises
%% But also for specifying labels (i.e. custom order) on nested lists
\usepackage{enumitem}
%% Lists of exercises in their own section, maximum depth 4
\newlist{exerciselist}{description}{4}
\setlist[exerciselist]{leftmargin=0pt,itemsep=1.0ex,topsep=1.0ex,partopsep=0pt,parsep=0pt}
%% Support for index creation
%% imakeidx package does not require extra pass (as with makeidx)
%% We set the title of the "Index" section via a keyword
%% And we provide language support for the "see" phrase
\usepackage{imakeidx}
\makeindex[title=Index, intoc=true]
\renewcommand{\seename}{see}
%% Package for tables spanning several pages
\usepackage{longtable}
%% hyperref driver does not need to be specified
\usepackage{hyperref}
%% Hyperlinking active in PDFs, all links solid and blue
\hypersetup{colorlinks=true,linkcolor=blue,citecolor=blue,filecolor=blue,urlcolor=blue}
\hypersetup{pdftitle={Discrete Mathematics}}
%% If you manually remove hyperref, leave in this next command
\providecommand\phantomsection{}
%% Graphics Preamble Entries
\usepackage{tikz}

\usetikzlibrary{positioning,matrix,arrows}

\usetikzlibrary{shapes,decorations,shadows,fadings}
%% If tikz has been loaded, replace ampersand with \amp macro
\ifdefined\tikzset
    \tikzset{ampersand replacement = \amp}
\fi
%% extpfeil package for certain extensible arrows,
%% as also provided by MathJax extension of the same name
%% NB: this package loads mtools, which loads calc, which redefines
%%     \setlength, so it can be removed if it seems to be in the 
%%     way and your math does not use:
%%     
%%     \xtwoheadrightarrow, \xtwoheadleftarrow, \xmapsto, \xlongequal, \xtofrom
%%     
%%     we have had to be extra careful with variable thickness
%%     lines in tables, and so also load this package late
\usepackage{extpfeil}
%% Custom Preamble Entries, late (use latex.preamble.late)
%% Begin: Author-provided macros
%% (From  docinfo/macros  element)
%% Plus three from MBX for XML characters
\def\d{\displaystyle}
\def\course{Math 228}
\newcommand{\f}[1]{\mathfrak #1}
\newcommand{\s}[1]{\mathscr #1}
\def\N{\mathbb N}
\def\B{\mathbf{B}}
\def\circleA{(-.5,0) circle (1)}
\def\Z{\mathbb Z}
\def\circleAlabel{(-1.5,.6) node[above]{$A$}}
\def\Q{\mathbb Q}
\def\circleB{(.5,0) circle (1)}
\def\R{\mathbb R}
\def\circleBlabel{(1.5,.6) node[above]{$B$}}
\def\C{\mathbb C}
\def\circleC{(0,-1) circle (1)}
\def\F{\mathbb F}
\def\circleClabel{(.5,-2) node[right]{$C$}}
\def\A{\mathbb A}
\def\twosetbox{(-2,-1.5) rectangle (2,1.5)}
\def\X{\mathbb X}
\def\threesetbox{(-2,-2.5) rectangle (2,1.5)}
\def\E{\mathbb E}
\def\O{\mathbb O}
\def\U{\mathcal U}
\def\pow{\mathcal P}
\def\inv{^{-1}}
\def\nrml{\triangleleft}
\def\st{:}
\def\~{\widetilde}
\def\rem{\mathcal R}
\def\sigalg{$\sigma$-algebra }
\def\Gal{\mbox{Gal}}
\def\iff{\leftrightarrow}
\def\Iff{\Leftrightarrow}
\def\land{\wedge}
\def\And{\bigwedge}
\def\entry{\entry}
\def\AAnd{\d\bigwedge\mkern-18mu\bigwedge}
\def\Vee{\bigvee}
\def\VVee{\d\Vee\mkern-18mu\Vee}
\def\imp{\rightarrow}
\def\Imp{\Rightarrow}
\def\Fi{\Leftarrow}
\def\var{\mbox{var}}
\def\r{.5}
\def\Th{\mbox{Th}}
\def\entry{\entry}
\def\sat{\mbox{Sat}}
\def\con{\mbox{Con}}
\def\iffmodels{\bmodels\models}
\def\dbland{\bigwedge \!\!\bigwedge}
\def\dom{\mbox{dom}}
\def\rng{\mbox{range}}
\DeclareMathOperator{\wgt}{wgt}
\newcommand{\vtx}[2]{node[fill,circle,inner sep=0pt, minimum size=4pt,label=#1:#2]{}}
\newcommand{\va}[1]{\vtx{above}{#1}}
\newcommand{\vb}[1]{\vtx{below}{#1}}
\newcommand{\vr}[1]{\vtx{right}{#1}}
\newcommand{\vl}[1]{\vtx{left}{#1}}
\renewcommand{\v}{\vtx{above}{}}
\def\circleA{(-.5,0) circle (1)}
\def\circleAlabel{(-1.5,.6) node[above]{$A$}}
\def\circleB{(.5,0) circle (1)}
\def\circleBlabel{(1.5,.6) node[above]{$B$}}
\def\circleC{(0,-1) circle (1)}
\def\circleClabel{(.5,-2) node[right]{$C$}}
\def\twosetbox{(-2,-1.4) rectangle (2,1.4)}
\def\threesetbox{(-2.5,-2.4) rectangle (2.5,1.4)}
\def\ansfilename{practice-answers}
\def\shadowprops{{fill=black!50,shadow xshift=0.5ex,shadow yshift=0.5ex,path fading={circle with fuzzy edge 10 percent}}}
\def\sb{.6}
\newcommand{\lt}{ < }
\newcommand{\gt}{ > }
\newcommand{\amp}{ & }
%% End: Author-provided macros
%% Title page information for book
\title{Discrete Mathematics\\
{\large An Open Introduction}}
\author{Oscar Levin\\
School of Mathematical Science\\
University of Northern Colorado
}
\date{July 22, 2016}
\begin{document}
\frontmatter
%% begin: half-title
\thispagestyle{empty}
{\centering
\vspace*{0.28\textheight}
{\Huge Discrete Mathematics}\\[2\baselineskip]
{\LARGE An Open Introduction}\\
}
\clearpage
%% end:   half-title
%% begin: adcard
\thispagestyle{empty}
\null%
\clearpage
%% end:   adcard
%% begin: title page
%% Inspired by Peter Wilson's "titleDB" in "titlepages" CTAN package
\thispagestyle{empty}
{\centering
\vspace*{0.14\textheight}
{\Huge Discrete Mathematics}\\[\baselineskip]
{\LARGE An Open Introduction}\\[3\baselineskip]
{\Large Oscar Levin}\\[0.5\baselineskip]
{\Large University of Northern Colorado}\\[3\baselineskip]
{\Large July 22, 2016}\\}
\clearpage
%% end:   title page
%% begin: copyright-page
\thispagestyle{empty}
\vspace*{\stretch{2}}
\vspace*{\stretch{1}}
\null\clearpage
%% end:   copyright-page
%% begin: acknowledgement
\chapter*{Acknowledgements}\label{acknowledgement-1}
\addcontentsline{toc}{chapter}{Acknowledgements}

  This book would not exist if not for ``Discrete and Combinatorial Mathematics'' by Richard Grassl and Tabitha Mingus. It is the book I learned discrete math out of, and taught out of the semester before I began writing this text. I wanted to maintain the inquiry based feel of their book but update, expand and rearrange some of the material.
  %
\par

  In Spring 2015, Alees Seehausen, a graduate student at the University of Northern Colorado, co-taught the Discrete Mathematics course with me and helped develop many of the \emph{Investigate!} activities and other problems currently used in the text. She also offered many suggestions for improvement of the expository text, for which I am quite grateful. Thanks also to Katie Morrison and Nate Eldredge for their suggestions after using parts of this text in their class.
  %
\par

  Finally, a thank you to the numerous students who have pointed out typos and made suggestions over the years and a thanks in advance to those who will do so in the future.
  %
%% end:   acknowledgement
%% begin: preface
\chapter*{Preface}\label{preface}
\addcontentsline{toc}{chapter}{Preface}

This text aims to give an introduction to select topics in discrete mathematics at a level appropriate for first or second year undergraduate math majors, especially those who intend to teach middle and high school mathematics. The book began as a set of notes for the Discrete Mathematics course at the University of Northern Colorado. This course serves both as a survey of the topics in discrete math and as the ``bridge'' course for math majors, as UNC does not offer a separate ``introduction to proofs'' course. Most students who take the course plan to teach, although there are a handful of students who will go on to graduate school or study applied math or computer science. For these students the current text hopefully is still of interest, but the intent is not to provide a solid mathematical foundation for computer science, unlike the majority of textbooks on the subject.
%
\par

Another difference between this text and most other discrete math books is that this book is intended to be used in a class taught using problem oriented or inquiry based methods. When I teach the class, I will assign sections for reading \emph{after} first introducing them in class by using a mix of group work and class discussion on a few interesting problems. The text is meant to consolidate what we \emph{discover} in class and serve as a reference for students as they master the concepts and techniques covered in the unit. None-the-less, every attempt has been made to make the text sufficient for self study as well, in a way that hopefully mimics an inquiry based classroom.
%
\par

The topics covered in this text were chosen to match the need of the students I teach at UNC. The main areas of study are combinatorics, sequences, logic and proofs, and graph theory, in that order. Induction is covered at the end of the chapter on sequences. Most discrete books put logic first as a preliminary, which certainly has its advantages. However, I wanted to discuss logic and proofs together, and found that doing both of these before anything else was overwhelming for my students given that they didn't yet have context of other problems in the subject. Also, after spending a couple weeks on proofs, we would hardly use that at all when covering combinatorics, so much of the progress we made was quickly lost.
%
\par

Depending on the speed of the class, it might be possible to include additional material. In past semesters I have included generating functions (after sequences) and some basic number theory (either after the logic and proofs chapter or at the very end of the course). These additional topics are covered in appendix A.
%
\par

While I (currently) believe this selection and order of topics is optimal, you should feel free to skip around to what interests you. There are occasionally examples and exercises that rely on earlier material, but I have tried to keep these to a minimum and usually can either be skipped or understood without too much additional study. If you are an instructor, feel free to edit the \LaTeX{}~source to fit your needs.
%
%% begin: preface
\chapter*{How to use this book}\label{preface-2}
\addcontentsline{toc}{chapter}{How to use this book}
\typeout{************************************************}
\typeout{Introduction  }
\typeout{************************************************}

  In addition to expository text, this book has a few features designed to encourage you to interact with the mathematics.
  %
%% begin: preface
\chapter*{\emph{Investigate!} activities}\label{preface-3}
\addcontentsline{toc}{chapter}{\emph{Investigate!} activities}

  Sprinkled throughout the sections (usually at the very beginning of a topic) you will find activities designed to get you acquainted with the topic soon to be discussed. These are similar (sometimes identical) to group activities I give students to introduce material. You really should spend some time thinking about, or even working through, these problems before reading the section. By priming yourself to the types of issues involved in the material you are about to read, you will better understand what is to come. There are no solutions provided for these problems, but don't worry if you can't solve them or are not confident in your answers. My hope is that you will take this frustration with you while you read the proceeding section. By the time you are done with the section, things should be much clearer.
  %
%% end:   preface
%% begin: preface
\chapter*{Examples}\label{preface-4}
\addcontentsline{toc}{chapter}{Examples}

  I have tried to include the ``correct'' number of examples. For those examples which include \emph{problems}, full solutions are included. Before reading the solution, try to at least have an understanding of what the problem is asking. Unlike some textbooks, the examples are not meant to be all inclusive for problems you will see in the exercises. They should not be used as a blueprint for solving other problems. Instead, use the examples to deepen our understanding of the concepts and techniques discussed in each section. Then use this understanding to solve the exercises at the end of each section.
  %
%% end:   preface
%% begin: preface
\chapter*{Exercises}\label{preface-5}
\addcontentsline{toc}{chapter}{Exercises}

  You get good at math through practice. Each section concludes with a small number of exercises meant to solidify concepts and basic skills presented in that section. At the end of each chapter, a larger collection of similar exercises is included (as a sort of ``chapter review'') which might bridge material of different sections in that chapter. Every exercise has either a hint, answer or full solution (which in the pdf version of the text can be found by clicking on the exercises number \textendash{} clicking on the solution number will bring you back to the exercise). Readers are encouraged to try these exercises before looking at the solution. When I teach with this book, I assign these exercises as practice and then use them, or similar problems, on quizzes and exams.
  %
%% end:   preface
%% begin: preface
\chapter*{Homework Problems}\label{preface-6}
\addcontentsline{toc}{chapter}{Homework Problems}

       Each chapter includes a small number of more involved problems \textendash{} the type I would assign as homework to be written up and collected each week. As many of these are problems I assign, solutions are not included. If you are using this book for self study, consider these additional \emph{Investigate!} problems.
    %
%% end:   preface
%% end:   preface
%% begin: preface
\chapter*{Previous and future editions}\label{pref_editions}
\addcontentsline{toc}{chapter}{Previous and future editions}

This current Fall 2015 edition of the text is essentially the first edition of the book. I have previously compiled many of the sections in a book format for easy distribution, but those were mostly just lecture notes and exercises (there was no index or Investigate problems; very little in the way of consistent formatting).
%
\par

My intent is to compile a new edition prior to each Fall semester which incorporate additions and corrections suggested by instructors and students who use the text the previous semester. Thus I encourage you to send along any suggestions and comments as you have them. For future editions, I will keep track of any major changes here.
%
%% end:   preface
\par\hfill\begin{tabular}{l@{}}
Oscar Levin, Ph.D.\\
University of Northern Colorado, 2016
\end{tabular}\\\par
%% end:   preface
%% begin: table of contents
\setcounter{tocdepth}{1}
\renewcommand*\contentsname{Contents}
\tableofcontents
%% end:   table of contents
\mainmatter
\typeout{************************************************}
\typeout{Chapter 1 Introduction and Preliminaries}
\typeout{************************************************}
\chapter[Introduction and Preliminaries]{Introduction and Preliminaries}\label{ch_intro}
\typeout{************************************************}
\typeout{Introduction  }
\typeout{************************************************}

      Welcome to Discrete Mathematics. If this is your first time encountering the subject, you will probably find discrete mathematics quite different from other math subjects. You might not even know what discrete math is! Hopefully this short introduction
      will shed some light on what the subject is about and what you can expect as you move forward in your studies.
    %
\typeout{************************************************}
\typeout{Section 1.1 What is Discrete Mathematics?}
\typeout{************************************************}
\section[What is Discrete Mathematics?]{What is Discrete Mathematics?}\label{sec_intro-intro}
\begin{quote}dis\textperiodcentered{}crete / dis'krët.%
\par
 \emph{Adjective}: Individually separate and distinct.%
\par
\emph{Synonyms}: separate - detached - distinct - abstract.%
\end{quote}

    Defining \emph{discrete mathematics} is hard because defining \emph{mathematics} is hard. What is mathematics? The study of numbers? In part, but you also study functions and lines and triangles and parallelepipeds and vectors and
    \dots{}. Or perhaps you want to say that mathematics is a collection of tools that allow you to solve problems. What sort of problems? Okay, those that involve numbers, functions, lines, triangles,
    \dots{}. Whatever your conception of what mathematics is, try applying the concept of ``discrete'' to it, as defined above. Some math fundamentally deals with
    \dots{} \emph{stuff}
    \dots{} that is individually separate and distinct.
  %
\par

    In an algebra or calculus class, you might have found a particular set of numbers (maybe the set of number in the range of a function). You would represent this set as an interval: \([0,\infty)\) is the range of \(f(x) = x^2\) since the set
    of outputs of the function are all real numbers 0 and greater. This set of numbers is NOT discrete. The numbers in the set are not separated by much at all. In fact, take any two numbers in the set and there are infinitely many more between
    them which are also in the set. Discrete math could still ask about the range of a function, but the set would not be an interval. Consider the function which gives the number of children each person reading this has. What is the range? I'm guessing
    it is something like \(\{0, 1, 2, 3\}\). Maybe 4 is in there too. But certainly there is nobody reading this that has 1.32419 children. This set \emph{is} discrete because the elements are separate. Also notice that the inputs to the function
    are a discrete set as each input is an individual person. You would not consider fractional inputs (there is nothing we care about \(2/3\) between a pair of readers).
  %
\par

    One way to get a feel for the subject is to consider the types of problems you solve in discrete math. Here are a few simple examples:
  %
\begin{activity}[]\label{activity-1}

      Here are a few Discrete Math problems for you to try.
    %
\par

      \emph{Note: Throughout the text you will see \emph{Investigate!} activities like this one. Answer the questions in these as best you can to give yourself a feel for what is coming next.}
    %
\leavevmode%
\begin{enumerate}
\item\hypertarget{li-1}{}
        The most popular mathematician in the world is throwing a party for all of his friends. As a way to kick things off, they decide that everyone should shake hands. Assuming all 10 people at the party each shake hands with every other person (but not themselves,
        obviously) exactly once, how many handshakes take place?
      \item\hypertarget{li-2}{}
        At the warm-up event for Oscar's All Star Hot Dog Eating Contest, Al ate one hot dog. Bob then showed him up by eating three hot dogs. Not to be outdone, Carl ate five. This continued with each contestant eating two more hot dogs than the previous contestant.
        How many hot dogs did Zeno (the 26th and final contestant) eat? How many hot dogs were eaten all together?
      \item\hypertarget{li-3}{}
        While walking through a fictional forest, you encounter three trolls. Each is either a \emph{knight}, who always tells the truth, or a \emph{knave}, who always lies. The trolls will not let you pass until you correctly identify each as either
        a knight or a knave. Each troll makes a single statement:


        \begin{itemize}[label=\textbullet]
\item{}Troll 1: If I am a knave, then there are exactly two knights here.\item{}Troll 2: Troll 1 is lying.\item{}Troll 3: Either we are all knaves or at least one of us is a knight.\end{itemize}



        Which troll is which?
      \item\hypertarget{li-7}{}
        Back in the days of yore, five small towns decided they wanted to build roads directly connecting each pair of towns. While the towns had plenty of money to build roads as long and as winding as they wished, it was very important that the roads not intersect
        with each other (as stop signs had not yet been invented). Also, tunnels and bridges were not allowed. Is it possible for each of these towns to build a road to each of the four other towns without creating any intersections?
      \end{enumerate}
\end{activity}
\par

    One reason it is difficult to define discrete math is that it is a very broad description which encapsulates a large number of subjects. In this course we will study four main topics: \emph{combinatorics} (the theory of ways things \emph{combine};
    in particular, how to count these ways), \emph{sequences}, \emph{logic}, and \emph{graph theory}. However, there are other topics that belong under the discrete umbrella, including computer science, abstract algebra, number theory, game theory,
    probability, and geometry (some of these, particularly the last two, have both discrete and non-discrete variants).
  %
\par

    Ultimately the best way to learn what discrete math is about is to \emph{do} it. Let's get started! Before we can begin answering more complicated (and fun) problems, we must lay down some foundation. We start by reviewing sets and functions in
    the framework of discrete mathematics.
  %
\typeout{************************************************}
\typeout{Section 1.2 Sets}
\typeout{************************************************}
\section[Sets]{Sets}\label{sec_intro-sets}
\typeout{************************************************}
\typeout{Introduction  }
\typeout{************************************************}

      The most fundamental objects we will use in our studies (and really in all of math) are
      \terminology{sets}
      \index{set}. Much of what follows might be review, but it is very important that you are fluent in the language of set theory. Most of the notation we use below is standard, although some might be a little different than what you have seen before.
    %
\par

      For us, a set will simply be an unordered collection of objects. Two examples: we could consider the set of all actors who have played \emph{The Doctor} on \emph{Doctor Who}\index{Doctor Who}, or the
      set of natural numbers between 1 and 10 inclusive. In the first case, Tom Baker is a element (or member) of the set, while Idris Elba, among many others, is not an element of the set. Also, the two examples are of different sets. Two sets are equal
      exactly if they contain the exact same elements.
    %
\typeout{************************************************}
\typeout{Subsection 1.2.1 Notation}
\typeout{************************************************}
\subsection[Notation]{Notation}\label{subsec_notation}

      We need some notation to make talking about sets easier. Consider,
      \begin{equation*}
        A = \{1, 2, 3\}.
      \end{equation*}
    %
\par

      This is read, ``\(A\) is the set containing the elements 1, 2 and 3.'' We use curly braces ``\(\{,~~ \}\)'' to enclose elements of a set. Some more notation:
      \begin{equation*}
        a \in \{a, b, c\}.
      \end{equation*}
    %
\par

      The symbol ``\(\in\)'' is read ``is in'' or ``is an element of.'' Thus the above means that \(a\) is an element of the set containing the letters \(a\), \(b\), and \(c\). Note that this is a true statement. It would also
      be true to say that \(d\) is not in that set:
      \begin{equation*}
        d \not\in \{a, b, c\}.
      \end{equation*}
    %
\par

      Be warned: we write ``\(x \in A\)'' when we wish to express that one of the elements of the set \(A\) is \(x\). For example, consider the set,
      \begin{equation*}
        A = \{1, b, \{x, y, z\}, \emptyset\}.
      \end{equation*}
    %
\par

      This is a strange set, to be sure. It contains four elements: the number 1, the letter b, the set \(\{x,y,z\}\), and the empty set (\(\emptyset = \{ \}\), the set containing no elements). Is \(x\) in \(A\)? The answer is no. None of
      the four elements in \(A\) are the letter \(x\), so we must conclude that \(x \notin A\). Similarly, if we considered the set \(B = \{1,b\}\), then again \(B \notin A\). Even though the elements of \(B\) are also elements of \(A\),
      we cannot say that the \emph{set} \(B\) is one of the things in the collection \(A\).
    %
\par

      If a set is
      \terminology{finite}
      \index{finite}, then we can describe it by simply listing the elements. Infinite sets exists though, so we need to be able to describe them as well. For instance, if we want \(A\) to be the set of all even natural numbers,
      would could write,
      \begin{equation*}
        A = \{0, 2, 4, 6, \ldots\},
      \end{equation*}
      but this is a little imprecise. Better would be
      \begin{equation*}
        A = \{x \in \N \st \exists n\in \N ( x = 2 n)\}.
      \end{equation*}
    %
\par

      Breaking that down: ``\(x \in \N\)'' means \(x\) is in the set \(\N\) \label{notation-1}
 (the set of natural numbers, starting with 0), \(:\) \label{notation-2}
      is read ``such that'' and ``\(\exists n\in \N (x = 2n) \)
      '' is read ``there exists an \(n\) in the natural numbers for which \(x\) is two times \(n\)'' (in other words, \(x\) is even). Slightly easier might be,
      \begin{equation*}
        A = \{x \st \mbox{  is even} \}.
      \end{equation*}
    %
\par

      Note: sometimes people use \(|\) or \(\backepsilon\) for the ``such that'' symbol instead of the colon.
    %
\par

      Defining a set using this sort of notation is very useful, although it takes some practice to read them correctly. It is a way to describe the set of all things that satisfy some condition (the condition is the logical statement after the ``:''      symbol). Here are some more examples. We use the symbols \(\wedge\) for ``and'' and \(\vee\) for ``or'' (which includes the ``or both'' for us)
      \index{connectives!and}
      \index{connectives!or}.
    %
\begin{example}[]\label{example-1}

          Describe each of the following sets both in words and by listing out enough elements to see the pattern.
        %
\leavevmode%
\begin{enumerate}
\item\hypertarget{li-8}{}\(\{x \st x + 3 \in \N\}\).\item\hypertarget{li-9}{}\(\{x \in \N \st x + 3 \in \N\}\).\item\hypertarget{li-10}{}\(\{x \st x \in \N \vee -x \in \N\}\).\item\hypertarget{li-11}{}\(\{x \st x \in \N \wedge -x \in \N\}\).\end{enumerate}
\par\medskip\noindent%
\textbf{Solution.}\quad \leavevmode%
\begin{enumerate}
\item\hypertarget{li-12}{}
              This is the set of all number which are 3 less than a natural number (i.e., that if you add 3 to them, you get a natural number). The set could also be written as \(\{-3, -2, -1, 0, 1, 2, \ldots\}\) (note that 0 is a natural number, so
              \(-3\) is in this set because \(-3 + 3 = 0\)).
            %
\item\hypertarget{li-13}{}
              This is the set of all natural numbers which are 3 less than a natural number. So here we just have \(\{0, 1, 2,3 \ldots\}\).
            %
\item\hypertarget{li-14}{}
              This is the set of all integers
              \index{integers} (positive and negative whole numbers, written \(\Z\)). In other words, \(\{\ldots, -2, -1, 0, 1, 2, \ldots\}\).
            %
\item\hypertarget{li-15}{}
              Here we want all numbers \(x\) such that \(x\) and \(-x\) are natural numbers. There is only one: 0. So we have the set \(\{0\}\).
            %
\end{enumerate}
\end{example}
\par

      We already have a lot of notation, and there is more yet. Below is a handy chart of symbols. Some of these will be discussed in greater detail as we move forward.
    %
\begin{mdframed}[style=assemblage]%
\noindent\textbf{\large Set Theory Notation}\label{assemblage-1}\par\medskip

        \begin{tabular}{lll}
Symbol:&Read:&Example:\tabularnewline\hrulethick
\(\{, \}\)

            &braces&\(\{1,2,3\}\). The braces enclose the elements of a set. This is the set which contains the numbers 1, 2, and 3.\tabularnewline[0pt]

              \(\st\)

            &such that&\(\{x \st x > 2\}\) is the set of all \(x\) such that \(x\) is greater than 2.\tabularnewline[0pt]

              \(\in\)

            &is an element of&\(2 \in \{1,2,3\}\) asserts that 2 is one of the elements in the set \(\{1,2,3\}\). However, \(4 \notin\{1,2,3\}\).\tabularnewline[0pt]

              \(\subseteq\)

            &is a subset of&\(A \subseteq B\) asserts that every element of \(A\) is also an element of \(B\).\tabularnewline[0pt]

              \(\subset\)
            &is a proper subset of&\(A \subset B\) asserts that every element of \(A\) is also an element of \(B\), but \(A \ne B\).\tabularnewline[0pt]

              \(\cap\)

            &intersection

            &\(A \cap B\) is the \emph{set} containing all elements which are elements of both \(A\) and \(B\).\tabularnewline[0pt]

              \(\cup\)

            &union

            &\(A \cup B\) is the \emph{set} containing all elements which are elements of \(A\) or \(B\) or both.\tabularnewline[0pt]

              \(\times\)

            &cross&\(A \times B\) is the set of all ordered pairs \((a,b)\) with \(a \in A\) and \(b \in B\).\tabularnewline[0pt]

              \(\setminus\)

            &set difference&\(A \setminus B\) is the \emph{set} containing all elements of \(A\) which are not elements of \(B\).\tabularnewline[0pt]

              \(\bar{A}\)
            &complement (of \(A\))&\(\bar{A}\) is the set of everything which is not an element of \(A\). The \(A\) can be any set here.\tabularnewline[0pt]

              \(\left|A\right|\)
            &cardinality (of \(A\))&\(|\{4,5,6\}| = 3\) because there are 3 elements in the set. Sometimes we call \(|A|\) the \emph{size} of \(A\).\tabularnewline[0pt]
 \terminology{Logic symbols:}&\tabularnewline[0pt]
\(\wedge\)
            &and&\(x \in A \wedge x \notin B\) means \(x\) is both in the set \(A\) and not in the set \(B\).\tabularnewline[0pt]

              \(\vee\)

            &or&\(x \in A \vee x \notin B\) asserts that \(x\) is an element of \(A\) or not an element of \(B\), or both.\tabularnewline[0pt]

              \(\neg\)

            &not&Another way to write \(x \notin A\) is \(\neg x \in A\).\tabularnewline[0pt]

              \(\forall\)

            &for all&\(\forall x (x \ge 0)\) claims that every number is greater than 0.

            \tabularnewline[0pt]

              \(\exists\)

            &there exists
            &\(\exists x (x \lt  0)\) claims that there is a number less than 0.
\end{tabular}

      %
\end{mdframed}
\begin{mdframed}[style=assemblage]%
\noindent\textbf{\large Special sets}\label{assemblage-2}\par\medskip

        \begin{tabular}{ll}
\(\emptyset\)
            &The \emph{empty set} is the set which contains no elements.\tabularnewline[0pt]

              \(\U\)
            &The \emph{universe set} is the set of all elements.\tabularnewline[0pt]

              \(\N\)
            &The set of natural numbers. That is, \(\N = \{0, 1, 2, 3\ldots\}\).\tabularnewline[0pt]

              \(\Z\)
            &The set of integers. That is, \(\Z = \{\ldots, -2, -1, 0, 1, 2, 3, \ldots\}\).\tabularnewline[0pt]

              \(\Q\)
            &The set of rational numbers.\tabularnewline[0pt]

              \(\R\)
            &The set of real numbers.\tabularnewline[0pt]

              \(\pow(A)\)

            &The \emph{power set} of any set \(A\) is the set of all subsets of \(A\).
\end{tabular}

      %
\end{mdframed}
\begin{activity}[]\label{activity-2}
\leavevmode%
\begin{enumerate}
\item\hypertarget{li-16}{}
            Find the cardinality of each set below.

          \begin{enumerate}
\item\hypertarget{li-17}{}\(A = \{3,4,\ldots, 15\}\).\item\hypertarget{li-18}{}\(B = \{n \in \N \st 2 \lt  n \le 200\}\).\item\hypertarget{li-19}{}\(C = \{n \le 100 \st n \in \N \wedge \exists m \in \N (n = 2m+1)\}\).\end{enumerate}
\item\hypertarget{li-20}{}
          Find two sets \(A\) and \(B\) for which \(|A| = 5\), \(|B| = 6\), and \(|A\cup B| = 9\). What is \(|A \cap B|\)?
      \item\hypertarget{li-21}{}
          Find sets \(A\) and \(B\) with \(|A| = |B|\) such that \(|A\cup B| = 7\) and \(|A \cap B| = 3\). What is \(|A|\)?

        \item\hypertarget{li-22}{}
              Let \(A = \{1,2,\ldots, 10\}\). Define \(\mathcal{B}_2 = \{B \subseteq A \st |B| = 2\}\). Find \(|\mathcal{B}_2|\).
        \item\hypertarget{li-23}{}
            For any sets \(A\) and \(B\), define \(AB = \{ab \st a\in A \wedge b \in B\}\). If \(A = \{1,2\}\) and \(B = \{2,3,4\}\), what is \(|AB|\)? What is \(|A \times B|\)?
        \end{enumerate}
\end{activity}
\typeout{************************************************}
\typeout{Subsection 1.2.2 Relationships Between Sets}
\typeout{************************************************}
\subsection[Relationships Between Sets]{Relationships Between Sets}\label{subsection-2}

      We have already said what it means for two sets to be equal: they have exactly the same elements. Thus, for example,
      \begin{equation*}
        \{1, 2, 3\} = \{2, 1, 3\}.
      \end{equation*}
    %
\par

      (Remember, the order the elements are written down in does not matter.) Also,
      \begin{equation*}
        \{1, 2, 3\} = \{1, 1+1, 1+1+1\} = \{I, II, III\}
      \end{equation*}
      since these are all ways to write the set containing the first three positive integers (how we write them doesn't matter, just what they are).
    %
\par

      What about the sets \(A = \{1, 2, 3\}\) and \(B = \{1, 2, 3, 4\}\)? Clearly \(A \ne B\), but notice that every element of \(A\) is also an element of \(B\). Because of this we say that \(A\) is a \emph{subset}
      \index{subset} of \(B\), or in symbols \(A \subset B\) or \(A \subseteq B\). (Both symbols are read ``is a subset of.'' The difference is that sometimes we want to say that \(A\) is either equal to or a subset of \(B\), in which
      case we use \(\subseteq\). This is analoguous to the difference between \(\lt\) and \(\le\).)
    %
\begin{example}[]\label{example-2}

          Let \(A = \{1, 2, 3, 4, 5, 6\}\), \(B = \{2, 4, 6\}\), \(C = \{1, 2, 3\}\) and \(D = \{7, 8, 9\}\). Determine which of the following are true, false, or meaningless.
        %
\leavevmode%
\begin{enumerate}
\item\hypertarget{li-24}{}\(A \subset B\).\item\hypertarget{li-25}{}\(B \subset A\).\item\hypertarget{li-26}{}\(B \in C\).\item\hypertarget{li-27}{}\(\emptyset \in A\).\item\hypertarget{li-28}{}\(\emptyset \subset A\).\item\hypertarget{li-29}{}\(A \lt  D\).\item\hypertarget{li-30}{}\(3 \in C\).\item\hypertarget{li-31}{}\(3 \subset C\).\item\hypertarget{li-32}{}\(\{3\} \subset C\).\end{enumerate}
\par\medskip\noindent%
\textbf{Solution.}\quad \leavevmode%
\begin{enumerate}
\item\hypertarget{li-33}{}
              False. For example, \(1\in A\) but \(1 \notin B\).
            %
\item\hypertarget{li-34}{}
              True. Every element in \(B\) is an element in \(A\).
            %
\item\hypertarget{li-35}{}
              False. The elements in \(C\) are 1, 2, and 3. The \emph{set} \(B\) is not equal to 1, 2, or 3.
            %
\item\hypertarget{li-36}{}
              False. \(A\) has exactly 6 elements, and none of them are the empty set.
            %
\item\hypertarget{li-37}{}
              True. Everything in the empty set (nothing) is also an element of \(A\). Notice that the empty set is a subset of every set.
            %
\item\hypertarget{li-38}{}
              Meaningless. A set cannot be less than another set.
            %
\item\hypertarget{li-39}{}
              True. \(3\) is one of the elements of the set \(C\).
            %
\item\hypertarget{li-40}{}
              Meaningless. \(3\) is not a set, so it cannot be a subset of another set.
            %
\item\hypertarget{li-41}{}
              True. \(3\) is the only element of the set \(\{3\}\), and is an element of \(C\), so every element in \(\{3\}\) is an element of \(C\).
            %
\end{enumerate}
\end{example}
\par

      In the example above, \(B\) is a subset of \(A\). You might wonder what other sets are subsets of \(A\). If you collect all these subsets of \(A\) into a new set, we get a set of sets. We call the set of all subsets of \(A\) the \emph{power set}
      \index{power set} of \(A\), and write it \(\pow(A)\).
    %
\begin{example}[]\label{example-3}

          Let \(A = \{1,2,3\}\). Find \(\pow(A)\).
        %
\par\medskip\noindent%
\textbf{Solution.}\quad 
          \(\pow(A)\) is a set of sets, all of which are subsets of \(A\). So
          \begin{equation*}
            \pow(A) = \{ \emptyset, \{1\}, \{2\}, \{3\}, \{1,2\}, \{1, 3\}, \{2,3\}, \{1,2,3\}\}.
          \end{equation*}
        %
\par

          Notice that while \(2 \in A\), it is wrong to write \(2 \in \pow(A)\) since none of the elements in \(\pow(A)\) are numbers! On the other hand, we do have \(\{2\} \in \pow(A)\) because \(\{2\} \subseteq A\).
        %
\par

          What does a subset of \(\pow(A)\) look like? Notice that \(\{2\} \not\subseteq \pow(A)\) because not everything in \(\{2\}\) is in \(\pow(A)\). But we do have \(\{ \{2\} \} \subseteq \pow(A)\). The only element of \(\{\{2\}\}\)          is the set \(\{2\}\) which is also an element of \(\pow(A)\). We could take the collection of all subsets of \(\pow(A)\) and call that \(\pow(\pow(A))\). Or even the power set of that set of sets of sets.
        %
\end{example}
\par

      Another way to compare sets is by their size. Notice that in the example above, \(A\) has 6 elements, \(B\), \(C\), and \(D\) all have 3 elements. The size of a set is called the set's \emph{cardinality}
      \index{cardinality}. We would write \(|A| = 6\), \(|B| = 3\), and so on. For sets that have a finite number of elements, the cardinality of the set is simply the number of elements in the set. Note that the cardinality of \(\{ 1, 2, 3, 2, 1\}\)      is 3. We do not count repeats (in fact, \(\{1, 2, 3, 2, 1\}\) is exactly the same set as \(\{1, 2, 3\}\)). There are sets with infinite cardinality, such as \(\N\), the set of rational numbers (written \(\mathbb Q\)), the set of even
      natural numbers, and the set of real numbers (\(\mathbb R\)). It is possible to distinguish between different infinite cardinalities, but that is beyond the scope of this text. For us, a set will either be infinite, or finite; if it is finite,
      the we can determine its cardinality by counting elements.
    %
\begin{example}[]\label{example-4}
\leavevmode%
\begin{enumerate}
\item\hypertarget{li-42}{}
              Find the cardinality of \(A = \{23, 24, \ldots, 37, 38\}\).
            %
\item\hypertarget{li-43}{}
              Find the cardinality of \(B = \{1, \{2, 3, 4\}, \emptyset\}\).
            %
\item\hypertarget{li-44}{}
              If \(C = \{1,2,3\}\), what is the cardinality of \(\pow(C)\)?
            %
\end{enumerate}
\par\medskip\noindent%
\textbf{Solution.}\quad \leavevmode%
\begin{enumerate}
\item\hypertarget{li-45}{}
              Since \(38 - 23 = 15\), we can conclude that the cardinality of the set is \(|A| = 16\) (you need to add one since 23 is included).
            %
\item\hypertarget{li-46}{}
              Here \(|B| = 3\). The three elements are the number 1, the set \(\{2,3,4\}\), and the empty set.
            %
\item\hypertarget{li-47}{}
              We wrote out the elements of the power set \(\pow(C)\) above, and there are 8 elements (each of which is a set). So \(|\pow(C)| = 8\).\footnotemark
            %
\end{enumerate}
\end{example}
\typeout{************************************************}
\typeout{Subsection 1.2.3 Operations On Sets}
\typeout{************************************************}
\subsection[Operations On Sets]{Operations On Sets}\label{subsection-3}

      Is it possible to add two sets? Not really, however there is something similar. If we want to combine two sets to get the collection of objects that are in either set, then we can take the \emph{union}
      \index{union} of the two sets. Symbolically,
      \begin{equation*}
        C = A \cup B,
      \end{equation*}
      read, ``\(C\) is the union of \(A\) and \(B\),'' means that the elements of \(C\) are exactly the elements which are either an element of \(A\) or an element of \(B\) (or an element of both). For example, if \(A = \{1, 2, 3\}\)      and \(B = \{2, 3, 4\}\), then \(A \cup B = \{1, 2, 3, 4\}\).
    %
\par

      The other common operation on sets is \emph{intersection}
      \index{intersection}. We write,
      \begin{equation*}
        C = A \cap B
      \end{equation*}
      and say, ``\(C\) is the intersection of \(A\) and \(B\),'' when the elements in \(C\) are precisely those both in \(A\) and in \(B\). So if \(A = \{1, 2, 3\}\) and \(B = \{2, 3, 4\}\), then \(A \cap B = \{2, 3\}\).
    %
\par

      Often when dealing with sets, we will have some understanding as to what ``everything'' is. Perhaps we are only concerned with natural numbers. In this case we would say that our \emph{universe} is \(\N\). Sometimes we denote this universe
      by \(\U\). Given this context, we might wish to speak of all the elements which are \emph{not} in a particular set. We say \(B\) is the \emph{complement}
      \index{complement} of \(A\), and write,
      \begin{equation*}
        B = \bar A
      \end{equation*}
      when \(B\) contains every element not contained in \(A\). So if our universe is \(\{1, 2,\ldots, 9, 10\}\), and \(A = \{2, 3, 5, 7\}\), then \(\bar A = \{1, 4, 6, 8, 9,10\}\).
    %
\par

      Of course we can perform more than one operation at a time. For example, consider
      \begin{equation*}
        A \cap \bar B.
      \end{equation*}
    %
\par

      This is the set of all elements which are both elements of \(A\) and not elements of \(B\). What have we done? We've started with \(A\) and removed all of the elements which were in \(B\). Another way to write this is the \emph{set difference}
      \index{set difference}
      \index{difference, of sets}:
      \begin{equation*}
        A \cap \bar B = A \setminus B.
      \end{equation*}
    %
\par

      It is important to remember that these operations (union, intersection, complement, and difference) on sets produce other sets. Don't confuse these with the symbols from the previous section (element of and subset of). \(A \cap B\) is a set,
      while \(A \subseteq B\) is true or false. This is the same difference as between \(3 + 2\) (which is a number) and \(3 \le 2\) (which is false).
    %
\begin{example}[]\label{example-5}

          Let \(A = \{1, 2, 3, 4, 5, 6\}\), \(B = \{2, 4, 6\}\), \(C = \{1, 2, 3\}\) and \(D = \{7, 8, 9\}\). If the universe is \(\U = \{1, 2, \ldots, 10\}\), find:
        %
\leavevmode%
\begin{enumerate}
\item\hypertarget{li-48}{}\(A \cup B\).\item\hypertarget{li-49}{}\(A \cap B\).\item\hypertarget{li-50}{}\(B \cap C\).\item\hypertarget{li-51}{}\(A \cap D\).\item\hypertarget{li-52}{}\(\bar{B \cup C}\).\item\hypertarget{li-53}{}\(A \setminus B\).\item\hypertarget{li-54}{}\((D \cap \bar C) \cup \bar{A \cap B}\).\item\hypertarget{li-55}{}\(\emptyset \cup C\).\item\hypertarget{li-56}{}\(\emptyset \cap C\).\end{enumerate}
\par\medskip\noindent%
\textbf{Solution.}\quad \leavevmode%
\begin{enumerate}
\item\hypertarget{li-57}{}\(A \cup B = \{1, 2, 3, 4, 5, 6\} = A\) since everything in \(B\) is already in \(A\).\item\hypertarget{li-58}{}\(A \cap B = \{2, 4, 6\} = B\) since everything in \(B\) is in \(A\).\item\hypertarget{li-59}{}\(B \cap C = \{2\}\) as the only element of both \(B\) and \(C\) is 2.\item\hypertarget{li-60}{}\(A \cap D = \emptyset\) since \(A\) and \(D\) have no common elements.\item\hypertarget{li-61}{}\(\bar{B \cup C} = \{5, 7, 8, 9, 10\}\). First we find that \(B \cup C = \{1, 2, 3, 4, 6\}\), then we take everything not in that set.\item\hypertarget{li-62}{}\(A \setminus B = \{1, 3, 5\}\) since the elements 1, 3, and 5 are in \(A\) but not in \(B\). This is the same as \(A \cap \bar B\).\item\hypertarget{li-63}{}\((D \cap \bar C) \cup \bar{A \cap B} = \{1, 3, 5, 7, 8, 9, 10\}.\) The set contains all elements that are either in \(D\) but not in \(C\) (\(\{7,8,9\}\)), or not in both \(A\) and \(B\) (\(\{1,3,5,7,8,9,10\}\)).\item\hypertarget{li-64}{}\(\emptyset \cup C = C\) since nothing is added by the empty set.\item\hypertarget{li-65}{}\(\emptyset \cap C = \emptyset\) since nothing can be both in a set and in the empty set.\end{enumerate}
\end{example}
\par

      You might notice that the symbols for union and intersection slightly resemble the logic symbols for ``or'' and ``and.'' This is no accident. What does it mean for \(x\) to be an element of \(A\cup B\)? It means that \(x\) is an element
      of \(A\emph{or}x\) is an element of \(B\) (or both). That is,
      \begin{equation*}
        x \in A \cup B \qquad \Iff \qquad x \in A \vee x \in B.
      \end{equation*}
    %
\par

      Similarly,
      \begin{equation*}
        x \in A \cap B \qquad \Iff \qquad x \in A \wedge x \in B.
      \end{equation*}
    %
\par

      Also,
      \begin{equation*}
        x \in \bar A \qquad \Iff \qquad \neg (x \in A).
      \end{equation*}
      which says \(x\) is an element of the complement of \(A\) if \(x\) is not an element of \(A\).
    %
\par

      There is one more way to combine sets which will be useful for us: the \emph{Cartesian product}, \(A \times B\). This sounds fancy but is nothing you haven't seen before. When you graph a function in calculus, you graph it in the Cartesian
      plane. This is the set of all ordered pairs of real numbers \((x,y)\). We can do this for \emph{any} pair of sets, not just the real numbers with themselves.
    %
\par

      Put another way, \(A \times B = \{(a,b) \st a \in A \wedge b \in B\}\). The first coordinate comes from the first set and the second coordinate comes from the second set. Sometimes we will want to take the Cartesian product of a set with itself,
      and this is fine: \(A \times A = \{(a,b) \st a, b \in A\}\) (we might also write \(A^2\) for this set). Notice that in \(A \times A\), we still want \emph{all} ordered pairs, not just the ones where the first and second coordinate are
      the same. We can also take products of 3 or more sets, getting ordered triples, or quadruples, and so on.
    %
\begin{example}[]\label{example-6}

          Let \(A = \{1,2\}\) and \(B = \{3,4,5\}\). Find \(A \times B\) and \(A \times A\). How many elements do you expect to be in \(B \times B\)?
        %
\par\medskip\noindent%
\textbf{Solution.}\quad 
          \(A \times B = \{(1,3), (1,4), (1,5), (2,3), (2,4), (2,5)\}\).
        %
\par

          \(A \times A = A^2 = \{(1,1), (1,2), (2,1), (2,2)\}\).
        %
\par

          \(|B\times B| = 9\). There will be 3 pairs with first coordinate \(3\), three more with first coordinate \(4\), and a final three with first coordinate \(5\).
        %
\end{example}
\typeout{************************************************}
\typeout{Subsection 1.2.4 Venn Diagrams}
\typeout{************************************************}
\subsection[Venn Diagrams]{Venn Diagrams}\label{subsection-4}

      \index{Venn diagram} There is a very nice visual tool we can use to represent operations on sets. Venn diagrams display sets as intersecting circles. We can shade the region we are talking about when we carry out an operation. We can also represent cardinality
      of a particular set by putting the number in the corresponding region.
    %
\leavevmode%
\begin{figure}
\centering
{
          \begin{tikzpicture}[fill=gray!50,scale=0.85]
 \draw[thick] \circleA \circleAlabel \circleB \circleBlabel \twosetbox;
\end{tikzpicture}
}
\end{figure}
{
        \begin{tikzpicture}[scale=.60, fill=gray!50]
 \draw[thick] \circleA \circleAlabel \circleB \circleBlabel \circleC \circleClabel \threesetbox;
\end{tikzpicture}
}
\par

      Each circle represents a set. The rectangle containing the circles represents the universe. To represent combinations of these sets, we shade the corresponding region. For example, we could draw \(A \cap B\) as:
    %
{
        \begin{tikzpicture}[fill=gray!50,scale=0.85]
	\begin{scope}
	\clip \circleA;
	\fill \circleB;
	\end{scope}
 \draw[thick] \circleA \circleAlabel \circleB \circleBlabel \twosetbox;
\end{tikzpicture}
}
\par

      Here is a representation of \(A \cap \bar B\), or equivalently \(A \setminus B\):
    %
{
        \begin{tikzpicture}[fill=gray!50,scale=0.85]
	\begin{scope}
	\clip \twosetbox \circleB;
	\fill \circleA;
	\end{scope}
 \draw[thick] \circleA \circleAlabel \circleB \circleBlabel \twosetbox;
\end{tikzpicture}
}
\par

      A more complicated example is \((B \cap C) \cup (C \cap \bar A)\), as seen below.
    %
{
        \begin{tikzpicture}[fill=gray!50,scale=0.65]
	\fill \circleC;
	\begin{scope}
	    \clip \circleC;
	    \fill[white] \circleA \circleB;
	  \end{scope}
	  \begin{scope}
	  	\clip \circleC;
	  	\fill \circleB;
	  \end{scope}
 \draw[thick] \circleA \circleAlabel \circleB \circleBlabel \circleC \circleClabel \threesetbox;
\end{tikzpicture}
}
\par

      Notice that the shaded regions above could also be arrived at in another way. We could have started with all of \(C\), then excluded the region where \(C\) and \(A\) overlap outside of \(B\). That region is \((A \cap C) \cap \bar B\).
      So the above Venn diagram also represents \(C \cap \bar{\left((A\cap C)\cap \bar B\right)}.\) So using just the picture, we have determined that
      \begin{equation*}
        (B \cap C) \cup (C \cap \bar A) = C \cap \bar{\left((A\cap C)\cap \bar B\right)}.
      \end{equation*}
    %
\typeout{************************************************}
\typeout{Exercises 1.2.4.1 Exercises}
\typeout{************************************************}
\subsubsection[Exercises]{Exercises}\label{exercises-1}
\begin{exerciselist}
\item[1.]\hypertarget{exercise-1}{}
            Let \(A = \{1,2,3,4,5\}\), \(B = \{3,4,5,6,7\}\), and \(C = \{2,3,5\}\).
          %
\leavevmode%
\begin{enumerate}[label=(\alph*)]
\item\hypertarget{li-66}{}
                Find \(A \cap B\).
              %
\item\hypertarget{li-67}{}
                Find \(A \cup B\).
              %
\item\hypertarget{li-68}{}
                Find \(A \setminus B\).
              %
\item\hypertarget{li-69}{}
                Find \(A \cap \bar{(B \cup C)}\).
              %
\item\hypertarget{li-70}{}
                Find \(A \times C\).
              %
\item\hypertarget{li-71}{}
                Is \(C \subseteq A\)? Explain.
              %
\item\hypertarget{li-72}{}
                Is \(C \subseteq B\)? Explain.
              %
\end{enumerate}
\par\smallskip
\par\smallskip
\noindent\textbf{Answer.}\hypertarget{answer-1}{}\quad
\leavevmode%
\begin{enumerate}[label=(\alph*)]
\item\hypertarget{li-73}{}\(A \cap B = \{3,4,5\}\).\item\hypertarget{li-74}{}\(A \cup B = \{1,2,3,4,5,6,7\}\).\item\hypertarget{li-75}{}\(A \setminus B = \{1,2\}\).\item\hypertarget{li-76}{}\(A \cap \bar{(B \cup C)} = \{1\}\).\item\hypertarget{li-77}{}\(A \times B = \{(1,2), (1,3), (1,5), (2,2), (2,3), (2,5), (3,2), (3,3), (3,5), (4,2), (4,3), (4,5), (5,2), (5,3), (5,5)\}\).\item\hypertarget{li-78}{}
                Yes.
              %
\item\hypertarget{li-79}{}
                No.
              %
\end{enumerate}
\item[2.]\hypertarget{exercise-2}{}
            Let \(A = \{x \in \N \st 3 \le x \le 13\}\), \(B = \{x \in \N \st x \mbox{ is even} \}\), and \(C = \{x \in \N \st x \mbox{ is odd} \}\).
          %
\leavevmode%
\begin{enumerate}[label=(\alph*)]
\item\hypertarget{li-80}{}
                Find \(A \cap B\).
              %
\item\hypertarget{li-81}{}
                Find \(A \cup B\).
              %
\item\hypertarget{li-82}{}
                Find \(B \cap C\).
              %
\item\hypertarget{li-83}{}
                Find \(B \cup C\).
              %
\end{enumerate}
\par\smallskip
\par\smallskip
\noindent\textbf{Answer.}\hypertarget{answer-2}{}\quad
\leavevmode%
\begin{enumerate}[label=(\alph*)]
\item\hypertarget{li-84}{}\(A \cap B = \{4,6,8,10,12\}\)\item\hypertarget{li-85}{}\(A \cup B = \{x \in \N \st (3 \le x \le 13) \vee x \mbox{ is even} \}.\) (the set of all natural numbers which are either even or between 3 and 13 inclusive).\item\hypertarget{li-86}{}\(B \cap C = \emptyset\).\item\hypertarget{li-87}{}\(B \cup C = \N\).\end{enumerate}
\item[3.]\hypertarget{exercise-3}{}
            Find an example of sets \(A\) and \(B\) such that \(A\cap B = \{3, 5\}\) and \(A \cup B = \{2, 3, 5, 7, 8\}\).
          %
\par\smallskip
\par\smallskip
\noindent\textbf{Answer.}\hypertarget{answer-3}{}\quad

            For example, \(A = \{2,3,5,7,8\}\) and \(B = \{3,5\}\).
          %
\item[4.]\hypertarget{exercise-4}{}
            Find an example of sets \(A\) and \(B\) such that \(A \subseteq B\) and \(A \in B\).
          %
\par\smallskip
\par\smallskip
\noindent\textbf{Answer.}\hypertarget{answer-4}{}\quad

            Let \(A = \{1,2,3\}\) and \(B = \{1,2,3,4,5,\{1,2,3\}\}\)
          %
\item[5.]\hypertarget{exercise-5}{}
            Recall \(\Z = \{\ldots,-2,-1,0, 1,2,\ldots\}\) (the integers). Let \(\Z^+ = \{1, 2, 3, \ldots\}\) be the positive integers. Let \(2\Z\) be the even integers, \(3\Z\) be the multiples of 3, and so on.
          %
\leavevmode%
\begin{enumerate}[label=(\alph*)]
\item\hypertarget{li-88}{}
                Is \(\Z^+ \subseteq 2\Z\)? Explain.
              %
\item\hypertarget{li-89}{}
                Is \(2\Z \subseteq \Z^+\)? Explain.
              %
\item\hypertarget{li-90}{}
                Find \(2\Z \cap 3\Z\). Describe the set in words, and also in symbols (using a \(\st\) symbol).
              %
\item\hypertarget{li-91}{}
                Express \(\{x \in \Z \st \exists y\in \Z (x = 2y \vee x = 3y)\}\) as a union or intersection of two sets above.
              %
\end{enumerate}
\par\smallskip
\par\smallskip
\noindent\textbf{Answer.}\hypertarget{answer-5}{}\quad
\leavevmode%
\begin{enumerate}[label=(\alph*)]
\item\hypertarget{li-92}{}
                No.
              %
\item\hypertarget{li-93}{}
                No.
              %
\item\hypertarget{li-94}{}\(2\Z \cap 3\Z\) is the set of all integers which are multiples of both 2 and 3 (so multiples of 6). Therefore \(2\Z \cap 3\Z = \{x \in \Z \st \exists y\in \Z(x = 6y)\}\).\item\hypertarget{li-95}{}\(2\Z \cup 3\Z\).\end{enumerate}
\item[6.]\hypertarget{exercise-6}{}
            Let \(A_2\) be the set of all multiples of 2 except for \(2\). Let \(A_3\) be the set of all multiples of 3 except for 3. And so on, so that \(A_n\) is the set of all multiple of \(n\) except for \(n\), for any \(n \ge 2\).
            Describe (in words) the set \(\bar{A_2 \cup A_3 \cup A_4 \cup \cdots}\).
          %
\par\smallskip
\par\smallskip
\noindent\textbf{Answer.}\hypertarget{answer-6}{}\quad

            The set of primes.
          %
\item[7.]\hypertarget{exercise-7}{}
            Draw a Venn diagram to represent each of the following:
          %
\leavevmode%
\begin{enumerate}[label=(\alph*)]
\item\hypertarget{li-96}{}\(A \cup \bar B\)\item\hypertarget{li-97}{}\(\bar{(A \cup B)}\)\item\hypertarget{li-98}{}\(A \cap (B \cup C)\)\item\hypertarget{li-99}{}\((A \cap B) \cup C\)\item\hypertarget{li-100}{}\(\bar A \cap B \cap \bar C\)\item\hypertarget{li-101}{}\((A \cup B) \setminus C\)\end{enumerate}
\par\smallskip
\par\smallskip
\noindent\textbf{Answer.}\hypertarget{answer-7}{}\quad
\leavevmode%
\begin{enumerate}[label=(\alph*)]
\item\hypertarget{li-102}{}\(A \cup \bar B\):
              {
               \begin{tikzpicture}[fill=gray!50]

\fill \circleA;

  \begin{scope}
  \clip \circleB \twosetbox;
  \fill \twosetbox;
  \end{scope}
  \draw[thick] \circleA \circleAlabel \circleB \circleBlabel \twosetbox;
\end{tikzpicture}
}
\item\hypertarget{li-103}{}\(\bar{(A \cup B)}\):
              {
               \begin{tikzpicture}[fill=gray!50]
  \fill \twosetbox;
  \fill[white] \circleA \circleB;
  \draw[thick] \circleA \circleAlabel \circleB \circleBlabel \twosetbox;
\end{tikzpicture}
}
\item\hypertarget{li-104}{}\(A \cap (B \cup C)\):
              {
               \begin{tikzpicture}[fill=gray!50]
\begin{scope}
  \clip \circleA;
  \fill \circleB \circleC;
\end{scope}
\draw[thick] \circleA \circleAlabel \circleB \circleBlabel \circleC \circleClabel \threesetbox;
\end{tikzpicture}
}
\item\hypertarget{li-105}{}\((A \cap B) \cup C\):
              {
               \begin{tikzpicture}[fill=gray!50]
\begin{scope}
  \clip \circleA;
  \fill \circleB;
\end{scope}
\fill \circleC;
\draw[thick] \circleA \circleAlabel \circleB \circleBlabel \circleC \circleClabel \threesetbox;
\end{tikzpicture}
}
\item\hypertarget{li-106}{}\(\bar A \cap B \cap \bar C\):
              {
               \begin{tikzpicture}[fill=gray!50]
\fill \circleB;
\begin{scope}
  \clip \circleB;
  \fill[white] \circleA \circleC;
\end{scope}

\draw[thick] \circleA \circleAlabel \circleB \circleBlabel \circleC \circleClabel \threesetbox;
\end{tikzpicture}
}
\item\hypertarget{li-107}{}\((A \cup B) \setminus C\):
              {
               \begin{tikzpicture}[fill=gray!50]
\fill \circleA;
\fill \circleB;
\fill[white] \circleC;
\draw[thick] \circleA \circleAlabel \circleB \circleBlabel \circleC \circleClabel \threesetbox;
\end{tikzpicture}
}
\end{enumerate}
\item[8.]\hypertarget{exercise-8}{}
            Describe a set in terms of \(A\) and \(B\) which has the following Venn diagram:
          %
{
              \begin{tikzpicture}[fill=gray!50, scale=0.75]
\scope
\clip (-2,-2) rectangle (2,2)
      (1,0) circle (1);
\fill (0,0) circle (1);
\endscope
\scope
\clip (-2,-2) rectangle (2,2)
      (0,0) circle (1);
\fill (1,0) circle (1);
\endscope
\draw[thick] (0,0) circle (1) (-1,.7)  node [text=black,above] {\(A\)}
      (1,0) circle (1) (2,.7)  node [text=black,above] {\(B\)}
      (-1.5,-1.5) rectangle (2.5,1.5);
\end{tikzpicture}
}
\par\smallskip
\par\smallskip
\noindent\textbf{Answer.}\hypertarget{answer-8}{}\quad

            For example, \(A \cup B \cap \bar{(A \cap B)}\). Note that \(\bar{A \cap B}\) would almost work, but also contain the area outside of both circles.
          %
\item[9.]\hypertarget{exercise-9}{}
            Find the following cardinalities:
          %
\leavevmode%
\begin{enumerate}[label=(\alph*)]
\item\hypertarget{li-108}{}\(|A|\) when \(A = \{4,5,6,\ldots,37\}\)\item\hypertarget{li-109}{}\(|A|\) when \(A = \{x \in \Z \st -2 \le x \le 100\}\)\item\hypertarget{li-110}{}\(|A \cap B|\) when \(A = \{x \in \N \st x \le 20\}\) and \(B = \{x \in \N \st x \mbox{ is prime} \}\)\end{enumerate}
\par\smallskip
\par\smallskip
\noindent\textbf{Answer.}\hypertarget{answer-9}{}\quad
\leavevmode%
\begin{enumerate}[label=(\alph*)]
\item\hypertarget{li-111}{}
                34.
              %
\item\hypertarget{li-112}{}
                103.
              %
\item\hypertarget{li-113}{}
                8.
              %
\end{enumerate}
\item[10.]\hypertarget{exercise-10}{}
            Let \(A = \{a, b, c\}\). Find \(\pow(A)\).
          %
\par\smallskip
\par\smallskip
\noindent\textbf{Answer.}\hypertarget{answer-10}{}\quad

            \(\pow(A) = \{\emptyset, \{a\}, \{b\}, \{c\}, \{a,b\}, \{a,c\}, \{b,c\}, \{a,b,c\}\}\).
          %
\item[11.]\hypertarget{exercise-11}{}
            Let \(A = \{1,2,\ldots, 10\}\). How many subsets of \(A\) contain exactly one element (i.e., how many \emph{singleton} subsets are there). How many \emph{doubleton} (containing exactly two elements) are there?
          %
\par\smallskip
\par\smallskip
\noindent\textbf{Answer.}\hypertarget{answer-11}{}\quad

            There are 10 singletons. There are 45 doubletons (because \(45 = 9+8+7+\cdots+2+1\)).
          %
\item[12.]\hypertarget{exercise-12}{}
            Let \(A = \{1,2,3,4,5,6\}\). Find all sets \(B \in \pow(A)\) which have the property \(\{2,3,5\} \subseteq B\).
          %
\par\smallskip
\par\smallskip
\noindent\textbf{Answer.}\hypertarget{answer-12}{}\quad

            \(\{2,3,5\}\), \(\{1,2,3,5\}\), \(\{2,3,4,5\}\), \(\{2,3,5,6\}\), \(\{1,2,3,4,5\}\), \(\{1,2,3,5,6\}\), \(\{2,3,4,5,6\}\), and \(\{1,2,3,4,5,6\}\).
          %
\item[13.]\hypertarget{exercise-13}{}
            Find an example of sets \(A\) and \(B\) such that \(|A| = 4\), \(|B| = 5\), and \(|A \cup B| = 9\).
          %
\par\smallskip
\par\smallskip
\noindent\textbf{Answer.}\hypertarget{answer-13}{}\quad

            For example \(A = \{1,2,3,4\}\) and \(B = \{5,6,7,8,9\}\).
          %
\item[14.]\hypertarget{exercise-14}{}
            Find an example of sets \(A\) and \(B\) such that \(|A| = 3\), \(|B| = 4\), and \(|A \cup B| = 5\).
          %
\par\smallskip
\par\smallskip
\noindent\textbf{Answer.}\hypertarget{answer-14}{}\quad

            For example, \(A = \{1,2,3\}\) and \(B = \{2,3,4,5\}\).
          %
\item[15.]\hypertarget{exercise-15}{}
            Are there sets \(A\) and \(B\) such that \(|A| = |B|\), \(|A\cup B| = 10\), and \(|A\cap B| = 5\)? Explain.
          %
\par\smallskip
\par\smallskip
\noindent\textbf{Answer.}\hypertarget{answer-15}{}\quad

            No. There must be 5 elements in common to both sets. Since there are 10 distinct elements all together in \(A\) and \(B\), this means that between \(A\) and \(B\), there must be 5 elements which they do not have in common (some
            in \(A\) but not in \(B\), some in \(B\) but not in \(A\)). But 5 is odd, so to have \(|A| = |B|\), we would need 7.5 elements in each set, which is impossible.
          %
\item[16.]\hypertarget{exercise-16}{}
            In a regular deck of playing cards there are 26 red cards and 12 face cards. Explain, using sets and what you have learned about cardinalities, why there are only 32 cards which are either red or a face card.
          %
\par\smallskip
\par\smallskip
\noindent\textbf{Answer.}\hypertarget{answer-16}{}\quad

            If \(R\) is the set of red cards and \(F\) is the set of face cards, we want to find \(|R \cup F|\). This is not simply \(|R| + |F|\) because there are 6 cards which are both red and a face card; \(|R \cap F| = 6\). We find
            \(|R \cup F| = 32\).
          %
\end{exerciselist}
%
%% A lineskip in table of contents as transition to appendices, backmatter
\addtocontents{toc}{\vspace{\normalbaselineskip}}
%
%
\appendix
%
\typeout{************************************************}
\typeout{Appendix A Notation}
\typeout{************************************************}
\chapter[Notation]{Notation}\label{appendix-1}
\begin{longtable}[l]{llr}
\textbf{Symbol}&\textbf{Description}&\textbf{Page}\\[1em]
\endfirsthead
\textbf{Symbol}&\textbf{Description}&\textbf{Page}\\[1em]
\endhead
\multicolumn{3}{r}{(Continued on next page)}\\
\endfoot
\endlastfoot
$\N$&The set of natural numbers&\pageref{notation-1}\\
$\st$&``such that''&\pageref{notation-2}\\
$$&&\pageref{notation-3}\\
$\st$&``such that''&\pageref{notation-4}\\
$\in$&``is an element of''&\pageref{notation-5}\\
$\neg$&Logical negation&\pageref{notation-6}\\
$\vee$&Logical ``or''&\pageref{notation-7}\\
$\{, \}$&braces, to contain set elements.&\pageref{notation-8}\\
\end{longtable}
%
\backmatter
%
%
%% The index is here, setup is all in preamble
\printindex
%
\cleardoublepage
\pagestyle{empty}
\vspace*{\stretch{1}}
\centerline{
    This book was authored in MathBook XML.
  %
}
\vspace*{\stretch{2}}
\end{document}