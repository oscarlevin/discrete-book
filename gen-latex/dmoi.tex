%**************************************%
%* Generated from MathBook XML source *%
%*    on 2016-08-03T11:03:09-06:00    *%
%*                                    *%
%*   http://mathbook.pugetsound.edu   *%
%*                                    *%
%**************************************%
\documentclass[10pt,]{book}
%% Load geometry package to allow page margin adjustments
\usepackage{geometry}
\geometry{letterpaper,total={5.0in,9.0in}}
%% Custom Preamble Entries, early (use latex.preamble.early)
%% Inline math delimiters, \(, \), need to be robust
%% 2016-01-31:  latexrelease.sty  supersedes  fixltx2e.sty
%% If  latexrelease.sty  exists, bugfix is in kernel
%% If not, bugfix is in  fixltx2e.sty
%% See:  https://tug.org/TUGboat/tb36-3/tb114ltnews22.pdf
%% and read "Fewer fragile commands" in distribution's  latexchanges.pdf
\IfFileExists{latexrelease.sty}{}{\usepackage{fixltx2e}}
%% Page Layout Adjustments (latex.geometry)
\geometry{papersize={6in,9in}, hmargin={0.75in, 0.75in}, height=7.5in, top=0.75in, twoside, ignoreheadfoot}
%% This LaTeX file may be compiled with pdflatex, xelatex, or lualatex
%% The following provides engine-specific capabilities
%% Generally, xelatex and lualatex will do better languages other than US English
%% You can pick from the conditional if you will only ever use one engine
\usepackage{ifthen}
\usepackage{ifxetex,ifluatex}
\ifthenelse{\boolean{xetex} \or \boolean{luatex}}{%
%% begin: xelatex and lualatex-specific configuration
%% fontspec package will make Latin Modern (lmodern) the default font
\ifxetex\usepackage{xltxtra}\fi
\usepackage{fontspec}
%% realscripts is the only part of xltxtra relevant to lualatex 
\ifluatex\usepackage{realscripts}\fi
%% 
%% Extensive support for other languages
\usepackage{polyglossia}
\setdefaultlanguage{english}
%% Magyar (Hungarian)
\setotherlanguage{magyar}
%% Spanish
\setotherlanguage{spanish}
%% Vietnamese
\setotherlanguage{vietnamese}
%% end: xelatex and lualatex-specific configuration
}{%
%% begin: pdflatex-specific configuration
%% translate common Unicode to their LaTeX equivalents
%% Also, fontenc with T1 makes CM-Super the default font
%% (\input{ix-utf8enc.dfu} from the "inputenx" package is possible addition (broken?)
\usepackage[T1]{fontenc}
\usepackage[utf8]{inputenc}
%% end: pdflatex-specific configuration
}
%% Monospace font: Inconsolata (zi4)
%% Sponsored by TUG: http://levien.com/type/myfonts/inconsolata.html
%% See package documentation for excellent instructions
%% One caveat, seem to need full file name to locate OTF files
%% Loads the "upquote" package as needed, so we don't have to
%% Upright quotes might come from the  textcomp  package, which we also use
%% We employ the shapely \ell to match Google Font version
%% pdflatex: "varqu" option produces best upright quotes
%% xelatex,lualatex: add StylisticSet 1 for shapely \ell
%% xelatex,lualatex: add StylisticSet 2 for plain zero
%% xelatex,lualatex: we add StylisticSet 3 for upright quotes
%% 
\ifthenelse{\boolean{xetex} \or \boolean{luatex}}{%
%% begin: xelatex and lualatex-specific monospace font
\usepackage{zi4}
\setmonofont[BoldFont=Inconsolatazi4-Bold.otf,StylisticSet={1,3}]{Inconsolatazi4-Regular.otf}
%% end: xelatex and lualatex-specific monospace font
}{%
%% begin: pdflatex-specific monospace font
\usepackage[varqu]{zi4}
%% end: pdflatex-specific monospace font
}
%% Symbols, align environment, bracket-matrix
\usepackage{amsmath}
\usepackage{amssymb}
%% allow more columns to a matrix
%% can make this even bigger by overriding with  latex.preamble.late  processing option
\setcounter{MaxMatrixCols}{30}
%%
%% Color support, xcolor package
%% Always loaded.  Used for:
%% mdframed boxes, add/delete text, author tools
\PassOptionsToPackage{usenames,dvipsnames,svgnames,table}{xcolor}
\usepackage{xcolor}
%%
%% Semantic Macros
%% To preserve meaning in a LaTeX file
%% Only defined here if required in this document
%% Used for inline definitions of terms
\newcommand{\terminology}[1]{\textbf{#1}}
%% Subdivision Numbering, Chapters, Sections, Subsections, etc
%% Subdivision numbers may be turned off at some level ("depth")
%% A section *always* has depth 1, contrary to us counting from the document root
%% The latex default is 3.  If a larger number is present here, then
%% removing this command may make some cross-references ambiguous
%% The precursor variable $numbering-maxlevel is checked for consistency in the common XSL file
\setcounter{secnumdepth}{3}
%% Environments with amsthm package
%% Theorem-like environments in "plain" style, with or without proof
\usepackage{amsthm}
\theoremstyle{plain}
%% Numbering for Theorems, Conjectures, Examples, Figures, etc
%% Controlled by  numbering.theorems.level  processing parameter
%% Always need a theorem environment to set base numbering scheme
%% even if document has no theorems (but has other environments)
\newtheorem{theorem}{Theorem}[section]
%% Only variants actually used in document appear here
%% Style is like a theorem, and for statements without proofs
%% Numbering: all theorem-like numbered consecutively
%% i.e. Corollary 4.3 follows Theorem 4.2
%% Example-like environments, normal text
%% Numbering is in sync with theorems, etc
\theoremstyle{definition}
\newtheorem{example}[theorem]{Example}
%% Numbering for Projects (independent of others)
%% Controlled by  numbering.projects.level  processing parameter
%% Always need a project environment to set base numbering scheme
%% even if document has no projectss (but has other blocks)
\newtheorem{project}{Project}[section]
%% Project-like environments, normal text
\theoremstyle{definition}
\newtheorem{investigation}[project]{\emph{Investigate!}}
%% assemblage: minimally structured content, high visibility presentation
%% Package for breakable highlight boxes
\usepackage[framemethod=tikz]{mdframed}
%% assemblage environment and style
\newenvironment{assemblage}[1]{\mdfsetup{frametitle={\colorbox{blue!20}{\space#1\space}},%
frametitlealignment={\hspace*{1ex}}, frametitleaboveskip=-1.5ex, frametitlebelowskip=0pt,%
roundcorner=1pt, leftmargin=3pt, rightmargin=3pt, backgroundcolor=blue!5,%
linecolor=blue!75!black,} \begin{mdframed}}{\end{mdframed}}
%% Miscellaneous environments, normal text
%% Numbering for inline exercises and lists is in sync with theorems, etc
\theoremstyle{definition}
\newtheorem{exercise}[theorem]{Exercise}
%% Localize LaTeX supplied names (possibly none)
\renewcommand*{\proofname}{Proof}
\renewcommand*{\appendixname}{Appendix}
\renewcommand*{\chaptername}{Chapter}
%% Equation Numbering
%% Controlled by  numbering.equations.level  processing parameter
\numberwithin{equation}{section}
%% For improved tables
\usepackage{array}
%% Some extra height on each row is desirable, especially with horizontal rules
%% Increment determined experimentally
\setlength{\extrarowheight}{0.2ex}
%% Define variable thickness horizontal rules, full and partial
%% Thicknesses are 0.03, 0.05, 0.08 in the  booktabs  package
\makeatletter
\newcommand{\hrulethin}  {\noalign{\hrule height 0.04em}}
\newcommand{\hrulemedium}{\noalign{\hrule height 0.07em}}
\newcommand{\hrulethick} {\noalign{\hrule height 0.11em}}
%% We preserve a copy of the \setlength package before other
%% packages (extpfeil) get a chance to load packages that redefine it
\let\oldsetlength\setlength
\newlength{\Oldarrayrulewidth}
\newcommand{\crulethin}[1]%
{\noalign{\global\oldsetlength{\Oldarrayrulewidth}{\arrayrulewidth}}%
\noalign{\global\oldsetlength{\arrayrulewidth}{0.04em}}\cline{#1}%
\noalign{\global\oldsetlength{\arrayrulewidth}{\Oldarrayrulewidth}}}%
\newcommand{\crulemedium}[1]%
{\noalign{\global\oldsetlength{\Oldarrayrulewidth}{\arrayrulewidth}}%
\noalign{\global\oldsetlength{\arrayrulewidth}{0.07em}}\cline{#1}%
\noalign{\global\oldsetlength{\arrayrulewidth}{\Oldarrayrulewidth}}}
\newcommand{\crulethick}[1]%
{\noalign{\global\oldsetlength{\Oldarrayrulewidth}{\arrayrulewidth}}%
\noalign{\global\oldsetlength{\arrayrulewidth}{0.11em}}\cline{#1}%
\noalign{\global\oldsetlength{\arrayrulewidth}{\Oldarrayrulewidth}}}
%% Single letter column specifiers defined via array package
\newcolumntype{A}{!{\vrule width 0.04em}}
\newcolumntype{B}{!{\vrule width 0.07em}}
\newcolumntype{C}{!{\vrule width 0.11em}}
\makeatother
%% Figures, Tables, Listings, Floats
%% The [H]ere option of the float package fixes floats in-place,
%% in deference to web usage, where floats are totally irrelevant
%% We re/define the figure, table and listing environments, if used
%%   1) New mbxfigure and/or mbxtable environments are defined with float package
%%   2) Standard LaTeX environments redefined to use new environments
%%   3) Standard LaTeX environments redefined to step theorem counter
%%   4) Counter for new environments is set to the theorem counter before caption
%% You can remove all this figure/table setup, to restore standard LaTeX behavior
%% HOWEVER, numbering of figures/tables AND theorems/examples/remarks, etc
%% WILL ALL de-synchronize with the numbering in the HTML version
%% You can remove the [H] argument of the \newfloat command, to allow flotation and 
%% preserve numbering, BUT the numbering may then appear "out-of-order"
\usepackage{float}
\usepackage[bf]{caption} % http://tex.stackexchange.com/questions/95631/defining-a-new-type-of-floating-environment 
\usepackage{newfloat}
% Figure environment setup so that it no longer floats
\SetupFloatingEnvironment{figure}{fileext=lof,placement={H},within=section,name=Figure}
% figures have the same number as theorems: http://tex.stackexchange.com/questions/16195/how-to-make-equations-figures-and-theorems-use-the-same-numbering-scheme 
\makeatletter
\let\c@figure\c@theorem
\makeatother
% Table environment setup so that it no longer floats
\SetupFloatingEnvironment{table}{fileext=lot,placement={H},within=section,name=Table}
% tables have the same number as theorems: http://tex.stackexchange.com/questions/16195/how-to-make-equations-figures-and-theorems-use-the-same-numbering-scheme 
\makeatletter
\let\c@table\c@theorem
\makeatother
%% Footnote Numbering
%% We reset the footnote counter, as given by numbering.footnotes.level
\makeatletter\@addtoreset{footnote}{section}\makeatother
%% Raster graphics inclusion, wrapped figures in paragraphs
%% \resizebox sometimes used for images in side-by-side layout
\usepackage{graphicx}
%%
%% More flexible list management, esp. for references and exercises
%% But also for specifying labels (i.e. custom order) on nested lists
\usepackage{enumitem}
%% Lists of exercises in their own section, maximum depth 4
\newlist{exerciselist}{description}{4}
\setlist[exerciselist]{leftmargin=0pt,itemsep=1.0ex,topsep=1.0ex,partopsep=0pt,parsep=0pt}
%% Support for index creation
%% imakeidx package does not require extra pass (as with makeidx)
%% We set the title of the "Index" section via a keyword
%% And we provide language support for the "see" phrase
\usepackage{imakeidx}
\makeindex[title=Index, intoc=true]
\renewcommand{\seename}{see}
%% Package for tables spanning several pages
\usepackage{longtable}
%% hyperref driver does not need to be specified
\usepackage{hyperref}
%% Hyperlinking active in PDFs, all links solid and blue
\hypersetup{colorlinks=true,linkcolor=blue,citecolor=blue,filecolor=blue,urlcolor=blue}
\hypersetup{pdftitle={Discrete Mathematics}}
%% If you manually remove hyperref, leave in this next command
\providecommand\phantomsection{}
%% Graphics Preamble Entries
\usepackage{tikz, pgfplots}

\usetikzlibrary{positioning,matrix,arrows}

\usetikzlibrary{shapes,decorations,shadows,fadings,patterns}

\usepackage{skak} %for chessboards etc.
%% If tikz has been loaded, replace ampersand with \amp macro
\ifdefined\tikzset
    \tikzset{ampersand replacement = \amp}
\fi
%% NB: calc redefines \setlength
\usepackage{calc}
%% used repeatedly for vertical dimensions of sidebyside panels
\newlength{\panelmax}
%% extpfeil package for certain extensible arrows,
%% as also provided by MathJax extension of the same name
%% NB: this package loads mtools, which loads calc, which redefines
%%     \setlength, so it can be removed if it seems to be in the 
%%     way and your math does not use:
%%     
%%     \xtwoheadrightarrow, \xtwoheadleftarrow, \xmapsto, \xlongequal, \xtofrom
%%     
%%     we have had to be extra careful with variable thickness
%%     lines in tables, and so also load this package late
\usepackage{extpfeil}
%% Custom Preamble Entries, late (use latex.preamble.late)
%This should load all the style information that mbx does not.
%%%  This is a set of styles to include in the preamble of the generated latex code for the book Discrete Mathematics: an Open Introduction.


\usepackage{bold-extra}
\usepackage{marvosym} %for stop signs.
\usepackage{textcomp}
\usepackage{multicol}


% % FONT OPTIONS (pick one group to uncomment):

%newpx is my current favorite.  This should work on a TEXLive distribution, but on MiKTeX it initially gave me problems.  Running the update wizard to update MiKTeX then showed ``newpx'' as an available package (only since 8/11/15).  Perhaps just synchronizing in the package manager would do the same thing.  Even then, needed to run initexmf --mkmaps from the command line.
%You could always uncomment all of these to use the default computer modern font.
\usepackage[utf8]{inputenc}
\usepackage[T1]{fontenc}
\usepackage{newpxtext}
\usepackage[vvarbb,cmintegrals,cmbraces,bigdelims]{newpxmath}
\usepackage[scr=rsfso]{mathalfa}% \mathscr is fancier than \mathcal
\linespread{1.04}         % adds more leading (space between lines)
% quantifiers look strange, so change those back to normal:
	\DeclareSymbolFont{mysymbols}{OMS}{cmsy}{b}{n} %note we make the figures bold to better match newpx.  Replace the ``b'' with an ``m'' to undo this.
	%\SetSymbolFont{mysymbols}  {bold}{OMS}{cmsy}{b}{n}
	%\DeclareSymbolFont{myoperators}   {OT1}{cmr} {m}{n}
	%\SetSymbolFont{myoperators}{bold}{OT1}{cmr} {bx}{n}
	\DeclareMathSymbol{\forall}{\mathord}{mysymbols}{"38}
	\DeclareMathSymbol{\exists}{\mathord}{mysymbols}{"39}
	%\DeclareMathSymbol{\pm}{\mathbin}{mysymbols}{"06}
	%\DeclareMathSymbol{+}{\mathbin}{myoperators}{"2B}
	%\DeclareMathSymbol{-}{\mathbin}{mysymbols}{"00}
	%\DeclareMathSymbol{=}{\mathrel}{myoperators}{"3D}



%\usepackage[T1]{fontenc}
%\usepackage{newtxtext,newtxmath}


%\usepackage[bitstream-charter]{mathdesign}
%\usepackage[T1]{fontenc}

%\usepackage[proportional,space,scaled=1.064]{erewhon}
%\usepackage[erewhon,vvarbb,bigdelims]{newtxmath}
%\usepackage[T1]{fontenc}
%\renewcommand*\oldstylenums[1]{\textosf{#1}}





% % % % % % Other packages % % % % % % % %

\usepackage{docmute}
\usepackage{pdfpages}

\usepackage{svg}

\usepackage[framemethod=tikz]{mdframed}



% % % % % % % % % % % % % %  END OF PACKAGES  % % % % % % % % % % % % % % % % % % %

%%%%%%%%%%%%%%%%%%%%%%%%%%%%%%%%%%%%%%%%%%%%%%%%%%%%%%%%%%%%%%%%%%%%%%%%
%%%%%%%%%%%%%%%%%  Nicely styled environments: %%%%%%%%%%%%%%%%%%%%%%%%%
%%%%%%%%%%%%%%%%%%%%%%%%%%%%%%%%%%%%%%%%%%%%%%%%%%%%%%%%%%%%%%%%%%%%%%%%

%create a mdframe style for examples:
\mdfdefinestyle{examplestyle}{%
	leftmargin=1.5ex, %left/right margins for oneside
	rightmargin=1.5ex,
	outermargin=1.5ex, %inner/outer margins for twoside
	innermargin=1.5ex,
	innertopmargin=0pt,
	innerbottommargin=1.5ex,
	skipbelow=1.5em,
	roundcorner=0pt,
	topline=false,
	rightline=false,
  frametitleaboveskip=2pt,
	frametitlebelowskip=2pt,
	needspace=3em,
}

%Create indented ``example'' environment:
\mdtheorem[style=examplestyle]{mdexample}[theorem]{Example}

%Redefine ``example'' to call the new mdexample, and put a paragraph break after it.
\renewenvironment{example}{%
\mdexample
}{%
\endmdexample\par
}


% \LetLtxMacro\mdexample\oldexample
% \let\endmdexample\endoldexample
%
% \newenvironment{mdexample}{%
% \oldexample
% }{%
% \endoldexample\par
% }

%Create ``investigation'' environment for in-text worksheet type problems:
\renewenvironment{investigation}{%
\mdfsetup{%
	frametitle={\colorbox{white}{\large\space\space \textit{Investigate!} \space} },
	frametitlealignment=\centering,
	frametitleaboveskip=-1.5ex,
	frametitlebelowskip=0pt,
	frametitlealignment={\hspace*{2pt}},
	innerbottommargin=2em,
	innermargin=1ex,
	outermargin=1ex,
	leftmargin=1ex,
	rightmargin=1ex,
%	topline=false,
%	bottomline=false,
	linecolor=green!50!black,
	linewidth=1pt,
	skipabove=2em,
	skipbelow=2em,
	roundcorner=15pt,
	needspace=4em,
}
\vskip 1em
\begin{mdframed}
}{%
\end{mdframed}\vspace{-4em} \begin{center}\colorbox{white}{ \space\space\quad {\LARGE \color{red} \Stopsign} \quad \textbf{Attempt the above activity before proceeding} \quad {\LARGE\color{red} \Stopsign}\quad\space\space }  \end{center}\par
}


% redefine the assemblage environment for minor tweaking %
\renewenvironment{assemblage}[1]{%
	\mdfsetup{%
	frametitle={\colorbox{blue!20}{\space#1\space}},
	% frametitlealignment={\hspace*{1ex}},
	frametitleaboveskip=-1.5ex,
	frametitlebelowskip=1ex,
	skipabove=2ex,
	skipbelow=1em,
	roundcorner=1pt,
	leftmargin=3pt,
	rightmargin=3pt,
	backgroundcolor=blue!5,
	linecolor=blue!30,
	linewidth=1.5pt,
	innertopmargin=-.5ex,
	innerbottommargin=1.15em,
	needspace=2em,
}
\begin{mdframed}\par\medskip
}{%
\end{mdframed}\par
}

%%%%%%%%%%%%%%%%%  End environments %%%%%%%%%%%%%%%%%%%%%%%%%%



% use QED to end proofs:
\renewcommand{\qedsymbol}{\textsc{qed}}


%%%%%%%% Set description list spacing the way I want %%%%%%%%%%%%%%%
\setlist[description]{style=nextline, leftmargin=3em}


%%%%%%% Set chapters to start at 0 %%%%%%%%%%%%%%%%%%%%%%%%%%%
\setcounter{chapter}{-1}


%Fix widows and orphans (single lines at top/bottom of page):
\clubpenalty=10000
\widowpenalty=10000
\raggedbottom



%%%%%%%%%%%%%%%%%%%%%%%%%%%%%%%%%%%%%%%%%%
%%%%%%%  Headers and footers %%%%%%%%%%%%%
%%%%%%%%%%%%%%%%%%%%%%%%%%%%%%%%%%%%%%%%%%

\usepackage{fancyhdr}
\pagestyle{fancy}
\renewcommand{\chaptermark}[1]{\markboth{\thechapter.\ #1}{}} %Removes word "chapter" from the \leftmark.

\fancyhead{} % clear header fields
\fancyhead[LE]{{\footnotesize \textsl{\thepage}}~~ \textsc{\scriptsize \nouppercase{\leftmark}}}
\fancyhead[RO]{\textsc{\scriptsize \nouppercase{\rightmark}} ~~ {\footnotesize \textsl{\thepage}}  }
\fancyfoot{}





%%%%%%%%%%%%%%%%%%%%%%%%%%%%%%%%%%%%%%%%%%
%%%%%%%    Chapter headings  %%%%%%%%%%%%%
%%%%%%%%%%%%%%%%%%%%%%%%%%%%%%%%%%%%%%%%%%

\usepackage[bf,sc,center,outermarks]{titlesec}


%%%%% FIX FOR BUG IN TITLESEC %%%%%%%%%%%
\usepackage{etoolbox}

\makeatletter
\patchcmd{\ttlh@hang}{\parindent\z@}{\parindent\z@\leavevmode}{}{}
\patchcmd{\ttlh@hang}{\noindent}{}{}{}
\makeatother
%%%%% END FIX %%%%%%%%%%%%%%%%%%%%%%%%%%%%


\titleformat{\chapter}[display]
	{\Large\filcenter}
	{\rule[4pt]{.3\textwidth}{2pt} \hspace{2ex} \large\textsc{\chaptertitlename} \thechapter \hspace{3ex} \rule[4pt]{0.3\textwidth}{2pt} }
	{1pc}
	{\titlerule\vspace{1ex}\huge\textsc}
	[\vspace{.75ex}\titlerule]
\titlespacing*{\chapter}{0pt}{-2em}{2em}

\titleformat{\paragraph}[block]
  {\normalfont\bfseries\filcenter}
  {\theparagraph}
  {}
  {\textsc}
%%%%%%%%%  End chapter/sectio headings %%%%%%%%%%%%%%%%%

%%% End of File.


%% Begin: Author-provided macros
%% (From  docinfo/macros  element)
%% Plus three from MBX for XML characters
\def\d{\displaystyle}
\def\course{Math 228}
\newcommand{\f}[1]{\mathfrak #1}
\newcommand{\s}[1]{\mathscr #1}
\def\N{\mathbb N}
\def\B{\mathbf{B}}
\def\circleA{(-.5,0) circle (1)}
\def\Z{\mathbb Z}
\def\circleAlabel{(-1.5,.6) node[above]{$A$}}
\def\Q{\mathbb Q}
\def\circleB{(.5,0) circle (1)}
\def\R{\mathbb R}
\def\circleBlabel{(1.5,.6) node[above]{$B$}}
\def\C{\mathbb C}
\def\circleC{(0,-1) circle (1)}
\def\F{\mathbb F}
\def\circleClabel{(.5,-2) node[right]{$C$}}
\def\A{\mathbb A}
\def\twosetbox{(-2,-1.5) rectangle (2,1.5)}
\def\X{\mathbb X}
\def\threesetbox{(-2,-2.5) rectangle (2,1.5)}
\def\E{\mathbb E}
\def\O{\mathbb O}
\def\U{\mathcal U}
\def\pow{\mathcal P}
\def\inv{^{-1}}
\def\nrml{\triangleleft}
\def\st{:}
\def\~{\widetilde}
\def\rem{\mathcal R}
\def\sigalg{$\sigma$-algebra }
\def\Gal{\mbox{Gal}}
\def\iff{\leftrightarrow}
\def\Iff{\Leftrightarrow}
\def\land{\wedge}
\def\And{\bigwedge}
\def\entry{\entry}
\def\AAnd{\d\bigwedge\mkern-18mu\bigwedge}
\def\Vee{\bigvee}
\def\VVee{\d\Vee\mkern-18mu\Vee}
\def\imp{\rightarrow}
\def\Imp{\Rightarrow}
\def\Fi{\Leftarrow}
\def\var{\mbox{var}}
\def\Th{\mbox{Th}}
\def\entry{\entry}
\def\sat{\mbox{Sat}}
\def\con{\mbox{Con}}
\def\iffmodels{\bmodels\models}
\def\dbland{\bigwedge \!\!\bigwedge}
\def\dom{\mbox{dom}}
\def\rng{\mbox{range}}
\DeclareMathOperator{\wgt}{wgt}
\newcommand{\vtx}[2]{node[fill,circle,inner sep=0pt, minimum size=4pt,label=#1:#2]{}}
\newcommand{\va}[1]{\vtx{above}{#1}}
\newcommand{\vb}[1]{\vtx{below}{#1}}
\newcommand{\vr}[1]{\vtx{right}{#1}}
\newcommand{\vl}[1]{\vtx{left}{#1}}
\renewcommand{\v}{\vtx{above}{}}
\def\circleA{(-.5,0) circle (1)}
\def\circleAlabel{(-1.5,.6) node[above]{$A$}}
\def\circleB{(.5,0) circle (1)}
\def\circleBlabel{(1.5,.6) node[above]{$B$}}
\def\circleC{(0,-1) circle (1)}
\def\circleClabel{(.5,-2) node[right]{$C$}}
\def\twosetbox{(-2,-1.4) rectangle (2,1.4)}
\def\threesetbox{(-2.5,-2.4) rectangle (2.5,1.4)}
\def\ansfilename{practice-answers}
\def\shadowprops{{fill=black!50,shadow xshift=0.5ex,shadow yshift=0.5ex,path fading={circle with fuzzy edge 10 percent}}}
\def\sb{.6}
\newcommand{\hexbox}[3]{
  \def\x{-cos{30}*\r*#1+cos{30}*#2*\r*2}
  \def\y{-\r*#1-sin{30}*\r*#1}
  \draw (\x,\y) +(90:\r) -- +(30:\r) -- +(-30:\r) -- +(-90:\r) -- +(-150:\r) -- +(150:\r) -- cycle;
  \draw (\x,\y) node{#3};
}
\renewcommand{\bar}{\overline}
\newcommand{\card}[1]{\left| #1 \right|}
\newcommand{\twoline}[2]{\begin{pmatrix}#1 \\ #2 \end{pmatrix}}
\newcommand{\lt}{ < }
\newcommand{\gt}{ > }
\newcommand{\amp}{ & }
%% End: Author-provided macros
%% Title page information for book
\title{Discrete Mathematics\\
{\large An Open Introduction}}
\author{Oscar Levin\\
School of Mathematical Science\\
University of Northern Colorado
}
\date{August 3, 2016}
\begin{document}
\frontmatter
%% begin: half-title
\thispagestyle{empty}
{\centering
\vspace*{0.28\textheight}
{\Huge Discrete Mathematics}\\[2\baselineskip]
{\LARGE An Open Introduction}\\
}
\clearpage
%% end:   half-title
%% begin: adcard
\thispagestyle{empty}
\null%
\clearpage
%% end:   adcard
%% begin: title page
%% Inspired by Peter Wilson's "titleDB" in "titlepages" CTAN package
\thispagestyle{empty}
{\centering
\vspace*{0.14\textheight}
{\Huge Discrete Mathematics}\\[\baselineskip]
{\LARGE An Open Introduction}\\[3\baselineskip]
{\Large Oscar Levin}\\[0.5\baselineskip]
{\Large University of Northern Colorado}\\[3\baselineskip]
{\Large August 3, 2016}\\}
\clearpage
%% end:   title page
%% begin: copyright-page
\thispagestyle{empty}
\vspace*{\stretch{2}}
\vspace*{\stretch{1}}
\null\clearpage
%% end:   copyright-page
%% begin: acknowledgement
\chapter*{Acknowledgements}\label{acknowledgement-1}
\addcontentsline{toc}{chapter}{Acknowledgements}

  This book would not exist if not for ``Discrete and Combinatorial Mathematics'' by Richard Grassl and Tabitha Mingus. It is the book I learned discrete math out of, and taught out of the semester before I began writing this text. I wanted to maintain the inquiry based feel of their book but update, expand and rearrange some of the material.
  %
\par

  In Spring 2015, Alees Seehausen, a graduate student at the University of Northern Colorado, co-taught the Discrete Mathematics course with me and helped develop many of the \emph{Investigate!} activities and other problems currently used in the text. She also offered many suggestions for improvement of the expository text, for which I am quite grateful. Thanks also to Katie Morrison and Nate Eldredge for their suggestions after using parts of this text in their class.
  %
\par

  Finally, a thank you to the numerous students who have pointed out typos and made suggestions over the years and a thanks in advance to those who will do so in the future.
  %
%% end:   acknowledgement
%% begin: preface
\chapter*{Preface}\label{preface}
\addcontentsline{toc}{chapter}{Preface}

This text aims to give an introduction to select topics in discrete mathematics at a level appropriate for first or second year undergraduate math majors, especially those who intend to teach middle and high school mathematics. The book began as a set of notes for the Discrete Mathematics course at the University of Northern Colorado. This course serves both as a survey of the topics in discrete math and as the ``bridge'' course for math majors, as UNC does not offer a separate ``introduction to proofs'' course. Most students who take the course plan to teach, although there are a handful of students who will go on to graduate school or study applied math or computer science. For these students the current text hopefully is still of interest, but the intent is not to provide a solid mathematical foundation for computer science, unlike the majority of textbooks on the subject.
%
\par

Another difference between this text and most other discrete math books is that this book is intended to be used in a class taught using problem oriented or inquiry based methods. When I teach the class, I will assign sections for reading \emph{after} first introducing them in class by using a mix of group work and class discussion on a few interesting problems. The text is meant to consolidate what we \emph{discover} in class and serve as a reference for students as they master the concepts and techniques covered in the unit. None-the-less, every attempt has been made to make the text sufficient for self study as well, in a way that hopefully mimics an inquiry based classroom.
%
\par

The topics covered in this text were chosen to match the need of the students I teach at UNC. The main areas of study are combinatorics, sequences, logic and proofs, and graph theory, in that order. Induction is covered at the end of the chapter on sequences. Most discrete books put logic first as a preliminary, which certainly has its advantages. However, I wanted to discuss logic and proofs together, and found that doing both of these before anything else was overwhelming for my students given that they didn't yet have context of other problems in the subject. Also, after spending a couple weeks on proofs, we would hardly use that at all when covering combinatorics, so much of the progress we made was quickly lost.
%
\par

Depending on the speed of the class, it might be possible to include additional material. In past semesters I have included generating functions (after sequences) and some basic number theory (either after the logic and proofs chapter or at the very end of the course). These additional topics are covered in appendix A.
%
\par

While I (currently) believe this selection and order of topics is optimal, you should feel free to skip around to what interests you. There are occasionally examples and exercises that rely on earlier material, but I have tried to keep these to a minimum and usually can either be skipped or understood without too much additional study. If you are an instructor, feel free to edit the \LaTeX{}~source to fit your needs.
%
\typeout{************************************************}
\typeout{Paragraphs  Previous and future editions}
\typeout{************************************************}
\paragraph[Previous and future editions]{Previous and future editions}\hypertarget{pref_editions}{}

This current Fall 2015 edition of the text is essentially the first edition of the book. I have previously compiled many of the sections in a book format for easy distribution, but those were mostly just lecture notes and exercises (there was no index or Investigate problems; very little in the way of consistent formatting).
%
\par

My intent is to compile a new edition prior to each Fall semester which incorporate additions and corrections suggested by instructors and students who use the text the previous semester. Thus I encourage you to send along any suggestions and comments as you have them. For future editions, I will keep track of any major changes here.
%
\par\hfill\begin{tabular}{l@{}}
Oscar Levin, Ph.D.\\
University of Northern Colorado, 2016
\end{tabular}\\\par
%% end:   preface
%% begin: preface
\chapter*{How to use this book}\label{preface-2}
\addcontentsline{toc}{chapter}{How to use this book}

  In addition to expository text, this book has a few features designed to encourage you to interact with the mathematics.
  %
\typeout{************************************************}
\typeout{Paragraphs  \emph{Investigate!} activities}
\typeout{************************************************}
\paragraph[\emph{Investigate!} activities]{\emph{Investigate!} activities}\hypertarget{paragraphs-2}{}

  Sprinkled throughout the sections (usually at the very beginning of a topic) you will find activities designed to get you acquainted with the topic soon to be discussed. These are similar (sometimes identical) to group activities I give students to introduce material. You really should spend some time thinking about, or even working through, these problems before reading the section. By priming yourself to the types of issues involved in the material you are about to read, you will better understand what is to come. There are no solutions provided for these problems, but don't worry if you can't solve them or are not confident in your answers. My hope is that you will take this frustration with you while you read the proceeding section. By the time you are done with the section, things should be much clearer.
  %
\typeout{************************************************}
\typeout{Paragraphs  Examples}
\typeout{************************************************}
\paragraph[Examples]{Examples}\hypertarget{paragraphs-3}{}

  I have tried to include the ``correct'' number of examples. For those examples which include \emph{problems}, full solutions are included. Before reading the solution, try to at least have an understanding of what the problem is asking. Unlike some textbooks, the examples are not meant to be all inclusive for problems you will see in the exercises. They should not be used as a blueprint for solving other problems. Instead, use the examples to deepen our understanding of the concepts and techniques discussed in each section. Then use this understanding to solve the exercises at the end of each section.
  %
\typeout{************************************************}
\typeout{Paragraphs  Exercises}
\typeout{************************************************}
\paragraph[Exercises]{Exercises}\hypertarget{paragraphs-4}{}

  You get good at math through practice. Each section concludes with a small number of exercises meant to solidify concepts and basic skills presented in that section. At the end of each chapter, a larger collection of similar exercises is included (as a sort of ``chapter review'') which might bridge material of different sections in that chapter. Every exercise has either a hint, answer or full solution (which in the pdf version of the text can be found by clicking on the exercises number \textendash{} clicking on the solution number will bring you back to the exercise). Readers are encouraged to try these exercises before looking at the solution. When I teach with this book, I assign these exercises as practice and then use them, or similar problems, on quizzes and exams.
  %
\typeout{************************************************}
\typeout{Paragraphs  Homework Problems}
\typeout{************************************************}
\paragraph[Homework Problems]{Homework Problems}\hypertarget{paragraphs-5}{}

       Each chapter includes a small number of more involved problems \textendash{} the type I would assign as homework to be written up and collected each week. As many of these are problems I assign, solutions are not included. If you are using this book for self study, consider these additional \emph{Investigate!} problems.
    %
%% end:   preface
%% begin: table of contents
\setcounter{tocdepth}{3}
\renewcommand*\contentsname{Contents}
\tableofcontents
%% end:   table of contents
\mainmatter
\typeout{************************************************}
\typeout{Chapter 1 Introduction and Preliminaries}
\typeout{************************************************}
\chapter[Introduction and Preliminaries]{Introduction and Preliminaries}\label{ch_intro}
\typeout{************************************************}
\typeout{Introduction  }
\typeout{************************************************}

      Welcome to Discrete Mathematics. If this is your first time encountering the subject, you will probably find discrete mathematics quite different from other math subjects. You might not even know what discrete math is! Hopefully this short introduction
      will shed some light on what the subject is about and what you can expect as you move forward in your studies.
    %
\typeout{************************************************}
\typeout{Section 1.1 What is Discrete Mathematics?}
\typeout{************************************************}
\section[What is Discrete Mathematics?]{What is Discrete Mathematics?}\label{sec_intro-intro}
\begin{quote}dis\textperiodcentered{}crete / dis'krët.%
\par
 \emph{Adjective}: Individually separate and distinct.%
\par
\emph{Synonyms}: separate - detached - distinct - abstract.%
\end{quote}

    Defining \emph{discrete mathematics} is hard because defining \emph{mathematics} is hard. What is mathematics? The study of numbers? In part, but you also study functions and lines and triangles and parallelepipeds and vectors and
    \dots{}. Or perhaps you want to say that mathematics is a collection of tools that allow you to solve problems. What sort of problems? Okay, those that involve numbers, functions, lines, triangles,
    \dots{}. Whatever your conception of what mathematics is, try applying the concept of ``discrete'' to it, as defined above. Some math fundamentally deals with\dots{} \emph{stuff}\dots{} that is individually separate and distinct.
  %
\par

    In an algebra or calculus class, you might have found a particular set of numbers (maybe the set of number in the range of a function). You would represent this set as an interval: \([0,\infty)\) is the range of \(f(x) = x^2\) since the set
    of outputs of the function are all real numbers 0 and greater. This set of numbers is NOT discrete. The numbers in the set are not separated by much at all. In fact, take any two numbers in the set and there are infinitely many more between
    them which are also in the set. Discrete math could still ask about the range of a function, but the set would not be an interval. Consider the function which gives the number of children each person reading this has. What is the range? I'm guessing
    it is something like \(\{0, 1, 2, 3\}\). Maybe 4 is in there too. But certainly there is nobody reading this that has 1.32419 children. This set \emph{is} discrete because the elements are separate. Also notice that the inputs to the function
    are a discrete set as each input is an individual person. You would not consider fractional inputs (there is nothing we care about \(2/3\) between a pair of readers).
  %
\par

    One way to get a feel for the subject is to consider the types of problems you solve in discrete math. Here are a few simple examples:
  %
\begin{investigation}[]\label{investigation-1}

      Here are a few Discrete Math problems for you to try.
    %
\par

      \emph{Note: Throughout the text you will see \emph{Investigate!} activities like this one. Answer the questions in these as best you can to give yourself a feel for what is coming next.}
    %
\leavevmode%
\begin{enumerate}
\item\hypertarget{li-1}{}
        The most popular mathematician in the world is throwing a party for all of his friends. As a way to kick things off, they decide that everyone should shake hands. Assuming all 10 people at the party each shake hands with every other person (but not themselves,
        obviously) exactly once, how many handshakes take place?
      %
\item\hypertarget{li-2}{}
        At the warm-up event for Oscar's All Star Hot Dog Eating Contest, Al ate one hot dog. Bob then showed him up by eating three hot dogs. Not to be outdone, Carl ate five. This continued with each contestant eating two more hot dogs than the previous contestant.
        How many hot dogs did Zeno (the 26th and final contestant) eat? How many hot dogs were eaten all together?
      %
\item\hypertarget{li-3}{}
      After excavating for weeks, you finally arrive at the burial chamber. The room is empty except for two large chests. On each is carved a message (strangely in English).
      %
\leavevmode%
\begin{figure}
\centering
{
\begin{tikzpicture}
          \node[shape=rectangle, draw=brown, thick, fill=brown!20!white, inner sep=5mm, minimum height=3cm, minimum width=3.5cm, text width=3.5cm, align=center] (a) { If this chest is empty, then the other chest's message is true.};
          \node[shape=rectangle, draw=brown, thick, fill=brown!20!white, inner sep=5mm, minimum height=3cm, minimum width=3.5cm, text width=3.5cm, align=center, right=of a] {
               This chest is filled with treasure or the other chest contains deadly scorpions.
              };
      \end{tikzpicture}
}
\end{figure}
\par

      You know exactly one of these messages is true. What should you do?
      %
\item\hypertarget{li-4}{}
        Back in the days of yore, five small towns decided they wanted to build roads directly connecting each pair of towns. While the towns had plenty of money to build roads as long and as winding as they wished, it was very important that the roads not intersect
        with each other (as stop signs had not yet been invented). Also, tunnels and bridges were not allowed. Is it possible for each of these towns to build a road to each of the four other towns without creating any intersections?
      %
\end{enumerate}
\end{investigation}
\par

    One reason it is difficult to define discrete math is that it is a very broad description which encapsulates a large number of subjects. In this course we will study four main topics: \terminology{combinatorics} (the theory of ways things \emph{combine};
    in particular, how to count these ways), \terminology{sequences}, \terminology{symbolic logic}, and \terminology{graph theory}. However, there are other topics that belong under the discrete umbrella, including computer science, abstract algebra, number theory, game theory,
    probability, and geometry (some of these, particularly the last two, have both discrete and non-discrete variants).
  %
\par

    Ultimately the best way to learn what discrete math is about is to \emph{do} it. Let's get started! Before we can begin answering more complicated (and fun) problems, we must lay down some foundation. We start by reviewing sets and functions in
    the framework of discrete mathematics.
  %
\typeout{************************************************}
\typeout{Section 1.2 
    Mathematical Statements
  }
\typeout{************************************************}
\section[
    Mathematical Statements
  ]{
    Mathematical Statements
  }\label{sec_intro-statements}
\typeout{************************************************}
\typeout{Introduction  }
\typeout{************************************************}
\begin{investigation}[]\label{investigation-2}

        While walking through a fictional forest, you encounter three trolls. Each is either a \emph{knight}, who always tells the truth, or a \emph{knave}, who always lies. The trolls will not let you pass until you correctly identify each as either a knight or a knave. Each troll makes a single statement:


        \begin{quote}Troll 1: If I am a knave, then there are exactly two knights here.%
\par
Troll 2: Troll 1 is lying.%
\par
Troll 3: Either we are all knaves or at least one of us is a knight.%
\end{quote}



        Which troll is which?%
\end{investigation}

      In order to \emph{do} mathematics, we must be able to \emph{talk} and
      \emph{write} about mathematics. Perhaps your experience with mathematics so far has mostly involved finding answers to problems. As we embark towards more advanced and abstract mathematics, writing will play a more prominent role in the mathematical process.
    %
\par

      Communication in mathematics requires more precision than many other subjects, and thus we should take a few pages here to consider the basic building blocks: \emph{mathematical statements}.
    %
\typeout{************************************************}
\typeout{Subsection 1.2.1 Atomic and Molecular Statements}
\typeout{************************************************}
\subsection[Atomic and Molecular Statements]{Atomic and Molecular Statements}\label{atomic-molecular-statements}

      A \terminology{statement}\index{statement} is any declarative sentence which is either true or false. A statement is
      \terminology{atomic} if it cannot be divided into smaller statements, otherwise it is called
      \terminology{molecular}.
    %
\begin{example}[]\label{example-1}

          These are statements (in fact,
          \emph{atomic} statements):
        %
\leavevmode%
\begin{itemize}[label=\textbullet]
\item{}
              Telephone numbers in the USA have 10 digits.
            %
\item{}
              The moon is made of cheese.
            %
\item{}
              42 is a perfect square.
            %
\item{}
              Every even number greater than 2 can be expressed as the sum of two primes.
            %
\item{}
              \(3+7 = 12\)
            %
\end{itemize}
\par

          And these are not statements:
        %
\leavevmode%
\begin{itemize}[label=\textbullet]
\item{}
              Would you like some cake?
            %
\item{}
              The sum of two squares.
            %
\item{}\(1+3+5+7+\cdots+2n+1\).%
\item{}
              Go to your room!
            %
\item{}
              \(3+x = 12\)
            %
\end{itemize}
\end{example}
\par

      The reason the last sentence is not a statement is because it contains a variable. Depending on what
      \(x\) is, the sentence is either true or false, but right now it is neither. One way to make the sentence into a statement is to specify the value of the variable in some way. This could be done in a number of ways. For example,
      ``\(3+x = 12\) where \(x = 9\)'' is a true statement, as is
      ``\(3+x = 12\) for some value of \(x\) ''. This is an example of
      \emph{quantifying} over a variable, which we will discuss more in a bit.
    %
\par

      You can build more complicated (molecular) statements out of simpler (atomic or molecular) ones using
      \terminology{logical connectives}
      \index{connectives}. For example, this is a statement:
    %
\begin{quote}
      Telephone numbers in the USA have 10 digits and 42 is a perfect square.
    \end{quote}
\par

      Note that we can break this down into two smaller statements. The two shorter statements are
      \emph{connected} by an
      ``and.'' We will consider 5 connectives:
      ``and'' (Sam is a man and Chris is a woman),
      ``or'' (Sam is a man or Chris is a woman),
      ``if\dots{}, then\dots{}'' (if Sam is a man, then Chris is a woman),
      ``if and only if'' (Sam is a man if and only if Chris is a woman), and
      ``not'' (Sam is not a man). The first four are called
      \emph{binary connectives} (because they connect two statements) while
      ``not'' is an example of a
      \emph{unary connective} (since it applies to a single statement).
    %
\par

      Which connective we use to modify statment(s) will determine the
      \terminology{truth value}
      \index{truth value} of the molecular statement (that is, whether the statement is true or false), based on the truth values of the statements being modified. It is important to realize that we do not need to know what the parts actually say, only whether those parts are true or false. So to analyze logical connectives, it is enough to consider \terminology{propositional variables} (sometimes called \emph{sentential} variables), usually capital letters in the middle of the alphabet: \(P, Q, R, S, \ldots\)
      \label{notation-1}
. These are variables that can take on one of two values: T or F. We also have symbols for the logical connectives:
      \(\wedge\),
      \(\vee\),
      \(\imp\),
      \(\iff\),
      \(\neg\).
    %
\begin{assemblage}{Logical Connectives}\label{assemblage-1}\par\medskip

        \leavevmode%
\begin{itemize}[label=\textbullet]
\item{}\(P \wedge Q\) means \(P\) and \(Q\), called a
            \terminology{conjunction}\index{conjunction}\index{connectives!and}.%
\item{}\(P \vee Q\) means
            \(P\) or
            \(Q\), called a
            \terminology{disjunction}\index{disjunction}\index{connectives!or}.%
\item{}\(P \imp Q\) means if
            \(P\) then
            \(Q\), called an
            \terminology{implication} or
            \terminology{conditional}\index{implication}\index{conditional}\index{connectives!implies}\index{if\dots{}
                                then}.%
\item{}\(P \iff Q\) means
            \(P\) if and only if
            \(Q\), called a
            \terminology{biconditional}\index{biconditional}\index{connectives!if and only if}\index{if and only if}.%
\item{}\(\neg P\) means not
            \(P\), called a
            \terminology{negation}\index{negation}\index{connectives!not}.%
\end{itemize}

      %
\end{assemblage}
\par

      The
      \terminology{truth value} of a statement is determined by the truth value(s) of its part(s), depending on the connectives:
    %
\begin{assemblage}{Truth Conditions for Connectives}\label{assemblage-2}\par\medskip

        \leavevmode%
\begin{itemize}[label=\textbullet]
\item{}\(P \wedge Q\) is true when both
            \(P\) and
            \(Q\) are true%
\item{}\(P \vee Q\) is true when
            \(P\) or
            \(Q\) or both are true.%
\item{}\(P \imp Q\) is true when
            \(P\) is false or
            \(Q\) is true or both.%
\item{}\(P \iff Q\) is true when
            \(P\) and
            \(Q\) are both true, or both false.%
\item{}\(\neg P\) is true when
            \(P\) is false.%
\end{itemize}

      %
\end{assemblage}
\par

      Note that for us,
      \emph{or} is the
      \terminology{inclusive or}
      \index{inclusive or} (and not the sometimes used
      \emph{exclusive or}) meaning that
      \(P \vee Q\) is in fact true when both
      \(P\) and
      \(Q\) are true. As for the other connectives,
      ``and'' behaves as you would expect, as does negation. The biconditional (if and only if) might seem a little strange, but you should think of this as saying the two parts of the statements are
      \emph{equivalent}. This leaves only the conditional
      \(P \imp Q\) which has a slightly different meaning in mathematics than it does in ordinary usage. However, implications are so common and useful in mathematics, that we must develop fluency with their use, and as such, they deserve their own subsection.
    %
\typeout{************************************************}
\typeout{Subsection 1.2.2 Implications}
\typeout{************************************************}
\subsection[Implications]{Implications}\label{subsec_implications}

      Easily the most common type of statement in mathematics is the conditional, or implication. Even statements that do not at first look like they have this form conceal an implication at their heart. Consider the
      \emph{Pythagorean Theorem}. Many a college freshman would quote this theorem as
      ``
                    \(a^2 + b^2 = c^2\).'' This is absolutely not correct. For one thing, that is not a statement since it has three variables in it. But perhaps they imply that this should be true for any values of the variables. So
      \(1^2 + 5^2 = 2^2\)??? How can we fix this? Well, the equation is true as long as
      \(a\) and
      \(b\) are the legs or a right triangle and
      \(c\) is the hypotenuse. In other words:

      \begin{quote}\emph{If
                    }\(a\) and
        \(b\) are the legs of a right triangle with hypotenuse
        \(c\),
        \emph{then
                    }\(a^2 + b^2 = c^2\).
      \end{quote}

    %
\par

      This is a reasonable way to think about implications: our claim is that the conclusion (``then'' part) is true, but on the assumption that the hypothesis (``if'' part) is true. We make no claim about the conclusion in situations when the hypothesis is false.
    %
\par

      Still, it is important to remember that an implication is a statement, and as such either true or false. The truth value of the implication is determined by the truth values of its two parts. To agree with the usage above, we say that an implication is true either when the hypothesis is false, or when the conclusion is true. This leaves only one way for an implication to be false: when the hypothesis is true and the conclusion is false.
    %
\begin{example}[]\label{example-2}
Consider the statement:
          \begin{quote}if Bob gets a 90 on the final, then Bob will pass the class.\end{quote}

          This is definitely an implication:
          \(P\) is the statement,
          ``Bob gets a 90 on the final,'' and
          \(Q\) is the statement,
          ``Bob will pass the class.''
        %
\par

          Suppose I made that statement to Bob. In what circumstances would it be fair to call me a liar? What if Bob really did get a 90 on the final, and he did pass the class? Then I have not lied; my statement is true. But if Bob did get a 90 on the final and did not pass the class, then I lied, making the statement false. The tricky case is this: what if Bob did not get a 90 on the final? Maybe he passes the class, maybe he doesn't. Did I lie in either case? I think not. In these last two cases,
          \(P\) was false, and the statement
          \(P \imp Q\) was true. In the first case,
          \(Q\) was true, and so was
          \(P \imp Q\). So
          \(P \imp Q\) is true when either
          \(P\) is false or
          \(Q\) is true.
        %
\end{example}
\par

      Just to be clear, although we sometimes read
      \(P \imp Q\) as
      ``
                    \(P\)
                    \emph{implies}
                    \(Q\)
                '', we are not insisting that there is some causal relationship between the statements
      \(P\) and
      \(Q\). In particular, if you claim that
      \(P \imp Q\) is
      \emph{false}, you are not saying that
      \(P\) does not imply
      \(Q\), but rather that
      \(P\) is true and
      \(Q\) is false.
    %
\begin{example}[]\label{example-3}
Decide which of the following statements are true and which are false. Briefly explain.
          \leavevmode%
\begin{enumerate}
\item\hypertarget{li-25}{}\(0=1 ~~ \imp ~~ 1=1\)%
\item\hypertarget{li-26}{}\(1=1 ~~ \imp ~~\) most horses have 4 legs%
\item\hypertarget{li-27}{}If 8 is a prime number, then the 7624th digit of
              \(\pi\) is an 8.%
\item\hypertarget{li-28}{}If the 7624th digit of
              \(\pi\) is an 8, then
              \(2+2 = 4\)%
\end{enumerate}

        %
\par\medskip\noindent%
\textbf{Solution.}\quad 
          All four of the statements are true. Remember, the only way for an implication to be false is for the
          \emph{if} part to be true and the
          \emph{then} part to be false.
          \leavevmode%
\begin{enumerate}
\item\hypertarget{li-29}{}Here the hypothesis is false and the conclusion is true, so the implication is true.
            %
\item\hypertarget{li-30}{}Here both the hypothesis and the conclusion is true, so the implication is true. It does not matter that there is no logical connection between the true mathematical fact and the fact about horses.%
\item\hypertarget{li-31}{}I have no idea what the 7624th digit of
              \(\pi\) is, but this does not matter. Since the hypothesis is false, the implication is automatically true.%
\item\hypertarget{li-32}{}Similarly here, regardless of the truth value of the hypothesis, the conclusion is true, making the implication true.%
\end{enumerate}

        %
\end{example}
\par

      It is important to understand the conditions under which an implication is true not only to decide whether a mathematical statement is true, but in order to
      \emph{prove} that it is. Proofs by seem scary (especially if you have had a bad high school geometry experience) but all we are really doing is explaining (very carefully) why a statement is true. And if you understand the truth conditions for an implication, you already have the outline for a proof.
    %
\begin{assemblage}{Direct Proofs of Implications}\label{assemblage-3}\par\medskip

        To prove an implication
        \(P \imp Q\), it is enough to assume
        \(P\) and from it deduce
        \(Q\).
      %
\end{assemblage}
\par

      There are other techniques to prove statements (implications and others) that we will encounter throughout our studies, and new proof techniques are discovered all the time. Direct proof is the easiest and most elegant style of proof and have the advantage that such a proof often does a great job of explaining
      \emph{why} the statement is true.
    %
\begin{example}[]\label{example-4}

          Prove: If two numbers
          \(a\) and
          \(b\) are even, then their sum
          \(a+b\) is even.
        %
\par\medskip\noindent%
\textbf{Solution.}\quad 
          Suppose the numbers
          \(a\) and
          \(b\) are even. This means that
          \(a = 2k\) and
          \(b=2j\) for some integers
          \(k\) and
          \(j\). The sum is then
          \(a+b = 2k+2j = 2(k+j)\). Since
          \(k+j\) is an integer, this means that
          \(a+b\) is even.
        %
\par

          Notice that since we get to assume the hypothesis of the implication we immediately have a place to start. The proof proceeds essentially by repeatedly asking and answering,
          ``what does that mean?''
        %
\end{example}
\par

      This sort of argument shows up outside of math as well. If you ever found yourself starting an argument with,
      ``hypothetically, let's assume
                    '' then you have attempted a direct proof of you desired conclusion.
    %
\par

      Since implications are so prevalent in mathematics, we have some special language to help discuss them:
    %
\begin{assemblage}{Converse and Contrapositive}\label{assemblage-4}\par\medskip

        \leavevmode%
\begin{itemize}[label=\textbullet]
\item{}
              The
              \terminology{converse}
              \index{converse} of an implication
              \(P \imp Q\) is the implication
              \(Q \imp P\). The converse is
              NOT logically equivalent to the original implication.
            %
\item{}
              The
              \terminology{contrapositive}
              \index{contrapositive} of an implication
              \(P \imp Q\) is the statement
              \(\neg Q \imp \neg P\). An implication and its contrapositive are logically equivalent (they are either both true or both false).
            %
\end{itemize}

      %
\end{assemblage}
\par

      Mathematics is overflowing with examples of true implications with a false converse. If a number greater than 2 is prime, then that number is odd. However, just because a number is odd does not mean it is prime. If a shape is a square, then it is a rectangle. But it is false that if a shape is a rectangle, then it is a square. While this happens often, it does not always happen. For example, they Pythagorean theorem has a true converse: if
      \(a^2 + b^2 = c^2\), then the triangle with sides
      \(a\),
      \(b\), and
      \(c\) is a
      \emph{right} triangle. Whenever you encounter a implication in mathematics, it is always reasonable to ask whether the converse is true.
    %
\par

      The contrapositive, on the other hand, always has the same truth value as its original implication. This can be very helpful in deciding whether an implication is true: often it is easier to analyze the contrapositive.
    %
\begin{example}[]\label{example-5}
True or false: If you draw any nine playing cards from a regular deck, then you will have at least three cards all of the same suit. Is the converse true?%
\par\medskip\noindent%
\textbf{Solution.}\quad 
          True. The original implication is a little hard to analyze because there are so many different combinations of nine cards. But consider the contrapositive: If you
          \emph{don't} have at least three cards all of the same suit, then you don't have nine cards. It is easy to see why this is true: you can at most have two cards of each of the four suits, for a total of eight cards (or fewer).
        %
\par

          The converse: If you have at least three card all of the same suit, then you have nine cards. This is false. You could have three spades and nothing else. Note that to demonstrate that the converse (an implication) is false, we provided an example where the hypothesis is true (you do have three cards of the same suit) but where the conclusion is false (you do not have nine cards).
        %
\end{example}
\par

      Understanding converses and contrapositives can help understand implications and their truth values:
    %
\begin{example}[]\label{example-6}

          Suppose I tell Sue that if she gets a 93\% on her final, then she will get an A in the class. Assuming that what I said is true, what can you conclude in the following cases:
        %
\leavevmode%
\begin{enumerate}
\item\hypertarget{li-35}{}Sue gets a 93\% on her final.%
\item\hypertarget{li-36}{}Sue gets an A in the class.%
\item\hypertarget{li-37}{}Sue does not get a 93\% on her final.%
\item\hypertarget{li-38}{}Sue does not get an A in the class.%
\end{enumerate}
\par\medskip\noindent%
\textbf{Solution.}\quad 
          Note first that whenever
          \(P \imp Q\) and
          \(P\) are both true statements,
          \(Q\) must be true as well. For this problem, take
          \(P\) to mean
          ``Sue gets a 93\%
                            on her final'' and
          \(Q\) to mean
          ``Sue will get an A in the class.''
        %
\leavevmode%
\begin{enumerate}
\item\hypertarget{li-39}{}We have
            \(P \imp Q\) and
            \(P\) so
            \(Q\) follows. Sue gets an A.%
\item\hypertarget{li-40}{}You cannot conclude anything. Sue could have gotten the A because she did extra credit for example. Notice that we do not know that if Sue gets an
            \(A\), then she gets a 93\% on her final. That is the converse of the original implication, so it might or might not be true.%
\item\hypertarget{li-41}{}The inverse of
            \(P \imp Q\) is
            \(\neg P \imp \neg Q\), which states that if Sue does not get a 93\% on the final then she will not get an A in the class. But this does not follow from the original implication. Again, we can conclude nothing. Sue could have done extra credit.%
\item\hypertarget{li-42}{}What would happen if Sue does not get an A but
            \emph{did} get a 93\% on the final. Then
            \(P\) would be true and
            \(Q\) would be false. But this makes the implication
            \(P \imp Q\) false! So it must be that Sue did not get a 93\% on the final. Notice now we have the implication
            \(\neg Q \imp \neg P\) which is the contrapositive of
            \(P \imp Q\). Since
            \(P \imp Q\) is assumed to be true, we know
            \(\neg Q \imp \neg P\) is true as well.%
\end{enumerate}
\end{example}
\par

      As we said above, an implication is not logically equivalent to its converse, but it is possible that both are true. In this case, when both
      \(P \imp Q\) and
      \(Q \imp P\) are true, we say that
      \(P\) and
      \(Q\) are equivalent. This is the biconditional we mentioned earlier:
    %
\begin{assemblage}{If and only if}\label{assemblage-5}\par\medskip

        \begin{quote}  \(P \iff Q\) is logically equivalent to \((P \imp Q) \wedge (Q \imp P)\).%
\end{quote}

      %
\par

        Example: Given an integer
        \(n\), it is true that
        \(n\) is even if and only if
        \(n^2\) is even. That is, if
        \(n\) is even, then
        \(n^2\) is even, as well as the converse: if
        \(n^2\) is even, then
        \(n\) is even.
      %
\end{assemblage}
\par

      You can think of
      ``if and only if'' statements as having two parts: an implication and its converse. We might say one is the
      ``if'' part, and the other is the
      ``only if'' part. We also sometimes say that
      ``if and only if'' statements have two directions: a forward direction
      \((P \imp Q)\) and a backwards direction (\(P \leftarrow Q\), which is really just sloppy notation for
      \(Q \imp P\)).
    %
\par

      Let's think a little about which part is which. Is
      \(P \imp Q\) the
      ``if'' part or the
      ``only if'' part? Perhaps we should look at an example.
    %
\begin{example}[]\label{example-7}

          Suppose it is true that I sing if and only if I'm in the shower. We know this means that both if I sing, then I'm in the shower, and also the converse, that if I'm in the shower, then I sing. Let
          \(P\) be the statement,
          ``I sing,'' and
          \(Q\) be,
          ``I'm in the shower.'' So
          \(P \imp Q\) is the statement
          ``if I sing, then I'm in the shower.'' Which part of the if and only if statement is this?
        %
\par

          What we are really asking is what is the meaning of
          ``I sing if I'm in the shower'' and
          ``I sing only if I'm in the shower.'' When is the first one (the
          ``if'' part)
          \emph{false}? When I am in the shower but not singing. That is the same condition on being false as the statement
          ``if I'm in the shower, then I sing.'' So the
          ``if'' part is
          \(Q \imp P\). On the other hand, to say,
          ``I sing only if I'm in the shower'' is equivalent to saying
          ``if I sing, then I'm in the shower,'' so the only if part is
          \(P \imp Q\).
        %
\end{example}
\par

      It is not terribly important to know which part is the if or only if part, but this does get at something very, very important: THERE ARE MANY WAYS TO STATE AN IMPLICATION! The problem is, since these are all different ways of saying the same implication, we cannot use truth tables to analyze the situation. Instead, we just need good English skills.
    %
\begin{example}[]\label{example-8}

          Rephrase the implication,
          ``if I dream, then I am asleep'' in as many different ways as possible. Then do the same for the converse.
        %
\par\medskip\noindent%
\textbf{Solution.}\quad 
          The following are all equivalent to the original implication:
        %
\leavevmode%
\begin{enumerate}
\item\hypertarget{li-43}{} I am asleep if I dream. %
\item\hypertarget{li-44}{} I dream only if I am asleep. %
\item\hypertarget{li-45}{} In order to dream, I must be asleep. %
\item\hypertarget{li-46}{} To dream, it is necessary that I am asleep. %
\item\hypertarget{li-47}{} To be asleep, it is sufficient to dream. %
\item\hypertarget{li-48}{} I am not dreaming unless I am asleep. %
\end{enumerate}
\par

          The following are equivalent to the converse (if I am asleep, then I dream):
        %
\leavevmode%
\begin{enumerate}
\item\hypertarget{li-49}{}
              I dream if I am asleep.
            %
\item\hypertarget{li-50}{}
              I am asleep only if I dream.
            %
\item\hypertarget{li-51}{}
              It is necessary that I dream in order to be asleep.
            %
\item\hypertarget{li-52}{}
              It is sufficient that I be asleep in order to dream.
            %
\item\hypertarget{li-53}{}
              If I don't dream, then I'm not asleep.
            %
\end{enumerate}
\end{example}
\par

      Hopefully you agree with the above example. We include the
      ``necessary and sufficient'' versions because those are common when discussing mathematics. In fact, let's agree once and for all what they mean:
    %
\begin{assemblage}{Necessary and Sufficient}\label{assemblage-6}\par\medskip

        \index{necessary condition} \index{sufficient condition}
      %
\par

        \leavevmode%
\begin{itemize}[label=\textbullet]
\item{}``
                                \(P\)
                                is necessary for
                                \(Q\)
                            '' means
            \(Q \imp P\).%
\item{}``
                                \(P\)
                                is sufficient for
                                \(Q\)
                            '' means
            \(P \imp Q\).%
\item{}
              If
              \(P\) is necessary and sufficient for
              \(Q\), then
              \(P \iff Q\).
            %
\end{itemize}

      %
\end{assemblage}
\par

      To be honest, I have trouble with these if I'm not very careful. I find it helps to have an example in mind:
    %
\begin{example}[]\label{example-9}

          Recall from calculus, if a function is differentiable at a point
          \(c\), then it is continuous at
          \(c\), but that the converse of this statement is not true (for example,
          \(f(x) = |x|\) at the point 0). Restate this fact using necessary and sufficient language.
        %
\par\medskip\noindent%
\textbf{Solution.}\quad 
          It is true that in order for a function to be differentiable at a point
          \(c\), it is necessary for the function to be continuous at
          \(c\). However, it is not necessary that a function be differentiable at
          \(c\) for it to be continuous at
          \(c\).
        %
\par

          It is true that to be continuous at a point
          \(c\), it is sufficient that the function be differentiable at
          \(c\). However, it is not the case that being continuous at
          \(c\) is sufficient for a function to be differentiable at
          \(c\).
        %
\end{example}
\par

      Thinking about the necessity and sufficiency of conditions can also help when writing proofs and justifying conclusions. If you want to establish some mathematical fact, it is helpful to think what other facts would
      \emph{be enough} (be sufficient) to prove your fact. If you have an assumption, think about what must also be necessary if that hypothesis is true.
    %
\typeout{************************************************}
\typeout{Subsection 1.2.3 Quantifiers}
\typeout{************************************************}
\subsection[Quantifiers]{Quantifiers}\label{subsec_quantifiers}
\typeout{************************************************}
\typeout{Introduction  }
\typeout{************************************************}
\begin{investigation}[]\label{investigation-3}

          Consider the statement below. Decide whether any are equivalent to each other, or whether any imply any others.
        %
\leavevmode%
\begin{enumerate}
\item\hypertarget{li-57}{}
              You can fool some people all of the time.
            %
\item\hypertarget{li-58}{}
              You can fool everyone some of the time.
            %
\item\hypertarget{li-59}{}
              You can always fool some people.
            %
\item\hypertarget{li-60}{}
              Sometimes you can fool everyone.
            %
\end{enumerate}
\end{investigation}

      It would be nice to use variables in our mathematical sentences. For example, suppose we wanted to claim that if
      \(n\) is prime, then
      \(n+7\) is not prime. This looks like an implication. I would like to write something like
      \begin{equation*}
        P(n) \imp \neg P(n+7)
      \end{equation*}
      where
      \(P(n)\) means
      ``
                    \(n\)
                    is prime.'' But this is not quite right. For one thing, because this sentence has a free variable (that is, a variable that we have not specified anything about), it is not a statement. Now if we plug in a specific value for
      \(n\), we do get a statement. In fact, it turns out that no matter what value we plug in for
      \(n\), we get a true implication. So what we really want to say is that
      \emph{for all} values of
      \(n\), if
      \(n\) is prime, then
      \(n+7\) is not. We need to
      \emph{quantify} the variable.
    %
\par

      Although there are many types of
      \emph{quantifiers} in English (e.g., many, few, most, etc.) in mathematics we for the most part stick to two: existential and universal.
    %
\begin{assemblage}{Universal and Existential Quantifiers}\label{assemblage-7}\par\medskip

        \index{quantifiers}
      %
\par

        The existential quantifier is
        \(\exists\) and is read
        ``there exists'' or
        ``there is.'' For example,\index{existential quantifier}
        \index{quantifiers!exists}
        \begin{equation*}
          \exists x (x \lt 0)
        \end{equation*}
        asserts that there is a number less than 0.
      %
\par

        The universal quantifier is
        \(\forall\) and is read
        ``for all'' or
        ``every.'' For example,\index{universal quantifier}
        \index{quantifiers!for all}
        \begin{equation*}
          \forall x (x \ge 0)
        \end{equation*}
        asserts that every number is greater than or equal to 0.
      %
\end{assemblage}
\par

      As with all mathematical statements, we would like to decide whether quantified statements are true or false. Consider the statement
      \begin{equation*}
        \forall x \exists y (y \lt x).
      \end{equation*}
      You would read this,
      ``for every
                    \(x\)
                    there is is some
                    \(y\)
                    such that
                    \(y\)
                    is less than
                    \(x\).'' Is this true? The answer depends on what our
      \emph{domain of discourse} is: when we say
      ``for all''
      \(x\), do we mean all positive integers or all real numbers or all elements of some other set? Usually this information is implied. In discrete mathematics, we almost always quantify over the
      \emph{natural numbers}, 0, 1, 2,
      , so let's take that for our domain of discourse here.
    %
\par

      For the statement to be true, we need it to be the case that no matter what natural number we select, there is always some natural number that is strictly smaller. Perhaps we could let
      \(y\) be
      \(x-1\)? But here is the problem: what if
      \(x = 0\)? Then
      \(y = -1\) and that is
      \emph{not a number!} (in our domain of discourse). Thus we see that the statement is not true because there is a number such that every number is at least as large as. or in symbols,
      \begin{equation*}
        \exists x \forall y (y \ge x).
      \end{equation*}
    %
\par

      To show that the original statement is false, we proved that the
      \emph{negation} was true. And notice how the negation and original statement compare. This is typical.
    %
\begin{assemblage}{Quantifiers and Negation}\label{assemblage-8}\par\medskip

        \begin{quote}
            \(\neg \forall x P(x)\) is equivalent to
            \(\exists x \neg P(x)\).%
\par

            \(\neg \exists x P(x)\) is equivalent to
            \(\forall x \neg P(x)
                            \).%
\end{quote}

      %
\end{assemblage}
\par

      Essentially, we can pass the negation symbol over a quantifier, but that causes the quantifier to switch type. This should not be surprising: if not everything is a property, then something doesn't have that property. And if there is not something with a property, then everything doesn't have that property.
    %
\typeout{************************************************}
\typeout{Exercises 1.2.4 Exercises}
\typeout{************************************************}
\subsection[Exercises]{Exercises}\label{exercises-1}
\begin{exerciselist}
\item[1.]\hypertarget{exercise-1}{}Classify each of the sentences below as an atomic statement, and molecular statement, or not a statement at all.  If the statement is molecular, say what kind it is (conjuction, disjunction, conditional, biconditional, negation).
        \leavevmode%
\begin{enumerate}[label=(\alph*)]
\item\hypertarget{li-61}{}The sum of the first 100 odd positive integers.%
\item\hypertarget{li-62}{}Everybody needs somebody sometime.%
\item\hypertarget{li-63}{}The Broncos will win the Super Bowl or I'll eat my hat.%
\item\hypertarget{li-64}{}We can have donuts for dinner, but only if it rains.%
\item\hypertarget{li-65}{}Every natural number greater than 1 is either prime or composite.%
\item\hypertarget{li-66}{}This sentences is false%
\end{enumerate}

        %
\par\smallskip
\item[2.]\hypertarget{exercise-2}{}
    Suppose \(P\) and \(Q\) are the statements:
    \(P\): Jack passed math.
    \(Q\): Jill passed math.
    %
\leavevmode%
\begin{enumerate}[label=(\alph*)]
\item\hypertarget{li-73}{} Translate ``Jack and Jill both passed math'' into symbols. %
\item\hypertarget{li-74}{} Translate ``If Jack passed math, then Jill did not'' into symbols. %
\item\hypertarget{li-75}{} Translate ``\(P \vee Q\)'' into English. %
\item\hypertarget{li-76}{} Translate ``\(\neg(P \wedge Q) \imp Q\)'' into English. %
\item\hypertarget{li-77}{} Suppose you know that if Jack passed math, then so did Jill.  What can you conclude if you know that:
      %
\begin{enumerate}[label=\roman*.]
\item\hypertarget{li-78}{} Jill passed math?  %
\item\hypertarget{li-79}{}  Jill did not pass math?%
\end{enumerate}

     %
\end{enumerate}
\par\smallskip
\item[3.]\hypertarget{exercise-3}{}
    Geoff Poshingten is out at a fancy pizza joint, and decides to order a calzone. When the waiter asks what he would like in it, he replies, ``I want either pepperoni or sausage. Also, if I have sausage, then I must also include quail. Oh, and if I have pepperoni or quail then I must also have ricotta cheese.''
    %
\leavevmode%
\begin{enumerate}[label=(\alph*)]
\item\hypertarget{li-87}{} Translate Geoff's order into logical symbols. %
\item\hypertarget{li-88}{} The waiter knows that Geoff is either a liar or a truth-teller (so either everything he says is false, or everything is true).  Which is it? %
\item\hypertarget{li-89}{} What, if anything, can the waiter conclude about the ingredients in Geoff's desired calzone? %
\end{enumerate}
\par\smallskip
\item[4.]\hypertarget{exercise-4}{} Consider the statement ``If Oscar eats Chinese food, then he drinks milk.'' %
\leavevmode%
\begin{enumerate}[label=(\alph*)]
\item\hypertarget{li-93}{} Write the converse of the statement. %
\item\hypertarget{li-94}{} Write the contrapositive of the statement. %
\item\hypertarget{li-95}{} Is it possible for the contrapositive to be false? If it was, what would that tell you? %
\item\hypertarget{li-96}{} Suppose the original statement is true, and that Oscar drinks milk. Can you conclude anything (about his eating Chinese food)? Explain. %
\item\hypertarget{li-97}{} Suppose the original statement is true, and that Oscar does not drink milk. Can you conclude anything (about his eating Chinese food)? Explain. %
\end{enumerate}
\par\smallskip
\item[5.]\hypertarget{exercise-5}{}
          Which of the following statements are equivalent to the implication, ``if you win the lottery, then you will be rich,'' and which are equivalent to the converse of the implication?
        %
\leavevmode%
\begin{enumerate}[label=(\alph*)]
\item\hypertarget{li-103}{} Either you win the lottery or else you are not rich. %
\item\hypertarget{li-104}{} Either you don't win the lottery or else you are rich. %
\item\hypertarget{li-105}{} You will win the lottery and be rich. %
\item\hypertarget{li-106}{} You will be rich if you win the lottery. %
\item\hypertarget{li-107}{} You will win the lottery if you are rich. %
\item\hypertarget{li-108}{} It is necessary for you to win the lottery to be rich. %
\item\hypertarget{li-109}{} It is sufficient to with the lottery to be rich. %
\item\hypertarget{li-110}{} You will be rich only if you win the lottery. %
\item\hypertarget{li-111}{} Unless you win the lottery, you won't be rich. %
\item\hypertarget{li-112}{} If you are rich, you must have one the lottery. %
\item\hypertarget{li-113}{} If you are not rich, then you did not win the lottery. %
\item\hypertarget{li-114}{} You will win the lottery if and only if you are rich. %
\end{enumerate}
\par\smallskip
\item[6.]\hypertarget{exercise-6}{}
          Consider the implication, ``if you clean your room, then you can watch TV.'' Rephrase the implication in as many ways as possible. Then do the same for the converse.
        %
\par\smallskip
\item[7.]\hypertarget{exercise-7}{}
          Translate into symbols. Use \(E(x)\) for ``\(x\) is even'' and \(O(x)\) for ``\(x\) is odd.''
        %
\leavevmode%
\begin{enumerate}[label=(\alph*)]
\item\hypertarget{li-127}{} No number is both even and odd. %
\item\hypertarget{li-128}{} One more than any even number is an odd number. %
\item\hypertarget{li-129}{} There is prime number that is even. %
\item\hypertarget{li-130}{} Between any two numbers there is a third number. %
\item\hypertarget{li-131}{} There is no number between a number and one more than that number. %
\end{enumerate}
\par\smallskip
\item[8.]\hypertarget{exercise-8}{}
          Translate into English:
        %
\leavevmode%
\begin{enumerate}[label=(\alph*)]
\item\hypertarget{li-137}{}\(\forall x (E(x) \imp E(x +2))\).%
\item\hypertarget{li-138}{}\(\forall x \exists y (\sin(x) = y)\).%
\item\hypertarget{li-139}{}\(\forall y \exists x (\sin(x) = y)\).%
\item\hypertarget{li-140}{}\(\forall x \forall y (x^3 = y^3 \imp x = y)\).%
\end{enumerate}
\par\smallskip
\item[9.]\hypertarget{exercise-9}{}
          Suppose \(P(x)\) is some predicate for which the statement \(\forall x P(x)\) is true. Is it also the case that \(\exists x P(x)\) is true? In other words, is the statement \(\forall x P(x) \imp \exists x P(x)\) always true? Is the converse always true? Explain.
        %
\par\smallskip
\item[10.]\hypertarget{exercise-10}{}
          For each of the statements below, give a domain of discourse for which the statement is true, and a domain for which the statement is false.
        %
\leavevmode%
\begin{enumerate}[label=(\alph*)]
\item\hypertarget{li-145}{}\(\forall x \exists y (y^2 = x)\).%
\item\hypertarget{li-146}{}\(\forall x \forall y \exists z (x \lt  z \lt  y)\).%
\item\hypertarget{li-147}{}\(\exists x \forall y \forall z (y \lt  z \imp y \le x \le z)\) Hint: domains need not be infinite.%
\end{enumerate}
\par\smallskip
\end{exerciselist}
\typeout{************************************************}
\typeout{Section 1.3 Sets}
\typeout{************************************************}
\section[Sets]{Sets}\label{sec_intro-sets}
\typeout{************************************************}
\typeout{Introduction  }
\typeout{************************************************}

      The most fundamental objects we will use in our studies (and really in all of math) are
      \terminology{sets}
      \index{set}. Much of what follows might be review, but it is very important that you are fluent in the language of set theory. Most of the notation we use below is standard, although some might be a little different than what you have seen before.
    %
\par

      For us, a set will simply be an unordered collection of objects. Two examples: we could consider the set of all actors who have played \emph{The Doctor} on \emph{Doctor Who}\index{Doctor Who}, or the set of natural numbers between 1 and 10 inclusive. In the first case, Tom Baker is a element (or member) of the set, while Idris Elba, among many others, is not an element of the set. Also, the two examples are of different sets. Two sets are equal exactly if they contain the exact same elements. For example, the set containing all of the vowels in the declaration of independence is precisely the same set as the set of vowels in the word ``questionably'' (namely, all of them); we do not care about order or repetitions, just whether the element is in the set or not.
    %
\typeout{************************************************}
\typeout{Subsection 1.3.1 Notation}
\typeout{************************************************}
\subsection[Notation]{Notation}\label{subsec_notation}

      We need some notation to make talking about sets easier. Consider,
      \begin{equation*}
        A = \{1, 2, 3\}.
      \end{equation*}
    %
\par

      This is read, ``\(A\) is the set containing the elements 1, 2 and 3.'' We use curly braces ``\(\{,~~ \}\)'' to enclose elements of a set. Some more notation:
      \begin{equation*}
        a \in \{a, b, c\}.
      \end{equation*}
    %
\par

      The symbol ``\(\in\)'' is read ``is in'' or ``is an element of.'' Thus the above means that \(a\) is an element of the set containing the letters \(a\), \(b\), and \(c\). Note that this is a true statement. It would also be true to say that \(d\) is not in that set:
      \begin{equation*}
        d \not\in \{a, b, c\}.
      \end{equation*}
    %
\par

      Be warned: we write ``\(x \in A\)'' when we wish to express that one of the elements of the set \(A\) is \(x\). For example, consider the set,
      \begin{equation*}
        A = \{1, b, \{x, y, z\}, \emptyset\}.
      \end{equation*}
    %
\par

      This is a strange set, to be sure. It contains four elements: the number 1, the letter b, the set \(\{x,y,z\}\), and the empty set (\(\emptyset = \{ \}\), the set containing no elements). Is \(x\) in \(A\)? The answer is no. None of the four elements in \(A\) are the letter \(x\), so we must conclude that \(x \notin A\). Similarly, consider the set \(B = \{1,b\}\). Even though the elements of \(B\) are elements of \(A\), we cannot say that the \emph{set} \(B\) is one of the elements of \(A\). Therefore \(B \notin A\). (Soon we will see that \(B\) is a \emph{subset} of \(A\), but this is different from being an element of \(A\).)
    %
\par

      We have described the sets above by listing their elements. Sometimes this is hard to do, especially when there are lots of elements in the set (perhaps infinitely many). For instance, if we want \(A\) to be the set of all even natural numbers, would could write,
      \begin{equation*}
        A = \{0, 2, 4, 6, \ldots\},
      \end{equation*}
      but this is a little imprecise. Better would be
      \begin{equation*}
        A = \{x \in \N \st \exists n\in \N ( x = 2 n)\}.
      \end{equation*}
    %
\par

      Breaking that down: ``\(x \in \N\)'' means \(x\) is in the set \(\N\) \label{notation-2}
 (the set of natural numbers, starting with 0), \(:\) \label{notation-3}
 is read ``such that'' and ``\(\exists n\in \N (x = 2n) \)
      '' is read ``there exists an \(n\) in the natural numbers for which \(x\) is two times \(n\)'' (in other words, \(x\) is even). Slightly easier might be,
      \begin{equation*}
        A = \{x \st \mbox{  is even} \}.
      \end{equation*}
    %
\par

      Note: sometimes people use \(|\) or \(\backepsilon\) for the ``such that'' symbol instead of the colon.
    %
\par

      Defining a set using this sort of notation is very useful, although it takes some practice to read them correctly. It is a way to describe the set of all things that satisfy some condition (the condition is the logical statement after the ``\(\st\)'' symbol). Here are some more examples.
    %
\begin{example}[]\label{example-10}

          Describe each of the following sets both in words and by listing out enough elements to see the pattern.
        %
\leavevmode%
\begin{enumerate}
\item\hypertarget{li-151}{}\(\{x \st x + 3 \in \N\}\).%
\item\hypertarget{li-152}{}\(\{x \in \N \st x + 3 \in \N\}\).%
\item\hypertarget{li-153}{}\(\{x \st x \in \N \vee -x \in \N\}\).%
\item\hypertarget{li-154}{}\(\{x \st x \in \N \wedge -x \in \N\}\).%
\end{enumerate}
\par\medskip\noindent%
\textbf{Solution.}\quad \leavevmode%
\begin{enumerate}
\item\hypertarget{li-155}{}
              This is the set of all number which are 3 less than a natural number (i.e., that if you add 3 to them, you get a natural number). The set could also be written as \(\{-3, -2, -1, 0, 1, 2, \ldots\}\) (note that 0 is a natural number, so
              \(-3\) is in this set because \(-3 + 3 = 0\)).
            %
\item\hypertarget{li-156}{}
              This is the set of all natural numbers which are 3 less than a natural number. So here we just have \(\{0, 1, 2,3 \ldots\}\).
            %
\item\hypertarget{li-157}{}
              This is the set of all integers
              \index{integers} (positive and negative whole numbers, written \(\Z\)). In other words, \(\{\ldots, -2, -1, 0, 1, 2, \ldots\}\).
            %
\item\hypertarget{li-158}{}
              Here we want all numbers \(x\) such that \(x\) and \(-x\) are natural numbers. There is only one: 0. So we have the set \(\{0\}\).
            %
\end{enumerate}
\end{example}
\par

      We already have a lot of notation, and there is more yet. Below is a handy chart of symbols. Some of these will be discussed in greater detail as we move forward.
    %
\begin{assemblage}{Set Theory Notation}\label{assemblage-9}\par\medskip

  \leavevmode%
\begin{description}
\item[\(\{, \}\)]\hypertarget{li-159}{} We use these \emph{braces} to enclose the elements of a set. So \(\{1,2,3\}\) is the set containing 1, 2, and 3. %
\item[\(\st\)]\hypertarget{li-160}{}\(\{x \st x > 2\}\) is the set of all \(x\) \emph{such that} \(x\) is greater than 2.%
\item[\(\in\)]\hypertarget{li-161}{} \(2 \in \{1,2,3\}\) asserts that 2 is \emph{an element of} the set \(\{1,2,3\}\). %
\item[\(\not\in\)]\hypertarget{li-162}{} \(4 \notin \{1,2,3\}\) because 4 \emph{is not an element of} the set \(\{1,2,3\}\). %
\item[\(\subseteq\)]\hypertarget{li-163}{}\(A \subseteq B\) asserts that \emph{\(A\) is a subset of \(B\)}: every element of \(A\) is also an element of \(B\).%
\item[\(\subset\)]\hypertarget{li-164}{}\(A \subset B\) asserts that  \emph{\(A\) is a proper subset of \(B\)}: every element of \(A\) is also an element of \(B\), but \(A \ne B\).%
\item[\(\cap\)]\hypertarget{li-165}{}\(A \cap B\), is the \emph{intersection of \(A\) and \(B\)}: the set containing all elements which are elements of both \(A\) and \(B\).%
\item[\(\cup\)]\hypertarget{li-166}{}\(A \cup B\) is the \emph{union of \(A\) and \(B\)}: is the set containing all elements which are elements of \(A\) or \(B\) or both.%
\item[\(\times\)]\hypertarget{li-167}{}\(A \times B\) is the \emph{Cartesian product of \(A\) and   \(B\)}: the set of all ordered pairs \((a,b)\) with \(a \in A\) and \(b \in B\).%
\item[\(\setminus\)]\hypertarget{li-168}{}\(A \setminus B\) is \emph{\(A\) set-minus \(B\)}: the set containing all elements of \(A\) which are not elements of \(B\).%
\item[\(\bar{A}\)]\hypertarget{li-169}{}The \emph{complement of \(A\)} is the set of everything which is not an element of \(A\). %
\item[\(\left|A\right|\)]\hypertarget{li-170}{}The \emph{cardinality (or size) of \(A\)} is the number of elements in \(A\).%
\end{description}

%
\end{assemblage}
\begin{assemblage}{Special sets}\label{assemblage-10}\par\medskip

        \leavevmode%
\begin{description}
\item[\(\emptyset\)]\hypertarget{li-171}{}The \emph{empty set} is the set which contains no elements.%
\item[\(\U\)]\hypertarget{li-172}{}The \emph{universe set} is the set of all elements.%
\item[\(\N\)]\hypertarget{li-173}{}The set of natural numbers. That is, \(\N = \{0, 1, 2, 3\ldots\}\).%
\item[\(\Z\)]\hypertarget{li-174}{}The set of integers. That is, \(\Z = \{\ldots, -2, -1, 0, 1, 2, 3, \ldots\}\).%
\item[\(\Q\)]\hypertarget{li-175}{}The set of rational numbers.%
\item[\(\R\)]\hypertarget{li-176}{}The set of real numbers.%
\item[\(\pow(A)\)]\hypertarget{li-177}{}The \emph{power set} of any set \(A\) is the set of all subsets of \(A\).%
\end{description}

      %
\end{assemblage}
\begin{investigation}[]\label{investigation-4}
\leavevmode%
\begin{enumerate}
\item\hypertarget{li-178}{}
            Find the cardinality of each set below.
          %
\begin{enumerate}
\item\hypertarget{li-179}{}\(A = \{3,4,\ldots, 15\}\).%
\item\hypertarget{li-180}{}\(B = \{n \in \N \st 2 \lt  n \le 200\}\).%
\item\hypertarget{li-181}{}\(C = \{n \le 100 \st n \in \N \wedge \exists m \in \N (n = 2m+1)\}\).%
\end{enumerate}
%
\item\hypertarget{li-182}{}
          Find two sets \(A\) and \(B\) for which \(|A| = 5\), \(|B| = 6\), and \(|A\cup B| = 9\). What is \(|A \cap B|\)?%
\item\hypertarget{li-183}{}
          Find sets \(A\) and \(B\) with \(|A| = |B|\) such that \(|A\cup B| = 7\) and \(|A \cap B| = 3\). What is \(|A|\)?%
\item\hypertarget{li-184}{}
          Let \(A = \{1,2,\ldots, 10\}\). Define \(\mathcal{B}_2 = \{B \subseteq A \st |B| = 2\}\). Find \(|\mathcal{B}_2|\).%
\item\hypertarget{li-185}{}
            For any sets \(A\) and \(B\), define \(AB = \{ab \st a\in A \wedge b \in B\}\). If \(A = \{1,2\}\) and \(B = \{2,3,4\}\), what is \(|AB|\)? What is \(|A \times B|\)?
        %
\end{enumerate}
\end{investigation}
\typeout{************************************************}
\typeout{Subsection 1.3.2 Relationships Between Sets}
\typeout{************************************************}
\subsection[Relationships Between Sets]{Relationships Between Sets}\label{subsection-5}

      We have already said what it means for two sets to be equal: they have exactly the same elements. Thus, for example,
      \begin{equation*}
        \{1, 2, 3\} = \{2, 1, 3\}.
      \end{equation*}
    %
\par

      (Remember, the order the elements are written down in does not matter.) Also,
      \begin{equation*}
        \{1, 2, 3\} = \{1, 1+1, 1+1+1\} = \{I, II, III\}
      \end{equation*}
      since these are all ways to write the set containing the first three positive integers (how we write them doesn't matter, just what they are).
    %
\par

      What about the sets \(A = \{1, 2, 3\}\) and \(B = \{1, 2, 3, 4\}\)? Clearly \(A \ne B\), but notice that every element of \(A\) is also an element of \(B\). Because of this we say that \(A\) is a \emph{subset}
      \index{subset} of \(B\), or in symbols \(A \subset B\) or \(A \subseteq B\). (Both symbols are read ``is a subset of.'' The difference is that sometimes we want to say that \(A\) is either equal to or a subset of \(B\), in which case we use \(\subseteq\). This is analoguous to the difference between \(\lt\) and \(\le\).)
    %
\begin{example}[]\label{example-11}

          Let \(A = \{1, 2, 3, 4, 5, 6\}\), \(B = \{2, 4, 6\}\), \(C = \{1, 2, 3\}\) and \(D = \{7, 8, 9\}\). Determine which of the following are true, false, or meaningless.
        %
\leavevmode%
\begin{enumerate}
\item\hypertarget{li-186}{}\(A \subset B\).%
\item\hypertarget{li-187}{}\(B \subset A\).%
\item\hypertarget{li-188}{}\(B \in C\).%
\item\hypertarget{li-189}{}\(\emptyset \in A\).%
\item\hypertarget{li-190}{}\(\emptyset \subset A\).%
\item\hypertarget{li-191}{}\(A \lt  D\).%
\item\hypertarget{li-192}{}\(3 \in C\).%
\item\hypertarget{li-193}{}\(3 \subset C\).%
\item\hypertarget{li-194}{}\(\{3\} \subset C\).%
\end{enumerate}
\par\medskip\noindent%
\textbf{Solution.}\quad \leavevmode%
\begin{enumerate}
\item\hypertarget{li-195}{}
              False. For example, \(1\in A\) but \(1 \notin B\).
            %
\item\hypertarget{li-196}{}
              True. Every element in \(B\) is an element in \(A\).
            %
\item\hypertarget{li-197}{}
              False. The elements in \(C\) are 1, 2, and 3. The \emph{set} \(B\) is not equal to 1, 2, or 3.
            %
\item\hypertarget{li-198}{}
              False. \(A\) has exactly 6 elements, and none of them are the empty set.
            %
\item\hypertarget{li-199}{}
              True. Everything in the empty set (nothing) is also an element of \(A\). Notice that the empty set is a subset of every set.
            %
\item\hypertarget{li-200}{}
              Meaningless. A set cannot be less than another set.
            %
\item\hypertarget{li-201}{}
              True. \(3\) is one of the elements of the set \(C\).
            %
\item\hypertarget{li-202}{}
              Meaningless. \(3\) is not a set, so it cannot be a subset of another set.
            %
\item\hypertarget{li-203}{}
              True. \(3\) is the only element of the set \(\{3\}\), and is an element of \(C\), so every element in \(\{3\}\) is an element of \(C\).
            %
\end{enumerate}
\end{example}
\par

      In the example above, \(B\) is a subset of \(A\). You might wonder what other sets are subsets of \(A\). If you collect all these subsets of \(A\) into a new set, we get a set of sets. We call the set of all subsets of \(A\) the \emph{power set}
      \index{power set} of \(A\), and write it \(\pow(A)\).
    %
\begin{example}[]\label{example-12}

          Let \(A = \{1,2,3\}\). Find \(\pow(A)\).
        %
\par\medskip\noindent%
\textbf{Solution.}\quad 
          \(\pow(A)\) is a set of sets, all of which are subsets of \(A\). So
          \begin{equation*}
            \pow(A) = \{ \emptyset, \{1\}, \{2\}, \{3\}, \{1,2\}, \{1, 3\}, \{2,3\}, \{1,2,3\}\}.
          \end{equation*}
        %
\par

          Notice that while \(2 \in A\), it is wrong to write \(2 \in \pow(A)\) since none of the elements in \(\pow(A)\) are numbers! On the other hand, we do have \(\{2\} \in \pow(A)\) because \(\{2\} \subseteq A\).
        %
\par

          What does a subset of \(\pow(A)\) look like? Notice that \(\{2\} \not\subseteq \pow(A)\) because not everything in \(\{2\}\) is in \(\pow(A)\). But we do have \(\{ \{2\} \} \subseteq \pow(A)\). The only element of \(\{\{2\}\}\) is the set \(\{2\}\) which is also an element of \(\pow(A)\). We could take the collection of all subsets of \(\pow(A)\) and call that \(\pow(\pow(A))\). Or even the power set of that set of sets of sets.
        %
\end{example}
\par

      Another way to compare sets is by their size. Notice that in the example above, \(A\) has 6 elements, \(B\), \(C\), and \(D\) all have 3 elements. The size of a set is called the set's \emph{cardinality}
      \index{cardinality}. We would write \(|A| = 6\), \(|B| = 3\), and so on. For sets that have a finite number of elements, the cardinality of the set is simply the number of elements in the set. Note that the cardinality of \(\{ 1, 2, 3, 2, 1\}\) is 3. We do not count repeats (in fact, \(\{1, 2, 3, 2, 1\}\) is exactly the same set as \(\{1, 2, 3\}\)). There are sets with infinite cardinality, such as \(\N\), the set of rational numbers (written \(\mathbb Q\)), the set of even natural numbers, and the set of real numbers (\(\mathbb R\)). It is possible to distinguish between different infinite cardinalities, but that is beyond the scope of this text. For us, a set will either be infinite, or finite; if it is finite, the we can determine its cardinality by counting elements.
    %
\begin{example}[]\label{example-13}
\leavevmode%
\begin{enumerate}
\item\hypertarget{li-204}{}
              Find the cardinality of \(A = \{23, 24, \ldots, 37, 38\}\).
            %
\item\hypertarget{li-205}{}
              Find the cardinality of \(B = \{1, \{2, 3, 4\}, \emptyset\}\).
            %
\item\hypertarget{li-206}{}
              If \(C = \{1,2,3\}\), what is the cardinality of \(\pow(C)\)?
            %
\end{enumerate}
\par\medskip\noindent%
\textbf{Solution.}\quad \leavevmode%
\begin{enumerate}
\item\hypertarget{li-207}{}
              Since \(38 - 23 = 15\), we can conclude that the cardinality of the set is \(|A| = 16\) (you need to add one since 23 is included).
            %
\item\hypertarget{li-208}{}
              Here \(|B| = 3\). The three elements are the number 1, the set \(\{2,3,4\}\), and the empty set.
            %
\item\hypertarget{li-209}{}
              We wrote out the elements of the power set \(\pow(C)\) above, and there are 8 elements (each of which is a set). So \(|\pow(C)| = 8\).\footnotemark
            %
\end{enumerate}
\end{example}
\typeout{************************************************}
\typeout{Subsection 1.3.3 Operations On Sets}
\typeout{************************************************}
\subsection[Operations On Sets]{Operations On Sets}\label{subsection-6}

      Is it possible to add two sets? Not really, however there is something similar. If we want to combine two sets to get the collection of objects that are in either set, then we can take the \terminology{union}
      \index{union} of the two sets. Symbolically,
      \begin{equation*}
        C = A \cup B,
      \end{equation*}
      read, ``\(C\) is the union of \(A\) and \(B\),'' means that the elements of \(C\) are exactly the elements which are either an element of \(A\) or an element of \(B\) (or an element of both). For example, if \(A = \{1, 2, 3\}\) and \(B = \{2, 3, 4\}\), then \(A \cup B = \{1, 2, 3, 4\}\).
    %
\par

      The other common operation on sets is \terminology{intersection}
      \index{intersection}. We write,
      \begin{equation*}
        C = A \cap B
      \end{equation*}
      and say, ``\(C\) is the intersection of \(A\) and \(B\),'' when the elements in \(C\) are precisely those both in \(A\) and in \(B\). So if \(A = \{1, 2, 3\}\) and \(B = \{2, 3, 4\}\), then \(A \cap B = \{2, 3\}\).
    %
\par

      Often when dealing with sets, we will have some understanding as to what ``everything'' is. Perhaps we are only concerned with natural numbers. In this case we would say that our \terminology{universe} is \(\N\). Sometimes we denote this universe by \(\U\). Given this context, we might wish to speak of all the elements which are \emph{not} in a particular set. We say \(B\) is the \terminology{complement}
      \index{complement} of \(A\), and write,
      \begin{equation*}
        B = \bar A
      \end{equation*}
      when \(B\) contains every element not contained in \(A\). So, if our universe is \(\{1, 2,\ldots, 9, 10\}\), and \(A = \{2, 3, 5, 7\}\), then \(\bar A = \{1, 4, 6, 8, 9,10\}\).
    %
\par

      Of course we can perform more than one operation at a time. For example, consider
      \begin{equation*}
        A \cap \bar B.
      \end{equation*}
    %
\par

      This is the set of all elements which are both elements of \(A\) and not elements of \(B\). What have we done? We've started with \(A\) and removed all of the elements which were in \(B\). Another way to write this is the \terminology{set difference}
      \index{set difference}
      \index{difference, of sets}:
      \begin{equation*}
        A \cap \bar B = A \setminus B.
      \end{equation*}
    %
\par

      It is important to remember that these operations (union, intersection, complement, and difference) on sets produce other sets. Don't confuse these with the symbols from the previous section (element of and subset of). \(A \cap B\) is a set, while \(A \subseteq B\) is true or false. This is the same difference as between \(3 + 2\) (which is a number) and \(3 \le 2\) (which is false).
    %
\begin{example}[]\label{example-14}

          Let \(A = \{1, 2, 3, 4, 5, 6\}\), \(B = \{2, 4, 6\}\), \(C = \{1, 2, 3\}\) and \(D = \{7, 8, 9\}\). If the universe is \(\U = \{1, 2, \ldots, 10\}\), find:
        %
\leavevmode%
\begin{enumerate}
\item\hypertarget{li-210}{}\(A \cup B\).%
\item\hypertarget{li-211}{}\(A \cap B\).%
\item\hypertarget{li-212}{}\(B \cap C\).%
\item\hypertarget{li-213}{}\(A \cap D\).%
\item\hypertarget{li-214}{}\(\bar{B \cup C}\).%
\item\hypertarget{li-215}{}\(A \setminus B\).%
\item\hypertarget{li-216}{}\((D \cap \bar C) \cup \bar{A \cap B}\).%
\item\hypertarget{li-217}{}\(\emptyset \cup C\).%
\item\hypertarget{li-218}{}\(\emptyset \cap C\).%
\end{enumerate}
\par\medskip\noindent%
\textbf{Solution.}\quad \leavevmode%
\begin{enumerate}
\item\hypertarget{li-219}{}\(A \cup B = \{1, 2, 3, 4, 5, 6\} = A\) since everything in \(B\) is already in \(A\).%
\item\hypertarget{li-220}{}\(A \cap B = \{2, 4, 6\} = B\) since everything in \(B\) is in \(A\).%
\item\hypertarget{li-221}{}\(B \cap C = \{2\}\) as the only element of both \(B\) and \(C\) is 2.%
\item\hypertarget{li-222}{}\(A \cap D = \emptyset\) since \(A\) and \(D\) have no common elements.%
\item\hypertarget{li-223}{}\(\bar{B \cup C} = \{5, 7, 8, 9, 10\}\). First we find that \(B \cup C = \{1, 2, 3, 4, 6\}\), then we take everything not in that set.%
\item\hypertarget{li-224}{}\(A \setminus B = \{1, 3, 5\}\) since the elements 1, 3, and 5 are in \(A\) but not in \(B\). This is the same as \(A \cap \bar B\).%
\item\hypertarget{li-225}{}\((D \cap \bar C) \cup \bar{A \cap B} = \{1, 3, 5, 7, 8, 9, 10\}.\) The set contains all elements that are either in \(D\) but not in \(C\) (\(\{7,8,9\}\)), or not in both \(A\) and \(B\) (\(\{1,3,5,7,8,9,10\}\)).%
\item\hypertarget{li-226}{}\(\emptyset \cup C = C\) since nothing is added by the empty set.%
\item\hypertarget{li-227}{}\(\emptyset \cap C = \emptyset\) since nothing can be both in a set and in the empty set.%
\end{enumerate}
\end{example}
\par

      You might notice that the symbols for union and intersection slightly resemble the logic symbols for ``or'' and ``and.'' This is no accident. What does it mean for \(x\) to be an element of \(A\cup B\)? It means that \(x\) is an element of \(A\) \emph{or} \(x\) is an element of \(B\) (or both). That is,
      \begin{equation*}
        x \in A \cup B \qquad \Iff \qquad x \in A \vee x \in B.
      \end{equation*}
    %
\par

      Similarly,
      \begin{equation*}
        x \in A \cap B \qquad \Iff \qquad x \in A \wedge x \in B.
      \end{equation*}
    %
\par

      Also,
      \begin{equation*}
        x \in \bar A \qquad \Iff \qquad \neg (x \in A).
      \end{equation*}
      which says \(x\) is an element of the complement of \(A\) if \(x\) is not an element of \(A\).
    %
\par

      There is one more way to combine sets which will be useful for us: the \terminology{Cartesian product}, \(A \times B\). This sounds fancy but is nothing you haven't seen before. When you graph a function in calculus, you graph it in the Cartesian plane. This is the set of all ordered pairs of real numbers \((x,y)\). We can do this for \emph{any} pair of sets, not just the real numbers with themselves.
    %
\par

      Put another way, \(A \times B = \{(a,b) \st a \in A \wedge b \in B\}\). The first coordinate comes from the first set and the second coordinate comes from the second set. Sometimes we will want to take the Cartesian product of a set with itself, and this is fine: \(A \times A = \{(a,b) \st a, b \in A\}\) (we might also write \(A^2\) for this set). Notice that in \(A \times A\), we still want \emph{all} ordered pairs, not just the ones where the first and second coordinate are the same. We can also take products of 3 or more sets, getting ordered triples, or quadruples, and so on.
    %
\begin{example}[]\label{example-15}

          Let \(A = \{1,2\}\) and \(B = \{3,4,5\}\). Find \(A \times B\) and \(A \times A\). How many elements do you expect to be in \(B \times B\)?
        %
\par\medskip\noindent%
\textbf{Solution.}\quad 
          \(A \times B = \{(1,3), (1,4), (1,5), (2,3), (2,4), (2,5)\}\).
        %
\par

          \(A \times A = A^2 = \{(1,1), (1,2), (2,1), (2,2)\}\).
        %
\par

          \(|B\times B| = 9\). There will be 3 pairs with first coordinate \(3\), three more with first coordinate \(4\), and a final three with first coordinate \(5\).
        %
\end{example}
\typeout{************************************************}
\typeout{Subsection 1.3.4 Venn Diagrams}
\typeout{************************************************}
\subsection[Venn Diagrams]{Venn Diagrams}\label{subsection-7}

      \index{Venn diagram} There is a very nice visual tool we can use to represent operations on sets. Venn diagrams display sets as intersecting circles. We can shade the region we are talking about when we carry out an operation. We can also represent cardinality of a particular set by putting the number in the corresponding region.
    %
% group protects changes to lengths, releases boxes (?)
{% begin: group for a single side-by-side
% set panel max height to practical minimum, created in preamble
\setlength{\panelmax}{0pt}
\newsavebox{\panelboxBimage}
\savebox{\panelboxBimage}{
\resizebox{0.34\linewidth}{!}{{
            \begin{tikzpicture}[fill=gray!50,scale=0.85]
 \draw[thick] \circleA \circleAlabel \circleB \circleBlabel \twosetbox;
\end{tikzpicture}
}
}}
\newlength{\phBimage}\setlength{\phBimage}{\ht\panelboxBimage+\dp\panelboxBimage}
\settototalheight{\phBimage}{\usebox{\panelboxBimage}}
\setlength{\panelmax}{\maxof{\panelmax}{\phBimage}}
\newsavebox{\panelboxCimage}
\savebox{\panelboxCimage}{
\resizebox{0.34\linewidth}{!}{{
            \begin{tikzpicture}[scale=.60, fill=gray!50]
 \draw[thick] \circleA \circleAlabel \circleB \circleBlabel \circleC \circleClabel \threesetbox;
\end{tikzpicture}
}
}}
\newlength{\phCimage}\setlength{\phCimage}{\ht\panelboxCimage+\dp\panelboxCimage}
\settototalheight{\phCimage}{\usebox{\panelboxCimage}}
\setlength{\panelmax}{\maxof{\panelmax}{\phCimage}}
\leavevmode%
% begin: side-by-side as figure/tabular
% \tabcolsep change local to group
\setlength{\tabcolsep}{0.08\textwidth}
% @{} suppress \tabcolsep at extremes, so margins behave as intended
\begin{figure}
\hspace*{0.08\textwidth}%
\begin{tabular}{@{}*{2}{c}@{}}
\begin{minipage}[c][\panelmax][t]{0.34\textwidth}\usebox{\panelboxBimage}\end{minipage}&
\begin{minipage}[c][\panelmax][t]{0.34\textwidth}\usebox{\panelboxCimage}\end{minipage}\end{tabular}
\end{figure}
% end: side-by-side as tabular/figure
}% end: group for a single side-by-side
\par

      Each circle represents a set. The rectangle containing the circles represents the universe. To represent combinations of these sets, we shade the corresponding region. For example, we could draw \(A \cap B\) as:
    %
\leavevmode%
\begin{figure}
\centering
{
        \begin{tikzpicture}[fill=gray!50,scale=0.85]
	\begin{scope}
	\clip \circleA;
	\fill \circleB;
	\end{scope}
 \draw[thick] \circleA \circleAlabel \circleB \circleBlabel \twosetbox;
\end{tikzpicture}
}
\end{figure}
\par

      Here is a representation of \(A \cap \bar B\), or equivalently \(A \setminus B\):
    %
\leavevmode%
\begin{figure}
\centering
{
        \begin{tikzpicture}[fill=gray!50,scale=0.85]
	\begin{scope}
	\clip \twosetbox \circleB;
	\fill \circleA;
	\end{scope}
 \draw[thick] \circleA \circleAlabel \circleB \circleBlabel \twosetbox;
\end{tikzpicture}
}
\end{figure}
\par

      A more complicated example is \((B \cap C) \cup (C \cap \bar A)\), as seen below.
    %
\leavevmode%
\begin{figure}
\centering
{
        \begin{tikzpicture}[fill=gray!50,scale=0.65]
	\fill \circleC;
	\begin{scope}
	    \clip \circleC;
	    \fill[white] \circleA \circleB;
	  \end{scope}
	  \begin{scope}
	  	\clip \circleC;
	  	\fill \circleB;
	  \end{scope}
 \draw[thick] \circleA \circleAlabel \circleB \circleBlabel \circleC \circleClabel \threesetbox;
\end{tikzpicture}
}
\end{figure}
\par

      Notice that the shaded regions above could also be arrived at in another way. We could have started with all of \(C\), then excluded the region where \(C\) and \(A\) overlap outside of \(B\). That region is \((A \cap C) \cap \bar B\). So the above Venn diagram also represents \(C \cap \bar{\left((A\cap C)\cap \bar B\right)}.\) So using just the picture, we have determined that
      \begin{equation*}
        (B \cap C) \cup (C \cap \bar A) = C \cap \bar{\left((A\cap C)\cap \bar B\right)}.
      \end{equation*}
    %
\typeout{************************************************}
\typeout{Exercises 1.3.5 Exercises}
\typeout{************************************************}
\subsection[Exercises]{Exercises}\label{exercises-2}
\begin{exerciselist}
\item[1.]\hypertarget{exercise-11}{}
          Let \(A = \{1,2,3,4,5\}\), \(B = \{3,4,5,6,7\}\), and \(C = \{2,3,5\}\).
        %
\leavevmode%
\begin{enumerate}[label=(\alph*)]
\item\hypertarget{li-228}{} Find \(A \cap B\). %
\item\hypertarget{li-229}{} Find \(A \cup B\). %
\item\hypertarget{li-230}{} Find \(A \setminus B\). %
\item\hypertarget{li-231}{} Find \(A \cap \overline{(B \cup C)}\). %
\item\hypertarget{li-232}{} Find \(A \times C\). %
\item\hypertarget{li-233}{} Is \(C \subseteq A\)? Explain. %
\item\hypertarget{li-234}{} Is \(C \subseteq B\)? Explain. %
\end{enumerate}
\par\smallskip
\item[2.]\hypertarget{exercise-12}{}
          Let \(A = \{x \in \N \st 3 \le x \le 13\}\), \(B = \{x \in \N \st x \mbox{ is even} \}\), and \(C = \{x \in \N \st x \mbox{ is odd} \}\).
        %
\leavevmode%
\begin{enumerate}[label=(\alph*)]
\item\hypertarget{li-242}{} Find \(A \cap B\). %
\item\hypertarget{li-243}{} Find \(A \cup B\). %
\item\hypertarget{li-244}{} Find \(B \cap C\). %
\item\hypertarget{li-245}{} Find \(B \cup C\). %
\end{enumerate}
\par\smallskip
\item[3.]\hypertarget{exercise-13}{}
          Find an example of sets \(A\) and \(B\) such that \(A\cap B = \{3, 5\}\) and \(A \cup B = \{2, 3, 5, 7, 8\}\).
        %
\par\smallskip
\item[4.]\hypertarget{exercise-14}{}
          Find an example of sets \(A\) and \(B\) such that \(A \subseteq B\) and \(A \in B\).
        %
\par\smallskip
\item[5.]\hypertarget{exercise-15}{}
          Recall \(\Z = \{\ldots,-2,-1,0, 1,2,\ldots\}\) (the integers). Let \(\Z^+ = \{1, 2, 3, \ldots\}\) be the positive integers. Let \(2\Z\) be the even integers, \(3\Z\) be the multiples of 3, and so on.
        %
\leavevmode%
\begin{enumerate}[label=(\alph*)]
\item\hypertarget{li-250}{} Is \(\Z^+ \subseteq 2\Z\)? Explain. %
\item\hypertarget{li-251}{} Is \(2\Z \subseteq \Z^+\)? Explain. %
\item\hypertarget{li-252}{} Find \(2\Z \cap 3\Z\). Describe the set in words, and using set notation. %
\item\hypertarget{li-253}{} Express \(\{x \in \Z \st \exists y\in \Z (x = 2y \vee x = 3y)\}\) as a union or intersection of two sets already described in this problem. %
\end{enumerate}
\par\smallskip
\item[6.]\hypertarget{exercise-16}{}
          Let \(A_2\) be the set of all multiples of 2 except for \(2\). Let \(A_3\) be the set of all multiples of 3 except for 3. And so on, so that \(A_n\) is the set of all multiple of \(n\) except for \(n\), for any \(n \ge 2\). Describe (in words) the set \(\bar{A_2 \cup A_3 \cup A_4 \cup \cdots}\).
        %
\par\smallskip
\item[7.]\hypertarget{exercise-17}{}
          Draw a Venn diagram to represent each of the following:
        %
\leavevmode%
\begin{enumerate}[label=(\alph*)]
\item\hypertarget{li-258}{}\(A \cup \bar B\)%
\item\hypertarget{li-259}{}\(\bar{(A \cup B)}\)%
\item\hypertarget{li-260}{}\(A \cap (B \cup C)\)%
\item\hypertarget{li-261}{}\((A \cap B) \cup C\)%
\item\hypertarget{li-262}{}\(\bar A \cap B \cap \bar C\)%
\item\hypertarget{li-263}{}\((A \cup B) \setminus C\)%
\end{enumerate}
\par\smallskip
\item[8.]\hypertarget{exercise-18}{}
          Describe a set in terms of \(A\) and \(B\) (using set notation) which has the following Venn diagram:
        %
\leavevmode%
\begin{figure}
\centering
{
\begin{tikzpicture}[fill=gray!50, scale=0.75]
\scope
\clip (-2,-2) rectangle (2,2)
    (1,0) circle (1);
\fill (0,0) circle (1);
\endscope
\scope
\clip (-2,-2) rectangle (2,2)
    (0,0) circle (1);
\fill (1,0) circle (1);
\endscope
\draw[thick] (0,0) circle (1) (-1,.7)  node [text=black,above] {\(A\)}
    (1,0) circle (1) (2,.7)  node [text=black,above] {\(B\)}
    (-1.5,-1.5) rectangle (2.5,1.5);
\end{tikzpicture}
}
\end{figure}
\par\smallskip
\item[9.]\hypertarget{exercise-19}{}
          Find the following cardinalities:
        %
\leavevmode%
\begin{enumerate}[label=(\alph*)]
\item\hypertarget{li-270}{}\(|A|\) when \(A = \{4,5,6,\ldots,37\}\)%
\item\hypertarget{li-271}{}\(|A|\) when \(A = \{x \in \Z \st -2 \le x \le 100\}\)%
\item\hypertarget{li-272}{}\(|A \cap B|\) when \(A = \{x \in \N \st x \le 20\}\) and \(B = \{x \in \N \st x \mbox{ is prime} \}\)%
\end{enumerate}
\par\smallskip
\item[10.]\hypertarget{exercise-20}{}
          Let \(A = \{a, b, c, d\}\). Find \(\pow(A)\).
        %
\par\smallskip
\item[11.]\hypertarget{exercise-21}{}
          Let \(A = \{1,2,\ldots, 10\}\). How many subsets of \(A\) contain exactly one element (i.e., how many \terminology{singleton} subsets are there). How many \terminology{doubleton} (containing exactly two elements) are there?
        %
\par\smallskip
\item[12.]\hypertarget{exercise-22}{}
          Let \(A = \{1,2,3,4,5,6\}\). Find all sets \(B \in \pow(A)\) which have the property \(\{2,3,5\} \subseteq B\).
        %
\par\smallskip
\item[13.]\hypertarget{exercise-23}{}
          Find an example of sets \(A\) and \(B\) such that \(|A| = 4\), \(|B| = 5\), and \(|A \cup B| = 9\).
        %
\par\smallskip
\item[14.]\hypertarget{exercise-24}{}
          Find an example of sets \(A\) and \(B\) such that \(|A| = 3\), \(|B| = 4\), and \(|A \cup B| = 5\).
        %
\par\smallskip
\item[15.]\hypertarget{exercise-25}{}
          Are there sets \(A\) and \(B\) such that \(|A| = |B|\), \(|A\cup B| = 10\), and \(|A\cap B| = 5\)? Explain.
        %
\par\smallskip
\item[16.]\hypertarget{exercise-26}{}
          In a regular deck of playing cards there are 26 red cards and 12 face cards. Explain, using sets and what you have learned about cardinalities, why there are only 32 cards which are either red or a face card.
        %
\par\smallskip
\end{exerciselist}
\typeout{************************************************}
\typeout{Section 1.4 Functions}
\typeout{************************************************}
\section[Functions]{Functions}\label{sec_intro-functions}
\typeout{************************************************}
\typeout{Introduction  }
\typeout{************************************************}

      A
      \terminology{function}\index{function} is a rule that assigns each input exactly one output. The set of all inputs for a function is called the
      \terminology{domain}\index{domain}. The set of all allowable outputs is called the
      \terminology{codomain}\index{codomain}. We would write \(f:X \to Y\) to describe a function with name \(f\), domain \(X\) and codomain \(Y\). This does not tell us \emph{which} function \(f\) is though. To define the function, we must describe the rule. This is often done by giving a formula to compute the output for any input (although this is certainly not the only way to describe the rule). %
\par
 For example, consider the function \(f:\N \to \N\) defined by \(f(x) = x^2 + 3\). Here the domain and codomain are the same set (the natural numbers). The rule is: take your input, multiply it by itself and add 3. This works because we can apply this rule to every natural number (every element of the domain) and the result is always a natural number (an element of the codomain). Notice though that not every natural number actually is an output (there is no way to get 0, 1, 2, 5, etc.). The set of natural numbers that are \emph{actually outputs} is called the
      \terminology{range}\index{range} of the function (in this case, the range is \(\{3, 4, 7, 12, 19, 28, \ldots\}\), all the natural numbers that are 3 more than a perfect square).
    %
\par

      The key thing that makes a rule actually a \emph{function} is that there is \emph{exactly one} output for each input. That is, it is important that the rule be a good rule. What output do we assign to the input 7? There can only be one answer for any particular function.
    %
\par

      The description of the rule can vary greatly. We might just give a list of each output for each input. You could also describe the function with a table or a graph or in words.
    %
\begin{example}[]\label{example-16}

          The following are all examples of functions:
        %
\leavevmode%
\begin{enumerate}
\item\hypertarget{li-276}{}\(f:\Z \to \Z\) defined by \(f(n) = 3n\). The domain and codomain are both the set of integers. However, the range is only the set of integer multiples of 3.%
\item\hypertarget{li-277}{}\(g: \{1,2,3\} \to \{a,b,c\}\) defined by \(g(1) = c\), \(g(2) = a\) and \(g(3) = a\). The domain is the set \(\{1,2,3\}\), the codomain is the set \(\{a,b,c\}\) and the range is the set \(\{a,c\}\). Note that \(g(2)\) and \(g(3)\) are the same element of the codomain. This is okay since each element in the domain still has only one output.%
\item\hypertarget{li-278}{}\(h:\{1,2,3\} \to \{1,2,3\}\) defined as follows:

            \leavevmode%
\begin{figure}
\centering
{
          \begin{tikzpicture}[scale=0.85]
  \draw[->] (-1,1) node[above] {1} -- (0,0) node[below] {2};
  \draw[->] (0,1) node[above] {2} -- (-1,0) node[below] {1};
  \draw[->] (1,1) node[above] {3} -- (1,0) node[below] {3};
\end{tikzpicture}
}
\end{figure}

            This means that the function \(f\) sends 1 to 2, 2 to 1 and 3 to 3: just follow the arrows.
          %
\end{enumerate}
\end{example}
\par

      The arrow diagram used to define the function above can be very helpful in visualizing functions. We will often be working with functions with \emph{finite} domains, so this kind of picture is often more useful than a traditional graph of a function. A graph of the function in example 3 above would look like this:
    %
\leavevmode%
\begin{figure}
\centering
{
      \begin{tikzpicture}[scale=0.75]
  \draw[thin, gray!50] (0,0) grid (3.5, 3.5);
  \draw[->, thick] (0,0) -- (0,3.5);
 \draw[->, thick] (0,0) -- (3.5,0);
 \fill (1,2) circle (3pt) (2,1) circle (3pt) (3,3) circle (3pt);
\end{tikzpicture}
}
\end{figure}
\par

      It would be absolutely WRONG to connect the dots or try to fit them to some curve. There are only three elements in the domain. A curve suggests that the domain contains an entire interval of real numbers. Remember, we are not in calculus any more!
    %
\par

      Since we will so often use functions with small domains and codomains, let's adopt some notation that is a little easier to work with than that of examples 2 and 3 above. All we need is some clear way of denoting which elements of the codomain each element in the domain is assigned. In fact, writing a table of values would work perfectly:

      \leavevmode%
\begin{table}
\centering
\begin{tabular}{llllll}
\multicolumn{1}{cA}{\(x\)}&0&1&2&3&4\tabularnewline\hrulethin
\multicolumn{1}{cA}{\(f(x)\)}&3&3&2&4&1
\end{tabular}
\end{table}

    %
\par

      We simplify this further by writing this as a matrix with each input directly over its output:
      \begin{equation*}
        f = \begin{pmatrix}0 \amp 1 \amp 2\amp 3 \amp 4 \\ 3 \amp 3 \amp 2 \amp 4 \amp 1\end{pmatrix}
      \end{equation*}
      Note this is just notation and not the same sort of matrix you would find in a linear algebra class (it does not make sense to do operations with these matrices, or row reduce them, for example).
    %
\par

      It is important to know how to determine if a rule is or is not a function. Drawing the arrow diagrams can help.
    %
\begin{example}[]\label{example-17}

          Which of the following diagrams represent a function? Let \(X = \{1,2,3,4\}\) and \(Y = \{a,b,c,d\}\).
        %
% group protects changes to lengths, releases boxes (?)
{% begin: group for a single side-by-side
% set panel max height to practical minimum, created in preamble
\setlength{\panelmax}{0pt}
\newsavebox{\panelboxPimage}
\savebox{\panelboxPimage}{
\resizebox{0.3\linewidth}{!}{{
        \begin{tikzpicture}[scale=0.9]
  \draw[->] (-1.5,1) node[above] {1} -- (1.5,0) node[below] {\(d\)};
  \draw[->] (-.5,1) node[above] {2} -- (-1.5,0) node[below] {\(a\)};
  \draw[->] (.5,1) node[above] {3} -- (.5, 0) node[below] {\(c\)};
  \draw[->] (1.5,1) node[above] {4} -- (-.5, 0) node[below] {\(b\)};
            \node[above] at (0,1.5) {$f:X \to Y$};
\end{tikzpicture}
}
}}
\newlength{\phPimage}\setlength{\phPimage}{\ht\panelboxPimage+\dp\panelboxPimage}
\settototalheight{\phPimage}{\usebox{\panelboxPimage}}
\setlength{\panelmax}{\maxof{\panelmax}{\phPimage}}
\newsavebox{\panelboxQimage}
\savebox{\panelboxQimage}{
\resizebox{0.3\linewidth}{!}{{
        \begin{tikzpicture}[scale=0.9]
  \draw[->] (-1.5,1) node[above] {1} -- (1.5,0) node[below] {\(d\)};
  \draw[->] (-.5,1) node[above] {2} -- (-1.6,0) node[below] {\(a\)};
  \draw[->] (.5,1) node[above] {3} -- (-1.4, 0);
  \draw[->] (1.5,1) node[above] {4} -- (-.5, 0) node[below] {\(b\)};
  \draw (.5,0) node[below] {\(c\)};
  \node[above] at (0,1.5) { $g:X \to Y$};
\end{tikzpicture}
}
}}
\newlength{\phQimage}\setlength{\phQimage}{\ht\panelboxQimage+\dp\panelboxQimage}
\settototalheight{\phQimage}{\usebox{\panelboxQimage}}
\setlength{\panelmax}{\maxof{\panelmax}{\phQimage}}
\newsavebox{\panelboxRimage}
\savebox{\panelboxRimage}{
\resizebox{0.3\linewidth}{!}{{
        \begin{tikzpicture}[scale=0.9]
  \draw (-1.5,1) node[above] {1};
  \draw[->] (-.5,1) node[above] {2} (-.6,1) -- (-1.5,0) node[below] {\(a\)};
  \draw[->] (-.4,1) -- (.5,0);
  \draw[->] (.5,1) node[above] {3} -- (1.5, 0) node[below] {\(d\)};
  \draw[->] (1.5,1) node[above] {4} -- (-.5, 0) node[below] {\(b\)};
  \draw (.5,0) node[below] {\(c\)};
            \node[above] at (0,1.5) {$h:X \to Y$};
\end{tikzpicture}
}
}}
\newlength{\phRimage}\setlength{\phRimage}{\ht\panelboxRimage+\dp\panelboxRimage}
\settototalheight{\phRimage}{\usebox{\panelboxRimage}}
\setlength{\panelmax}{\maxof{\panelmax}{\phRimage}}
\leavevmode%
% begin: side-by-side as figure/tabular
% \tabcolsep change local to group
\setlength{\tabcolsep}{0.0166666666666667\textwidth}
% @{} suppress \tabcolsep at extremes, so margins behave as intended
\begin{figure}
\hspace*{0.0166666666666667\textwidth}%
\begin{tabular}{@{}*{3}{c}@{}}
\begin{minipage}[c][\panelmax][t]{0.3\textwidth}\usebox{\panelboxPimage}\end{minipage}&
\begin{minipage}[c][\panelmax][t]{0.3\textwidth}\usebox{\panelboxQimage}\end{minipage}&
\begin{minipage}[c][\panelmax][t]{0.3\textwidth}\usebox{\panelboxRimage}\end{minipage}\end{tabular}
\end{figure}
% end: side-by-side as tabular/figure
}% end: group for a single side-by-side
\par\medskip\noindent%
\textbf{Solution.}\quad 
          \(f\) is a function. So is \(g\). There is no problem with an element of the codomain not being the output for any input, and there is no problem with \(a\) from the codomain being the output of both 2 and 3 from the domain. We could use our two-line matrix notation to write these as
          \begin{equation*}
            f= \begin{pmatrix} 1 \amp 2 \amp 3 \amp 4 \\ d \amp a \amp c \amp b \end{pmatrix} \qquad g = \begin{pmatrix} 1 \amp 2 \amp 3 \amp 4 \\ d \amp a \amp a \amp b \end{pmatrix}.
          \end{equation*}

        %
\par

          However, \(h\) is not a function. In fact, it fails for two reasons. First, the element 1 from the domain has not been mapped to any element from the codomain. Second, the element 2 from the domain has been mapped to more than one element from the codomain (\(a\) and \(c\)). Note that either one of these problems is enough to make a rule not a function. In general, neither of the following mappings are functions:
        %
% group protects changes to lengths, releases boxes (?)
{% begin: group for a single side-by-side
% set panel max height to practical minimum, created in preamble
\setlength{\panelmax}{0pt}
\newsavebox{\panelboxSimage}
\savebox{\panelboxSimage}{
\resizebox{0.18\linewidth}{!}{{
        \begin{tikzpicture}[scale=0.9]
  \fill (-1, 1.2) circle (.1) (0,1.2) circle (.1) (1, 1.2) circle (.1);
  \draw[->] (-1, 1) -- (-.5,0);
  \draw[->] (1,1) -- (.5, 0);
  \draw (-.5, -0.2) circle (.1) (.5, -0.2) circle (.1);
\end{tikzpicture}
}
}}
\newlength{\phSimage}\setlength{\phSimage}{\ht\panelboxSimage+\dp\panelboxSimage}
\settototalheight{\phSimage}{\usebox{\panelboxSimage}}
\setlength{\panelmax}{\maxof{\panelmax}{\phSimage}}
\newsavebox{\panelboxTimage}
\savebox{\panelboxTimage}{
\resizebox{0.27\linewidth}{!}{{
       \begin{tikzpicture}[scale=0.9]
  \fill (-1, 1.2) circle (.1) (0,1.2) circle (.1) (1, 1.2) circle (.1);
  \draw[->] (-1.1, 1) -- (-1.5, 0);
  \draw[->] (-.9, 1) -- (-.5, 0);
  \draw[->] (0,1) -- (.5,0);
  \draw[->] (1,1) -- (1.5, 0);
  \draw (-.5, -0.2) circle (.1) (.5, -0.2) circle (.1) (-1.5, -0.2) circle (.1) (1.5, -0.2) circle (.1);
\end{tikzpicture}
}
}}
\newlength{\phTimage}\setlength{\phTimage}{\ht\panelboxTimage+\dp\panelboxTimage}
\settototalheight{\phTimage}{\usebox{\panelboxTimage}}
\setlength{\panelmax}{\maxof{\panelmax}{\phTimage}}
\leavevmode%
% begin: side-by-side as figure/tabular
% \tabcolsep change local to group
\setlength{\tabcolsep}{0.1375\textwidth}
% @{} suppress \tabcolsep at extremes, so margins behave as intended
\begin{figure}
\hspace*{0.1375\textwidth}%
\begin{tabular}{@{}*{2}{c}@{}}
\begin{minipage}[c][\panelmax][t]{0.18\textwidth}\usebox{\panelboxSimage}\end{minipage}&
\begin{minipage}[c][\panelmax][t]{0.27\textwidth}\usebox{\panelboxTimage}\end{minipage}\end{tabular}
\end{figure}
% end: side-by-side as tabular/figure
}% end: group for a single side-by-side
\par

          It might also be helpful to think about how you would write the two-line notation for \(h\). We would have something like:
          \begin{equation*}
            h=\begin{pmatrix} 1 \amp 2 \amp 3 \amp 4 \\ \amp a,c? \amp d \amp b\end{pmatrix}.
          \end{equation*}
          There is nothing under 1 (bad) and we needed to put more than one thing under 2 (very bad). With a rule that is actually a function, two-line notation will always ``work''.
        %
\end{example}
\typeout{************************************************}
\typeout{Subsection 1.4.1 Surjections, Injections, and Bijections}
\typeout{************************************************}
\subsection[Surjections, Injections, and Bijections]{Surjections, Injections, and Bijections}\label{subsec_surj-inj-bij}

      We now turn to investigating special properties functions might or might not possess.
    %
\par

      In the examples above, you may have noticed that sometimes there are elements of the codomain which are not in the range. When this sort of the thing \emph{does not} happen, (that is, when everything in the codomain is in the range) we say the function is
      \terminology{onto}\index{onto} or that the function maps the domain \emph{onto} the codomain. This terminology should make sense: the function puts the domain (entirely) on top of the codomain. The fancy math term for an onto function is a
      \terminology{surjection}\index{surjection}, and we say that an onto function is a
      \terminology{surjective} function.
    %
\par

      In pictures:
    %
% group protects changes to lengths, releases boxes (?)
{% begin: group for a single side-by-side
% set panel max height to practical minimum, created in preamble
\setlength{\panelmax}{0pt}
\newsavebox{\panelboxUimage}
\savebox{\panelboxUimage}{
\resizebox{0.3\linewidth}{!}{{
  \begin{tikzpicture}
  \fill (-1.5, 1.2) circle (.1) (-.5,1.2) circle (.1) (.5, 1.2) circle (.1) (1.5,1.2) circle (.1);
  \draw[->] (-1.5, 1) -- (-1,0);
  \draw[->] (-.5,1) -- (0, 0);
  \draw[->] (.5, 1) -- (.9,0);
  \draw[->] (1.5,1) -- (1.1,0);
  \draw (-1, -0.2) circle (.1) (0, -0.2) circle (.1) (1, -0.2) circle (.1);
\end{tikzpicture}
}
}}
\newlength{\phUimage}\setlength{\phUimage}{\ht\panelboxUimage+\dp\panelboxUimage}
\settototalheight{\phUimage}{\usebox{\panelboxUimage}}
\setlength{\panelmax}{\maxof{\panelmax}{\phUimage}}
\newsavebox{\panelboxVimage}
\savebox{\panelboxVimage}{
\resizebox{0.3\linewidth}{!}{{
  \begin{tikzpicture}
  \fill (-1.5, 1.2) circle (.1) (-.5,1.2) circle (.1) (.5, 1.2) circle (.1) (1.5,1.2) circle (.1);
  \draw[->] (-1.5, 1) -- (-1.1,0);
  \draw[->] (-.5,1) -- (-.9, 0);
  \draw[->] (.5, 1) -- (.9,0);
  \draw[->] (1.5,1) -- (1.1,0);
  \draw (-1, -0.2) circle (.1) (0, -0.2) circle (.1) (1, -0.2) circle (.1);
\end{tikzpicture}
}
}}
\newlength{\phVimage}\setlength{\phVimage}{\ht\panelboxVimage+\dp\panelboxVimage}
\settototalheight{\phVimage}{\usebox{\panelboxVimage}}
\setlength{\panelmax}{\maxof{\panelmax}{\phVimage}}
\leavevmode%
% begin: side-by-side as figure/tabular
% \tabcolsep change local to group
\setlength{\tabcolsep}{0.1\textwidth}
% @{} suppress \tabcolsep at extremes, so margins behave as intended
\begin{figure}
\hspace*{0.1\textwidth}%
\begin{tabular}{@{}*{2}{c}@{}}
\begin{minipage}[c][\panelmax][t]{0.3\textwidth}\usebox{\panelboxUimage}\end{minipage}&
\begin{minipage}[c][\panelmax][t]{0.3\textwidth}\usebox{\panelboxVimage}\end{minipage}\tabularnewline
\parbox[t]{0.3\textwidth}{\captionof{figure}{Surjective.\label{figure-19}}
}&
\parbox[t]{0.3\textwidth}{\captionof{figure}{Not surjective.\label{figure-20}}
}\end{tabular}
\end{figure}
% end: side-by-side as tabular/figure
}% end: group for a single side-by-side
\begin{example}[]\label{example-18}

          Which functions are surjective (i.e., onto)?
        %
\leavevmode%
\begin{enumerate}
\item\hypertarget{li-279}{}\(f:\Z \to \Z\) defined by \(f(n) = 3n\).%
\item\hypertarget{li-280}{}\(g: \{1,2,3\} \to \{a,b,c\}\) defined by \(g = \begin{pmatrix}1 \amp 2 \amp 3 \\ c \amp a \amp a \end{pmatrix}\).%
\item\hypertarget{li-281}{}\(h:\{1,2,3\} \to \{1,2,3\}\) defined as follows:
            \leavevmode%
\begin{figure}
\centering
{
            \begin{tikzpicture}
  \draw[->] (-1,1) node[above] {1} -- (0,0) node[below] {2};
  \draw[->] (0,1) node[above] {2} -- (-1,0) node[below] {1};
  \draw[->] (1,1) node[above] {3} -- (1,0) node[below] {3};
\end{tikzpicture}
}
\end{figure}
%
\end{enumerate}
\par\medskip\noindent%
\textbf{Solution.}\quad \leavevmode%
\begin{enumerate}
\item\hypertarget{li-282}{}\(f\) is not surjective. There are elements in the codomain which are not in the range. For example, no \(n \in \Z\) gets mapped to the number 1 (the rule would say that \(\frac{1}{3}\) would be sent to 1, but \(\frac{1}{3}\) is not in the domain). In fact, the range of the function is \(3\Z\) (the integer multiples of 3), which is not equal to \(\Z\).%
\item\hypertarget{li-283}{}\(g\) is not surjective. There is no \(x \in \{1,2,3\}\) (the domain) for which \(g(x) = b\), so \(b\), which is in the codomain, is not in the range. Notice that there is an element from the codomain ``missing'' from the bottom row of the matrix.%
\item\hypertarget{li-284}{}\(h\) is surjective. Every element of the codomain is also in the range. Nothing in the codomain is missed.%
\end{enumerate}
\end{example}
\par

      To be a function, a rule cannot assign a single element of the domain to two or more different elements of the codomain. However, we have seen that the reverse \emph{is} permissible: a function might assign the same element of the codomain to two or more different elements of the domain. When this \emph{does not} occur (that is, when each element of the codomain is assigned to at most one element of the domain) then we say the function is
      \terminology{one-to-one}\index{one-to-one}. Again, this terminology makes sense: we are sending at most one element from the domain to one element from the codomain. One input to one output. The fancy math term for a one-to-one function is an
      \terminology{injection}\index{injection}. We call one-to-one functions
      \terminology{injective} functions.
    %
\par

      In pictures:
    %
% group protects changes to lengths, releases boxes (?)
{% begin: group for a single side-by-side
% set panel max height to practical minimum, created in preamble
\setlength{\panelmax}{0pt}
\newsavebox{\panelboxXimage}
\savebox{\panelboxXimage}{
\resizebox{0.4\linewidth}{!}{{
  \begin{tikzpicture}
  \fill (-1.5, 1.2) circle (.1) (-.5,1.2) circle (.1) (.5, 1.2) circle (.1) (1.5,1.2) circle (.1);
  \draw[->] (-1.5, 1) -- (-2,0);
  \draw[->] (-.5,1) -- (-1, 0);
  \draw[->] (.5, 1) -- (1,0);
  \draw[->] (1.5,1) -- (2,0);
  \draw (-2, -0.2) circle (.1) (-1, -.2) circle (.1) (0, -0.2) circle (.1) (1, -0.2) circle (.1) (2, -0.2) circle (.1);
\end{tikzpicture}
}
}}
\newlength{\phXimage}\setlength{\phXimage}{\ht\panelboxXimage+\dp\panelboxXimage}
\settototalheight{\phXimage}{\usebox{\panelboxXimage}}
\setlength{\panelmax}{\maxof{\panelmax}{\phXimage}}
\newsavebox{\panelboxYimage}
\savebox{\panelboxYimage}{
\resizebox{0.4\linewidth}{!}{{
  \begin{tikzpicture}
  \fill (-1.5, 1.2) circle (.1) (-.5,1.2) circle (.1) (.5, 1.2) circle (.1) (1.5,1.2) circle (.1);
  \draw[->] (-1.5, 1) -- (-2,0);
  \draw[->] (-.5,1) -- (-1, 0);
  \draw[->] (.5, 1) -- (.9,0);
  \draw[->] (1.5,1) -- (1.1,0);
  \draw (-2, -0.2) circle (.1) (-1, -.2) circle (.1) (0, -0.2) circle (.1) (1, -0.2) circle (.1) (2, -0.2) circle (.1);
\end{tikzpicture}
}
}}
\newlength{\phYimage}\setlength{\phYimage}{\ht\panelboxYimage+\dp\panelboxYimage}
\settototalheight{\phYimage}{\usebox{\panelboxYimage}}
\setlength{\panelmax}{\maxof{\panelmax}{\phYimage}}
\leavevmode%
% begin: side-by-side as figure/tabular
% \tabcolsep change local to group
\setlength{\tabcolsep}{0.05\textwidth}
% @{} suppress \tabcolsep at extremes, so margins behave as intended
\begin{figure}
\hspace*{0.05\textwidth}%
\begin{tabular}{@{}*{2}{c}@{}}
\begin{minipage}[c][\panelmax][t]{0.4\textwidth}\usebox{\panelboxXimage}\end{minipage}&
\begin{minipage}[c][\panelmax][t]{0.4\textwidth}\usebox{\panelboxYimage}\end{minipage}\tabularnewline
\parbox[t]{0.4\textwidth}{\captionof{figure}{Injective.\label{figure-22}}
}&
\parbox[t]{0.4\textwidth}{\captionof{figure}{
            Not injective.
        \label{figure-23}}
}\end{tabular}
\end{figure}
% end: side-by-side as tabular/figure
}% end: group for a single side-by-side
\begin{example}[]\label{example-19}

          Which functions are injective (i.e., one-to-one)?
        %
\leavevmode%
\begin{enumerate}
\item\hypertarget{li-285}{}\(f:\Z \to \Z\) defined by \(f(n) = 3n\).%
\item\hypertarget{li-286}{}\(g: \{1,2,3\} \to \{a,b,c\}\) defined by \(g = \begin{pmatrix}1 \amp 2 \amp 3 \\ c \amp a \amp a \end{pmatrix}\).%
\item\hypertarget{li-287}{}\(h:\{1,2,3\} \to \{1,2,3\}\) defined as follows:
            \leavevmode%
\begin{figure}
\centering
{
          \begin{tikzpicture}
  \draw[->] (-1,1) node[above] {1} -- (0,0) node[below] {2};
  \draw[->] (0,1) node[above] {2} -- (-1,0) node[below] {1};
  \draw[->] (1,1) node[above] {3} -- (1,0) node[below] {3};
\end{tikzpicture}
}
\end{figure}
%
\end{enumerate}
\par\medskip\noindent%
\textbf{Solution.}\quad \leavevmode%
\begin{enumerate}
\item\hypertarget{li-288}{}\(f\) is injective. Each element in the codomain is assigned to at \emph{most} one element from the domain. If \(x\) is a multiple of three, then only \(x/3\) is mapped to \(x\). If \(x\) is not a multiple of 3, then there is no input corresponding to the output \(x\).%
\item\hypertarget{li-289}{}\(g\) is not injective. Both inputs \(2\) and \(3\) are assigned the output \(a\). Notice that there is an element from the codomain that appears more than once on the bottom row of the matrix.%
\item\hypertarget{li-290}{}\(h\) is injective. Each output is only an output once.%
\end{enumerate}
\end{example}
\par

      From the examples above, it should be clear that there are functions which are surjective, injective, both, or neither. In the case when a function is both one-to-one and onto (an injection and surjection), we say the function is a
      \terminology{bijection}\index{bijection}, or that the function is a
      \terminology{bijective} function.
    %
\typeout{************************************************}
\typeout{Subsection 1.4.2 Inverse Image}
\typeout{************************************************}
\subsection[Inverse Image]{Inverse Image}\label{subsection-9}

      When discussing functions, we have notation for talking about an element of the domain (say \(x\)) and its corresponding element in the codomain (we write \(f(x)\), which \emph{is} the element corresponding to \(x\)). It would also be nice to start with some element of the codomain (say \(y\)) and talk about which element or elements (if any) from the domain get sent to it. We could write ``those \(x\) in the domain such that \(f(x) = y\),'' but this is a lot of writing. Here is some notation to make our lives easier.
    %
\par

      Suppose \(f:X \to Y\) is a function. For \(y \in Y\) (an element of the codomain), we write \(f\inv(y)\)\label{notation-4}
 to represent the \emph{set} of all elements in the domain \(X\) which get sent to \(y\). That is, \(f\inv(y) = \{x \in X \st f(x) = y\}\). We say that \(f\inv(y)\) is the
      \terminology{complete inverse image}\index{inverse image} of \(y\) under \(f\).
    %
\par

      WARNING: \(f\inv(y)\) is not an inverse function! Inverse functions only exist for bijections, but \(f\inv(y)\) is defined for any function \(f\). The point: \(f\inv(y)\) is a \emph{>set}, not an \emph{element} of the domain.
    %
\begin{example}[]\label{example-20}

          Consider the function \(f:\{1,2,3,4,5,6\} \to \{a,b,c,d\}\) given by
          \begin{equation*}
            f = \begin{pmatrix}1 \amp 2 \amp 3 \amp 4 \amp 5 \amp 6 \\ a \amp a \amp b \amp c \amp c \amp c\end{pmatrix}.
          \end{equation*}
          Find the complete inverse image of each element in the codomain.
        %
\par\medskip\noindent%
\textbf{Solution.}\quad 
          Remember, we are looking for sets.
          \begin{equation*}
            f\inv(a) = \{1,2\}
          \end{equation*}
        %
\begin{equation*}
          f\inv(b) = \{3\}
        \end{equation*}\begin{equation*}
          f\inv(c) = \{4,5,6\}
        \end{equation*}\begin{equation*}
          f\inv(d) = \emptyset.
        \end{equation*}\end{example}
\begin{example}[]\label{example-21}

          Consider the function \(g:\Z \to \Z\) defined by \(g(n) = n^2 + 1\). Find \(g\inv(1)\), \(g\inv(2)\), \(g\inv(3)\) and \(g\inv(10)\).
        %
\par\medskip\noindent%
\textbf{Solution.}\quad 
          To find \(g\inv(1)\), we need to find all integers \(n\) such that \(n^2 + 1 = 1\). Clearly only 0 works, so \(g\inv(1) = \{0\}\) (note that even though there is only one element, we still write it as a set with one element in it).
        %
\par

          To find \(g\inv(2)\), we need to find all \(n\) such that \(n^2 + 1 = 2\). We see \(g\inv(2) = \{-1,1\}\).
        %
\par

          If \(n^2 + 1 = 3\), then we are looking for an \(n\) such that \(n^2 = 2\). There are no such integers so \(g\inv(3) = \emptyset\).
        %
\par

          Finally, \(g\inv(10) = \{-3, 3\}\) because \(g(-3) = 10\) and \(g(3) = 10\).
        %
\end{example}
\par

      Since \(f\inv(y)\) is a set, it makes sense to ask for \(\card{f\inv(y)}\), the number of elements in the domain which map to \(y\).
    %
\begin{example}[]\label{example-22}

          Find a function \(f:\{1,2,3,4,5\} \to \N\) such that \(\card{f\inv(7)} = 5\).
        %
\par\medskip\noindent%
\textbf{Solution.}\quad 
          There is only one such function. We need five elements of the domain to map to the number \(7 \in \N\). Since there are only five elements in the domain, all of them must map to 7. So
          \begin{equation*}
            f = \begin{pmatrix}1 \amp 2 \amp 3 \amp 4 \amp 5 \\ 7 \amp 7 \amp 7 \amp 7 \amp 7\end{pmatrix}.
          \end{equation*}
        %
\end{example}
\begin{assemblage}{Function Definitions}\label{assemblage-11}\par\medskip

        \leavevmode%
\begin{itemize}[label=\textbullet]
\item{}
              A
              \terminology{function} is a rule that assigns each element of a set, called the
              \terminology{domain}, to exactly one element of a second set, called the
              \terminology{codomain}.
            %
\item{}
              Notation: \(f:X \to Y\) is our way of saying that the function is called \(f\), the domain is the set \(X\), and the codomain is the set \(Y\).
            %
\item{}
              To specify the rule for a function with small domain, use
              \terminology{two-line notation} by writing a matrix with each output directly below its corresponding input, as in:
              \begin{equation*}
                f = \begin{pmatrix}1 \amp 2 \amp 3 \amp 4 \\ 2 \amp 1 \amp 3 \amp 1 \end{pmatrix}.
              \end{equation*}

            %
\item{}\(f(x) = y\) means the element \(x\) of the domain (input) is assigned to the element \(y\) of the codomain. We say \(y\) is an output. Alternatively, we call \(y\) the
              \terminology{image of \(x\) under \(f\)}.%
\item{}
              The
              \terminology{range} is a subset of the codomain. It is the set of all elements which are assigned to at least one element of the domain by the function. That is, the range is the set of all outputs.
            %
\item{}
              A function is
              \terminology{injective} (an
              \terminology{injection} or
              \terminology{one-to-one}) if every element of the codomain is the output for
              \terminology{at most} one element from the domain.
            %
\item{}
              A function is
              \terminology{surjective} (a
              \terminology{surjection} or
              \terminology{onto}) if every element of the codomain is the output of
              \terminology{at least} one element of the domain.
            %
\item{}
              A
              \terminology{bijection} is a function which is both an injection and surjection. In other words, if every element of the codomain is the output of
              \terminology{exactly one} element of the domain.
            %
\item{}
              The
              \terminology{complete inverse image} of an element in the codomain, written \(f\inv(y)\), is the \emph{set} of all elements in the domain which are assigned to \(y\) by the function.
            %
\end{itemize}

      %
\end{assemblage}
\typeout{************************************************}
\typeout{Exercises 1.4.3 Exercises}
\typeout{************************************************}
\subsection[Exercises]{Exercises}\label{exercises-3}
\begin{exerciselist}
\item[1.]\hypertarget{exercise-27}{}
            Write out all functions \(f: \{1,2,3\} \to \{a,b\}\) (using two-line notation). How many are there? How many are injective? How many are surjective? How many are both?
          %
\par\smallskip
\item[2.]\hypertarget{exercise-28}{}
            Write out all functions \(f: \{1,2\} \to \{a,b,c\}\) (in two-line notation). How many are there? How many are injective? How many are surjective? How many are both?
          %
\par\smallskip
\item[3.]\hypertarget{exercise-29}{}
            Consider the function \(f:\{1,2,3,4,5\} \to \{1,2,3,4\}\) given by the table below:
          %
\leavevmode%
\begin{table}
\centering
\begin{tabular}{llllll}
\multicolumn{1}{cA}{\(x\)}&1&2&3&4&5\tabularnewline\hrulethin
\multicolumn{1}{cA}{\(f(x)\)}&3&2&4&1&2
\end{tabular}
\end{table}
\leavevmode%
\begin{enumerate}[label=(\alph*)]
\item\hypertarget{li-300}{}
                Is \(f\) injective? Explain.
              %
\item\hypertarget{li-301}{}
                Is \(f\) surjective? Explain.
              %
\item\hypertarget{li-302}{}
                Write the function using two-line notation.
              %
\end{enumerate}
\par\smallskip
\item[4.]\hypertarget{exercise-30}{}
            Consider the function \(f:\{1,2,3,4\} \to \{1,2,3,4\}\) given by the graph below.
          %
\leavevmode%
\begin{figure}
\centering
{
            \begin{tikzpicture}[scale=1]
  \draw[thin, gray!50] (0,0) grid (4.5, 4.5);
  \draw[<->, thick] (0,4.5) node[left] {\(f(x)\)} -- (0,0) -- (4.5,0) node[below right] {\(x\)};
  \foreach \x in {1,2,3,4}
    \draw (\x,0) node[below] { \x} (0, \x) node[left] { \x};
  \fill (1,3) circle (.1) (2,4) circle (.1) (3,1) circle (.1) (4,3) circle (.1);
\end{tikzpicture}
}
\end{figure}
\leavevmode%
\begin{enumerate}[label=(\alph*)]
\item\hypertarget{li-306}{}
                Is \(f\) injective? Explain.
              %
\item\hypertarget{li-307}{}
                Is \(f\) surjective? Explain.
              %
\item\hypertarget{li-308}{}
                Write the function using two-line notation.
              %
\end{enumerate}
\par\smallskip
\item[5.]\hypertarget{exercise-31}{}
            For each function given below, determine whether or not the function is injective and whether or not the function is surjective.
          %
\leavevmode%
\begin{enumerate}[label=(\alph*)]
\item\hypertarget{li-312}{}\(f:\N \to \N\) given by \(f(n) = n+4\).%
\item\hypertarget{li-313}{}\(f:\Z \to \Z\) given by \(f(n) = n+4\).%
\item\hypertarget{li-314}{}\(f:\Z \to \Z\) given by \(f(n) = 5n - 8\).%
\item\hypertarget{li-315}{}\(f:\Z \to \Z\) given by \(f(n) = \begin{cases}n/2 \amp  \mbox{ if  is even} \\ (n+1)/2 \amp \mbox{ if  is odd} . \end{cases}
              \)%
\end{enumerate}
\par\smallskip
\item[6.]\hypertarget{exercise-32}{}
            Let \(A = \{1,2,3,\ldots,10\}\). Consider the function \(f:\pow(A) \to \N\) given by \(f(B) = |B|\). That is, \(f\) takes a subset of \(A\) as an input and outputs the cardinality of that set.
          %
\leavevmode%
\begin{enumerate}[label=(\alph*)]
\item\hypertarget{li-320}{}
                Is \(f\) injective? Prove your answer.
              %
\item\hypertarget{li-321}{}
                Is \(f\) surjective? Prove your answer.
              %
\item\hypertarget{li-322}{}
                Find \(f\inv(1)\).
              %
\item\hypertarget{li-323}{}
                Find \(f\inv(0)\).
              %
\item\hypertarget{li-324}{}
                Find \(f\inv(12)\).
              %
\end{enumerate}
\par\smallskip
\item[7.]\hypertarget{exercise-33}{}
            Let \(A = \{n \in \N \st 0 \le n \le 999\}\) be the set of all numbers with three or fewer digits. Define the function \(f:A \to \N\) by \(f(abc) = a+b+c\), where \(a\), \(b\), and \(c\) are the digits of the number in \(A\). For example, \(f(253) = 2 + 5 + 3 =  10\).
          %
\leavevmode%
\begin{enumerate}[label=(\alph*)]
\item\hypertarget{li-330}{}
                Find \(f\inv(3)\).
              %
\item\hypertarget{li-331}{}
                Find \(f\inv(28)\).
              %
\item\hypertarget{li-332}{}
                Is \(f\) injective. Explain.
              %
\item\hypertarget{li-333}{}
                Is \(f\) surjective. Explain.
              %
\end{enumerate}
\par\smallskip
\item[8.]\hypertarget{exercise-34}{}
            Let \(f:X \to Y\) be some function. Suppose \(3 \in Y\). What can you say about \(f\inv(3)\) if you know,
          %
\leavevmode%
\begin{enumerate}[label=(\alph*)]
\item\hypertarget{li-338}{}\(f\) is injective? Explain.%
\item\hypertarget{li-339}{}\(f\) is surjective? Explain.%
\item\hypertarget{li-340}{}\(f\) is bijective? Explain.%
\end{enumerate}
\par\smallskip
\item[9.]\hypertarget{exercise-35}{}
            Find a set \(X\) and a function \(f:X \to \N\) so that \(f\inv(0) \cup f\inv(1) = X\).
          %
\par\smallskip
\item[10.]\hypertarget{exercise-36}{}
            What can you deduce about the sets \(X\) and \(Y\) if you know
            \dots{}
          %
\leavevmode%
\begin{enumerate}[label=(\alph*)]
\item\hypertarget{li-344}{}
                there is an injective function \(f:X \to Y\)? Explain.
              %
\item\hypertarget{li-345}{}
                there is a surjective function \(f:X \to Y\)? Explain.
              %
\item\hypertarget{li-346}{}
                there is a bijectitve function \(f:X \to Y\)? Explain.
              %
\end{enumerate}
\par\smallskip
\item[11.]\hypertarget{exercise-37}{}
            Suppose \(f:X \to Y\) is a function. Which of the following are possible? Explain.
          %
\leavevmode%
\begin{enumerate}[label=(\alph*)]
\item\hypertarget{li-350}{}\(f\) is injective but not surjective.%
\item\hypertarget{li-351}{}\(f\) is surjective but not injective.%
\item\hypertarget{li-352}{}\(|X| = |Y|\) and \(f\) is injective but not surjective.%
\item\hypertarget{li-353}{}\(|X| = |Y|\) and \(f\) is surjective but not injective.%
\item\hypertarget{li-354}{}\(|X| = |Y|\), \(X\) and \(Y\) are finite, and \(f\) is injective but not surjective.%
\item\hypertarget{li-355}{}\(|X| = |Y|\), \(X\) and \(Y\) are finite, and \(f\) is surjective but not injective.%
\end{enumerate}
\par\smallskip
\item[12.]\hypertarget{exercise-38}{}
            Consider the function \(f:\Z \to \Z\) given by \(f(n) = \begin{cases}n+1 \amp  \mbox{ if  is even} \\ n-3 \amp \mbox{ if  is odd} . \end{cases}
            \)
          %
\leavevmode%
\begin{enumerate}[label=(\alph*)]
\item\hypertarget{li-362}{}
                Is \(f\) injective? Prove your answer.
              %
\item\hypertarget{li-363}{}
                Is \(f\) surjective? Prove your answer.
              %
\end{enumerate}
\par\smallskip
\item[13.]\hypertarget{exercise-39}{}
            At the end of the semester a teacher assigns letter grades to each of her students. Is this a function? If so, what sets make up the domain and codomain, and is the function injective, surjective, bijective, or neither?
          %
\par\smallskip
\item[14.]\hypertarget{exercise-40}{}
            In the game of \emph{Hearts}, four players are each dealt 13 cards from a deck of 52. Is this a function? If so, what sets make up the domain and codomain, and is the function injective, surjective, bijective, or neither?
          %
\par\smallskip
\item[15.]\hypertarget{exercise-41}{}
            Suppose 7 players are playing 5-card stud. Each player initially receives 5 cards from a deck of 52. Is this a function? If so, what sets make up the domain and codomain, and is the function injective, surjective, bijective, or neither?
          %
\par\smallskip
\end{exerciselist}
\typeout{************************************************}
\typeout{Chapter 2 Counting}
\typeout{************************************************}
\chapter[Counting]{Counting}\label{ch_counting}
\typeout{************************************************}
\typeout{Introduction  }
\typeout{************************************************}

One of the first things you learn in mathematics is how to count. Now we want to count large collections of things quickly and precisely. For example:
%
\leavevmode%
\begin{itemize}[label=\textbullet]
\item{}
In a group of 10 people, if everyone shakes hands with everyone else exactly once, how many handshakes took place?
%
\item{}
How many ways can you distribute \(10\) girl scout cookies to \(7\) boy scouts?
%
\item{}
How many anagrams are there of ``anagram''?
%
\item{}
How many subsets of \(\{1,2,3,\ldots, 10\}\) have cardinality \(7\)?
%
\end{itemize}
\par

Before tackling these difficult questions, let's look at the basics of counting.
%
\typeout{************************************************}
\typeout{Section 2.1 Additive and Multiplicative Principles}
\typeout{************************************************}
\section[Additive and Multiplicative Principles]{Additive and Multiplicative Principles}\label{sec_counting-addmult}
\typeout{************************************************}
\typeout{Introduction  }
\typeout{************************************************}
\begin{investigation}[]\label{investigation-5}
\leavevmode%
\begin{enumerate}
\item\hypertarget{li-370}{}A restaurant offers 8 appetizers and 14 entrées. How many choices do you have if:

          %
\begin{enumerate}
\item\hypertarget{li-371}{} you will eat one dish, either an appetizer or an entrée?
            %
\item\hypertarget{li-372}{}
              you are extra hungry and want to eat both an appetizer and an entrée?
            %
\end{enumerate}
%
\item\hypertarget{li-373}{}Think about the methods you used to solve question 1. Write down the rules for these methods.
        %
\item\hypertarget{li-374}{}Do your rules work? A standard deck of playing cards has 26 red cards and 12 face cards.


          %
\begin{enumerate}
\item\hypertarget{li-375}{}
              How many ways can you select a card which is either red or a face card?
            %
\item\hypertarget{li-376}{}
              How many ways can you select a card which is both red and a face card?
            %
\item\hypertarget{li-377}{}
              How many ways can you select two cards so that the first one is red and the second one is a face card?
            %
\end{enumerate}
%
\end{enumerate}
\end{investigation}

    Consider this rather simple counting problem: at Red Dogs and Donuts, there are 14 varieties of donuts, and 16 types of hot dogs. If you want either a donut or a dog, how many options do you have? This isn't too hard, you just add 14 and 16. Will that always work? What is important here?
  %
\begin{assemblage}{Additive Principle}\label{assemblage-12}\par\medskip

      The
      \terminology{additive principle}\index{additive principle} states that if event \(A\) can occur in \(m\) ways, and event \(B\) can occur in \(n\)
      \terminology{disjoint}\index{disjoint} ways, then the event ``\(A\) or \(B\)'' can occur in \(m + n\) ways.
    %
\end{assemblage}
\par

    It is important that the events be disjoint: i.e., that there is no way for \(A\) and \(B\) to both happen at the same time. For example, a standard deck of 52 cards contains \(26\) red cards and \(12\) face cards. However, the number of ways to select a card which is either red or a face card is not \(26 + 12 = 38\). This is because there are 6 cards which are both red and face cards.
  %
\begin{example}[]\label{example-23}

        How many two letter ``words''\index{words} start with either A or B? (A word is just a strings of letters; it doesn't have to be English, or even pronounceable.)
      %
\par\medskip\noindent%
\textbf{Solution.}\quad 
        First, how many two letter words start with A? We just need to select the second letter, which can be accomplished in 26 ways. So there are 26 words starting with A. There are also 26 words that start with B. To select a word which starts with either A or B, we can pick the word from the first 26 or the second 26, for a total of 52 words.
      %
\end{example}
\par

    The additive principle also works with more than two events. Say, in addition to your 14 choices for donuts and 16 for dogs, you would also consider eating one of 15 waffles? How many choices do you have now? You would have \(14 + 16 + 15 = 45\) options.
  %
\begin{example}[]\label{example-24}

        How many two letter words start with one of the 5 vowels?
      %
\par\medskip\noindent%
\textbf{Solution.}\quad 
        There are 26 two letter words starting with A, another 26 starting with E, and so on. We will have 5 groups of 26. So we add 26 to itself 5 times. Of course it would be easier to just multiply \(5\cdot 26\). We are really using the additive principle again, just using multiplication as a shortcut.
      %
\end{example}
\begin{example}[]\label{example-25}

        Suppose you are going for some fro-yo. You can pick one of 6 yogurt choices, and one of 4 toppings. How many choices do you have?
      %
\par\medskip\noindent%
\textbf{Solution.}\quad 
        Break your choices up into disjoint events: \(A\) are the choices with the first topping, \(B\) the choices featuring the second topping, and so on. There are four events; each can occur in 6 ways (one for each yogurt flavor). The events are disjoint, so the total number of choices is \(6 + 6 + 6 + 6 = 24\).
      %
\end{example}
\par

    Note that in both of the previous examples, when using the additive principle on a bunch of events all the same size, it is quicker to multiply. This really is the same, and not just because \(6 + 6 + 6 + 6 = 4\cdot 6\). We can first select the topping in 4 ways (that is, we first select which of the disjoint events we will take). For each of those first 4 choices, we now have 6 choices of yogurt. We have:
  %
\begin{assemblage}{Multiplicative Principle}\label{assemblage-13}\par\medskip

      The
      \terminology{multiplicative principle}\index{multiplicative principle} states that if event \(A\) can occur in \(m\) ways, and each possibility for \(A\) allows for exactly \(n\) ways for event \(B\), then the event ``\(A\) and \(B\)'' can occur in \(m \cdot n\) ways.
    %
\end{assemblage}
\par

    The multiplicative principle generalizes to more than two events as well.
  %
\begin{example}[]\label{example-26}

        How many license plates can you make out of three letters followed by three numerical digits?
      %
\par\medskip\noindent%
\textbf{Solution.}\quad 
        Here we have six events: the first letter, the second letter, the third letter, the first digit, the second digit, and the third digit. The first three events can each happen in 26 ways; the last three can each happen in 10 ways. So the total number of license plates will be \(26\cdot 26\cdot 26 \cdot 10 \cdot 10 \cdot 10\), using the multiplicative principle.
      %
\par

        Does this make sense? Think about how we would pick a license plate. How many choices we would have? First, we need to pick the first letter. There are 26 choices. Now for each of those, there are 26 choices for the second letter: 26 second letters with first letter A, 26 second letters with first letter B, and so on. We add 26 to itself 26 times. Or quicker: there are \(26 \cdot 26\) choices for the first two letters.
      %
\par

        Now for each choice of the first two letters, we have 26 choices for the third letter. That is, 26 third letters for the first two letters AA, 26 choices for the third letter after starting AB, and so on. There are \(26 \cdot 26\) of these \(26\) third letter choices, for a total of \((26\cdot26)\cdot 26\) choices for the first three letters. And for each of these \(26\cdot26\cdot26\) choices of letters, we have a bunch of choices for the remaining digits.
      %
\par

        In fact, there are going to be exactly 1000 choices for the numbers. We can see this because there are 1000 three-digit numbers (000 through 999). This is 10 choices for the first digit, 10 for the second, and 10 for the third. The multiplicative principle says we multiply: \(10\cdot 10 \cdot 10 = 1000\).
      %
\par

        All together, there were \(26^3\) choices for the three letters, and \(10^3\) choices for the numbers, so we have a total of \(26^3 \cdot 10^3\) choices of license plates.
      %
\end{example}
\par

    Careful: ``and'' doesn't mean ``times.'' For example, how many playing cards are both red and a face card? Not \(26 \cdot 12\). The answer is 6, and we needed to know something about cards to answer that question.
  %
\par

    Another caution: how many ways can you select two cards, so that the first one is a red card and the second one is a face card? This looks more like the multiplicative principle (you are counting two separate events) but the answer is not \(26 \cdot 12\) here either. The problem is that while there are 26 ways for the first card to be selected, it is not the case that \emph{for each} of those there are 12 ways to select the second card. If the first card was both red and a face card, then there would be only 11 choices for the second card.
    \footnote{To solve this problem, you could break it into two cases. First, count how many ways there are to select the two cards when the first card is a red non-face card. Second, count how many ways when the first card is a red face card. Doing so makes the events in each separate case independent, so the multiplicative principle can be applied.\label{fn-1}}
  %
\begin{example}[Counting functions]\label{ex_counting-functions-all}

      How many functions \(f:\{1,2,3,4,5\} \to \{a,b,c,d\}\) are there?
    \par\medskip\noindent%
\textbf{Solution.}\quad 
        Remember that a function sends each element of the domain to exactly one element of the codomain.  To determine a function, we just need to specify the image of each element in the domain.  Where can we send 1?  There are 4 choices.  Where can we send 2?  Again, 4 choices.  What we have here is 5 ``events'' (picking the image of an element in the domain) each of which can happen in 4 ways (the choices for that image).  Thus there are \(4 \cdot 4 \cdot 4 \cdot 4 \cdot 4 = 4^5\) functions.
      %
\par

        This is more than just an example of how we can use the multiplicative principle in a particular counting question.  What we have here is a general interpretation of certain applications of the multiplicative principle using rigorously defined mathematical objects: functions.  Whenever we have a counting question that asks for the the number of outcomes of a repeated event, we can interpret that as asking for the number of functions from \(\{1,2,\ldots, n\}\) (where \(n\) is the number of times the event is repeated) to  \(\{1,2,\ldots,k\}\) (where \(k\) is the number of ways that event can occur).
      %
\end{example}
\typeout{************************************************}
\typeout{Subsection 2.1.1 Counting With Sets}
\typeout{************************************************}
\subsection[Counting With Sets]{Counting With Sets}\label{subsec_countingWithSets}

      Do you believe the additive and multiplicative principles? How would you convince someone they are correct? This is surprisingly difficult. They seem so simple, so obvious. But why do they work?
    %
\par

      To make things clearer, and more mathematically rigorous, we will use sets. Do not skip this section! It might seem like we are just trying to give a proof of these principles, but we are doing a lot more. If we understand the additive and multiplicative principles rigorously, we will be better at applying them, and knowing when and when not to apply them at all.
    %
\par

      We will look at the additive and multiplicative principles in a slightly different way. Instead of thinking about event \(A\) and event \(B\), we want to think of a set \(A\) and a set \(B\). The sets will contain all the different ways the event can happen. (It will be helpful to be able to switch back and forth between these two models when checking that we have counted correctly.) Here's what we mean:
    %
\begin{example}[]\label{example-28}

          Suppose you own 9 shirts and 5 pairs of pants.
        %
\leavevmode%
\begin{enumerate}
\item\hypertarget{li-378}{}
              How many outfits can you make?
            %
\item\hypertarget{li-379}{}
              If today is half-naked-day, and you will wear only a shirt or only a pair of pants, how many choices do you have?
            %
\end{enumerate}
\par\medskip\noindent%
\textbf{Solution.}\quad 
          By now you should agree that the answer to the first question is \(9 \cdot 5 = 45\) and the answer to the second question is \(9 + 5 = 14\). These are the multiplicative and additive principles. There are two events: picking a shirt and picking a pair of pants. The first event can happen in 9 ways and the second event can happen in 5 ways. To get both a shirt and a pair of pants, you multiply. To get just one article of clothing, you add.
        %
\par

          Now look at this using sets. There are two sets, call them \(S\) and \(P\). The set \(S\) contains all 9 shirts so \(|S| = 9\) while \(|P| = 5\), since there are 5 elements in the set \(P\) (namely your 5 pairs of pants). What are we asking in terms of these sets? Well in question 2, we really want \(|S \cup P|\), the number of elements in the union of shirts and pants. This is just \(|S| + |P|\) (since there is no overlap; \(|S \cap P| = 0\)). Question 1 is slightly more complicated. Your first guess might be to find \(|S \cap P|\), but this is not right (there is nothing in the intersection). We are not asking for how many clothing items are both a shirt and a pair of pants. Instead, we want one of each. We could think of this as asking how many pairs \((x,y)\) there are, where \(x\) is a shirt and \(y\) is a pair of pants. As we will soon verify, this number is \(|S| \cdot |P|\).
        %
\end{example}
\par

      From this example we can see right away how to rephrase our additive principle in terms of sets:
    %
\begin{assemblage}{Additive Principle (with sets)}\label{assemblage-14}\par\medskip

        \index{additive principle} Given two sets \(A\) and \(B\), if \(A \cap B = \emptyset\) (that is, if there is no element in common to both \(A\) and \(B\)), then
        \begin{equation*}
          \card{A \cup B} = \card{A} + \card{B}.
        \end{equation*}
      %
\end{assemblage}
\par

      This hardly needs a proof. To find \(A \cup B\), you take everything in \(A\) and throw in everything in \(B\). Since there is no element in both sets already, you will have \(\card{A}\) things and add \(\card{B}\) new things to it. This is what adding does! Of course, we can easily extend this to any number of disjoint sets.
    %
\par

      From the example above, we see that in order to investigate the multiplicative principle carefully, we need to consider ordered pairs. We should define this carefully:
    %
\begin{assemblage}{Cartesian Product}\label{assemblage-15}\par\medskip

        Given sets \(A\) and \(B\), we can form the
        \terminology{set} \(A \times B = \{(x,y) \st x \in A \wedge y \in B\}\) to be the set of all ordered pairs \((x,y)\) where \(x\) is an element of \(A\) and \(y\) is an element of \(B\). We call \(A \times B\) the
        \terminology{Cartesian product}\index{Cartesian product} of \(A\) and \(B\).
      %
\end{assemblage}
\begin{example}[]\label{example-29}

          Let \(A = \{1,2\}\) and \(B=\{3,4,5\}\). Find \(A \times B\).
        %
\par\medskip\noindent%
\textbf{Solution.}\quad 
        We want to find ordered pairs \((a,b)\) where \(a\) can be either \(1\) or \(2\) and \(b\) can be either 3, 4, or 5. \(A \times B\) is the set of all of these pairs:
        \begin{equation*}
          A \times B = \{(1,3), (1,4), (1,5), (2,3), (2,4), (2,5)\}
        \end{equation*}\end{example}
\par

      The question is, what is \(\card{A \times B}\)? To figure this out, write out \(A \times B\).
    %
\par

      Let \(A = \{a_1,a_2, a_3, \ldots, a_m\}\) and \(B = \{b_1,b_2, b_3, \ldots, b_n\}\) (so \(\card{A} = m\) and \(\card{B} = n\)). The set \(A \times B\) contains all pairs with the first half of the pair being some \(a_i \in A\) and the second being one of the \(b_j \in B\). In other words:
      \begin{align*}
 A \times B = \{ \amp (a_1, b_1), (a_1, b_2), (a_1, b_3), \ldots (a_1, b_n),\\
 \amp (a_2, b_1), (a_2, b_2), (a_2, b_3), \ldots, (a_2, b_n),\\
 \amp (a_3, b_1), (a_3, b_2), (a_3, b_3), \ldots, (a_3, b_n),\\
 \amp \vdots\\
 \amp (a_m, b_1), (a_m, b_2), (a_m, b_3), \ldots, (a_m, b_n)\}.
\end{align*}
    %
\par

      Notice what we have done here: we made \(m\) rows of \(n\) pairs, for a total of \(m \cdot n\) pairs.
    %
\par

      Each row above is really \(\{a_i\} \times B\) for some \(a_i \in A\). That is, we fixed the \(A\)-element. Broken up this way, we have
      \begin{equation*}
        A \times B = (\{a_1\} \times B) \cup (\{a_2\} \times B) \cup (\{a_3\}\times B) \cup \cdots \cup (\{a_m\} \times B).
      \end{equation*}
    %
\par

      So \(A \times B\) is really the union of \(m\) disjoint sets. Each of those sets has \(n\) elements in them. The total (using the additive principle) is \(n + n + n + \cdots + n = m \cdot n\).
    %
\par

      To summarize:
    %
\begin{assemblage}{Multiplicative Principle (with sets)}\label{assemblage-16}\par\medskip

        \index{multiplicative principle} Given two sets \(A\) and \(B\), we have \(\card{A \times B} = \card{A} \cdot \card{B}\).
      %
\end{assemblage}
\par

      Again, we can easily extend this to any number of sets.
    %
\typeout{************************************************}
\typeout{Subsection 2.1.2 Principle of Inclusion/Exclusion}
\typeout{************************************************}
\subsection[Principle of Inclusion/Exclusion]{Principle of Inclusion/Exclusion}\label{subsec_PIE}
\begin{investigation}[]\label{investigation-6}
 A recent buzz marketing campaign for \emph{Village Inn} surveyed patrons on their pie preferences. People were asked whether they enjoyed (A) Apple, (B) Blueberry or (C) Cherry pie (respondents answered yes or no to each type of pie, and could say yes to more than one type). The following table shows the results of the survey.%
\leavevmode%
\begin{table}
\centering
\begin{tabular}{cccccccc}
\multicolumn{1}{rA}{Pies enjoyed:}&A&B&C&AB&AC&BC&ABC\tabularnewline\hrulethin
\multicolumn{1}{rA}{Number of people:}&20&13&26&9&15&7&5
\end{tabular}
\end{table}
\par

        How many of those asked enjoy at least one of the kinds of pie? Also, explain why the answer is not 95.
      %
\end{investigation}

      \index{principle of inclusion/exclusion}\index{PIE} While we are thinking about sets, consider what happens to the additive principle when the sets are NOT disjoint. Suppose we want to find \(\card{A \cup B}\) and know that \(\card{A} = 10\) and \(\card{B} = 8\). This is not enough information though. We do not know how many of the 8 elements in \(B\) are also elements of \(A\). However, if we also know that \(\card{A \cap B} = 6\), then we can say exactly how many elements are in \(A\), and, of those, how many are in \(B\) and how many are not (6 of the 10 elements are in \(B\), so 4 are in \(A\) but not in \(B\)). We could fill in a Venn diagram \index{Venn diagram} as follows:
    %
\leavevmode%
\begin{figure}
\centering
{
       \begin{tikzpicture}
  \draw[thick] \circleA \circleAlabel \circleB \circleBlabel \twosetbox;
  \draw (0,0) node{6} (-1,0) node{4} (1,0) node{2};
\end{tikzpicture}
}
\end{figure}
\par

      This says there are 6 elements in \(A \cap B\), 4 elements in \(A \setminus B\) and 2 elements in \(B \setminus A\). Now \emph{these} three sets \emph{are} disjoint, so we can use the additive principle to find the number of elements in \(A \cup B\). It is \(6 + 4 + 2 = 12\).
    %
\par

      This will always work, but drawing a Venn diagram is more than we need to do. In fact, it would be nice to relate this problem to the case where \(A\) and \(B\) are disjoint. Is there one rule we can make that works in either case?
    %
\par

      Here is another way to get the answer to the problem above. Start by just adding \(\card{A} + \card{B}\). This is \(10 + 8 = 18\), which would be the answer if \(\card{A \cap B} = 0\). We see that we are off by exactly 6, which just so happens to be \(\card{A \cap B}\). So perhaps we guess,
      \begin{equation*}
        \card{A \cup B} = \card{A} + \card{B} - \card{A \cap B}.
      \end{equation*}
    %
\par

      This works for this one example. Will it always work? Think about what we are doing here. We want to know how many things are either in \(A\) or \(B\) (or both). We can throw in everything in \(A\), and everything in \(B\). This would give \(\card{A} + \card{B}\) many elements. But of course when you actually take the union, you do not repeat elements that are in both. So far we have counted every element in \(A \cap B\) exactly twice: once when we put in the elements from \(A\) and once when we included the elements from \(B\). We correct by subtracting out the number of elements we have counted twice. So we added them in twice, subtracted once, leaving them counted only one time.
    %
\par

      In other words, we have:
    %
\begin{assemblage}{Cardinality of a union (2 sets)}\label{assemblage-17}\par\medskip

        For any finite sets \(A\) and \(B\),
        \begin{equation*}
          \card{A \cup B} = \card{A} + \card{B} - \card{A \cap B}.
        \end{equation*}
      %
\end{assemblage}
\par

      We can do something similar with three sets.
    %
\begin{example}[]\label{example-30}

          An examination in three subjects, Algebra, Biology, and Chemistry, was taken by 41 students. The following table shows how many students failed in each single subject and in their various combinations:
        %
\leavevmode%
\begin{table}
\centering
\begin{tabular}{cccccccc}
\multicolumn{1}{rA}{Subject:}&A&B&C&AB&AC&BC&ABC\tabularnewline\hrulethin
\multicolumn{1}{rA}{Failed:}&12&5&8&2&6&3&1
\end{tabular}
\end{table}
\par

          How many students failed at least one subject?
        %
\par\medskip\noindent%
\textbf{Solution.}\quad 
          The answer is not 37, even though the sum of the numbers above is 37. For example, while 12 students failed Algebra, 2 of those students also failed Biology, 6 also failed Chemestry, and 1 of those failed all three subjects. In fact, that 1 student who failed all three subjects is counted a total of 7 times in the total 37. To clarify things, let us think of the students who failed Algebra as the elements of the set \(A\), and similarly for sets \(B\) and \(C\). The one student who failed all three subjects is the lone element of the set \(A \cap B \cap C\). Thus, in Venn diagrams:
        %
\leavevmode%
\begin{figure}
\centering
{
           \begin{tikzpicture}[scale=0.9]
  \draw[thick] \circleA \circleAlabel \circleB \circleBlabel \circleC \circleClabel \threesetbox;
  \draw (0,-.35) node{1};
\end{tikzpicture}
}
\end{figure}
\par

          Now let's fill in the other intersections. We know \(A\cap B\) contains 2 elements, but 1 element has already been counted. So we should put a 1 in the region where \(A\) and \(B\) intersect (but \(C\) does not). Similarly, we calculate the cardinality of \((A\cap C) \setminus B\), and \((B \cap C) \setminus A\):
        %
\leavevmode%
\begin{figure}
\centering
{
           \begin{tikzpicture}[scale=0.9]
  \draw[thick] \circleA \circleAlabel \circleB \circleBlabel \circleC \circleClabel \threesetbox;
  \draw (0,-.35) node{1} (0,.4) node{1} (-.6,-.65) node{5} (.6,-.65) node{2};
\end{tikzpicture}
}
\end{figure}
\par

          Next, we determine the numbers which should go in the remaining regions, including outside of all three circles. This last number is the number of students who did not fail any subject:
        %
\leavevmode%
\begin{figure}
\centering
{
           \begin{tikzpicture}[scale=0.9]
  \draw[thick] \circleA \circleAlabel \circleB \circleBlabel \circleC \circleClabel \threesetbox;
  \draw (0,-.35) node{1} (0,.4) node{1} (-.6,-.65) node{5} (.6,-.65) node{2};
  \draw (-1,.3) node{5} (1,.3) node{1} (0,-1.5) node{0} (-1.5,-1.75) node{26};
\end{tikzpicture}
}
\end{figure}
\par

          We found 5 goes in the ``\(A\) only'' region because the entire circle for \(A\) needed to have a total of 12, and 7 were already accounted for. Similarly, we calculate the ``\(B\) only'' region to contain only 1 student and the ``\(C\) only'' region to contain no students.
        %
\par

          Thus the number of students who failed at least one class is 15 (the sum of the numbers in each of the eight disjoint regions). The number of students who passed all three classes is 26: the total number of students, 41, less the 15 who failed at least one class.
        %
\par

          Note that we can also answer other questions. For example, now many students failed just Chemistry? None. How many passed Algebra but failed both Biology and Chemistry? This corresponds to the region inside both \(B\) and \(C\) but outside of \(A\), containing 2 students.
        %
\end{example}
\par

      Could we have solved the problem above in an algebraic way? While the additive principle generalizes to any number of sets, when we add a third set here, we must be careful. With two sets, we needed to know the cardinalities of \(A\), \(B\), and \(A \cap B\) in order to find the cardinality of \(A \cup B\). With three sets we need more information. There are more ways the sets can combine. Not surprisingly then, the formula for cardinality of the union of three non-disjoint sets is more complicated:
    %
\begin{assemblage}{Cardinality of a union (3 sets)}\label{assemblage-18}\par\medskip

        For any finite sets \(A\), \(B\), and \(C\),
        \begin{equation*}
          \card{A \cup B \cup C} = \card{A} + \card{B} + \card{C} - \card{A \cap B} - \card{A \cap C} - \card{B \cap C} + \card{A \cap B \cap C}
        \end{equation*}
      %
\end{assemblage}
\par

      To determine how many elements are in at least one of \(A\), \(B\), or \(C\) we add up all the elements in each of those sets. However, when we do that, any element in both \(A\) and \(B\) is counted twice. Also, each element in both \(A\) and \(C\) is counted twice, as are elements in \(B\) and \(C\), so we take each of those out of our sum once. But now what about the elements which are in \(A \cap B \cap C\) (in all three sets)? We added them in three times, but also removed them three times. They have not yet been counted. Thus we add those elements back in at the end.
    %
\par

      Returning to our example above, we have \(\card{A} = 12\), \(\card{B} = 5\), \(\card{C} = 8\). We also have \(\card{A \cap B} = 2\), \(\card{A \cap C} = 6\), \(\card{B \cap C} = 3\), and \(\card{A \cap B \cap C} = 1\). Therefore:
      \begin{equation*}
        \card{A \cup B \cup C} = 12 + 5 + 8 - 2 - 6 - 3 + 1 = 15
      \end{equation*}
    %
\par

      This is what we got when we solved the problem using Venn diagrams.
    %
\par

      This process of adding in, then taking out, then adding back in, and so on is called the \emph{Principle of Inclusion/Exclusion}, or simply PIE. We will return to this counting technique later to solve for more complicated problems (involving more than 3 sets).
    %
\typeout{************************************************}
\typeout{Exercises 2.1.3 Exercises}
\typeout{************************************************}
\subsection[Exercises]{Exercises}\label{exercises-4}
\begin{exerciselist}
\item[1.]\hypertarget{exercise-42}{}
          Your wardrobe consists of 5 shirts, 3 pairs of pants, and 17 bow ties\index{bow ties}. How many different outfits can you make?
        %
\par\smallskip
\item[2.]\hypertarget{exercise-43}{}
          For your college interview, you must wear a tie. You own 3 regular (boring) ties and 5 (cool) bow ties.


          \leavevmode%
\begin{enumerate}[label=(\alph*)]
\item\hypertarget{li-380}{}
              How many choices do you have for your neck-wear?
            %
\item\hypertarget{li-381}{}
                You realize that the interview is for clown college, so you should probably wear both a regular tie and a bow tie. How many choices do you have now?
              %
\item\hypertarget{li-382}{}
                For the rest of your outfit, you have 5 shirts, 4 skirts, 3 pants, and 7 dresses. You want to select either a shirt to wear with a skirt or pants, or just a dress. How many outfits do you have to choose from?
              %
\end{enumerate}

        %
\par\smallskip
\item[3.]\hypertarget{exercise-44}{}
          Your Blu-ray collection consists of 9 comedies and 7 horror movies. Give an example of a question for which the answer is:
        %
\leavevmode%
\begin{enumerate}[label=(\alph*)]
\item\hypertarget{li-386}{}
              16.
            %
\item\hypertarget{li-387}{}
              63.
            %
\end{enumerate}
\par\smallskip
\item[4.]\hypertarget{exercise-45}{}
          Suppose you have sets \(A\) and \(B\) with \(\card{A} = 10\) and \(\card{B} = 15\).
          \leavevmode%
\begin{enumerate}[label=(\alph*)]
\item\hypertarget{li-390}{}What is the largest possible value for \(\card{A \cap B}\)?%
\item\hypertarget{li-391}{} What is the smallest possible value for \(\card{A \cap B}\)?%
\item\hypertarget{li-392}{} What are the possible values for \(\card{A \cup B}\)?%
\end{enumerate}

        %
\par\smallskip
\item[5.]\hypertarget{exercise-46}{}
          If \(\card{A} = 8\) and \(\card{B} = 5\), what is \(\card{A \cup B} + \card{A \cap B}\)?
        %
\par\smallskip
\item[6.]\hypertarget{exercise-47}{}
          A group of college students were asked about their TV watching habits. Of those surveyed, 28 students watch \emph{The Walking Dead}, 19 watch \emph{The Blacklist}, and 24 watch \emph{Game of Thrones}. Additionally, 16 watch \emph{The Walking Dead} and \emph{The Blacklist}, 14 watch \emph{The Walking Dead} and \emph{Game of Thrones}, and 10 watch \emph{The Blacklist} and \emph{Game of Thrones}. There are 8 students who watch all three shows. How many students surveyed watched at least one of the shows?
        %
\par\smallskip
\item[7.]\hypertarget{exercise-48}{}
          \leavevmode%
\begin{enumerate}[label=(\alph*)]
\item\hypertarget{li-396}{}
              Find \(\card{(A \cup C)\setminus B}\) provided \(\card{A} = 50\), \(\card{B} = 45\), \(\card{C} = 40\), \(\card{A\cap B} = 20\), \(\card{A \cap C} = 15\), \(\card{B \cap C} = 23\), and \(\card{A \cap B \cap C} = 12\).
            %
\item\hypertarget{li-397}{}
              Describe a set in terms of \(A\), \(B\), and \(C\) with cardinality 26.
            %
\end{enumerate}

        %
\par\smallskip
\item[8.]\hypertarget{exercise-49}{}
          Consider all 5 letter ``words'' made from the letters \(a\) through \(h\). (Recall, words are just strings of letters, not necessarily actual English words.)
        %
\leavevmode%
\begin{enumerate}[label=(\alph*)]
\item\hypertarget{li-400}{}
              How many of these words are there total?
            %
\item\hypertarget{li-401}{}
              How many of these words contain no repeated letters?
            %
\item\hypertarget{li-402}{}
              How many of these words start with the sub-word ``aha''?
            %
\item\hypertarget{li-403}{}
              How many of these words either start with ``aha'' or end with ``bah'' or both?
            %
\item\hypertarget{li-404}{}
              How many of the words containing no repeats also do not contain the sub-word ``bad''?%
\end{enumerate}
\par\smallskip
\end{exerciselist}
\typeout{************************************************}
\typeout{Section 2.2 Binomial Coefficients}
\typeout{************************************************}
\section[Binomial Coefficients]{Binomial Coefficients}\label{sec_counting-binom}
\begin{investigation}[]\label{investigation-7}

      In Chess, a rook can move only in straight lines (not diagonally). Fill in each square of the chess board below with the number of different shortest paths the rook, in the upper left corner, can take to get to that square. For example, one square is already filled in. There are six different paths from the rook to the square: DDRR (down down right right), DRDR, DRRD, RDDR, RDRD and RRDD.
    %
\leavevmode%
\begin{figure}
\centering
{
\begin{tikzpicture}[scale=.6]
\foreach \row in {0, 2, 4,6}{
  \foreach \col in {0,2,4,6}{
  \draw[fill=gray!30] (\row,\col) rectangle (\row+1, \col+1) rectangle (\row+2, \col+2);
  }
}
\draw[thick] (0,0) rectangle (8,8);
\node at (0.5,7.5) {\Large \symrook};
\node at (2.5,5.5) {$6$};
\end{tikzpicture}
}
\end{figure}
\end{investigation}

    Here are some apparently different discrete objects we can count: subsets, bit strings, lattice paths, and binomial coefficients. We will give an example of each type of counting problem (and say what these things even are). As we will see, these counting problems are surprisingly similar.
  %
\typeout{************************************************}
\typeout{Subsection 2.2.1 Subsets}
\typeout{************************************************}
\subsection[Subsets]{Subsets}\label{subsection-12}

        Subsets should be familiar, otherwise read over
        \hyperref[sec_intro-sets]{Section~\ref{sec_intro-sets}} again. Suppose we look at the set \(A = \{1,2,3,4,5\}\). How many subsets of \(A\) contain exactly 3 elements?
      %
\par

        First, a simpler question. How many subsets of \(A\) are there total? In other words, what is \(|\pow(A)|\) (the cardinality of the power set of \(A\))? Think about how we would build a subset. We need to decide, for each of the elements of \(A\), whether or not to include the element in our subset. So we need to decide ``yes'' or ``no'' for the element 1. And for each choice we make, we need to decide ``yes'' or ``no'' for the element 2. And so on. For each of the 5 elements, we have 2 choices. Therefore the number of subsets is simply \(2\cdot 2\cdot 2 \cdot 2\cdot 2 = 2^5\) (by the multiplicative principle).
      %
\par

        Of those 32 subsets, how many have 3 elements? This is not obvious. Note that we cannot just use the multiplicative principle. Maybe we want to say we have 2 choices (yes/no) for the first element, 2 choices for the second, 2 choices for the third, and then only 1 choice for the other two. But what if we said ``no'' to one of the first three elements? Then we would have two choices for the 4th element. What a mess!
      %
\par

        Another (bad) idea: we need to pick three elements to be in our subset. There are 5 elements to choose from. So there are 5 choices for the first element, and for each of those 4 choices for the second, and then 3 for the third (last) element. The multiplicative principle would say then that there are a total of \(5 \cdot 4 \cdot 3 = 60\) ways to select the 3 element subset. But this cannot be correct (\(60 > 32\) for one thing). One of the outcomes we would get from these choices would be the set \(\{3,2,5\}\), by choosing the element 3 first, then the element 2, then the element 5. Another outcome would be \(\{5,2,3\}\) by choosing the element 5 first, then the element 2, then the element 3. But these are the same set! We can correct this by dividing: for each set of three elements, there are 6 outcomes counted amoung our 60 (since there are 3 choices for which element we list first, 2 for which we list second, and 1 for which we list last). So we expect there to be
        10 3-element subsets of \(A\).
      %
\par

        Is this right? Well, we could list out all 10 of them, being very systematic in doing so, to make sure we don't miss any or list any twice. Or we could try to count how many subsets of \(A\) \emph{don't} have 3 elements in them. How many have no elements? Just 1 (the empty set). How many have 5? Again, just 1. These are the cases in which we say ``no'' to all elements, or ``yes'' to all elements. Okay, what about the subsets which contain a single element? There are 5 of these. We must say ``yes'' to exactly one element, and there are 5 to choose from. This is also the number of subsets containing 4 elements. Those are the ones for which we must say ``no'' to exactly one element.
      %
\par

        So far we have counted 12 of the 32 subsets. We have not yet counted the subsets with cardinality 2 and with cardinality 3. There are a total of 20 subsets left to split up between these two groups. But the number of each must be the same! If we say ``yes'' to exactly two elements, that can be accomplished in exactly the same number of ways as the number of ways we can say ``no'' to exactly two elements. So the number of 2-element subsets is equal to the number of 3-element subsets. Together there are 20 of these subsets, so 10 each.
      %
\leavevmode%
\begin{table}
\centering
\begin{tabular}{lllllll}
\multicolumn{1}{c}{Number of elements:}&\multicolumn{1}{c}{0}&\multicolumn{1}{c}{1}&\multicolumn{1}{c}{2}&\multicolumn{1}{c}{3}&\multicolumn{1}{c}{4}&\multicolumn{1}{c}{5}\tabularnewline\hrulethin
\multicolumn{1}{c}{Number of subsets:}&\multicolumn{1}{c}{1}&\multicolumn{1}{c}{5}&\multicolumn{1}{c}{10}&\multicolumn{1}{c}{10}&\multicolumn{1}{c}{5}&\multicolumn{1}{c}{1}
\end{tabular}
\end{table}
\typeout{************************************************}
\typeout{Subsection 2.2.2 
        Bit Strings
      }
\typeout{************************************************}
\subsection[
        Bit Strings
      ]{
        Bit Strings
      }\label{subsection-13}
 ``Bit'' is short for ``binary digit,'' so a bit string is a string of binary digits. The binary digits are simply the numbers 0 and 1. All of the following are bit strings:
      \begin{equation*}
        1001 \quad 0 \quad 1111 \quad 1010101010
      \end{equation*}
      %
\par

        The number of bits (0's or 1's) in the string is the \emph{length} of the string; the strings above have lengths 4, 1, 4, and 10 respectively. We also can ask how many of the bits are 1's. The number of 1's in a bit string is the \emph{weight} of the string; the weights of the above strings are 2, 0, 4, and 5 respectively.
      %
\begin{assemblage}{Bit Strings}\label{assemblage-19}\par\medskip

          \index{bit string}


          \leavevmode%
\begin{itemize}[label=\textbullet]
\item{}
A \(n\)-bit string is a bit string of length \(n\).  That is, it is a string containing \(n\) symbols, each of which is a bit, either 0 or 1.
%
\item{}
The \emph{weight}\index{weight, of a string} of a bit string is the number of 1's in it.
%
\item{}\(\B^n\)\label{notation-5}
 is the \emph{set} of all \(n\)-bit strings.%
\item{}\(\B^n_k\)\label{notation-6}
 is the set of all \(n\)-bit strings of weight \(k\).%
\end{itemize}

        %
\end{assemblage}
\par

        For example, the elements of the set \(\B^3_2\) are the bit strings 011, 101, and 110. Those are the only strings containing three bits exactly two of which are 1's.
      %
\par

        The counting questions: How many bit strings have length 5? How many of those have weight 3? In other words, we are asking for the cardinalities \(|\B^5|\) and \(|\B^5_3|\).
      %
\par

        To find the number of 5-bit strings is straight forward. We have 5 bits, and each can either be a 0 or a 1. So there are 2 choices for the first bit, 2 choices for the second, and so on. By the multiplicative principle, there are \(2 \cdot 2 \cdot 2\cdot 2 \cdot 2 = 2^5 = 32\) such strings.
      %
\par

        Finding the number of 5-bit strings of weight 3 is harder. Think about how such a string could start. The first bit must be either a 0 or a 1. In the first case (the string starts with a 0), we must then decide on four more bits. To have a total of three 1's, among those four remaining bits there must be three 1's. To count all of these strings, we must include all 4-bit strings of weight 3. In the second case (the string starts with a 1), we still have four bits to choose, but now only two of them can be 1's, so we should look at all the 4-bit strings of weight 2. So the strings in \(\B^5_3\) all have the form \(1\B^4_2\) (that is, a 1 followed by a string from \(\B^4_2\)) or \(0\B^4_3\). These two sets are disjoint, so we can use the additive principle:
        \begin{equation*}
          |\B^5_3| = |\B^4_2| + |\B^4_3|.
        \end{equation*}
      %
\par

        This is an example of a \emph{recurrence relation}. We represented one instance of our counting problem in terms of two simpler instances of the problem. If only we knew the cardinalities of \(\B^4_2\) and \(\B^4_3\). Repeating the same reasoning,
        \begin{equation*}
          |\B^4_2| = |\B^3_1| + |\B^3_2| \quad \mbox{and} \quad |\B^4_3| = |\B^3_2| + |\B^3_3|.
        \end{equation*}
      %
\par

        We can keep going down, but this should be good enough. Both \(\B^3_1\) and \(\B^3_2\) contain 3 bit strings: we must pick one of the three bits to be a 1 (three ways to do that) or one of the three bits to be a 0 (three ways to do that). Also, \(\B^3_3\) contains just one string: 111. Thus \(|\B^4_2| = 6\) and \(|\B^4_3| = 4\), which puts \(\B^5_3\) at a total of 10 strings.
      %
\par

        But wait
        \textemdash{}32 and 10 were the answers to the counting questions about subsets. Coincidence? Not at all. Each bit string can be thought of as a \emph{code} for a subset. For the set \(A = \{1,2,3,4,5\}\), we would use 5-bit strings, one bit for each element of \(A\). Each bit in the string is a 0 if its corresponding element of \(A\) is not in the subset, and a 1 if the element of \(A\) is in the subset. Remember, deciding the subset amounted to a sequence of five yes/no votes for the elements of \(A\). Instead of yes, we put a 1; instead of no, we put a 0.
      %
\par

        For example, the bit string \(11001\) represents the subset \(\{1,2,5\}\) since the first, second and fifth bits are 1's. The subset \(\{3,5\}\) would be coded by the string \(00101\). What we really have here is a bijection from \(\pow(A)\) to \(\B^5\).
      %
\par

        Now for a subset to contain exactly three elements, the corresponding bit string must contain exactly three 1's. In other words, the weight must be 3. Thus counting the number of 3-element subsets of \(A\) is the same as counting the number 5-bit strings of weight 3.
      %
\typeout{************************************************}
\typeout{Subsection 2.2.3  Lattice Paths}
\typeout{************************************************}
\subsection[ Lattice Paths]{ Lattice Paths}\label{subsection-14}

        The \emph{integer lattice} is the set of all points in the Cartesian plane for which both the \(x\) and \(y\) coordinates are integers. If you like to draw graphs on graph paper, the lattice is the set of all the intersections of the grid lines.
      %
\par

        A \emph{lattice path}\index{lattice path} is one of the shortest possible paths connecting two points on the lattice, moving only horizontally and vertically. For example, here are three possible lattice paths from the points \((0,0)\) to \((3,2)\):
      %
% group protects changes to lengths, releases boxes (?)
{% begin: group for a single side-by-side
% set panel max height to practical minimum, created in preamble
\setlength{\panelmax}{0pt}
\newsavebox{\panelboxAGimage}
\savebox{\panelboxAGimage}{
\resizebox{0.3\linewidth}{!}{{
      \begin{tikzpicture}
  \draw[very thin, color=gray!50] (-.5,-.5) grid (3.5, 2.5);
  \foreach \x in {0,...,3}
  \foreach \y in {0,...,2}
  \fill (\x,\y) circle (1.5pt);
  \draw (0,0) node[below left] { (0,0)} (3,2) node[above right] { (3,2)};
  \draw[very thick] (0,0) -- (2,0) -- (2,2) -- (3,2);
\end{tikzpicture}
}
}}
\newlength{\phAGimage}\setlength{\phAGimage}{\ht\panelboxAGimage+\dp\panelboxAGimage}
\settototalheight{\phAGimage}{\usebox{\panelboxAGimage}}
\setlength{\panelmax}{\maxof{\panelmax}{\phAGimage}}
\newsavebox{\panelboxAHimage}
\savebox{\panelboxAHimage}{
\resizebox{0.3\linewidth}{!}{{
      \begin{tikzpicture}
  \draw[very thin, color=gray!50] (-.5,-.5) grid (3.5, 2.5);
  \foreach \x in {0,...,3}
  \foreach \y in {0,...,2}
  \fill (\x,\y) circle (1.5pt);
  \draw (0,0) node[below left] { (0,0)} (3,2) node[above right] { (3,2)};
  \draw[very thick] (0,0) -- (0,2) -- (3,2);
\end{tikzpicture}
}
}}
\newlength{\phAHimage}\setlength{\phAHimage}{\ht\panelboxAHimage+\dp\panelboxAHimage}
\settototalheight{\phAHimage}{\usebox{\panelboxAHimage}}
\setlength{\panelmax}{\maxof{\panelmax}{\phAHimage}}
\newsavebox{\panelboxAIimage}
\savebox{\panelboxAIimage}{
\resizebox{0.3\linewidth}{!}{{
      \begin{tikzpicture}
  \draw[very thin, color=gray!50] (-.5,-.5) grid (3.5, 2.5);
  \foreach \x in {0,...,3}
  \foreach \y in {0,...,2}
  \fill (\x,\y) circle (1.5pt);
  \draw (0,0) node[below left] { (0,0)} (3,2) node[above right] { (3,2)};
  \draw[very thick] (0,0) -- (1,0) -- (1,1) -- (3,1) -- (3,2);
\end{tikzpicture}
}
}}
\newlength{\phAIimage}\setlength{\phAIimage}{\ht\panelboxAIimage+\dp\panelboxAIimage}
\settototalheight{\phAIimage}{\usebox{\panelboxAIimage}}
\setlength{\panelmax}{\maxof{\panelmax}{\phAIimage}}
\leavevmode%
% begin: side-by-side as figure/tabular
% \tabcolsep change local to group
\setlength{\tabcolsep}{0.0166666666666667\textwidth}
% @{} suppress \tabcolsep at extremes, so margins behave as intended
\begin{figure}
\hspace*{0.0166666666666667\textwidth}%
\begin{tabular}{@{}*{3}{c}@{}}
\begin{minipage}[c][\panelmax][t]{0.3\textwidth}\usebox{\panelboxAGimage}\end{minipage}&
\begin{minipage}[c][\panelmax][t]{0.3\textwidth}\usebox{\panelboxAHimage}\end{minipage}&
\begin{minipage}[c][\panelmax][t]{0.3\textwidth}\usebox{\panelboxAIimage}\end{minipage}\end{tabular}
\end{figure}
% end: side-by-side as tabular/figure
}% end: group for a single side-by-side
\par

        Notice to ensure the path is the \emph{shortest} possible, each move must be either to the right or up. Additionally, in this case, note that no matter what path we take, we must make three steps right and two steps up. No matter what order we make these steps, there will always be 5 steps. Thus each path has \terminology{length} 5.
      %
\par

        The counting question: how many lattice paths are there between \((0,0)\) and \((3,2)\)? We could try to draw all of these, or instead of drawing them, maybe just list which direction we travel on each of the 5 steps. One path might be RRUUR, or maybe UURRR, or perhaps RURRU (those correspond to the three paths drawn above). So how many such strings of R's and U's are there?
      %
\par

        Notice that each of these strings must contain 5 symbols. Exactly 3 of them must be R's (since our destination is 3 units to the right). This seems awfully familiar. In fact, what if we used \(1\)'s instead of R's and 0's instead of U's? Then we would just have 5-bit strings of weight 3. There are 10 of those, so there are 10 lattice paths from (0,0) to (3,2).
      %
\par

        The correspondence between bit strings and lattice paths does not stop there. Here is another way to count lattice paths. Consider the lattice shown below:
      %
\leavevmode%
\begin{figure}
\centering
{
      \begin{tikzpicture}
  \draw[very thin, color=gray!50] (-.5,-.5) grid (3.5, 2.5);
  \foreach \x in {0,...,3}
  \foreach \y in {0,...,2}
  \fill (\x,\y) circle (1.5pt);
  \draw (0,0) node[below left] { (0,0)} (3,2) node[above right] { (3,2)};
  \draw (3,1) node[above right] { \(B\)} (2,2) node[above right]{ \(A\)};
\end{tikzpicture}
}
\end{figure}
\par

        Any lattice path from (0,0) to (3,2) must pass through exactly one of \(A\) and \(B\). The point \(A\) is 4 steps away from (0,0) and two of them are towards the right. The number of lattice paths to \(A\) is the same as the number of 4-bit strings of weight 2, namely 6. The point \(B\) is 4 steps away from (0,0), but now 3 of them are towards the right. So the number of paths to point \(B\) is the same as the number of 4-bit strings of weight 3, namely 4. So the total number of paths to (3,2) is just \(6+4\). This is the same way we calculated the number of 5-bit strings of weight 3. The point: the exact same recurrence relation exists for bit strings and for lattice paths.
      %
\typeout{************************************************}
\typeout{Subsection 2.2.4 Binomial Coefficients}
\typeout{************************************************}
\subsection[Binomial Coefficients]{Binomial Coefficients}\label{subsection-15}

        Binomial coefficients are the coefficients in the expanded version of a binomial, such as \((x+y)^5\). What happens when we multiply such a binomial out? We will expand \((x+y)^n\) for various values of \(n\). Each of these are done by multiplying everything out (i.e., FOIL-ing) and then collecting like terms.
        \begin{equation*}
          (x+y)^1 = x + y
        \end{equation*}
      %
\begin{equation*}
        (x+y)^2 = x^2 + 2xy + y^2
      \end{equation*}\begin{equation*}
        (x+y)^3 = x^3 + 3x^2y + 3xy^2 + y^3
      \end{equation*}\begin{equation*}
        (x+y)^4 = x^4 + 4x^3y + 6x^2y^2 + 4xy^3 + y^4.
      \end{equation*}\par

        In fact, there is a quicker way to expand the above binomials. For example, consider the next one, \((x+y)^5\). What we are really doing is multiplying out,
        \begin{equation*}
          (x+y)(x+y)(x+y)(x+y)(x+y).
        \end{equation*}
      %
\par
If that looks daunting, go back to the case of \((x+y)^3 = (x+y)(x+y)(x+y)\). Why do we only have one \(x^3\) and \(y^3\) but three \(x^2y\) and \(xy^2\) terms? Every time we distribute over an \((x+y)\) we create two copies of what is left, one multiplied by \(x\), the other multiplied by \(y\). To get \(x^3\), we need to pick the ``multiplied by \(x\)'' side every time (we don't have any \(y\)'s in the term). This will only happen once. On the other hand, to get \(x^2y\) we need to select the \(x\) side twice and the \(y\) side once. In other words, we need to pick one of the three \((x+y)\) terms to ``contribute'' their \(y\).
      %
\par

        Similarly, in the expansion of \((x+y)^5\), there will be only one \(x^5\) term and one \(y^5\) term. This is because to get an \(x^5\), we need to use the \(x\) term in each of the copies of the binomial \((x+y)\), and similarly for \(y^5\). What about \(x^4y\)? To get terms like this, we need to use four \(x\)'s and one \(y\), so we need exactly one of the five binomials to contribute a \(y\). There are 5 choices for this, so there are 5 ways to get \(x^4y\), so the coefficient of \(x^4y\) is 5. This is also the coefficient for \(xy^4\) for the same (but opposite) reason: there are 5 ways to pick which of the 5 binomials contribute the single \(x\). So far we have
        \begin{equation*}
          (x+y)^5 = x^5 + 5x^4y + \underline{~?~}~x^3y^2 + \underline{~?~}~x^2y^3 + 5 xy^4 + y^5.
        \end{equation*}
      %
\par

        We still need the coefficients of \(x^3y^2\) and \(x^2y^3\). In both cases, we need to pick exactly 3 of the 5 binomials to contribute one variable, the other two to contribute the other. Wait. This sounds familiar. We have 5 things, each can be one of two things, and we need a total of 3 of one of them. That's just like taking 5 bits and making sure exactly 3 of them are 1's. So the coefficient of \(x^3y^2\) (and also \(x^2y^3\)) will be exactly the same as the number of bit strings of length 5 and weight 3, which we found earlier to be 10. So we have:
        \begin{equation*}
          (x+y)^5 = x^5 + 5x^4y + 10x^3y^2 + 10x^2y^3 + 5 xy^4 + y^5.
        \end{equation*}
      %
\par

        These numbers we keep seeing over and over again. They are the number of subsets of a particular size, the number of bit strings of a particular weight, the number of lattice paths, and the coefficients of these binomial products. We will call them \emph{binomial coefficients}. We even have a special symbol for them: \({n \choose k}\).
      %
\begin{assemblage}{Binomial Coefficients}\label{assemblage-20}\par\medskip

          \index{binomial coefficients} For each integer \(n \ge 0\) and integer \(k\) with \(0 \le k \le n\) there is a number
          \begin{equation*}
            {n\choose k}
          \end{equation*}
          read ``\(n\) choose \(k\).'' We have:


          \leavevmode%
\begin{itemize}[label=\textbullet]
\item{}\({n\choose k} = |\B^n_k|\), the number of \(n\)-bit strings of weight \(k\).%
\item{}\({n \choose k}\) is the number of subsets of a set of size \(n\) each with cardinality \(k\).%
\item{}\({n \choose k}\) is the number of lattice paths of length \(n\) containing \(k\) steps to the right.%
\item{}\({n \choose k}\) is the coefficient of \(x^ky^{n-k}\) in the expansion of \((x+y)^n\).%
\item{}\({n \choose k}\) is the number of ways to select \(k\) objects from a total of \(n\) objects.%
\end{itemize}

        %
\end{assemblage}
\par

        The last bullet point is usually taken as the definition of \({n \choose k}\). Out of \(n\) objects we must choose \(k\) of them, so there are \(n\) choose \(k\) ways of doing this. Each of our counting problems above can be viewed in this way:
      %
\leavevmode%
\begin{itemize}[label=\textbullet]
\item{}
How many subsets of \(\{1,2,3,4,5\}\) contain exactly 3 elements?  We must choose \(3\) of the 5 elements to be in our subset.  There are \({5 \choose 3}\) ways to do this, so there are \({5 \choose 3}\) such subsets.
%
\item{}
How many bit strings have length 5 and weight 3?  We must choose \(3\) of the 5 bits to be 1's.  There are \({5 \choose 3}\) ways to do this, so there are \({5 \choose 3}\) such bit strings.
%
\item{}
How many lattice paths are there from (0,0) to (3,2)?  We must choose 3 of the 5 steps to be towards the right.  There are \({5 \choose 3}\) ways to do this, so there are \({5 \choose 3}\) such lattice paths.
%
\item{}
What is the coefficient of \(x^3y^2\) in the expansion of \((x+y)^5\)?  We must choose 3 of the 5 copies of the binomial to contribute an \(x\).  There are \({5 \choose 3}\) ways to do this, so the coefficient is \({5 \choose 3}\).
%
\end{itemize}
\par

        It should be clear that in each case above, we have the right answer. All we had to do is phrase the question correctly and it became obvious that \({5 \choose 3}\) is correct. However, this does not tell us that the answer is in fact 10 in each case. We will eventually find a formula for \({n \choose k}\), but for now, look back at how we arrived at the answer 10 in our counting problems above. It all came down to bit strings, and we have a recurrence relation for bit strings:
        \begin{equation*}
          |\B^n_k| = |\B^{n-1}_{k-1}| + |\B^{n-1}_k|.
        \end{equation*}
      %
\par

        Remember, this is because we can start the bit string with either a 1 or a 0. In both cases, we have \(n-1\) more bits to pick. The strings starting with 1 must contain \(k-1\) more 1's, while the strings starting with 0 still need \(k\) more 1's.
      %
\par

        Since \(|\B^n_k| = {n \choose k}\), the same recurrence relation holds for binomial coefficients:
      %
\begin{assemblage}{Recurrence relation for \({n \choose k}\)}\label{assemblage-21}\par\medskip

          \begin{equation*}
            {n \choose k} = {n-1 \choose k-1} + {n-1 \choose k}
          \end{equation*}
        %
\end{assemblage}
\typeout{************************************************}
\typeout{Subsection 2.2.5 Pascal's Triangle}
\typeout{************************************************}
\subsection[Pascal's Triangle]{Pascal's Triangle}\label{subsec_Pascal}

        Let's arrange the binomial coefficients \({n \choose k}\) into a triangle like follows:
      %
\leavevmode%
\begin{figure}
\centering
{
        \begin{tikzpicture}
  \foreach \n in {0,...,4}
  \foreach \k in {0,...,\n}
  \draw (-\n+2*\k, -\n) node {\(\displaystyle{\n \choose \k}\)};
\end{tikzpicture}
}
\end{figure}
\par

        This can continue as far down as we like. The recurrence relation for \({n \choose k}\) tells us that each entry in the triangle is the sum of the two entries above it. The entries on the sides of the triangle are always 1. This is because \({n \choose 0} = 1\) for all \(n\) since there is only one way to pick 0 of \(n\) objects and \({n \choose n} = 1\) since there is one way to select all \(n\) out of \(n\) objects. Using the recurrence relation, and the fact that the sides of the triangle are 1's, we can easily replace all the entries above with the correct values of \({n \choose k}\). Doing so gives us \emph{Pascal's triangle}.
      %
\par

        We can use Pascal's triangle to calculate binomial coefficients. For example, using the triangle on the next page, we can find \({12 \choose 6} = 924\).
      %
\index{Pascal's triangle}% group protects changes to lengths, releases boxes (?)
{% begin: group for a single side-by-side
% set panel max height to practical minimum, created in preamble
\setlength{\panelmax}{0pt}
\newsavebox{\panelboxALimage}
\savebox{\panelboxALimage}{
\resizebox{1\linewidth}{!}{{
  \begin{tikzpicture}
\def\r{.55}
\foreach \row in {0,...,16} {
  \hexbox{\row}{0}{\large 1}
}
%fill in the rest of the triangle:
\foreach \row in {1,...,16} {
  \pgfmathsetmacro{\entry}{1};
  \foreach \col in {1,...,\row} {
    % iterative formula : val = precval * (row-col+1)/col
    % (+ 0.5 to bypass rounding errors)
    \pgfmathtruncatemacro{\entry}{\entry*((\row-\col+1)/\col)+0.5};
    \global\let\entry=\entry
    \ifnum \entry<100
\hexbox{\row}{\col}{\large \entry}
    \else \ifnum \entry<1000
\hexbox{\row}{\col}{\entry}
    \else \ifnum \entry<10000
\hexbox{\row}{\col}{\footnotesize \entry}
\else
\hexbox{\row}{\col}{\scriptsize \entry}
\fi
    \fi
    \fi
  }
}
\node[above] at (0,3) {\Huge Pascal's Triangle};
\end{tikzpicture}
}
}}
\newlength{\phALimage}\setlength{\phALimage}{\ht\panelboxALimage+\dp\panelboxALimage}
\settototalheight{\phALimage}{\usebox{\panelboxALimage}}
\setlength{\panelmax}{\maxof{\panelmax}{\phALimage}}
\leavevmode%
% begin: side-by-side as figure/tabular
% \tabcolsep change local to group
\setlength{\tabcolsep}{0\textwidth}
% @{} suppress \tabcolsep at extremes, so margins behave as intended
\begin{figure}
\begin{tabular}{@{}*{1}{c}@{}}
\begin{minipage}[c][\panelmax][t]{1\textwidth}\usebox{\panelboxALimage}\end{minipage}\end{tabular}
\end{figure}
% end: side-by-side as tabular/figure
}% end: group for a single side-by-side
\typeout{************************************************}
\typeout{Exercises 2.2.6 Exercises}
\typeout{************************************************}
\subsection[Exercises]{Exercises}\label{exercises-5}
\begin{exerciselist}
\item[1.]\hypertarget{exercise-50}{}
            Let \(S = \{1, 2, 3, 4, 5, 6\}\)
          %
\leavevmode%
\begin{enumerate}[label=(\alph*)]
\item\hypertarget{li-423}{} How many subsets are there total? %
\item\hypertarget{li-424}{} How many subsets have \(\{2,3,5\}\) as a subset? %
\item\hypertarget{li-425}{} How many subsets contain at least one odd number? %
\item\hypertarget{li-426}{} How many subsets contain exactly one even number? %
\end{enumerate}
\par\smallskip
\item[2.]\hypertarget{exercise-51}{}
            Let \(S = \{1, 2, 3, 4, 5, 6\}\)
          %
\leavevmode%
\begin{enumerate}[label=(\alph*)]
\item\hypertarget{li-431}{} How many subsets are there of cardinality 4? %
\item\hypertarget{li-432}{} How many subsets of cardinality 4 have \(\{2,3,5\}\) as a subset? %
\item\hypertarget{li-433}{} How many subsets of cardinality 4 contain at least one odd number? %
\item\hypertarget{li-434}{} How many subsets of cardinality 4 contain exactly one even number? %
\end{enumerate}
\par\smallskip
\item[3.]\hypertarget{exercise-52}{}
            You break your piggy-bank to discover lots of pennies and nickels. You start arranging these in rows of 6 coins.
          %
\leavevmode%
\begin{enumerate}[label=(\alph*)]
\item\hypertarget{li-439}{} You find yourself making rows containing an equal number of pennies and nickels.  For fun, you decide to lay out every possible such row.  How many coins will you need? %
\item\hypertarget{li-440}{} How many coins would you need to make all possible rows of 6 coins (not necessarily with equal number of pennies and nickels)? %
\end{enumerate}
\par\smallskip
\item[4.]\hypertarget{exercise-53}{}
            How many 10-bit strings contain 6 or more 1's?
          %
\par\smallskip
\item[5.]\hypertarget{exercise-54}{}
            How many subsets of \(\{0,1,\ldots, 9\}\) have cardinality 6 or more?
          %
\par\smallskip
\item[6.]\hypertarget{exercise-55}{}
            What is the coefficient of \(x^{12}\) in \((x+2)^{15}\)?
          %
\par\smallskip
\item[7.]\hypertarget{exercise-56}{}
            What is the coefficient of \(x^9\) in the expansion of \((x+1)^{14} + x^3(x+2)^{15}\)?
          %
\par\smallskip
\item[8.]\hypertarget{exercise-57}{}
            How many shortest lattice paths start at (3,3) and
          %
\leavevmode%
\begin{enumerate}[label=(\alph*)]
\item\hypertarget{li-443}{} end at (10,10)? %
\item\hypertarget{li-444}{} end at (10,10) and pass through (5,7)? %
\item\hypertarget{li-445}{} end at (10,10) and avoid (5,7)? %
\end{enumerate}
\par\smallskip
\item[9.]\hypertarget{exercise-58}{}
            Suppose you are ordering a large pizza from \emph{D.P.~Dough}. You want 3 distinct toppings, chosen from their list of 11 vegetarian toppings.
          %
\leavevmode%
\begin{enumerate}[label=(\alph*)]
\item\hypertarget{li-449}{} How many choices do you have for your pizza? %
\item\hypertarget{li-450}{} How many choices do you have for your pizza if you refuse to have pineapple as one of your toppings? %
\item\hypertarget{li-451}{} How many choices do you have for your pizza if you \emph{insist} on having pineapple as one of your toppings? %
\item\hypertarget{li-452}{} How do the three questions above relate to each other? %
\end{enumerate}
\par\smallskip
\item[10.]\hypertarget{exercise-59}{}
            Explain why the coefficient of \(x^5y^3\) the same as the coefficient of \(x^3y^5\) in the expansion of \((x+y)^8\)?
          %
\par\smallskip
\end{exerciselist}
\typeout{************************************************}
\typeout{Section 2.3 Combinations and Permutations}
\typeout{************************************************}
\section[Combinations and Permutations]{Combinations and Permutations}\label{sec_counting-combperm}
\begin{investigation}[]\label{investigation-8}

      You have a bunch of chips which come in five different colors: red, blue, green, purple and yellow.

      \leavevmode%
\begin{enumerate}
\item\hypertarget{li-457}{}
            How many different two-chip stacks can you make if the bottom chip must be red or blue? Explain your answer using both the additive and multiplicative principles.
          %
\item\hypertarget{li-458}{}
            How many different three-chip stacks can you make if the bottom chip must be red or blue and the top chip must be green, purple or yellow? How does this problem relate to the previous one?
          %
\item\hypertarget{li-459}{}
            How many different three-chip stacks are there in which no color is repeated? What about four-chip stacks?
          %
\item\hypertarget{li-460}{}
            Suppose you wanted to take three different colored chips and put them in your pocket. How many different choices do you have? What if you wanted four different colored chips? How do these problems relate to the previous one?
          %
\end{enumerate}

    %
\end{investigation}

    A \terminology{permutation} is a (possible) rearrangement of objects. For example, there are 6 permutations of the letters \emph{a, b, c}:
    \begin{equation*}
      abc, ~~ acb, ~~ bac, ~~bca, ~~ cab, ~~ cba.
    \end{equation*}
  %
\par

    We know that we have them all listed above
    \textemdash{}there are 3 choices for which letter we put first, then 2 choices for which letter comes next, which leaves only 1 choice for the last letter. The multiplicative principle says we multiply \(3\cdot 2 \cdot 1\).
  %
\begin{example}[]\label{example-31}

        How many permutations are there of the letters \emph{a, b, c, d, e, f}?
      %
\par\medskip\noindent%
\textbf{Solution.}\quad 
        We do NOT want to try to list all of these out. However, if we did, we would need to pick a letter to write down first. There are 6 choices for that letter. For each choice of first letter, there are 5 choices for the second letter (we cannot repeat the first letter; we are rearranging letters and only have one of each), and for each of those, there are 4 choices for the third, 3 choices for the fourth, 2 choices for the fifth and finally only 1 choice for the last letter. So there are \(6 \cdot 5 \cdot 4 \cdot 3 \cdot 2 \cdot 1 = 720\) permutations of the 6 letters.
      %
\end{example}
\par
 A piece of notation is helpful here: \(n!\), read ``\(n\) factorial'',\index{factorial} is the product of all positive integers less than or equal to \(n\) (for reasons of convenience, we also define 0! to be 1). So the number of permutation of 6 letters, as seen in the previous example is \(6! = 6\cdot 5 \cdot 4 \cdot 3 \cdot 2 \cdot 1\). This generalizes:%
\begin{assemblage}{Permutations of \(n\) elements}\label{assemblage-22}\par\medskip

        There are \(n! = n\cdot (n-1)\cdot (n-2)\cdot \cdots \cdot 2\cdot 1\) permutations of \(n\) (distinct) elements.
      %
\end{assemblage}
\begin{example}[Counting Bijective Functions]\label{ex_counting-functions-bijective}

          How many functions \(f:\{1,2,\ldots,8\} \to \{1,2,\ldots, 8\}\) are \emph{bijective}?
        %
\par\medskip\noindent%
\textbf{Solution.}\quad 
          Remember what it means for a function to be bijective: each element in the codomain must be the image of exactly one element of the domain. Using two line notation, we could write one of these bijections as
          \begin{equation*}
            f = \twoline{1 \amp 2 \amp 3 \amp 4 \amp 5 \amp 6 \amp 7 \amp 8} {3 \amp 1 \amp 5 \amp 8 \amp 7 \amp 6 \amp 2 \amp 4}
          \end{equation*}
          What we are really doing is just rearranging the elements of the codomain, so we are creating a permutation of 8 elements. In fact, ``permutation'' is another term used to describe bijective functions from a finite set to itself.
        %
\par

          If you believe this, then you see the answer must be \(8! = 8 \cdot 7 \cdot\cdots\cdot 1 = 40320\). You can see this directly as well: for each element of the domain, we must pick a distinct element of the codomain to map to. There are 8 choices for where to send 1, then 7 choices for where to send 2, and so on. We multiply using the multiplicative principle.
        %
\end{example}
\par

        Sometimes we do not want to permute all of the letters/numbers/elements we are givin.
      %
\begin{example}[]\label{example-33}

            How many 4 letter ``words'' can you make from the letters \emph{a} through \emph{f}, with no repeated letters?
          %
\par\medskip\noindent%
\textbf{Solution.}\quad 
            This is just like the problem of permuting 4 letters, only now we have more choices for each letter. For the first letter, there are 6 choices. For each of those, there are 5 choices for the second letter. Then there are 4 choices for the third letter, and 3 choices for the last letter. The total number of words is \(6\cdot 5\cdot 4 \cdot 3 = 360\). This is not \(6!\) because we never multiplied by 2 and 1. We could start with \(6!\) and then cancel the 2 and 1, and thus write \(\frac{6!}{2!}\).
          %
\end{example}
\par

        In general, we can ask how many permutations exist of \(k\) objects choosing those objects from a larger collection of \(n\) objects. (In the example above, \(k = 4\), and \(n = 6\).) We write this number \(P(n,k)\) and sometimes call it a \terminology{\(k\)-permutation of \(n\) elements}. From the example above, we see that to compute \(P(n,k)\) we must apply the multiplicative principle to \(k\) numbers, starting with \(n\) and counting backwards. So for example
        \begin{equation*}
          P(10, 4) = 10\cdot 9 \cdot 8 \cdot 7.
        \end{equation*}
      %
\par

        Notice again that \(P(10,4)\) starts out looking like \(10!\), but we stop after 7. We can formally account for this ``stopping'' by dividing away the part of the factorial we do not want:
        \begin{equation*}
          P(10,4) = \frac{10\cdot 9 \cdot 8 \cdot 7 \cdot 6 \cdot 5 \cdot 4 \cdot 3 \cdot 2 \cdot 1}{6 \cdot 5 \cdot 4 \cdot 3 \cdot 2 \cdot 1} = \frac{10!}{6!}.
        \end{equation*}
      %
\par

        Careful: The factorial in the denominator is not \(4!\) but rather \((10-4)!\).
      %
\begin{assemblage}{\(k\)-permutations of \(n\) elements}\label{assemblage-23}\par\medskip

          \index{\(k\)-permutation}\index{permutation}
          \(P(n,k)\) is the number of \terminology{\(k\)-permutations} of \(n\) elements, the number of ways to \emph{arrange} \(k\) objects chosen from \(n\) distinct objects.
          \begin{equation*}
            P(n,k) = \frac{n!}{(n-k)!}.
          \end{equation*}
        %
\end{assemblage}
\par

        Note that when \(n = k\), we have \(P(n,n) = \frac{n!}{(n-n)!} = n!\) (since we defined \(0!\) to be 1). This makes sense
        \textemdash{}we already know \(n!\) gives the number of permutations of all \(n\) objects.
      %
\begin{example}[Counting injective functions]\label{ex_counting-functions-injective}
How many functions \(f:\{1,2,3\} \to \{1,2,3,4,5,6,7\}\) are \emph{injective}?%
\par\medskip\noindent%
\textbf{Solution.}\quad 
            Note that it doesn't make sense to ask for the number of \emph{bijections} here, as there are none (because the codomain is larger than the domain, there are no surjections). But for a function to be injective, we just can't use an element of the codomain more than once.
          %
\par

            We need to pick an element from the codomain to be the image of 1. There are 8 choices. Then we need to pick one of the remaining 7 elements to be the image of 2. Finally, one of the remaining 6 elements must be the image of 3. So the total number of functions is \(8\cdot 7 \cdot 6 = P(8,3)\).
          %
\par

            What this demonstrates in general is that the number of injections \(f:A \to B\), where \(\card{A} = k\) and \(\card{B} = n\), is \(P(n,k)\).
          %
\end{example}
\par

          Here is another way to find the number of \(k\)-permutations of \(n\) elements: first select which \(k\) elements will be in the permutation, then count how many ways there are to arrange them. Once you have selected the \(k\) objects, we know there are \(k!\) ways to arrange (permute) them. But how do you select \(k\) objects from the \(n\)? You have \(n\) objects, and you need to \emph{choose} \(k\) of them. You can do that in \({n \choose k}\) ways. Then for each choice of those \(k\) elements, we can permute \emph{them} in \(k!\) ways. Using the multiplicative principle, we get another formula for \(P(n,k)\):
          \begin{equation*}
            P(n,k) = {n \choose k}\cdot k!.
          \end{equation*}
        %
\par

          Now since we have a closed formula for \(P(n,k)\) already, we can substitute that in:
          \begin{equation*}
            \frac{n!}{(n-k)!} = {n \choose k} \cdot k!.
          \end{equation*}
        %
\par

          If we divide both sides by \(k!\) we get a closed formula for \({n \choose k}\).
        %
\begin{assemblage}{Closed formula for \({n \choose k}\)}\label{assemblage-24}\par\medskip

            \begin{equation*}
              {n \choose k} = \frac{n!}{(n-k)!k!}
            \end{equation*}
          %
\end{assemblage}
\par

          We say \(P(n,k)\) counts permutations, and \({n \choose k}\) counts \emph{combinations}\index{combination}. The formulas for each are very similar, there is just an extra \(k!\) in the denominator of \({n \choose k}\). That extra \(k!\) accounts for the fact that \({n \choose k}\) does not distinguish between the different orders that the \(k\) objects can appear in. We are just selecting (or choosing) the \(k\) objects, not arranging them. Perhaps ``combination'' is a misleading label. We don't mean it like a combination lock (where the order would definitely matter). Perhaps a better metaphor is a combination of flavors
          \textemdash{} you just need to decide which flavors to combine, not the order in which to combine them.
        %
\par

          To further illustrate the connection between combinations and permutations, we close with an example.
        %
\begin{example}[]\label{example-35}

              You decide to have a dinner party. Even though you are incredibly popular and have 14 different friends, you only have enough chairs to invite 6 of them.
            %
\leavevmode%
\begin{enumerate}
\item\hypertarget{li-461}{}
How many choices do you have for which 6 friends to invite?
%
\item\hypertarget{li-462}{}
What if you need to decide not only which friends to invite but also where to seat them along your long table?  How many choices do you have then?
%
\end{enumerate}
\par\medskip\noindent%
\textbf{Solution.}\quad \leavevmode%
\begin{enumerate}
\item\hypertarget{li-463}{}
You must simply choose 6 friends from a group of 14.  This can be done in \({14 \choose 6}\) ways.  We can find this number either by using Pascal's triangle or the closed formula: \(\frac{14!}{8!\cdot 6!} = 3003\).
%
\item\hypertarget{li-464}{}
Here you must count all the ways you can permute 6 friends chosen from a group of 14.  So the answer is \(P(14, 6)\), which can be calculated as \(\frac{14!}{8!} = 2192190\).
%
\par

  Notice that we can think of this counting problem as a question about counting functions: how many injective functions are there from your set of 6 chairs to your set of 14 friends (the functions are injective because you can't have a single chair go to two of your friends).
%
\end{enumerate}

              How are these numbers related? Notice that \(P(14,6)\) is \emph{much} larger than \({14 \choose 6}\). This makes sense. \({14 \choose 6}\) picks 6 friends, but \(P(14,6)\) arranges the 6 friends as well as picks them. In fact, we can say exactly how much larger \(P(14,6)\) is. In both counting problems we choose 6 out of 14 friends. For the first one, we stop there, at 3003 ways. But for the second counting problem, each of those 3003 choices of 6 friends can be arranged in exactly \(6!\) ways. So now we have \(3003\cdot 6!\) choices and that is exactly \(2192190\).
            %
\par

              Alternatively, look at the first problem another way. We want to select 6 out of 14 friends, but we do not care about the order they are selected in. To select 6 out of 14 friends, we might try this:
              \begin{equation*}
                14 \cdot 13 \cdot 12 \cdot 11 \cdot 10 \cdot 9.
              \end{equation*}
            %
\par

              This is a reasonable guess, since we have 14 choices for the first guest, then 13 for the second, and so on. But the guess is wrong (in fact, that product is exactly \(2192190 = P(14,6)\)). It distinguishes between the different orders in which we could invite the guests. To correct for this, we could divide by the number of different arrangements of the 6 guests (so that all of these would count as just one outcome). There are precisely \(6!\) ways to arrange 6 guests, so the correct answer to the first question is
              \begin{equation*}
                \frac{14 \cdot 13 \cdot 12 \cdot 11\cdot 10 \cdot 9}{6!}.
              \end{equation*}
            %
\par

              Note that another way to write this is
              \begin{equation*}
                \frac{14!}{8!\cdot 6!}.
              \end{equation*}
              which is what we had originally.
            %
\end{example}
\typeout{************************************************}
\typeout{Exercises 2.3.1 Exercises}
\typeout{************************************************}
\subsection[Exercises]{Exercises}\label{exercises-6}
\begin{exerciselist}
\item[1.]\hypertarget{exercise-60}{}
                A pizza parlor offers 10 toppings.
              %
\leavevmode%
\begin{enumerate}[label=(\alph*)]
\item\hypertarget{li-465}{} How many 3-topping pizzas could they put on their menu?  Assume double toppings are not allowed. %
\item\hypertarget{li-466}{} How many total pizzas are possible, with between zero and ten toppings (but not double toppings) allowed? %
\item\hypertarget{li-467}{} The pizza parlor will list the 10 toppings in two equal-sized columns on their menu.  How many ways can they arrange the toppings in the left column? %
\end{enumerate}
\par\smallskip
\item[2.]\hypertarget{exercise-61}{}
                A combination lock consists of a dial with 40 numbers on it. To open the lock, you turn the dial to the right until you reach a first number, then to the left until you get to second number, then to the right again to the third number. The numbers must be distinct. How many different combinations are possible?
              %
\par\smallskip
\item[3.]\hypertarget{exercise-62}{}
                Using the digits 2 through 8, find the number of different 5-digit numbers such that:
              %
\leavevmode%
\begin{enumerate}[label=(\alph*)]
\item\hypertarget{li-471}{} Digits can be used more than once. %
\item\hypertarget{li-472}{} Digits cannot be repeated, but can come in any order. %
\item\hypertarget{li-473}{} Digits cannot be repeated and must be written in increasing order. %
\item\hypertarget{li-474}{} Which of the above counting questions is a combination and which is a permutation?  Explain why this makes sense. %
\end{enumerate}
\par\smallskip
\item[4.]\hypertarget{exercise-63}{}
                How many quadrilaterals can you draw using the dots below as vertices (corners)?
              %
\leavevmode%
\begin{figure}
\centering
{
                 \begin{tikzpicture}[scale=.9]
 \foreach \x in {-3,...,3}
 \foreach \y in {-1,1}
 \fill (\x,\y) circle (3pt);
\end{tikzpicture}
}
\end{figure}
\par\smallskip
\item[5.]\hypertarget{exercise-64}{}
                How many of the quadrilaterals possible in the previous problem are:
              %
\leavevmode%
\begin{enumerate}[label=(\alph*)]
\item\hypertarget{li-479}{} Squares? %
\item\hypertarget{li-480}{} Rectangles? %
\item\hypertarget{li-481}{} Parallelograms? %
\item\hypertarget{li-482}{} Trapezoids?\footnote{Here, as in calculus, a trapezoid is defined as a quadrilateral with \emph{at least} one pair of parallel sides.  In particular, parallelograms are trapezoids.\label{fn-2}} %
\item\hypertarget{li-483}{} Trapezoids that are not parallelograms? %
\end{enumerate}
\par\smallskip
\item[6.]\hypertarget{exercise-65}{}
                An \emph{anagram} of a word is just a rearrangement of its letters. How many different anagrams of ``uncopyrightable'' are there? (This happens to be the longest common English word without any repeated letters.)
              %
\par\smallskip
\item[7.]\hypertarget{exercise-66}{}
                How many anagrams are there of the word ``assesses'' that start with the letter ``a''?
              %
\par\smallskip
\item[8.]\hypertarget{exercise-67}{}
                How many anagrams are there of ``anagram''?
              %
\par\smallskip
\item[9.]\hypertarget{exercise-68}{}
                On a business retreat, your company of 20 businessmen and businesswomen go golfing.
              %
\leavevmode%
\begin{enumerate}[label=(\alph*)]
\item\hypertarget{li-489}{}
You need to divide up into foursomes (groups of 4 people): a first foursome, a second foursome, and so on.  How many ways can you do this?
%
\item\hypertarget{li-490}{}
After all your hard work, you realize that in fact, you want each foursome to include one of the five Board members.  How many ways can you do this?
%
\end{enumerate}
\par\smallskip
\item[10.]\hypertarget{exercise-69}{}
                How many different seating arrangements are possible for King Arthur and his 9 knights around their round table?
              %
\par\smallskip
\item[11.]\hypertarget{exercise-70}{}
                Consider sets \(A\) and \(B\) with \(|A| = 10\) and \(|B| = 17\).
              %
\leavevmode%
\begin{enumerate}[label=(\alph*)]
\item\hypertarget{li-493}{}
    How many functions \(f: A \to B\) are there?
    %
\item\hypertarget{li-494}{}
    How many functions \(f: A \to B\) are injective?
    %
\end{enumerate}
\par\smallskip
\item[12.]\hypertarget{exercise-71}{}
                Consider functions \(f: \{1,2,3,4\} \to \{1,2,3,4,5,6\}\).
              %
\leavevmode%
\begin{enumerate}[label=(\alph*)]
\item\hypertarget{li-497}{}
    How many functions are there total?
    %
\item\hypertarget{li-498}{}
    How many functions are injective?
    %
\item\hypertarget{li-499}{} How many of the injective functions are \emph{increasing}?  To be increasing means that if \(a \lt b\) then \(f(a) \lt f(b)\), or in other words, the outputs get larger as the inputs get larger.
  %
\end{enumerate}
\par\smallskip
\end{exerciselist}
\typeout{************************************************}
\typeout{Section 2.4 Combinatorial Proofs}
\typeout{************************************************}
\section[Combinatorial Proofs]{Combinatorial Proofs}\label{sec_comb-proofs}
\typeout{************************************************}
\typeout{Introduction  }
\typeout{************************************************}
\begin{investigation}[]\label{investigation-9}
\leavevmode%
\begin{enumerate}
\item\hypertarget{li-503}{}
        The Stanley Cup is decided in a best of 7 tournament between two teams. In how many ways can your team win? Let's answer this question two ways:
        %
%
\begin{enumerate}
\item\hypertarget{li-504}{}
        How many of the 7 games does your team need to win?  How many ways can this happen?
        %
\item\hypertarget{li-505}{}
        What if the tournament goes all 7 games?  So you win the last game.  How many ways can the first 6 games go down?
        %
\item\hypertarget{li-506}{}
        What if the tournament goes just 6 games?  How many ways can this happen?  What about 5 games?  4 games?
        %
\item\hypertarget{li-507}{}
        What are the two different ways to compute the number of ways your team can win?  Write down an equation involving binomial coefficients (that is, \({n \choose k}\)'s).  What pattern in Pascal's triangle is this an example of?
        %
\end{enumerate}
\item\hypertarget{li-508}{}
        Generalize. What if the rules changed and you played a best of \(9\) tournament (5 wins required)? What if you played an \(n\) game tournament with \(k\) wins required to be named champion?
        %
\end{enumerate}
\end{investigation}
\typeout{************************************************}
\typeout{Subsection 2.4.1 Patterns in Pascal's Triangle}
\typeout{************************************************}
\subsection[Patterns in Pascal's Triangle]{Patterns in Pascal's Triangle}\label{subsec_patternsPascal}

      Have a look again at Pascal's triangle. Forget for a moment where it comes from - just look at it as a mathematical object. What do you notice?
    %
\leavevmode%
\begin{figure}
\centering
{
        \begin{tikzpicture}
\def\r{.5}

% Pascal's triangle
%put row of 1's down left side:
  \foreach \row in {0,...,7} {
    \hexbox{\row}{0}{ 1}
  }
%fill in the rest of the triangle:
  \foreach \row in {1,...,7} {
    \pgfmathsetmacro{\entry}{1};
    \foreach \col in {1,...,\row} {
      % iterative formula : val = precval * (row-col+1)/col
      % (+ 0.5 to bypass rounding errors)
     \pgfmathtruncatemacro{\entry}{\entry*((\row-\col+1)/\col)+0.5};
      \global\let\entry=\entry
      \ifnum \entry<100
	\hexbox{\row}{\col}{\entry}
      \else \ifnum \entry<1000
	\hexbox{\row}{\col}{\footnotesize \entry}
      \else \ifnum \entry<10000
	\hexbox{\row}{\col}{\footnotesize \entry}
	\else
	\hexbox{\row}{\col}{\scriptsize \entry}
	\fi
      \fi
      \fi
    }
  }
\end{tikzpicture}
}
\end{figure}
\par

      There are lots of patterns hidden away in the triangle, enough to fill a reasonably sized book. Here are just a few of the most obvious ones:
    %
\leavevmode%
\begin{enumerate}
\item\hypertarget{li-509}{}
The entries on the border of the triangle are all 1.
%
\item\hypertarget{li-510}{}
Any entry not on the border is the sum of the two entries above it.
%
\item\hypertarget{li-511}{}
The triangle is symmetric.  On any row, entries on the left side are mirrored on the right side.
%
\item\hypertarget{li-512}{}
The sum of all entries on a given row is a power of 2. (You should check this!)
%
\end{enumerate}
\par

      We would like to state these observations in a more precise way, and then prove that they are correct. Now each entry in Pascal's triangle is in fact a binomial coefficient. The 1 on the very top of the triangle is \({0 \choose 0}\). The next row (which we will call row 1, even though it is not the top-most row) consists of \({1 \choose 0}\) and \({1 \choose 1}\). Row 4 (the row 1, 4, 6, 4, 1) consists of the binomial coefficients
      \begin{equation*}
        {4 \choose 0} ~~ {4 \choose 1} ~~ {4 \choose 2} ~~ {4 \choose 3} ~~ {4 \choose 4}.
      \end{equation*}
    %
\par

      Given this description of the elements in Pascal's triangle, we can rewrite the above observations as follows:
    %
\leavevmode%
\begin{enumerate}
\item\hypertarget{li-513}{}\({n \choose 0} = 1\) and \({n \choose n} = 1\).%
\item\hypertarget{li-514}{}\({n \choose k} = {n-1 \choose k-1} + {n-1 \choose k}\).%
\item\hypertarget{li-515}{}\({n \choose k} = {n \choose n-k}\).%
\item\hypertarget{li-516}{}\({n\choose 0} + {n \choose 1} + {n \choose 2} + \cdots + {n \choose n} = 2^n\).%
\end{enumerate}
\par

      Each of these are an example of a \terminology{binomial identity}\index{binomial identity}: an identity (i.e., equation) involving binomial coefficients.
    %
\par

      Our goal is to establish these identities. We wish to prove that they hold for all values of \(n\) and \(k\). These proofs can be done in many ways. One option would be to give algebraic proofs, using the formula for \({n \choose k}\):
      \begin{equation*}
        {n \choose k} = \frac{n!}{(n-k)!\,k!}.
      \end{equation*}
    %
\par

      Here's how you might do that for the second identity above.
    %
\begin{example}[]\label{example-36}

          Give an algebraic proof for the binomial identity
          \begin{equation*}
            {n \choose k} = {n-1\choose k-1} + {n-1 \choose k}.
          \end{equation*}
        %
\begin{proof}\hypertarget{proof-3}{}

          By the definition of \({n \choose k}\), we have
          \begin{equation*}
            {n-1 \choose k-1} = \frac{(n-1)!}{(n-1-(k-1))!(k-1)!} = \frac{(n-1)!}{(n-k)!(k-1)!} \hfill\end{equation*} and
          \begin{equation*} {n-1 \choose k} = \frac{(n-1)!}{(n-1-k)!k!}.
          \end{equation*}
        %
\par

          Thus, starting with the right hand side of the equation:
          \begin{align*}
 {n-1 \choose k-1} + {n-1 \choose k} \amp = \frac{(n-1)!}{(n-k)!(k-1)!}+ \frac{(n-1)!}{(n-1-k)!\,k!}\\
 \amp = \frac{(n-1)!k}{(n-k)!\,k!} + \frac{(n-1)!(n-k)}{(n-k)!\,k!}\\
 \amp = \frac{(n-1)!(k+n-k)}{(n-k)!\,k!}\\
 \amp = \frac{n!}{(n-k)!\, k!}\\
 \amp = {n \choose k}.
\end{align*}
        %
\par

          The second line (where the common denominator is found) works because \(k(k-1)! = k!\) and \((n-k)(n-k-1)! = (n-k)!\).
        %
\end{proof}
\end{example}
\par

      This is certainly a valid proof, but also is entirely useless. Even if you understand the proof perfectly, it does not tell you \emph{why} the identity is true. A better approach would be to explain what \({n \choose k}\) \emph{means} and then say why that is also what \({n-1 \choose k-1} + {n-1 \choose k}\) means. Let's see how this works for the four identities we observed above.
    %
\begin{example}[]\label{example-37}

          Explain why \({n \choose 0} = 1\) and \({n \choose n} = 1\).
        %
\par\medskip\noindent%
\textbf{Solution.}\quad 
          What do these binomial coefficients tell us? Well, \({n \choose 0}\) gives the number of ways to select 0 objects from a collection of \(n\) objects. There is only one way to do this, namely to not select any of the objects. Thus \({n \choose 0} = 1\). Similarly, \({n \choose n}\) gives the number of ways to select \(n\) objects from a collection of \(n\) objects. There is only one way to do this: select all \(n\) objects. Thus \({n \choose n} = 1\).
        %
\par

          Alternatively, we know that \({n \choose 0}\) is the number of \(n\)-bit strings with weight 0. There is only one such string, the string of all 0's. So \({n \choose 0} = 1\). Similarly \({n \choose n}\) is the number of \(n\)-bit strings with weight \(n\). There is only one string with this property, the string of all 1's.
        %
\par

          Another way: \({n \choose 0}\) gives the number of subsets of a set of size \(n\) containing 0 elements. There is only one such subset, the empty set. \({n \choose n}\) gives the number of subsets containing \(n\) elements. The only such subset is the original set (of all elements).
        %
\end{example}
\begin{example}[]\label{example-38}

          Explain why \({n \choose k} = {n-1 \choose k-1} + {n-1 \choose k}\).
        %
\par\medskip\noindent%
\textbf{Solution.}\quad 
          The easiest way to see this is to consider bit strings. \({n \choose k}\) is the number of bit strings of length \(n\) containing \(k\) 1's. Of all of these strings, some start with a 1 and the rest start with a 0. First consider all the bit strings which start with a 1. After the 1, there must be \(n-1\) more bits (to get the total length up to \(n\)) and exactly \(k-1\) of them must be 1's (as we already have one, and we need \(k\) total). How many strings are there like that? There are exactly \({n-1 \choose k-1}\) such bit strings, so of all the length \(n\) bit strings containing \(k\) 1's, \({n-1 \choose k-1}\) of them start with a 1. Similarly, there are \({n-1\choose k}\) which start with a 0 (we still need \(n-1\) bits and now \(k\) of them must be 1's). Since there are \({n-1 \choose k}\) bit strings containing \(n-1\) bits with \(k\) 1's, that is the number of length \(n\) bit strings with \(k\) 1's which start
          with a 0. Therefore \({n \choose k} = {n-1\choose k-1} + {n-1 \choose k}\).
        %
\par

          Another way: consider the question, how many ways can you select \(k\) pizza toppings from a menu containing \(n\) choices? One way to do this is just \({n \choose k}\). Another way to answer the same question is to first decide whether or not you want anchovies. If you do want anchovies, you still need to pick \(k-1\) toppings, now from just \(n-1\) choices. That can be done in \({n-1 \choose k-1}\) ways. If you do not want anchovies, then you still need to select \(k\) toppings from \(n-1\) choices (the anchovies are out). You can do that in \({n-1 \choose k}\) ways. Since the choices with anchovies are disjoint from the choices without anchovies, the total choices are \({n-1 \choose k-1}+{n-1 \choose k}\). But wait. We answered the same question in two different ways, so the two answers must be the same. Thus \({n \choose k} = {n-1\choose k-1} + {n-1 \choose k}\).
        %
\par

          You can also explain (prove) this identity by counting subsets, or even lattice paths.
        %
\end{example}
\begin{example}[]\label{ex_symmetry-formula}

          Prove the binomial identity \({n \choose k} = {n \choose n-k}\).
        %
\par\medskip\noindent%
\textbf{Solution.}\quad 
          Why is this true? \({n \choose k}\) counts the number of ways to select \(k\) things from \(n\) choices. On the other hand, \({n \choose n-k}\) counts the number of ways to select \(n-k\) things from \(n\) choices. Are these really the same? Well, what if instead of selecting the \(n-k\) things you choose to exclude them. How many ways are there to choose \(n-k\) things to exclude from \(n\) choices. Clearly this is \({n \choose n-k}\) as well (it doesn't matter whether you include or exclude the things once you have chosen them). And if you exclude \(n-k\) things, then you are including the other \(k\) things. So the set of outcomes should be the same.
        %
\par

          Let's try the pizza counting example like we did above. How many ways are there to pick \(k\) toppings from a list of \(n\) choices? On the one hand, the answer is simply \({n \choose k}\). Alternatively, you could make a list of all the toppings you don't want. To end up with a pizza containing exactly \(k\) toppings, you need to pick \(n-k\) toppings to not put on the pizza. You have \({n \choose n-k}\) choices for the toppings you don't want. Both of these ways give you a pizza with \(k\) toppings, in fact all the ways to get a pizza with \(k\) toppings. Thus these two answers must be the same: \({n \choose k} = {n \choose n-k}\).
        %
\par

          You can also prove (explain) this identity using bit strings, subsets, or lattice paths. The bit string argument is nice: \({n \choose k}\) counts the number of bit strings of length \(n\) with \(k\) 1's. This is also the number of bit string of length \(n\) with \(k\) 0's (just replace each 1 with a 0 and each 0 with a 1). But if a string of length \(n\) has \(k\) 0's, it must have \(n-k\) 1's. And there are exactly \({n\choose n-k}\) strings of length \(n\) with \(n-k\) 1's.
        %
\end{example}
\begin{example}[]\label{example-40}

          Prove the binomial identity \({n\choose 0} + {n \choose 1} + {n\choose 2} + \cdots + {n \choose n} = 2^n\).
        %
\par\medskip\noindent%
\textbf{Solution.}\quad 
          Let's do a ``pizza proof'' again. We need to find a question about pizza toppings which has \(2^n\) as the answer. How about this: If a pizza joint offers \(n\) toppings, how many pizzas can you build using any number of toppings from no toppings to all toppings, using each topping at most once?
        %
\par

          On one hand, the answer is \(2^n\). For each topping you can say ``yes'' or ``no,'' so you have two choices for each topping.
        %
\par

          On the other hand, divide the possible pizzas into disjoint groups: the pizzas with no toppings, the pizzas with one topping, the pizzas with two toppings, etc. If we want no toppings, there is only one pizza like that (the empty pizza, if you will) but it would be better to think of that number as \({n \choose 0}\) since we choose 0 of the \(n\) toppings. How many pizzas have 1 topping? We need to choose 1 of the \(n\) toppings, so \({n \choose 1}\). We have:
        %
\leavevmode%
\begin{itemize}[label=\textbullet]
\item{}Pizzas with 0 toppings: \({n \choose 0}\)%
\item{}Pizzas with 1 topping: \({n \choose 1}\)%
\item{}Pizzas with 2 toppings: \({n \choose 2}\)%
\item{}\(\vdots\)%
\item{}Pizzas with \(n\) toppings: \({n \choose n}\).%
\end{itemize}
\par

          The total number of possible pizzas will be the sum of these, which is exactly the left hand side of the identity we are trying to prove.
        %
\par

          Again, we could have proved the identity using subsets, bit strings, or lattice paths (although the lattice path argument is a little tricky).
        %
\end{example}
\par

      Hopefully this gives some idea of how explanatory proofs of binomial identities can go. It is worth pointing out that sometimes more traditional proofs are also nice.
      \footnote{Most every binomial identity can be proved using mathematical induction, using the recursive definition for \({n \choose k}\). We will discuss induction in
        {$\langle\langle$sec_seq-induction$\rangle\rangle$}.\label{fn-3}} For example, consider the following rather slick proof of the last identity.
    %
\par

      Recall the binomial theorem:
      \begin{equation*}
        (x + y)^n = {n \choose 0}x^n + {n \choose 1}x^{n-1}y + {n \choose 2}x^{n-2}y^2 + \cdots + {n \choose n-1}x\cdot y^n + {n \choose n}y^n.
      \end{equation*}
    %
\par

      Let \(x = 1\) and \(y = 1\). We get:
      \begin{equation*}
        (1 + 1)^n = {n \choose 0}1^n + {n \choose 1}1^{n-1}1 + {n \choose 2}1^{n-2}1^2 + \cdots + {n \choose n-1}1\cdot 1^n + {n \choose n}1^n.
      \end{equation*}
    %
\par

      Of course this simplifies to:
      \begin{equation*}
        (2)^n = {n \choose 0} + {n \choose 1} + {n \choose 2} + \cdots + {n \choose n-1} + {n \choose n}.
      \end{equation*}
    %
\par

      Something fun to try: Let \(x = 1\) and \(y = 2\). Neat huh?
    %
\typeout{************************************************}
\typeout{Subsection 2.4.2 More Proofs}
\typeout{************************************************}
\subsection[More Proofs]{More Proofs}\label{subsec_moreProofs}

      The explanatory proofs given in the above examples are typically called \terminology{combinatorial proofs}. In general, to give a combinatorial proof for a binomial identity, say \(A = B\) you do the following:
    %
\leavevmode%
\begin{enumerate}
\item\hypertarget{li-522}{}
      Find a counting problem you will be able to answer in two ways.
      %
\item\hypertarget{li-523}{}
      Explain why one answer to the counting problem is \(A\).
      %
\item\hypertarget{li-524}{}
      Explain why the other answer to the counting problem is \(B\).
      %
\end{enumerate}
\par

      Since both \(A\) and \(B\) are the answers to the same question, we must have \(A = B\).
    %
\par

      The tricky thing is coming up with the question. This is not always obvious, but it gets easier the more counting problems you solve. You will start to recognize types of answers as the answers to types of questions. More often what will happen is you will be solving a counting problem and happen to think up two different ways of finding the answer. Now you have a binomial identity and the proof is right there. The proof \emph{is} the problem you just solved together with your two solutions.
    %
\par

      For example, consider this counting question:

      \begin{quote}
        How many 10-letter words use exactly four A's, three B's, two C's and one D?
      \end{quote}


      Let's try to solve this problem. We have 10 spots for letters to go. Four of those need to be A's. We can pick the four A-spots in \({10 \choose 4}\) ways. Now where can we put the B's? Well there are only 6 spots left, we need to pick \(3\) of them. This can be done in \({6 \choose 3}\) ways. The two C's need to go in two of the 3 remaining spots, so we have \({3 \choose 2}\) ways of doing that. That leaves just one spot of the D, but we could write that 1 choice as \({1 \choose 1}\). Thus the answer is:
      \begin{equation*}
        {10 \choose 4}{6 \choose 3}{3 \choose 2}{1 \choose 1}.
      \end{equation*}
    %
\par

      But why stop there? We can find the answer another way too. First let's decide where to put the one D: we have 10 spots, we need to choose 1 of them, so this can be done in \({10 \choose 1}\) ways. Next, choose one of the \({9 \choose 2}\) ways to place the two C's. We now have \(7\) spots left, and three of them need to be filled with B's. There are \({7 \choose 3}\) ways to do this. Finally the A's can be placed in \({4 \choose 4}\) (that is, only one) ways. So another answer to the question is
      \begin{equation*}
        {10 \choose 1}{9 \choose 2}{7 \choose 3}{4 \choose 4}.
      \end{equation*}
    %
\par

      Interesting. This gives us the binomial identity:
      \begin{equation*}
        {10 \choose 4}{6 \choose 3}{3 \choose 2}{1 \choose 1} = {10 \choose 1}{9 \choose 2}{7 \choose 3}{4 \choose 4}.
      \end{equation*}
    %
\par

      Here are a couple of other binomial identities with combinatorial proofs.
    %
\begin{example}[]\label{example-41}

          Prove the identity
          \begin{equation*}
            1 n + 2(n-1) + 3 (n-2) + \cdots + (n-1) 2 + n 1 = {n+2 \choose 3}.
          \end{equation*}
        %
\par\medskip\noindent%
\textbf{Solution.}\quad 
          To give a combinatorial proof we need to think up a question we can answer in two ways: one way needs to give the left-hand-side of the identity, the other way needs to be the right-hand-side of the identity. Our clue to what question to ask comes from the right hand side: \({n+2 \choose 3}\) counts the number of ways to select 3 things from a group of \(n+2\) things. Let's name those things \(1, 2, 3, \ldots, n+2\). In other words, we want to find 3-element subsets of those numbers (since order should not matter, subsets are exactly the right thing to think about). We will have to be a bit clever to explain why the left-hand-side also gives the number of these subsets. Here's the proof.
        %
\begin{proof}\hypertarget{proof-4}{}

            Consider the question ``How many 3-element subsets are there of the set \(\{1,2,3,\ldots, n+2\}\)?'' We answer this in two ways:
          %
\par

            Answer 1: We must select 3 elements from the collection of \(n+2\) elements. This can be done in \({n+2 \choose 3}\) ways.
          %
\par

            Answer 2: Break this problem up into cases by what the middle number in the subset is. Say each subset is \(\{a,b,c\}\) written in increasing order. We count the number of subsets for each distinct value of \(b\). The smallest possible value of \(b\) is \(2\), and the largest is \(n+1\).
          %
\par

            When \(b = 2\), there are \(1 \cdot n\) subsets: 1 choice for \(a\) and \(n\) choices (3 through \(n+2\)) for \(c\).
          %
\par

            When \(b = 3\), there are \(2 \cdot (n-1)\) subsets: 2 choices for \(a\) and \(n-1\) choices for \(c\).
          %
\par

            When \(b = 4\), there are \(3 \cdot (n-2)\) subsets: 3 choices for \(a\) and \(n-2\) choices for \(c\).
          %
\par

            And so on. When \(b = n+1\), there are \(n\) choices for \(a\) and only 1 choice for \(c\), so \(n \cdot 1\) subsets.
          %
\par

            Therefore the total number of subsets is
            \begin{equation*}
              1 n + 2 (n-1) + 3 (n-2) + \cdots + (n-1)2 + n 1.
            \end{equation*}
          %
\par

            Since Answer 1 and Answer 2 are answers to the same question, they must be equal. Therefore
            \begin{equation*}
              1 n + 2 (n-1) + 3 (n-2) + \cdots + (n-1) 2 + n 1 = {n+2 \choose 3}.
            \end{equation*}
          %
\end{proof}
\end{example}
\begin{example}[]\label{example-42}

          Prove the binomial identity
          \begin{equation*}
            {n \choose 0}^2 + {n \choose 1}^2 + {n \choose 2}^2 + \cdots + {n \choose n}^2 = {2n \choose n}.
          \end{equation*}
        %
\par\medskip\noindent%
\textbf{Solution.}\quad 
          We will give two different proofs of this fact. The first will be very similar to the previous example (counting subsets). The second proof is a little slicker, using lattice paths.
        %
\begin{proof}\hypertarget{proof-5}{}

            Consider the question: ``How many pizzas can you make using \(n\) toppings when there are \(2n\) toppings to choose from?''
          %
\par

            Answer 1: There are \(2n\) toppings, from which you must choose \(n\). This can be done in \({2n \choose n}\) ways.
          %
\par

            Answer 2: Divide the toppings into two groups of \(n\) toppings (perhaps \(n\) meats and \(n\) veggies). Any choice of \(n\) toppings must include some number from the first group and some number from the second group. Consider each possible number of meat toppings separately:
          %
\par

            0 meats: \({n \choose 0}{n \choose n}\), since you need to choose 0 of the \(n\) meats and \(n\) of the \(n\) veggies.
          %
\par

            1 meat: \({n \choose 1}{n \choose n-1}\), since you need 1 of \(n\) meats so \(n-1\) of \(n\) veggies.
          %
\par

            2 meats: \({n \choose 2}{n \choose n-2}\). Choose 2 meats and the remaining \(n-2\) toppings from the \(n\) veggies.
          %
\par

            And so on. The last case is \(n\) meats, which can be done in \({n \choose n}{n \choose 0}\) ways.
          %
\par

            Thus the total number of pizzas possible is
            \begin{equation*}
              {n \choose 0}{n \choose n} + {n \choose 1}{n \choose n-1} + {n \choose 2}{n \choose n-2} + \cdots + {n \choose n}{n \choose 0}.
            \end{equation*}
          %
\par

            This is not quite the left hand side \dots{} yet. Notice that \({n \choose n} = {n \choose 0}\) and \({n \choose n-1} = {n  \choose 1}\) and so on, by the identity in
            \hyperref[ex_symmetry-formula]{Example~\ref{ex_symmetry-formula}}. Thus we do indeed get
            \begin{equation*}
              {n \choose 0}^2 + {n \choose 1}^2 + {n \choose 2}^2 + \cdots + {n \choose n}^2.
            \end{equation*}
          %
\par

            Since these two answers are answers to the same question, they must be equal, and thus
            \begin{equation*}
              {n \choose 0}^2 + {n \choose 1}^2 + {n \choose 2}^2 + \cdots + {n \choose n}^2 = {2n \choose n}.
            \end{equation*}
          %
\end{proof}
\par

          For an alternative proof, we use lattice paths. This is reasonable to consider because the right hand side of the identity reminds us of the number of paths from \((0,0)\) to \((n,n)\).
        %
\begin{proof}\hypertarget{proof-6}{}

            Consider the question: How many lattice paths are there from \((0,0)\) to \((n,n)\)?
          %
\par

            Answer 1: We must travel \(2n\) steps, and \(n\) of them must be in the up direction. Thus there are \({2n \choose n}\) paths.
          %
\par

            Answer 2: Note that any path from \((0,0)\) to \((n,n)\) must cross the line \(x + y = n\). That is, any path must pass through exactly one of the points: \((0,n)\), \((1,n-1)\), \((2,n-2)\), \dots{}, \((n, 0)\). For example, this is what happens in the case \(n = 4\):
          %
\leavevmode%
\begin{figure}
\centering
{
          \begin{tikzpicture}[scale=0.8]
  \draw[very thin, color=gray!50] (-.5,-.5) grid (4.5, 4.5);
  \draw (0,0) circle (1.5pt) node[below left] { (0,0)} (4,4) circle (1.5pt) node[above right] { (4,4)};
  \draw[thick] (-.5, 4.5) -- (4.5, -.5) node[right]{ \(x + y = 4\)};
  \draw (0,4) circle (1.5pt) node[above right]{ (0,4)} (1,3) circle (1.5pt) node[above right]{ (1,3)} (2,2) circle (1.5pt) node[above right]{ (2,2)} (3,1) circle (1.5pt) node[above right]{ (3,1)} (4,0) circle (1.5pt) node[above right]{ (4,0)};
\end{tikzpicture}
}
\end{figure}
\par

            How many paths pass through \((0,n)\)? To get to that point, you must travel \(n\) units, and \(0\) of them are to the right, so there are \({n \choose 0}\) ways to get to \((0,n)\). From \((0,n)\) to \((n,n)\) takes \(n\) steps, and \(0\) of them are up. So there are \({n \choose 0}\) ways to get from \((0,n)\) to \((n,n)\). Therefore there are \({n \choose 0}{n \choose 0}\) paths from \((0,0)\) to \((n,n)\) through the point \((0,n)\).
          %
\par

            What about through \((1,n-1)\). There are \({n \choose 1}\) paths to get there (\(n\) steps, 1 to the right) and \({n \choose 1}\) paths to complete the journey to \((n,n)\) (\(n\) steps, \(1\) up). So there are \({n \choose 1}{n \choose 1}\) paths from \((0,0)\) to \((n,n)\) through \((1,n-1)\).
          %
\par

            In general, to get to \((n,n)\) through the point \((k,n-k)\) we have \({n \choose k}\) paths to the midpoint and then \({n \choose k}\) paths from the midpoint to \((n,n)\). So there are \({n \choose k}{n \choose k}\) paths from \((0,0)\) to \((n,n)\) through \((k, n-k)\).
          %
\par

            All together then the total paths from \((0,0)\) to \((n,n)\) passing through exactly one of these midpoints is
            \begin{equation*}
              {n \choose 0}^2 + {n \choose 1}^2 + {n \choose 2}^2 + \cdots + {n \choose n}^2.
            \end{equation*}
          %
\par

            Since these two answers are answers to the same question, they must be equal, and thus
            \begin{equation*}
              {n \choose 0}^2 + {n \choose 1}^2 + {n \choose 2}^2 + \cdots + {n \choose n}^2 = {2n \choose n}.
            \end{equation*}
          %
\end{proof}
\end{example}
\typeout{************************************************}
\typeout{Exercises 2.4.3 Exercises}
\typeout{************************************************}
\subsection[Exercises]{Exercises}\label{exercises-7}
\begin{exerciselist}
\item[1.]\hypertarget{exercise-72}{}
          Prove the identity \({n\choose k} = {n-1 \choose k-1} + {n-1 \choose k}\) using a question about subsets.
        %
\par\smallskip
\item[2.]\hypertarget{exercise-73}{}
          Give a combinatorial proof of the identity \(2+2+2 = 3\cdot 2\).
        %
\par\smallskip
\item[3.]\hypertarget{exercise-74}{}
          Give a combinatorial proof for the identity \(1 + 2 + 3 + \cdots + n = {n+1 \choose 2}\).
        %
\par\smallskip
\item[4.]\hypertarget{exercise-75}{}
          A woman is getting married. She has 15 best friends but can only select 6 of them to be her bridesmaids, one of which needs to be her maid of honor. How many ways can she do this?
        %
\leavevmode%
\begin{enumerate}[label=(\alph*)]
\item\hypertarget{li-525}{} What if she first selects the 6 bridesmaids, and then selects one of them to be the maid of honor? %
\item\hypertarget{li-526}{} What if she first selects her maid of honor, and then 5 other bridemaids? %
\item\hypertarget{li-527}{} Explain why \(6 {15 \choose 6} = 15 {14 \choose 5}\). %
\end{enumerate}
\par\smallskip
\item[5.]\hypertarget{exercise-76}{}
          Give a combinatorial proof of the identity \({n \choose 2}{n-2 \choose k-2} = {n\choose k}{k \choose 2}\).
        %
\par\smallskip
\item[6.]\hypertarget{exercise-77}{}
          Consider the bit strings in \(\B^6_2\) (bit strings of length 6 and weight 2).
        %
\leavevmode%
\begin{enumerate}[label=(\alph*)]
\item\hypertarget{li-531}{} How many of those bit strings start with 01? %
\item\hypertarget{li-532}{} How many of those bit strings start with 001? %
\item\hypertarget{li-533}{} Are there any other strings we have not counted yet?  Which ones, and how many are there? %
\item\hypertarget{li-534}{} How many bit strings are there total in \(\B^6_2\)? %
\item\hypertarget{li-535}{} What binomial identity have you just given a combinatorial proof for? %
\end{enumerate}
\par\smallskip
\item[7.]\hypertarget{exercise-78}{}
          Let's count \terminology{ternary} digit strings, that is, strings in which each digit can be 0, 1, or 2.
        %
\leavevmode%
\begin{enumerate}[label=(\alph*)]
\item\hypertarget{li-542}{} How many ternary digit strings contain exactly \(n\) digits? %
\item\hypertarget{li-543}{} How many ternary digit strings contain exactly \(n\) digits and \(n\) 2's. %
\item\hypertarget{li-544}{} How many ternary digit strings contain exactly \(n\) digits and \(n-1\) 2's.  (Hint: where can you put the non-2 digit, and then what could it be?) %
\item\hypertarget{li-545}{} How many ternary digit strings contain exactly \(n\) digits and \(n-2\) 2's.  (Hint: see previous hint) %
\item\hypertarget{li-546}{} How many ternary digit strings contain exactly \(n\) digits and \(n-k\) 2's. %
\item\hypertarget{li-547}{} How many ternary digit strings contain exactly \(n\) digits and no 2's. (Hint: what kind of a string is this?) %
\item\hypertarget{li-548}{} Use the above parts to give a combinatorial proof for the identity \begin{equation*} {n \choose 0} + 2{n \choose 1} + 2^2{n \choose 2} + 2^3{n \choose 3} + \cdots + 2^n{n \choose n} = 3^n. \end{equation*} %
\end{enumerate}
\par\smallskip
\item[8.]\hypertarget{exercise-79}{}
          How many ways are there to rearrange the letters in the word ``rearrange''? Answer this question in at least two different ways to establish a binomial identity.
        %
\par\smallskip
\item[9.]\hypertarget{exercise-80}{}
          Give a combinatorial proof for the identity \(P(n,k) = {n \choose k}k!\)
        %
\par\smallskip
\item[10.]\hypertarget{exercise-81}{}
          Establish the identity below using a combinatorial proof.
          \begin{equation*}
            {2 \choose 2}{n \choose 2} + {3 \choose 2}{n-1 \choose 2} + {4\choose 2}{n-2 \choose 2} + \cdots + {n\choose 2}{2\choose 2} = {n+3 \choose 5}.
          \end{equation*}
        %
\par\smallskip
\end{exerciselist}
\typeout{************************************************}
\typeout{Section 2.5 Stars and Bars}
\typeout{************************************************}
\section[Stars and Bars]{Stars and Bars}\label{sec_stars-and-bars}

\index{stars and bars}
%
\begin{investigation}[]\label{investigation-10}

  Suppose you have some number of identical Rubik's cubes to distribute to your friends. Imagine you start with a single row of the cubes.


\leavevmode%
\begin{enumerate}
\item\hypertarget{li-556}{}
 Find the number of different ways you can distribute the cubes provided:
%
%
\begin{enumerate}
\item\hypertarget{li-557}{}
you have 3 cubes to give to 2 people.
%
\item\hypertarget{li-558}{}
you have 4 cubes to give to 2 people.
%
\item\hypertarget{li-559}{}
you have 5 cubes to give to 2 people.
%
\item\hypertarget{li-560}{}
you have 3 cubes to give to 3 people.
%
\item\hypertarget{li-561}{}
you have 4 cubes to give to 3 people.
%
\item\hypertarget{li-562}{}
you have 5 cubes to give to 3 people.
%
\end{enumerate}
\item\hypertarget{li-563}{}
Make a conjecture about how many different ways you could distribute 7 cubes to 4 people. Explain.
%
\item\hypertarget{li-564}{}
What if each person were required to get \emph{at least one} cube? How would your answers change?
%
\end{enumerate}

%
\end{investigation}
\par

Consider the following counting problem:
%
\begin{quote}
You have 7 cookies to give to 4 kids.  How many ways can you do this?
\end{quote}
\par

Take a moment to think about how you might solve this problem. You may assume that it is acceptable to give a kid no cookies. Also, the cookies are all identical and the order in which you give out the cookies does not matter.
%
\par

Before solving the problem, here is a wrong answer: You might guess that the answer should be \(4^7\) because for each of the 7 cookies, there are 4 choices of kids to which you can give the cookie. This is reasonable, but wrong. To see why, consider a few possible outcomes: we could assign the first six cookies to kid A, and the seventh cookie to kid B. Another outcome would assign the first cookie to kid B and the six remaining cookies to kid A. Both outcomes are included in the \(4^7\) answer. But for our counting problem, both outcomes are really the same \textendash{} kid A gets six cookies and kid B gets one cookie.
%
\par

What do outcomes actually look like? How can we represent them? One approach would be to write a outcome as a string of four numbers like this:
\begin{equation*}
  3112,
\end{equation*}
which represent the outcome in which the first kid gets 3 cookies, the second and third kid each get 1 cookie, and the fourth kid gets 2 cookies. Represented this way, the order in which the numbers occur matters. 1312 is a different outcome, because the first kid gets a one cookie instead of 3. Each number in the string can be any integer between 0 and 7. But the answer is not \(7^4\). We need the \emph{sum} of the numbers to be 7.
%
\par

Another way we might represent outcomes is to write a string of seven letters:
\begin{equation*}
  \mbox{ABAADCD} ,
\end{equation*}
which represents that the first cookie goes to kid A, the second cookie goes to kid B, the third and fourth cookies go to kid A, and so on. In fact, this outcome is identical to the previous one\textemdash{}A gets 3 cookies, B and C get 1 each and D gets 2. Each of the seven letters in the string can be any of the 4 possible letters (one for each kid), but the number of such strings is not \(4^7\), because here order does \emph{not} matter. In fact, another way to write the same outcome is
\begin{equation*}
  \mbox{AAABCDD} .
\end{equation*}
%
\par

This will be the preferred representation of the outcome. Since we can write the letters in any order, we might as well write them in \emph{alphabetical} order for the purposes of counting. So we will write all the A's first, then all the B's, and so on.
%
\par

Now think about how you could specify such an outcome. All we really need to do is say when to switch from one letter to the next. In terms of cookies, we need to say after how many cookies do we stop giving cookies to the first kid and start giving cookies to the second kid. And then after how many do we switch to the third kid? And after how many do we switch to the fourth? So yet another way to represent an outcome is like this:
\begin{equation*}
  ***|*|*|**
\end{equation*}
%
\par

Three cookies go to the first kid, then we switch and give one cookie to the second kid, then switch, one to the third kid, switch, two to the fourth kid. Notice that we need 7 stars and 3 bars \textendash{} one star for each cookie, and one bar for each switch between kids, so one fewer bars than there are kids (we don't need to switch after the last kid \textendash{} we are done).
%
\par

Why have we done all of this? Simple: to count the number of ways to distribute 7 cookies to 4 kids, all we need to do is count how many \emph{stars and bars} charts there are. But a stars and bars chart is just a string of symbols, some stars and some bars. If instead of stars and bars we would use 0's and 1's, it would just be a bit string. We know how to count those.
%
\par

Before we get too excited, we should make sure that really \emph{any} string of (in our case) 7 stars and 3 bars corresponds to a different way to distribute cookies to kids. In particular consider a string like this:
\begin{equation*}
  |***||****
\end{equation*}
%
\par

Does that correspond to a cookie distribution? Yes. It represents the distribution in which kid A gets 0 cookies (because we switch to kid B before any stars), kid B gets three cookies (three stars before the next bar), kid C gets 0 cookies (no stars before the next bar) and kid D gets the remaining 4 cookies. No matter how the stars and bars are arranged, we can distribute cookies in that way. Also, given any way to distribute cookies, we can represent that with a stars and bars chart. For example, the distribution in which kid A gets 6 cookies and kid B gets 1 cookie has the following chart:
\begin{equation*}
  ******|*||
\end{equation*}
%
\par

After all that work we are finally ready to count. Each way to distribute cookies corresponds to a stars and bars chart with 7 stars and 3 bars. So there are 10 symbols, and we must choose 3 of them to be bars. Thus:
\begin{equation*}
  \mbox{ There are } {10 \choose 3}\mbox{ ways to distribute 7 cookies to 4 kids.}
\end{equation*}
%
\par

While we are at it, we can also answer a related question: how many ways are there to distribute 7 cookies to 4 kids so that each kid gets at least one cookie? What can you say about the corresponding stars and bars charts? The charts must start and end with at least one star (so that kids A and D) get cookies, and also no two bars can be adjacent (so that kids B and C are not skipped). One way to assure this is to only place bars in the spaces \emph{between} the stars. With 7 stars, there are 6 spots between the stars, so we must choose 3 of those 6 spots to fill with bars. Thus there are \({6 \choose 3}\) ways to distribute 7 cookies to 4 kids giving at least one cookie to each kid.
%
\par

Another (and more general) way to approach this modified problem is to first give each kid one cookie. Now the remaining 3 cookies can be distributed to the 4 kids without restrictions. So we have 3 stars and 3 bars for a total of 6 symbols, 3 of which must be bars. So again we see that there are \({6 \choose 3}\) ways to distribute the cookies.
%
\par

Stars and bars can be used in counting problems other than kids and cookies. Here are a few examples:
%
\begin{example}[]\label{example-43}

Your favorite mathematical pizza chain offers 10 toppings. How many pizzas can you make if you are allowed 6 toppings? The order of toppings does not matter but now you are allowed repeats. So one possible pizza is triple sausage, double pineapple, and onions.
%
\par\medskip\noindent%
\textbf{Solution.}\quad 
We get 6 toppings (counting possible repeats). Represent each of these toppings as a star. Think of going down the menu one topping at a time: you see anchovies first, and skip to the next, sausage. You say yes to sausage 3 times (use 3 stars), then switch to the next topping on the list. You keep skipping until you get to pineapple, which you say yes to twice. Another switch and you are at onions. You say yes once. Then you keep switching until you get to the last topping, never saying yes again (since you already have said yes 6 times. There are 10 toppings to choose from, so we must switch from considering one topping to the next 9 times. These are the bars.
%
\par

Now that we are confident that we have the right number of stars and bars, we answer the question simply: there are 6 stars and 9 bars, so 15 symbols. We need to pick 9 of them to be bars, so there number of pizzas possible is
\begin{equation*}
  {15 \choose 9}.
\end{equation*}
%
\end{example}
\begin{example}[]\label{example-44}

How many 7 digit phone numbers are there in which the digits are non-increasing? That is, every digit is less than or equal to the previous one.
%
\par\medskip\noindent%
\textbf{Solution.}\quad 
We need to decide on 7 digits so we will use 7 stars. The bars will represent a switch from each possible single digit number down the next smaller one. So the phone number 866-5221 is represented by the stars and bars chart
\begin{equation*}
  |*||**|*|||**|*|
\end{equation*}
%
\par

There are 10 choices for each digit (0-9) so we must switch between choices 9 times. We have 7 stars and 9 bars, so the total number of phone numbers is
\begin{equation*}
  {16 \choose 9}.
\end{equation*}
%
\end{example}
\begin{example}[]\label{example-45}

How many integer solutions are there to the equation
\begin{equation*}
  x_1 + x_2 + x_3 + x_4 + x_5 = 13.
\end{equation*}
%
\par

(An integer solution to an equation is a solution in which the unknown must have an integer value.)
%
\leavevmode%
\begin{enumerate}
\item\hypertarget{li-565}{}
where \(x_i \ge 0\) for each \(x_i\)?
%
\item\hypertarget{li-566}{}
where \(x_i > 0\) for each \(x_i\)?
%
\item\hypertarget{li-567}{}
where \(x_i \ge 2\) for each \(x_i\)?
%
\end{enumerate}
\par\medskip\noindent%
\textbf{Solution.}\quad 
This problem is just like giving 13 cookies to 5 kids. We need to say how many of the 13 units go to each of the 5 variables. In other words, we have 13 stars and 4 bars (the bars are like the ``+'' signs in the equation).
%
\leavevmode%
\begin{enumerate}
\item\hypertarget{li-568}{}
If \(x_i\) can be 0 or greater, we are in the standard case with no restrictions.  So 13 stars and 4 bars can be arranged in \({17 \choose 4}\) ways.
%
\item\hypertarget{li-569}{}
Now each variable must be at least 1.  So give one unit to each variable to satisfy that restriction.  Now there are 8 stars left, and still 4 bars, so the number of solutions is \({12 \choose 4}\).
%
\item\hypertarget{li-570}{}
Now each variable must be 2 or greater.  So before any counting, give each variable 2 units.  We now have 3 remaining stars and 4 bars, so there are \({7 \choose 4}\) solutions.
%
\end{enumerate}
\end{example}
\typeout{************************************************}
\typeout{Paragraphs  Counting with Functions:}
\typeout{************************************************}
\paragraph[Counting with Functions:]{Counting with Functions:}\hypertarget{paragraphs-6}{}
 Many of the counting problems in this section might at first appear to be examples of counting \emph{functions}.  After all, when we try to count the number of ways to distribute cookies to kids, we are assigning each cookie to a kid, just like you assign elements of the domain of a function to elements in the codomain.  However, the number of ways to assign 7 cookies to 4 kids is \({10 \choose 7} = 120\), while the number of functions \(f: \{1,2,3,4,5,6,7\} \to \{a,b,c,d\}\) is \(4^7 = 16384\).  What is going on here? %
\par

    When we count functions, we consider the following two functions, for example, to be different:
    \begin{equation*}
      f = \twoline{1 \amp 2 \amp 3 \amp 4\amp 5 \amp 6 \amp 7}{a \amp b \amp c \amp c \amp c \amp c \amp c}  \qquad g = \twoline{1 \amp 2 \amp 3 \amp 4\amp 5 \amp 6 \amp 7}{b \amp a \amp c \amp c \amp c \amp c \amp c}.
    \end{equation*}
    But these two functions would correspond to the \emph{same} cookie distribution: kids \(a\) and \(b\) each get one cookie, kid \(c\) gets the rest (and none for kid \(d\)).
  %
\par
 The point: elements of the domain are distinguished, cookies are indistinguishable.  This is analagous to the distinction between permutations (like counting functions) and combinations (not).
  %
\typeout{************************************************}
\typeout{Exercises 2.5.1 Exercises}
\typeout{************************************************}
\subsection[Exercises]{Exercises}\label{exercises-8}
\begin{exerciselist}
\item[1.]\hypertarget{exercise-82}{}
A \emph{multiset} is a collection of objects, just like a set, but can contain an object more than once (the order of the elements still doesn't matter). For example, \(\{1,1, 2, 5, 5, 7\}\) is a multiset of size 6.
%
\leavevmode%
\begin{enumerate}[label=(\alph*)]
\item\hypertarget{li-571}{}
How many sets of size 5 can be made using the 10 numeric digits 0 through 9?
%
\item\hypertarget{li-572}{}
How many \emph{multi}sets of size 5 can be made using the 10 numeric digits 0 through 9?
%
\end{enumerate}
\par\smallskip
\item[2.]\hypertarget{exercise-83}{}
Each of the counting problems below can be solved with stars and bars. For each, say what outcome the diagram
\begin{equation*}
  ***|*||**|
\end{equation*}
represents, if there are the correct number of stars and bars for the problem. Otherwise, say why the diagram does not represent any outcome, and what a correct diagram would look like.
%
\leavevmode%
\begin{enumerate}[label=(\alph*)]
\item\hypertarget{li-575}{}
How many ways are there to select a handful of 6 jellybeans from a jar that contains 5 different flavors?
%
\item\hypertarget{li-576}{}
How many ways can you distribute 5 identical lollipops to 6 kids?
%
\item\hypertarget{li-577}{}
How many 6-letter words can you make using the 5 vowels?
%
\item\hypertarget{li-578}{}
How many solutions are there to the equation \(x_1 + x_2 + x_3 + x_4 = 6\).
%
\end{enumerate}
\par\smallskip
\item[3.]\hypertarget{exercise-84}{}
After gym class you are tasked with putting the 14 identical dodgeballs away into 5 bins.
%
\leavevmode%
\begin{enumerate}[label=(\alph*)]
\item\hypertarget{li-583}{}
How many ways can you do this if there are no restrictions?
%
\item\hypertarget{li-584}{}
How many ways can you do this if each bin must contain at least one dodgeball?
%
\end{enumerate}
\par\smallskip
\item[4.]\hypertarget{exercise-85}{}
How many integer solutions are there to the equation \(x + y + z = 8\)
for which
%
\leavevmode%
\begin{enumerate}[label=(\alph*)]
\item\hypertarget{li-587}{}\(x\), \(y\), and \(z\) are all positive?%
\item\hypertarget{li-588}{}\(x\), \(y\), and \(z\) are all non-negative?%
\item\hypertarget{li-589}{}\(x\), \(y\), and \(z\) are all greater than \(-3\).%
\end{enumerate}
\par\smallskip
\item[5.]\hypertarget{exercise-86}{}
Using the digits 2 through 8, find the number of different 5-digit numbers such that:
%
\leavevmode%
\begin{enumerate}[label=(\alph*)]
\item\hypertarget{li-593}{}
Digits cannot be repeated and must be written in increasing order.  For example, 23678 is okay, but 32678 is not.
%
\item\hypertarget{li-594}{}
Digits \emph{can} be repeated and must be written in \emph{non-decreasing} order.  For example, 24448 is okay, but 24484 is not.
%
\end{enumerate}
\par\smallskip
\item[6.]\hypertarget{exercise-87}{}
When playing Yahtzee, you roll five regular 6-sided dice. How many different outcomes are possible from a single roll? The order of the dice does not matter.
%
\par\smallskip
\item[7.]\hypertarget{exercise-88}{}
Your friend tells you she has 7 coins in her hand (just pennies, nickels, dimes and quarters). If you guess how many of each kind of coin she has, she will give them to you. If you guess randomly, what is the probability that you will be correct?
%
\par\smallskip
\item[8.]\hypertarget{exercise-89}{}
How many integer solutions to \(x_1 + x_2 + x_3 + x_4  = 25\) are there for which \(x_1 \ge 1\), \(x_2 \ge 2\), \(x_3 \ge 3\) and \(x_4 \ge 4\)?
%
\par\smallskip
\item[9.]\hypertarget{exercise-90}{}
Solve the three counting problems below. Then say why it makes sense that they all have the same answer. That is, say how you can interpret them as each other.
%
\leavevmode%
\begin{enumerate}[label=(\alph*)]
\item\hypertarget{li-597}{}
How many ways are there to distribute 8 cookies to 3 kids?
%
\item\hypertarget{li-598}{}
How many solutions in non-negative integers are there to \(x+y+z = 8\)?
%
\item\hypertarget{li-599}{}
How many different packs of 8 crayons can you make using crayons that come in red, blue and yellow?
%
\end{enumerate}
\par\smallskip
\item[10.]\hypertarget{exercise-91}{}
\emph{Conic}, your favorite math themed fast food drive-in offers 20 flavors which can be added to your soda. You have enough money to buy a large soda with 4 added flavors. How many different soda concoctions can you order if:
%
\leavevmode%
\begin{enumerate}[label=(\alph*)]
\item\hypertarget{li-600}{}
You refuse to use any of the flavors more than once?
%
\item\hypertarget{li-601}{}
You refuse repeats but care about the order the flavors are added?
%
\item\hypertarget{li-602}{}
You allow yourself multiple shots of the same flavor?
%
\item\hypertarget{li-603}{}
You allow yourself multiple shots, and care about the order the flavors are added?
%
\end{enumerate}
\par\smallskip
\end{exerciselist}
\typeout{************************************************}
\typeout{Section 2.6 Advanced Counting Using PIE}
\typeout{************************************************}
\section[Advanced Counting Using PIE]{Advanced Counting Using PIE}\label{sec_advPIE}
\typeout{************************************************}
\typeout{Introduction  }
\typeout{************************************************}
\begin{investigation}[]\label{investigation-11}

        You have 11 identical mini key-lime pies to give to 4 children. However, you don't want any kid to get more than 3 pies. How many ways can you distribute the pies?

      \leavevmode%
\begin{enumerate}
\item\hypertarget{li-608}{}
            How many ways are there to distribute the pies without any restriction?
          %
\item\hypertarget{li-609}{}
            Let's get rid of the ways that one or more kid gets too many pies. How many ways are there to distribute the pies if Al gets too many pies? What if Bruce gets too many? Or Cat? Or Dent?
          %
\item\hypertarget{li-610}{}
            What if two kids get too many pies? How many ways can this happen? Does it matter which two kids you pick to overfeed?
          %
\item\hypertarget{li-611}{}
            Is it possible that three kids get too many pies? If so, how many ways can this happen?
          %
\item\hypertarget{li-612}{}
            How should you combine all the numbers you found above to answer the original question?
          %
\end{enumerate}

%
\par

        Suppose now you have 13 pies and 7 children. No child can have more than 2 pies. How many ways can you distribute the pies?
      %
\end{investigation}

      Stars and bars allows us to count the number of ways to distribute 10 cookies to 3 kids and natural number solutions to \(x+y+z = 11\), for example. A relatively easy modification allows us to put a \emph{lower bound} restriction on these problems: perhaps each kid must get at least two cookies or \(x,y,z \ge 2\). This was done by first assigning each kid (or variable) 2 cookies (or units) and then distributing the rest using stars and bars.
    %
\par

      What if we wanted an \emph{upper bound} restriction? For example, we might insist that no kid gets more than 4 cookies or that \(x, y, z \le 4\). It turns out this is considerably harder, but still possible. The idea is to count all the distributions and then remove those that violate the condition. In other words, we must count the number of ways to distribute 11 cookies to 3 kids in which \emph{one or more} of the kids gets more than 4 cookies. For any particular kid, this is not a problem; we do this using stars and bars. But how to combine the number of ways for kid A, or B or C? We must use the PIE.
    %
\par

      The Principle of Inclusion/Exclusion (PIE) gives a method for finding the cardinality of the union of not necessarily disjoint sets. We saw in
      \hyperref[subsec_PIE]{Subsection~\ref{subsec_PIE}} how this works with three sets. To find how many things are in \emph{one or more} of the sets \(A\), \(B\), and \(C\), we should just add up the number of things in each of these sets. However, if there is any overlap among the sets, those elements are counted multiple times. So we subtract the things in each intersection of a pair of sets. But doing this removes elements which are in all three sets once too often, so we need to add it back in. In terms of cardinality of sets, we have
      \begin{equation*}
        |A \cup B \cup C| = |A| + |B| + |C| - |A \cap B| - |A \cap C| - |B \cap C| + |A\cap B \cap C|.
      \end{equation*}
    %
\begin{example}[]\label{example-46}

          Three kids, Alberto, Bernadette, and Carlos, decide to share 11 cookies. They wonder how many ways they could split the cookies up provided that none of them receive more than 4 cookies (someone receiving no cookies is for some reason acceptable to these kids).
        %
\par\medskip\noindent%
\textbf{Solution.}\quad 
          Without the ``no more than 4'' restriction, the answer would be \({13 \choose 2}\), using 11 stars and 2 bars (separating the three kids). Now count the number of ways that one or more of the kids violates the condition, i.e., gets at least 4 cookies.
        %
\par

          Let \(A\) be the set of outcomes in which Alberto gets more than 4 cookies. Let \(B\) be the set of outcomes in which Bernadette gets more than 4 cookies. Let \(C\) be the set of outcomes in which Carlos gets more than 4 cookies. We then are looking (for the sake of subtraction) for the size of the set \(A \cup B \cup C\). Using PIE, we must find the sizes of \(|A|\), \(|B|\), \(|C|\), \(|A\cap B|\) and so on. Here is what we find.
        %
\leavevmode%
\begin{itemize}[label=\textbullet]
\item{}\(|A| = {8 \choose 2}\). First give Alberto 5 cookies, then distribute the remaining 6 to the three kids without restrictions, using 6 stars and 2 bars.%
\item{}\(|B| = {8 \choose 2}\). Just like above, only now Bernadette gets 5 cookies at the start.%
\item{}\(|C| = {8 \choose 2}\). Carlos gets 5 cookies first.%
\item{}\(|A \cap B| = {3 \choose 2}\). Give Alberto and Bernadette 5 cookies each, leaving 1 (star) to distribute to the three kids (2 bars).%
\item{}\(|A \cap C| = {3 \choose 2}\). Alberto and Carlos get 5 cookies first.%
\item{}\(|B \cap C| = {3 \choose 2}\). Bernadette and Carlos get 5 cookies first.%
\item{}\(|A \cap B \cap C| = 0\). It is not possible for all three kids to get 4 or more cookies.%
\end{itemize}
\par

          Combining all of these we see
          \begin{equation*}
            |A \cup B \cup C| = {8 \choose 2} + {8 \choose 2} + {8 \choose 2} - {3 \choose 2} - {3 \choose 2} - {3 \choose 2} + 0 = 75.
          \end{equation*}
        %
\par

          Thus the answer to the original question is \({13 \choose 2} - 75 = 78 - 75 = 3\). This makes sense now that we see it. The only way to ensure that no kid gets more than 4 cookies is to give two kids 4 cookies and one kid 3; there are three choices for which kid that should be. We could have found the answer much quicker through this observation, but the point of the example is to illustrate that PIE works!
        %
\end{example}
\par

      For four or more sets, we do not write down a formula for PIE. Instead, we just think of the principle: add up all the elements in single sets, then subtract out things you counted twice (elements in the intersection of a \emph{pair} of sets), then add back in elements you removed too often (elements in the intersection of groups of three sets), then take back out elements you added back in too often (elements in the intersection of groups of four sets), then add back in, take back out, add back in, etc. This would be very difficult if it wasn't for the fact that in these problems, all the cardinalities of the single sets are equal, as are all the cardinalities of the intersections of two sets, and that of three sets, and so on. Thus we can group all of these together and multiply by how many different combinations of 1, 2, 3,
      \dots{} sets there are.
    %
\begin{example}[]\label{example-47}

          How many ways can you distribute 10 cookies to 4 kids so that no kid gets more than 2 cookies?
        %
\par\medskip\noindent%
\textbf{Solution.}\quad 
          To answer this, we will subtract all the outcomes in which a kid gets 3 or more cookies. How many outcomes are there like that? We can force kid A to eat 3 or more cookies by giving him 3 cookies before we start. Doing so reduces the problem to one in which we have 7 cookies to give to 4 kids without any restrictions. In that case, we have 7 stars (the 7 remaining cookies) and 3 bars (one less than the number of kids) so we can distribute the cookies in \({10 \choose 3}\) ways. Of course we could choose any one of the 4 kids to give too many cookies, so it would appear that there are \({4 \choose 1}{10 \choose 3}\) ways to distribute the cookies giving too many to one kid. But in fact, we have over counted.
        %
\par

          We must get rid of the outcomes in which two kids have too many cookies. There are \({4 \choose 2}\) ways to select 2 kids to give extra cookies. It takes 6 cookies to do this, leaving only 4 cookies. So we have 4 stars and still 3 bars. The remaining 4 cookies can thus be distributed in \({7 \choose 3}\) ways (for each of the \({4 \choose 2}\) choices of which 2 kids to over-feed).
        %
\par

          But now we have removed too much. We must add back in all the ways to give too many cookies to three kids. This uses 9 cookies, leaving only 1 to distribute to the 4 kids using stars and bars, which can be done in \({4 \choose 3}\) ways. We must consider this outcome for every possible choice of which three kids we over-feed, and there are \({4 \choose 3}\) ways of selecting that set of 3 kids.
        %
\par

          Next we would subtract all the ways to give four kids too many cookies, but in this case, that number is 0.
        %
\par

          All together we get that the number of ways to distribute 10 cookies to 4 kids without giving any kid more than 2 cookies is:
          \begin{equation*}
            {13 \choose 3} - \left[{4 \choose 1}{10 \choose 3} - {4 \choose 2}{7 \choose 3} + {4\choose 3}{4\choose 3}\right] = 286 - [480 - 210 + 16] = 0.
          \end{equation*}
        %
\par

          This makes sense: there is NO way to distribute 10 cookies to 4 kids and make sure that nobody gets more than 2. It is slightly surprising that
          \begin{equation*}
            {13 \choose 3} = \left[{4 \choose 1}{10 \choose 3} - {4 \choose 2}{7 \choose 3} + {4\choose 3}{4\choose 3}\right]
          \end{equation*}
          but since PIE works, this equality must hold.
        %
\end{example}
\par

      Just so you don't think that these problems always have easier solutions, consider the following example.
    %
\begin{example}[]\label{example-48}

          Earlier we counted the number of solutions to the equation
          \begin{equation*}
            x_1 + x_2 + x_3 + x_4 + x_5 = 13.
          \end{equation*}
        %
\par

          How many of those solutions have \(0 \le x_i \le 3\) for each \(x_i\)?
        %
\par\medskip\noindent%
\textbf{Solution.}\quad 
          We must subtract off the number of solutions in which one or more of the variables has a value greater than 3. We will need to use PIE because counting the number of solutions for which each of the five variables separately are greater than 3 counts solutions multiple times. Here is what we get:
        %
\leavevmode%
\begin{itemize}[label=\textbullet]
\item{}
              Total solutions: \({17 \choose 4}\).
            %
\item{}
              Solutions where \(x_1 > 3\): \({13 \choose 4}\). Give \(x_1\) 4 units first, then distribute the remaining 9 units to the 5 variables.
            %
\item{}
              Solutions where \(x_1 > 3\) and \(x_2 > 3\): \({9 \choose 4}\). After you give 4 units to \(x_1\) and another 4 to \(x_2\), you only have 5 units left to distribute.
            %
\item{}
              Solutions where \(x_1 > 3\), \(x_2 > 3\) and \(x_3 > 3\): \({5 \choose 4}\).
            %
\item{}
              Solutions where \(x_1 > 3\), \(x_2 > 3\), \(x_3 > 3\), and \(x_4 > 3\): 0.
            %
\end{itemize}
\par

          We also need to account for the fact that we could choose any of the five variables in the place of \(x_1\) above, any pair of variables in the place of \(x_1\) and \(x_2\) and so on. It is because of this that the double counting occurs, so we need to use PIE. All together we have that the number of solutions with \(0 \le x_i \le 3\) is
          \begin{equation*}
            {17 \choose 4} - \left[{5\choose 1}{13 \choose 4} - {5 \choose 2}{9 \choose 4} + {5 \choose 3}{5 \choose 4}\right] = 15.
          \end{equation*}
        %
\end{example}
\typeout{************************************************}
\typeout{Subsection 2.6.1 Counting Derangements}
\typeout{************************************************}
\subsection[Counting Derangements]{Counting Derangements}\label{subsec_derangements}
\begin{investigation}[]\label{investigation-12}

        For your senior prank, you decide to switch the nameplates on your favorite 5 professors' doors. So that none of them feel left out, you want to make sure that all of the nameplates end up on the wrong door. How many ways can this be accomplished?
      %
\end{investigation}

      The advanced use of PIE has applications beyond stars and bars. A \emph{derangement}\index{derangement} of \(n\) elements \(\{1,2,3,\ldots, n\}\) is a permutation in which no element is fixed. For example, there are \(6\) permutations of the three elements \(\{1,2,3\}\):
      \begin{equation*}
        123 ~~ 132 ~~ 213 ~~ 231 ~~ 312 ~~ 321.
      \end{equation*}
      but most of these have one or more elements fixed: \(123\) has all three elements fixed, \(132\) has the first element fixed (1 is in its original first position), and so on. In fact, the only derangements of three elements are
      \begin{equation*}
        231 \mbox{ and } 312.
      \end{equation*}
    %
\par

      If we go up to 4 elements, there are 24 permutations (because we have 4 choices for the first element, 3 choices for the second, 2 choices for the third leaving only 1 choice for the last). How many of these are derangements? If you list out all 24 permutations and eliminate those which are not derangements, you will be left with just 9 derangements. Let's see how we can get that number using PIE.
    %
\begin{example}[]\label{example-49}

          How many derangements are there of 4 elements?
        %
\par\medskip\noindent%
\textbf{Solution.}\quad 
          We count all permutations, and subtract those which are not derangements. There are \(4! = 24\) permutations of 4 elements. Now for a permutation to \emph{not} be a derangement, at least one of the 4 elements must be fixed. There are \({4 \choose 1}\) choices for which single element we fix. Once fixed, we need to find a permutation of the other three elements. There are \(3!\) permutations on 3 elements. But now we have counted too many non-derangements, so we must subtract those permutations which fix two elements. There are \({4 \choose 2}\) choices for which two elements we fix, and then for each pair, \(2!\) permutations of the remaining elements. But this subtracts too many, so add back in permutations which fix 3 elements, all \({4 \choose 3}1!\) of them. Finally subtract the \({4 \choose 4}0!\) permutations (recall \(0! = 1\)) which fix all four elements. All together we get that the number of derangements of 4 elements is:
          \begin{equation*}
            4! - \left[{4 \choose 1}3! - {4 \choose 2}2! + {4 \choose 3} 1! - {4 \choose 4}0!\right] = 24 - 15 = 9.
          \end{equation*}
        %
\end{example}
\par

      Of course we can use a similar formula to count the derangements on any number of elements. However, the more elements we have, the longer the formula gets. Here is another example.
    %
\begin{example}[]\label{example-50}

          Five gentlemen attend a party, leaving their hats at the door. At the end of the party, they hastily grab hats on their way out. How many different ways could this happen so that none of the gentlemen leave with their own hat?
        %
\par\medskip\noindent%
\textbf{Solution.}\quad 
          We are counting derangements on 5 elements. There are \(5!\) ways for the gentlemen to grab hats in any order - but many of these permutations will result in someone getting their own hat. So we subtract all the ways in which one or more of the men get their own hat. In other words, we subtract the non-derangements. Doing so requires PIE. Thus the answer is:
          \begin{equation*}
            5! - \left[{5 \choose 1}4! - {5 \choose 2}3! + {5 \choose 3}2! - {5 \choose 4}1! + {5 \choose 5}0!\right].
          \end{equation*}
        %
\end{example}
\typeout{************************************************}
\typeout{Subsection 2.6.2 Counting Functions}
\typeout{************************************************}
\subsection[Counting Functions]{Counting Functions}\label{subsection-20}
\begin{investigation}[]\label{investigation-13}

    \leavevmode%
\begin{itemize}[label=\textbullet]
\item{}
    Consider all functions \(f: \{1,2,3,4,5\} \to \{1,2,3,4,5\}\). How many functions are there all together? How many of those are injective? Remember, a function is an injection if every input goes to a different output.
    %
\item{}
    Consider all functions \(f: \{1,2,3,4,5\} \to \{1,2,3,4,5\}\). How many of the \emph{injections} have the property that \(f(x) \ne x\) for any \(x \in \{1,2,3,4,5\}\)?
    %
\par

    Your friend claims that the answer is:
    \begin{equation*}
      5! - \left[ {5\choose 1}4! - {5 \choose 2}3! + {5\choose 3}2! - {5 \choose 4}1! + {5\choose 5}0! \right].
    \end{equation*}
    %
\par

    Explain why this is correct.
    %
\item{}
    Recall that a \emph{surjection} is a function for which every element of the codomain is in the range. How many of the functions \(f: \{1,2,3,4,5\} \to \{1,2,3,4,5\}\) are surjective? Use PIE!
    %
\end{itemize}

    %
\end{investigation}

      We have seen throughout this chapter that many counting questions can be rephrased as questions about counting functions with certain properties.  This is reasonable since many counting questions can be thought of as counting the number of ways to assign elements from one set to elements of another.
    %
\begin{example}[]\label{example-51}

    You decide to give away your video game collection so to better spend your time studying advance mathematics. How many ways can you do this, provided:
      \leavevmode%
\begin{enumerate}
\item\hypertarget{li-628}{}You want to distribute your 3 different PS4 games among 5 friends, so that no friend gets more than one game?%
\item\hypertarget{li-629}{}You want to distribute your 8 different 3DS games among 5 friends?%
\item\hypertarget{li-630}{}You want to distribute your 8 different SNES games among 5 friends, so that each friend gets at least one game?%
\end{enumerate}

      In each case, model the counting question as a function counting question.
    %
\par\medskip\noindent%
\textbf{Solution.}\quad 
      \leavevmode%
\begin{enumerate}
\item\hypertarget{li-631}{}We must use the three games (call them 1, 2, 3) as the domain and the 5 friends (a,b,c,d,e) as the codomain (otherwise the function would not be defined for the whole domain when a friend didn't get any game).  So how many functions are there with domain \(\{1,2,3\}\) and codomain \(\{a,b,c,d,e\}\)?  The answer to this is \(5^3=125\), since we can assign any of 5 elements to be the image of 1, any of 5 elements to be the image of 2 and any of 5 elements to be the image of 3.%
\par

          But this is not the correct answer to our counting problem, because one of these functions is \(f= \twoline{1\amp 2\amp 3}{a\amp a\amp a}\); one friend can get more than one game.  What we really need to do is count \emph{injective} functions.  This gives \(P(5,3) = 60\) functions, which is the answer to our counting question.
        %
\item\hypertarget{li-632}{}
        Again, we need to use the 8 games as the domain and the 5 friends as the codomain.  We are counting all functions, so the number of ways to distribute the games is \(5^8\).
      %
\item\hypertarget{li-633}{}
        This question is harder.  Use the games as the domain and friends as the codomain (otherwise an element of the domain would have more than one image, which is impossible).  To ensure that every friend gets at least one game means that every element of the codomain is in the range.  In other words, we are looking for \emph{surjective} functions. How do you count those??
      %
\end{enumerate}


%
\end{example}
\par

      In \hyperref[ex_counting-functions-all]{Example~\ref{ex_counting-functions-all}} we saw how to count all functions (using the multiplicative principle) and in \hyperref[ex_counting-functions-injective]{Example~\ref{ex_counting-functions-injective}} we learned how to count injective functions (using permutations).  Surjective functions are not as easily counted (unless the size of the domain is smaller than the codomain, in which case there are none).
    %
\par

    The idea is to count the functions which are \emph{not} surjective, and then subtract that from the total number of functions. This works very well when the codomain has two elements in it:
    %
\begin{example}[]\label{example-52}

    How many functions \(f: \{1,2,3,4,5\} \to \{a,b\}\) are surjective?
    %
\par\medskip\noindent%
\textbf{Solution.}\quad 
    There are \(2^5\) functions all together, two choices for where to send each of the 5 elements of the domain. Now of these, the functions which are \emph{not} surjective must exclude one or more elements of the codomain from the range. So first, consider functions for which \(a\) is not in the range. This can only happen one way: everything gets sent to \(b\). Alternatively, we could exclude \(b\) from the range. Then everything gets sent to \(a\), so there is only one function like this. These are the only ways in which a function could not be surjective (no function excludes both \(a\) and \(b\) from the range) so there are exactly \(2^5 - 2\) surjective functions.
    %
\end{example}
\par

    When there are three elements in the codomain, there are now three choices for a single element to exclude from the range. Additionally, we could pick pairs of two elements to exclude from the range, and we must make sure we don't over count these. It's PIE time!
    %
\begin{example}[]\label{example-53}

    How many functions \(f: \{1,2,3,4,5\} \to \{a,b,c\}\) are surjective?
    %
\par\medskip\noindent%
\textbf{Solution.}\quad 
    Again start with the total number of functions: \(3^5\) (as each of the five elements of the domain can go to any of three elements of the codomain). Now we count the functions which are \emph{not} surjective.
    %
\par

    Start by excluding \(a\) from the range. Then we have two choices (\(b\) or \(c\)) for where to send each of the five elements of the domain. Thus there are \(2^5\) functions which exclude \(a\) from the range. Similarly, there are \(2^5\) functions which exclude \(b\), and another \(2^5\) which exclude \(c\). Now have we counted all functions which are not surjective? Yes, but in fact, we have counted some multiple times. For example, the function which sends everything to \(c\) was one of the \(2^5\) functions we counted when we excluded \(a\) from the range, and also one of the \(2^5\) functions we counted when we excluded \(b\) from the range. We must subtract out all the functions which specifically exclude two elements from the range. There is 1 function when we exclude \(a\) and \(b\) (everything goes to \(c\)), one function when we exclude \(a\) and \(c\), and one function when we exclude \(b\) and \(c\).
    %
\par

    We are using PIE: to count the functions which are not surjective, we added up the functions which exclude \(a\), \(b\), and \(c\) separately, then subtracted the functions which exclude pairs of elements. We would then add back in the functions which exclude groups of three elements, except that there are no such functions. We find that the number of functions which are \emph{not} surjective is
    \begin{equation*}
      2^5 + 2^5 + 2^5 - 1 - 1 - 1 + 0.
    \end{equation*}
    %
\par

    Perhaps a more descriptive way to write this is
    \begin{equation*}
      {3 \choose 1}2^5 - {3 \choose 2}1^5 + {3 \choose 3}0^5.
    \end{equation*}
    since each of the \(2^5\)'s was the result of choosing 1 of the 3 elements of the codomain to exclude from the range, each of the three \(1^5\)'s was the result of choosing 2 of the 3 elements of the codomain to exclude. Writing \(1^5\) instead of 1 makes sense too: we have 1 choice of were to send each of the 5 elements of the domain.
    %
\par

    Now we can finally count the number of surjective functions:
    \begin{equation*}
      3^5 - \left[{3 \choose 1}2^5 - {3 \choose 2}1^5\right] = 150.
    \end{equation*}
    %
\end{example}
\par

    You might worry that to count surjective functions when the codomain is larger than 3 elements would be too tedious. We need to use PIE but with more than 3 sets the formula for PIE is very long. However, we have lucked out. As we saw in the example above, the number of functions which exclude a single element from the range is the same no matter which single element is excluded. Similarly, the number of functions which exclude a pair of elements will be the same for every pair. With larger codomains,
    we will see the same behavior with groups of 3, 4, and more elements excluded. So instead of adding/subtracting each of these, we can simply add or subtract all of them at once, if you know how many there are. This works just like it did in for the other types of counting questions in this section, only now the size of the various combinations of sets is a number raised to a power, as opposed to a binomial coefficient or factorial. Here's what happens with \(4\) and \(5\) elements in the codomain.
    %
\begin{example}[]\label{example-54}
\leavevmode%
\begin{enumerate}
\item\hypertarget{li-634}{}
    How many functions \(f: \{1,2,3,4,5\} \to \{a,b,c,d\}\) are surjective?
    %
\item\hypertarget{li-635}{}
    How many functions \(f: \{1,2,3,4,5\} \to \{a,b,c,d,e\}\) are surjective?
    %
\end{enumerate}
\par\medskip\noindent%
\textbf{Solution.}\quad \leavevmode%
\begin{enumerate}
\item\hypertarget{li-636}{}
    There are \(4^5\) functions all together; we will subtract the functions which are not surjective.  We could exclude any one of the four elements of the codomain, and doing so will leave us with \(3^5\) functions for each excluded element.  This counts too many so we subtract the functions which exclude two of the four elements of the codomain, each pair giving \(2^5\) functions.  But this excludes too many, so we add back in the functions which exclude three of the four elements of the codomain, each triple giving \(1^5\) function.  There are \({4 \choose 1}\) groups of functions excluding a single element, \({4 \choose 2}\) groups of functions excluding a pair of elements, and \({4 \choose 3}\) groups of functions excluding a triple of elements.  This means that the number of functions which are \emph{not} surjective is:
    \begin{equation*}
      {4 \choose 1}3^5 - {4 \choose 2}2^5 + {4 \choose 3}1^5.
    \end{equation*}
    We can now say that the number of functions which are surjective is:
    \begin{equation*}
      4^5 - \left[{4 \choose 1}3^5 - {4 \choose 2}2^5 + {4 \choose 3}1^5\right].
    \end{equation*}
    %
\item\hypertarget{li-637}{}
    The number of surjective functions is:
    \begin{equation*}
      5^5 - \left[{5 \choose 1}4^5 - {5 \choose 2}3^5 + {5 \choose 3}2^5 - {5 \choose 4}1^5\right].
    \end{equation*}
    We took the total number of functions \(5^5\) and subtracted all that were not surjective.  There were \({5 \choose 1}\) ways to select a single element from the codomain to exclude from the range, and for each there were \(4^5\) functions.  But this double counts, so we use PIE and subtract functions excluding two elements from the range: there are \({5 \choose 2}\) choices for the two elements to exclude, and for each pair, \(3^5\) functions.  This takes out too many functions, so we add back in functions which exclude 3 elements from the range: \({5 \choose 3}\) choices for which three to exclude, and then \(2^5\) functions for each choice of elements.  Finally we take back out the 1 function which excludes 4 elements for each of the \({5 \choose 4}\) choices of 4 elements.
    %
\end{enumerate}
\end{example}
\par

    We have seen that counting surjective functions is another nice example of the advanced use of the Principle of Inclusion/Exclusion. Also, counting injective functions turns out to be equivalent to permutations, and counting all functions has a solution akin to those counting problems where order matters but repeats are allowed (like counting the number of words you can make from a given set of letters).
    %
\par

    These are not just a few more examples of the techniques we have developed in this chapter. Quite the opposite: everything we have learned in this chapter are examples of \emph{counting functions}!
    %
\begin{example}[]\label{example-55}

    How many 5-letter words can you make using the eight letters \(a\) through \(h\)? How many contain no repeated letters?
    %
\par\medskip\noindent%
\textbf{Solution.}\quad 
    By now it should be no surprise that there are \(8^5\) words, and \(P(8,5)\) words without repeated letters. The new piece here is that we are actually counting functions. For the first problem, we are counting all functions from \(\{1,2,\ldots, 5\}\) to \(\{a,b,\ldots, h\}\). The numbers in the domain represent the \emph{position} of the letter in the word, the codomain represents the letter that could be assigned to that position. If we ask for no repeated letters, we are asking for injective functions.
    %
\par

    If \(A\) and \(B\) are \emph{any} sets with \(|A| = 5\) and \(|B| = 8\), then the number of functions \(f: A \to B\) is \(8^5\) and the number of injections is \(P(8,5)\). So if you can represent your counting problem as a function counting problem, most of the work is done.
    %
\end{example}
\begin{example}[]\label{example-56}

    How many subsets are there of \(\{1,2,\ldots, 9\}\)? How many 9-bit strings are there (of any weight)?
    %
\par\medskip\noindent%
\textbf{Solution.}\quad 
    We saw in \hyperref[sec_counting-binom]{Section~\ref{sec_counting-binom}} that the answer to both these questions is \(2^9\), as we can say yes or no (or 0 or 1) to each of the 9 elements in the set (positions in the bit-string). But \(2^9\) also looks like the answer you get from counting functions. In fact, if you count all functions \(f: A \to B\) with \(|A| = 9\) and \(|B| = 2\), this is exactly what you get.
    %
\par

    This makes sense! Let \(A = \{1,2,\ldots, 9\}\) and \(B = \{y, n\}\). We are assigning each element of the set either a yes or a no. Or in the language of bit-strings, we would take the 9 positions in the bit string as our domain and the set \(\{0,1\}\) as the codomain.
    %
\end{example}
\par

    So far we have not used a function as a model for binomial coefficients (combinations). Think for a moment about the relationship between combinations and permutations, say specifically \({9 \choose 3}\) and \(P(9,3)\). We \emph{do} have a function model for \(P(9,3)\). This is the number of \emph{injective} functions from a set of size 3 (say \(\{1,2,3\}\) to a set of size 9 (say \(\{1,2,\ldots, 9\}\)) since there are 9 choices for where to send the first element of the domain, then only 8 choices for the second, and 7 choices for the third. For example, the function might look like this:
    \begin{equation*}
      f(1) = 5 \qquad f(2) = 8 \qquad f(3) = 4.
    \end{equation*}
    %
\par

    This is a different function from:
    \begin{equation*}
      f(1) = 4 \qquad f(2) = 5 \qquad f(3) = 8.
    \end{equation*}
    %
\par

    Now \(P(9,3)\) counts these as different outcomes correctly, but \({9\choose 3}\) will count these (among others) as just one outcome. In fact, in terms of functions \({9 \choose 3}\) just counts the number of different ranges possible of injective functions. This should not be a surprise since binomial coefficients counts subsets, and the range is a possible subset of the codomain.\footnote{A more mathematically sophisticated interpretation of combinations is that we are defining two injective functions to be \emph{equivalent} if they have the same range, and then counting the number of equivalence classes under this notion of equivalence.\label{fn-4}}
    %
\par

    While it is possible to interpret combinations as functions, perhaps the better advice is to instead use combinations (or stars and bars) when functions are not quite the right way to interpret the counting question.
    %
\typeout{************************************************}
\typeout{Exercises 2.6.3 Exercises}
\typeout{************************************************}
\subsection[Exercises]{Exercises}\label{exercises-9}
\begin{exerciselist}
\item[1.]\hypertarget{exercise-92}{}
            The dollar menu at your favorite tax-free fast food restaurant has 7 items. You have
            \textdollar{}10 to spend. How many different meals can you buy if you spend all your money and:
          %
\leavevmode%
\begin{enumerate}[label=(\alph*)]
\item\hypertarget{li-638}{}
                Purchase at least one of each item.
              %
\item\hypertarget{li-639}{}
                Possibly skip some items.
              %
\item\hypertarget{li-640}{}
                Don't get more than 2 of any particular item.
              %
\end{enumerate}
\par\smallskip
\item[2.]\hypertarget{exercise-93}{}
            After another gym class you are tasked with putting the 14 identical dodgeballs away into 5 bins. This time, no bin can hold more than 6 balls. How many ways can you clean up?
          %
\par\smallskip
\item[3.]\hypertarget{exercise-94}{}
            Consider the equation \(x_1 + x_2 + x_3 + x_4 = 15\). How many solutions are there with \(2 \le x_i \le 5\) for all \(i \in \{1,2,3,4\}\)?
          %
\par\smallskip
\item[4.]\hypertarget{exercise-95}{}
            Suppose you planned on giving 7 gold stars to some of the 13 star students in your class. Each student can receive at most one star. How many ways can you do this? Use PIE, and also an easier method, and compare your results.
          %
\par\smallskip
\item[5.]\hypertarget{exercise-96}{}
            Based on the previous question, give a combinatorial proof for the identity:
            \begin{equation*}
              {n \choose k} = {n+k-1 \choose k} - \sum_{j=1}^n (-1)^{j+1}{n \choose j}{n+k-(2j+1) \choose k}.
            \end{equation*}
          %
\par\smallskip
\item[6.]\hypertarget{exercise-97}{}
            Illustrate how the counting of derangements works by writing all permutations of \(\{1,2,3,4\}\) and the crossing out those which are not derangements. Keep track of the permutations you cross out more than once, using PIE.
          %
\par\smallskip
\item[7.]\hypertarget{exercise-98}{}
            Ten ladies of a certain age drop off their red hats at the hat check of a museum. As they are leaving, the hat check attendant gives the hats back randomly. In how many ways can exactly six of the ladies receive their own hat (and the other four not)?
          %
\par\smallskip
\item[8.]\hypertarget{exercise-99}{}
      Consider functions \(f: \{1,2,3,4\} \to \{a,b,c,d,e,f\}\).
      How many functions have the property that \(f(1) \ne a\) or \(f(2) \ne b\), or both?
      %
\par\smallskip
\item[9.]\hypertarget{exercise-100}{}
      Consider sets \(A\) and \(B\) with \(|A| = 10\) and \(|B| = 5\). How many functions \(f: A \to B\) are surjective?
      %
\par\smallskip
\item[10.]\hypertarget{exercise-101}{}
      Let \(A = \{1,2,3,4,5\}\). How many injective functions \(f:A \to A\) have the property that for each \(x \in A\), \(f(x) \ne x\)?
      %
\par\smallskip
\end{exerciselist}
\typeout{************************************************}
\typeout{Section 2.7 Chapter Summary}
\typeout{************************************************}
\section[Chapter Summary]{Chapter Summary}\label{sec_count-conc}
\typeout{************************************************}
\typeout{Introduction  }
\typeout{************************************************}
\begin{investigation}[]\label{investigation-14}

        Suppose you have a huge box of animal crackers containing plenty of each of 10 different animals. For the counting questions below, carefully examine their similarities and differences, and then give an answer. The answers are all one of the following:

        \leavevmode%
\begin{itemize}[label=\textbullet]
\item{}\(P(10,6)\)%
\item{}\({10 \choose 6}\)%
\item{}\(10^6\)%
\item{}\({15 \choose 9}.\)%
\end{itemize}


        \leavevmode%
\begin{enumerate}
\item\hypertarget{li-648}{}
              How many animal parades containing 6 crackers can you line up?
            %
\item\hypertarget{li-649}{}
              How many animal parades of 6 crackers can you line up so that the animals appear in alphabetical order?
            %
\item\hypertarget{li-650}{}
              How many ways could you line up 6 different animals in alphabetical order?
            %
\item\hypertarget{li-651}{}
              How many ways could you line up 6 different animals if they can come in any order?
            %
\item\hypertarget{li-652}{}
              How many ways could you give 6 children one animal cracker each?
            %
\item\hypertarget{li-653}{}
              How many ways could you give 6 children one animal cracker each so that no two kids get the same animal?
            %
\item\hypertarget{li-654}{}
              How many ways could you give out 6 giraffes to 10 kids?
            %
\item\hypertarget{li-655}{}
              Write a question about giving animal crackers to kids that has the answer \({10\choose 6}\).
            %
\end{enumerate}

      %
\end{investigation}

      With all the different counting techniques we have mastered in this last chapter, it might be difficult to know when to apply which technique. Indeed, it is very easy to get mixed up and use the wrong counting method for a given problem. You get better with practice. As you practice you start to notice some trends that can help you distinguish between types of counting problems. Here are some suggestions that you might find helpful when deciding how to tackle a counting problem and checking whether your solution is correct.
    %
\leavevmode%
\begin{itemize}[label=\textbullet]
\item{}
          Remember that you are counting the number of items in some \emph{list of outcomes}. Write down part of this list. Write down an element in the middle of the list \textendash{} how are you deciding whether your element really is in the list. Could you get this element more than once using your proposed answer?
        %
\item{}
          If generating an element on the list involves selecting something (for example, picking a letter or picking a position to put a letter, etc), can the things you select be repeated? Remember, permutations and combinations select objects from a set \emph{without} repeats.
        %
\item{}
          Does order matter? Be careful here and be sure you know what your answer really means. We usually say that order matters when you get different outcomes when the same objects are selected in different orders. Combinations and ``Stars \& Bars'' are used when order \emph{does not} matter.
        %
\item{}
          There are four possibilities when it comes to order and repeats. If order matters and repeats are allowed, the answer will look like \(n^k\). If order matters and repeats are not allowed, we have \(P(n,k)\). If order doesn't matter and repeats are allowed, use stars and bars. If order doesn't matter and repeats are not allowed, use \({n\choose k}\). But be careful: this only applies when you are selecting things, and you should make sure you know exactly what you are selecting before determining which case you are in.
        %
\item{}
          Think about how you would represent your counting problem in terms of sets or functions. We know how to count different sorts of sets and different types of functions.
        %
\item{}
          As we saw with combinatorial proofs, you can often solve a counting problem in more than one way. Do that, and compare your numerical answers. If they don't match, something is amiss.
        %
\end{itemize}
\par

      While we have covered many counting techniques, we have really only scratched the surface of the large subject of \emph{enumerative combinatorics}. There are mathematicians doing original research in this area even as you read this. Counting can be really hard.
    %
\par

      In the next chapter, we will approach counting questions from a very different direction, and in doing so, answer infinitely many counting questions at the same time. We will create \emph{sequences} of answers to related questions.
    %
\typeout{************************************************}
\typeout{Exercises 2.7.1 Exercises}
\typeout{************************************************}
\subsubsection[Exercises]{Exercises}\label{exercises-10}
\begin{exerciselist}
\item[1.]\hypertarget{exercise-102}{}
            You have 9 presents to give to your 4 kids. How many ways can this be done if:
          %
\leavevmode%
\begin{enumerate}[label=(\alph*)]
\item\hypertarget{li-662}{}
                The presents are identical, and each kid gets at least one present?
              %
\item\hypertarget{li-663}{}
                The presents are identical, and some kids might get no presents?
              %
\item\hypertarget{li-664}{}
                The presents are unique, and some kids might get no presents?
              %
\item\hypertarget{li-665}{}
                The presents are unique and each kid gets at least one present?
              %
\end{enumerate}
\par\smallskip
\item[2.]\hypertarget{exercise-103}{}
            For each of the following counting problems, say whether the answer is
          %
\leavevmode%
\begin{itemize}[label=\textbullet]
\item{}\({10\choose 4}\)%
\item{}\(P(10,4)\)%
\item{}Neither%
\end{itemize}
\par

            If you answer is ``Neither,'' say what the answer should be instead.
          %
\leavevmode%
\begin{enumerate}[label=(\alph*)]
\item\hypertarget{li-673}{}
                How many shortest lattice paths are there from \((0,0)\) to \((10,4)\)?
              %
\item\hypertarget{li-674}{}
                If you have 10 bow ties\index{bow ties}, and you want to select 4 of them for next week, how many choices do you have?
              %
\item\hypertarget{li-675}{}
                Suppose you have 10 bow ties and you will wear one on each of the next 4 days. How many choices do you have?
              %
\item\hypertarget{li-676}{}
                If you want to wear 4 of your 10 bow ties next week (Monday through Sunday), how many ways can this be accomplished?
              %
\item\hypertarget{li-677}{}
                Out of a group of 10 classmates, how many ways can you rank your top 4 friends?
              %
\item\hypertarget{li-678}{}
                If 10 students come to their professor's office but only 4 can fit at a time, how different combinations of 4 students can see the prof first?
              %
\item\hypertarget{li-679}{}
                How many 4 letter words can be made from the first 10 letters of the alphabet?
              %
\item\hypertarget{li-680}{}
                How many ways can you make the word ``cake'' from the first 10 letters of the alphabet?
              %
\item\hypertarget{li-681}{}
                How many ways are there to distribute 10 apples among 4 children?
              %
\item\hypertarget{li-682}{}
                If you have 10 kids (and live in a shoe) and 4 types of cereal, how many ways can your kids eat breakfast?
              %
\item\hypertarget{li-683}{}
                How many ways can you arrange exactly 4 ones in a string of 10 binary digits?
              %
\item\hypertarget{li-684}{}
                You want to select 4 single digit numbers as your lotto picks. How many choices do you have?
              %
\item\hypertarget{li-685}{}
                10 kids want ice-cream. You have 4 varieties. How many ways are there to give the kids as much ice-cream as they want?
              %
\item\hypertarget{li-686}{}
                How many 1-1 functions are there from \(\{1,2,\ldots, 10\}\) to \(\{a,b,c,d\}\)?
              %
\item\hypertarget{li-687}{}
                How many surjective functions are there from \(\{1,2,\ldots, 10\}\) to \(\{a,b,c,d\}\)?
              %
\item\hypertarget{li-688}{}
                Each of your 10 bow ties match 4 pairs of suspenders. How many outfits can you make?
              %
\item\hypertarget{li-689}{}
                After the party, the 10 kids each choose one of 4 party-favors. How many outcomes?
              %
\item\hypertarget{li-690}{}
                How many 6-elements subsets are there of the set \(\{1,2,\ldots, 10\}\)
              %
\item\hypertarget{li-691}{}
                How many ways can you split up 11 kids into 5 teams?
              %
\item\hypertarget{li-692}{}
                How many solutions are there to \(x_1 + x_2 + \cdots + x_5 = 6\) where each \(x_i\) is non-negative?
              %
\item\hypertarget{li-693}{}
                Your band goes on tour. There are 10 cities within driving distance, but only enough time to play 4 of them. How many choices do you have for the cities on your tour?
              %
\item\hypertarget{li-694}{}
                In how many different ways can you play the 4 cities you choose?
              %
\item\hypertarget{li-695}{}
                Out of the 10 breakfast cereals available, you want to have 4 bowls. How many ways can you do this?
              %
\item\hypertarget{li-696}{}
                There are 10 types of cookies available. You want to make a 4 cookie stack. How many different stacks can you make?
              %
\item\hypertarget{li-697}{}
                From you home at (0,0) you want to go to either the donut shop at (5,4) or the one at (3,6). How many paths could you take?
              %
\item\hypertarget{li-698}{}
                How many 10-digit numbers do not contain a sub-string of 4 repeated digits?
              %
\end{enumerate}
\par\smallskip
\item[3.]\hypertarget{exercise-104}{}
            Recall, you own 3 regular ties and 5 bow ties\index{bow ties}. You realize that it would be okay to wear more than two ties to your clown college interview.
          %
\leavevmode%
\begin{enumerate}[label=(\alph*)]
\item\hypertarget{li-725}{}
                You must select some of your ties to wear. Everything is okay, from no ties up to all ties. How many choices do you have?
              %
\item\hypertarget{li-726}{}
                If you want to wear at least one regular tie and one bow tie, but are willing to wear up to all your ties, how many choices do you have for which ties to wear?
              %
\item\hypertarget{li-727}{}
                How many choices do you have if you wear exactly 2 of the 3 regular ties and 3 of the 5 bow ties?
              %
\item\hypertarget{li-728}{}
                Once you have selected 2 regular and 3 bow ties, in how many orders could you put the ties on, assuming you must have one of the three bow ties on top?
              %
\end{enumerate}
\par\smallskip
\item[4.]\hypertarget{exercise-105}{}
            Give a counting question where the answer is \(8\cdot 3 \cdot 3 \cdot 5\). Give another question where the answer is \(8 + 3 + 3 + 5\).
          %
\par\smallskip
\item[5.]\hypertarget{exercise-106}{}
            Consider five digit numbers \(\alpha = a_1a_2a_3a_4a_5\), with each digit from the set \(\{1,2,3,4\}\).
          %
\leavevmode%
\begin{enumerate}[label=(\alph*)]
\item\hypertarget{li-733}{}
                How many such numbers are there?
              %
\item\hypertarget{li-734}{}
                How many such numbers are there for which the \emph{sum} of the digits is even?
              %
\item\hypertarget{li-735}{}
                How many such numbers contain more even digits than odd digits?
              %
\end{enumerate}
\par\smallskip
\item[6.]\hypertarget{exercise-107}{}
            Let \(A\) and \(B\) be finite sets. Explain, in words, why \(|A \cup B| \le |A| + |B|\) and why \(|A \cup B| = |A| + |B| - |A \cap B|\).
          %
\par\smallskip
\item[7.]\hypertarget{exercise-108}{}
            For how many \(n \in \{1,2, \ldots, 500\}\) is \(n\) a multiple of one or more of 5, 6, or 7?
          %
\par\smallskip
\item[8.]\hypertarget{exercise-109}{}
            In a recent small survey of airline passengers, 25 said they had flown American in the last year, 30 had flown Jet Blue, and 20 had flown Continental. Of those, 10 reported they had flown on American and Jet Blue, 12 had flown on Jet Blue and Continental, and 7 had flown on American and Continental. 5 passengers had flown on all three airlines.
          %
\par

            How many passengers were surveyed? (Assume the results above make up the entire survey.)
          %
\par\smallskip
\item[9.]\hypertarget{exercise-110}{}
            Recall, by \(8\)-bit strings, we mean strings of binary digits, of length 8.
          %
\leavevmode%
\begin{enumerate}[label=(\alph*)]
\item\hypertarget{li-739}{}
                How many \(8\)-bit strings are there total?
              %
\item\hypertarget{li-740}{}
                How many \(8\)-bit strings have weight 5?
              %
\item\hypertarget{li-741}{}
                How many subsets of the set \(\{a,b,c,d,e,f,g,h\}\) contain exactly 5 elements?
              %
\item\hypertarget{li-742}{}
                Explain why your answers to parts (b) and (c) are the same. Why are these questions equivalent?
              %
\end{enumerate}
\par\smallskip
\item[10.]\hypertarget{exercise-111}{}
            What is the coefficient of \(x^{10}\) in the expansion of \((x+1)^{13} + x^2(x+1)^{17}\)?
          %
\par\smallskip
\item[11.]\hypertarget{exercise-112}{}
            How many 8-letter words contain exactly 5 vowels (a,e,i,o,u)? What if repeated letters were not allowed?
          %
\par\smallskip
\item[12.]\hypertarget{exercise-113}{}
            For each of the following, find the number of shortest lattice paths from \((0,0)\) to \((8,8)\) which:
          %
\leavevmode%
\begin{enumerate}[label=(\alph*)]
\item\hypertarget{li-747}{}
                pass through the point \((2,3)\).
              %
\item\hypertarget{li-748}{}
                avoid (do not pass through) the point \((7,5)\).
              %
\item\hypertarget{li-749}{}
                either pass through \((2,3)\) or \((5,7)\) (or both).
              %
\end{enumerate}
\par\smallskip
\item[13.]\hypertarget{exercise-114}{}
            You live in Grid-Town on the corner of 2nd and 3rd, and work in a building on the corner of 10th and 13th. How many routes are there which take you from home to work and then back home, but by a different route?
          %
\par\smallskip
\item[14.]\hypertarget{exercise-115}{}
            Give an example of a problem for which \(P(n,k)\) is the solution. Give another example of a problem for which \({n\choose k}\) is the solution.
          %
\par\smallskip
\item[15.]\hypertarget{exercise-116}{}
            How many 10-bit strings start with \(111\) or end with \(101\) or both?
          %
\par\smallskip
\item[16.]\hypertarget{exercise-117}{}
            How many 10-bit strings of weight 6 start with \(111\) or end with \(101\) or both?
          %
\par\smallskip
\item[17.]\hypertarget{exercise-118}{}
            How many 6 letter words made from the letters \(a,b,c,d,e,f\) without repeats do not contain the sub-word ``bad'' in (a) consecutive letters? or (b) not-necessarily consecutive letters (but in order)?
          %
\par\smallskip
\item[18.]\hypertarget{exercise-119}{}
            Explain using lattice paths why \(\sum_{k=0}^n {n \choose k} = 2^n\).
          %
\par\smallskip
\item[19.]\hypertarget{exercise-120}{}
            Explain the relationship between \(\d{n\choose k}\) and \(P(n,k)\). Be sure to say both how the formulas for each are related, and why that relationship makes sense.
          %
\par\smallskip
\item[20.]\hypertarget{exercise-121}{}
            Give your favorite argument for why Pascal's Triangle is symmetric. That is, explain why \({n \choose k} = {n \choose n-k}\).
          %
\par\smallskip
\item[21.]\hypertarget{exercise-122}{}
            Suppose you have 20 one-dollar bills to give out as prizes to your top 5 discrete math students. How many ways can you do this if:
          %
\leavevmode%
\begin{enumerate}[label=(\alph*)]
\item\hypertarget{li-753}{}
                each of the 5 students gets at least 1 dollar?
              %
\item\hypertarget{li-754}{}
                some students might get nothing?
              %
\item\hypertarget{li-755}{}
                each student gets at least 1 dollar but no more than 7 dollars?
              %
\end{enumerate}
\par\smallskip
\item[22.]\hypertarget{exercise-123}{}
            How many functions \(f: \{1,2,3,4,5\} \to \{a,b,c,d,e\}\) are there for which
          %
\leavevmode%
\begin{enumerate}[label=(\alph*)]
\item\hypertarget{li-759}{}\(f(1) = a\) or \(f(2) = b\) (or both)?%
\item\hypertarget{li-760}{}\(f(1) \ne a\) or \(f(2) \ne b\) (or both)?%
\item\hypertarget{li-761}{}\(f(1) \ne a\emph{and}f(2) \ne b\), and are also one-to-one?%
\item\hypertarget{li-762}{}
                are onto but have \(f(1) \ne a\), \(f(2) \ne b\), \(f(3) \ne c\), \(f(4) \ne d\) and \(f(5) \ne e\)?
              %
\end{enumerate}
\par\smallskip
\item[23.]\hypertarget{exercise-124}{}
            How many permutations of \(\{1,2,3,4,5\}\) leave exactly 1 element fixed?
          %
\par\smallskip
\item[24.]\hypertarget{exercise-125}{}
            How many functions map \(\{1,2,3,4,5,6\}\emph{onto}\{a,b,c,d\}\) (i.e., how many \emph{surjections} are there)?
          %
\par\smallskip
\item[25.]\hypertarget{exercise-126}{}
            To thank your math professor for doing such an amazing job all semester, you decide to bake him (or her) cookies. You know how to make 10 different types of cookies.
          %
\leavevmode%
\begin{enumerate}[label=(\alph*)]
\item\hypertarget{li-767}{}
                If you want to give your professor 4 different types of cookies, how many different combinations of cookie type can you select? Explain your answer.
              %
\item\hypertarget{li-768}{}
                To keep things interesting, you decide to make a different number of each type of cookie. If again you want to select 4 cookie types, how many ways can you select the cookie types and decide for which there will be the most, second most, etc. Explain your answer.
              %
\item\hypertarget{li-769}{}
                You change your mind again. This time you decide you will make a total of 12 cookies. Each cookie could be any one of the 10 types of cookies you know how to bake (and it's okay if you leave some types out). How many choices do you have? Explain.
              %
\item\hypertarget{li-770}{}
                You realize that the previous plan did not account for presentation. This time, you once again want to make 12 cookies, each one could be any one of the 10 types of cookies. However, now you plan to shape the cookies into the numerals 1, 2,
                \dots{}, 12 (and probably arrange them to make a giant clock, but you haven't decided on that yet). How many choices do you have for which types of cookies to bake into which numerals? Explain.
              %
\item\hypertarget{li-771}{}
                The only flaw with the last plan is that your professor might not get to sample all 10 different varieties of cookies. How many choices do you have for which types of cookies to make into which numerals, given that each type of cookie should be present at least once? Explain.
              %
\end{enumerate}
\par\smallskip
\item[26.]\hypertarget{exercise-127}{}
            For which of the parts above does it make sense to interpret the counting question as counting some number of functions? Say what the domain and codomain should be, and whether you are counting all functions, injections, surjections, or something else.
          %
\par\smallskip
\end{exerciselist}
\typeout{************************************************}
\typeout{Subsection 2.7.1 Homework Problems}
\typeout{************************************************}
\subsection[Homework Problems]{Homework Problems}\label{subsection-21}

      The following are some more involved problems for you to try, which might be assigned as homework.
    %
\typeout{************************************************}
\typeout{Exercises 2.7.1.1 Exercises}
\typeout{************************************************}
\subsubsection[Exercises]{Exercises}\label{exercises-11}
\begin{exerciselist}
\item[1.]\hypertarget{exercise-128}{}
            We usually write numbers in decimal form (or base 10), meaning numbers are composed using 10 different ``digits'' \(\{0,1,\ldots, 9\}\). Sometimes though it is useful to write numbers \emph{hexadecimal} or base 16. Now there are 16 distinct digits that can be used to form numbers: \(\{0, 1, \ldots, 9, \mathrm{A, B, C, D, E, F}\}\). So for example, a 3 digit hexadecimal number might be 3B8.
          %
\leavevmode%
\begin{enumerate}[label=(\alph*)]
\item\hypertarget{li-782}{}
                How many 2-digit hexadecimals are there in which the first digit is E or F? Explain your answer in terms of the additive principle (using either events or sets).

                
                    There are 16 hexadecimals in which the first digit is an E (one for each choice of second digit). Similarly, there are 16 hexadecimals in which the first digit is an F. We want the union of these two disjoint sets, so there are \(16 + 16 = 32\) two digits hexadecimals in which the first digit is either an E or an F.
                  %

              %
\item\hypertarget{li-783}{}
                Explain why your answer to the previous part is correct in terms of the multiplicative principle (using either events or sets). Why do both the additive and multiplicative principles give you the same answer?

                
                    We can first select the first digit in 2 ways. We then select the second digit in 16 ways. The multiplicative principle says that the number of ways to accomplish both these tasks together is \(2 \cdot 16 = 32\). Of course \(2 \cdot 16 = 16 + 16\) so we get the same answer as in part (a). There we divided the total number of outcomes into two groups of size 16, each group based on the choice we made for the first task (selecting the first digit).
                  %

              %
\item\hypertarget{li-784}{}
                How many 3-digit hexadecimals start with a letter (A-F) and end with a numeral (0-9)? Explain.

                
                    We can select the first digit in 6 ways, the second digit in 16 ways, and the third digit in 10 ways. Thus there are \(6\cdot 16 \cdot 10 = 960\) hexadecimals given these restrictions.
                  %

              %
\item\hypertarget{li-785}{}
                How many 3-digit hexadecimals start with a letter (A-F) or end with a numeral (0-9) (or both)? Explain.

                
                    The number of 3-digit hexadecimals that start with a letter is \(6 \cdot 16 \cdot 16 = 1536\). The number of 3-hexadecimals that end with a numeral is \(16 \cdot 16 \cdot 10 = 2560\). We want all the elements from both these sets. However, both sets include those 3-digit hexadecimals which both start with a letter and end with a numeral (found to be 960 in the previous part), so we must subtract these (once). Thus the number of 3-digit hexadecimals starting with a letter or ending with a numeral is:
                    \begin{equation*}
                      1536 + 2560 - 960 = 3136
                    \end{equation*}
                  %

              %
\end{enumerate}
\par\smallskip
\item[2.]\hypertarget{exercise-129}{}
            For how many three digit numbers (100 to 999) is the \emph{sum of the digits} even? (For example, \(343\) has an even sum of digits: \(3+4+3 = 10\) which is even.) Find the answer and explain why it is correct in at least two \emph{different} ways.
          %
\par\smallskip
\item[3.]\hypertarget{exercise-130}{}
            In a recent survey, 30 students reported whether they liked their potatoes Mashed, French-fried, or Twice-baked. 15 liked them mashed, 20 liked French fries, and 9 liked twice baked potatoes. Additionally, 12 students liked both mashed and fried potatoes, 5 liked French fries and twice baked potatoes, 6 liked mashed and baked, and 3 liked all three styles. How many students
            \emph{hate} potatoes? Explain why your answer is correct.
          %
\par\smallskip
\item[4.]\hypertarget{exercise-131}{}(8)\space\space{}
            Let \(A = \{1,2,3,\ldots,9\}\).
          %
\leavevmode%
\begin{enumerate}[label=(\alph*)]
\item\hypertarget{li-789}{}
                How many subsets of \(A\) are there? That is, find \(|\pow(A)|\). Explain.

                
                    There are \(512\) subsets. This is \(2^9\), which makes sense because we are deciding yes or no on whether to include each element of \(A\) in the subset.
                  %

              %
\item\hypertarget{li-790}{}
                How many subsets of \(A\) contain exactly 5 elements? Explain.

                
                    Of the nine elements in \(A\), we must choose five of them to be in the subset. So \({9 \choose 5} = 126\).
                  %

              %
\item\hypertarget{li-791}{}
                How many subsets of \(A\) contain only even numbers? Explain.

                
                    For each of the 9 elements from \(A\), we must decide yes or no on whether to include them in the subset. However, for the odd numbers, we only have one choice: no. So there are only 4 elements we have two choices for, so the answer is \(2^4 = 16\). (Note, if you wish to exclude the empty set - it does not contain odd numbers, but no evens either - then you could subtract 1).
                  %

              %
\item\hypertarget{li-792}{}
                How many subsets of \(A\) contain an even number of elements? Explain.

                
                    Count the number of subsets with each possible even cardinality:
                    \begin{equation*}
                      {9 \choose 0} + {9 \choose 2} + {9\choose 4} + {9 \choose 6} + {9 \choose 8} = 256
                    \end{equation*}
                  %

              %
\end{enumerate}
\par\smallskip
\item[5.]\hypertarget{exercise-132}{}
            How many \(9\)-bit strings (that is, bit strings of length 9) are there which:
          %
\leavevmode%
\begin{enumerate}[label=(\alph*)]
\item\hypertarget{li-793}{}
                Start with the sub-string 101? Explain.

                
                    \(2^6 = 64\). You have 2 choices for each of the remaining 6 bits.
                  %

              %
\item\hypertarget{li-794}{}
                Have weight 5 (i.e., contain exactly five 1's) and start with the sub-string 101? Explain.

                
                    \({6 \choose 3} = 20\). You need to choose 3 of the remaining 6 bits to be 1's.
                  %

              %
\item\hypertarget{li-795}{}
                Either start with \(101\) or end with \(11\) (or both)? Explain.

                
                    176. There are 64 strings that start with 101, and another 128 which end with 11 (we choose 1 or 0 for 7 bits, so \(2^7\)). However, we count the strings that start with 101 and end with 11 twice - there are \(16\) such strings (\(2^4\)). So using PIE, we have \(64 + 128 - 16 = 176\)
                  %

              %
\item\hypertarget{li-796}{}
                Have weight 5 and either start with 101 or end with 11 (or both)? Explain.

                
                    51. There are \({6 \choose 3} = 20\) strings of weight 5 which start with 101, and another \({7 \choose 3} = 35\) which end with 11. We have over counted again - there are weight 5 strings which both start with 101 and end with 11, in fact \({4 \choose 1} = 4\) of them. So all together we have \(20 + 35 - 4 = 51\) strings.
                  %

              %
\end{enumerate}
\par\smallskip
\item[6.]\hypertarget{exercise-133}{}
            How many triangles are there with vertices from the points shown below? Note, we are not allowing degenerate triangles - ones with all three vertices on the same line, but we do allow non-right triangles. Explain why your answer is correct. (HINT: you need at exactly two points on either the \(x\)- or \(y\)-axis, but don't over-count the right triangles.)
          %
\leavevmode%
\begin{figure}
\centering
{
            \begin{tikzpicture}[scale=0.7]
  \foreach \i in {0,...,6} {
    \fill (\i,0) circle (2pt);
  }
  \foreach \i in {1,...,4} {
    \fill (0,\i) circle (2pt);
  }
\end{tikzpicture}
}
\end{figure}
\par\smallskip
\item[7.]\hypertarget{exercise-134}{}
            Gridtown USA, besides having excellent donut shoppes, is known for its precisely laid out grid of streets and avenues. Streets run east-west, and avenues north-south, for the entire stretch of the town, never curving and never interrupted by parks or schools or the like.
          %
\par

            Suppose you live on the corner of 1st and 1st and work on the corner of 12th and 12th. Thus you must travel 22 blocks to get to work as quickly as possible.
          %
\leavevmode%
\begin{enumerate}[label=(\alph*)]
\item\hypertarget{li-799}{}
                How many different routes can you take to work, assuming you want to get there as quickly as possible?

                
                    \({22 \choose 11}\) since you must choose 11 of the 22 blocks to travel east.
                  %

              %
\item\hypertarget{li-800}{}
                Now suppose you want to stop and get a donut on the way to work, from your favorite donut shoppe on the corner of 8th st and 10th ave. How many routes to work, via the donut shoppe, can you take (again, ensuring the shortest possible route)?

                
                    The donut shoppe is 16 blocks away, 7 one way, 9 the other. So to get from home to the donut shoppe, there are \({16 \choose 7}\) routes (or equivalently, \({16 \choose 9}\)). Then from the donut shopped to work, you need to travel 6 more blocks, 2 on way and 4 the other. So there are \({6 \choose 2}\) (or \({6 \choose 4}\)) routes from the donut shoppe to work.
                  %
\par

                    For each of the ways to the donut shoppe, there are so many ways to work, so the multiplicative principle says the total number of ways from home to work via the donut shoppe is
                    \begin{equation*}
                      {16 \choose 7}{6 \choose 2}
                    \end{equation*}
                  %

              %
\item\hypertarget{li-801}{}
                Disaster Strikes Gridtown: there is a pothole on 4th avenue between 5th and 6th street. How many routes to work can you take avoiding that unsightly (and dangerous) stretch of road?

                
                    Routes to work that hit the pothole: \({7 \choose 3}1{14 \choose 8}\).
                  %
\par

                    There for the number of routes to work which \emph{avoid} the pothole are
                    \begin{equation*}
                      {22 \choose 11} - {7 \choose 3}{14 \choose 8}
                    \end{equation*}
                  %

              %
\item\hypertarget{li-802}{}
                How many routes are there both avoiding the pothole and visiting the donut shoppe?

                
                    First compute the number of routes to the donut shoppe avoiding the pothole:
                    \begin{equation*}
                      {16 \choose 7} - {7 \choose 3}{8 \choose 6}
                    \end{equation*}
                  %
\par

                    Then you still need to go to work from there. Thus the answer is:
                    \begin{equation*}
                      \left({16 \choose 7} - {7 \choose 3}{8 \choose 6}\right){6 \choose 2}
                    \end{equation*}
                  %

              %
\end{enumerate}
\par\smallskip
\item[8.]\hypertarget{exercise-135}{}
            Recall that the formula for \(P(n,k)\) is \(\dfrac{n!}{(n-k)!}\). Your task here is to explain \emph{why} this is the right formula.
          %
\leavevmode%
\begin{enumerate}[label=(\alph*)]
\item\hypertarget{li-803}{}
                Suppose you have 12 chips, each a different color. How many different stacks of 5 chips can you make? Explain your answer and why it is the same as using the formula for \(P(12,5)\).

                
                    There are \(12\cdot 11\cdot 10\cdot 9\cdot 8 = 95040\) different stacks of 5 chips. This is because there are 12 choices for which chip goes on bottom, then 11 choices for which comes next, and so on for a total of 5 chips. The formula for \(P(12,5)\) is \(\frac{12!}{7!}\) - the \(7!\) cancels out the part of \(12!\) we don't include.
                  %

              %
\item\hypertarget{li-804}{}
                Using the scenario of the 12 chips again, what does \(12!\) count? What does \(7!\) count? Explain.

                
                    \(12!\) is the number of different stacks of all 12 chips (12 choices for the bottom chip, 11 for the next, and so on for 12 chips total). \(7!\) is the number of 7-chip stacks you can make using just 7 different colored chips (not all 12) because you have 7 choices for the first chip, 6 for the second, and so on.
                  %

              %
\item\hypertarget{li-805}{}
                Explain why it makes sense to divide \(12!\) by \(7!\) when computing \(P(12,5)\) (in terms of the chips).

                
                    Say you made all \(12!\) stacks of 12 chips. For any particular arrangement of the bottom 5 chips, you will have \(7!\) stacks that have that bottom arrangement (one for each arrangement of the remaining 7 chips). But we just want to count the number of 5-chip stacks, so we count all of these as the same outcome (once the top 7 chips are removed, the bottom stacks look identical). Thus we really want to count how many groups there are. All \(12!\) stacks are grouped into groups of size \(7!\), so there are \(12!/7!\) groups.
                  %

              %
\item\hypertarget{li-806}{}
                Does your explanation work for numbers other than 12 and 5? Explain the formula \(P(n,k) = \frac{n!}{(n-k)!}\) using the variables \(n\) and \(k\).

                
                    \(P(n,k)\) counts the number of stacks of size \(k\) where the chips come in \(n\) different colors (with at most one chip of each color in the stack). You could count this by first forming all stacks of \(n\) chips, and there are \(n!\) of these. But to get stacks of size \(k\), we don't want the top \(n-k\) chips. When we remove \(n-k\) chips from the stacks, we sometimes get the same arrangement of the bottom chips. In fact, for each arrangement of the bottom \(k\) chips, there are \((n-k)!\) different ways to arrange the top \(n-k\) chips, and all of these should count as just 1 outcome. Thus we divide \(n!\) by \((n-k)!\) to find the number of \(k\)-chip stacks.
                  %

              %
\end{enumerate}
\par\smallskip
\item[9.]\hypertarget{exercise-136}{}
            Suppose you own \(x\) fezzes and \(y\) bow ties. Of course, \(x\) and \(y\) are both greater than 1.
          %
\leavevmode%
\begin{enumerate}[label=(\alph*)]
\item\hypertarget{li-807}{}
                How many combinations of fez and bow tie can you make? You can wear only one fez and one bow tie at a time. Explain.

                
                    You have \(x\) choices for the fez, and for each choice of fez you have \(y\) choices for the bow tie. Thus you have \(x \cdot y\) choices for fez and bow tie combination.
                  %

              %
\item\hypertarget{li-808}{}
                Explain why the answer is \emph{also} \({x+y \choose 2} - {x \choose 2} - {y \choose 2}\). (If this is what you claimed the answer was in part (a), try it again.)

                
                    Line up all \(x+y\) quirky clothing items - the \(x\) fezzes and \(y\) bow ties. Now pick 2 of them. This can be done in \({x+y \choose 2}\) ways. However, we might have picked 2 fezzes, which is not allowed. There are \({x \choose 2}\) ways to pick 2 fezzes. Similarly, the \({x+y \choose 2}\) ways to pick two items includes \({y \choose 2}\) ways to select 2 bow ties, also not allowed. Thus the total number of ways to pick a fez and a bow ties is
                    \begin{equation*}
                      {x+y \choose 2} - {x \choose 2} - {y \choose 2}
                    \end{equation*}
                  %

              %
\item\hypertarget{li-809}{}
                Use your answers to parts (a) and (b) to give a combinatorial proof of the identity
                \begin{equation*}
                  {x+y \choose 2} - {x \choose 2} - {y \choose 2} = xy
                \end{equation*}
                \begin{proof}\hypertarget{proof-13}{}

                      The question is how many ways can you select one of \(x\) fezzes and one of \(y\) bow ties. We answer this question in two ways. First, the answer could be \(a\cdot b\). This is correct as described in part (a) above. Second, the answer could be \({x+y \choose 2} - {x \choose 2} - {y \choose 2}\). This is correct as described in part (b) above. Therefore
                      \begin{equation*}
                        {x+y \choose 2} - {x \choose 2} - {y \choose 2} = xy
                      \end{equation*}
                    %
\end{proof}

              %
\end{enumerate}
\par\smallskip
\item[10.]\hypertarget{exercise-137}{}
            Consider the identity:
            \begin{equation*}
              k{n\choose k} = n{n-1 \choose k-1}
            \end{equation*}
          %
\leavevmode%
\begin{enumerate}[label=(\alph*)]
\item\hypertarget{li-810}{}
                Is this true? Try it for a few values of \(n\) and \(k\).

                
                    Yes. For example, if \(n = 7\) and \(k = 4\), we have
                    \begin{equation*}
                      4 \cdot {7 \choose 4} = 4 \cdot 35 = 140 = 7 \cdot 20 = 7 \cdot {6 \choose 3}
                    \end{equation*}
                  %

              %
\item\hypertarget{li-811}{}
                Use the formula for \({n \choose k}\) to give an algebraic proof of the identity.

                \begin{equation*}
                    k{n \choose k} = k \frac{n!}{(n-k)!\,k!} = \frac{n!}{(n-k)!(k-1)!} = n\frac{(n-1)!}{(n-1-(k-1))!(k-1)!} = n {n-1 \choose k-1}
                  \end{equation*}
              %
\item\hypertarget{li-812}{}
                Give a combinatorial proof of the identity. Hint: How many ways can you select a team of \(k\) people from a group of \(n\) people \emph{and} select one of them to be the team captain?

                \begin{proof}\hypertarget{proof-14}{}

                      Question: How many ways can you select a chaired committee of \(k\) people from a group of \(n\) people? That is, you need to select \(k\) people to be on the committee and one of them needs to be in charge. How many ways can this happen?
                    %
\par

                      Answer 1: First select \(k\) of the \(n\) people to be on the committee. This can be done in \({n \choose k}\) ways. Now select one of those \(k\) people to be in charge - this can be done in \(k\) ways. So there are a total of \(k {n \choose k}\) ways to select the chaired committee.
                    %
\par

                      Answer 2: First select the chair of the committee. You have \(n\) people to choose from, so this can be done in \(n\) ways. Now fill the rest of the committee. There are \(n-1\) people to choose from (you cannot select the person you picked to be the chair) and \(k-1\) spots to fill (the chair's spot is already taken). So this can be done in \({n-1 \choose k-1}\) ways. Therefore there are \(n{n-1 \choose k-1}\) ways to select the chaired committee.
                    %
\end{proof}

              %
\end{enumerate}
\par\smallskip
\item[11.]\hypertarget{exercise-138}{}
            After a late night of math studying, you and your friends decide to go to your favorite tax-free fast food Mexican restaurant, \emph{Burrito Chime}. You decide to order off of the dollar menu, which has 7 items. Your group has
            \textdollar{}16 to spend (and will spend all of it).
          %
\leavevmode%
\begin{enumerate}[label=(\alph*)]
\item\hypertarget{li-813}{}
                How many different orders are possible? Explain. (The \emph{order} in which the order is placed does not matter - just which and how many of each item that is ordered.)

                
                    \(\d{22 \choose 6}\) - there are 16 stars and 6 bars.
                  %

              %
\item\hypertarget{li-814}{}
                How many different orders are possible if you want to get at least one of each item? Explain.

                
                    \(\d{15 \choose 6}\) - buy one of each item, using
                    \textdollar{}7. This leaves you
                    \textdollar{}11 to distribute to the 7 items, so 11 stars and 6 bars.
                  %

              %
\item\hypertarget{li-815}{}
                How many different orders are possible if you don't get more than 4 of any one item? Explain. Hint: get rid of the bad orders using PIE.

                \begin{equation*}
                    {22 \choose 6} - \left[{7 \choose 1}{17 \choose 6} - {7 \choose 2}{12 \choose 6} + {7 \choose 3}{7 \choose 6} \right]
                  \end{equation*}
              %
\end{enumerate}
\par\smallskip
\item[12.]\hypertarget{exercise-139}{}
            Consider functions \(f:\{1,2,3,4,5\} \to \{0,1,2,\ldots,9\}\).
          %
\leavevmode%
\begin{enumerate}[label=(\alph*)]
\item\hypertarget{li-816}{}
                How many of these functions are strictly increasing? Explain. (A function is strictly increasing provided if \(a \lt  b\), then \(f(a) \lt  f(b)\).)

                
                    \({10 \choose 5}\). Note that a strictly increasing function is automatically injective. So the five outputs must all be different. So let's first pick which five outputs we will use: there are \({10 \choose 5}\) ways to do this. Now how many ways are there to assign those outputs to the inputs \(1\) through 5? Only one way, since there is only one way to arrange numbers in increasing order.
                  %

              %
\item\hypertarget{li-817}{}
                How many of the functions are non-decreasing? Explain. (A function is non-decreasing provided if \(a \lt  b\), then \(f(a) \le f(b)\).)

                
                    \({14 \choose 5}\). This is in fact a stars and bars problem. The stars are the 5 inputs and the bars are the 9 spots between the 10 possible outputs. Think of it this way - we will specify \(f(1)\), then \(f(2)\), then \(f(3)\), and so on in that order. Start with the possible output 0. We can use it as the output of \(f(1)\), or we can switch to 1 as a potential output. Say we put \(f(1) = 1\). Now we are at 1 (can't go back to 0). Should \(f(2) = 1\)? If yes, then we are putting down another star. If no, put down a bar and switch to 2. Maybe you switch to 3, then assign \(f(2) = 3\) and \(f(3) = 3\) (two more stars) before switching to 4 as a possible output. And so on.
                  %

              %
\end{enumerate}
\par\smallskip
\item[13.]\hypertarget{exercise-140}{}
            The Grinch sneaks into a room with 6 Christmas presents to 6 different people. He proceeds to switch the name-labels on the presents. How many ways could he do this if:
          %
\leavevmode%
\begin{enumerate}[label=(\alph*)]
\item\hypertarget{li-818}{}
                No present is allowed to end up with its original label? Explain what each term in your answer represents.

                \begin{equation*}
                    6! - \left[{6 \choose 1}5! - {6 \choose 2}4! + {6 \choose 3}3! - {6 \choose 4}2! + {6 \choose 5}1! - {6 \choose 6}0!\right]
                  \end{equation*}
              %
\item\hypertarget{li-819}{}
                Exactly 2 presents keep their original labels? Explain.

                \begin{equation*}
                    {6 \choose 2}\left(4! - \left[{4\choose 1}3! - {4 \choose 2}2! + {4 \choose 3}1! - {4 \choose 4}0!\right]\right)
                  \end{equation*}
              %
\item\hypertarget{li-820}{}
                Exactly 5 presents keep their original labels? Explain.

                
                    0. Once 5 presents have their original label, there is only one present left and only one label left, so the 6th present must get its own label.
                  %

              %
\end{enumerate}
\par\smallskip
\end{exerciselist}
%
%% A lineskip in table of contents as transition to appendices, backmatter
\addtocontents{toc}{\vspace{\normalbaselineskip}}
%
%
\appendix
%
\typeout{************************************************}
\typeout{Appendix A Solutions to Exercises}
\typeout{************************************************}
\chapter[Solutions to Exercises]{Solutions to Exercises}\label{appendix-1}
\subsection*{1.2.4 Exercises}
\noindent\textbf{1.}\quad{}\leavevmode%
\begin{enumerate}[label=(\alph*)]
\item\hypertarget{li-67}{}This is not a statement; it does not make sense to say it is true or false.%
\item\hypertarget{li-68}{}This is an atomic statement (there are some quantifiers, but no connectives).%
\item\hypertarget{li-69}{}This is a molecular statement, specifically a disjunction.  Although if we read into it a bit more, what the speaker is really saying is that if the Broncos do not win the super bowl, then he will eat his hat, which would be a conditional.%
\item\hypertarget{li-70}{}This is a molecular statement, a conditional.%
\item\hypertarget{li-71}{}This is an atomic statement.  Even though there is an ``or'' in the statement, it would not make sense to consider the two halves of the disjuction.  This is because we quantified \emph{over} the disjunction.  In symbols, we have \(\forall x (x > 1 \imp (P(x) \vee C(x)))\).  If we drop the quantifier, we are not left with a statement, since there is a free variable.%
\item\hypertarget{li-72}{}This is not a statement, although it certainly looks like one.  Remember that statements must be true or false.  If this sentence were true, that would make it false.  If it were false, that would make it true.  Examples like this are rare and usually arise from some sort of self reference.%
\end{enumerate}
\par\smallskip
\noindent\textbf{2.}\quad{}\leavevmode%
\begin{enumerate}[label=(\alph*)]
\item\hypertarget{li-80}{}\(P \wedge Q\).%
\item\hypertarget{li-81}{}\(P \imp \neg Q\).%
\item\hypertarget{li-82}{}
    Jack passed math or Jill passed math (or both).
    %
\item\hypertarget{li-83}{}
    If Jack and Jill did not both pass math, then Jill did.
    %
\item\hypertarget{li-84}{}
    %
\begin{enumerate}[label=\roman*.]
\item\hypertarget{li-85}{} Nothing else. %
\item\hypertarget{li-86}{} Jack did not pass math either.%
\end{enumerate}

    %
\end{enumerate}
\par\smallskip
\noindent\textbf{3.}\quad{}\leavevmode%
\begin{enumerate}[label=(\alph*)]
\item\hypertarget{li-90}{} Three statements: \(P \vee S\), \(S \imp Q\), \((P \vee Q) \imp R\).  You could also connect the first two with a \(\wedge\). %
\item\hypertarget{li-91}{} He cannot be lying about all three sentences, so he is telling the truth. %
\item\hypertarget{li-92}{} No matter what, Geoff wants ricotta.  If he doesn't have quail, then he must have pepperoni but not sausage. %
\end{enumerate}
\par\smallskip
\noindent\textbf{4.}\quad{}\leavevmode%
\begin{enumerate}[label=(\alph*)]
\item\hypertarget{li-98}{} If Oscar drinks milk, then he eats Chinese food. %
\item\hypertarget{li-99}{} If Oscar does not drink milk, then he does not eat Chinese food. %
\item\hypertarget{li-100}{} Yes. The original statement would be false too. %
\item\hypertarget{li-101}{} Nothing. The converse need not be true. %
\item\hypertarget{li-102}{} He does not eat Chinese food. The contrapositive would be true. %
\end{enumerate}
\par\smallskip
\noindent\textbf{5.}\quad{}
          The statements are equivalent to the\dots{}
        %
\leavevmode%
\begin{enumerate}[label=(\alph*)]
\item\hypertarget{li-115}{} converse. %
\item\hypertarget{li-116}{} implication. %
\item\hypertarget{li-117}{} neither. %
\item\hypertarget{li-118}{} implication. %
\item\hypertarget{li-119}{} converse. %
\item\hypertarget{li-120}{} converse. %
\item\hypertarget{li-121}{} implication. %
\item\hypertarget{li-122}{} converse. %
\item\hypertarget{li-123}{} converse. %
\item\hypertarget{li-124}{} converse (in fact, this IS the converse). %
\item\hypertarget{li-125}{} implication (the statement is the contrapositive of the implication). %
\item\hypertarget{li-126}{} neither. %
\end{enumerate}
\par\smallskip
\noindent\textbf{6.}\quad{}
          Hint: of course there are many answers. It helps to assume that the statement is true and the converse is NOT true. Think about what that means in the real world and then start saying it in different ways. Some ideas: use necessary and sufficient language, use ``only if,'' consider negations, use ``or else'' language.
        %
\par\smallskip
\noindent\textbf{7.}\quad{}\leavevmode%
\begin{enumerate}[label=(\alph*)]
\item\hypertarget{li-132}{}\(\neg \exists x (E(x) \wedge O(x))\).%
\item\hypertarget{li-133}{}\(\forall x (E(x) \imp O(x+1))\).%
\item\hypertarget{li-134}{}\(\exists x(P(x) \wedge E(x))\) (where \(P(x)\) means ``\(x\) is prime'').%
\item\hypertarget{li-135}{}\(\forall x \forall y \exists z(x \lt  z \lt  y \vee y \lt  z \lt  x)\).%
\item\hypertarget{li-136}{}\(\forall x \neg \exists y (x \lt  y \lt  x+1)\).%
\end{enumerate}
\par\smallskip
\noindent\textbf{8.}\quad{}\leavevmode%
\begin{enumerate}[label=(\alph*)]
\item\hypertarget{li-141}{} Any even number plus 2 is an even number. %
\item\hypertarget{li-142}{} For any \(x\) there is a \(y\) such that \(\sin(x) = y\). In other words, every number \(x\) is in the domain of sine. %
\item\hypertarget{li-143}{} For every \(y\) there is an \(x\) such that \(\sin(x) = y\). In other words, every number \(y\) is in the range of sine (which is false). %
\item\hypertarget{li-144}{} For any numbers, if the cubes of two numbers are equal, then the numbers are equal. %
\end{enumerate}
\par\smallskip
\noindent\textbf{9.}\quad{}
          If \(P(x)\) is true of every \(x\), then in particular it is true of \(x = 0\) (or any fixed element of the universe). So then there is definitely some \(x\) (namely 0) for which \(P(x)\) holds. Thus \(\forall x P(x) \imp \exists x P(x)\) is always true. The converse is not always true though. Consider the predicate \(x = 5\). So \(P(x)\) is true if and only if \(x = 5\). Certainly it is true that \(\exists x P(x)\) (since we can take \(x = 5\)), but false that \(\forall x P(x)\).
        %
\par\smallskip
\noindent\textbf{10.}\quad{}\leavevmode%
\begin{enumerate}[label=(\alph*)]
\item\hypertarget{li-148}{}
              This says that everything has a square root (every element is the square of something). This is true of the positive real numbers, and also of the complex numbers. It is false of the natural numbers though, as for \(x = 2\) there is no natural number \(y\) such that \(y^2 = 2\).
            %
\item\hypertarget{li-149}{}
              This asserts that between every pair of numbers there is some number strictly between them. This is true of the rationals (and reals) but false of the integers. If \(x = 1\) and \(y = 2\), then there is nothing we can take for \(z\).
            %
\item\hypertarget{li-150}{}
              Here we are saying that something is between every pair of numbers. For almost every domain, this is false. In fact, if the domain contains \(\{1,2,3, 4\}\), then no matter what we take \(x\) to be, there will be a pair that \(x\) is NOT between. However, the set \(\{1,2,3\}\) as our domain makes the statement true. Let \(x = 2\). Then no matter what \(y\) and \(z\) we pick, if \(y \lt  z\), then 2 is between them.
            %
\end{enumerate}
\par\smallskip
\subsection*{1.3.5 Exercises}
\noindent\textbf{1.}\quad{}\leavevmode%
\begin{enumerate}[label=(\alph*)]
\item\hypertarget{li-235}{}\(A \cap B = \{3,4,5\}\).%
\item\hypertarget{li-236}{}\(A \cup B = \{1,2,3,4,5,6,7\}\).%
\item\hypertarget{li-237}{}\(A \setminus B = \{1,2\}\).%
\item\hypertarget{li-238}{}\(A \cap \bar{(B \cup C)} = \{1\}\).%
\item\hypertarget{li-239}{}\(A \times C = \{(1,2), (1,3), (1,5), (2,2), (2,3), (2,5), (3,2), (3,3), (3,5), (4,2), (4,3), (4,5), (5,2), (5,3), (5,5)\}\).%
\item\hypertarget{li-240}{}Yes.%
\item\hypertarget{li-241}{}No.%
\end{enumerate}
\par\smallskip
\noindent\textbf{2.}\quad{}\leavevmode%
\begin{enumerate}[label=(\alph*)]
\item\hypertarget{li-246}{}\(A \cap B = \{4,6,8,10,12\} = \{x \in \N \st (3 \le x \le 13) \wedge x \mbox{ is even}\).%
\item\hypertarget{li-247}{}\(A \cup B = \{3, 4, 5, \ldots, 12, 13\} = \{x \in \N \st (3 \le x \le 13) \vee x \mbox{ is even} \}\). %
\item\hypertarget{li-248}{}\(B \cap C = \emptyset\).%
\item\hypertarget{li-249}{}\(B \cup C = \N\).%
\end{enumerate}
\par\smallskip
\noindent\textbf{3.}\quad{}
          For example, \(A = \{2,3,5,7,8\}\) and \(B = \{3,5\}\).
        %
\par\smallskip
\noindent\textbf{4.}\quad{}
          For example, \(A = \{1,2,3\}\) and \(B = \{1,2,3,4,5,\{1,2,3\}\}\)
        %
\par\smallskip
\noindent\textbf{5.}\quad{}\leavevmode%
\begin{enumerate}[label=(\alph*)]
\item\hypertarget{li-254}{} No. %
\item\hypertarget{li-255}{} No. %
\item\hypertarget{li-256}{}\(2\Z \cap 3\Z\) is the set of all integers which are multiples of both 2 and 3 (so multiples of 6). Therefore \(2\Z \cap 3\Z = \{x \in \Z \st \exists y\in \Z(x = 6y)\}\).%
\item\hypertarget{li-257}{}\(2\Z \cup 3\Z\).%
\end{enumerate}
\par\smallskip
\noindent\textbf{6.}\quad{}
          The set of primes.
        %
\par\smallskip
\noindent\textbf{7.}\quad{}\leavevmode%
\begin{enumerate}[label=(\alph*)]
\item\hypertarget{li-264}{}\(A \cup \bar B\):

            \leavevmode%
\begin{figure}
\centering
{
\begin{tikzpicture}[fill=gray!50]
    \fill \circleA;
      \begin{scope}
      \clip \circleB \twosetbox;
      \fill \twosetbox;
      \end{scope}
      \draw[thick] \circleA \circleAlabel \circleB \circleBlabel \twosetbox;
    \end{tikzpicture}
}
\end{figure}
%
\item\hypertarget{li-265}{}\(\bar{(A \cup B)}\):
            \leavevmode%
\begin{figure}
\centering
{
              \begin{tikzpicture}[fill=gray!50]
  \fill \twosetbox;
  \fill[white] \circleA \circleB;
  \draw[thick] \circleA \circleAlabel \circleB \circleBlabel \twosetbox;
\end{tikzpicture}
}
\end{figure}
%
\item\hypertarget{li-266}{}\(A \cap (B \cup C)\):
            \leavevmode%
\begin{figure}
\centering
{
              \begin{tikzpicture}[fill=gray!50]
\begin{scope}
  \clip \circleA;
  \fill \circleB \circleC;
\end{scope}
\draw[thick] \circleA \circleAlabel \circleB \circleBlabel \circleC \circleClabel \threesetbox;
\end{tikzpicture}
}
\end{figure}
%
\item\hypertarget{li-267}{}\((A \cap B) \cup C\):
            \leavevmode%
\begin{figure}
\centering
{
\begin{tikzpicture}[fill=gray!50]
  \begin{scope}
    \clip \circleA;
    \fill \circleB;
  \end{scope}
  \fill \circleC;
  \draw[thick] \circleA \circleAlabel \circleB \circleBlabel \circleC \circleClabel \threesetbox;
  \end{tikzpicture}
}
\end{figure}
%
\item\hypertarget{li-268}{}\(\bar A \cap B \cap \bar C\):
            \leavevmode%
\begin{figure}
\centering
{
\begin{tikzpicture}[fill=gray!50]
  \fill \circleB;
  \begin{scope}
    \clip \circleB;
    \fill[white] \circleA \circleC;
  \end{scope}

  \draw[thick] \circleA \circleAlabel \circleB \circleBlabel \circleC \circleClabel \threesetbox;
  \end{tikzpicture}
}
\end{figure}
%
\item\hypertarget{li-269}{}\((A \cup B) \setminus C\):
            \leavevmode%
\begin{figure}
\centering
{
\begin{tikzpicture}[fill=gray!50]
\fill \circleA;
\fill \circleB;
\fill[white] \circleC;
\draw[thick] \circleA \circleAlabel \circleB \circleBlabel \circleC \circleClabel \threesetbox;
\end{tikzpicture}
}
\end{figure}
%
\end{enumerate}
\par\smallskip
\noindent\textbf{8.}\quad{}
          For example, \(A \cup B \cap \bar{(A \cap B)}\). Note that \(\bar{A \cap B}\) would almost work, but it also contains the area outside of both circles.
        %
\par\smallskip
\noindent\textbf{9.}\quad{}\leavevmode%
\begin{enumerate}[label=(\alph*)]
\item\hypertarget{li-273}{} 34. %
\item\hypertarget{li-274}{} 103. %
\item\hypertarget{li-275}{} 8. %
\end{enumerate}
\par\smallskip
\noindent\textbf{10.}\quad{}
          \(\pow(A) = \{\emptyset, \{a\}, \{b\}, \{c\}, \{d\}, \{a,b\}, \{a,c\}, \{a,d\}, \{b,c\}, \{b,d\}, \{c,d\} \{a,b,c\}, \{a,b,d\}, \{a,c,d\}, \{b,c,d\}, \{a,b,c,d\}\}\).
        %
\par\smallskip
\noindent\textbf{11.}\quad{}
          There are 10 singletons. There are 45 doubletons: nine that include 1, eight that include 2 (but not 1), 7 that include 3 (but not 1 or 2) and so on. \(9+8+7+\cdots+2+1 = 45\); ).
        %
\par\smallskip
\noindent\textbf{12.}\quad{}
          \(\{2,3,5\}\), \(\{1,2,3,5\}\), \(\{2,3,4,5\}\), \(\{2,3,5,6\}\), \(\{1,2,3,4,5\}\), \(\{1,2,3,5,6\}\), \(\{2,3,4,5,6\}\), and \(\{1,2,3,4,5,6\}\).
        %
\par\smallskip
\noindent\textbf{13.}\quad{}
          For example, \(A = \{1,2,3,4\}\) and \(B = \{5,6,7,8,9\}\) gives \(A \cup B = \{1,2,3,4,5,6,7,8,9\}\).
        %
\par\smallskip
\noindent\textbf{14.}\quad{}
          For example, \(A = \{1,2,3\}\) and \(B = \{2,3,4,5\}\) gives \(A\cup B = \{1,2,3,4,5\}\).
        %
\par\smallskip
\noindent\textbf{15.}\quad{}
          No. There must be 5 elements in common to both sets. Since there are 10 distinct elements all together in \(A\) and \(B\), this means that between \(A\) and \(B\), there must be 5 elements which they do not have in common (some in \(A\) but not in \(B\), some in \(B\) but not in \(A\)). But to have \(|A| = |B|\), we would need to exclude the same number of elements from both sets.  Since 5 is odd, we would need to exclude 2.5 elements from each set making \(|A| = |B| = 7.5\) which is impossible.
        %
\par\smallskip
\noindent\textbf{16.}\quad{}
          If \(R\) is the set of red cards and \(F\) is the set of face cards, we want to find \(|R \cup F|\). This is not simply \(|R| + |F|\) because there are 6 cards which are both red and a face card; \(|R \cap F| = 6\). We find
          \(|R \cup F| = 32\).
        %
\par\smallskip
\subsection*{1.4.3 Exercises}
\noindent\textbf{1.}\quad{}
            There are 8 different functions. In two-line notation these are:

            
              \begin{equation*} f = \begin{pmatrix} 1 \amp 2 \amp 3 \\ a \amp a\amp a \end{pmatrix} \quad f = \begin{pmatrix} 1 \amp 2 \amp 3 \\ b \amp b \amp b \end{pmatrix}\end{equation*}
              \begin{equation*} f = \begin{pmatrix} 1 \amp 2 \amp 3 \\ a \amp a\amp b \end{pmatrix} \quad f = \begin{pmatrix} 1 \amp 2 \amp 3 \\ a \amp b \amp a \end{pmatrix} \quad f = \begin{pmatrix} 1 \amp 2 \amp 3 \\ b \amp a\amp a \end{pmatrix}
              \end{equation*}
              \begin{equation*}
                \quad f = \begin{pmatrix} 1 \amp 2 \amp 3 \\ b \amp b \amp a \end{pmatrix} \quad f = \begin{pmatrix} 1 \amp 2 \amp 3 \\ b \amp a\amp b \end{pmatrix} \quad f = \begin{pmatrix} 1 \amp 2 \amp 3 \\ a \amp b \amp b \end{pmatrix} \end{equation*}

            %


            None of the functions are injective. Exactly 6 of the functions are surjective. No functions are both (since no functions here are injective).
          %
\par\smallskip
\noindent\textbf{2.}\quad{}
            There are 9 functions: you have a choice of three outputs for \(f(1)\), and for each, you have three choices for the output \(f(2)\). Of these functions, 6 are injective, 0 are surjective, and 0 are both:

            \begin{equation*}
              f = \twoline{1 \amp 2}{a\amp a} \quad f = \twoline{1 \amp 2}{b \amp b} \quad f = \twoline{1 \amp 2}{c \amp c}
            \end{equation*}
             \begin{equation*}
               f = \twoline{1 \amp 2}{a\amp b} \quad f = \twoline{1 \amp 2}{a \amp c} \quad f = \twoline{1 \amp 2}{b \amp c}
             \end{equation*}
             \begin{equation*}
               f = \twoline{1 \amp 2}{b \amp a} \quad f = \twoline{1 \amp 2}{c \amp a} \quad f = \twoline{1 \amp 2}{c \amp b}
             \end{equation*}
          %
\par\smallskip
\noindent\textbf{3.}\quad{}\leavevmode%
\begin{enumerate}[label=(\alph*)]
\item\hypertarget{li-303}{}\(f\) is not injective, since \(f(2) = f(5)\) - two different inputs have the same output.%
\item\hypertarget{li-304}{}\(f\) is surjective, since every element of the codomain is an element of the range.%
\item\hypertarget{li-305}{}\(f=\begin{pmatrix}1 \amp 2 \amp 3 \amp 4 \amp 5 \\ 3 \amp 2 \amp 4 \amp 1 \amp 2\end{pmatrix}\).%
\end{enumerate}
\par\smallskip
\noindent\textbf{4.}\quad{}\leavevmode%
\begin{enumerate}[label=(\alph*)]
\item\hypertarget{li-309}{}\(f\) is not injective, since \(f(1) = 3\) and \(f(4) = 3\).%
\item\hypertarget{li-310}{}\(f\) is not surjective, since there is no input which gives 2 as an output.%
\item\hypertarget{li-311}{}\(f=\begin{pmatrix} 1 \amp 2 \amp 3 \amp 4 \\ 3 \amp 4 \amp 1 \amp 3\end{pmatrix}\).%
\end{enumerate}
\par\smallskip
\noindent\textbf{5.}\quad{}\leavevmode%
\begin{enumerate}[label=(\alph*)]
\item\hypertarget{li-316}{}\(f\) is injective, but not surjective (since 0, for example, is never an output).%
\item\hypertarget{li-317}{}\(f\) is injective and surjective. Unlike in the previous question, every integers is an output (of the integer 4 less than it).%
\item\hypertarget{li-318}{}\(f\) is injective, but not surjective (10 is not 8 less than a multiple of 5, for example).%
\item\hypertarget{li-319}{}\(f\) is not injective, but is surjective. Every integer is an output (of twice itself, for example) but some integers are outputs of more than one input: \(f(5) = 3 = f(6)\).%
\end{enumerate}
\par\smallskip
\noindent\textbf{6.}\quad{}\leavevmode%
\begin{enumerate}[label=(\alph*)]
\item\hypertarget{li-325}{}\(f\) is not injective. To prove this, we must simply find two different elements of the domain which map to the same element of the codomain. Since \(f(\{1\}) = 1\) and \(f(\{2\}) = 1\), we see that \(f\) is not injective.%
\item\hypertarget{li-326}{}\(f\) is not surjective. The largest subset of \(A\) is \(A\) itself, and \(|A| = 10\). So no natural number greater than 10 will ever be an output.%
\item\hypertarget{li-327}{}\(f\inv(1) = \{\{1\}, \{2\}, \{3\}, \ldots \{10\}\}\) (the set of all the singleton subsets of \(A\)).%
\item\hypertarget{li-328}{}\(f\inv(0) = \{\emptyset\}\). Note, it would be wrong to write \(f\inv(0) = \emptyset\) - that would claim that there is no input which has 0 as an output.%
\item\hypertarget{li-329}{}\(f\inv(12) = \emptyset\), since there are no subsets of \(A\) with cardinality 12.%
\end{enumerate}
\par\smallskip
\noindent\textbf{7.}\quad{}\leavevmode%
\begin{enumerate}[label=(\alph*)]
\item\hypertarget{li-334}{}\(f\inv(3) = \{003, 030, 300, 012, 021, 102, 201, 120, 210, 111\}\)%
\item\hypertarget{li-335}{}\(f\inv(28) = \emptyset\) (since the largest sum of three digits is \(9+9+9 = 27\))%
\item\hypertarget{li-336}{}
                Part (a) proves that \(f\) is not injective. The output 3 is assigned to 10 different inputs.
              %
\item\hypertarget{li-337}{}
                Part (b) proves that \(f\) is not surjective. There is an element of the codomain (28) which is not assigned to any inputs.
              %
\end{enumerate}
\par\smallskip
\noindent\textbf{8.}\quad{}\leavevmode%
\begin{enumerate}[label=(\alph*)]
\item\hypertarget{li-341}{}\(|f\inv(3)| \le 1\). In other words, either \(f\inv(3)\) is the emptyset or is a set containing exactly one element. Injective functions cannot have two elements from the domain both map to 3.%
\item\hypertarget{li-342}{}\(|f\inv(3)| \ge 1\). In other words, \(f\inv(3)\) is a set containing at least one elements, possibly more. Surjective functions must have something map to 3.%
\item\hypertarget{li-343}{}\(|f\inv(3)| = 1\). There is exactly one element from \(X\) which gets mapped to 3, so \(f\inv(3)\) is the set containing that one element.%
\end{enumerate}
\par\smallskip
\noindent\textbf{9.}\quad{}
            \(X\) can really be any set, as long as \(f(x) = 0\) or \(f(x) = 1\) for every \(x \in X\). For example, \(X = \N\) and \(f(n) = 0\) works.
          %
\par\smallskip
\noindent\textbf{10.}\quad{}\leavevmode%
\begin{enumerate}[label=(\alph*)]
\item\hypertarget{li-347}{}\(|X| \le |Y|\). Otherwise two or more of the elements of \(X\) would need to map to the same element of \(Y\).%
\item\hypertarget{li-348}{}\(|X| \ge |Y|\). Otherwise there would be one or more elements of \(Y\) which were never an output.%
\item\hypertarget{li-349}{}\(|X| = |Y|\). This is the only way for both of the above to occur.%
\end{enumerate}
\par\smallskip
\noindent\textbf{11.}\quad{}\leavevmode%
\begin{enumerate}[label=(\alph*)]
\item\hypertarget{li-356}{} Possible. For example, let \(X=\{1,2\}\) and \(Y = \{1,2,3\}\) and consider \(f=\begin{pmatrix}1 \amp 2 \\ 1 \amp 3\end{pmatrix}\). %
\item\hypertarget{li-357}{} Possible. For example, let \(X = \{1,2\}\) and \(Y = \{1\}\) with \(f(1) = f(2) = 1\).%
\item\hypertarget{li-358}{} Possible. This is possible, but only if both \(X\) and \(Y\) are infinite. For example, if \(X = Y = \Z\), consider \(f(x) = 2x\).%
\item\hypertarget{li-359}{} Possible. For example, again take \(X = Y = \N\) and consider \(f(x) = \begin{cases} 0 \amp \mathrm{ if } x = 0 \\ x-1 \amp \mathrm{ if } x \ge 1\end{cases}\)%
\item\hypertarget{li-360}{} Not possible. If \(f\) is injective, then each input corresponds to a different output, so the range is the size of \(X\), which is the size of \(Y\), so the range is all of \(Y\).%
\item\hypertarget{li-361}{} Not possible. If \(f\) is surjective, then every element of the codomain is an output. If more than one output corresponded to a single input, we would not have enough inputs to cover all the outputs.%
\end{enumerate}
\par\smallskip
\noindent\textbf{12.}\quad{}\leavevmode%
\begin{enumerate}[label=(\alph*)]
\item\hypertarget{li-364}{}\(f\) is injective.

              \begin{proof}\hypertarget{proof-1}{}

                  Let \(x\) and \(y\) be elements of the domain \(\Z\). Assume \(f(x) = f(y)\). If \(x\) and \(y\) are both even, then \(f(x) = x+1\) and \(f(y) = y+1\). Since \(f(x) = f(y)\), we have \(x + 1 = y + 1\) which implies that \(x = y\). Similarly, if \(x\) and \(y\) are both odd, then \(x - 3 = y-3\) so again \(x = y\). The only other possibility is that \(x\) is even an \(y\) is odd (or visa-versa). But then \(x + 1\) would be odd and \(y - 3\) would be even, so it cannot be that \(f(x) = f(y)\). Therefore if \(f(x) = f(y)\) we then have \(x = y\), which proves that \(f\) is injective.
                %
\end{proof}
%
\item\hypertarget{li-365}{}\(f\) is surjective.

              \begin{proof}\hypertarget{proof-2}{}

                  Let \(y\) be an element of the codomain \(\Z\). We will show there is an element \(n\) of the domain (\(\Z\)) such that \(f(n) = y\). There are two cases: First, if \(y\) is even, then let \(n = y+3\). Since \(y\) is even, \(n\) is odd, so \(f(n) = n-3 = y+3-3 = y\) as desired. Second, if \(y\) is odd, then let \(n = y-1\). Since \(y\) is odd, \(n\) is even, so \(f(n) = n+1 = y-1+1 = y\) as needed. Therefore \(f\) is surjective.
                %
\end{proof}
%
\end{enumerate}
\par\smallskip
\noindent\textbf{13.}\quad{}
            Yes, this is a function, if you choose the domain and codomain correctly. The domain will be the set of students, and the codomain will be the set of possible grades. The function is almost certainly not injective, because it is likely that two students will get the same grade. The function might be surjective \textendash{} it will be if there is at least one student who gets each grade.
          %
\par\smallskip
\noindent\textbf{14.}\quad{}
            Yes, as long as the set of cards is the domain and the set of players is the codomain. The function is not injective because multiple cards go to each player. It is surjective since all players get cards.
          %
\par\smallskip
\noindent\textbf{15.}\quad{}
            This cannot be a function. If the domain were the set of cards, then it is not a function because not every card gets dealt to a player. If the domain were the set of players, it would not be a function because a single player would get mapped to multiple cards. Since this is not a function, it doesn't make sense to say whether it is injective/surjective/bijective.
          %
\par\smallskip
\subsection*{2.1.3 Exercises}
\noindent\textbf{1.}\quad{}
          There are 255 outfits. Use the multiplicative principle.
        %
\par\smallskip
\noindent\textbf{2.}\quad{}\leavevmode%
\begin{enumerate}[label=(\alph*)]
\item\hypertarget{li-383}{}
              8 ties.  Use the additive principle.
            %
\item\hypertarget{li-384}{}15 ties. Use the multiplicative principle%
\item\hypertarget{li-385}{}\(5\cdot (4+3) + 7 = 42\) outfits.%
\end{enumerate}
\par\smallskip
\noindent\textbf{3.}\quad{}\leavevmode%
\begin{enumerate}[label=(\alph*)]
\item\hypertarget{li-388}{}
              For example, 16 is the number of choices you have if you want to watch one movie, either a comedy or horror flick.
            %
\item\hypertarget{li-389}{}
              For example, 63 is the number of choices you have if you will watch two movies, first a comedy and then a horror.
            %
\end{enumerate}
\par\smallskip
\noindent\textbf{4.}\quad{}
          \leavevmode%
\begin{enumerate}[label=(\alph*)]
\item\hypertarget{li-393}{}\(A\) and \(B\) might have no elements in common, giving \(\card{A\cap B} = 0\).%
\item\hypertarget{li-394}{}To maximize the number of elements in common between \(A\) and \(B\), make \(A \subset B\).  This would give \(\card{A \cap B} = 10\).%
\item\hypertarget{li-395}{}\(15 \le \card{A \cup B} \le 25\).  In fact, when \(\card{A \cap B} = 0\) then \(\card{A \cup B} = 25\) and when \(\card{A \cap B} = 10\) then \(\card{A \cup B} = 15\). %
\end{enumerate}

        %
\par\smallskip
\noindent\textbf{5.}\quad{}
          \(\card{A \cup B} + \card{A \cap B} = 13\).  Use PIE: we know \(\card{A \cup B} = 8 + 5 - \card{A \cap B}\).
        %
\par\smallskip
\noindent\textbf{6.}\quad{}
          39 students.  Use PIE or a Venn diagram.
        %
\par\smallskip
\noindent\textbf{7.}\quad{}
          \leavevmode%
\begin{enumerate}[label=(\alph*)]
\item\hypertarget{li-398}{}\(\card{(A \cup C)\setminus B} = 44\). Use PIE or a Venn diagram.
            %
\item\hypertarget{li-399}{}
              One possibility: \((A \cup B) \cap C\).  Here using a Venn diagram is quite a bit easier.
            %
\end{enumerate}

        %
\par\smallskip
\noindent\textbf{8.}\quad{}\leavevmode%
\begin{enumerate}[label=(\alph*)]
\item\hypertarget{li-405}{}\(8^5 = 32768\) words, since you select from 8 letters 5 times.%
\item\hypertarget{li-406}{}\(8\cdot 7\cdot 6\cdot 5\cdot 4 = 6720\) words. After selecting a letter, you have fewer letters to select for the next one.%
\item\hypertarget{li-407}{}
              \(8 \cdot 8 =64\)words: you need to select the 4th and 5th letters.
            %
\item\hypertarget{li-408}{}\(64 + 64 - 0 = 128\) words. There are 64 words which start with ``aha'' and another 64 words that end with ``bah.'' Perhaps we over counted the words that both start with ``aha'' and end with ``bah'', but since the words are only 5 letters long, there are no such words.%
\item\hypertarget{li-409}{}\((8\cdot 7\cdot 6\cdot 5\cdot 4) - 3\cdot (5\cdot 4) = 6660\) words. All the words minus the bad ones. The taboo word can be in any of three positions (starting with letter 1, 2, or 3) and for each position we must choose the other two letters (from the remaining 5 letters).%
\end{enumerate}
\par\smallskip
\subsection*{2.2.6 Exercises}
\noindent\textbf{1.}\quad{}\leavevmode%
\begin{enumerate}[label=(\alph*)]
\item\hypertarget{li-427}{}\(2^6 = 64\) subsets. We need to select yes/no for each of the six elements.%
\item\hypertarget{li-428}{}\(2^3 = 8\) subsets.  We need to select yes/no for each of the remaining three elements.%
\item\hypertarget{li-429}{}\(2^6 - 2^3 = 56\) subsets.  There are 8 subsets which do not contain any odd numbers (select yes/no for each even number).%
\item\hypertarget{li-430}{}\(3\cdot 2^3 = 24\) subsets.  First pick the even number.  Then say yes or no to each of the odd numbers.%
\end{enumerate}
\par\smallskip
\noindent\textbf{2.}\quad{}\leavevmode%
\begin{enumerate}[label=(\alph*)]
\item\hypertarget{li-435}{}\({6\choose 4} = 15\).%
\item\hypertarget{li-436}{}\({3 \choose 1} = 3\).  We need to select 1 of the 3 remaining elements to be in the subset.%
\item\hypertarget{li-437}{}\({6 \choose 4} = 15\).  All subsets of cardinality 4 must contain at least one odd number.%
\item\hypertarget{li-438}{}\({3 \choose 1} = 3\).  Select 1 of the 3 even numbers.  The remaining three odd numbers of \(S\) must all be in the set.%
\end{enumerate}
\par\smallskip
\noindent\textbf{3.}\quad{}\leavevmode%
\begin{enumerate}[label=(\alph*)]
\item\hypertarget{li-441}{} We can think of each row as a 6-bit string of weight 3 (since of the 6 coins, we require 3 to be pennies).  Thus there are \({6 \choose 3} = 20\) rows possible.  Each row requires 6 coins, so if we want to make all the rows at the same time, we will need 120 coins (60 of each). %
\item\hypertarget{li-442}{} Now there are \(2^6 = 64\) rows possible, which is also \({6 \choose 0} + {6\choose 1} + {6 \choose 2} + {6 \choose 3} + {6 \choose 4} + {6 \choose 5} + {6 \choose 6}\), if you break them up into rows containing 0, 1, 2, etc. pennies.  Thus we need \(6 \cdot 64 = 384\) coins (192 of each). %
\end{enumerate}
\par\smallskip
\noindent\textbf{4.}\quad{}
            \({10 \choose 6} + {10\choose 7} + {10\choose 8} + {10 \choose 9} + {10\choose 10} = 386\) strings.  Count the number of strings with each permissible number of 1's separately, then add them up.
          %
\par\smallskip
\noindent\textbf{5.}\quad{}
            \({10 \choose 6} + {10\choose 7} + {10\choose 8} + {10 \choose 9} + {10\choose 10} = 386\) subsets. This is the same as the previous question, since we can think of each subset as a 10-bit string with a 1 representing that we include that element in the subset.
          %
\par\smallskip
\noindent\textbf{6.}\quad{}
            To get an \(x^{12}\), we must pick 12 of the 15 factors to contribute an \(x\), leaving the other 3 to contribute a 2. There are \({15 \choose 12}\) ways to select these 12 factors. So the term containing an \(x^{12}\) will be \({15 \choose 12}x^{12}2^{3}\). In other words, the coefficient of \(x^{12}\) is \({15\choose 12}2^3 = 3640\).
          %
\par\smallskip
\noindent\textbf{7.}\quad{}
            Reason as in the previous question. We get \({14\choose 9} + {15 \choose 6}2^9\).
          %
\par\smallskip
\noindent\textbf{8.}\quad{}\leavevmode%
\begin{enumerate}[label=(\alph*)]
\item\hypertarget{li-446}{}\({14 \choose 7} = 3432\) paths.  The paths all have length 14 (7 steps up and 7 steps right), we just select which 7 of those 14 should be up.%
\item\hypertarget{li-447}{}\({6 \choose 2}{8\choose 5} = 840\) paths.  First travel to (5,7), and then continue on to (10,10).%
\item\hypertarget{li-448}{}\({14 \choose 7} - {6\choose 2}{8 \choose 5}\) paths.  Remove all the paths that you found in part (b).%
\end{enumerate}
\par\smallskip
\noindent\textbf{9.}\quad{}\leavevmode%
\begin{enumerate}[label=(\alph*)]
\item\hypertarget{li-453}{}\({11 \choose 3} = 165\) choices, since you have to select a 3-element subset of the set of 11 toppings.%
\item\hypertarget{li-454}{}\({10 \choose 3} = 120\) choices, since you must select 3 of the 10 non-pineapple toppings.%
\item\hypertarget{li-455}{}\({10 \choose 2} = 45\) choices, since you must select 2 of the remaining 10 non-pineapple toppings to have in addition to the pineapple.%
\item\hypertarget{li-456}{}\(165  = 120 + 45\) choices, which makes sense because every 3-topping pizza either has pineapple or does not have pineapple as a topping.%
\end{enumerate}
\par\smallskip
\noindent\textbf{10.}\quad{}
            The coefficient of \(x^5y^3\) is \({8\choose 5}\), since we must pick 5 of the 8 factors to contribute an \(x\). The coefficient of \(x^3y^5\) is \({8 \choose 3}\), since we pick 3 out of the 8 factors to contribute an \(x\). But \({8 \choose 5} = {8\choose 3}\), because we could just as easily have picked 5 out of the 8 factors to contribute a \(y\).
          %
\par\smallskip
\subsection*{2.3.1 Exercises}
\noindent\textbf{1.}\quad{}\leavevmode%
\begin{enumerate}[label=(\alph*)]
\item\hypertarget{li-468}{}\({10 \choose 3} = 120\) pizzas.  We must choose (in no particular order) 3 out of the 10 toppings.%
\item\hypertarget{li-469}{}\(2^{10} = 1024\) pizzas.  Say yes or no to each topping.%
\item\hypertarget{li-470}{}\(P(10,5) 30240\) ways.  Assign each of the 5 spots in the left column to a unique pizza topping.%
\end{enumerate}
\par\smallskip
\noindent\textbf{2.}\quad{}
                Despite its name, we are not looking for a combination here. The order in which the three numbers appears matters. There are \(P(40,3) = 40\cdot 39 \cdot 38\) different possibilities for the ``combination''. This is assuming you cannot repeat any of the numbers (if you could, the answer would be \(40^3\)).
              %
\par\smallskip
\noindent\textbf{3.}\quad{}\leavevmode%
\begin{enumerate}[label=(\alph*)]
\item\hypertarget{li-475}{} This is just the multiplicative principle.  There are 7 digits which we can select for each of the 5 positions, so we have \(7^5 = 16807\) such numbers. %
\item\hypertarget{li-476}{} Now we have 7 choices for the first number, 6 for the second, etc.  So there are \(7 \cdot 6 \cdot 5 \cdot 4 \cdot 3 = P(7,5) = 2520\) such numbers. %
\item\hypertarget{li-477}{} To build such a number we simply must select 5 different digits.  After doing so, there will only be one way to arrange them into a 5-digit number.  Thus there are \({7 \choose 5} = 21\) such numbers. %
\item\hypertarget{li-478}{} The permutation is in part (b), while the combination is in part (c).  At first this seems backwards, since usually we use combinations for when order does not matter.  Here it looks like in part (c) that order does matter.  The better way to distinguish between combinations and permutations is to ask whether we are counting different arrangements as different outcomes.  In part (c), there is only one arrangement of any set of 5 digits, while in part (b) each set of 5 digits gives \(5!\) different outcomes. %
\end{enumerate}
\par\smallskip
\noindent\textbf{4.}\quad{}
                \({7\choose 2}{7\choose 2} = 441\) quadrilaterals. We must pick two of the seven dots from the top row and two of the seven dots on the bottom row. However, it does not make a difference which of the two (on each row) we pick first because once these four dots are selected, there is exactly one quadrilateral that they determine.
              %
\par\smallskip
\noindent\textbf{5.}\quad{}\leavevmode%
\begin{enumerate}[label=(\alph*)]
\item\hypertarget{li-484}{} 5 squares. You need to skip exactly one dot on the top and on the bottom to make the side lengths equal.  Once you pick a dot on the top, the other three dots are determined. %
\item\hypertarget{li-485}{}\({7 \choose 2}\) rectangles.  Once you select the two dots on the top, the bottom two are determined.%
\item\hypertarget{li-486}{} This is tricky since you need to worry about running out of space.  One way to count: break into cases by the location of the top left corner.  You get \({7 \choose 2} + ({7 \choose 2}-1) + ({7 \choose 2} - 3) + ({7 \choose 2} - 6) + ({7 \choose 2} - 10) + ({7 \choose 2} - 15) = 91\) parallelograms. %
\item\hypertarget{li-487}{} All of them %
\item\hypertarget{li-488}{} All of them, except the parallelograms.  So \({7\choose 2}{7\choose 2} - \left[ {7 \choose 2} + ({7 \choose 2}-1) + ({7 \choose 2} - 3) + ({7 \choose 2} - 6) + ({7 \choose 2} - 10) + ({7 \choose 2} - 15) \right]\). %
\end{enumerate}
\par\smallskip
\noindent\textbf{6.}\quad{}
                Since there are 15 different letters, we have 15 choices for the first letter, 14 for the next, and so on. Thus there are \(15!\) anagrams.
              %
\par\smallskip
\noindent\textbf{7.}\quad{}
                After the first letter (a), we must rearrange the remaining 7 letters. There are only two letters (s and e), so this is really just a bit-string question (think of s as 1 and e as 0). Thus there \({7 \choose 2} = 21\) anagrams starting with ``a''.
              %
\par\smallskip
\noindent\textbf{8.}\quad{}
                First, decide where to put the ``a''s. There are 7 positions, and we must choose 3 of them to fill with an ``a''. This can be done in \({7 \choose 3}\) ways. The remaining 4 spots all get a different letter, so there are \(4!\) ways to finish off the anagram. This gives a total of \({7 \choose 3}\cdot 4!\) anagrams. Strangely enough, this is 840, which is also equal to \(P(7,4)\). To get the answer that way, start by picking one of the 7 \emph{positions} to be filled by the ``n'', one of the remaining 6 positions to be filled by the ``g'', one of the remaining 5 positions to be filled by the ``r'', one of the remaining 4 positions to be filled by the ``m'' and then put ``a''s in the remaining 3 positions.
              %
\par\smallskip
\noindent\textbf{9.}\quad{}\leavevmode%
\begin{enumerate}[label=(\alph*)]
\item\hypertarget{li-491}{}\({20 \choose 4}{16 \choose 4}{12 \choose 4}{8 \choose 4}{4 \choose 4}\) ways. Pick 4 out of 20 people to be in the first foursome, then 4 of the remaining 16 for the second foursome, and so on (use the multiplicative principle to combine).%
\item\hypertarget{li-492}{}\(5!{15 \choose 3}{12 \choose 3}{9 \choose 3}{6 \choose 3}{3 \choose 3}\) ways.  First determine the tee time of the 5 board members, then select 3 of the 15 non board members to golf with the first board member, then 3 of the remaining 12 to golf with the second, and so on.%
\end{enumerate}
\par\smallskip
\noindent\textbf{10.}\quad{}
                \(9!\) (there are 10 people seated around the table, but it does not matter where King Arthur sits, only who sits to his left, two seats to his left, and so on).
              %
\par\smallskip
\noindent\textbf{11.}\quad{}\leavevmode%
\begin{enumerate}[label=(\alph*)]
\item\hypertarget{li-495}{}\(17^{10}\) functions.  There are 17 choices for the image of each element in the domain.%
\item\hypertarget{li-496}{}\(P(17, 10)\) injective functions.  There are 17 choices for image of the first element of the domain, then only 16 choices for the second, and so on.%
\end{enumerate}
\par\smallskip
\noindent\textbf{12.}\quad{}\leavevmode%
\begin{enumerate}[label=(\alph*)]
\item\hypertarget{li-500}{}\(6^4 = 1296\)functions, since there are six choices of where to send each of the 4 elements of the domain.%
\item\hypertarget{li-501}{}\(P(6, 4) = 6 \cdot 5 \cdot 4 \cdot 3 = 360\)functions, since outputs cannot be repeated.%
\item\hypertarget{li-502}{}\({6 \choose 4} = 15\) functions. Since the function must be injective and increasing, we just need to select the four distinct elements of the range from the six elements of the codomain.  Once selected, we must put the smallest as the image of 1, the next smallest as  the image of 2, and so on (doing this does not increase the number of functions, since there is one choice for how this event can occur).  %
\end{enumerate}
\par\smallskip
\subsection*{2.4.3 Exercises}
\noindent\textbf{1.}\quad{}\begin{proof}\hypertarget{proof-7}{}

            Question: How many subsets of size \(k\) are there of the set \(\{1,2,\ldots, n\}\)?
          %
\par

            Answer 1: You must choose \(k\) out of \(n\) elements to put in the set, which can be done in \({n \choose k}\) ways.
          %
\par

            Answer 2: First count the number of \(k\)-element subsets of \(\{1,2,\ldots, n\}\) which contain the number \(n\). We must choose \(k-1\) of the \(n-1\) other element to include in this set. Thus there are \({n-1\choose k-1}\) such subsets. We have not yet counted all the \(k\)-element subsets of \(\{1,2,\ldots, n\}\) though. In fact, we have missed exactly those subsets which do NOT contain \(n\). To form one of these subsets, we need to choose \(k\) of the other \(n-1\) elements, so this can be done in \({n-1 \choose k}\) ways. Thus the answer to the question is \({n-1 \choose k-1} + {n-1 \choose k}\).
          %
\par

             Since the two answers are both answers tot eh same question, they are equal, establishing the identity \({n\choose k} = {n-1 \choose k-1} + {n-1 \choose k}\).
           %
\end{proof}
\par\smallskip
\noindent\textbf{2.}\quad{}\begin{proof}\hypertarget{proof-8}{}

            Question: How many 2-letter words start with \emph{a}, \emph{b}, or \emph{c} and end with either \emph{y} or \emph{z}?
          %
\par

            Answer 1: There are two words that start with \emph{a}, two that start with \emph{b}, two that start with \emph{c}, for a total of \(2+2+2\).
          %
\par

            Answer 2: There are three choices for the first letter and two choices for the second letter, for a total of \(3 \cdot 2\).
          %
\par

            Since the two answers are both answers to the same question, they are equal. Thus \(2 + 2 + 2 = 3\cdot 2\).
          %
\end{proof}
\par\smallskip
\noindent\textbf{3.}\quad{}\begin{proof}\hypertarget{proof-9}{}

            Question: How many subsets of \(A = {1,2,3, \ldots, n+1}\) contain exactly two elements?
          %
\par

            Answer 1: We must choose 2 elements from \(n+1\) choices, so there are \({n+1 \choose 2}\) subsets.
          %
\par

            Answer 2: We break this question down into cases, based on what the larger of the two elements in the subset is. The larger element can't be 1, since we need at least one element smaller than it.
          %
\par

            Larger element is 2: there is 1 choice for the smaller element.
          %
\par

            Larger element is 3: there are 2 choices for the smaller element.
          %
\par

            Larger element is 4: there are 3 choices for the smaller element.
          %
\par

            And so on. When the larger element is \(n+1\), there are \(n\) choices for the smaller element. Since each two element subset must be in exactly one of these cases, the total number of two element subsets is \(1 + 2 + 3 + \cdots + n\).
          %
\par

            Answer 1 and answer 2 are both correct answers to the same question, so they must be equal. Therefore,
            \begin{equation*}
              1 + 2 + 3 + \cdots + n = {n+1 \choose 2}
            \end{equation*}
          %
\end{proof}
\par\smallskip
\noindent\textbf{4.}\quad{}\leavevmode%
\begin{enumerate}[label=(\alph*)]
\item\hypertarget{li-528}{} She has \({15 \choose 6}\) ways to select the 6 bridesmaids, and then for each way, has 6 choices for the maid of honor.  Thus she has \({15 \choose 6}6\) choices. %
\item\hypertarget{li-529}{} She has 15 choices for who will be her maid of honor.  Then she needs to select 5 of the remaining 14 friends to be bridesmaids, which she can do in \({14 \choose 5}\) ways.  Thus she has \(15 {14 \choose 5}\) choices. %
\item\hypertarget{li-530}{} We have answered the question (how many wedding parties can the bride choose from) in two ways.  The first way gives the left-hand side of the identity and the second way gives the right-hand side of the identity.  Therefore the identity holds. %
\end{enumerate}
\par\smallskip
\noindent\textbf{5.}\quad{}\begin{proof}\hypertarget{proof-10}{}

            Question: You have a large container filled with ping-pong balls, all with a different number of them. You must select \(k\) of the balls, putting two of them in a jar and the others in a box. How many ways can you do this?
          %
\par

            Answer 1: First select 2 of the \(n\) balls to put in the jar. Then select \(k-2\) of the remaining \(n-2\) balls to put in the box. The first task can be completed in \({n \choose 2}\) different ways, the second task in \({n-2 \choose k-2}\) ways. Thus there are \({n \choose 2}{n-2 \choose k-2}\) ways to select the balls.
          %
\par

            Answer 2: First select \(k\) balls from the \(n\) in the container. Then pick 2 of the \(k\) balls you picked to put in the jar, placing the remaining \(k-2\) in the box. The first task can be completed in \({n \choose k}\) ways, the second task in \({k \choose 2}\) ways. Thus there are \({n \choose k}{k \choose 2}\) ways to select the balls.
          %
\par

            Since both answers count the same thing, they must be equal and the identity is established.
          %
\end{proof}
\par\smallskip
\noindent\textbf{6.}\quad{}\leavevmode%
\begin{enumerate}[label=(\alph*)]
\item\hypertarget{li-536}{} After the 1, we need to find a 5-bit string with one 1.  There are \({5 \choose 1}\) ways to do this. %
\item\hypertarget{li-537}{}\({4 \choose 1}\) strings (we need to pick 1 of the remaining 4 slots to be the second 1).%
\item\hypertarget{li-538}{}\({3 \choose 1}\) strings.%
\item\hypertarget{li-539}{} Yes.  We still need strings starting with 0001 (there are \({2 \choose 1}\) of these) and strings starting 00001 (there is only \({1 \choose 1} = 1\) of these). %
\item\hypertarget{li-540}{}\({6 \choose 2}\) strings%
\item\hypertarget{li-541}{} An example of the Hockey Stick Theorem: \begin{equation*} {1 \choose 1} + {2 \choose 1} + {3 \choose 1} + {4 \choose 1} + {5 \choose 1} = {6 \choose 2} \end{equation*} %
\end{enumerate}
\par\smallskip
\noindent\textbf{7.}\quad{}\leavevmode%
\begin{enumerate}[label=(\alph*)]
\item\hypertarget{li-549}{}\(3^n\) strings, since there are 3 choices for each of the \(n\) digits.%
\item\hypertarget{li-550}{}\(1\) string, since all the digits need to be 2's.  However, we might write this as \({n \choose 0}\) strings.%
\item\hypertarget{li-551}{} There are \({n \choose 1}\) places to put the non-2 digit.  That digit can be either a 0 or a 1, so there are \(2{n \choose 1}\) such strings. %
\item\hypertarget{li-552}{} We must choose two slots to fill with 0's or 1's.  There are \({n \choose 2}\) ways to do that.  Once the slots are picked, we have two choices for the first slot (0 or 1) and two choices for the second slot (0 or 1).  So there are a total of \(2^2{n \choose 2}\) such strings. %
\item\hypertarget{li-553}{} There are \({n \choose k}\) ways to pick which slots don't have the 2's.  Then those slots can be filled in \(2^k\) ways (0 or 1 for each slot).  So there are \(2^k{n \choose k}\) such strings. %
\item\hypertarget{li-554}{} These strings contain just 0's and 1's, so they are bit strings.  There are \(2^n\) bit strings.  But keeping with the pattern above, we might write this as \(2^n {n \choose n}\) strings. %
\item\hypertarget{li-555}{} We answer the question of how many length \(n\) ternary digit strings there are in two ways.  First, each digit can be one of three choices, so the total number of strings is \(3^n\).  On the other hand, we could break the question down into cases by how many of the digits are 2's.  If they are all 2's, then there are \({n \choose 0}\) strings.  If all but one is a 2, then there are \(2{n \choose 1}\) strings.  If all but 2 of the digits are 2's, then there are \(2^2{n \choose 2}\) strings.  We choose 2 of the \(n\) digits to be non-2, and then there are 2 choices for each of those digits.  And so on for every possible number of 2's in the string. Therefore \( {n \choose 0} + 2{n \choose 1} + 2^2{n \choose 2} + 2^3{n \choose 3} + \cdots + 2^n{n \choose n} = 3^n. \) %
\end{enumerate}
\par\smallskip
\noindent\textbf{8.}\quad{}
          The word contains 9 letters: 3 ``r''s, 2 ``a''s and 2 ``e''s, along with an ``n'' and a ``g''. We could first select the positions for the ``r''s in \({9 \choose 3}\) ways, then the ``a''s in \({6 \choose 2}\) ways, the ``e''s in \({4 \choose 2}\) ways and then select one of the remaining two spots to put the ``n'' (placing the ``g'' in the last spot). This gives the answer
          \begin{equation*}
            {9 \choose 3}{6 \choose 2}{4 \choose 2}{2\choose 1}{1\choose 1}.
          \end{equation*}
        %
\par

          Alternatively, we could select the positions of the letters in the opposite order, which would give an answer
          \begin{equation*}
            {9 \choose 1}{8\choose 1}{7 \choose 2}{5\choose 2}{3\choose 3}.
          \end{equation*}
        %
\par

          (where the 3 ``r''s go in the remaining 3 spots). These two expressions are equal:
          \begin{equation*}
            {9 \choose 3}{6 \choose 2}{4 \choose 2}{2\choose 1}{1\choose 1} = {9 \choose 1}{8\choose 1}{7 \choose 2}{5\choose 2}{3\choose 3}.
          \end{equation*}
        %
\par\smallskip
\noindent\textbf{9.}\quad{}\begin{proof}\hypertarget{proof-11}{}

            Question: How many \(k\)-letter words can you make using \(n\) different letters without repeating any letter?
          %
\par

            Answer 1: There are \(n\) choices for the first letter, \(n-1\) choices for the second letter, \(n-2\) choices for the third letter, and so on until \(n - (k-1)\) choices for the \(k\)th letter (since \(k-1\) letters have already been assigned at that point). The product of these numbers can be written \(\frac{n!}{(n-k)!}\) which is \(P(n,k)\).  Therefore there are \(P(n,k)\) words.
          %
\par

            Answer 2: First pick \(k\) letters to be in the word from the \(n\) choices. This can be done in \({n \choose k}\) ways. Now arrange those letters into a word. There are \(k\) choices for the first letter, \(k-1\) choices for the second, and so on, for a total of \(k!\) arrangements of the \(k\) letters. Thus the total number of words is \({n \choose k}k!\).
          %
\par

            Since the two answers are correct answers to the same question, we have established that \(P(n,k) = {n \choose k}k!\).
          %
\end{proof}
\par\smallskip
\noindent\textbf{10.}\quad{}\begin{proof}\hypertarget{proof-12}{}

            Question: How many 5-element subsets are there of the set \(\{1,2,\ldots, n+3\}\).
          %
\par

            Answer 1: We choose 5 out of the \(n+3\) elements, so \({n+3 \choose 5}\) subsets.
          %
\par

            Answer 2: Break this up into cases by what the ``middle'' (third smallest) element of the 5 element subset is. The smallest this could be is a 3. In that case, we have \({2 \choose 2}\) choices for the numbers below it, and \({n \choose 2}\) choices for the numbers above it. Alternatively, the middle number could be a 4. In this case there are \({3 \choose 2}\) choices for the bottom two numbers and \({n-1 \choose 2}\) choices for the top two numbers. If the middle number is 5, then there are \({4 \choose 2}\) choices for the bottom two numbers and \({n-2 \choose 2}\) choices for the top two numbers. An so on, all the way up to the largest the middle number could be, which is \(n+1\). In that case there are \({n \choose 2}\) choices for the bottom two numbers and \({2 \choose 2}\) choices for the top number. Thus the number of 5 element subsets is
            \begin{equation*}
              {2 \choose 2}{n \choose 2} + {3 \choose 2}{n-1 \choose 2} + {4\choose 2}{n-2 \choose 2} + \cdots + {n\choose 2}{2\choose 2}.
            \end{equation*}
          %
\par

            Since the two answers correctly answer the same question, we have
            \begin{equation*}
              {2 \choose 2}{n \choose 2} + {3 \choose 2}{n-1 \choose 2} + {4\choose 2}{n-2 \choose 2} + \cdots + {n\choose 2}{2\choose 2} = {n+3 \choose 5}.
            \end{equation*}
          %
\end{proof}
\par\smallskip
\subsection*{2.5.1 Exercises}
\noindent\textbf{1.}\quad{}\leavevmode%
\begin{enumerate}[label=(\alph*)]
\item\hypertarget{li-573}{}\({10\choose 5}\) sets.  We must select 5 of the 10 digits to put in the set.%
\item\hypertarget{li-574}{}
Use stars and bars: each star represents one of the 5 elements of the set, each bar represents a switch between digits.  So there are 5 stars and 9 bars, giving us \({14 \choose 9}\) sets.
%
\end{enumerate}
\par\smallskip
\noindent\textbf{2.}\quad{}\leavevmode%
\begin{enumerate}[label=(\alph*)]
\item\hypertarget{li-579}{}
You take 3 strawberry, 1 lime, 0 licorice, 2 blueberry and 0 bubblegum.
%
\item\hypertarget{li-580}{}
This is backwards.  We don't want the stars to represent the kids because the kids are not identical, but the stars are.  Instead we should use 5 stars (for the lollipops) and use 5 bars to switch between the 6 kids.  For example, **||***||| would represent the outcome with the first kid getting 2 lollipops, the third kid getting 3, and the rest of the kids getting none.
%
\item\hypertarget{li-581}{}
This is the word AAAEOO.
%
\item\hypertarget{li-582}{}
This doesn't represent a solution.  Each star should represent one of the 6 units that add up to 6, and the bars should \emph{switch} between the different variables.  We have one too many bars.  An example of a correct diagram would be *|**||***, representing that \(x_1 = 1\), \(x_2 = 2\), \(x_3 = 0\), and   \(x_4 = 3\).
%
\end{enumerate}
\par\smallskip
\noindent\textbf{3.}\quad{}\leavevmode%
\begin{enumerate}[label=(\alph*)]
\item\hypertarget{li-585}{}\({18 \choose 4}\) ways.  Each outcome can be represented by a sequence of 14 stars and 4 bars.%
\item\hypertarget{li-586}{}\({13 \choose 4}\) ways.  First put one ball in each bin.  This leaves 9 stars and 4 bars.%
\end{enumerate}
\par\smallskip
\noindent\textbf{4.}\quad{}\leavevmode%
\begin{enumerate}[label=(\alph*)]
\item\hypertarget{li-590}{}\({7 \choose 2}\) solutions.  After each variable gets 1 star for free, we are left with 5 stars and 2 bars.%
\item\hypertarget{li-591}{}\({10 \choose 2}\) solutions.  We have 8 stars and 2 bars.%
\item\hypertarget{li-592}{}\({19 \choose 2}\) solutions.  This problem is equivalent to finding the number of solutions to \(x' + y' + z' = 17\) where \(x'\), \(y'\) and \(z'\) are non-negative.  (In fact, we really just do a substitution.  Let \(x = x'- 3\), \(y = y' - 3\) and \(z = z' - 3\)).%
\end{enumerate}
\par\smallskip
\noindent\textbf{5.}\quad{}\leavevmode%
\begin{enumerate}[label=(\alph*)]
\item\hypertarget{li-595}{}
There are \({7 \choose 5}\) numbers.  We simply choose five of the seven digits and once chosen put them in increasing order.
%
\item\hypertarget{li-596}{}
This requires stars and bars.  Use a star to represent each of the 5 digits in the number, and use their position relative to the bars to say what numeral fills that spot.  So we will have 5 stars and 6 bars, giving \({11 \choose 6}\) numbers.
%
\end{enumerate}
\par\smallskip
\noindent\textbf{6.}\quad{}
\({10 \choose 5}\) outcomes. We have 5 stars (the five dice) and 5 bars (the five switches between the numbers 1-6).
%
\par\smallskip
\noindent\textbf{7.}\quad{}
We must figure out how many different combinations of 7 coins are possible. Let a star represent each coin, and a bar represent switching between type of coin. For example, **|*||**** represents 2 pennies, one nickel, no dimes and 4 quarters. The number of such star and bar diagrams (with 7 stars and 3 bars) is \({10 \choose 3} = 120\). Thus you have a 1 in 120 chance of guessing correctly.
%
\par\smallskip
\noindent\textbf{8.}\quad{}
\({18 \choose 3}\) solutions. Distribute 10 units to the variables before finding all solutions to \(x_1' + x_2' + x_3' + x_4' = 15\) in non-negative integers.
%
\par\smallskip
\noindent\textbf{9.}\quad{}
The answer to each of the counting questions is \({10 \choose 2}\), using 8 stars and 2 bars. The bars separate the kids, the variables, and the colors. To see why these are really the same, notice that we can think of the 3 kids as named \(x\), \(y\), and \(z\). Or think of \(x\) as red, \(y\) as blue and \(z\) as yellow. Notice that in each case, the three things are distinguishable, while the 8 things (cookies, units, crayons) are all identical.
%
\par\smallskip
\noindent\textbf{10.}\quad{}\leavevmode%
\begin{enumerate}[label=(\alph*)]
\item\hypertarget{li-604}{}\({20 \choose 4}\) sodas (order does not matter and repeats are not allowed).%
\item\hypertarget{li-605}{}\(P(20, 4) = 20\cdot 19\cdot 18 \cdot 17\) sodas (order matters and repeats are not allowed).%
\item\hypertarget{li-606}{}\({23 \choose 19}\) sodas (order does not matter and repeats are allowed; 4 stars and 19 bars).%
\item\hypertarget{li-607}{}\(20^4\) sodas (order matters and repeats are allowed; 20 choices 4 times).%
\end{enumerate}
\par\smallskip
\subsection*{2.6.3 Exercises}
\noindent\textbf{1.}\quad{}
            Hint: Stars and bars.
          %
\leavevmode%
\begin{enumerate}[label=(\alph*)]
\item\hypertarget{li-641}{}\({9 \choose 6}\).%
\item\hypertarget{li-642}{}\({16 \choose 6}\).%
\item\hypertarget{li-643}{}\({16 \choose 6} - \left[{7 \choose 1}{13 \choose 6} - {7 \choose 2}{10 \choose 6} + {7 \choose 3}{7 \choose 6}\right]\).%
\end{enumerate}
\par\smallskip
\noindent\textbf{2.}\quad{}
            \({18 \choose 4} - \left[ {5 \choose 1}{11 \choose 4} - {5 \choose 2}{4 \choose 4}\right]\). Subtract all the distributions for which one or more bins contain 7 or more balls.
          %
\par\smallskip
\noindent\textbf{3.}\quad{}
            The easiest way to solve this is to instead count the solutions to \(y_1 + y_2 + y_3 + y_4 = 7\) with \(0 \le y_i \le 3\). By taking \(x_i = y_i+2\), each solution to this new equation corresponds to exactly one solution to the original equation.
          %
\par

            Now all the ways to distribute the 7 units to the four \(y_i\) variables can be found using stars and bars, specifically 7 stars and 3 bars, so \({10 \choose 3}\). But this includes the ways that one or more \(y_i\) variables can be assigned more than 3 units. So subtract, using PIE. We get
            \begin{equation*}
              {10 \choose 3} - {4\choose 1} {6 \choose 3}.
            \end{equation*}
          %
\par

            Note that this is the final answer because it is not possible to have two variables both get 4 units.
          %
\par\smallskip
\noindent\textbf{4.}\quad{}
            Without any restriction, there would be \({19\choose 12}\) ways to distribute the stars. Now we must use PIE to eliminate all distributions in which one or more student gets more than one star:
            \begin{equation*}
              {19 \choose 12} - \left[{13 \choose 1}{17 \choose 12} - {13\choose 2}{15 \choose 12} + {13\choose 3}{13 \choose 12}\right] = 1716.
            \end{equation*}
          %
\par

            Interestingly enough, this number is also the value of \({13 \choose 7}\), which makes sense: if each student can have at most one star, we must just pick the 7 out of 13 students to receive them.
          %
\par\smallskip
\noindent\textbf{5.}\quad{}
            The question is, how many ways can you distribute \(k\) cookies to \(n\) kids so that each kid gets at most one cookie. On one hand, the answer is just \({n \choose k}\) since you must choose \(k\) kids to get a cookie. Alternatively, we can use stars and bars with PIE, which is how we get the right hand side of the identity. Note that lots of the terms on the right hand side will be zero, as soon as \(n+k-(2j+1)\) drops below \(k\).
          %
\par\smallskip
\noindent\textbf{6.}\quad{}
            The 9 derangements are: 2143, 2341, 2413, 3142, 3412, 3421, 4123, 4312, 4321.
          %
\par\smallskip
\noindent\textbf{7.}\quad{}
            \({10 \choose 6}\left(4! - \left[{4 \choose 1} 3! - {4 \choose 2}2! + {4 \choose 3}1! - {4 \choose 4}0!\right]\right)\). We choose 6 of the 10 ladies to get their own hat, and the multiply by the number of ways the remaining hats can be deranged.
          %
\par\smallskip
\noindent\textbf{8.}\quad{}
      There are \(5 \cdot 6^3\) functions for which \(f(1) \ne a\) and another \(5 \cdot 6^3\) functions for which \(f(2) \ne b\).  There are \(5^2 \cdot 6^2\) functions for which both \(f(1) \ne a\) and \(f(2) \ne b\).  So the total number of functions for which \(f(1) \ne a\) or \(f(2) \ne b\) or both is
      \begin{equation*}
        5 \cdot 6^3 + 5 \cdot 6^3 - 5^2 \cdot 6^2 = 1260.
      \end{equation*}
      %
\par\smallskip
\noindent\textbf{9.}\quad{}
      \(5^{10} - \left[{5 \choose 1}4^{10} - {5 \choose 2}3^{10} + {5 \choose 3}2^{10} - {5 \choose 4}1^{10}\right]\)
      %
\par\smallskip
\noindent\textbf{10.}\quad{}
      \(5! - \left[{5 \choose 1}4! - {5 \choose 2}3! + {5 \choose 3}2! - {5 \choose 4}1! + {5 \choose 5}0!\right]\). This is a sneaky way to as for the number of derangements on 5 elements.
      %
\par\smallskip
\subsubsection*{2.7.1 Exercises}
\noindent\textbf{1.}\quad{}\leavevmode%
\begin{enumerate}[label=(\alph*)]
\item\hypertarget{li-666}{}\({8 \choose 3}\), after giving one present to each kid, you are left with 5 presents (stars) which need to be divide among the 4 kids (giving 3 bars).%
\item\hypertarget{li-667}{}\({12 \choose 3}\). You have 9 stars and 3 bars.%
\item\hypertarget{li-668}{}\(4^9\). You have 4 choices for whom to give each present. This is like making a function from the set of present to the set of kids.%
\item\hypertarget{li-669}{}\(4^9 - \left[{4 \choose 1}3^9 - {4\choose 2}2^9 + {4 \choose 3}1^9 \right]\). Now the function from the set of present to the set of kids must be surjective.%
\end{enumerate}
\par\smallskip
\noindent\textbf{2.}\quad{}\leavevmode%
\begin{enumerate}[label=(\alph*)]
\item\hypertarget{li-699}{}
                Neither. \({14 \choose 4}\).
              %
\item\hypertarget{li-700}{}\({10\choose 4}\).%
\item\hypertarget{li-701}{}\(P(10,4)\), since order is important.%
\item\hypertarget{li-702}{}
                Neither. Assuming you will wear each of the 4 ties on just 4 of the 7 days, without repeats: \({10\choose 4}P(7,4)\).
              %
\item\hypertarget{li-703}{}\(P(10,4)\).%
\item\hypertarget{li-704}{}\({10\choose 4}\).%
\item\hypertarget{li-705}{}
                Neither. Since you could repeat letters: \(10^4\). If no repeats are allowed, it would be \(P(10,4)\).
              %
\item\hypertarget{li-706}{}
                Neither. Actually, ``k'' is the 11th letter of the alphabet, so the answer is 0. If ``k'' was among the first 10 letters, there would only be 1 way - write it down.
              %
\item\hypertarget{li-707}{}
                Neither. Either \({9\choose 3}\) (if every kid gets an apple) or \({13 \choose 3}\) (if appleless kids are allowed).
              %
\item\hypertarget{li-708}{}
                Neither. Note that this could not be \({10 \choose 4}\) since the 10 things and 4 things are from different groups. \(4^{10}\).
              %
\item\hypertarget{li-709}{}\({10 \choose 4}\) - don't be fooled by the ``arrange'' in there - you are picking 4 out of 10 \emph{spots} to put the 1's.%
\item\hypertarget{li-710}{}\({10 \choose 4}\) (assuming order is irrelevant).%
\item\hypertarget{li-711}{}
                Neither. \(16^{10}\) (each kid chooses yes or no to 4 varieties).
              %
\item\hypertarget{li-712}{}
                Neither. 0.
              %
\item\hypertarget{li-713}{}
                Neither. \(4^{10} - [{4\choose 1}3^{10} - {4\choose 2}2^{10} + {4 \choose 3}1^{10}]\).
              %
\item\hypertarget{li-714}{}
                Neither. \(10\cdot 4\).
              %
\item\hypertarget{li-715}{}
                Neither. \(4^{10}\).
              %
\item\hypertarget{li-716}{}\({10 \choose 4}\) (which is the same as \({10 \choose 6}\)).%
\item\hypertarget{li-717}{}
                Neither. If all the kids were identical, and you wanted no empty teams, it would be \({10 \choose 4}\). Instead, this will be the same as the number of surjective functions from a set of size 11 to a set of size 5.
              %
\item\hypertarget{li-718}{}\({10 \choose 4}\).%
\item\hypertarget{li-719}{}\({10 \choose 4}\).%
\item\hypertarget{li-720}{}
                Neither. \(4!\).
              %
\item\hypertarget{li-721}{}
                Neither. It's \({10 \choose 4}\) if you won't repeat any choices. If repetition is allowed, then this becomes \(x_1 + x_2 + \cdots +x_{10} = 4\), which has \({13 \choose 9}\) solutions in non-negative integers.
              %
\item\hypertarget{li-722}{}
                Neither. Since repetition of cookie type is allowed, the answer is \(10^4\). Without repetition, you would have \(P(10,4)\).
              %
\item\hypertarget{li-723}{}\({10 \choose 4}\) since that is equal to \({9 \choose 4} + {9 \choose 3}\).%
\item\hypertarget{li-724}{}
                Neither. It will be a complicated (possibly PIE) counting problem.
              %
\end{enumerate}
\par\smallskip
\noindent\textbf{3.}\quad{}\leavevmode%
\begin{enumerate}[label=(\alph*)]
\item\hypertarget{li-729}{}\(2^8 = 256\). You have two choices for each tie: wear it or don't.%
\item\hypertarget{li-730}{}
                You have 7 choices for regular ties (the 8 choices less the ``no regular tie'' option) and 31 choices for bow ties (32 total minus the ``no bow tie'' option). Thus total you have \(7 \cdot 31 = 217\).
              %
\item\hypertarget{li-731}{}\({3\choose 2}{5\choose 3} = 30\).%
\item\hypertarget{li-732}{}
                Select one of the 3 bow ties to go on top. There are then 4 choices for the next tie, 3 for the tie after that, and so on. Thus \(3\cdot 4! = 72\).
              %
\end{enumerate}
\par\smallskip
\noindent\textbf{4.}\quad{}
            You own 8 purple bow ties\index{bow ties}, 3 red bow ties, 3 blue bow ties and 5 green bow ties. How many ways can you select one of each color bow tie to take with you on a trip? \(8 \cdot 3 \cdot 3 \cdot 5\). How many choices do you have for a single bow tie to wear tomorrow? \(8 + 3 + 3 + 5\).
          %
\par\smallskip
\noindent\textbf{5.}\quad{}\leavevmode%
\begin{enumerate}[label=(\alph*)]
\item\hypertarget{li-736}{}\(4^5\).%
\item\hypertarget{li-737}{}\(4^4\cdot 2\) (choose any digits for the first four digits - then pick either an even or an odd last digit to make the sum even).%
\item\hypertarget{li-738}{}
                We need 3 or more even digits. 3 even digits: \({5 \choose 3}2^3 2^2\). 4 even digits: \({5 \choose 4}2^4 2\). 5 even digits: \({5 \choose 5}2^5\). So all together: \({5 \choose 3}2^3 2^2 + {5 \choose 4}2^4 2 + {5 \choose 5}2^5\).
              %
\end{enumerate}
\par\smallskip
\noindent\textbf{6.}\quad{}
            \(|A \cup B|\) is the number of things that are in \(A\) or in \(B\) or in both. If you count up everything in each set independently, then anything which is in both sets (in \(A \cap B\)) is counted twice.
          %
\par\smallskip
\noindent\textbf{7.}\quad{}
            215. Use PIE: \(100 + 83 + 71 - 16 - 14 -11 + 2 = 215\) or a Venn diagram. To find out how many numbers are divisible by 6 and 7, for example, take \(500/42\) and round down.
          %
\par\smallskip
\noindent\textbf{8.}\quad{}
            51.
          %
\par\smallskip
\noindent\textbf{9.}\quad{}\leavevmode%
\begin{enumerate}[label=(\alph*)]
\item\hypertarget{li-743}{}\(2^8\).%
\item\hypertarget{li-744}{}\({8 \choose 5}\).%
\item\hypertarget{li-745}{}\({8 \choose 5}\).%
\item\hypertarget{li-746}{}
                There is a bijection between subsets and bit strings: a 1 means that element in is the subset, a 0 means that element is not in the subset. To get a subset of an 8 element set we have a 8-bit string. To make sure the subset contains exactly 5 elements, there must be 5 1's, so the weight must be 5.
              %
\end{enumerate}
\par\smallskip
\noindent\textbf{10.}\quad{}
            \({13 \choose 10} + {17 \choose 8}\).
          %
\par\smallskip
\noindent\textbf{11.}\quad{}
            With repeated letters allowed: \({8 \choose 5}5^5 21^3\). Without repeats: \({8 \choose 5}5! P(21, 3)\).
          %
\par\smallskip
\noindent\textbf{12.}\quad{}\leavevmode%
\begin{enumerate}[label=(\alph*)]
\item\hypertarget{li-750}{}\({5 \choose 2}{11 \choose 6}\).%
\item\hypertarget{li-751}{}\({16 \choose 8} - {12 \choose 7}{4 \choose 1}\).%
\item\hypertarget{li-752}{}\({5 \choose 2}{11 \choose 6} + {12 \choose 5}{4 \choose 3} - {5 \choose 2}{7 \choose 3}{4 \choose 3}\).%
\end{enumerate}
\par\smallskip
\noindent\textbf{13.}\quad{}
            \({18 \choose 8}\left({18 \choose 8} - 1\right)\).
          %
\par\smallskip
\noindent\textbf{14.}\quad{}
            A test had \(n\) questions on it, of which you must answer any \(k\) questions. How many choices do you have as to what order you answer the questions on the test? \(P(n,k)\). When grading the test, how many different combinations of question might the professor see? \({n \choose k}\).
          %
\par\smallskip
\noindent\textbf{15.}\quad{}
            \(2^7 + 2^7 - 2^4\).
          %
\par\smallskip
\noindent\textbf{16.}\quad{}
            \({7 \choose 3} + {7 \choose 4} - {4 \choose 1}\).
          %
\par\smallskip
\noindent\textbf{17.}\quad{}
            (a) \(6! - 4\cdot 3!\). (b) \(6! - {6 \choose 3}3!\).
          %
\par\smallskip
\noindent\textbf{18.}\quad{}
            \(2^n\) is the number of lattice paths which have length \(n\), since for each step you can go up or right. Such a path would end along the line \(x + y = n\). So you will end at \((0,n)\), or \((1,n-1)\) or \((2, n-2)\) or
            \dots{} or \((n,0)\). Counting the paths to each of these points separately, give \({n \choose 0}\), \({n \choose 1}\), \({n \choose 2}\),
            \dots{}, \({n \choose n}\) (each time choosing which of the \(n\) steps to be to the right).
          %
\par\smallskip
\noindent\textbf{19.}\quad{}
            Hint: give a combinatorial proof for the identity \(P(n,k) = {n \choose k} k!\).
          %
\par\smallskip
\noindent\textbf{20.}\quad{}
            Of your \(n\) bow ties\index{bow ties}, you decide to give \(k\) away to charity. How many ways can you do this? On one hand, you can choose \(k\) of the \(n\) bow ties to give away in \({n \choose k}\) ways. Alternatively, you can choose which bow ties to keep. You must keep \(n -k\) of them if you will give \(k\) away, so you can choose the bow ties to keep in \({n \choose n-k}\) ways. This gives a combinatorial proof for the identity.
          %
\par\smallskip
\noindent\textbf{21.}\quad{}
            Hint: stars and bars
          %
\leavevmode%
\begin{enumerate}[label=(\alph*)]
\item\hypertarget{li-756}{}\({19 \choose 4}\).%
\item\hypertarget{li-757}{}\({24 \choose 4}\).%
\item\hypertarget{li-758}{}\({19 \choose 4} - \left[{5 \choose 1}{12 \choose 4} - {5 \choose 2}{5 \choose 4}  \right]\).%
\end{enumerate}
\par\smallskip
\noindent\textbf{22.}\quad{}\leavevmode%
\begin{enumerate}[label=(\alph*)]
\item\hypertarget{li-763}{}\(5^4 + 5^4 - 5^3\).%
\item\hypertarget{li-764}{}\(4\cdot 5^4 + 5 \cdot 4 \cdot 5^3 - 4 \cdot 4 \cdot 5^3\).%
\item\hypertarget{li-765}{}\(5! - \left[ 4! + 4! - 3! \right]\).%
\item\hypertarget{li-766}{}\(5! - \left[{5 \choose 1}4! - {5 \choose 2}3! + {5 \choose 3}2! - {5 \choose 4}1! + {5 \choose 5} 0!\right]\).%
\end{enumerate}
\par\smallskip
\noindent\textbf{23.}\quad{}
            \({5 \choose 1}\left( 4! - \left[{4 \choose 1}3! - {4 \choose 2}2! + {4 \choose 3} 1! - {4 \choose 4} 0!\right] \right)\).
          %
\par\smallskip
\noindent\textbf{24.}\quad{}
            \(4^6 - \left[{4 \choose 1}3^6 - {4 \choose 2}2^6 + {4 \choose 3} 1^6 \right]\).
          %
\par\smallskip
\noindent\textbf{25.}\quad{}\leavevmode%
\begin{enumerate}[label=(\alph*)]
\item\hypertarget{li-772}{}\({10 \choose 4}\). You need to choose 4 of the 10 cookie types. Order doesn't matter.%
\item\hypertarget{li-773}{}\(P(10, 4) = 10 \cdot 9 \cdot 8 \cdot 7\). You are choosing and arranging 4 out of 10 cookies. Order matters now.%
\item\hypertarget{li-774}{}\({21 \choose 9}\). You must switch between cookie type 9 times as you make your 12 cookies. The cookies are the stars, the switches between cookie types are the bars.%
\item\hypertarget{li-775}{}\(10^{12}\). You have 10 choices for the ``1'' cookie, 10 choices for the ``2'' cookie, and so on.%
\item\hypertarget{li-776}{}\(10^{12} - \left[{10 \choose 1}9^{12} - {10 \choose 2}8^{12} + \cdots - {10 \choose 10}0^{12}   \right]\). We must use PIE to remove all the ways in which one or more cookie type is not selected.%
\end{enumerate}
\par\smallskip
\noindent\textbf{26.}\quad{}\leavevmode%
\begin{enumerate}[label=(\alph*)]
\item\hypertarget{li-777}{}
                You are giving your professor 4 types of cookies coming from 10 different types of cookies. This does not lend itself well to a function interpretation. We \emph{could} say that the domain contains the 4 types you will give your professor and the codomain contains the 10 you can choose from, but then counting injections would be too much (it doesn't matter if you pick type 3 first and type 2 second, or the other way around, just that you pick those two types).
              %
\item\hypertarget{li-778}{}
                We want to consider injective functions from the set \(\{\)most, second most, second least, least\(\}\) to the set of 10 cookie types. We want injections because we cannot pick the same type of cookie to give most and least of (for example).
              %
\item\hypertarget{li-779}{}
                This is not a good problem to interpret as a function. The problem is that the domain would have to be the 12 cookies you bake, but these elements are indistinguishable (there is not a first cookie, second cookie, etc.).
              %
\item\hypertarget{li-780}{}
                The domain should be the 12 shapes, the codomain the 10 types of cookies. Since we can use the same type for different shapes, we are interested in counting all functions here.
              %
\item\hypertarget{li-781}{}
                Here we insist that each type of cookie be given at least once, so now we are asking for the number of surjections of those functions counted in the previous part.
              %
\end{enumerate}
\par\smallskip
\subsubsection*{2.7.1.1 Exercises}
\noindent\textbf{2.}\quad{}
            There are multiple ways to do this.
          %
\leavevmode%
\begin{enumerate}[label=(\alph*)]
\item\hypertarget{li-786}{}
                An even sum can occur in 4 ways: EEE, EOO, OEO, and OOE. There are \(4 \cdot 5 \cdot 5\) ways to build numbers of the first two types (there are only 4 choices for a starting even number - it cannot be 0) and \(5 \cdot 5 \cdot 5\) ways to build the second two types. This gives a total of 450 numbers.
              %
\item\hypertarget{li-787}{}
                To build a 3 digit number with an even sum, you can choose any of 9 digits for the first digit, any of 10 digits for the second digit. Then the last digit must either be even (if the sum of the first two digits are even) or odd (if the sum of the first two digits are odd). Luckily there are the same number of even last digits and odd last digits - 5. So there are a total of \(9 \cdot 10 \cdot 5 = 450\) numbers with an even sum of digits.
              %
\item\hypertarget{li-788}{}
                Start finding sums of digits from 3-digit numbers: \(100 \to\) odd, \(101 \to\) even, \(102 \to\)odd, \(103 \to\) even, and so on. So the numbers appear to alternate between even and odd sums. However, notice that 109 has an even sum while 110 does as well. But then 111 is odd, 112 is even, and so on. So we can conclude that half of the numbers 100 to 109 have even sum, half of the number 110 to 119 have even sum, half from 120 to 129, and so on. This means that overall half of the numbers will have even sum, so half of the 900 3-digit numbers will have even sum, namely 450 of them.
              %
\end{enumerate}
\par\smallskip
\noindent\textbf{3.}\quad{}
            Using the principle of inclusion/exclusion, the number of students who like their potatoes in at least one of the ways described is
            \begin{equation*}
              15 + 20 + 9 - 12 - 5 - 6 + 3 = 24.
            \end{equation*}
          %
\par

            Therefore there are \(30-24 = 6\) students who do not like potatoes. You can also do this problem with a Venn diagram.
          %
\par\smallskip

            If you consider the factorization of any divisor of 735000 it must have at most three 2s, at most one 3, at most four 5s and at most two 7s, with no other prime factors. Thus to select a divisor, we just need to pick how many of these prime factors are present. There are 4 choices for how many 2s to include (between zero and four), 2 choices for how many 3s, 5 choices for how many 5s and 3 choices for how many 7s. Thus the number of divisors is:
            \begin{equation*}
              4\cdot 2 \cdot 5 \cdot 3 = 120
            \end{equation*}
          %
\par\smallskip
\noindent\textbf{6.}\quad{}
            There are 120 triangles. Here are two ways (there are others as well) to get this:
          %
\leavevmode%
\begin{enumerate}[label=(\alph*)]
\item\hypertarget{li-797}{}
                First count the triangles with the base on the \(x\)-axis. There are \({7 \choose 2}\) ways to pick the base. The third vertex of the triangle must be one of the 4 dots on the \(y\)-axis (not the origin) so there are a total of \({7 \choose 2}4\) of these triangles. The triangles with base on the \(y\) axis can be counted similarly: \({5 \choose 2}6\). However, we have counted all the right triangles twice - they have a base on the \(x\)-axis and also on the \(y\)-axis. There are \(4 \cdot 6\) right triangles. Thus the total number of triangles is:
                \begin{equation*}
                  {7 \choose 2}4 + {5 \choose 2}6 - 6\cdot 4 = 120
                \end{equation*}
              %
\item\hypertarget{li-798}{}
                We must select 3 of the 11 dots. This can be done in \({11 \choose 3}\) ways. However, this will also give us degenerate triangles when all three vertices are on the \(x\)-axis or on the \(y\)-axis. There are \({7 \choose 3}\) ways we could have picked all three vertices on the \(x\)-axis. There are \({5 \choose 3}\) ways we could have picked all three vertices on the \(y\)-axis. Therefore the total number of triangles is
                \begin{equation*}
                  {11 \choose 3} - {7 \choose 3} - {5 \choose 3} = 120
                \end{equation*}
              %
\end{enumerate}
\par\smallskip
\typeout{************************************************}
\typeout{Appendix B Notation}
\typeout{************************************************}
\chapter[Notation]{Notation}\label{appendix-2}
\begin{longtable}[l]{llr}
\textbf{Symbol}&\textbf{Description}&\textbf{Page}\\[1em]
\endfirsthead
\textbf{Symbol}&\textbf{Description}&\textbf{Page}\\[1em]
\endhead
\multicolumn{3}{r}{(Continued on next page)}\\
\endfoot
\endlastfoot
$
                    P, Q, R, S, \ldots
                $&Propositional (sentential) variables&\pageref{notation-1}\\
$\N$&The set of natural numbers&\pageref{notation-2}\\
$\st$&``such that''&\pageref{notation-3}\\
$f\inv(y)$&The complete inverse image of \(y\) under \(f\).&\pageref{notation-4}\\
$\B^n$&The set of length \(n\) bit strings&\pageref{notation-5}\\
$\B^n_k$&The set of legth \(n\) bit strings with weight \(k\).&\pageref{notation-6}\\
\end{longtable}
%
\backmatter
%
%
%% The index is here, setup is all in preamble
\printindex
%
\cleardoublepage
\pagestyle{empty}
\vspace*{\stretch{1}}
\centerline{
    This book was authored in MathBook XML.
  %
}
\vspace*{\stretch{2}}
\end{document}