%**************************************%
%* Generated from MathBook XML source *%
%*    on 2016-04-26T09:15:17-07:00    *%
%*                                    *%
%*   http://mathbook.pugetsound.edu   *%
%*                                    *%
%**************************************%
\documentclass[10pt,]{article}
%% Load geometry package to allow page margin adjustments
\usepackage{geometry}
\geometry{letterpaper,total={5.0in,9.0in}}
%% Custom Preamble Entries, early (use latex.preamble.early)
%% Inline math delimiters, \(, \), need to be robust
%% 2016-01-31:  latexrelease.sty  supersedes  fixltx2e.sty
%% If  latexrelease.sty  exists, bugfix is in kernel
%% If not, bugfix is in  fixltx2e.sty
%% See:  https://tug.org/TUGboat/tb36-3/tb114ltnews22.pdf
%% and read "Fewer fragile commands" in distribution's  latexchanges.pdf
\IfFileExists{latexrelease.sty}{}{\usepackage{fixltx2e}}
%% Page Layout Adjustments (latex.geometry)
%% For unicode character support, use the "xelatex" executable
%% If never using xelatex, the next three lines can be removed
\usepackage{ifxetex}
\ifxetex\usepackage{xltxtra}\fi
%% Symbols, align environment, bracket-matrix
\usepackage{amsmath}
\usepackage{amssymb}
%% allow more columns to a matrix
%% can make this even bigger by overiding with  latex.preamble.late  processing option
\setcounter{MaxMatrixCols}{30}
%% Semantic Macros
%% To preserve meaning in a LaTeX file
%% Only defined here if required in this document
%% Subdivision Numbering, Chapters, Sections, Subsections, etc
%% Subdivision numbers may be turned off at some level ("depth")
%% A section *always* has depth 1, contrary to us counting from the document root
%% The latex default is 3.  If a larger number is present here, then
%% removing this command may make some cross-references ambiguous
%% The precursor variable $numbering-maxlevel is checked for consistency in the common XSL file
\setcounter{secnumdepth}{3}
%% Environments with amsthm package
%% Theorem-like enviroments in "plain" style, with or without proof
\usepackage{amsthm}
\theoremstyle{plain}
%% Numbering for Theorems, Conjectures, Examples, Figures, etc
%% Controlled by  numbering.theorems.level  processing parameter
%% Always need a theorem environment to set base numbering scheme
%% even if document has no theorems (but has other environments)
\newtheorem{theorem}{Theorem}[section]
%% Only variants actually used in document appear here
%% Numbering: all theorem-like numbered consecutively
%% i.e. Corollary 4.3 follows Theorem 4.2
%% Definition-like environments, normal text
%% Numbering for definition, examples is in sync with theorems, etc
%% also for free-form exercises, not in exercise sections
\theoremstyle{definition}
\newtheorem{example}[theorem]{Example}
\newtheorem{exercise}[theorem]{Exercise}
%% Localize LaTeX supplied names (possibly none)
%% Raster graphics inclusion, wrapped figures in paragraphs
\usepackage{graphicx}
%% Colors for Sage boxes and author tools (red hilites)
\usepackage[usenames,dvipsnames,svgnames,table]{xcolor}
%% hyperref driver does not need to be specified
\usepackage{hyperref}
%% Hyperlinking active in PDFs, all links solid and blue
\hypersetup{colorlinks=true,linkcolor=blue,citecolor=blue,filecolor=blue,urlcolor=blue}
\hypersetup{pdftitle={On and regarding an example MBX article}}
%% If you manually remove hyperref, leave in this next command
\providecommand\phantomsection{}
%% Graphics Preamble Entries
%% extpfeil package for certain extensible arrows,
%% as also provided by MathJax extension of the same name
%% NB: this package loads mtools, which loads calc, which redefines
%%     \setlength, so it can be removed if it seems to be in the 
%%     way and your math does not use:
%%     
%%     \xtwoheadrightarrow, \xtwoheadleftarrow, \xmapsto, \xlongequal, \xtofrom
%%     
%%     we have had to be extra careful with variable thickness
%%     lines in tables, and so also load this package late
\usepackage{extpfeil}
%% Custom Preamble Entries, late (use latex.preamble.late)
%% Begin: Author-provided macros
%% (From  docinfo/macros  element)
%% Plus three from MBX for XML characters

\newcommand{\lt}{ < }
\newcommand{\gt}{ > }
\newcommand{\amp}{ & }
%% End: Author-provided macros
%% Title page information for article
\title{On and regarding an example MBX article}
\date{}
\begin{document}
\thispagestyle{empty}
\maketitle
\typeout{************************************************}
\typeout{Introduction  Introduction}
\typeout{************************************************}
\section*{Introduction}
This is the introduction.  It comes before the other sections.  For example it comes before \hyperref[sec-background]{Section~\ref{sec-background}}.%
\typeout{************************************************}
\typeout{Section 1 Background}
\typeout{************************************************}
\section[Background]{Background}\label{sec-background}
The first thing you should know about this sentence is that it is false.  Which is a strange thing for a sentence to claim about itself, but then of course it is self referenctial, so what do you expect.%
\par
A second paragraph is also possible, but would not change the fact that this whole seciton is rather rediculous.%
\typeout{************************************************}
\typeout{Section 2 A few examples.}
\typeout{************************************************}
\section[A few examples.]{A few examples.}\label{sec-examples}
\begin{example}\label{example-1}
This is the first example.  It doesn't have a title.%
\end{example}
\begin{example}[Another example]\label{ex-2}
Here is another example.  This one does have a title.%
This is the solution to the second exercise.\end{example}
\begin{exercise}\label{exercise-1}
This is an exercise which asks you to compute something.  It is really just an excuse to play with math mode.%
\par
For example, we could ask, what is the quadratic equation?%
\par\smallskip
\noindent\textbf{Solution.}\hypertarget{solution-2}{}\quad
\(x = \frac{-b \pm \sqrt{b^2 - 4ac}}{2a}\)%
\end{exercise}
\begin{example}\label{ex-3}
What is the number of this example?%
It's not the first or the second, and the fourth is right out of the question.3If you count from the top, you will realize this is the third example.  Don't confuse this with \hyperref[ex-2]{\ref{ex-2}}.%
\end{example}
\typeout{************************************************}
\typeout{Introduction  Another introduction}
\typeout{************************************************}
\section*{Another introduction}
Can we really have two introductions?%
This is a paragraph without a section.  Where will it show up?%
\end{document}