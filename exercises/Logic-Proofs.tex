\begin{questions}
 \question Consider the statement ``for all integers $a$ and $b$, if $a + b$ is even, then $a$ and $b$ are even''
 \begin{parts}
  \part Write the contrapositive of the statement
  \part Write the converse of the statement
  \part Write the negation of the statement.
  \part Is the original statement true or false?  Prove your answer.
  \part Is the contrapositive of the original statement true or false?  Prove your answer.
  \part Is the converse of the original statement true or false?  Prove your answer.
  \part Is the negation of the original statement true or false?  Prove your answer.
 \end{parts}
 
   \begin{answer}
     \begin{parts}
 	\part For all integers $a$ and $b$, if $a$ or $b$ are not even, then $a+b$ is not even.
 	\part For all integers $a$ and $b$, if $a$ and $b$ are even, then $a+b$ is even.
 	\part There are numbers $a$ and $b$ such that $a+b$ is even but $a$ and $b$ are not both even.
 	\part False.  For example, $a = 3$ and $b = 5$.  $a+b = 8$, but neither $a$ nor $b$ are even.
 	\part False, since it is equivalent to the original statement.
 	\part True.  Let $a$ and $b$ be integers.  Assume both are even.  Then $a = 2k$ and $b = 2j$ for some integers $k$ and $j$.  But then $a+b = 2k + 2j = 2(k+j)$ which is even.
 	\part True, since the statement is false.
       \end{parts}
   \end{answer}
   



\question Consider the statement: for all integers $n$, if $n$ is even then $8n$ is even.
\begin{parts}
  \part Prove the statement.  What sort of proof are you using?
  \part Is the converse true?  Prove or disprove.
\end{parts}

	\begin{answer}
	\begin{parts}
	  \part Direct proof.  
	  \begin{proof}
	    Let $n$ be an integer.  Assume $n$ is even.  Then $n = 2k$ for some integer $k$.  Thus $8n = 16k = 2(8k)$.  Therefore $8n$ is even.
	  \end{proof}
	
	  \part The converse is false.  That is, there is an integer $n$ such that $8n$ is even but $n$ is odd.  For example, consider $n = 3$.  Then $8n = 24$ which is even but $n = 3$ is odd.
	\end{parts}
	\end{answer}











 \question Prove that $\sqrt 3$ is irrational.
 
   \begin{answer}
     \begin{proof}
      Suppose $\sqrt{3}$ were rational.  Then $\sqrt{3} = \frac{a}{b}$ for some integers $a$ and $b \ne 0$.  Without loss of generality, assume $\frac{a}{b}$ is reduced.  Now
 \[3 = \frac{a^2}{b^2}\]
 \[b^2 3 = a^2\]
 So $a^2$ is a multiple of 3.  This can only happen if $a$ is a multiple of 3, so $a = 3k$ for some integer $k$.  Then we have
 \[b^2 3 = 9k^2\]
 \[b^2 = 3k^2\]
 So $b^2$ is a multiple of 3, making $b$ a multiple of 3 as well.  But this contradicts our assumption that $\frac{a}{b}$ is in lowest terms.
     \end{proof}
   \end{answer}


   
   
   
 \question Consider the statement: for all integers $a$ and $b$, if $a$ is even and $b$ is a multiple of 3, then $ab$ is a multiple of 6.
	\begin{parts}
	  \part Prove the statement.  What sort of proof are you using?
	  \part State the converse.  Is it true?  Prove or disprove.
	\end{parts}
	
	\begin{answer}
	\begin{parts}
	  \part Direct proof.
	  \begin{proof}
	    Let $a$ and $b$ be integers.  Assume $a$ is even and $b$ is a multiple of 3.  Then $a = 2k$ and $b = 3j$ for some integers $k$ and $j$.  Now
	    \[ab = (2k)(3j) = 6(kj)\]
	    Since $kj$ is an integer, we have that $ab$ is a multiple of 6.
	  \end{proof}
	
	  \part The converse is: for all integers $a$ and $b$, if $ab$ is a multiple of 6, then $a$ is even and $b$ is a multiple of 3.  This is false.  Consider $a = 3$ and $b = 10$.  Then $ab = 30$ which is a multiple of 6, but $a$ is not even and $b$ is not divisible by 3.
	\end{parts}
	\end{answer}
 
 
   
   
\question Prove the statement: For all integers $n$, if $5n$ is odd, then $n$ is odd.  Clearly state the style of proof you are using.
	\begin{answer}
	We will prove the contrapositive: if $n$ is even, then $5n$ is even.
	  \begin{proof}
	    Let $n$ be an arbitrary integer, and suppose $n$ is even.  Then $n = 2k$ for some integer $k$.  Thus $5n = 5\cdot 2k = 10k = 2(5k)$.  Since $5k$ is an integer, we see that $5n$ must be even.  This completes the proof.
	  \end{proof}
	
	\end{answer}




% This could also be a homework problem.
\question Prove the statement: For all integers $a$, $b$, and $c$, if $a^2 + b^2 = c^2$, then $a$ or $b$ is even. 
	\begin{answer}
	  \begin{proof}
	    Suppose, contrary to stipulation, that there are integers $a$, $b$ and $c$ such that $a^2 + b^2 = c^2$ but $a$ and $b$ are both odd.  Then $a = 2k+1$ and $b = 2j + 1$ for some integers $k$ and $j$.  We then have
	    \[a^2 + b^2 = (2k+1)^2 + (2j+1)^2 = 4k^2 + 4k + 1 + 4j^2 + 4j + 1 = 4(k^2 + j^2 + k + j) + 2\]
	    So $c^2 = 4(k^2 + j^2 + k + j) + 2$.  This means that $c^2$ is even, which can only happen if $c$ is even.  But then $c^2$ must be a multiple of 4.  However, this is a contradiction because $4(k^2 + j^2 + k + j) + 2$ is not a multiple of 4.
	  \end{proof}
	
	\end{answer}
 



\question The game TENZI comes with 40 six-sided dice (each numbered 1 to 6).  Suppose you roll all 40 dice.  Prove that there will be at least seven dice that land on the same number.

	\begin{answer}
	This is an example of the pigeonhole principle.  We can prove it by contrapositive.
	
	\begin{proof}
	Suppose that each number only came up 6 or fewer times.  So there are at most six 1's, six 2's, and so on.  That's a total of 36 dice, so you must not have rolled all 40 dice.
	\end{proof}
	\end{answer}





\question How many dice would you have to roll before you were guaranteed that some four of them would all match or all be different?  Prove your answer.

	\begin{answer}
		We can have 9 dice without any four matching or all being different: three 1's, three 2's, three 3's.  We will prove that whenever you roll 10 dice, you will always get four matching or all being different.
		\begin{proof}
			Suppose you roll 10 dice, but that there are NOT four matching rolls.  This means at most, there are three of any given value.  If we only had three different values, that would be only 9 dice, so there must be 4 different values, giving 4 dice that are all different.
		\end{proof} 
	\end{answer}
	




\question Prove that $\log(7)$ is irrational.

	\begin{answer}
	 We give a proof by contradiction.
	\begin{proof}
	  Suppose, contrary to stipulation that $\log(7)$ is rational.  Then $\log(7) = \frac{a}{b}$ with $a$ and $b \ne 0$ integers.  By properties of logarithms, this implies
	  \[7 = 10^{\frac{a}{b}}\]
	  Equivalently,
	  \[7^b = 10^a\]
	  But this is impossible as any power of 7 will be odd while any power of 10 will be even.
	\end{proof}
	\end{answer}


\question Prove that there are no integer solutions to the equation $x^2 = 4y + 3$.

	\begin{answer}
		\begin{proof}
		  Suppose there were integers $x$ and $y$ such that $x^2 = 4y + 3$.  Now $x^2$ must be odd, since $4y + 3$ is odd.  Since $x^2$ is odd, we know $x$ must be odd as well.  So $x = 2k + 1$ for some integer $k$.  Then $x^2 = 4k^2 + 4k + 1 = 4(k^2 + k) + 1$.  Therefore we have,
		  \[4(k^2 + k) + 1 = 4y + 3\]
		  which implies
		  \[4(k^2 + k) = 4y + 2\]
		  and therefore
		  \[2(k^2 + k) = 2y + 1.\]
		  But this is a contradiction -- the left hand side is even while the right hand side is odd. 
		\end{proof}
		
	\end{answer}
 


\question For each of the statements below, say what method of proof you should use to prove them.  Then say how the proof starts and how it ends.  Bonus points for filling in the middle.
\begin{parts}
 \part There are no integers $x$ and $y$ such that $x$ is a prime greater than 5 and $x = 6y + 3$.
 \part For all integers $n$, if $n$ is a multiple of 3, then $n$ can be written as the sum of consecutive integers.
 \part For all integers $a$ and $b$, if $a^2 + b^2$ is odd, then $a$ or $b$ is odd.
\end{parts}

	\begin{answer}
		\begin{parts}
		% There are no integers $x$ and $y$ such that $x$ is a prime greater than 5 and $x = 6y + 3$.
		 \part Proof by contradiction.  Start of proof: Assume, for the sake of contradiction, that there are integers $x$ and $y$ such that $x$ is a prime greater than 5 and $x = 6y + 3$.  End of proof: \ldots this is a contradiction, so there are no such integers.
		%  For all integers $n$, if $n$ is a multiple of 3, then $n$ can be written as the sum of consecutive integers.
		 \part Direct proof.  Start of proof: Let $n$ be an integers.  Assume $n$ is a multiple of 3.  End of proof: Therefore $n$ can be written as the sum of consecutive integers.
		%  For all integers $a$ and $b$, if $a^2 + b^2$ is odd, then $a$ or $b$ is odd.
		 \part Proof by contrapositive.  Start of proof: Let $a$ and $b$ be integers.  Assume that $a$ and $b$ are even.  End of proof: Therefore $a^2 + b^2$ is even.
		\end{parts}
	\end{answer}

\end{questions}