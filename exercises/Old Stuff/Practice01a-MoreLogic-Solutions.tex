\documentclass[11pt]{exam}

\usepackage{amsmath, amssymb, amsthm, multicol}
\usepackage{graphicx}
\usepackage{textcomp}
\usepackage{chessboard}

\def\d{\displaystyle}
\def\b{\mathbf}
\def\R{\mathbf{R}}
\def\Z{\mathbf{Z}}
\def\st{~:~}
\def\bar{\overline}
\def\inv{^{-1}}
\def\imp{\rightarrow}
\def\and{\wedge}


%\pointname{pts}
\pointsinmargin
\marginpointname{pts}
\addpoints
\pagestyle{head}
%\printanswers

\firstpageheader{Math 228}{\bf Practice Problems 1a: More Logic \\ Solutions}{Spring 2012}


\begin{document}

\begin{questions}
\question 
\begin{parts}
  \part  Negation: The power goes off and the food does not spoil.\\
  Converse: If the food spoils, then the power went off.\\
  Contrapositive: If the food does not spoil, then the power did not go off.
  
  \part   Negation: The door is closed and the light is on.\\
  Converse: If the light is off then the door is closed.\\
  Contrapositive: If the light is on then the door is open.
  \part 
    Negation: $\exists x (x < 1 \and x^2 \ge 1)$\\
  Converse: $\forall x( x^2 < 1 \imp x < 1)$\\
  Contrapositive: $\forall x (x^2 \ge 1 \imp x \ge 1)$.
  \part Negation: There is a natural number $n$ which is prime but not solitary.\\
  Converse: For all natural numbers $n$, if $n$ is solitary, then $n$ is prime.\\
  Contrapositive: For all natural numbers $n$, if $n$ is not solitary then $n$ is not prime.
  
  \part Negation: There is a function which is differentiable and not continuous.\\
  Converse: For all functions $f$, if $f$ is continuous then $f$ is differentiable. \\
  Contrapositive: For all functions $f$, if $f$ is not continuous then $f$ is not differentiable.
  
  \part Negation: There are integers $a$ and $b$ for which $a\cdot b$ is even but $a$ or $b$ is odd.\\
  Converse: For all integers $a$ and $b$, if $a$ and $b$ are even then $ab$ is even.\\
  Contrapositive: For all integers $a$ and $b$, if $a$ or $b$ is odd, then $ab$ is odd.
  
  \part Negation: There are integers $x$ and $y$ such that for every integer $n$, $x \le 0$ and $nx \le y$. \\
  Converse: For every integer $x$ and every integer $y$ there is an integer $n$ such that if $nx > y$ then $x > 0$.\\
  Contrapositive: For every integer $x$ and every integer $y$ there is an integer $n$ such that if $nx \le y$ then $x \le 0$.
  
  \part  Negation: There are real numbers $x$ and $y$ such that $xy = 0$ but $x \ne 0$ and $y \ne 0$.\\
  Converse: For all real numbers $x$ and $y$, if $x = 0$ or $y = 0$ then $xy = 0$\\
  Contrapositive: For all real numbers $x$ and $y$, if $x \ne 0$ and $y \ne 0$ then $xy \ne 0$.
  
  \part Negation: There is at least one student in Math 228 who does not understand implications but will still pass the exam.\\
  Converse: For every student in Math 228, if they fail the exam, then they did not understand implications.\\
  Contrapositive: For every student in Math 228, if they pass the exam, then they understood implications. 
  
\end{parts}

\question 
\begin{parts}
  \part Direct proof.  
  \begin{proof}
    Let $n$ be an integer.  Assume $n$ is even.  Then $n = 2k$ for some integer $k$.  Thus $8n = 16k = 2(8k)$.  Therefore $8n$ is even.
  \end{proof}

  \part The converse is false.  That is, there is an integer $n$ such that $8n$ is even but $n$ is odd.  For example, consider $n = 3$.  Then $8n = 24$ which is even but $n = 3$ is odd.
\end{parts}

\question 
\begin{parts}
  \part Direct proof.
  \begin{proof}
    Let $n$ be an integer.  Assume $n$ is odd.  So $n = 2k+1$ for some integer $k$.  Then 
    \[7n = 7(2k+1) = 14k + 7 = 2(7k +3) + 1\]
    Since $7k + 3$ is an integer, we see that $7n$ is odd.  
  \end{proof}

  \part The converse is: for all integers $n$ if $7n$ is odd, then $n$ is odd.  We will prove this by contrapositive.
  \begin{proof}
    Let $n$ be an integer.  Assume $n$ is not odd.  Then $n = 2k$ for some integer $k$.  So $7n = 14k = 2(7k)$ which is to say $7n$ is even.  Therefore $7n$ is not odd.
  \end{proof}

\end{parts}

\question 
\begin{parts}
  \part Direct proof.
  \begin{proof}
    Let $a$ and $b$ be integers.  Assume $a$ is even and $b$ is a multiple of 3.  Then $a = 2k$ and $b = 3j$ for some integers $k$ and $j$.  Now
    \[ab = (2k)(3j) = 6(kj)\]
    Since $kj$ is an integer, we have that $ab$ is a multiple of 6.
  \end{proof}

  \part The converse is: for all integers $a$ and $b$, if $ab$ is a multiple of 6, then $a$ is even and $b$ is a multiple of 3.  This is false.  Consider $a = 3$ and $b = 10$.  Then $ab = 30$ which is a multiple of 6, but $a$ is not even and $b$ is not divisible by 3.
\end{parts}


\question We give a proof by contradiction.
\begin{proof}
  Suppose, contrary to stipulation that $\log(7)$ is rational.  Then $\log(7) = \frac{a}{b}$ with $a$ and $b \ne 0$ integers.  By properties of logarithms, this implies
  \[7 = 10^{\frac{a}{b}}\]
  Equivalently,
  \[7^b = 10^a\]
  But this is impossible as any power of 7 will be odd while any power of 10 will be even.
\end{proof}

\question Again, by contradiction.
\begin{proof}
  Suppose there were integers $x$ and $y$ such that $x^2 = 4y + 3$.  Now $x^2$ must be odd, since $4y + 3$ is odd.  Since $x^2$ is odd, we know $x$ must be odd as well.  So $x = 2k + 1$ for some integer $k$.  Then $x^2 = 4k^2 + 4k + 1 = 4(k^2 + k) + 1$.  Therefore we have,
  \[4(k^2 + k) + 1 = 4y + 3\]
  which implies
  \[4(k^2 + k) = 4y + 2\]
  and therefore
  \[2(k^2 + k) = 2y + 1.\]
  But this is a contradiction - the left hand side is even while the right hand side is odd. 
\end{proof}



\end{questions}




\end{document}


