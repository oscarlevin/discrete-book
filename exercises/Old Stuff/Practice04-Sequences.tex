\documentclass[11pt]{exam}

\usepackage{amsmath, amssymb, multicol}
\usepackage{graphicx}
\usepackage{textcomp}
\usepackage{tikz}

\def\d{\displaystyle}
\def\b{\mathbf}
\def\R{\mathbb{R}}
\def\Z{\mathbb{Z}}
\def\N{\mathbb{N}}
\def\pow{\mathcal{P}}
\def\st{~:~}
\def\bar{\overline}
\def\inv{^{-1}}
\def\imp{\rightarrow}
\def\and{\wedge}


%\pointname{pts}
\pointsinmargin
\marginpointname{pts}
\addpoints
\pagestyle{head}
%\printanswers

\firstpageheader{Math 228}{\bf Practice Problems 4: Sequences}{Spring 2012}


\begin{document}
\noindent \textbf{Instructions}: The problems below are purely for you to practice.  I will not collect these, but it is still a good idea to write out your solutions in full.  Any of these problems or problems similar are fair game for quizzes and exams.  

\begin{questions}
\question Find the closed formula for each of the following sequences by relating them to a well know sequence.  Assume the first term given is $a_1$.
\begin{parts}
  \part $2, 5, 10, 17, 26, \ldots$
  \part $0, 2, 5, 9, 14, 20, \ldots$
  \part $8, 12, 17, 23, 30,\ldots$
  \part $1, 5, 23, 119, 719,\ldots$
\end{parts}

\question The Fibonacci sequence is $0, 1, 1, 2, 3, 5, 8, 13, \ldots$ (where $F_0 = 0$).
\begin{parts}
  \part Give the recursive definition for the sequence.
  \part Write out the first few terms of the sequence of partial sums.  
  \part Give a closed formula for the sequence of partial sums in terms of $F_n$  (for example, you might say $F_0 + F_1 + \cdots + F_n = 3F_{n-1}^2 + n$, although that is definitely not correct).
\end{parts}

\question Write out the first few terms of the sequence given by $a_1 = 3$; $a_n = 2a_{n-1} + 4$.  Then find a recursive definition for the sequence $10, 24, 52, 108, \ldots$.

\question Write out the first few terms of the sequence given by $a_n = n^2 - 3n + 1$.  Then find a closed formula for the sequence (starting with $a_1$) $0, 2, 6, 12, 20, \ldots$.

\question Consider the sequence $8, 14, 20, 26, \ldots, $. 
\begin{parts}
\part What is the next term in the sequence?  
\part Find a formula for the $n$th term of this sequence, assuming $a_1 = 8$.
\part Find the sum of the first 100 terms of the sequence: $\sum_{k=1}^{100}a_k$.
\end{parts}

\question Consider the sequence $1, 7, 13, 19, \ldots, 6n + 7$.  
\begin{parts}
\part How many terms are there in the sequence?
\part What is the second-to-last term?
\part Find the sum of all the terms in the sequence.
\end{parts}


\question Find $5 + 7 + 9 + 11+ \cdots + 521$.

\question Find $5 + 15 + 45 + \cdots + 5\cdot 3^{20}$

\question Find $1 - \frac{2}{3} + \frac{4}{9} - \cdots + \frac{2^{30}}{3^{30}}$

\question Find $x$ and $y$ such that $27, x, y, 1$ is part of an arithmetic sequence.  Then find $x$ and $y$ so that the sequence is part of a geometric sequence.  ($x$ and $y$ might not be integers.) 

\question Use summation ($\sum$) or product ($\prod$) notation to rewrite the following.
\begin{parts}
  \part $2 + 4 + 6 + 8 + \cdots + 2n$
  \part $1 + 5 + 9 + 13 + \cdots + 425$
  \part $1 + \frac{1}{2} + \frac{1}{3} + \frac{1}{4} + \cdots + \frac{1}{50}$
  \part $2 \cdot 4 \cdot 6 \cdot \cdots \cdot 2n$
  \part $(\frac{1}{2})(\frac{2}{3})(\frac{3}{4})\cdots(\frac{100}{101})$
\end{parts}

\question Expand the following sums and products.  That is, write them out the long way.
\begin{parts}
  \part $\d\sum_{k=1}^{100} (3+4k)$
  \part $\d\sum_{k=0}^n 2^k$
  \part $\d\sum_{k=2}^{50}\frac{1}{(k^2 - 1)}$
  \part $\d\prod_{k=2}^{100}\frac{k^2}{(k^2-1}$
  \part $\d\prod_{k=0}^n (2+3k)$
\end{parts}


\question Use polynomial fitting to find the formula for the $n$th term of the following sequences:
\begin{parts}
\part 2, 5, 11, 21, 36,\ldots
\part 0, 2, 6, 12, 20, \ldots
\end{parts}

\question Can you use polynomial fitting to find the formula for the $n$th term of the sequence 4, 7, 11, 18, 29, 47, \ldots?  Explain why or why not. 

\question Find the next 2 terms in the sequence $3, 5, 11, 21, 43, 85\ldots.$.  Then give a recursive definition for the sequence.  Finally, use the characteristic root technique to find a closed formula for the sequence.

\question Solve the recurrence relation $a_n = a_{n-1} + 2^n$ with $a_0 = 5$.

\question Show that $4^n$ is a solution to the recurrence relation $a_n = 3a_{n-1} + 4a_{n-2}$.

\question Find the solution to the recurrence relation $a_n = 3a_{n-1} + 4a_{n-2}$ with initial terms $a_0 = 2$ and $a_1 = 3$.

\question Find the solution to the recurrence relation $a_n = 3a_{n-1} + 4a_{n-2}$ with initial terms $a_0 = 5$ and $a_1 = 8$.

\question Solve the recurrence relation $a_n = 2a_{n-1} - a_{n-2}$.
\begin{parts}
  \part What is the solution if the initial terms are $a_0 = 1$ and $a_1 = 2$?
  \part What do the initial terms need to be in order for $a_9 = 30$?
  \part For which $x$ are there initial terms which make $a_9 = x$?
\end{parts}

\question Solve the recurrence relation $a_n = 3a_{n-1} + 10a_{n-2}$ with initial terms $a_0 = 4$ and $a_1 = 1$.  

\end{questions}




\end{document}


