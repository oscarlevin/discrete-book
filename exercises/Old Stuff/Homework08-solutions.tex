\documentclass[10pt]{exam}

\usepackage{amssymb, amsmath, amsthm, mathrsfs, multicol, graphicx} 
\usepackage{tikz}

\def\d{\displaystyle}
\def\?{\reflectbox{?}}
\def\b#1{\mathbf{#1}}
\def\f#1{\mathfrak #1}
\def\c#1{\mathcal #1}
\def\s#1{\mathscr #1}
\def\r#1{\mathrm{#1}}
\def\N{\mathbb N}
\def\Z{\mathbb Z}
\def\Q{\mathbb Q}
\def\R{\mathbb R}
\def\C{\mathbb C}
\def\F{\mathbb F}
\def\A{\mathbb A}
\def\X{\mathbb X}
\def\E{\mathbb E}
\def\O{\mathbb O}
\def\pow{\mathscr P}
\def\inv{^{-1}}
\def\nrml{\triangleleft}
\def\st{:}
\def\~{\widetilde}
\def\rem{\mathcal R}
\def\iff{\leftrightarrow}
\def\Iff{\Leftrightarrow}
\def\and{\wedge}
\def\And{\bigwedge}
\def\AAnd{\d\bigwedge\mkern-18mu\bigwedge}
\def\Vee{\bigvee}
\def\VVee{\d\Vee\mkern-18mu\Vee}
\def\imp{\rightarrow}
\def\Imp{\Rightarrow}
\def\Fi{\Leftarrow}

\def\={\equiv}
\def\var{\mbox{var}}
\def\mod{\mbox{Mod}}
\def\Th{\mbox{Th}}
\def\sat{\mbox{Sat}}
\def\con{\mbox{Con}}
\def\bmodels{=\joinrel\mathrel|}
\def\iffmodels{\bmodels\models}
\def\dbland{\bigwedge \!\!\bigwedge}
\def\dom{\mbox{dom}}
\def\rng{\mbox{range}}
\DeclareMathOperator{\wgt}{wgt}

\def\circleA{(-.5,0) circle (1)}
\def\circleAlabel{(-1.5,.6) node[above]{$A$}}
\def\circleB{(.5,0) circle (1)}
\def\circleBlabel{(1.5,.6) node[above]{$B$}}
\def\circleC{(0,-1) circle (1)}
\def\circleClabel{(.5,-2) node[right]{$C$}}
\def\twosetbox{(-2,-1.5) rectangle (2,1.5)}
\def\threesetbox{(-2,-2.5) rectangle (2,1.5)}


\def\bar{\overline}

%\pointname{pts}
\pointsinmargin
\marginpointname{pts}
\marginbonuspointname{pts-bns}
\addpoints
\pagestyle{head}
\printanswers

\firstpageheader{Math 228}{\bf Homework 8 Solutions}{April 4, 2012}


\begin{document}

\begin{questions}
\question[5] How many triangles are there with vertices from the points shown below?  Note, we are not allowing degenerate triangles - ones with all three vertices on the same line.  Explain why your answer is correct. (HINT: you need at exactly two points on either the $x$- or $y$-axis, but don't over-count the right triangles.)

\begin{center}
  \begin{tikzpicture}[scale=.7]
    \foreach \i in {0,...,6} {
      \fill (\i,0) circle (2pt);
    }
    \foreach \i in {1,...,4} {
      \fill (0,\i) circle (2pt);
    }
  \end{tikzpicture}

  
\end{center}

\begin{solution}
  There are 120 triangles.  Here are two ways (there are others as well) to get this:
  
  \begin{enumerate}
    \item First count the triangles with the base on the $x$-axis.  There are ${7 \choose 2}$ ways to pick the base.  The third vertex of the triangle must be one of the 4 dots on the $y$-axis (not the origin) so there are a total of ${7 \choose 2}4$ of these triangles.  The triangles with base on the $y$ axis can be counted similarly: ${5 \choose 2}6$.  However, we have counted all the right triangles twice - they have a base on the $x$-axis and also on the $y$-axis.  There are $4 \cdot 6$ right triangles.  Thus the total number of triangles is:
    \[{7 \choose 2}4 + {5 \choose 2}6 - 6\cdot 4 = 120\]
    \item We must select 3 of the 11 dots.  This can be done in ${11 \choose 3}$ ways.  However, this will also give us degenerate triangles when all three vertices are on the $x$-axis or on the $y$-axis.  There are ${7 \choose 3}$ ways we could have picked all three vertices on the $x$-axis.  There are ${5 \choose 3}$ ways we could have picked all three vertices on the $y$-axis.  Therefore the total number of triangles is
    \[{11 \choose 3} - {7 \choose 3} - {5 \choose 3} = 120\]
  \end{enumerate}

\end{solution}

\question[6] Suppose you own $a$ fezzes and $b$ bow ties.  Of course, $a$ and $b$ are both greater than 1.
\begin{parts}
  \part How many combinations of fez and bow tie can you make?  You can wear only one fez and one bow tie at a time.  Explain.
  \begin{solution}
    You have $a$ choices for the fez, and for each choice of fez you have $b$ choices for the bow tie.  Thus you have $a \cdot b$ choices for fez and bow tie combination.
  \end{solution}

  \part Explain why the answer is {\em also} ${a+b \choose 2} - {a \choose 2} - {b \choose 2}$.  (If this is what you claimed the answer was in part (a), try it again.)
  \begin{solution}
    Line up all $a+b$ quirky clothing items - the $a$ fezzes and $b$ bow ties.  Now pick 2 of them.  This can be done in ${a+b \choose 2}$ ways.  However, we might have picked 2 fezzes, which is not allowed.  There are ${a \choose 2}$ ways to pick 2 fezzes.  Similarly, the ${a+b \choose 2}$ ways to pick two items includes ${b \choose 2}$ ways to select 2 bow ties, also not allowed.  Thus the total number of ways to pick a fez and a bow ties is
    \[{a+b \choose 2} - {a \choose 2} - {b \choose 2}\]
  \end{solution}

  \part Use your answers to parts (a) and (b) to give a combinatorial proof of the identity
  \[{a+b \choose 2} - {a \choose 2} - {b \choose 2} = ab\]
  \begin{solution}
  \begin{proof}
       The question is how many ways can you select one of $a$ fezzes and one of $b$ bow ties.  We answer this question in two ways.  First, the answer could be $a\cdot b$. This is correct as described in part (a) above.  Second, the answer could be ${a+b \choose 2} - {a \choose 2} - {b \choose 2}$.  This is correct as described in part (b) above.  Therefore 
    \[{a+b \choose 2} - {a \choose 2} - {b \choose 2} = ab\]
  \end{proof}
  \end{solution}

\end{parts}


\question Consider the identity:
\[k{n\choose k} = n{n-1 \choose k-1}\]
\begin{parts}
  \part[2] Is this true?  Try it for a few values of $n$ and $k$.
  \begin{solution}
    Yes.  For example, if $n = 7$ and $k = 4$, we have \[4 \cdot {7 \choose 4} = 4 \cdot 35 = 140 = 7 \cdot 20 = 7 \cdot {6 \choose 3}\]
  \end{solution}

  \part[3] Use the formula for ${n \choose k}$ to give an algebraic proof of the identity.
  \begin{solution}
    \[k{n \choose k} = k \frac{n!}{(n-k)!\,k!} = \frac{n!}{(n-k)!(k-1)!} = n\frac{(n-1)!}{(n-1-(k-1))!(k-1)!} = n {n-1 \choose k-1}\]
  \end{solution}

  \part[4] Give a combinatorial proof of the identity. Hint: How many ways can you select a chaired committee of $k$ people from a group of $n$ people?  
  \begin{solution}
    \begin{proof}
      Question: How many ways can you select a chaired committee of $k$ people from a group of $n$ people?  That is, you need to select $k$ people to be on the committee and one of them needs to be in charge.  How many ways can this happen?
      
      Answer 1: First select $k$ of the $n$ people to be on the committee.  This can be done in ${n \choose k}$ ways.  Now select one of those $k$ people to be in charge - this can be done in $k$ ways.  So there are a total of $k {n \choose k}$ ways to select the chaired committee.
      
      Answer 2: First select the chair of the committee.  You have $n$ people to choose from, so this can be done in $n$ ways.  Now fill the rest of the committee.  There are $n-1$ people to choose from (you cannot select the person you picked to be the chair) and $k-1$ spots to fill (the chair's spot is already taken).  So this can be done in ${n-1 \choose k-1}$ ways.  Therefore there are $n{n-1 \choose k-1}$ ways to select the chaired committee.
    \end{proof}

  \end{solution}

\end{parts}




\end{questions}




\end{document}


