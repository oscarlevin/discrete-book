\documentclass[11pt]{exam}

\usepackage{amsmath, amssymb, multicol}
\usepackage{graphicx}
\usepackage{textcomp}
\usepackage{tikz}

\def\d{\displaystyle}
\def\b{\mathbf}
\def\R{\mathbb{R}}
\def\Z{\mathbb{Z}}
\def\N{\mathbb{N}}
\def\pow{\mathcal{P}}
\def\st{~:~}
\def\bar{\overline}
\def\inv{^{-1}}
\def\imp{\rightarrow}
\def\and{\wedge}


%\pointname{pts}
\pointsinmargin
\marginpointname{pts}
\addpoints
\pagestyle{head}
%\printanswers

\firstpageheader{Math 228}{\bf Practice Problems 2: Sets\\ Hints and Answers}{Spring 2012}


\begin{document}
\noindent \textbf{Instructions}: The problems below are purely for you to practice.  I will not collect these, but it is still a good idea to write out your solutions in full.  Any of these problems or problems similar are fair game for quizzes and exams.  

\begin{questions}
\question % Let $A = \{1,2,3,4,5\}$, $B = \{3,4,5,6,7\}$ and $C = \{2,3,5\}$.
\begin{parts}
 \part $A \cap B = \{3,4,5\}$.  %Find $A \cap B$.
 \part $A \cup B = \{1,2,3,4,5,6,7\}$. %Find $A \cup B$.
 \part $A \setminus B = \{1,2\}$. %Find $A \setminus B$.
 \part Yes.  %Is $C \subseteq A$?
 \part No. %Is $C \subseteq B$?
\end{parts}

 \question %Let $A = \{x \in \N \st 3 \le x \le 13\}$, $B = \{x \in \N \st x \mbox{ is even}\}$, and $C = \{x \in \N \st x \mbox{ is odd}\}$.
\begin{parts}
  %Find $A \cap B$
  \part $A \cap B = \{4,6,8,10,12\}$
  % Find $A \cup B$.
  \part $A \cup B = \{x \in \N \st (3 \le x \le 13) \vee x \mbox{ is even}\}.$ (the set of all natural numbers which are either even or between 3 and 13 inclusive).
  %Find $B \cap C$.
  \part $B \cap C = \emptyset$.
  %Find $B \cup C$.
  \part $B \cup C = \N$.
\end{parts}


%  Find an example of sets $A$ and $B$ such that $A\cap B = \{3, 5\}$ and $A \cup B = \{2, 3, 5, 7, 8\}$.
 \question For example, $A = \{2,3,5,7,8\}$ and $B = \{3,5\}$.

%  Find an example of sets $A$ and $B$ such that $A \subseteq B$ and $A \in B$.
\question Let $A = \{1,2,3\}$ and $B = \{1,2,3,4,5,\{1,2,3\}\}$

%  Recall $\Z = \{\ldots,-2,-1,0, 1,2,\ldots\}$ (the integers).  Let $\Z^+$ be the positive integers.  Let $2\Z$ be the even integers, $3\Z$ be the multiples of 3, and so on.
\question
\begin{parts}
%    Is $\Z^+ \subseteq 2\Z$? 
\part No.
%   Is $2\Z \subseteq \Z^+$? 
\part No.
%    Find $2\Z \cap 3\Z$.  Describe the set in words, and also in symbols (using a $\st$ symbol).
\part $2\Z \cap 3\Z$ is the set of all integers which are multiples of both 2 and 3 (so multiples of 6).  Therefore $2\Z \cap 3\Z = \{x \in \Z \st \exists y\in \Z(x = 6y)\}$.
%    Express $\{x \in \Z \st \exists y\in \Z (x = 2y \vee x = 3y)\}$ as a union or intersection of two sets above.
\part $2\Z \cup 3\Z$.
 \end{parts}


%  Let $A_2$ be the set of all multiples of 2 except for $2$.  Let $A_3$ be the set of all multiples of 3 except for 3.  And so on, so that $A_n$ is the set of all multiple of $n$ except for $n$, for any $n \ge 2$.  Describe (in words) the set $\bar{A_2 \cup A_3 \cup A_4 \cup \cdots}$
\question The set of primes.


%  Draw a Venn diagram to represent each of the following:
\question 


  \begin{parts}

\def\circleA{(-.5,0) circle (1)}
\def\circleAlabel{(-1.5,.6) node[above]{$A$}}
\def\circleB{(.5,0) circle (1)}
\def\circleBlabel{(1.5,.6) node[above]{$B$}}
\def\circleC{(0,-1) circle (1)}
\def\circleClabel{(.5,-2) node[right]{$C$}}
\def\twosetbox{(-2,-1.5) rectangle (2,1.5)}
\def\threesetbox{(-2,-2.5) rectangle (2,1.5)}


\begin{multicols}{3}

\part  $A \cup \bar B$:

\begin{tikzpicture}[fill=gray!50]
 %Fill A:
 \fill \circleA;
 %Fill \bar B:
  \begin{scope}
  \clip \circleB \twosetbox; %This defines the scope to everything in the twosetbox which is not in circleB.
  \fill \twosetbox;
  \end{scope}
  \draw[thick] \circleA \circleAlabel \circleB \circleBlabel \twosetbox;
\end{tikzpicture}

\vfill

%   
\part $\bar{(A \cup B)}$:

\begin{tikzpicture}[fill=gray!50]
  \fill \twosetbox;
  \fill[white] \circleA \circleB;
  \draw[thick] \circleA \circleAlabel \circleB \circleBlabel \twosetbox;
\end{tikzpicture}

\columnbreak

%   
\part $A \cap (B \cup C)$:

\begin{tikzpicture}[fill=gray!50]
\begin{scope}
  \clip \circleA;
  \fill \circleB \circleC;
\end{scope}
\draw[thick] \circleA \circleAlabel \circleB \circleBlabel \circleC \circleClabel \threesetbox;
\end{tikzpicture}



%  
\part $(A \cap B) \cup C$:

\begin{tikzpicture}[fill=gray!50]
\begin{scope}
  \clip \circleA;
  \fill \circleB;
\end{scope}
\fill \circleC;
\draw[thick] \circleA \circleAlabel \circleB \circleBlabel \circleC \circleClabel \threesetbox;
\end{tikzpicture}
\vfill
\columnbreak

%   
\part $\bar A \cap B \cap \bar C$:

\begin{tikzpicture}[fill=gray!50]
\fill \circleB;
\begin{scope}
  \clip \circleB;
  \fill[white] \circleA \circleC;
\end{scope}

\draw[thick] \circleA \circleAlabel \circleB \circleBlabel \circleC \circleClabel \threesetbox;
\end{tikzpicture}

%   
\part $(A \cup B) \setminus C$:

\begin{tikzpicture}[fill=gray!50]
\fill \circleA;
\fill \circleB;
\fill[white] \circleC;
\draw[thick] \circleA \circleAlabel \circleB \circleBlabel \circleC \circleClabel \threesetbox;
\end{tikzpicture}
\end{multicols}
 \end{parts}





%  Describe a set in terms of $A$ and $B$ which has the following Venn diagram:
\question For example, $A \cup B \cap \bar{(A \cap B)}$.  Note that $\bar{A \cap B}$ would almost work, but also contain the area outside of both circles.

% \begin{center}
% \begin{tikzpicture}[fill=gray]
% % left hand
% \scope
% \clip (-2,-2) rectangle (2,2)
%       (1,0) circle (1);
% \fill (0,0) circle (1);
% \endscope
% % right hand
% \scope
% \clip (-2,-2) rectangle (2,2)
%       (0,0) circle (1);
% \fill (1,0) circle (1);
% \endscope
% % outline
% \draw[thick] (0,0) circle (1) (-1,.7)  node [text=black,above] {$A$}
%       (1,0) circle (1) (2,.7)  node [text=black,above] {$B$}
%       (-1.5,-1.5) rectangle (2.5,1.5);
% \end{tikzpicture}
% \end{center}
% 


% Find the cardinalities:
\question 
 \begin{parts}
%    Find $|A|$ when $A = \{4,5,6,\ldots,37\}$
\part 34.
%   Find $|A|$ when $A = \{x \in \Z \st -2 \le x \le 100\}$
\part 103.
%    Find $|A \cap B|$ when $A = \{x \in \N \st x \le 20\}$ and $B = \{x \in \N \st x \mbox{ is prime}\}$
\part 8.
 \end{parts}
 
%  Let $A = \{a, b, c\}$.  Find $\pow(A)$.
\question $\pow(A) = \{\emptyset, \{a\}, \{b\}, \{c\}, \{a,b\}, \{a,c\}, \{b,c\}, \{a,b,c\}\}$.

%  Let $A = \{1,2,\ldots, 10\}$.  How many subsets of $A$ contain exactly one element (i.e., how many {\em singleton} subsets are there).  How many {\em doubleton} (containing exactly two elements) are there?
\question There are 10 singletons.  There are 45 doubletons (because $45 = 9+8+7+\cdots+2+1$).


%  Let $A = \{1,2,3,4,5,6\}$.  Find all sets $B \in \pow(A)$ which have the property $\{2,3,5\} \subseteq B$.
\question $\{2,3,5\}, \{1,2,3,5\}, \{2,3,4,5\}, \{2,3,5,6\}, \{1,2,3,4,5\}, \{1,2,3,5,6\}, \{2,3,4,5,6\}$, and $\{1,2,3,4,5,6\}$.


%  Find an example of sets $A$ and $B$ such that $|A| = 4$, $|B| = 5$ and $|A \cup B| = 9$.  
\question For example $A = \{1,2,3,4\}$ and $B = \{5,6,7,8,9\}$.

% Find an example of sets $A$ and $B$ such that $|A| = 3$, $|B| = 4$ and $|A \cup B| = 5$.
\question For example, $A = \{1,2,3\}$ and $B = \{2,3,4,5\}$.

% If $|A| = 10$ and $|B| = 15$, what is the largest possible value for $|A \cap B|$?  What is the smallest?  What are the possible values for $|A \cup B|$?
\question $0 \le |A \cap B| \le 10$ and $15 \le |A \cup B| \le 25$.

%  If $|A| = 8$ and $|B| = 5$, what is $|A \cup B| + |A \cap B|$?
\question $|A \cup B| + |A \cap B| = 13$


%  In a regular deck of playing cards there are 26 red cards and 12 face cards.  Explain in terms of sets why there are only 32 cards which are either red or a face card.
\question If $R$ is the set of red cards and $F$ is the set of face cards, we have $|R \cup F| = |R| + |F| - |R \cap F|$.  There are 6 cards which are both red and a face card, so $|R \cup F| = 32$.


%  A group of college students were asked about their TV watching habits.  Of those surveyed, 28 students watch {\em House}, 19 watch {\em Castle} and 24 watch re-runs of {\em 24}.  Additionally, 16 watch {\em House} and {\em Castle}, 14 watch {\em House} and {\em 24} and 10 watch {\em Castle} and {\em 24}.  There are 8 students who watch all three shows.  How many students surveyed watched at least one of the shows?
\question 39.

%Find $|(A \cup C)\cap \bar B|$ provided $|A| = 50$, $|B| = 45$, $|C| = 40$, $|A\cap B| = 20$, $|A \cap C| = 15$, $|B \cap C| = 23$ and $|A \cap B \cap C| = 12$.
\question $|(A \cup C)\cap \bar B| = 44$


%Using the same data as the previous question, describe a set with cardinality 26.
\question One possibility: $(A \cup B) \cap C$.



\end{questions}




\end{document}


