\documentclass[11pt]{exam}

\usepackage{amsmath, amssymb, amsthm, multicol}
\usepackage{graphicx}
\usepackage{textcomp}
\usepackage{tikz}

\def\d{\displaystyle}
\def\b{\mathbf}
\def\R{\mathbb{R}}
\def\Z{\mathbb{Z}}
\def\N{\mathbb{N}}
\def\pow{\mathcal{P}}
\def\st{~:~}
\def\bar{\overline}
\def\inv{^{-1}}
\def\imp{\rightarrow}
\def\and{\wedge}


%\pointname{pts}
\pointsinmargin
\marginpointname{pts}
\addpoints
\pagestyle{head}
%\printanswers

\firstpageheader{Math 228}{\bf Practice Problems 3: Functions\\ Hints and Answers}{Spring 2012}


\begin{document}

\begin{questions}
%Find all functions $f: \{1,2,3\}$ to $\{a,b\}$.  How many are there?  How many are one-to-one?  How many are onto?  How many are both?
\question There are 8 different functions.  For example, $f(1) = a$, $f(2) = a$, $f(3) = a$; or $f(1) = a$, $f(2) = b$, $f(3) = a$, and so on.  None of the functions are one-to-one.  Exactly 6 of the functions are onto.  No functions are both (since no functions here are one-to-one).


%Find all functions $f: \{1,2\}$ to $\{a,b,c\}$.  How many are there?  How many are one-to-one?  How many are onto?  How many are both?
\question There are nine functions - you have a choice of three outputs for $f(1)$, and for each, you have three choices for the output $f(2)$.  Of these functions, 6 are one-to-one, 0 are onto, and 0 are both.



%  Consider the function $f:\{1,2,3,4,5\} \to \{1,2,3,4\}$ given by the table below:
% \centerline{\begin{tabular}{c||c|c|c|c|c}
%               $x$ & 1 & 2 & 3 & 4 & 5 \\ \hline
%               $f(x)$ & 3 & 2 & 4 & 1 & 2
%             \end{tabular}
%             }
\question
\begin{parts}
%Is $f$ one-to-one?  Explain.
\part $f$ is not one-to-one, since $f(2) = f(5)$ - two different inputs have the same output. 
% Is $f$ onto?  Explain.
\part $f$ is onto, since every element of the codomain is an element of the range.
\end{parts}


% Consider the function $f:\{1,2,3,4\} \to \{1,2,3,4\}$ given by the graph below.
% \begin{multicols}{2}
% \begin{center}
%   \begin{tikzpicture}[scale=1]
%     \draw[thin, gray!50] (0,0) grid (4.5, 4.5);
%     \draw[<->, thick] (0,4.5) node[left] {$f(x)$} -- (0,0) -- (4.5,0) node[below right] {$x$};
%     \foreach \x in {1,2,3,4}
%       \draw (\x,0) node[below] {\tiny \x} (0, \x) node[left] {\tiny \x};
%     \fill (1,3) circle (.1) (2,4) circle (.1) (3,1) circle (.1) (4,3) circle (.1);
%   \end{tikzpicture}
% \end{center}
\question 

\begin{parts}
%Is $f$ one-to-one?  Explain.
  \part $f$ is not one-to-one, since $f(1) = 3$ and $f(4) = 3$.
%   Is $f$ onto?  Explain.
  \part $f$ is not onto, since there is no input which gives 2 as an output.
\end{parts}

% \end{multicols}

% For each function given below, determine whether or not the function is one-to-one and whether or not the function is onto.
\question 
\begin{parts}
% $f:\N \to \N$ given by $f(n) = n+4$.
  \part $f$ is one-to-one, but not onto.
%   $f:\Z \to \Z$ given by $f(n) = n+4$.
  \part $f$ is one-to-one and onto.
  %$f:\Z \to \Z$ given by $f(n) = 5n - 8$.
  \part $f$ is one-to-one, but not onto.
%   $f:\Z \to \Z$ given by $f(n) = \begin{cases}
%                                          n/2 & \mbox{ if $n$ is even}\\
%                                          (n+1)/2 & \mbox{ if $n$ is odd}.
%                                        \end{cases}$
 \part $f$ is not one-to-one, but is onto.
\end{parts}


%Let $A = \{1,2,3,\ldots,10\}$.  Consider the function $f:\pow(A) \to \N$ given by $f(B) = |B|$.  So $f$ takes a subset of $A$ as an input and outputs the cardinality of that set
\question  
\begin{parts}
% Is $f$ one-to-one?  Prove your answer.
  \part $f$ is not one-to-one.  To prove this, we must simply find two different elements of the domain which map to the same element of the codomain.  Since $f(\{1\}) = 1$ and $f(\{2\}) = 1$, we see that $f$ is not one-to-one.
%   Is $f$ onto?  Prove your answer.
  \part $f$ is not onto.  The largest subset of $A$ is $A$ itself, and $|A| = 10$.  So no natural number greater than 10 will ever be an output.
%   Find $f\inv(1)$.
  \part $f\inv(1) = \{\{1\}, \{2\}, \{3\}, \ldots \{10\}\}$ (the set of all the singleton subsets of $A$).
%   Find $f\inv(0)$.
  \part $f\inv(0) = \{\emptyset\}$.  Note, it would be wrong to write $f\inv(0) = \emptyset$ - that would claim that there is no input which has 0 as an output.
   % Find $f\inv(12)$.
  \part $f\inv(12) = \emptyset$, since there are no subsets of $A$ with cardinality 12.
\end{parts}

% Let $A = \{n \in \N \st 0 \le n \le 999\}$ be the set of all numbers with three or fewer digits.  Define the function $f:A \to \N$ by $f(abc) = a+b+c$, where $a$, $b$, and $c$ are the digits of the number in $A$.  For example, $f(253) = 2 + 5 + 3 =  10$.
\question 
\begin{parts}
% Find $f\inv(3)$.
  \part $f\inv(3) = \{003, 030, 300, 012, 021, 102, 201, 120, 210, 111\}$
%   Find $f\inv(28)$.
  \part $f\inv(28) = \emptyset$ (since the largest sum of three digits is $9+9+9 = 27$)
%   Use one of the parts above to prove that $f$ is not one-to-one.
  \part Part (a) proves that $f$ is not one-to-one - the output 3 is assigned to 10 different inputs.
%   Use one of the parts above to prove that $f$ is not onto.
  \part Part (b) proves that $f$ is not onto - there is an element of the codomain (28) which is assigned to no inputs.
\end{parts}

% Find a set $X$ and a function $f:X \to \N$ so that $f\inv(0) \cup f\inv(1) = X$.
\question $X$ can really be any set, as long as $f(x) = 0$ or $f(x) = 1$ for every $x \in X$.  For example, $X = \N$ and $f(n) = 0$ works.

% What can you deduce about the sets $X$ and $Y$ if you know,
\question 
\begin{parts}
% there is a one-to-one function $f:X \to Y$.  Explain.
  \part $|X| \le |Y|$ - otherwise two or more of the elements of $X$ would need to map to the same element of $Y$.
%   there is a onto function $f:X \to Y$.  Explain.
  \part $|X| \ge |Y|$ - otherwise there would be one or more elements of $Y$ which were never an output.
%   there is a bijection $f:X \to Y$.  Explain.
  \part $|X| = |Y|$.  This is the only way for both of the above to occur.
\end{parts}

% Suppose $f:X \to Y$ is a function.  Which of the following are possible?  Explain.
\question 
\begin{parts}
% $f$ is one-to-one but not onto.
  \part Yes. (Can you give an example?)
  \part Yes. %$f$ is onto but not one-to-one.
  \part Yes. %$|X| = |Y|$ and $f$ is one-to-one but not onto.
  \part Yes. %$|X| = |Y|$ and $f$ is onto but not one-to-one.
  \part No. %$|X| = |Y|$, $X$ and $Y$ are finite, and $f$ is one-to-one but not onto.
  \part No. %$|X| = |Y|$, $X$ and $Y$ are finite, and $f$ is onto but not one-to-one.
\end{parts}

% Consider the function $f:\Z \to \Z$ given by $f(n) = \begin{cases}
%                                                                  n+1 & \mbox{ if $n$ is even}\\
%                                                                  n-3 & \mbox{ if $n$ is odd}.
%                                                                \end{cases}$
\question 
\begin{parts}
% Is $f$ one-to-one?  Prove your answer.
  \part $f$ is one-to-one. 
  \begin{proof}
   Let $x$ and $y$ be elements of the domain $\Z$.  Assume $f(x) = f(y)$.  If $x$ and $y$ are both even, then $f(x) = x+1$ and $f(y) = y+1$.  Since $f(x) = f(y)$, we have $x + 1 = y + 1$ which implies that $x = y$.  Similarly, if $x$ and $y$ are both odd, then $x - 3 = y-3$ so again $x = y$.  The only other possibility is that $x$ is even an $y$ is odd (or visa-versa).  But then $x + 1$ would be odd and $y - 3$ would be even, so it cannot be that $f(x) = f(y)$.  Therefore if $f(x) = f(y)$ we then have $x = y$, which proves that $f$ is one-to-one.
  \end{proof}
% Is $f$ onto?  Prove your answer.
  \part $f$ is onto.
  \begin{proof}
   Let $y$ be an element of the codomain $\Z$.  We will show there is an element $n$ of the domain ($\Z$) such that $f(n) = y$.  There are two cases.  First, if $y$ is even, then let $n = y+3$.  Since $y$ is even, $n$ is odd, so $f(n) = n-3 = y+3-3 = y$ as desired.  Second, if $y$ is odd, then let $n = y-1$.  Since $y$ is odd, $n$ is even, so $f(n) = n+1 = y-1+1 = y$ as needed.  Therefore $f$ is onto.
  \end{proof}

\end{parts}


\end{questions}




\end{document}


