\documentclass[11pt]{exam}

\usepackage{amsmath, amssymb, multicol}
\usepackage{graphicx}
\usepackage{textcomp}
\usepackage{chessboard}

\def\d{\displaystyle}
\def\b{\mathbf}
\def\R{\mathbf{R}}
\def\Z{\mathbf{Z}}
\def\st{~:~}
\def\bar{\overline}
\def\inv{^{-1}}
\def\imp{\rightarrow}
\def\and{\wedge}


%\pointname{pts}
\pointsinmargin
\marginpointname{pts}
\addpoints
\pagestyle{head}
%\printanswers

\firstpageheader{Math 228}{\bf Practice Problems 1: Logic}{January 18, 2012}


\def\filename{Practice01-Logic}
\def\ansfilename{\filename-solutions}

				
\Opensolutionfile{\ansfilename}
\Newassociation{answer}{Ans}{\ansfilename}


\begin{document}
 

\begin{questions}
\question Make a truth table for the statement $(P \vee Q) \imp (P \and Q)$. 

\begin{answer}
  \begin{tabular}{c|c|c}
             $P$ & $Q$ & $(P \vee Q) \imp (P \and Q)$\\ \hline
             T & T & T \\
             T & F & F \\
             F & T & F \\
             F & F & T
          \end{tabular}
\end{answer}


\question Make a truth table for the statement $\neg P \wedge (Q \imp P)$.  What can you conclude about $P$ and $Q$ if you know the statement is true?

\begin{answer}
  \begin{tabular}{c|c|c}
             $P$ & $Q$ & $\neg P \and (Q \imp P)$\\ \hline
             T & T & F \\
             T & F & F \\
             F & T & F \\
             F & F & T
          \end{tabular}
If the statement is true, then both $P$ and $Q$ are false.
\end{answer}


\question Make a truth table for the statement $\neg P \imp (Q \and R)$.

\begin{answer}
  Hint: Like above, only now you will need 8 rows instead of just 4.
\end{answer}


\question Determine whether the following two statements are logically equivalent: $\neg(P \imp Q)$ and $P \and \neg Q$.  Explain how you know you are correct.

\begin{answer}
  Make a truth table for each and compare.  The statements are logically equivalent.
\end{answer}


\question Are the statements $P \imp (Q\vee R)$ and $(P \imp Q) \vee (P \imp R)$ logically equivalent?

\begin{answer}
   Again, make two truth tables.  The statements are logically equivalent.
\end{answer}


\question Consider the statement about a party, ``If it's your birthday or there will be cake, then there will be cake.''
\begin{parts}
 \part Translate the above statement into symbols.  Clearly state which statement is $P$ and which is $Q$.
 \part Make a truth table for the statement.
 \part Assuming the statement is true, what (if anything) can you conclude if there will be cake?
 \part Assuming the statement is true, what (if anything) can you conclude if there will not be cake?
 \part Suppose you found out that the statement was a lie.  What can you conclude?
\end{parts}

\begin{answer}
  \begin{parts}
 \part $P$: it's your birthday; $Q$: there will be cake.  $(P \vee Q) \imp Q$
 \part Hint: you should get three T's and one F.
 \part Only that there will be cake.
 \part It's your birthday!
 \part The cake is a lie.
\end{parts}
\end{answer}


\question Suppose $P$ and $Q$ are the statements:
$P$: Jack passed math.
$Q$: Jill passed math.
\begin{parts}
 \part Translate ``Jack and Jill both passed math'' into symbols.
\part Translate ``If Jack passed math, then Jill did not'' into symbols.
\part Translate ``$P \vee Q$'' into English.
\part Translate ``$\neg(P \and Q) \imp Q$'' into English.
\part Suppose you know that if Jack passed math, then so did Jill.  What can you conclude if you know that:
\begin{subparts}
 \subpart Jill passed math?  
\subpart  Jill did not pass math?
\end{subparts}
\end{parts}

\begin{answer}
  \begin{parts}
\part $P \and Q$
\part $P \imp \neg Q$
\part Jack passed math or Jill passed math (or both).
\part If Jack and Jill did not both pass math, then Jill did.
\part 
\begin{subparts}
 \subpart Nothing else. 
\subpart  Jack did not pass math either.
\end{subparts}
\end{parts}

\end{answer}



\question Geoff Poshingten is out at a fancy pizza joint, and decides to order a calzone.  When the waiter asks what he would like in it, he replies, ``I want either pepperoni or sausage, and if I have sausage, I must also include quail.  Oh, and if I have pepperoni or quail then I must also have ricotta cheese.''  
\begin{parts}
	\part Translate Geoff's order into logical symbols.
	\part The waiter knows that Geoff is either a liar or a truth-teller (so either everything he says is false, or everything is true).  Which is it?
	\part What, if anything, can the waiter conclude about the ingredients in Geoff's desired calzone?
\end{parts}

\begin{answer}
  \begin{parts}
	\part Three statements: $P \vee S$, $S \imp Q$, $(P \vee Q) \imp R$.  You could also connect the first two with a $\and$.
	\part He cannot be lying about all three sentences, so he is telling the truth.
	\part No matter what, Geoff wants ricotta.  If he doesn't have quail, then he must have pepperoni but not sausage.
\end{parts}
\end{answer}



\question Consider the statement ``If Oscar eats Chinese food, then he drinks milk.''
\begin{parts}
 \part Write the converse of the statement.
 \part Write the contrapositive of the statement.
 \part Is it possible for the contrapositive to be false?  If it was, what would that tell you?
 \part Suppose the original statement is true, and that Oscar drinks milk.  Can you conclude anything (about his eating Chinese food)?  Explain.
 \part Suppose the original statement is true, and that Oscar does not drink milk.  Can you conclude anything (about his eating Chinese food)?  Explain.
\end{parts}

\begin{answer}
  \begin{parts}
 \part If Oscar drinks milk, then he eats Chinese food.
 \part If Oscar does not drink milk, then he does not eat Chinese food.
 \part Yes.  The original statement would be false too.
 \part Nothing. The converse need not be true.
 \part He does not eat Chinese food. The contrapositive would be true.
\end{parts}
\end{answer}



\question Simplify the following statements (so that negation only appears right before variables).
\begin{parts}
  \part $\neg(P \imp \neg Q)$
  \part $(\neg P \vee \neg Q) \imp \neg (\neg Q \and R)$
  \part $\neg((P \imp \neg Q) \vee \neg (R \and \neg R))$
  \part It is false that if Sam is not a man then Chris is a woman, and that Chris is not a woman.
\end{parts}

\begin{answer}
  \begin{parts}
  \part $P \and Q$ 
  \part $(P \vee Q) \vee (Q \and \neg R)$
  \part F.  Or $(P \and Q) \and (R \and \neg R)$ 
  \part Either Sam is a woman and Chris is a man, or Chris is a woman.
\end{parts}
\end{answer}


\question Translate into symbols.  Use $E(x)$ for ``$x$ is even'' and $O(x)$ for ``$x$ is odd.''
 \begin{parts}
  \part No number is both even and odd.
\part One more than any even number is an odd number.
\part There is prime number that is even.
\part Between any two numbers there is a third number.
\part There is no number between a number and one more than that number.
 \end{parts}

\begin{answer}
  \begin{parts}
  \part $\neg \exists x (E(x) \and O(x))$
\part $\forall x (E(x) \imp O(x+1))$
\part $\exists x(P(x) \and E(x))$ (where $P(x)$ means ``$x$ is prime'')
\part $\forall x \forall y \exists z(x < z < y \vee y < z < x)$
\part $\forall x \neg \exists y (x < y < x+1)$
 \end{parts}
\end{answer}


\question Translate into English:
\begin{parts}
 \part $\forall x (E(x) \imp E(x +2))$
\part $\forall x \exists y (\sin(x) = y)$
\part $\forall y \exists x (\sin(x) = y)$
\part $\forall x \forall y (x^3 = y^3 \imp x = y)$
\end{parts}

\begin{answer}
  \begin{parts}
 \part Any even number plus 2 is an even number.
\part For any $x$ there is a $y$ such that $\sin(x) = y$.  In other words, every number $x$ is in the domain of sine. 
\part For every $y$ there is an $x$ such that $\sin(x) = y$.  In other words, every number $y$ is in the range of sine (which is false).
\part For any numbers, if the cubes of two numbers are equal, then the numbers are equal.
\end{parts}
\end{answer}


\question Simplify the statements (so negation appears only directly next to predicates).
\begin{parts}
  \part $\neg \exists x \forall y (\neg O(x) \vee E(y))$
  \part $\neg \forall x \neg \forall y \neg(x < y \and \exists z (x < z \vee y < z))$
  \part There is a number $n$ for which no other number is either less $n$ than or equal to $n$.
  \part It is false that for every number $n$ there are two other numbers which $n$ is between.
\end{parts}

\begin{answer}
  \begin{parts}
  \part $\forall x \exists y (O(x) \and \neg E(y))$
  \part $\exists x \forall y (x \ge y \vee \forall z (x \ge z \and y \ge z))$
  \part There is a number $n$ for which every other number is strictly greater than $n$.
  \part There is a number $n$ which is not between any other two numbers.
\end{parts}
\end{answer}



\question Consider the statement ``for all integers $a$ and $b$, if $a + b$ is even, then $a$ and $b$ are even''
\begin{parts}
 \part Write the contrapositive of the statement
 \part Write the converse of the statement
 \part Write the negation of the statement.
 \part Is the original statement true or false?  Prove your answer.
 \part Is the contrapositive of the original statement true or false?  Prove your answer.
 \part Is the converse of the original statement true or false?  Prove your answer.
 \part Is the negation of the original statement true or false?  Prove your answer.
\end{parts}

\begin{answer}
  \begin{parts}
 \part For all integers $a$ and $b$, if $a$ or $b$ are not even, then $a+b$ is not even.
 \part For all integers $a$ and $b$, if $a$ and $b$ are even, then $a+b$ is even.
 \part There are numbers $a$ and $b$ such that $a+b$ is even but $a$ and $b$ are not both even.
 \part False.  For example, $a = 3$ and $b = 5$.  $a+b = 8$, but neither $a$ nor $b$ are even.
 \part False, since it is equivalent to the original statement.
 \part True.  Let $a$ and $b$ be integers.  Assume both are even.  Then $a = 2k$ and $b = 2j$ for some integers $k$ and $j$.  But then $a+b = 2k + 2j = 2(k+j)$ which is even.
 \part True, since the statement is false.
\end{parts}
\end{answer}



\question Prove that $\sqrt 3$ is irrational.

\begin{answer}
  Suppose $\sqrt{3}$ were rational.  Then $\sqrt{3} = \frac{a}{b}$ for some integers $a$ and $b \ne 0$.  Without loss of generality, assume $\frac{a}{b}$ is reduced.  Now
\[3 = \frac{a^2}{b^2}\]
\[b^2 3 = a^2\]
So $a^2$ is a multiple of 3.  This can only happen if $a$ is a multiple of 3, so $a = 3k$ for some integer $k$.  Then we have
\[b^2 3 = 9k^2\]
\[b^2 = 3k^2\]
So $b^2$ is a multiple of 3, making $b$ a multiple of 3 as well.  But this contradicts our assumption that $\frac{a}{b}$ is in lowest terms.
\end{answer}


\question Write the negation, converse and contrapositive for each of the statements below.
\begin{parts}
  \part If the power goes off, then the food will spoil.
  \part If the door is closed, then the light is off.
  \part $\forall x (x < 1 \imp x^2 < 1)$
  \part For all natural numbers $n$, if $n$ is prime, then $n$ is solitary.
  \part For all functions $f$, if $f$ is differentiable, then $f$ is continuous.
  \part For all integers $a$ and $b$, if $a\cdot b$ is even, then $a$ and $b$ are even.
  \part For every integer $x$ and every integer $y$ there is an integer $n$ such that if $x > 0$ then $nx > y$.
  \part For all real numbers $x$ and $y$, if $xy = 0$ then $x = 0$ or $y = 0$.
  \part For every student in Math 228, if they do not understand implications, then they will fail the exam.
\end{parts}

\begin{answer}
  \begin{parts}
  \part  Negation: The power goes off and the food does not spoil.\\
  Converse: If the food spoils, then the power went off.\\
  Contrapositive: If the food does not spoil, then the power did not go off.
  
  \part   Negation: The door is closed and the light is on.\\
  Converse: If the light is off then the door is closed.\\
  Contrapositive: If the light is on then the door is open.
  \part 
    Negation: $\exists x (x < 1 \and x^2 \ge 1)$\\
  Converse: $\forall x( x^2 < 1 \imp x < 1)$\\
  Contrapositive: $\forall x (x^2 \ge 1 \imp x \ge 1)$.
  \part Negation: There is a natural number $n$ which is prime but not solitary.\\
  Converse: For all natural numbers $n$, if $n$ is solitary, then $n$ is prime.\\
  Contrapositive: For all natural numbers $n$, if $n$ is not solitary then $n$ is not prime.
  
  \part Negation: There is a function which is differentiable and not continuous.\\
  Converse: For all functions $f$, if $f$ is continuous then $f$ is differentiable. \\
  Contrapositive: For all functions $f$, if $f$ is not continuous then $f$ is not differentiable.
  
  \part Negation: There are integers $a$ and $b$ for which $a\cdot b$ is even but $a$ or $b$ is odd.\\
  Converse: For all integers $a$ and $b$, if $a$ and $b$ are even then $ab$ is even.\\
  Contrapositive: For all integers $a$ and $b$, if $a$ or $b$ is odd, then $ab$ is odd.
  
  \part Negation: There are integers $x$ and $y$ such that for every integer $n$, $x \le 0$ and $nx \le y$. \\
  Converse: For every integer $x$ and every integer $y$ there is an integer $n$ such that if $nx > y$ then $x > 0$.\\
  Contrapositive: For every integer $x$ and every integer $y$ there is an integer $n$ such that if $nx \le y$ then $x \le 0$.
  
  \part  Negation: There are real numbers $x$ and $y$ such that $xy = 0$ but $x \ne 0$ and $y \ne 0$.\\
  Converse: For all real numbers $x$ and $y$, if $x = 0$ or $y = 0$ then $xy = 0$\\
  Contrapositive: For all real numbers $x$ and $y$, if $x \ne 0$ and $y \ne 0$ then $xy \ne 0$.
  
  \part Negation: There is at least one student in Math 228 who does not understand implications but will still pass the exam.\\
  Converse: For every student in Math 228, if they fail the exam, then they did not understand implications.\\
  Contrapositive: For every student in Math 228, if they pass the exam, then they understood implications. 
  
\end{parts}
\end{answer}


\question Consider the statement: for all integers $n$, if $n$ is even then $8n$ is even.
\begin{parts}
  \part Prove the statement.  What sort of proof are you using?
  \part Is the converse true?  Prove or disprove.
\end{parts}

\begin{answer}
  \begin{parts}
  \part Direct proof.  
  \begin{proof}
    Let $n$ be an integer.  Assume $n$ is even.  Then $n = 2k$ for some integer $k$.  Thus $8n = 16k = 2(8k)$.  Therefore $8n$ is even.
  \end{proof}

  \part The converse is false.  That is, there is an integer $n$ such that $8n$ is even but $n$ is odd.  For example, consider $n = 3$.  Then $8n = 24$ which is even but $n = 3$ is odd.
\end{parts}
\end{answer}


\question Consider the statement: for all integers $n$, if $n$ is odd, then $7n$ is odd.
\begin{parts}
  \part Prove the statement.  What sort of proof are you using?
  \part Prove the converse.  What sort of proof are you using?
\end{parts}

\begin{answer}
  \begin{parts}
  \part Direct proof.
  \begin{proof}
    Let $n$ be an integer.  Assume $n$ is odd.  So $n = 2k+1$ for some integer $k$.  Then 
    \[7n = 7(2k+1) = 14k + 7 = 2(7k +3) + 1\]
    Since $7k + 3$ is an integer, we see that $7n$ is odd.  
  \end{proof}

  \part The converse is: for all integers $n$ if $7n$ is odd, then $n$ is odd.  We will prove this by contrapositive.
  \begin{proof}
    Let $n$ be an integer.  Assume $n$ is not odd.  Then $n = 2k$ for some integer $k$.  So $7n = 14k = 2(7k)$ which is to say $7n$ is even.  Therefore $7n$ is not odd.
  \end{proof}

\end{parts}
\end{answer}


\question Consider the statement: for all integers $a$ and $b$, if $a$ is even and $b$ is a multiple of 3, then $ab$ is a multiple of 6.
\begin{parts}
  \part Prove the statement.  What sort of proof are you using?
  \part State the converse.  Is it true?  Prove or disprove.
\end{parts}

\begin{answer}
  \begin{parts}
  \part Direct proof.
  \begin{proof}
    Let $a$ and $b$ be integers.  Assume $a$ is even and $b$ is a multiple of 3.  Then $a = 2k$ and $b = 3j$ for some integers $k$ and $j$.  Now
    \[ab = (2k)(3j) = 6(kj)\]
    Since $kj$ is an integer, we have that $ab$ is a multiple of 6.
  \end{proof}

  \part The converse is: for all integers $a$ and $b$, if $ab$ is a multiple of 6, then $a$ is even and $b$ is a multiple of 3.  This is false.  Consider $a = 3$ and $b = 10$.  Then $ab = 30$ which is a multiple of 6, but $a$ is not even and $b$ is not divisible by 3.
\end{parts}
\end{answer}



\question Prove that $\log(7)$ is irrational.

\begin{answer}
  We give a proof by contradiction.
\begin{proof}
  Suppose, contrary to stipulation that $\log(7)$ is rational.  Then $\log(7) = \frac{a}{b}$ with $a$ and $b \ne 0$ integers.  By properties of logarithms, this implies
  \[7 = 10^{\frac{a}{b}}\]
  Equivalently,
  \[7^b = 10^a\]
  But this is impossible as any power of 7 will be odd while any power of 10 will be even.
\end{proof}
\end{answer}



\question Prove that there are no integer solutions to the equation $x^2 = 4y + 3$.

\begin{answer}
  Again, by contradiction.
\begin{proof}
  Suppose there were integers $x$ and $y$ such that $x^2 = 4y + 3$.  Now $x^2$ must be odd, since $4y + 3$ is odd.  Since $x^2$ is odd, we know $x$ must be odd as well.  So $x = 2k + 1$ for some integer $k$.  Then $x^2 = 4k^2 + 4k + 1 = 4(k^2 + k) + 1$.  Therefore we have,
  \[4(k^2 + k) + 1 = 4y + 3\]
  which implies
  \[4(k^2 + k) = 4y + 2\]
  and therefore
  \[2(k^2 + k) = 2y + 1.\]
  But this is a contradiction - the left hand side is even while the right hand side is odd. 
\end{proof}

\end{answer}


 
\end{questions}


\Closesolutionfile{\ansfilename}

\end{document}


