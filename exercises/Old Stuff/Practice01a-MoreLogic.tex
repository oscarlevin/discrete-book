\documentclass[11pt]{exam}

\usepackage{amsmath, amssymb, multicol}
\usepackage{graphicx}
\usepackage{textcomp}
\usepackage{chessboard}

\def\d{\displaystyle}
\def\b{\mathbf}
\def\R{\mathbf{R}}
\def\Z{\mathbf{Z}}
\def\st{~:~}
\def\bar{\overline}
\def\inv{^{-1}}
\def\imp{\rightarrow}
\def\and{\wedge}


%\pointname{pts}
\pointsinmargin
\marginpointname{pts}
\addpoints
\pagestyle{head}
%\printanswers

\firstpageheader{Math 228}{\bf Practice Problems 1a: More Logic}{Spring 2012}


\begin{document}
\noindent \textbf{Instructions}: Here are some extra practice problem on the logic and proofs stuff we have been doing.  Again, these are just for you.

\begin{questions}
\question Write the negation, converse and contrapositive for each of the statements below.
\begin{parts}
  \part If the power goes off, then the food will spoil.
  \part If the door is closed, then the light is off.
  \part $\forall x (x < 1 \imp x^2 < 1)$
  \part For all natural numbers $n$, if $n$ is prime, then $n$ is solitary.
  \part For all functions $f$, if $f$ is differentiable, then $f$ is continuous.
  \part For all integers $a$ and $b$, if $a\cdot b$ is even, then $a$ and $b$ are even.
  \part For every integer $x$ and every integer $y$ there is an integer $n$ such that if $x > 0$ then $nx > y$.
  \part For all real numbers $x$ and $y$, if $xy = 0$ then $x = 0$ or $y = 0$.
  \part For every student in Math 228, if they do not understand implications, then they will fail the exam.
\end{parts}

\question Consider the statement: for all integers $n$, if $n$ is even then $8n$ is even.
\begin{parts}
  \part Prove the statement.  What sort of proof are you using?
  \part Is the converse true?  Prove or disprove.
\end{parts}

\question Consider the statement: for all integers $n$, if $n$ is odd, then $7n$ is odd.
\begin{parts}
  \part Prove the statement.  What sort of proof are you using?
  \part Prove the converse.  What sort of proof are you using?
\end{parts}

\question Consider the statement: for all integers $a$ and $b$, if $a$ is even and $b$ is a multiple of 3, then $ab$ is a multiple of 6.
\begin{parts}
  \part Prove the statement.  What sort of proof are you using?
  \part State the converse.  Is it true?  Prove or disprove.
\end{parts}


\question Prove that $\log(7)$ is irrational.


\question Prove that there are no integer solutions to the equation $x^2 = 4y + 3$.




\end{questions}




\end{document}


