\documentclass[11pt]{exam}

\usepackage{amssymb, amsmath, amsthm, mathrsfs, multicol, graphicx} 
\usepackage{tikz}

\def\d{\displaystyle}
\def\?{\reflectbox{?}}
\def\b#1{\mathbf{#1}}
\def\f#1{\mathfrak #1}
\def\c#1{\mathcal #1}
\def\s#1{\mathscr #1}
\def\r#1{\mathrm{#1}}
\def\N{\mathbb N}
\def\Z{\mathbb Z}
\def\Q{\mathbb Q}
\def\R{\mathbb R}
\def\C{\mathbb C}
\def\F{\mathbb F}
\def\A{\mathbb A}
\def\X{\mathbb X}
\def\E{\mathbb E}
\def\O{\mathbb O}
\def\pow{\mathscr P}
\def\inv{^{-1}}
\def\nrml{\triangleleft}
\def\st{:}
\def\~{\widetilde}
\def\rem{\mathcal R}
\def\iff{\leftrightarrow}
\def\Iff{\Leftrightarrow}
\def\and{\wedge}
\def\And{\bigwedge}
\def\AAnd{\d\bigwedge\mkern-18mu\bigwedge}
\def\Vee{\bigvee}
\def\VVee{\d\Vee\mkern-18mu\Vee}
\def\imp{\rightarrow}
\def\Imp{\Rightarrow}
\def\Fi{\Leftarrow}

\def\={\equiv}
\def\var{\mbox{var}}
\def\mod{\mbox{Mod}}
\def\Th{\mbox{Th}}
\def\sat{\mbox{Sat}}
\def\con{\mbox{Con}}
\def\bmodels{=\joinrel\mathrel|}
\def\iffmodels{\bmodels\models}
\def\dbland{\bigwedge \!\!\bigwedge}
\def\dom{\mbox{dom}}
\def\rng{\mbox{range}}
\DeclareMathOperator{\wgt}{wgt}

\def\circleA{(-.5,0) circle (1)}
\def\circleAlabel{(-1.5,.6) node[above]{$A$}}
\def\circleB{(.5,0) circle (1)}
\def\circleBlabel{(1.5,.6) node[above]{$B$}}
\def\circleC{(0,-1) circle (1)}
\def\circleClabel{(.5,-2) node[right]{$C$}}
\def\twosetbox{(-2,-1.5) rectangle (2,1.5)}
\def\threesetbox{(-2,-2.5) rectangle (2,1.5)}


\def\bar{\overline}

%\pointname{pts}
\pointsinmargin
\marginpointname{pts}
\marginbonuspointname{pts-bns}
\addpoints
\pagestyle{head}
%\printanswers

\firstpageheader{Math 228}{\bf Homework 7}{Due: Wed March 28, 2012}


\begin{document}
\noindent \textbf{Instructions}: Complete the homework problems below on a {\em separate} sheet of paper (and not all jammed up between the questions). Each solution should be accompanied with supporting work or an explanation why the solution is correct. Your work will be graded on correctness as well as the clarity of your explanations. 



\begin{questions}
\question[4] Let $A = \{1,2,3,\ldots,9\}$.  
\begin{parts}
  \part How many subsets of $A$ contain only even numbers?
  \begin{solution}
    For each of the 9 elements from $A$, we must decide yes or no on whether to include them in the subset.  However, for the odd numbers, we only have one choice: no.  So there are only 4 elements we have two choices for, so the answer is $2^4$.  (Note, if you wish to exclude the empty set - it does not contain odd numbers, but no evens either - then you could subtract 1).
  \end{solution}

  \part How many subsets of $A$ contain an even number of elements?
  \begin{solution}
    Count the number of subsets with each possible even cardinality:
    \[{9 \choose 0} + {9 \choose 2} + {9\choose 4} + {9 \choose 6} + {9 \choose 8} = 256\]
  \end{solution}

\end{parts}

\question[4] For how many three digit numbers (100 to 999) is the {\em sum of the digits} even? (For example, $343$ has an even sum of digits: $3+4+3 = 10$ which is even.)  Explain.

\begin{solution}
  There are multiple ways to do this.
  \begin{enumerate}
    \item An even sum can occur in 4 ways: EEE, EOO, OEO, and OOE.  There are $4 \cdot 5 \cdot 5$ ways to build numbers of the first two types (there are only 4 choices for a starting even number - it cannot be 0) and $5 \cdot 5 \cdot 5$ ways to build the second two types.  This gives a total of 450 numbers.
    \item To build a 3 digit number with an even sum, you can choose any of 9 digits for the first digit, any of 10 digits for the second digit.  Then the last digit must either be even (if the sum of the first two digits are even) or odd (if the sum of the first two digits are odd).  Luckily there are the same number of even last digits and odd last digits - 5.  So there are a total of $9 \cdot 10 \cdot 5 = 450$ numbers with an even sum of digits.
  \end{enumerate}

\end{solution}


\question[8] Gridtown USA, besides having excellent donut shoppes, is known for its precisely laid out grid of streets and avenues.  Streets run east-west, and avenues north-south, for the entire stretch of the town, never curving and never interrupted by parks or schools or the like.
\begin{parts}
\part Suppose you live on the corner of 1st and 1st and work on the corner of 12th and 12th.  How many blocks must you drive to get to work as quickly as possible?
\begin{solution}
  22 blocks (11 east, 11 north).
\end{solution}

\part How many different routes can you take to work, assuming you want to get there as quickly as possible?
\begin{solution}
  ${22 \choose 11}$ since you must choose 11 of the 22 blocks to travel east.
\end{solution}

\part Now suppose you want to stop and get a donut on the way to work, from your favorite donut shoppe on the corner of 8th st and 10th ave.  How many routes to work, via the donut shoppe, can you take (again, ensuring the shortest possible route)?
\begin{solution}
  The donut shoppe is 16 blocks away, 7 one way, 9 the other.  So to get from home to the donut shoppe, there are ${16 \choose 7}$ routes (or equivalently, ${16 \choose 9}$).  Then from the donut shopped to work, you need to travel 6 more blocks, 2 on way and 4 the other.  So there are ${6 \choose 2}$ (or ${6 \choose 4}$) routes from the donut shoppe to work.  
  
  For each of the ways to the donut shoppe, there are so many ways to work, so the multiplicative principle says the total number of ways from home to work via the donut shoppe is
  \[{16 \choose 7}{6 \choose 2}\]
\end{solution}

\part Disaster Strikes Gridtown: there is a pothole on 4th avenue between 5th and 6th street.  How many routes to work can you take avoiding that unsightly (and dangerous) stretch of road?
\begin{solution}
  Routes to work that hit the pothole: ${7 \choose 3}1{14 \choose 8}$.
  
  There for the number of routes to work which {\em avoid} the pothole are
  \[{22 \choose 11} - {7 \choose 3}{14 \choose 8}\]
\end{solution}

\part How many routes are there both avoiding the pothole and visiting the donut shoppe?
\begin{solution}
  First compute the number of routes to the donut shoppe avoiding the pothole:
  \[{16 \choose 7} - {7 \choose 3}{8 \choose 6}\]
  Then you still need to go to work from there.  Thus the answer is:
  \[({16 \choose 7} - {7 \choose 3}{8 \choose 6}){6 \choose 2}\]
\end{solution}

\end{parts}

\question[4] How many $9$-bit strings (that is, bit strings of length 9) are there which:
\begin{parts}
  \part Start with the sub-string 101?
  \begin{solution}
    $2^6 = 64$.  You have 2 choices for each of the remaining 6 bits.
  \end{solution}

  \part Have weight 5 (i.e., contain exactly five 1's) and start with the sub-string 101?
  \begin{solution}
    ${6 \choose 3} = 20$.  You need to choose 3 of the remaining 6 bits to be 1's.
  \end{solution}

\end{parts}

\end{questions}




\end{document}


