\documentclass[10pt]{exam}

\usepackage{amssymb, amsmath, amsthm, mathrsfs, multicol, graphicx} 
\usepackage{tikz}

\def\d{\displaystyle}
\def\?{\reflectbox{?}}
\def\b#1{\mathbf{#1}}
\def\f#1{\mathfrak #1}
\def\c#1{\mathcal #1}
\def\s#1{\mathscr #1}
\def\r#1{\mathrm{#1}}
\def\N{\mathbb N}
\def\Z{\mathbb Z}
\def\Q{\mathbb Q}
\def\R{\mathbb R}
\def\C{\mathbb C}
\def\F{\mathbb F}
\def\A{\mathbb A}
\def\X{\mathbb X}
\def\E{\mathbb E}
\def\O{\mathbb O}
\def\pow{\mathscr P}
\def\inv{^{-1}}
\def\nrml{\triangleleft}
\def\st{:}
\def\~{\widetilde}
\def\rem{\mathcal R}
\def\iff{\leftrightarrow}
\def\Iff{\Leftrightarrow}
\def\and{\wedge}
\def\And{\bigwedge}
\def\AAnd{\d\bigwedge\mkern-18mu\bigwedge}
\def\Vee{\bigvee}
\def\VVee{\d\Vee\mkern-18mu\Vee}
\def\imp{\rightarrow}
\def\Imp{\Rightarrow}
\def\Fi{\Leftarrow}

\def\={\equiv}
\def\var{\mbox{var}}
\def\mod{\mbox{Mod}}
\def\Th{\mbox{Th}}
\def\sat{\mbox{Sat}}
\def\con{\mbox{Con}}
\def\bmodels{=\joinrel\mathrel|}
\def\iffmodels{\bmodels\models}
\def\dbland{\bigwedge \!\!\bigwedge}
\def\dom{\mbox{dom}}
\def\rng{\mbox{range}}
\DeclareMathOperator{\wgt}{wgt}


\def\bar{\overline}

%\pointname{pts}
\pointsinmargin
\marginpointname{pts}
\addpoints
\pagestyle{head}
\printanswers

\firstpageheader{Math 228}{\bf Homework 2 Solutions}{ February 1, 2012}


\begin{document}

\begin{questions}
\question[6] For each of the following statements, write the converse, contrapositive, and the negation.
\begin{parts}
  \part $\exists x \forall y (x \ge 2y \imp x > y+1)$
  \begin{solution}
    Converse: $\exists x \forall y (x > y +1 \imp x \ge 2y)$
    Contrapositive: $\exists x \forall y (x \le y + 1 \imp x < 2y)$
    Negation: $\forall x \exists y (x \ge 2y \and x \le y+1)$
  \end{solution}

  \part For all integers $a$ and $b$, if $a$ is even and $b$ is odd, then $a + b$ is odd.
  \begin{solution}
    Converse: For all integers $a$ and $b$, if $a + b$ is odd, then $a$ is even and $b$ is odd.
    Contrapositive: For all integers $a$ and $b$, if $a+b$ is not odd, then $a$ is not even or $b$ is not odd.
    Negation: There are integers $a$ and $b$ such that $a$ is even and $b$ is odd, but $a+b$ is not odd.
  \end{solution}

\end{parts}

\question[5] Prove the statement: For all integers $n$, if $5n$ is odd, then $n$ is odd.  Clearly state the style of proof you are using.
\begin{solution}
We will prove the contrapositive: if $n$ is even, then $5n$ is even.
  \begin{proof}
    Let $n$ be an arbitrary integer, and suppose $n$ is even.  Then $n = 2k$ for some integer $k$.  Thus $5n = 5\cdot 2k = 10k = 2(5k)$.  Since $5k$ is an integer, we see that $5n$ must be even.  This completes the proof.
  \end{proof}

\end{solution}


\question[5] Prove the statement: For all integers $a$, $b$, and $c$, if $a^2 + b^2 = c^2$, then $a$ or $b$ is even.  Hint: try a proof by contradiction.
\begin{solution}
  \begin{proof}
    Suppose, contrary to stipulation, that there are integers $a$, $b$ and $c$ such that $a^2 + b^2 = c^2$ but $a$ and $b$ are both odd.  Then $a = 2k+1$ and $b = 2j + 1$ for some integers $k$ and $j$.  We then have
    \[a^2 + b^2 = (2k+1)^2 + (2j+1)^2 = 4k^2 + 4k + 1 + 4j^2 + 4j + 1 = 4(k^2 + j^2 + k + j) + 2\]
    So $c^2 = 4(k^2 + j^2 + k + j) + 2$.  This means that $c^2$ is even, which can only happen if $c$ is even.  But then $c^2$ must be a multiple of 4.  However, this is a contradiction because $4(k^2 + j^2 + k + j) + 2$ is not a multiple of 4.
  \end{proof}

\end{solution}


\question[4] Order matters with quantifiers! Sometimes. 
\begin{parts}
  \part Find a formula $\varphi$ such that $\forall x \exists y \,\varphi$ is true but $\exists y \forall x \,\varphi$ is false.
  \begin{solution}
    There are many examples, but for instance $\varphi$ could be $x < y$.  It is true that for every $x$ there is a $y$ greater than it.  However, there is not a $y$ greater than every $x$
  \end{solution}

  \part Explain why you cannot find a formula $\psi$ such that $\forall x \exists y \, \psi$ is false but $\exists y \forall x \, \psi$ is true.
  \begin{solution}
    Let's say $\exists y \forall x \psi$ is true.  That means you can pick a $y$ such that no matter what $x$ is picked, $\psi$ holds.  Now say you pick that same $y$ but keep it secret.  Now you pick any $x$ you like, and then reveal your previously selected $y$.  Since you picked the same $y$ as before, $\forall x \exists y\psi$ will also be true.   
  \end{solution}

\end{parts}



\end{questions}




\end{document}


