\documentclass[10pt]{exam}

\usepackage{amssymb, amsmath, amsthm, mathrsfs, multicol, graphicx} 
\usepackage{tikz}

\def\d{\displaystyle}
\def\?{\reflectbox{?}}
\def\b#1{\mathbf{#1}}
\def\f#1{\mathfrak #1}
\def\c#1{\mathcal #1}
\def\s#1{\mathscr #1}
\def\r#1{\mathrm{#1}}
\def\N{\mathbb N}
\def\Z{\mathbb Z}
\def\Q{\mathbb Q}
\def\R{\mathbb R}
\def\C{\mathbb C}
\def\F{\mathbb F}
\def\A{\mathbb A}
\def\X{\mathbb X}
\def\E{\mathbb E}
\def\O{\mathbb O}
\def\pow{\mathscr P}
\def\inv{^{-1}}
\def\nrml{\triangleleft}
\def\st{:}
\def\~{\widetilde}
\def\rem{\mathcal R}
\def\iff{\leftrightarrow}
\def\Iff{\Leftrightarrow}
\def\and{\wedge}
\def\And{\bigwedge}
\def\AAnd{\d\bigwedge\mkern-18mu\bigwedge}
\def\Vee{\bigvee}
\def\VVee{\d\Vee\mkern-18mu\Vee}
\def\imp{\rightarrow}
\def\Imp{\Rightarrow}
\def\Fi{\Leftarrow}

\def\={\equiv}
\def\var{\mbox{var}}
\def\mod{\mbox{Mod}}
\def\Th{\mbox{Th}}
\def\sat{\mbox{Sat}}
\def\con{\mbox{Con}}
\def\bmodels{=\joinrel\mathrel|}
\def\iffmodels{\bmodels\models}
\def\dbland{\bigwedge \!\!\bigwedge}
\def\dom{\mbox{dom}}
\def\rng{\mbox{range}}
\DeclareMathOperator{\wgt}{wgt}

\def\circleA{(-.5,0) circle (1)}
\def\circleAlabel{(-1.5,.6) node[above]{$A$}}
\def\circleB{(.5,0) circle (1)}
\def\circleBlabel{(1.5,.6) node[above]{$B$}}
\def\circleC{(0,-1) circle (1)}
\def\circleClabel{(.5,-2) node[right]{$C$}}
\def\twosetbox{(-2,-1.5) rectangle (2,1.5)}
\def\threesetbox{(-2,-2.5) rectangle (2,1.5)}


\def\bar{\overline}

%\pointname{pts}
\pointsinmargin
\marginpointname{pts}
\marginbonuspointname{pts-bns}
\addpoints
\pagestyle{head}
\printanswers

\firstpageheader{Math 228}{\bf Homework 4 Solutions}{February 22, 2012}


\begin{document}




\begin{questions}
\question[4] We have a formula $n$th term of an arithmetic sequence with first term $a$ and common difference $d$.  It is $a_n = a+d(n-1)$.  Find a closed formula for the \underline{sum} of the first $n$ terms of the sequence: $S_n = a_1 + a_2 + \cdots + a_n$.  Show your work.

\begin{solution}
  \begin{align*}
    S = &~ a + (a+d) + (a+2d) + \cdots + (a+d(n-1))\\
    \underline{+ S    = }& \underline{~(a+d(n-1)) + (a+d(n-2)) + \cdots + (a+d) + a}\\
    2S  = &(2a + d(n-1)) + (2a + d(n-1)) + \cdots + (2a + d(n-1))\\
    2S  = &(2a + d(n-1))n \mbox{~~~~ {\footnotesize b/c there are $n$ terms in the sum}}
  \end{align*}
Thus $S = \dfrac{(2a + d(n-1))n}{2}$
\end{solution}


\question[4] The $n$th term of a geometric sequence with first term $a$ and common ratio $r$ is $a_n = a\cdot r^{n-1}$.  Find a close formula for the \underline{sum} of the first $n$ terms of the sequence: $S_n = a_1 + a_2 + \cdots + a_n$.  Show your work.

\begin{solution}
  \begin{align*}
    S = &~ a + ar + ar^2 + ar^3 + \cdots + ar^{n-1} \\
    \underline{- rS = }& \underline{~~~~~~ ar + ar^2 + ar^3 + \cdots + ar^{n-1} + ar^n}\\
    S - rS = & ~ a - ar^n\\
    S(1-r) = & ~ a(1 - r^n)
  \end{align*}
So $S = \dfrac{a(1-r^n)}{1-r}$
\end{solution}


\question[6] In their down time, ghost pirates enjoy stacking cannonballs in triangular based pyramids (aka, tetrahedrons), like those pictured here:

\centerline{\includegraphics[height=.3in]{cannonballs.png}}

Note, in the picture on the right, there are some cannonballs (actually just one) you cannot see.  

The pirates wonder how many cannonballs would be required to build a pyramid 15 layers high (thus breaking the world cannonball stacking record).  Can you help?

\begin{parts}
\part Let $P(n)$ denote the number of cannonballs needed to create a pyramid $n$ layers high.  So $P(1) = 1$, $P(2) = 4$, and so on.  Calculate $P(3)$, $P(4)$ and $P(5)$.
\begin{solution}
  To get the next larger pyramid, we add a triangle of cannonballs to the previous pyramid.  Thus to get $P(n)$, we add $P(n-1)$ to the $n$th triangular number:
  $P(3) = 4 + 6 = 10$, $P(4) = 10 + 10 = 20$, $P(5) = 20 + 15 = 35$.
\end{solution}


\part Use polynomial fitting to find a closed formula for $P(n)$.  Show your work.
\begin{solution}
  The first differences are $3, 6, 10, 15, \ldots$.  The second differences are $3, 4, 5, 6, \ldots$.  The third differences are $1,1,1,\ldots$.  Since third differences are constant, we know the closed formula for $P(n)$ will be a degree 3 polynomial.  So $P(n) = an^3 + b n^2 + cn + d$.  Note that $P(0) = 0$, so $d = 0$.  To solve for $a$, $b$, and $c$, we solve the system of equations:
  \begin{align*}
    1 & = a + b + c \\
    4 & = 8a+ 4b + 2c \\
    10 & = 27a + 9b + 3c
  \end{align*}
  Doing so gives $a = \frac{1}{6}$, $b = \frac{1}{2}$ and $c = \frac{1}{3}$ so 
  \[P(n) = \frac{1}{6}n^3 + \frac{1}{2} n^2 + \frac{1}{3} n\]
\end{solution}


\part Answer the pirate's question: how many cannonballs do they need to make a pyramid 15 layers high?

\begin{solution}
  \[P(15) = \frac{1}{6}15^3 + \frac{1}{2} 15^2 + \frac{1}{3} 15 = 680\]
\end{solution}

\end{parts}

\question[6] If you have enough toothpicks, you can make a large triangular grid.  Below, are the triangular grids of size 1 and of size 2.  The size 1 grid requires 3 toothpicks, the size 2 grid requires 9 toothpicks.

\centerline{\includegraphics[height=.3in]{triangles.png}}
 
\begin{parts}
  \part Let $t_n$ be the number of toothpicks required to make a size $n$ triangular grid.  Write out the first 5 terms of the sequence $t_1, t_2, \ldots$.
  \begin{solution}
    $ 3, 9, 18, 30, 45, \ldots$
  \end{solution}

  \part Find a recursive definition for the sequence.  Explain why you are correct.
  \begin{solution}
    $t_n = t_{n-1} + 3n$.  This works because to get the next larger triangular grid, we must add a row of $n$ triangles, each requiring 3 toothpicks.
  \end{solution}

  \part Find a closed formula for the sequence.  Explain why you are correct.
  \begin{solution}
    You could do this using polynomial fitting, but there is an easier way - notice that each term in the sequence is a multiple of 3.  Dividing each term by 3 gives the sequence $1,3, 6, 10, 15,\ldots$ - the triangular numbers. This makes sense because we are forming triangles out of 3-toothpick collections.   So a closed formula is therefore
    $t_n = 3\frac{n(n+1)}{2}$
  \end{solution}

\end{parts}

\bonusquestion[4] Bonus: How many triangles (of all sizes and orientations) are contained in a size $n$ triangular grid?  For example, there is one triangle in a size 1 grid, and 5 triangles in a size 2 grid.
\begin{solution}
  To count the triangles, you need to separate the upright triangles from the upside down triangles.  The upright triangles are counted as the sum of triangular numbers, so should be described by a degree 3 polynomial.  The formula for upside down triangles is harder - the formula will be slightly different depending on whether $n$ is even or odd.
\end{solution}

\end{questions}




\end{document}


