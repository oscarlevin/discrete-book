\documentclass[11pt]{exam}

\usepackage{amssymb, amsmath, amsthm, mathrsfs, multicol, graphicx} 
\usepackage{tikz}

\def\d{\displaystyle}
\def\?{\reflectbox{?}}
\def\b#1{\mathbf{#1}}
\def\f#1{\mathfrak #1}
\def\c#1{\mathcal #1}
\def\s#1{\mathscr #1}
\def\r#1{\mathrm{#1}}
\def\N{\mathbb N}
\def\Z{\mathbb Z}
\def\Q{\mathbb Q}
\def\R{\mathbb R}
\def\C{\mathbb C}
\def\F{\mathbb F}
\def\A{\mathbb A}
\def\X{\mathbb X}
\def\E{\mathbb E}
\def\O{\mathbb O}
\def\pow{\mathscr P}
\def\inv{^{-1}}
\def\nrml{\triangleleft}
\def\st{:}
\def\~{\widetilde}
\def\rem{\mathcal R}
\def\iff{\leftrightarrow}
\def\Iff{\Leftrightarrow}
\def\and{\wedge}
\def\And{\bigwedge}
\def\AAnd{\d\bigwedge\mkern-18mu\bigwedge}
\def\Vee{\bigvee}
\def\VVee{\d\Vee\mkern-18mu\Vee}
\def\imp{\rightarrow}
\def\Imp{\Rightarrow}
\def\Fi{\Leftarrow}

\def\={\equiv}
\def\var{\mbox{var}}
\def\mod{\mbox{Mod}}
\def\Th{\mbox{Th}}
\def\sat{\mbox{Sat}}
\def\con{\mbox{Con}}
\def\bmodels{=\joinrel\mathrel|}
\def\iffmodels{\bmodels\models}
\def\dbland{\bigwedge \!\!\bigwedge}
\def\dom{\mbox{dom}}
\def\rng{\mbox{range}}
\DeclareMathOperator{\wgt}{wgt}

\def\circleA{(-.5,0) circle (1)}
\def\circleAlabel{(-1.5,.6) node[above]{$A$}}
\def\circleB{(.5,0) circle (1)}
\def\circleBlabel{(1.5,.6) node[above]{$B$}}
\def\circleC{(0,-1) circle (1)}
\def\circleClabel{(.5,-2) node[right]{$C$}}
\def\twosetbox{(-2,-1.5) rectangle (2,1.5)}
\def\threesetbox{(-2,-2.5) rectangle (2,1.5)}


\def\bar{\overline}

%\pointname{pts}
\pointsinmargin
\marginpointname{pts}
\marginbonuspointname{pts-bns}
\addpoints
\pagestyle{head}
%\printanswers

\firstpageheader{Math 228}{\bf Homework 8}{Due: Wed April 4, 2012}


\begin{document}
\noindent \textbf{Instructions}: Complete the homework problems below on a {\em separate} sheet of paper (and not all jammed up between the questions). Each solution should be accompanied with supporting work or an explanation why the solution is correct. Your work will be graded on correctness as well as the clarity of your explanations. 



\begin{questions}
\question[5] How many triangles are there with vertices from the points shown below?  Note, we are not allowing degenerate triangles - ones with all three vertices on the same line.  Explain why your answer is correct. (HINT: you need at exactly two points on either the $x$- or $y$-axis, but don't over-count the right triangles.)

\begin{center}
  \begin{tikzpicture}
    \foreach \i in {0,...,6} {
      \fill (\i,0) circle (2pt);
    }
    \foreach \i in {1,...,4} {
      \fill (0,\i) circle (2pt);
    }
  \end{tikzpicture}

  
\end{center}

\question[6] Suppose you own $a$ fezzes and $b$ bow ties.  Of course, $a$ and $b$ are both greater than 1.
\begin{parts}
  \part How many combinations of fez and bow tie can you make?  You can wear only one fez and one bow tie at a time.  Explain.
  \part Explain why the answer is {\em also} ${a+b \choose 2} - {a \choose 2} - {b \choose 2}$.  (If this is what you claimed the answer was in part (a), try it again.)
  \part Use your answers to parts (a) and (b) to give a combinatorial proof of the identity
  \[{a+b \choose 2} - {a \choose 2} - {b \choose 2} = ab\]
\end{parts}


\question Consider the identity:
\[k{n\choose k} = n{n-1 \choose k-1}\]
\begin{parts}
  \part[2] Is this true?  Try it for a few values of $n$ and $k$.
  \part[3] Use the formula for ${n \choose k}$ to give an algebraic proof of the identity.
  \part[4] Give a combinatorial proof of the identity. Hint: How many ways can you select a chaired committee of $k$ people from a group of $n$ people?  
\end{parts}




\end{questions}




\end{document}


