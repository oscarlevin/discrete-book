\documentclass[10pt]{exam}

\usepackage{amssymb, amsmath, amsthm, mathrsfs, multicol, graphicx} 
\usepackage{tikz}

\def\d{\displaystyle}
\def\?{\reflectbox{?}}
\def\b#1{\mathbf{#1}}
\def\f#1{\mathfrak #1}
\def\c#1{\mathcal #1}
\def\s#1{\mathscr #1}
\def\r#1{\mathrm{#1}}
\def\N{\mathbb N}
\def\Z{\mathbb Z}
\def\Q{\mathbb Q}
\def\R{\mathbb R}
\def\C{\mathbb C}
\def\F{\mathbb F}
\def\A{\mathbb A}
\def\X{\mathbb X}
\def\E{\mathbb E}
\def\O{\mathbb O}
\def\pow{\mathscr P}
\def\inv{^{-1}}
\def\nrml{\triangleleft}
\def\st{:}
\def\~{\widetilde}
\def\rem{\mathcal R}
\def\iff{\leftrightarrow}
\def\Iff{\Leftrightarrow}
\def\and{\wedge}
\def\And{\bigwedge}
\def\AAnd{\d\bigwedge\mkern-18mu\bigwedge}
\def\Vee{\bigvee}
\def\VVee{\d\Vee\mkern-18mu\Vee}
\def\imp{\rightarrow}
\def\Imp{\Rightarrow}
\def\Fi{\Leftarrow}

\def\={\equiv}
\def\var{\mbox{var}}
\def\mod{\mbox{Mod}}
\def\Th{\mbox{Th}}
\def\sat{\mbox{Sat}}
\def\con{\mbox{Con}}
\def\bmodels{=\joinrel\mathrel|}
\def\iffmodels{\bmodels\models}
\def\dbland{\bigwedge \!\!\bigwedge}
\def\dom{\mbox{dom}}
\def\rng{\mbox{range}}
\DeclareMathOperator{\wgt}{wgt}


\def\bar{\overline}

%\pointname{pts}
\pointsinmargin
\marginpointname{pts}
\addpoints
\pagestyle{head}
\printanswers

\firstpageheader{Math 228}{\bf Homework 1 Solutions}{Jan 25, 2012}


\begin{document}


\begin{questions}
\question[6] 
\begin{parts}
  \part Make a truth table for the statement $P \imp (\neg Q \vee R)$.
  \begin{solution}
    \begin{center}
  \begin{tabular}{c|c|c||c|c}
    $P$ & $Q$ & $R$ & $\neg Q \vee R$ & $P \imp (\neg Q \vee R)$\\ \hline
    T & T & T & T & T\\
    T & T & F & F & F \\
    T & F & T & T & T \\
    T & F & F & T & T \\
    F & T & T & T & T \\
    F & T & F & F & T \\
    F & F & T & T & T \\
    F & F & F & T & T
  \end{tabular}
\end{center}
  \end{solution}

  \part If Tommy {\bf lies} when he says, ``if I ate pizza, then either I didn't eat cucumber sandwiches or I did eat raisins,'' what can you conclude about what Tommy ate?  Explain.
  \begin{solution}
    The statement made is the same as the one we made a truth table for above.  If the statement is a lie, then we are in the case(s) in which the statement is false.  This turns out to be only the second case, so we see that $P$ and $Q$ are true and $R$ is false.  Therefore Tommy ate pizza a cucumber sandwiches, but not raisins.
  \end{solution}

\end{parts}
  
\question[6] Can you distribute conjunctions over disjunctions?  Disjunctions over conjunctions?  Let's find out.  Remember, two statements are logically equivalent if they are true in exactly the same cases.
\begin{parts}
  \part Are the statements $P \vee (Q \and R)$ and $(P \vee Q) \and (P \vee R)$ logically equivalent?  
  \begin{solution}
    Yes they are.  We prove this by showing that their truth tables are identical:
    \begin{center}
        \begin{tabular}{c|c|c||c||c}
    $P$ & $Q$ & $R$ & $P \vee (Q \and R)$ & $(P \vee Q) \and (P \vee R)$\\ \hline
    T & T & T & T & T\\
    T & T & F & T & T \\
    T & F & T & T & T \\
    T & F & F & T & T \\
    F & T & T & T & T \\
    F & T & F & F & F \\
    F & F & T & F & F \\
    F & F & F & F & F
  \end{tabular}
    \end{center}
  \end{solution}
%\newpage
  \part Are the statements $P \and (Q \vee R)$ and $(P \and Q) \vee (P \and R)$ logically equivalent?
  \begin{solution}
    It works again.  Here are the two truth tables which prove it:
        \begin{center}
        \begin{tabular}{c|c|c||c||c}
    $P$ & $Q$ & $R$ & $P \and (Q \vee R)$ & $(P \and Q) \vee (P \and R)$\\ \hline
    T & T & T & T & T\\
    T & T & F & T & T \\
    T & F & T & T & T \\
    T & F & F & F & F \\
    F & T & T & F & F \\
    F & T & F & F & F \\
    F & F & T & F & F \\
    F & F & F & F & F
  \end{tabular}
    \end{center}
  \end{solution}

\end{parts}

\question[4] Use De Morgan's Laws, and any other logical equivalence facts you know to simplify the following statements.  Show all your steps, justifying each.  Your final statements should have negations only appear directly next to the propositional variables ($P$, $Q$, etc.), and no double negations.  
\begin{parts}
  \part $\neg((\neg P \and Q) \vee \neg(R \vee \neg S))$.
  \begin{solution}
    $\neg((\neg P \and Q) \vee \neg(R \vee \neg S))$\\
    $\neg(\neg P \and Q) \and \neg\neg(R \vee \neg S)$ by De Morgan's law.\\
    $\neg(\neg P \and Q) \and (R \vee \neg S)$ by double negation.\\
    $(\neg\neg P \vee \neg Q) \and (R \vee \neg S)$ by De Morgan's law.\\
    $(P \vee \neg Q) \and (R \vee \neg S)$ by double negation.
  \end{solution}

  \part $\neg((\neg P \imp \neg Q) \and (\neg Q \imp R))$ (careful with the implications).
  \begin{solution}
    We will need to convert the implications to disjunctions so we can apply De Morgan's law:
    
    $\neg((\neg P \imp \neg Q) \and (\neg Q \imp R))$\\
    $\neg((\neg \neg P \vee \neg Q) \and (\neg\neg Q \vee R))$ by implication/disjunction equivalence.\\
    $\neg((P \vee \neg Q) \and (Q \vee R))$ by double negation.\\
    $\neg(P \vee \neg Q) \vee \neg (Q \vee R)$ by De Morgan's law.\\
    $(\neg P \and \neg \neg Q) \vee (\neg Q \and \neg R)$ by De Morgan's law.\\
    $(\neg P \and Q) \vee (\neg Q \and \neg R)$ by double negation.
  \end{solution}

\end{parts}

\question[4] Find a statement which has the following truth table.  You final answer should contain only one instance of each of the variables $P$, $Q$ and $R$.

\begin{center}
  \begin{tabular}{c|c|c||c}
    $P$ & $Q$ & $R$ & ???\\ \hline
    T & T & T & F \\
    T & T & F & F \\
    T & F & T & T \\
    T & F & F & T \\
    F & T & T & T \\
    F & T & F & F \\
    F & F & T & T \\
    F & F & F & T
  \end{tabular}
\end{center}
\begin{solution}
  One possible statement is this: $\neg Q \vee (\neg P \and R)$.  Of course, there are many statements logically equivalent to this such as $Q \imp (\neg P \and R)$ or $Q \imp \neg(R \imp P)$.  
  
  To find the solution, you have lots of choices.  Notice that whenever $Q$ is false, the statement is true.  But also the statement is true in one case when $Q$ is true - when $P$ is false and $R$ is true.  The other case which has $P$ false and $R$ true (and $Q$ false) also results in a true statement, so we to get truth we must have either $Q$ false (so $\neg Q$ true) or $\neg P$ and $R$ both true (so $P$ is false and $R$ is true).
  
  An alternate method of arriving at the solution is to notice that the statement is false more than it is true, so first find a the negation of the statement.  This leads to $\neg ((P \and Q) \vee (\neg P \and Q \and \neg R)$.  If you simplify this, you get the same answer as above.
\end{solution}


\end{questions}




\end{document}


