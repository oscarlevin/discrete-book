\documentclass[11pt]{exam}

\usepackage{amsmath, amssymb, multicol}
\usepackage{graphicx}
\usepackage{textcomp}
\usepackage{chessboard}

\def\d{\displaystyle}
\def\b{\mathbf}
\def\R{\mathbf{R}}
\def\Z{\mathbf{Z}}
\def\st{~:~}
\def\bar{\overline}
\def\inv{^{-1}}
\def\imp{\rightarrow}
\def\and{\wedge}


%\pointname{pts}
\pointsinmargin
\marginpointname{pts}
\addpoints
\pagestyle{head}
%\printanswers

\firstpageheader{Math 228}{\bf Practice Problems 1: Logic\\ Hints and Answers}{Spring 2012}


\begin{document}

\begin{questions}
\question \begin{tabular}{c|c|c}
             $P$ & $Q$ & $(P \vee Q) \imp (P \and Q)$\\ \hline
             T & T & T \\
             T & F & F \\
             F & T & F \\
             F & F & T
          \end{tabular}
 

\question \begin{tabular}{c|c|c}
             $P$ & $Q$ & $\neg P \and (Q \imp P)$\\ \hline
             T & T & F \\
             T & F & F \\
             F & T & F \\
             F & F & T
          \end{tabular}
If the statement is true, then both $P$ and $Q$ are false.

\question Hint: Like above, only now you will need 8 rows instead of just 4.

\question Make a truth table for each and compare.  The statements are logically equivalent.

\question Again, make two truth tables.  The statements are logically equivalent.

\question 
\begin{parts}
 \part $P$: it's your birthday; $Q$: there will be cake.  $(P \vee Q) \imp Q$
 \part Hint: you should get three T's and one F.
 \part Only that there will be cake.
 \part It's your birthday!
 \part The cake is a lie.
\end{parts}

\question 
\begin{parts}
\part $P \and Q$
\part $P \imp \neg Q$
\part Jack passed math or Jill passed math (or both).
\part If Jack and Jill did not both pass math, then Jill did.
\part 
\begin{subparts}
 \subpart Nothing else. 
\subpart  Jack did not pass math either.
\end{subparts}
\end{parts}


\question   
\begin{parts}
	\part Three statements: $P \vee S$, $S \imp Q$, $(P \vee Q) \imp R$.  You could also connect the first two with a $\and$.
	\part He cannot be lying about all three sentences, so he is telling the truth.
	\part No matter what, Geoff wants ricotta.  If he doesn't have quail, then he must have pepperoni but not sausage.
\end{parts}


\question Consider the statement ``If Oscar eats Chinese food, then he drinks milk.''
\begin{parts}
 \part If Oscar drinks milk, then he eats Chinese food.
 \part If Oscar does not drink milk, then he does not eat Chinese food.
 \part Yes.  The original statement would be false too.
 \part Nothing. The converse need not be true.
 \part He does not eat Chinese food. The contrapositive would be true.
\end{parts}


\question 
\begin{parts}
  \part $P \and Q$ 
  \part $(P \vee Q) \vee (Q \and \neg R)$
  \part F.  Or $(P \and Q) \and (R \and \neg R)$ 
  \part Either Sam is a woman and Chris is a man, or Chris is a woman.
\end{parts}


\question
 \begin{parts}
  \part $\neg \exists x (E(x) \and O(x))$
\part $\forall x (E(x) \imp O(x+1))$
\part $\exists x(P(x) \and E(x))$ (where $P(x)$ means ``$x$ is prime'')
\part $\forall x \forall y \exists z(x < z < y \vee y < z < x)$
\part $\forall x \neg \exists y (x < y < x+1)$
 \end{parts}

\question 
\begin{parts}
 \part Any even number plus 2 is an even number.
\part For any $x$ there is a $y$ such that $\sin(x) = y$.  In other words, every number $x$ is in the domain of sine. 
\part For every $y$ there is an $x$ such that $\sin(x) = y$.  In other words, every number $y$ is in the range of sine (which is false).
\part For any numbers, if the cubes of two numbers are equal, then the numbers are equal.
\end{parts}

\question 
\begin{parts}
  \part $\forall x \exists y (O(x) \and \neg E(y))$
  \part $\exists x \forall y (x \ge y \vee \forall z (x \ge z \and y \ge z))$
  \part There is a number $n$ for which every other number is strictly greater than $n$.
  \part There is a number $n$ which is not between any other two numbers.
\end{parts}


\question 
\begin{parts}
 \part For all integers $a$ and $b$, if $a$ or $b$ are not even, then $a+b$ is not even.
 \part For all integers $a$ and $b$, if $a$ and $b$ are even, then $a+b$ is even.
 \part There are numbers $a$ and $b$ such that $a+b$ is even but $a$ and $b$ are not both even.
 \part False.  For example, $a = 3$ and $b = 5$.  $a+b = 8$, but neither $a$ nor $b$ are even.
 \part False, since it is equivalent to the original statement.
 \part True.  Let $a$ and $b$ be integers.  Assume both are even.  Then $a = 2k$ and $b = 2j$ for some integers $k$ and $j$.  But then $a+b = 2k + 2j = 2(k+j)$ which is even.
 \part True, since the statement is false.
\end{parts}


\question Suppose $\sqrt{3}$ were rational.  Then $\sqrt{3} = \frac{a}{b}$ for some integers $a$ and $b \ne 0$.  Without loss of generality, assume $\frac{a}{b}$ is reduced.  Now
\[3 = \frac{a^2}{b^2}\]
\[b^2 3 = a^2\]
So $a^2$ is a multiple of 3.  This can only happen if $a$ is a multiple of 3, so $a = 3k$ for some integer $k$.  Then we have
\[b^2 3 = 9k^2\]
\[b^2 = 3k^2\]
So $b^2$ is a multiple of 3, making $b$ a multiple of 3 as well.  But this contradicts our assumption that $\frac{a}{b}$ is in lowest terms.
 
\end{questions}




\end{document}


