\documentclass[11pt]{exam}

\usepackage{amssymb, amsmath, amsthm, mathrsfs, multicol, graphicx} 
\usepackage{tikz}

\def\d{\displaystyle}
\def\?{\reflectbox{?}}
\def\b#1{\mathbf{#1}}
\def\f#1{\mathfrak #1}
\def\c#1{\mathcal #1}
\def\s#1{\mathscr #1}
\def\r#1{\mathrm{#1}}
\def\N{\mathbb N}
\def\Z{\mathbb Z}
\def\Q{\mathbb Q}
\def\R{\mathbb R}
\def\C{\mathbb C}
\def\F{\mathbb F}
\def\A{\mathbb A}
\def\X{\mathbb X}
\def\E{\mathbb E}
\def\O{\mathbb O}
\def\pow{\mathscr P}
\def\inv{^{-1}}
\def\nrml{\triangleleft}
\def\st{:}
\def\~{\widetilde}
\def\rem{\mathcal R}
\def\iff{\leftrightarrow}
\def\Iff{\Leftrightarrow}
\def\and{\wedge}
\def\And{\bigwedge}
\def\AAnd{\d\bigwedge\mkern-18mu\bigwedge}
\def\Vee{\bigvee}
\def\VVee{\d\Vee\mkern-18mu\Vee}
\def\imp{\rightarrow}
\def\Imp{\Rightarrow}
\def\Fi{\Leftarrow}

\def\={\equiv}
\def\var{\mbox{var}}
\def\mod{\mbox{Mod}}
\def\Th{\mbox{Th}}
\def\sat{\mbox{Sat}}
\def\con{\mbox{Con}}
\def\bmodels{=\joinrel\mathrel|}
\def\iffmodels{\bmodels\models}
\def\dbland{\bigwedge \!\!\bigwedge}
\def\dom{\mbox{dom}}
\def\rng{\mbox{range}}
\DeclareMathOperator{\wgt}{wgt}

\def\circleA{(-.5,0) circle (1)}
\def\circleAlabel{(-1.5,.6) node[above]{$A$}}
\def\circleB{(.5,0) circle (1)}
\def\circleBlabel{(1.5,.6) node[above]{$B$}}
\def\circleC{(0,-1) circle (1)}
\def\circleClabel{(.5,-2) node[right]{$C$}}
\def\twosetbox{(-2,-1.5) rectangle (2,1.5)}
\def\threesetbox{(-2,-2.5) rectangle (2,1.5)}


\def\bar{\overline}

%\pointname{pts}
\pointsinmargin
\marginpointname{pts}
\marginbonuspointname{pts-bns}
\addpoints
\pagestyle{head}
%\printanswers

\firstpageheader{Math 228}{\bf Homework 5}{Due: Wed February 29, 2012}


\begin{document}
\noindent \textbf{Instructions}: Complete the homework problems below on a {\em separate} sheet of paper (and not all jammed up between the questions). Each solution should be accompanied with supporting work or an explanation why the solution is correct. Your work will be graded on correctness as well as the clarity of your explanations. 



\begin{questions}
\question[4] Solve the recurrence relation $a_n = a_{n-1} + 3$ using telescoping or iteration.  Show your work.
\begin{solution}
  Both telescoping and iteration work.  For telescoping:
  \begin{align*}
    a_1 - a_0 & = 3\\
    a_2 - a_1 & = 3 \\
    a_3 - a_2 & = 3 \\
    \vdots & ~~ \vdots \\
    \underline{ + a_n - a_{n-1}} & \underline{ = 3}\\
    a_n - a_0 & = 3n
  \end{align*}
  Thus the solution is $a_n = 3n + a_0$.
\end{solution}


\question[6] Let $a_n$ be the number of  $1 \times n$ tile designs can you make using $1 \times 1$ tiles available in 4 colors and $1 \times 2$ tiles available in 5 colors.
\begin{parts}
  \part First, find a recurrence relation to describe the problem. 
  \begin{solution}
    $a_n = 4a_{n-1} + 5a_{n-2}$
  \end{solution}

  \part Write out the first 6 terms of the sequence $a_1, a_2, \ldots$.
  \begin{solution}
    4, 21, 104, 521, 2604, 13021
  \end{solution}

  \part Solve the recurrence relation.  That is, find a closed formula for $a_n$.
  \begin{solution}
    The characteristic equation is $x^2 - 4x - 5 = 0$ so the characteristic roots are $x = 5$ and $x = -1$.  Therefore the general solution is 
    \[a_n = a 5^n + b (-1)^n\]
    We solve for $a$ and $b$ using the fact that $a_1 = 4$ and $a_2 = 21$.  We get $a = \frac{5}{6}$ and $b = \frac{1}{6}$.  Therefore the solution is
    \[a_n = \frac{5}{6} 5^n + \frac{1}{6}(-1)^n\]
  \end{solution}

\end{parts}


\question[5] Consider the recurrence relation $a_n = 4a_{n-1} - 4a_{n-2}$.
\begin{parts}
  \part Find the general solution to the recurrence relation (beware the repeated root).
  \begin{solution}
    The characteristic polynomial is $x^2 - 4x + 4$ which factors as $(x -2)^2$, so the only characteristic root is $x = 2$.  Thus the general solution is
    \[a_n = a2^n + bn2^n\]
  \end{solution}

  \part Find the solution when $a_0 = 1$ and $a_1 = 2$.
  \begin{solution}
    Since $1 = a2^0 + b\cdot 0 \cdot 2^0$ have have $a = 1$.  Then $2 = 2^1 + b 2^1$ so $b = 0$.  We have the solution 
    \[a_n = 2^n\]
  \end{solution}

  \part Find the solution when $a_0 = 1$ and $a_1 = 8$.
  \begin{solution}
    Again, we have $a = 1$.  Now when we plug in $n = 1$ we bet $8 = 2 + 2b$ so $b = 3$.  The solution:
    \[a_n = 2^n + 3n2^n\]
  \end{solution}

\end{parts}

\question[5] Prove, by mathematical induction, that $F_0 + F_1 + F_2 + \cdots + F_{n} = F_{n+2} - 1$, where $F_n$ is the $n$th Fibonacci number ($F_0 = 0$, $F_1 = 1$ and $F_n = F_{n-1} + F_{n-2}$).
\begin{solution}
  \begin{proof}
    Let $P(n)$ be the statement $F_0 + F_1 + F_2 + \cdots + F_n = F_{n+2} - 1$.  We will prove that $P(n)$ is true for all $n \ge 0$.  
    
    Base case: $P(0)$ states that $F_0 = F_2 - 1$, which is true because $F_0 = 0$ and $F_2 = 1$.
    
    Inductive case:  Assume $P(k)$ is true for an arbitrary fixed $k \ge 0$.  That is, \[F_0 + F_1 + F_2 + \cdots + F_k = F_{k+2} - 1\]
    We must prove that $P(k+1)$ is true as well (i.e. that $F_0 + F_1 + \cdots +F_{k+1} = F_{k+3} - 1$).  Start with the left hand side:
    \begin{align*}
      F_0 + F_1 + F_2 + \cdots + F_k + F_{k+1} & = F_{k+2} - 1 + F_{k+1} & \mbox{ by the inductive hypothesis}\\
      & = F_{k+3} - 1 & \mbox{ by the definition of the Fibonacci numbers}
    \end{align*}
    Thus $P(k+1)$ is true.
    
    Therefore by the principle of mathematical induction, $P(n)$ is true for all $n \ge 0$.
  \end{proof}

\end{solution}

\end{questions}




\end{document}


