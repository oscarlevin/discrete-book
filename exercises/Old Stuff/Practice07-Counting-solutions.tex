\documentclass[10pt]{exam}

\usepackage{amsmath, amssymb, amsthm, multicol}
\usepackage{graphicx}
\usepackage{textcomp}
\usepackage{tikz}

\def\d{\displaystyle}
\def\b{\mathbf}
\def\R{\mathbb{R}}
\def\Z{\mathbb{Z}}
\def\N{\mathbb{N}}
\def\pow{\mathcal{P}}
\def\st{~:~}
\def\bar{\overline}
\def\inv{^{-1}}
\def\imp{\rightarrow}
\def\and{\wedge}


%\pointname{pts}
\pointsinmargin
\marginpointname{pts}
\addpoints
\pagestyle{head}
%\printanswers

\firstpageheader{Math 228}{\bf Practice Problems 7: Counting\\Solutions}{Spring 2012}


\begin{document}
 

\begin{questions}
\question 255 %Your wardrobe consists of 5 shirts, 3 pairs of pants, and 17 bow ties.  How many different outfits can you make?

\question 8 %For an interview, you must wear a tie.  You own 3 regular (boring) ties and 5 (cool) bow ties.  How many choices do you have for your neck-wear? 

\question 15 %You realize that the interview is for clown-college, so you should probably wear both a regular tie and a bow tie.  How many choices do you have now?

\question %You realize that it would also be okay to wear more than two ties.
\begin{parts}
 \part $2^8 = 256$.  You have two choices for each tie - wear it or don't. %You must select some of your ties to wear - everything is okay, from no ties up to all ties.  How many choices do you have?
 \part You have 7 choices for regular ties (the 8 choices less the ``no regular tie'' option) and 31 choices for bow ties (32 total minus the ``no bow tie'' option).  Thus total you have $7 \cdot 31 = 217$.  %If you want to wear at least one regular tie and one bow tie, but are willing to wear up to all your ties, how many choices do you have for which ties to wear?
 \part ${2\choose 3}{5\choose 3} = 30$  %How many choices do you have if you wear exactly 2 of the 3 regular ties and 3 of the 5 bow ties?
 \part $5! = 120$  %Once you have selected 2 regular and 3 bow ties, in how many orders could you put the ties on, assuming you must have one of the three bow ties on top?
\end{parts}

\question %Your Blu-ray collection consists of 9 comedies and 7 horror movies. Give an example of a question for which the answer is:
\begin{parts}
 \part 16 is the number of choices you have if you want to watch one movie, either a comedy or horror flick.
 \part 63 is the number of choices you have if you will watch two movies, first a comedy and then a horror.
\end{parts}

\question %Consider all 5 letter ``words'' made from the letters $a$ through $h$.
\begin{parts}
 \part $8^5$, since you select from 8 letters 5 times.  %How many of these words are there total?
 \part $P(8,5) = 8\cdot 7\cdot 6\cdot 5\cdot 4$.  After selecting a letter, you have fewer letters to select for the next one.  %How many of these words contain no repeated letters?
 \part 64 - you need to select the 4th and 5th letters. %How many of these words (repetitions allowed) start with the sub-word ``aha''?
 \part $64 + 64 - 0 = 128$.  There are 64 words which start with ``aha'' and another 64 words that end with ``bah.''  Perhaps we over counted the words that both start with ``aha'' and end with ``bah'' but since the words are only 5 letters long, there are no such words.  %How many of these words (repetitions allowed) either start with ``aha'' or end with ``bah'' or both?
 \part $P(8,5) - 3\cdot P(5,2) = 6660$ - all the words minus the bad ones.  The taboo word can be in any of three positions (starting with letter 1, 2, or 3) and for each position we must choose the other two letters (from the remaining 5 letters) %How many of the words containing no repeats also do not contain the sub-word ``bad'' (in consecutive letters)?  
\end{parts}

\question ${10 \choose 6} + {10\choose 7} + {10\choose 8} + {10 \choose 9} + {10\choose 10} = 386$ %How many 10-bit strings contain 6 or more 1's?

\question Use the binomial theorem.  ${14\choose 9} + {15 \choose 6}2^9$.  %What is the coefficient of $x^9$ in the expansion of $(x+1)^{14} + x^3(x+2)^{15}$?

\question %Let $S = \{1, 2, 3, 4, 5, 6\}$
\begin{parts}
  \part $2^6 = 64$  %How many subsets are there total? 
  \part $2^3 = 8$.  We need to select yes/no for each of the remaining three elements.  %How many subsets contain $\{2,3,5\}$ as a subset?
  \part $2^3 = 8$.  We need to decide yes/no for the three non-prime elements.  %How many subsets of $S$ contain no prime numbers?
  \part $2^6 - 2^3 = 56$.  There are 8 subsets which do not contain any odd numbers. %How many subsets contain at least one odd number? 
  \part 9.  We need to select one odd (3 choices) and one even (3 choices).  %How many doubletons (i.e., subsets of two elements) contain exactly one even number?
\end{parts}
   
\question %How many shortest lattice paths start at (3,3) and
\begin{parts}
  \part ${14 \choose 7}$ %end at (10,10)?
  \part ${6 \choose 2}{8\choose 5}$ %end at (10,10) and pass through (5,7)?
  \part ${14 \choose 7} - {6\choose 2}{8 \choose 5}$ %end at (10,10) and avoid (5,7)?
\end{parts}   
   
\question %A pizza parlor offers 10 toppings.
\begin{parts}
 \part ${10 \choose 3}$ %How many 3-topping pizzas could they put on their menu?  Assume double toppings are not allowed.
 \part $2^{10}$ %How many total pizzas are possible, with between zero and ten toppings (but not double toppings) allowed?
 \part $P(10,5)$  %The pizza parlor will list the 10 toppings in two columns on their menu.  How many ways can they arrange the toppings in the left column?
\end{parts}

\question ${7\choose 2}{7\choose 2}$ %How many quadrilaterals can you draw using the dots below as vertices (corners)?

% %\begin{center}
%  \begin{tikzpicture}
%   \foreach \x in {-3,...,3}
%   \foreach \y in {-1,1}
%   \fill (\x,\y) circle (3pt);
%  \end{tikzpicture}
% \end{center}

\question %How many of the quadrilaterals possible in the previous problem are:
\begin{parts}
 \part 5 (you need to skip one dot the top and the bottom). %Squares?
 \part ${7 \choose 2}$ - once you select the two dots on the top, the bottom two are determined. % Rectangles?
 \part This is tricky - you need to worry about running out of space.  One way to count: break into cases by the location of the top left corner.  You get ${7 \choose 2} + ({7 \choose 2}-1) + ({7 \choose 2} - 3) + ({7 \choose 2} - 6) + ({7 \choose 2} - 10) + ({7 \choose 2} - 15)$ %Parallelograms?
 \part All of them %Trapezoids?
\end{parts}


\question %On a business retreat, your company of 20 businessmen go golfing.
\begin{parts}
 \part ${20 \choose 4}{16 \choose 4}{12 \choose 4}{8 \choose 4}{4 \choose 4}$ %You need to divide up into foursomes (groups of 4 people): a first foursome, a second foursome, and so on.  How many ways can you do this?
 \part $5!{15 \choose 3}{12 \choose 3}{9 \choose 3}{6 \choose 3}{3 \choose 3}$ %After all your hard work, you realize that in fact, you want each foursome to include one of the five CEO's.  How many ways can you do this?
\end{parts}

\question $9!$ (there are 10 people seated around the table, but it does not matter where King Arthur sits, only who sits to his left, two seats to his left, and so on).  %How many different seating arrangements are possible for King Arthur and his 9 knights around their round table?
\end{questions}


\end{document}


