\section{Final Exam Study Guide}

The Final Exam is cumulative, so anything we have discussed this semester is fair game.  I would strongly advise you to make sure you can solve every problem from the first three exams as well as the quizzes.  If you can do this, and {\em understand} what you are doing (so that small changes to the problems won't mess you up) you are very likely to do well on the final.  Don't forget, the final will also cover the material on number theory (studied after exam 3).

You also have lots of homework problems and practice problems (all of which have solutions on blackboard), and the problems on the three previous study guides.  This might be overwhelming, so here is a sort of checklist of topics on concepts you should understand:

\begin{itemize} 
 \item Logic, Quantifiers, and Sets
    \begin{itemize}
      \item Truth tables.
      \item Valid argument forms.
      \item Converse, Contrapositive, Negations.
      \item Expressing statements with Quantifiers and Sets.
    \end{itemize}
   
    \item Graph Theory
       \begin{itemize}
         \item Degrees of vertices
         \item Euler paths and circuits
         \item Planar graphs (including Euler's Formula: $V - E + F = 2$)
         \item Vertex coloring
       \end{itemize}
    
  \item Counting techniques
    \begin{itemize}
      \item Additive and Multiplicative principles
      \item Combinations (aka binomial coefficients, aka ${n\choose k}$)
      \item Permutations
      \item Principal of Inclusion Exclusion
      \item Stars and Bars
      \item Combinatorial proofs
      \item Functions: all, injective, surjective
    \end{itemize}

  \item Sequences
    \begin{itemize}
    \item Arithmetic and Geometric
    \item Summing terms and the sequence of partial sums
    \item Recursive and closed formulas
    \item Characteristic Root Technique, Telescoping, Iteration
    \item Generating functions
    \end{itemize}

\item Mathematical Induction

\item Number Theory
	\begin{itemize}
		\item Divisibility and division with remainder
		\item Congruence and modular arithmetic
		\item Diophantine equations
	\end{itemize}

 
\end{itemize}





Below are some of my favorite question from the semester.  Some are new, some are taken (or modified) from the previous study guides, practice problems, and homework.  We start with number theory questions, after which the problems are mostly in the same order we learned them this semester.

\begin{squestions}

\question Suppose $a \equiv b \pmod{n}$.  Explain what this means in terms of remainders after division, as well as what this means in terms of differences.  Explain why these are the same.

	\begin{answer}
		$a \equiv b \pmod{n}$ means that you get the same remainder when you divide $a$ by $n$ as you do when you divide $b$ by $n$.  It also means that $a-b$ is a multiple of $n$, or equivalently, that $n \mid a-b$.  These are the same: if the remainders are the same, then if you take the difference, those remainders will cancel, leaving you with remainder 0.
	\end{answer}





\question What is the remainder when you divide $7^{787}$ by 12?

	\begin{answer}
		$7^{787} = 7(7^2)^{393} = 7(49)^{393} \equiv 7(1)^{393} \pmod{12}$.  Thus the remainder will be 7.
	\end{answer}





\question Dividing $14x + 10$ by $18$ gives the same remainder as dividing 394 by 18.  What could the remainder be when you divide $x$ by 18?

	\begin{answer}
		In other words, we must solve $14x + 10 \equiv 394 \pmod{18}$.  First reduce $394 \equiv 16\pmod{18}$, and then subtract 10 from both sides.  This gives $14x \equiv 6 \pmod{18}$.  Now divide both sides by 2: $7x \equiv 3 \pmod{9}$ (since $18/\gcd{14,18} = 9$).  Add 9 repeatedly: $7x \equiv 21 \pmod{9}$ so $x \equiv 3 \pmod{9}$.  Since $x = 3 + 9k$, if we divide by 18, the remainder could be 3 or 12.
	\end{answer}





\question Solve the Diophantine equation $46x + 33y = 600$.

	\begin{answer}
		Rewrite the equation as a congruence, modulo 33.  This gives: $46x \equiv 600 \pmod{33}$, which reduces to $13x \equiv 6 \pmod{33}$, or more helpfully, $13x \equiv 39 \pmod{33}$ which yields $x \equiv 3 \pmod{33}$.  So $x = 3 + 33k$.  We then have $46(3+33k) + 33y = 600$ so $y = 14 - 46k$
	\end{answer}




\question Is it possible to make exactly \$6 using just 46-cent and 33-cent stamps?  If so, how would you do it?

	\begin{answer}
		Yes.  We are just asking whether the Diophantine equation $46x + 33y = 600$ has a solution.  Above, we found one solution was $x = 3$ and $y = 14$, so you can make \$6 using 3 of the 46-cent stamps and 14 of the 33-cent stamps.
	\end{answer}




\question Make a truth table for the statement $(\neg P \rightarrow Q) \wedge (Q \rightarrow R)$.  What can you conclude if you know the statement is false?

  \begin{answer}
  
  \begin{tabular}{c|c|c|c}
	 $P$ & $Q$ & $R$ & $(\neg P \rightarrow Q) \wedge (Q \rightarrow R)$ \\ \hline
	 T & T & T & T\\
	 T & T & F & F\\
	 T & F & T & T\\
	 T & F & F & T\\
	 F & T & T & T\\
	 F & T & F & F\\
	 F & F & T & F\\
	 F & F & F & F\\
  \end{tabular}

  If the statement is false, you cannot conclude much - there are 4 cases.
  \end{answer}






\question Consider the statement ``for all integers $a$ and $b$, if $a$ or $b$ is even, then $a\cdot b$ is even''
\begin{parts}
 \part Write the contrapositive of the statement
 \part Write the converse of the statement
 \part Write the negation of the statement.
 \part Is the original statement true or false?  Prove your answer.
 \part Is the contrapositive of the original statement true or false?  Prove your answer.
 \part Is the converse of the original statement true or false?  Prove your answer.
 \part Is the negation of the original statement true or false?  Prove your answer.
\end{parts}

  \begin{answer}
  \begin{parts}
  \part For all integers $a$ and $b$, if $a\cdot b$ is not even, then $a$ is not even and $b$ is not even.
  \part For all integers $a$ and $b$, if $a \cdot b$ is even then $a$ or $b$ is even.
  \part There are integers $a$ and $b$ such that $a$ or $b$ is even but $a\cdot b$ is not even.
  \part True.  Suppose $a$ or $b$ is even.  If $a$ is even, then $a = 2k$ for some $k$.  So $a\cdot b = 2k b$ which is even.  If $b$ is even, then $b = 2k$ for some $k$.  So $a \cdot b = 2ka$ which is even again.  So in either case, $a\cdot b$ is even.  QED.
  \part True.  The contrapositive is equivalent to the original statement, which is true.
  \part True.  We prove this by contrapositive.  Suppose $a$ and $b$ are odd.  Then $a = 2k + 1$ and $b = 2l+1$ for some integers $k$ and $l$.  Then $a\cdot b = (2k+1)(2l+1) = 4kl + 2k + 2l + 1 = 2(2kl + k + l) + 1$ which is odd.  QED.
  \part False, since the original statement is true.
  \end{parts}
  \end{answer}




\question Is the following a valid argument form?  Check using a truth table, and then explain why in words.
\begin{tabular}{c}
$P \imp (Q \vee R)$ \\
$P \and \neg Q$ \\ \hline
$\therefore$ ~~ $R$ ~~~~
\end{tabular}

	\begin{answer}
		Yes, the argument is valid: whenever the premises are true, so is the conclusion.  If you make a truth table that includes columns for $P \imp (Q \vee R)$ and also for $P \and \neg Q$, then for every row in which {\em both} of these columns are true, the entry for $R$ on that row will also be true.
		
		This makes sense.  If $P \imp (Q \vee R)$ is true, as well as $P \and \neg Q$, then in particular, $P$ is true.  A true implication with a true hypothesis must have a true conclusion, so $Q \vee R$ is true.  But if $\neg Q$ is true, then $Q$ must be false.  So the only way for $Q \vee R$ to be true is if $R$ is true.
	\end{answer}
	




\question Simplify the statements (so negation appears only directly next to predicates).
\begin{parts}
  \part $\neg \exists x \forall y (\neg O(x) \vee E(y))$
  \part $\neg \forall x \neg \forall y \neg(x < y \and \exists z (x < z \vee y < z))$
  \part There is a number $n$ for which no other number is either less $n$ than or equal to $n$.
  \part It is false that for every number $n$ there are two other numbers which $n$ is between.
\end{parts}

  \begin{answer}
  \begin{parts}
	 \part $\forall x \exists y (O(x) \and \neg E(y))$
	 \part $\exists x \forall y (x \ge y \vee \forall z (x \ge z \and y \ge z))$
	 \part There is a number $n$ for which every other number is strictly greater than $n$.
	 \part There is a number $n$ which is not between any other two numbers.
  \end{parts}
  \end{answer}



\question Let $A = \{1, 2, 3, 4, 5\}$ and $B = \{2, 3, 5, 7\}$ be subsets of the universal set $S = \{1, 2, \ldots, 10\}$.  Find:
\begin{parts}
\part $A \cup B$
\part $A \cap B$
\part $\bar A$
\part $A \cap \bar B$
\end{parts}

  \begin{answer}
  \begin{parts}
  \part $A \cup B = \{1,2,3,4,5,7\}$
  \part $A \cap B = \{2, 3, 5\}$
  \part $\bar A = \{6, 7, 8, 9, 10\}$
  \part $A \cap \bar B = \{1, 4\}$
  \end{parts}
  \end{answer}



\question Give an example of set $A$ and $B$ such that $|A| = 5$, $|B| = 2$, $|A \cap B| = 1$ and $A \in B$.

  \begin{answer}
  $A = \{1,2,3,4,5\}$, $B = \{1, \{1,2,3,4,5\}\}$.  The sets have only one element in common (1).
  \end{answer}





\question Which of the statements below are equivalent to the statement, ``if a number is prime, then it is the difference of squares,'' and which are equivalent to the converse of the statement?  
\begin{parts}
	\part Either a number is not prime, or else it is the difference of squares.
	\part To be a prime number, it is sufficient to be the difference of squares.
	\part To be a prime number, it is necessary to be the difference of squares.
	\part A number is the difference of squares if and only if it is prime.
	\part There does not exists a number which is both prime and not a difference of squares.
	\part For all numbers $x$, $x$ is prime if $x$ is the difference of squares.
	\part The set of numbers which are differences of squares is a subset of the set of primes.
\end{parts}

	\begin{answer}
		\begin{parts}
			\part This is equivalent to the original implication.  $P \imp Q$ is logically equivalent to $\neg P \vee Q$.
			\part Converse. Another way to say this, is that as soon as a number is a difference of squares, it must also be prime.  It would therefore be possible for a number to be prime, even though it is not a difference of squares (this rules out $P \imp Q$), but impossible for the number to be a difference of squares and not a prime. 
			\part Original implication.  It cannot be that the number is prime and not a difference of squares.  So the statement is false if $P$ is true and $Q$ is false, in other words, $P \imp Q$.
			\part Neither.  $P \iff Q$ is logically equivalent to $(P \imp Q) \and (Q \imp P)$
			\part Original implication.  $\neg \exists x (P(x) \and Q(x))$ is equivalent to $\forall x (\neg P(x) \vee Q(x))$.  
			\part Converse.  Note the ``if'' is directly in front of ``$x$ is a difference of squares.'' 
			\part Converse.  Another way to say this is that every element of the set of difference of squares is an element of the set of primes.  So if a number is the difference of squares, then it is prime, which is the converse.
		\end{parts}
	\end{answer}


\question At a school dance, 6 girls and 4 boys take turns dancing (as couples) with each other.
\begin{parts}
  \part How many couples danced if every girl dances with ever boy?
  \part How many couples danced if every one danced with everyone else (regardless of gender)?
  \part Explain what graphs can be used to represent these situations.
\end{parts}

  \begin{answer}
  \begin{parts}
	 \part There were 24 couples - 6 choices for the girl and 4 choices for the boy.
	 \part There were 45 couples - ${10 \choose 2}$ since we must choose two of the 10 people to dance together.
	 \part For part (a), we are counting the number of edges in $K_{4,6}$.  In part (b) we count the edges of $K_{10}$.
  \end{parts}
  \end{answer}


\question Among a group of $n$ people, is it possible for everyone to be friends with an odd number of people in the group?  If so, what can you say about $n$?

  \begin{answer}
  Yes, as long as $n$ is even.  If $n$ were odd, then corresponding graph would have an odd number of odd degree vertices, which is impossible.
  \end{answer}


\question How many edges does the graph $K_{n,n}$ have?  For which values of $n$ does the graph contain an Euler circuit?  For which values of $n$ is the graph planar?

  \begin{answer}
  $K_{n,n}$ has $n^2$ edges.  The graph will have an Euler circuit when $n$ is even.  The graph will be planar only when $n < 3$.
  \end{answer}


\question The graph $G$ has 6 vertices with degrees $1, 2, 2, 3, 3, 5$.  How many edges does $G$ have?  If $G$ was planar how many faces would it have?  Does $G$ have an Euler path?

  \begin{answer}
  $G$ has 8 edges (since the sum of the degrees is 16).  If $G$ is planar, then it will have 4 faces (since $6 - 8 + 4 = 2$).  $G$ does not have an Euler path since there are more than 2 vertices of odd degree.
  \end{answer}


\question Explain why a graph with a vertex with odd degree cannot have an Euler circuit? 

  \begin{answer}
  If a vertex in $G$ has an odd degree, then any path which does not start there will eventually get stuck there: the number of edges used in the path towards the vertex will be greater than the number of edges used away from the vertex.  If the path started at the given vertex, it would not be able to end there for the same (although opposite) reason.
  \end{answer}


\question What is the smallest number of colors you need to properly color the vertices of $K_{7}$.  Can you say whether $K_7$ is planar based on your answer?

  \begin{answer}
  $7$ colors.  Thus $K_7$ is not planar (by the contrapositive of the Four Color Theorem).
  \end{answer}


\question What is the smallest number of colors you need to properly color the vertices of $K_{3,4}$?  Can you say whether $K_{3,4}$ is planar based on your answer?

  \begin{answer}
  The chromatic number of $K_{3,4}$ is 2, since the graph is bipartite.  You cannot say whether the graph is planar based on this coloring (the converse of the Four Color Theorem is not true).  In fact, the graph is {\em not} planar, since it contains $K_{3,3}$ as a subgraph.
  \end{answer}


\question If a planar graph $G$ with $7$ vertices divides the the plane into 8 regions, how many edges must $G$ have?

  \begin{answer}
  $G$ has $13$ edges, since we need $7 - E + 8 = 2$.
  \end{answer}








\question For each part below, say whether the statement is true or false.  Explain why the true statements are true, and given counter-examples for the false statements.
\begin{parts}
  \part Every bipartite graph is planar.
  \part Every bipartite graph has chromatic number 2.
  \part Every bipartite graph has an Euler path.
  \part Every vertex of a bipartite graph has even degree.
  \part A graph is bipartite if and only if the sum of the degrees of all the vertices is even.
\end{parts}

  \begin{answer}
  \begin{parts}
	 \part False.  For example, $K_{3,3}$ is not planar.
	 \part True.  The graph is bipartite so it is possible to divide the vertices into two groups with no edges between vertices in the same group.  Thus we can color all the vertices of one group red and the other group blue.
	 \part False.  $K_{3,3}$ has 6 vertices with degree 3, so contains no Euler path.
	 \part False.  $K_{3,3}$ again.
	 \part False.  The sum of the degrees of all vertices is even for {\em all} graphs so this property does not imply that the graph is bipartite.
  \end{parts}
  \end{answer}



\question Consider the statement ``If a graph is planar, then it has an Euler path.''
\begin{parts}
 \part Write the converse of the statement.
 \part Write the contrapositive of the statement.
 \part Write the negation of the statement.
 \part Is it possible for the contrapositive to be false?  If it was, what would that tell you?
 \part Is the original statement true or false?  Prove your answer.
 \part Is the converse of the statement true or false?  Prove your answer.
\end{parts}

  \begin{answer}
  \begin{parts}
  \part If a graph has an Euler path, then it is planar.
  \part If a graph does not have an Euler path, then it is not planar.
  \part There is a graph which is planar and does not have an Euler path.
  \part Yes.  In fact, in this case it is because the original statement is false.
  \part False.  $K_4$ is planar but does not have an Euler path.
  \part False.  $K_5$ has an Euler path but is not planar.
  \end{parts}
  \end{answer}








\question If $S = \{1, 2, 3, 4, 5, 6\}$, how many subsets of $S$ contain no prime numbers?  How many subsets are there total?  How many subsets contain at least one odd number?  How many doubletons contain exactly one even number?
  
  \begin{answer}
   If $S = \{1, 2, 3, 4, 5, 6\}$, how many subsets of $S$ contain no prime numbers? 8.  How many subsets are there total? 64. How many subsets contain at least one odd number? 56.  How many doubletons contain exactly one even number? 9.
  \end{answer}

  

\question For how many $n \in \{1,2, \ldots, 500\}$ is $n$ a multiple of $3$, $4$, or $7$?

\begin{answer}
 Using PIE: 286.
\end{answer}





\question Give an example of a counting problem for which ${9\choose 3}$ is the solution.  What would need to be different about your problem for the solution to be $P(9,3)$?  How about $9^3$.  How about ${11 \choose 2}$?  How about ${8 \choose 2}$?

  \begin{answer}
   There are ${9 \choose 3}$ ways to select 3 of your 9 favorite books to take on a trip.  There are $P(9,3)$ ways to read 3 of your favorite 9 books (assuming you read only one at a time).  If you want to read three books, but don't mind reading the same book multiple times, there are $9^3$ ways to read 3 books from your 9 choices.  If it's time to give away your 9 favorite books to 3 friends, then there are $\{8 \choose 2\}$ ways to do that (if each friend gets at least one book) or $\{11 \choose 2\}$ (if some of your friends might be illiterate).
  \end{answer}


\question How many subsets of $\{1,2,\ldots, 10\}$ contain at least 6 elements?

  \begin{answer}
   ${10 \choose 6} + {10 \choose 7} + {10 \choose 8} + {10 \choose 9} + {10 \choose 10}$
  \end{answer}


\question How many subsets of $\{1,2,\ldots, 10\}$ contain (as subsets) either $\{1,2,3\}$ or $\bar{\{8,9,10\}}$ or both?

  \begin{answer}
   $2^7 + 2^7 - 2^4$.
  \end{answer}


\question How many of the subset in the previous question have cardinality 6?

  \begin{answer}
   ${7 \choose 3} + {7 \choose 6} - {4 \choose 3}$
  \end{answer}


\question How many 5 letter words made from the letters $a,b,c,d,e,f$ without repeats do not contain the sub-word ``bad'' in (a) consecutive letters? or (b) not-necessarily consecutive letters (but in order)? 

  \begin{answer}
   (a) 702  (b) 660
  \end{answer}



\question How many shortest lattice paths start at (2,3) and
\begin{parts}
  \part end at (8,11)?
  \part end at (8,11) and pass through (4,7)?
  \part end at (8,11) and avoid (4,7)?
\end{parts}

  \begin{answer}
   \begin{parts}
    \part ${14 \choose 6}$
    \part ${6 \choose 2}{8 \choose 4}$
    \part ${14 \choose 6} - {6 \choose 2}{8 \choose 4}$
   \end{parts}
  \end{answer}


\question The dollar menu at your favorite tax-free fast food restaurant has 7 items.  You have \$10 to spend. How many different meals can you buy if you spend all your money and: 
\begin{parts}
  \part Purchase at least one of each item.
  \part Possibly skip some items.
  \part Don't get more than 2 of any particular item.
\end{parts}

  \begin{answer}
  Hint: Stars and bars.
   \begin{parts}
    \part ${9 \choose 6}$
    \part ${16 \choose 6}$
    \part ${16 \choose 6} - \left[{7 \choose 1}{13 \choose 6} - {7 \choose 2}{10 \choose 6} + {7 \choose 3}{7 \choose 6}\right]$
   \end{parts}
  \end{answer}



\question Explain using lattice paths why ${n \choose k} = {n \choose n-k}$.  What sort of proof have you just given?

  \begin{answer}
   ${n\choose k}$ counts the number of paths starting at the origin and ending at a point $n$ steps away, $k$ steps of which are up.  Instead of selecting which of the step are up, we could have selected which of the steps go right.  If there are $k$ steps up, there are $n-k$ steps right.  Thus the total number of paths to the same point is ${n \choose n-k}$.  This is an example of a combinatorial proof.
  \end{answer}


\question Explain using bit-strings why $\sum_{k=0}^n {n \choose k} = 2^n$.  What sort of proof have you just given.

  \begin{answer}
   There are $2^n$ bit strings with length $n$, since you have 2 choices for each of the $n$ bits.  We can count this same quantity by first considering the number of $n$-bit strings with zero 1s, ${n\choose 0}$, then all the $n$-bit strings with one 1, ${n \choose 1}$, then with two 1s, and so on.  This gives the left hand side.  Another combinatorial proof.
  \end{answer}



\question What is the coefficient of $x^9$ in the expansion of $(x+3)^{20} + x^2(x+1)^{10}$?

  \begin{answer}
   ${20 \choose 9}3^{11} + {10 \choose 7}$
  \end{answer}


\question The Grinch sneaks into a room with 5 Christmas presents to 5 different people.  He proceeds to switch the name-labels on the presents.  How many ways could he do this if:
\begin{parts}
  \part No present is allowed to end up with its original label?
  \part Exactly 2 presents keep their original labels?
\end{parts}

  \begin{answer}
   \begin{parts}
    \part $5! - \left[{5 \choose 1} 4! - {5 \choose 2}3! + {5 \choose 3}2! - {5 \choose 4}1! + {5 \choose 5}0!\right]$
    \part ${5 \choose 2}\left(3! - \left[{3 \choose 1}2! - {3 \choose 2} 1! + {3 \choose 3}0!\right]\right)$
   \end{parts}
  \end{answer}


\question How many functions map $\{1,2,3,4,5,6\}$ {\em onto} $\{a,b,c,d\}$.

 \begin{answer}
  $4^6 - \left[{4 \choose 1}3^6 - {4 \choose 2} 2^6 + {4 \choose 1}1^6\right]$
 \end{answer}


\question How many functions from $\{1, 2, 3, 4, 5, 6\}$ to $\{a,b,c,d\}$ are one-to-one?

  \begin{answer}
   None.
  \end{answer}


\question \textit{Conic}, your favorite math themed fast food drive-in offers 20 flavors which can be added to your soda.  You have enough money to buy a large soda with 4 added flavors.  How many different soda concoctions can you order if:
\begin{parts}
  \part you refuse to use any of the flavors more than once?
  \part you refuse repeats but care about the order the flavors are added?
  \part you allow yourself multiple shots of the same flavor?
  \part you allow yourself multiple shots, and care about the order the flavors are added?
\end{parts}

  \begin{answer}
   \begin{parts}
    \part ${20 \choose 4}$ (order does not matter and repeats are not allowed)
    \part $P(20, 4) = 20\cdot 19\cdot 18 \cdot 17$ (order matters and repeats are not allowed)
    \part ${23 \choose 19}$ (order does not matter and repeats are allowed - stars and bars)
    \part $20^4$ (order matters and repeats are allowed - 20 choices 4 times)
   \end{parts}
  \end{answer}


\question Give a combinatorial proof for the identity ${n\choose k} = {n-1 \choose k} + {n-1 \choose k-1}$
  
  \begin{answer}
    How many pizzas are possible if you are allowed $k$ toppings chosen from a list of $n$ toppings?  On one hand the answer is just ${n \choose k}$.  Alternatively, we could first count all the pizzas that do not have anchovies (there are ${n-1 \choose k}$ of those, since you need to pick $k$ toppings from now one few choices) and the count all the pizzas that do have anchovies (there are ${n-1 \choose k-1}$ of those, since you only need to pick $k-1$ more toppings from the $n-1$ other available choices).
  \end{answer}









\question If $|A| = 4$ and $|B| = 7$, how many functions are there from $A$ to $B$?  Of those, how many are one-to-one?  How many are onto?  How many are both? 

  \begin{answer}
	 There are $7^4$ functions, $P(7,4)$ of which are one-to-one.  None of the functions are onto (or both).
  \end{answer}


\question Repeat the previous problem using $|A| = 7$ and $|B| = 4$.

  \begin{answer}
	 There are $4^7$ functions, none of which are one-to-one.  The number of onto functions is $4^7 - \left[{4 \choose 1}3^7 - {4 \choose 2}2^7 + {4 \choose 3}1^7\right]$
  \end{answer}



\question Consider the sequence $5, 11, 17, 23, \ldots, $. 
\begin{parts}
\part What is the next term in the sequence?  
\part Find a recurrence relation for the sequence.
\part Find a formula for the $n$th term of this sequence, assuming $a_1 = 5$.
\part Find a generating function for the sequence.
\part Find the sum of the first 100 terms of the sequence: $\sum_{k=1}^{100}a_k$.
\part Write out the first 5 terms of the sequence of partial sums: $S(n) = \sum_{k=1}^n a_k$.
\part Find a generating function for $S(n)$.
\end{parts}

  \begin{answer}
    \begin{parts}
		\part 29 
		\part $a_n = a_{n-1} + 6$ with initial term $a_1 = 5$
		\part $a_n = 5 + 6(n-1)$
		\part $\frac{5x}{1-x} + \frac{6x^2}{(1-x)^2}$
		\part $\frac{604 \cdot 100}{2} = 30200$
		\part $5, 16, 33, 56, 85,\ldots$
		\part $(\frac{5x}{1-x} + \frac{6x^2}{(1-x)^2})(\frac{1}{1-x})$
	 \end{parts}
  \end{answer}





\question Consider the sequence $2, 6, 10, 14, \ldots, 4n + 6$.  
\begin{parts}
\part How many terms are there in the sequence?
\part What is the second-to-last term?
\part Find the sum of all the terms in the sequence.
\end{parts}

  \begin{answer}
	 \begin{parts}
	 \part $n+2$ terms
	 \part $4n+2$
	 \part $\frac{(4n+8)(n+2)}{2}$
	 \end{parts}
  \end{answer}



\question Find $3 + 7 + 11+ \cdots + 427$.

  \begin{answer}
   $\frac{430\cdot 107}{2} = 23005$
  \end{answer}


\question Find $5 + 15 + 45 + \cdots + 5\cdot 3^{20}$

  \begin{answer}
   $\frac{5-5\cdot 3^{21}}{-2}$
  \end{answer}




\question Find the next 2 terms in the sequence $3, 5, 11, 21, 43, 85\ldots.$.  Then give a recursive definition for the sequence.

  \begin{answer}
   $\ldots, 171, 341,\ldots$.  $a_1 = 3$, $a_2 = 5$, $a_n = a_{n-1} + 2a_{n-2}$
  \end{answer}



\question Use polynomial fitting to find the formula for the $n$th term of the following sequences:
\begin{parts}
\part 2, 5, 11, 21, 36,\ldots
\part 0, 2, 6, 12, 20, \ldots
\end{parts}

  \begin{answer}
  \begin{parts}
	 \part $\frac{n^3}{6} + \frac{n^2}{2} + \frac{n}{3} + 1$. Third differences are constant, so $a_n = an^3 + bn^2 + cn + d$.  Use the terms of the sequence to solve for $a, b, c,$ and $d$.  
	 \part $a_n = n^2 - n$ 
	 \end{parts}
  \end{answer}



\question Can you use polynomial fitting to find the formula for the $n$th term of the sequence 4, 7, 11, 18, 29, 47, \ldots?  Explain why or why not. 

  \begin{answer}
   No.  The differences between terms give back the original sequence, so will never be constant.  We know that the closed formula will not be a polynomial.
  \end{answer}


\question Suppose the closed formula for a particular sequence is a degree 3 polynomial.  What can you say about the closed formula for:
\begin{parts}
  \part The sequence of partial sums.
  \part The sequence of second differences.
\end{parts}

  \begin{answer}
	 \begin{parts}
	  \part The sequence of partial sums will be a degree 4 polynomial (its sequence of differences will be the original sequence).
	  \part The sequence of second differences will be a degree 1 polynomial - an arithmetic sequence.
	 \end{parts}
  \end{answer}



\question Find the solution to the recurrence relation $a_n = 3a_{n-1} + 4a_{n-2}$ with initial terms $a_0 = 7$ and $a_1 = 2$.

  \begin{answer}
   $a_n = 1.8(4)^n + 5.2(-1)^n$
  \end{answer}



\question Solve the recurrence relation $a_n = 5 a_{n-1} + 1$ with $a_0 = 3$.

  \begin{answer}
   $a_n = 3\cdot 5^n + \frac{5^n - 1}{4}$ (using iteration).
  \end{answer}



\question Find the generating function for the sequence $-1, 3, -9, 27, -81, \ldots$
  
  \begin{answer}
   $\frac{-1}{1+3x}$
  \end{answer}


\question Find the generating function for the sequence $1, 1, 2, 3, 5, 8, 13 \ldots$

  \begin{answer}
   $\frac{1}{1-x-x^2}$
  \end{answer}

  
\question Find the generating function for the sequence $0, 2, 0, 0, 4, 0, 0, 6, 0, 0, 8, \ldots$

  \begin{answer}
   $\frac{2x}{(1-x^3)^2}$
  \end{answer}


\question Find the generating function for the sequence defined by the recurrence relation $a_n = 2a_{n-1} - 3a_{n-2} + 5$.
  
  \begin{answer}
   No matter what $a_0$ and $a_1$ are, we have the generating function $\frac{a_0 + (a_1 - 2a_0)x}{1-2x + 3x^2} + \frac{5x^2}{(1 - 2x + 3x^2)(1-x)}$
  \end{answer}


\question Prove $1^3 + 2^3 + 3^3 + \cdots + n^3 = \left(\frac{n(n+1)}{2}\right)^2$ holds for all $n \ge 1$, by mathematical induction.

	\begin{answer}
		Hint: This is a straight forward induction proof.  Note you will need to simplify $\left(\frac{n(n+1)}{2}\right)^2 + (n+1)^3$ and get $\left(\frac{(n+1)(n+2)}{2}\right)^2$.
	\end{answer}



\question By our rules for modular arithmetic, you know that $8^n \equiv 1 \pmod{7}$ for all $n \ge 0$.  Give a proof using mathematical induction.

	\begin{answer}
		Hint: For the inductive case, you can assume $8^k \equiv 1 \pmod{7}$.  We also know that $8 \equiv 1 \pmod{7}$.  Use property 3 for modular arithmetic to multiply these equations.
	\end{answer}



\question Suppose $a_0 = 1$, $a_1 = 1$ and $a_n = 3a_{n-1} - 2a_{n-1}$.  Prove, using strong induction, that $a_n = 1$ for all $n$.

	\begin{answer}
		Hint: there are two base cases $P(0)$ and $P(1)$.  Then, for the inductive case, assume $P(k)$ is true for all $k < n$.  This allows you to assume $a_{n-1} = 1$ and $a_{n-2} = 1$.  Apply the recurrence relation.
	\end{answer}




\question Prove, by induction, that the {\em greatest} amount of postage you {\em cannot} make exactly using 4 and 9 cent stamps is 23 cents.


  \begin{answer}
 Hint: one 9-cent stamp is 1 more than two 4-cent stamps, and seven 4-cent stamps is 1 more than three 9-cent stamps.
  \end{answer}



\question Recall, a {\em tree} is a connected graph containing no cycles.  Prove, using induction, that for any tree, the number of vertices is one larger than the number of edges.  

	\begin{answer}
		Let $V$ be the number of vertices and $E$ be the number of edges.  We will do induction on $E$. The base case can either be just a single vertex ($E=0$), or two vertices connected by an edge $(E=1)$.  In either case, $V = E+1$.  For the inductive case, suppose we have a tree with $E$ vertices, and $V = E+1$.  Now add an edge to the tree.  This can only be done by also adding a vertex (we cannot connect two vertices, because that would create a cycle).  So our new graph will have one more vertex and one more edge, but will therefore still maintain $V = E+1$.
	\end{answer}

	
\end{squestions}




