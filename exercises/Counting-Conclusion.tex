\begin{questions}


	


\question You have 9 presents to give to your 4 kids.  How many ways can this be done if
\begin{parts}
  \part The presents are identical, and each kid gets at least one present?
  \part The presents are identical, and some kids might get no presents?
  \part The presents are unique, and some kids might get no presents?
  \part the presents are unique and each kid gets at least one present?
\end{parts}

	\begin{answer}
	\begin{parts}
	  \part ${8 \choose 3}$, after giving one present to each kid, you are left with 5 presents (stars) which need to be divide among the 4 kids (giving 3 bars). %The presents are identical, and each kid gets at least one present?
	  \part ${12 \choose 3}$.  You have 9 stars and 3 bars.  %The presents are identical, and some kids might get no presents?
	  \part $4^9$.  You have 4 choices for whom to give each present.  This is like making a function from the set of present to the set of kids. %The presents are unique, and some kids might get no presents?
	  \part $4^9 - \left[{4 \choose 1}3^9 - {4\choose 2}2^9 + {4 \choose 3}1^9 \right]$.  Now the function from the set of present to the set of kids must be surjective. %the presents are unique and each kid gets at least one present?
	\end{parts}
	\end{answer}
	
	
\question For each of the following counting problems, say whether the answer is


\begin{itemize}
\begin{multicols}{3}
  \item[A] ${10\choose 4}$
  \item[B] $P(10,4)$
  \item[C] Neither
  \end{multicols}
\end{itemize}

If you answer is ``Neither,'' say what the answer should be instead.
\begin{parts}
  \part How many shortest lattice paths are there from $(0,0)$ to $(10,4)$?
  \part If you have 10 bow ties\index{bow ties}, and you want to select 4 of them for next week, how many choices do you have?
  \part Suppose you have 10 bow ties and you will wear one on each of the next 4 days.  How many choices do you have?
  \part If you want to wear 4 of your 10 bow ties next week (Monday through Sunday), how many ways can this be accomplished?
  \part Out of a group of 10 classmates, how many ways can you rank your top 4 friends?
  \part If 10 students come to their professor's office but only 4 can fit at a time, how different combinations of 4 students can see the prof first?
  \part How many 4 letter words can be made from the first 10 letters of the alphabet?
  \part How many ways can you make the word ``cake'' from the first 10 letters of the alphabet?
  \part How many ways are there to distribute 10 apples among 4 children?
  \part If you have 10 kids (and live in a shoe) and 4 types of cereal, how many ways can your kids eat breakfast?
  \part How many ways can you arrange exactly 4 ones in a string of 10 binary digits?
  \part You want to select 4 single digit numbers as your lotto picks.  How many choices do you have?
  \part 10 kids want ice-cream.  You have 4 varieties.  How many ways are there to give the kids as much ice-cream as they want?
  \part How many 1-1 functions are there from $\{1,2,\ldots, 10\}$ to $\{a,b,c,d\}$?
  \part How many surjective functions are there from $\{1,2,\ldots, 10\}$ to $\{a,b,c,d\}$?
  \part Each of your 10 bow ties match 4 pairs of suspenders.  How many outfits can you make?
  \part After the party, the 10 kids each choose one of 4 party-favors.  How many outcomes?
  \part How many 6-elements subsets are there of the set $\{1,2,\ldots, 10\}$
  \part How many ways can you split up 11 kids into 5 teams?
  \part How many solutions are there to $x_1 + x_2 + \cdots + x_5 = 6$ where each $x_i$ is non-negative?
  \part Your band goes on tour.  There are 10 cities within driving distance, but only enough time to play 4 of them.  How many choices do you have for the cities on your tour?
  \part In how many different ways can you play the 4 cities you choose?
  \part Out of the 10 breakfast cereals available, you want to have 4 bowls.  How many ways can you do this?
  \part There are 10 types of cookies available.  You want to make a 4 cookie stack.  How many different stacks can you make?
  \part From you home at (0,0) you want to go to either the donut shop at (5,4) or the one at (3,6).  How many paths could you take?
  \part How many 10-digit numbers do not contain a sub-string of 4 repeated digits?
\end{parts}

	\begin{answer}
	\begin{parts}
	    \part Neither.  ${14 \choose 4}$.
	  \part ${10\choose 4}$.
	  \part $P(10,4)$, since order is important.
	  \part Neither. Assuming you will wear each of the 4 ties on just 4 of the 7 days, without repeats: ${10\choose 4}P(7,4)$.
	  \part $P(10,4)$.
	  \part ${10\choose 4}$.
	  \part Neither. Since you could repeat letters: $10^4$. If no repeats are allowed, it would be $P(10,4)$.
	  \part Neither.  Actually, ``k'' is the 11th letter of the alphabet, so the answer is 0.  If ``k'' was among the first 10 letters, there would only be 1 way - write it down.
	  \part Neither.  Either ${9\choose 3}$ (if every kid gets an apple) or ${13 \choose 3}$ (if appleless kids are allowed).
	  \part Neither.  Note that this could not be ${10 \choose 4}$ since the 10 things and 4 things are from different groups.  $4^{10}$.
	  \part ${10 \choose 4}$ - don't be fooled by the ``arrange'' in there - you are picking 4 out of 10 {\em spots} to put the 1's. 
	  \part ${10 \choose 4}$ (assuming order is irrelevant). 
	  \part Neither.  $16^{10}$ (each kid chooses yes or no to 4 varieties).
	  \part Neither.  0.
	  \part Neither.  $4^{10} - [{4\choose 1}3^{10} - {4\choose 2}2^{10} + {4 \choose 3}1^{10}]$.
	  \part Neither.  $10\cdot 4$.
	  \part Neither. $4^{10}$.
	  \part ${10 \choose 4}$ (which is the same as ${10 \choose 6}$).
	  \part Neither.  If all the kids were identical, and you wanted no empty teams, it would be ${10 \choose 4}$.  Instead, this will be the same as the number of surjective functions from a set of size 11 to a set of size 5. 
	  \part ${10 \choose 4}$.
	  \part ${10 \choose 4}$.
	  \part Neither.  $4!$.
	  \part Neither.  It's ${10 \choose 4}$ if you won't repeat any choices.  If repetition is allowed, then this becomes $x_1 + x_2 + \cdots +x_{10} = 4$, which has ${13 \choose 9}$ solutions in non-negative integers.
	  \part Neither.  Since repetition of cookie type is allowed, the answer is $10^4$.  Without repetition, you would have $P(10,4)$.
	  \part ${10 \choose 4}$ since that is equal to ${9 \choose 4} + {9 \choose 3}$.
	  \part Neither.  It will be a complicated (possibly PIE) counting problem.
	\end{parts}
	\end{answer}
	


\question Recall, you own 3 regular ties and 5 bow ties\index{bow ties}.  You realize that it would be okay to wear more than two ties to your clown college interview.
\begin{parts}
 \part You must select some of your ties to wear. Everything is okay, from no ties up to all ties.  How many choices do you have?
 \part If you want to wear at least one regular tie and one bow tie, but are willing to wear up to all your ties, how many choices do you have for which ties to wear?
 \part How many choices do you have if you wear exactly 2 of the 3 regular ties and 3 of the 5 bow ties?
 \part Once you have selected 2 regular and 3 bow ties, in how many orders could you put the ties on, assuming you must have one of the three bow ties on top?
\end{parts}

  \begin{answer}
    \begin{parts}
      \part $2^8 = 256$.  You have two choices for each tie: wear it or don't. %You must select some of your ties to wear - everything is okay, from no ties up to all ties.  How many choices do you have?
      \part You have 7 choices for regular ties (the 8 choices less the ``no regular tie'' option) and 31 choices for bow ties (32 total minus the ``no bow tie'' option).  Thus total you have $7 \cdot 31 = 217$.  %If you want to wear at least one regular tie and one bow tie, but are willing to wear up to all your ties, how many choices do you have for which ties to wear?
      \part ${3\choose 2}{5\choose 3} = 30$.  %How many choices do you have if you wear exactly 2 of the 3 regular ties and 3 of the 5 bow ties?
      \part Select one of the 3 bow ties to go on top.  There are then 4 choices for the next tie, 3 for the tie after that, and so on.  Thus $3\cdot 4! = 72$.  %Once you have selected 2 regular and 3 bow ties, in how many orders could you put the ties on, assuming you must have one of the three bow ties on top?
    \end{parts}
  \end{answer}	

\question Give a counting question where the answer is $8\cdot 3 \cdot 3 \cdot 5$.  Give another question where the answer is $8 + 3 + 3 + 5$.

	\begin{answer}
		You own 8 purple bow ties\index{bow ties}, 3 red bow ties, 3 blue bow ties and 5 green bow ties.  How many ways can you select one of each color bow tie to take with you on a trip?  $8 \cdot 3 \cdot 3 \cdot 5$.  How many choices do you have for a single bow tie to wear tomorrow?  $8 + 3 + 3 + 5$.
	\end{answer}
	
	

\question Consider five digit numbers $\alpha = a_1a_2a_3a_4a_5$, with each digit from the set $\{1,2,3,4\}$.
\begin{parts}
\part How many such numbers are there?
\part How many such numbers are there for which the {\em sum} of the digits is even?
\part How many such numbers contain more even digits than odd digits?
\end{parts}

	\begin{answer}
		\begin{parts}
		\part $4^5$. %How many such numbers are there?
		\part $4^4\cdot 2$ (choose any digits for the first four digits - then pick either an even or an odd last digit to make the sum even). %How many such numbers are there for which the {\em sum} of the digits is even?
		\part We need 3 or more even digits.  3 even digits: ${5 \choose 3}2^3 2^2$.  4 even digits: ${5 \choose 4}2^4 2$.  5 even digits: ${5 \choose 5}2^5$.  So all together: ${5 \choose 3}2^3 2^2 + {5 \choose 4}2^4 2 + {5 \choose 5}2^5$.  %  How many such numbers contain more even digits than odd digits?
		\end{parts}
	\end{answer}
	
	
	
\question Let $A$ and $B$ be finite sets.  Explain, in words, why $|A \cup B| \le |A| + |B|$ and why $|A \cup B| = |A| + |B| - |A \cap B|$.

	\begin{answer}
		$|A \cup B|$ is the number of things that are in $A$ or in $B$ or in both.  If you count up everything in each set independently, then anything which is in both sets (in $A \cap B$) is counted twice.
	\end{answer}
	
	
	
\question For how many $n \in \{1,2, \ldots, 500\}$ is $n$ a multiple of one or more of 5, 6, or 7?

	\begin{answer}
		215.  Use PIE: $100 + 83 + 71 - 16 - 14 -11 + 2 = 215$ or a Venn diagram.  To find out how many numbers are divisible by 6 and 7, for example, take $500/42$ and round down.
	\end{answer}
	
	


\question In a recent small survey of airline passengers, 25 said they had flown American in the last year, 30 had flown Jet Blue, and 20 had flown Continental.  Of those, 10 reported they had flown on American and Jet Blue, 12 had flown on Jet Blue and Continental, and 7 had flown on American and Continental.  5 passengers had flown on all three airlines. 

How many passengers were surveyed?  (Assume the results above make up the entire survey.)

	\begin{answer}
		51.
	\end{answer}
	
	


	

\question Recall, by $8$-bit strings, we mean strings of binary digits, of length 8.
\begin{parts}
  \part How many $8$-bit strings are there total?
  \part How many $8$-bit strings have weight 5?
  \part How many subsets of the set $\{a,b,c,d,e,f,g,h\}$ contain exactly 5 elements?
  \part Explain why your answers to parts (b) and (c) are the same.  Why are these questions equivalent?
\end{parts}

	\begin{answer}
		\begin{parts}
		  \part $2^8$. %How many $8$-bit strings are there total?
		  \part ${8 \choose 5}$.  %How many $8$-bit strings have weight 5?
		  \part ${8 \choose 5}$. %How many subsets of the set $\{a,b,c,d,e,f,g,h\}$ contain exactly 5 elements?
		  \part There is a bijection between subsets and bit strings: a 1 means that element in is the subset, a 0 means that element is not in the subset.  To get a subset of an 8 element set we have a 8-bit string.  To make sure the subset contains exactly 5 elements, there must be 5 1's, so the weight must be 5. %Explain why your answers to parts (b) and (c) are the same.  Why are these questions equivalent?
		\end{parts}
	\end{answer}
	
	

\question What is the coefficient of $x^{10}$ in the expansion of $(x+1)^{13} + x^2(x+1)^{17}$?

	\begin{answer}
		${13 \choose 10} + {17 \choose 8}$.
	\end{answer}
	
	


\question How many 8-letter words contain exactly 5 vowels (a,e,i,o,u)?  What if repeated letters were not allowed?

	\begin{answer}
		 With repeated letters allowed: ${8 \choose 5}5^5 21^3$.  Without repeats: ${8 \choose 5}5! P(21, 3)$.
	\end{answer}
	
	


\question For each of the following, find the number of shortest lattice paths from $(0,0)$ to $(8,8)$ which:
\begin{parts}
  \part pass through the point $(2,3)$.
  \part avoid (do not pass through) the point $(7,5)$.
  \part either pass through $(2,3)$ or $(5,7)$ (or both).
\end{parts}

	\begin{answer}
		\begin{parts}
		  \part ${5 \choose 2}{11 \choose 6}$. %pass through the point $(2,3)$.
		  \part ${16 \choose 8} - {12 \choose 7}{4 \choose 1}$.  %avoid (do not pass through) the point $(7,5)$.
		  \part ${5 \choose 2}{11 \choose 6} + {12 \choose 5}{4 \choose 3} - {5 \choose 2}{7 \choose 3}{4 \choose 3}$. %either pass through $(2,3)$ or $(5,7)$ (or both).
		\end{parts}
	\end{answer}
	
	


\question You live in Grid-Town on the corner of 2nd and 3rd, and work in a building on the corner of 10th and 13th.  How many routes are there which take you from home to work and then back home, but by a different route?

	\begin{answer}
		 ${18 \choose 8}\left({18 \choose 8} - 1\right)$. 
	\end{answer}
	
	


\question Give an example of a problem for which $P(n,k)$ is the solution.  Give another example of a problem for which ${n\choose k}$ is the solution.

	\begin{answer}
		 A test had $n$ questions on it, of which you must answer any $k$ questions.  How many choices do you have as to what order you answer the questions on the test?  $P(n,k)$.  When grading the test, how many different combinations of question might the professor see?  ${n \choose k}$.
	\end{answer}
	
	


%\question How many 10-bit strings contain 6 or more 1's?
%
%	\begin{answer}
%		${10 \choose 6} + {10 \choose 7} + {10 \choose 8} + {10 \choose 9} + {10 \choose 10}$ 
%	\end{answer}
	
	


\question How many 10-bit strings start with $111$ or end with $101$ or both?

	\begin{answer}
		 $2^7 + 2^7 - 2^4$.
	\end{answer}
	
	


\question How many 10-bit strings of weight 6 start with $111$ or end with $101$ or both?

	\begin{answer}
		${7 \choose 3} + {7 \choose 4} - {4 \choose 1}$.
	\end{answer}
	
	


\question How many 6 letter words made from the letters $a,b,c,d,e,f$ without repeats do not contain the sub-word ``bad'' in (a) consecutive letters? or (b) not-necessarily consecutive letters (but in order)? 

	\begin{answer}
		(a) $6! - 4\cdot 3!$.  (b) $6! - {6 \choose 3}3!$.
	\end{answer}
	
	


\question Explain using lattice paths why $\sum_{k=0}^n {n \choose k} = 2^n$.

	\begin{answer}
		$2^n$ is the number of lattice paths which have length $n$, since for each step you can go up or right.  Such a path would end along the line $x + y = n$.  So you will end at $(0,n)$, or $(1,n-1)$ or $(2, n-2)$ or \ldots or $(n,0)$.  Counting the paths to each of these points separately, give ${n \choose 0}$, ${n \choose 1}$, ${n \choose 2}$, \ldots, ${n \choose n}$ (each time choosing which of the $n$ steps to be to the right).
	\end{answer}
	
	

 
\question Explain the relationship between $\d{n\choose k}$ and $P(n,k)$.  Be sure to say both how the formulas for each are related, and why that relationship makes sense.

	\begin{answer}
		Hint: give a combinatorial proof for the identity $P(n,k) = {n \choose k} k!$.
	\end{answer}
	
	


\question Give your favorite argument for why Pascal's Triangle is symmetric.  That is, explain why \({n \choose k} = {n \choose n-k}\).

	\begin{answer}
		Of your $n$ bow ties\index{bow ties}, you decide to give $k$ away to charity.  How many ways can you do this?  On one hand, you can choose $k$ of the $n$ bow ties to give away in ${n \choose k}$ ways.  Alternatively, you can choose which bow ties to keep.  You must keep $n -k$ of them if you will give $k$ away, so you can choose the bow ties to keep in ${n \choose n-k}$ ways.  This gives a combinatorial proof for the identity.
	\end{answer}
	
	



\question Suppose you have 20 one-dollar bills to give out as prizes to your top 5 discrete math students.  How many ways can you do this if:
\begin{parts}
  \part each of the 5 students gets at least 1 dollar?
  \part some students might get nothing?
  \part each student gets at least 1 dollar but no more than 7 dollars?
\end{parts}

	\begin{answer}
		Hint: stars and bars%Suppose you have 20 one-dollar bills to give out as prizes to your top 5 discrete math students.  How many ways can you do this if:
		\begin{parts}
		  \part ${19 \choose 4}$. %each of the 5 students gets at least 1 dollar?
		  \part ${24 \choose 4}$. %some students might get nothing?
		  \part ${19 \choose 4} - \left[{5 \choose 1}{12 \choose 4} - {5 \choose 2}{5 \choose 4}  \right]$. %each student gets at least 1 dollar but no more than 7 dollars?
		\end{parts}
	\end{answer}
		
	

	


\question How many functions $f: \{1,2,3,4,5\} \to \{a,b,c,d,e\}$ are there for which
\begin{parts}
  \part $f(1) = a$ or $f(2) = b$ (or both)?
  \part $f(1) \ne a$ or $f(2) \ne b$ (or both)?
  \part $f(1) \ne a$ {\em and} $f(2) \ne b$, and are also one-to-one?
  \part are onto but have $f(1) \ne a$, $f(2) \ne b$, $f(3) \ne c$, $f(4) \ne d$ and $f(5) \ne e$?
\end{parts}

	\begin{answer}
		\begin{parts}
		  \part $5^4 + 5^4 - 5^3$. %$f(1) = a$ or $f(2) = b$ (or both)?
		  \part $4\cdot 5^4 + 5 \cdot 4 \cdot 5^3 - 4 \cdot 4 \cdot 5^3$. %$f(1) \ne a$ or $f(2) \ne b$ (or both)?
		  \part $5! - \left[ 4! + 4! - 3! \right]$. %$f(1) \ne a$ {\em and} $f(2) \ne b$, and are also one-to-one?
		  \part $5! - \left[{5 \choose 1}4! - {5 \choose 2}3! + {5 \choose 3}2! - {5 \choose 4}1! + {5 \choose 5} 0!\right]$. %are onto but have $f(1) \ne a$, $f(2) \ne b$, $f(3) \ne c$, $f(4) \ne d$ and $f(5) \ne e$?
		\end{parts}
	\end{answer}
	
	


\question How many permutations of $\{1,2,3,4,5\}$ leave exactly 1 element fixed?

	\begin{answer}
		 ${5 \choose 1}\left( 4! - \left[{4 \choose 1}3! - {4 \choose 2}2! + {4 \choose 3} 1! - {4 \choose 4} 0!\right] \right)$.
	\end{answer}
	
	


\question How many functions map $\{1,2,3,4,5,6\}$ {\em onto} $\{a,b,c,d\}$ (i.e., how many {\em surjections} are there)?

	\begin{answer}
		$4^6 - \left[{4 \choose 1}3^6 - {4 \choose 2}2^6 + {4 \choose 3} 1^6 \right]$.
	\end{answer}
	



\question To thank your math professor for doing such an amazing job all semester, you decide to bake him (or her) cookies.  You know how to make 10 different types of cookies.
\begin{parts}
 \part If you want to give your professor 4 different types of cookies, how many different combinations of cookie type can you select?  Explain your answer.
 \part To keep things interesting, you decide to make a different number of each type of cookie.  If again you want to select 4 cookie types, how many ways can you select the cookie types and decide for which there will be the most, second most, etc.  Explain your answer.
 \part You change your mind again.  This time you decide you will make a total of 12 cookies.  Each cookie could be any one of the 10 types of cookies you know how to bake (and it's okay if you leave some types out).  How many choices do you have?  Explain.
 \part You realize that the previous plan did not account for presentation.  This time, you once again want to make 12 cookies, each one could be any one of the 10 types of cookies.  However, now you plan to shape the cookies into the numerals 1, 2, \ldots, 12 (and probably arrange them to make a giant clock, but you haven't decided on that yet).  How many choices do you have for which types of cookies to bake into which numerals?  Explain.
 \part The only flaw with the last plan is that your professor might not get to sample all 10 different varieties of cookies.  How many choices do you have for which types of cookies to make into which numerals, given that each type of cookie should be present at least once?  Explain.
\end{parts}

	\begin{answer}
		\begin{parts}
		 \part ${10 \choose 4}$.  You need to choose 4 of the 10 cookie types.  Order doesn't matter. %If you want to give your professor 4 different types of cookies, how many different combinations of cookie type can you select?  Explain your answer.
		 \part $P(10, 4) = 10 \cdot 9 \cdot 8 \cdot 7$.  You are choosing and arranging 4 out of 10 cookies.  Order matters now.  %To keep things interesting, you decide to make a different number of each type of cookie.  If again you want to select 4 cookie types, how many ways can you select the cookie types and decide for which there will be the most, second most, etc.  Explain your answer.
		 \part ${21 \choose 9}$.  You must switch between cookie type 9 times as you make your 12 cookies.  The cookies are the stars, the switches between cookie types are the bars. %You change your mind again.  This time you decide you will make a total of 12 cookies.  Each cookie could be any one of the 10 types of cookies you know how to bake (and it's okay if you leave some types out).  How many choices do you have?  Explain.
		 \part $10^{12}$.  You have 10 choices for the ``1'' cookie, 10 choices for the ``2'' cookie, and so on. %You realize that the previous plan did not account for presentation.  This time, you once again want to make 12 cookies, each one could be any one of the 10 types of cookies.  However, now you plan to shape the cookies into the numerals 1, 2, \ldots, 12 (and probably arrange them to make a giant clock - but you haven't decided on that yet).  How many choices do you have for which types of cookies to bake into which numerals?  Explain.
		 \part $10^{12} - \left[{10 \choose 1}9^{12} - {10 \choose 2}8^{12} + \cdots - {10 \choose 10}0^{12}   \right]$.  We must use PIE to remove all the ways in which one or more cookie type is not selected.
		\end{parts}
	\end{answer}
	
	
\question For which of the parts above does it make sense to interpret the counting question as counting some number of functions?  Say what the domain and codomain should be, and whether you are counting all functions, injections, surjections, or something else.
	
	\begin{answer}
		\begin{parts}
			\part You are giving your professor 4 types of cookies coming from 10 different types of cookies.  This does not lend itself well to a function interpretation.  We {\em could} say that the domain contains the 4 types you will give your professor and the codomain contains the 10 you can choose from, but then counting injections would be too much (it doesn't matter if you pick type 3 first and type 2 second, or the other way around, just that you pick those two types).
			\part We want to consider injective functions from the set $\{$most, second most, second least, least$\}$ to the set of 10 cookie types.  We want injections because we cannot pick the same type of cookie to give most and least of (for example).  
			\part This is not a good problem to interpret as a function.  The problem is that the domain would have to be the 12 cookies you bake, but these elements are indistinguishable (there is not a first cookie, second cookie, etc.).
			\part The domain should be the 12 shapes, the codomain the 10 types of cookies.  Since we can use the same type for different shapes, we are interested in counting all functions here.  
			\part Here we insist that each type of cookie be given at least once, so now we are asking for the number of surjections of those functions counted in the previous part.
		\end{parts}
	\end{answer}
	



\end{questions}