\begin{questions}

\question Your wardrobe consists of 5 shirts, 3 pairs of pants, and 17 bow ties.  How many different outfits can you make?

  \begin{answer}
    255.
  \end{answer}


  
\question For your college interview, you must wear a tie.  You own 3 regular (boring) ties and 5 (cool) bow ties.  How many choices do you have for your neck-wear? 

  \begin{answer}
    8.
  \end{answer}




\question You realize that the interview is for clown-college, so you should probably wear both a regular tie and a bow tie.  How many choices do you have now?

  \begin{answer}
    15.
  \end{answer}



\question You realize that it would also be okay to wear more than two ties.
\begin{parts}
 \part You must select some of your ties to wear - everything is okay, from no ties up to all ties.  How many choices do you have?
 \part If you want to wear at least one regular tie and one bow tie, but are willing to wear up to all your ties, how many choices do you have for which ties to wear?
 \part How many choices do you have if you wear exactly 2 of the 3 regular ties and 3 of the 5 bow ties?
 \part Once you have selected 2 regular and 3 bow ties, in how many orders could you put the ties on, assuming you must have one of the three bow ties on top?
\end{parts}

  \begin{answer}
    \begin{parts}
      \part $2^8 = 256$.  You have two choices for each tie - wear it or don't. %You must select some of your ties to wear - everything is okay, from no ties up to all ties.  How many choices do you have?
      \part You have 7 choices for regular ties (the 8 choices less the ``no regular tie'' option) and 31 choices for bow ties (32 total minus the ``no bow tie'' option).  Thus total you have $7 \cdot 31 = 217$.  %If you want to wear at least one regular tie and one bow tie, but are willing to wear up to all your ties, how many choices do you have for which ties to wear?
      \part ${3\choose 2}{5\choose 3} = 30$  %How many choices do you have if you wear exactly 2 of the 3 regular ties and 3 of the 5 bow ties?
      \part $5! = 120$  %Once you have selected 2 regular and 3 bow ties, in how many orders could you put the ties on, assuming you must have one of the three bow ties on top?
    \end{parts}
  \end{answer}



\question Your Blu-ray collection consists of 9 comedies and 7 horror movies. Give an example of a question for which the answer is:
\begin{parts}
 \part 16.
 \part 63.
\end{parts}

  \begin{answer}
    \begin{parts}
      \part 16 is the number of choices you have if you want to watch one movie, either a comedy or horror flick.
      \part 63 is the number of choices you have if you will watch two movies, first a comedy and then a horror.
    \end{parts}
  \end{answer}


  
\question If $|A| = 10$ and $|B| = 15$, what is the largest possible value for $|A \cap B|$?  What is the smallest?  What are the possible values for $|A \cup B|$?

  \begin{answer}
    $0 \le |A \cap B| \le 10$ and $15 \le |A \cup B| \le 25$.
  \end{answer}



\question If $|A| = 8$ and $|B| = 5$, what is $|A \cup B| + |A \cap B|$?

  \begin{answer}
      $|A \cup B| + |A \cap B| = 13$
  \end{answer}

  
  
  
\question A group of college students were asked about their TV watching habits.  Of those surveyed, 28 students watch {\em Elementary}, 19 watch {\em Castle} and 24 watch of {\em The Mentalist}.  Additionally, 16 watch {\em Elementary} and {\em Castle}, 14 watch {\em elementary} and {\em The Mentalist} and 10 watch {\em Castle} and {\em The Mentalist}.  There are 8 students who watch all three shows.  How many students surveyed watched at least one of the shows?

  \begin{answer}
    39.
  \end{answer}



\question Find $|(A \cup C)\cap \bar B|$ provided $|A| = 50$, $|B| = 45$, $|C| = 40$, $|A\cap B| = 20$, $|A \cap C| = 15$, $|B \cap C| = 23$ and $|A \cap B \cap C| = 12$.

    \begin{answer}
      $|(A \cup C)\cap \bar B| = 44$.  Use a Venn diagram.
    \end{answer}



\question Using the same data as the previous question, describe a set with cardinality 26.

    \begin{answer}
	One possibility: $(A \cup B) \cap C$.
    \end{answer}

    
    
    
\question Consider all 5 letter ``words'' made from the letters $a$ through $h$.
\begin{parts}
 \part How many of these words are there total?
 \part How many of these words contain no repeated letters?
 \part How many of these words (repetitions allowed) start with the sub-word ``aha''?
 \part How many of these words (repetitions allowed) either start with ``aha'' or end with ``bah'' or both?
 \part How many of the words containing no repeats also do not contain the sub-word ``bad'' (in consecutive letters)?  
\end{parts}

  \begin{answer}
    \begin{parts}
      \part $8^5$, since you select from 8 letters 5 times.  %How many of these words are there total?
      \part $8\cdot 7\cdot 6\cdot 5\cdot 4$.  After selecting a letter, you have fewer letters to select for the next one.  %How many of these words contain no repeated letters?
      \part 64 - you need to select the 4th and 5th letters. %How many of these words (repetitions allowed) start with the sub-word ``aha''?
      \part $64 + 64 - 0 = 128$.  There are 64 words which start with ``aha'' and another 64 words that end with ``bah.''  Perhaps we over counted the words that both start with ``aha'' and end with ``bah'' but since the words are only 5 letters long, there are no such words.  %How many of these words (repetitions allowed) either start with ``aha'' or end with ``bah'' or both?
      \part $(8\cdot 7\cdot 6\cdot 5\cdot 4) - 3\cdot (5\cdot 4) = 6660$ - all the words minus the bad ones.  The taboo word can be in any of three positions (starting with letter 1, 2, or 3) and for each position we must choose the other two letters (from the remaining 5 letters) %How many of the words containing no repeats also do not contain the sub-word ``bad'' (in consecutive letters)?  
    \end{parts}
  \end{answer}



\end{questions}