\begin{questions}


\question What is the smallest number of colors you need to properly color the vertices of $K_{4,5}$.  That is, find the chromatic number of the graph.

	\begin{answer}
		2, since the graph is bipartite.  One color for the top set of vertices, another color for the bottom set of vertices.  %What is the smallest number of colors you need to properly color the vertices of $K_{4,5}$.  That is, find the chromatic number of the graph.
	\end{answer}
	
	



\question Draw a graph with chromatic number 6 (i.e., which requires 6 colors to properly color the vertices).  Could your graph be planar?  Explain.

	\begin{answer}
		For example, $K_6$.  If the chromatic number is 6, then the graph is not planar - the 4-color theorem states that all planar graphs can be colored with 4 or fewer colors. %Draw a graph with chromatic number 6 (i.e., which requires 6 colors to properly color the vertices).  Could your graph be planar?  Explain.
	\end{answer}
	
	
	



\question Find the chromatic number of each of the following graphs.

\begin{center}
  \begin{tikzpicture}
    \draw[thick] (-1,1) \v -- (0,2) \v -- (1,1) \v -- (0,0) \v -- (-1,1) -- (0,1) \v -- (1,1);
  \end{tikzpicture}
  \hfill
  \begin{tikzpicture}
    \draw[thick] (360/7:1) \v -- (2*360/7:1) \v -- (3*360/7:1) \v -- (4*360/7:1) \v -- (5*360/7:1) \v -- (6*360/7:1) \v -- (0:1) \v -- cycle;
  \end{tikzpicture}
  \hfill 
  \begin{tikzpicture}
    \draw (0,0) \v;
    \foreach \x in {0,...,4}
    \draw[thick] (0,0) -- (\x*360/5:1) \v -- (\x*360/5+72:1);
  \end{tikzpicture}
  \hfill
  \begin{tikzpicture}
    \foreach \x in {0,...,4}
    \draw[thick] (\x*72+18:1) \v -- (\x*72+90:1) -- (\x*72-54:1);
  \end{tikzpicture}
  \hfill
    \begin{tikzpicture}[scale=.6]
    \draw[thick] (18:2) -- (90:2) -- (162:2)  -- (234:2) -- (306:2) -- cycle; 
    \draw[thick] (18:1) --  (162:1)  -- (306:1) -- (90:1) -- (234:1) --cycle;
    \foreach \x in {18, 90, 162, 234, 306}
    \draw[thick] (\x:1) \v -- (\x:2) \v;
  \end{tikzpicture}
\end{center}

	\begin{answer}
		The chromatic numbers are 2, 3, 4, 5, and 3 respectively from left to right. %Find the chromatic number of each of the following graphs.
	\end{answer}
	
	

\question What is the smallest number of colors that can be used to color the vertices of a cube so that no two adjacent vertices are colored identically?

	\begin{answer}
		The cube can be represented as a planar graph and colored with two colors as follows:
		
		\begin{center}
		\begin{tikzpicture}
		\foreach \ang in {45, 135, 225, 315} {
		\draw (\ang:.4) \v -- (\ang:1) \v -- (\ang+90:1) (\ang:.4) -- (\ang+90:.4);
		}
		\draw (45:.4) node[above]{\footnotesize R} (135:.4) node[above]{\footnotesize B} (225:.4) node[above]{\footnotesize R} (315:.4) node[above]{\footnotesize B} (45:1) node[above]{\footnotesize B} (135:1) node[above]{\footnotesize R} (225:1) node[above]{\footnotesize B} (315:1) node[above]{\footnotesize R};
		\end{tikzpicture}
		\end{center}
		
		Since it would be impossible to color the vertices with a single color, we see that the cube has chromatic number 2 (it is bipartite).
	\end{answer}

\end{questions}