\begin{questions}


\question What is the smallest number of colors you need to properly color the vertices of $K_{4,5}$.  That is, find the chromatic number of the graph.

	\begin{answer}
		2, since the graph is bipartite.  One color for the top set of vertices, another color for the bottom set of vertices.  %What is the smallest number of colors you need to properly color the vertices of $K_{4,5}$.  That is, find the chromatic number of the graph.
	\end{answer}
	
	



\question Draw a graph with chromatic number 6 (i.e., which requires 6 colors to properly color the vertices).  Could your graph be planar?  Explain.

	\begin{answer}
		For example, $K_6$.  If the chromatic number is 6, then the graph is not planar - the 4-color theorem states that all planar graphs can be colored with 4 or fewer colors. %Draw a graph with chromatic number 6 (i.e., which requires 6 colors to properly color the vertices).  Could your graph be planar?  Explain.
	\end{answer}
	
	
	



\question Find the chromatic number of each of the following graphs.

\begin{center}
  \begin{tikzpicture}
    \draw[thick] (-1,1) \v -- (0,2) \v -- (1,1) \v -- (0,0) \v -- (-1,1) -- (0,1) \v -- (1,1);
  \end{tikzpicture}
  \hfill
  \begin{tikzpicture}
    \draw[thick] (360/7:1) \v -- (2*360/7:1) \v -- (3*360/7:1) \v -- (4*360/7:1) \v -- (5*360/7:1) \v -- (6*360/7:1) \v -- (0:1) \v -- cycle;
  \end{tikzpicture}
  \hfill 
  \begin{tikzpicture}
    \draw (0,0) \v;
    \foreach \x in {0,...,4}
    \draw[thick] (0,0) -- (\x*360/5:1) \v -- (\x*360/5+72:1);
  \end{tikzpicture}
  \hfill
  \begin{tikzpicture}
    \foreach \x in {0,...,4}
    \draw[thick] (\x*72+18:1) \v -- (\x*72+90:1) -- (\x*72-54:1);
  \end{tikzpicture}
  \hfill
    \begin{tikzpicture}[scale=.6]
    \draw[thick] (18:2) -- (90:2) -- (162:2)  -- (234:2) -- (306:2) -- cycle; 
    \draw[thick] (18:1) --  (162:1)  -- (306:1) -- (90:1) -- (234:1) --cycle;
    \foreach \x in {18, 90, 162, 234, 306}
    \draw[thick] (\x:1) \v -- (\x:2) \v;
  \end{tikzpicture}
\end{center}

	\begin{answer}
		The chromatic numbers are 2, 3, 4, 5, and 3 respectively from left to right. %Find the chromatic number of each of the following graphs.
	\end{answer}
	
	

\question What is the smallest number of colors that can be used to color the vertices of a cube so that no two adjacent vertices are colored identically?

	\begin{answer}
		The cube can be represented as a planar graph and colored with two colors as follows:
		
		\begin{center}
		\begin{tikzpicture}
		\foreach \ang in {45, 135, 225, 315} {
		\draw (\ang:.4) \v -- (\ang:1) \v -- (\ang+90:1) (\ang:.4) -- (\ang+90:.4);
		}
		\draw (45:.4) node[right]{\tiny R} (135:.4) node[left]{\tiny B} (225:.4) node[left]{\tiny R} (315:.4) node[right]{\tiny B} (45:1) node[right]{\tiny B} (135:1) node[left]{\tiny R} (225:1) node[left]{\tiny B} (315:1) node[right]{\tiny R};
		\end{tikzpicture}
		\end{center}
		
		Since it would be impossible to color the vertices with a single color, we see that the cube has chromatic number 2 (it is bipartite).
	\end{answer}
	

\question Not all graphs are perfect.  Give an example of a graph with chromatic number 4 that does not contain a copy of $K_4$.  That is, there should be no 4 vertices all pairwise adjacent.

	\begin{answer}
		The wheel graph below has this property.  The outside of the wheel forms an odd cycle, so requires 3 colors, the center of the wheel must be different than all the outside vertices.
		
		\begin{center}
		\begin{tikzpicture}
		
		\foreach \ang in {18, 90, ..., 306}{
		\draw (0,0) -- (\ang:1) \v -- (\ang+72:1); 
		}
		\draw (0,0) \v;
		\end{tikzpicture}
		\end{center}
	\end{answer}


\question Prove by induction on vertices that any graph $G$ which contains at least one vertex of degree less than $\Delta(G)$ (the maximal degree of all vertices in $G$) has chromatic number at most $\Delta(G)$.

	\begin{answer}
		\begin{proof}
		Let $G$ be a graph with $n$ vertices, maximal degree $\Delta(G)$ and at least one vertex of degree less than $\Delta(G)$.  Assume for the sake of induction that all graphs $G'$ with fewer than $n$ vertices and a vertex of degree less than $\Delta(G')$ have chromatic number less than $\Delta(G')$.
		
		Find a vertex of $G$ with degree less than $\Delta(G)$ and remove it.  This forms a subgraph $G'$ which has $n-1$ vertices.  Also, since we removed edges, we know that $\Delta(G') \le \Delta(G)$.  If these maximal degrees are equal, then $G'$ must also have a vertex of degree less than $\Delta(G')$, since at least one of its vertices had one more edge in $G$.  In this case, we can apply our inductive hypothesis to produce a coloring of the vertices of $G'$ using just $\Delta(G)$ colors.  If $\Delta(G') < \Delta(G)$, then we also can easily find a proper coloring of the vertices of $G'$ using just $\Delta(G)$ colors by starting with any vertex and coloring it and all of its neighbors differently, and then fanning out.  Thus $G'$ has a proper vertex coloring using just $\Delta(G)$-many colors.
		
		Now move back to $G$.  Put the removed vertex back into the graph.  Since it is adjacent to at most $\Delta(G) - 1$ other vertices, there will be one of the $\Delta(G)$ colors that is not present among its neighbors, which we could use to color the newly inserted vertex.
		\end{proof}
	\end{answer}
	
	
\question You have a set of magnetic alphabet letters (one of each of the 26 letters in the alphabet) that you need to put into boxes.  For obvious reasons, you don't want to put two consecutive letters in the same box.  What is the fewest number of boxes you need (assuming the boxes are able to hold as many letters as they need to)?

	\begin{answer}
		If we drew a graph with each letter representing a vertex, and each edge connecting two letters that were consecutive in the alphabet, we would have a graph containing two vertices of degree 1 (A and Z) and the remaining 24 vertices all of degree 2 (for example, $D$ would be adjacent to both $C$ and $E$).  By Brooks' theorem, this graph has chromatic number at most 2, as that is the maximal degree in the graph and the graph is not a complete graph or odd cycle.  Thus only two boxes are needed.
	\end{answer}


\question Prove that if you color every edge of $K_6$ either red or blue, you are guaranteed a monochromatic triangle.

	\begin{answer}
		\begin{proof}
		Start with a single vertex of $K_6$, call it $v_0$.  There are five edges incident to $v_0$, and by the pigeonhole principle, three of these must be colored identically.  Without loss of generality, say these edges are $(v_0, v_1)$, $(v_0, v_2)$ and $(v_0,v_3)$, and are colored red.  Consider the edges $(v_1,v_2)$, $(v_2,v_3)$, and $(v_3, v_1)$.  If any of these are colored red, we would have a monochromatic red triangle (it plus the two edges incident to $v_0$).  If they are all colored blue, then we have a monochromatic blue triangle (those three edges).  Either way, we are guaranteed to monochromatic triangle.
		
		\end{proof}
	\end{answer}	

\end{questions}