\begin{questions}

\question\label{tt} Complete a truth table for the statement $\neg P \imp (Q \wedge R)$

  \begin{answer}
    \begin{tabular}{c|c|c||c}
                     $P$&$Q$&$R$& $\neg P \imp (Q \wedge R)$ \\ \hline
                     T & T & T & T\\
                     T & T & F & T\\
                     T & F & T & T\\
                     T & F & F & T \\
                     F & T & T & T\\
                     F & T & F & F\\
                     F & F & T & F\\
                     F & F & F & F
                    \end{tabular}
  \end{answer}

  
\question Suppose you know that the statement ``if Peter is not tall, then Quincy is fat and Robert is skinny'' is \underline{false}.  What, if anything, can you conclude about Peter and Robert if you know that Quincy is indeed fat?  Explain (you may reference problem \ref{tt}).

  \begin{answer}
    Peter is not tall and Robert is not skinny.  You must be in row 6 in the truth table above.
  \end{answer}


\question Are the statement $P \imp (Q \vee R)$ and $(P \imp Q) \vee (P \imp R)$ logically equivalent?  Explain your answer.

  \begin{answer}
    Yes.  To see this, make a truth table for each statement and compare.
  \end{answer}


  
\question Is the following a valid deduction rule: \begin{tabular}{rl} & $P \imp Q$ \\ & $P\imp R$ \\ \hline $\therefore$ & $P \imp (Q \wedge R)$.\end{tabular}  Explain.

  \begin{answer}
    Make a truth table that includes all three statements in the argument:
    
    \begin{tabular}{c|c|c||c|c|c}
     $P$ & $Q$ & $R$ & $P \imp Q$ & $P \imp R$ & $P \imp (Q \wedge R)$ \\ \hline
      T  &  T  &  T  &      T     &      T     &   T \\
      T  &  T  &  F  &      T     &      F     &   F \\
      T  &  F  &  T  &      F     &      T     &   F \\
      T  &  F  &  F  &      F     &      F     &   F \\
      F  &  T  &  T  &      T     &      T     &   T \\
      F  &  T  &  F  &      T     &      T     &   T \\
      F  &  F  &  T  &      T     &      T     &   T \\
      F  &  F  &  F  &      T     &      T     &   T 
    \end{tabular}
  
  Notice that in every row for which both $P \imp Q$ and $P \imp R$ is true, so is $P \imp (Q \wedge R)$.  Therefore, whenever the premises of the argument are true, so is the conclusion.  In other words, the deduction rule is valid.
  \end{answer}


\question Consider the statement: for all integers $n$, if $n$ is even and $n \le 7$ then $n$ is negative or $n \in \{0,2,4,6\}$.
\begin{parts}
 \part Is the statement true?  Explain why.
 \part Write the negation of the statement.  Is it true?  Explain.
 \part State the contrapositive of the statement.  Is it true?  Explain.
 \part State the converse of the statement.  Is it true?  Explain.
\end{parts}

  \begin{answer}
      \begin{parts}
	% Is the statement true?  Explain why.
	\part The statement is true.  If $n$ is an even integer less than or equal to 8, then the only way it could not be negative is if $n$ was equal to 0, 2, 4, or 6.
	% Write the negation of the statement.  Is it true?  Explain.
	\part There is an integer $n$ such that $n$ is even and $n \le 7$ but $n$ is not negative and $n \not\in \{0,2,4,6\}$.  This is false, since the original statement is true.
	%  State the contrapositive of the statement.  Is it true?  Explain.
	\part For all integers $n$, if $n$ is not negative and $n \not\in\{0,2,4,6\}$ then $n$ is odd or $n > 7$.  This is true, since the contrapositive is equivalent to the original statement (which is true).
	%  State the converse of the statement.  Is it true?  Explain.
	\part For all integers $n$, if $n$ is negative or $n \in \{0,2,4,6\}$ then $n$ is even and $n \le 7$.  This is false.  $n = -3$ is a counter-example.
      \end{parts}
  \end{answer}


\question Consider the statement: $\forall x (\forall y (x + y = y) \imp \forall z (x\cdot z = 0))$
\begin{parts}
   \part Explain what the statement says in words.  Is this statement true?  Be sure to state what you are taking the universe of discourse to be.
   \part Write the converse of the statement, both in words and in symbols.  Is the converse true?
   \part Write the contrapositive of the statement, both in words and in symbols.  Is the contrapositive true?
   \part Write the negation of the statement, both in words and in symbols.  Is the negation true?
\end{parts}

  \begin{answer}
      \begin{parts}
	\part For any number $x$, if it is the case that adding any number to $x$ gives that number back, then multiplying any number by $x$ will give 0.  This is true (of the integers or the reals) - the ``if'' part only holds if $x = 0$, and in that case, anything times $x$ will be 0.
	\part The converse in words is this: for any number $x$, if everything times $x$ is zero, then everything added to $x$ gives itself.  Or in symbols: $\forall x (\forall z (x \cdot z = 0) \imp \forall y (x + y = y))$.  The converse is true - the only number which when multiplied by any other number gives 0 is $x = 0$.  And if $x = 0$, then $x + y = y$.
	\part The contrapositive in words is: for any number $x$, if there is some number which when multiplied by $x$ does not give zero, then there is some number which when added to $x$ does not give that number.  In symbols: $\forall x (\exists z (x\cdot z \ne 0) \imp \exists y (x + y \ne y))$.  We know the contrapositive must be true because the original implication is true.
	\part The negation: there is a number $x$ such that any number added to $x$ gives the number back again, but there is a number you can multiply $x$ by and not get 0.  In symbols: $\exists x (\forall y (x + y = y) \wedge \exists z (x \cdot z \ne 0))$.  Of course since the original implication is true, the negation is false.
      \end{parts}
  \end{answer}


  
\question Write each of the following statements in the form, ``if \ldots, then \ldots.''  Careful, some of the statements might be false (which is alright for the purposes of this question).
\begin{parts}
  \part To loose weight, you must exercise.
  \part To loose weight, all you need to do is exercise.
  \part Every American is patriotic.
  \part You are patriotic only if you are American.
  \part The set of rational numbers is a subset of the real numbers.
  \part A number is prime if it is not even.
  \part Either the Broncos will win the Super Bowl, or they won't play in the Super Bowl.
\end{parts}

  \begin{answer}
      \begin{parts}
	  \part If you have lost weight, then you exercised.
	  \part If you exercise, then you will lose weight.
	  \part If you are American, then you are patriotic.
	  \part If you are patriotic, then you are American.
	  \part If a number is rational, then it is real.
	  \part If a number is not even, then it is prime.  (Or the contrapositive: if a number is not prime, then it is even.)
	  \part If the Broncos don't win the Super Bowl, then they didn't play in the Super Bowl.  Alternatively, if the Broncos play in the Super Bowl, then they will win the Super Bowl.
      \end{parts}
  \end{answer}


\question Simplify the following.
\begin{parts}
 \part $\neg (\neg (P \wedge \neg Q) \imp \neg(\neg R \vee \neg(P \imp R)))$
 \part $\neg \exists x \neg \forall y \neg \exists z (z = x + y \imp \exists w (x - y = w))$
\end{parts}

  \begin{answer}
      \begin{parts}
	    % $\neg (\neg (P \wedge \neg Q) \imp \neg(\neg R \vee \neg(P \imp R)))$
	    \part $(\neg P \vee Q) \wedge (\neg R \vee (P \wedge \neg R))$
	    %  $\neg \exists x \neg \forall y \neg \exists z (z = x + y \imp \exists w (x - y = w))$
	    \part $\forall x \forall y \forall z (z = x+y \wedge \forall w (x-y \ne w))$
      \end{parts}
  \end{answer}


\question Suppose you break your piggy bank and scoop up a handful of 22 coins (pennies, nickels, dimes and quarters).
\begin{parts}
\part Prove that you must have at least 6 coins of a single denomination.
\part Suppose you have an odd number of pennies.  Prove that you must have an odd number of at least one of the other types of coins.
\part How many coins would you need to scoop up to be sure that you either had 4 coins that were all the same or 4 coins that were all different?  Prove your answer.
\end{parts}

	\begin{answer}
		\begin{parts}
		\part Suppose you only had 5 coins of each denomination.  This means you have 5 pennies, 5 nickels, 5 dimes and 5 quarters.  This is a total of 20 coins.  But you have more than 20 coins, so you must have more than 5 of at least one type.
		\part Suppose you have 22 coins, including $2k$ nickels, $2j$ dimes, and $2l$ quarters (so an even number of each of these three types of coins).  The number of pennies you have will then be
		\[22 - 2k - 2j - 2l = 2(11-k-j-l)\]
		But this says that the number of pennies is also even (it is 2 times an integer).  Thus we have established the contrapositive of the statement, ``If you have an odd number of pennies then you have an odd number of at least one other coin type.''
		\part You need 10 coins.  You could have 3 pennies, 3 nickels, and 3 dimes.  The 10th coin must either be a quarter, giving you 4 coins that are all different, or else a 4th penny, nickel or dime.  To prove this, assume you don't have 4 coins that are all the same or all different.  In particular, this says that you only have 3 coin types, and each of those types can only contain 3 coins, for a total of 9 coins, which is less than 10.
		\end{parts}
	\end{answer}


\end{questions}