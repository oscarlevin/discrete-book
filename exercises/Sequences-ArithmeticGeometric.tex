\begin{questions}
\question Consider the sequence $8, 14, 20, 26, \ldots $.
\begin{parts}
\part What is the next term in the sequence?
\part Find a formula for the $n$th term of this sequence, assuming $a_1 = 8$.
\part Find the sum of the first 100 terms of the sequence: $\sum_{k=1}^{100}a_k$.
\end{parts}

	\begin{answer}
		\begin{parts}
		% What is the next term in the sequence?
		\part 32.
		% Find a formula for the $n$th term of this sequence, assuming $a_1 = 8$.
		\part $a_n = 8 + 6(n-1)$.
		% Find the sum of the first 100 terms of the sequence: $\sum_{k=1}^{100}a_k$.
		\part $30500$.
		\end{parts}
	\end{answer}





\question Consider the sequence $1, 7, 13, 19, \ldots, 6n + 7$.
\begin{parts}
\part How many terms are there in the sequence?
\part What is the second-to-last term?
\part Find the sum of all the terms in the sequence.
\end{parts}

	\begin{answer}
		\begin{parts}
		% How many terms are there in the sequence?
		\part $n+2$ terms.
		\part $6n+1$. %second to last term
		%Find the sum of all the terms in the sequence.
		\part $\frac{(6n+8)(n+2)}{2}$.
		\end{parts}
	\end{answer}






\question Find $5 + 7 + 9 + 11+ \cdots + 521$.

	\begin{answer}
		68117.
	\end{answer}





\question Find $5 + 15 + 45 + \cdots + 5\cdot 3^{20}$.

	\begin{answer}
		$\frac{5-5\cdot 3^{21}}{-2}$.
	\end{answer}





\question Find $1 - \frac{2}{3} + \frac{4}{9} - \cdots + \frac{2^{30}}{3^{30}}$.

	\begin{answer}
		$\frac{1 + \frac{2^{31}}{3^{31}}}{5/3}$.
	\end{answer}





\question Find $x$ and $y$ such that $27, x, y, 1$ is part of an arithmetic sequence.  Then find $x$ and $y$ so that the sequence is part of a geometric sequence.  (Warning: $x$ and $y$ might not be integers.)

	\begin{answer}
		For arithmetic: $x = 55/3$, $y = 29/3$.  For geometric: $x = 9$ and $y = 3$.
	\end{answer}




\question Consider the sequence $2, 7, 15, 26, 40, 57, \ldots$ (with $a_0 = 2$).  By looking at the differences between terms, express the sequence as a sequence of partial sums.  Then find a closed formula for the sequence by computing the $n$th partial sum.

	\begin{answer}
		We have $2 = 2$, $7 = 2+5$, $15 = 2 + 5 + 8$, $26 = 2+5+8+11$, and so on.  The terms in the sums are given by the arithmetic sequence $b_n = 2+3n$.  In other words, $a_n = \sum_{k=0}^n 2+3k$.  To find this, we reverse and add.  We get $a_n = \frac{(4+3n)(n+1)}{2}$ (we have $n+1$ there because there are $n+1$ terms in the sum for $a_n$).
	\end{answer}





\question Use summation ($\sum$) or product ($\prod$) notation to rewrite the following.
\begin{parts}
  \part $2 + 4 + 6 + 8 + \cdots + 2n$.
  \part $1 + 5 + 9 + 13 + \cdots + 425$.
  \part $1 + \frac{1}{2} + \frac{1}{3} + \frac{1}{4} + \cdots + \frac{1}{50}$.
  \part $2 \cdot 4 \cdot 6 \cdot \cdots \cdot 2n$.
  \part $(\frac{1}{2})(\frac{2}{3})(\frac{3}{4})\cdots(\frac{100}{101})$.
\end{parts}

	\begin{answer}
		\begin{parts}
		  \part $\d\sum_{k=1}^n 2k$.		%$2 + 4 + 6 + 8 + \cdots + 2n$
		  \part $\d\sum_{k=1}^{107} (1 + 4(k-1))$.		%$1 + 5 + 9 + 13 + \cdots + 425$
		  \part $\d\sum_{k=1}^{50} \frac{1}{k}$.		%$1 + \frac{1}{2} + \frac{1}{3} + \frac{1}{4} + \cdots + \frac{1}{50}$
		  \part $\d\prod_{k=1}^n 2k$.		%$2 \cdot 4 \cdot 6 \cdot \cdots \cdot 2n$
		  \part $\d\prod_{k=1}^{100} \frac{k}{k+1}$.	%$(\frac{1}{2})(\frac{2}{3})(\frac{3}{4})\cdots(\frac{100}{101})$
		\end{parts}
	\end{answer}





\question Expand the following sums and products.  That is, write them out the long way.
\begin{parts}
  \part $\d\sum_{k=1}^{100} (3+4k)$.
  \part $\d\sum_{k=0}^n 2^k$.
  \part $\d\sum_{k=2}^{50}\frac{1}{(k^2 - 1)}$.
  \part $\d\prod_{k=2}^{100}\frac{k^2}{(k^2-1)}$.
  \part $\d\prod_{k=0}^n (2+3k)$.
\end{parts}

	\begin{answer}
		\begin{parts}
		  \part $\d\sum_{k=1}^{100} (3+4k) = 7 + 11 + 15 + \cdots + 403$.
		  \part $\d\sum_{k=0}^n 2^k = 1 + 2 + 4 + 8 + \cdots + 2^n$.
		  \part $\d\sum_{k=2}^{50}\frac{1}{(k^2 - 1)} = 1 + \frac{1}{3} + \frac{1}{8} + \frac{1}{15} + \cdots + \frac{1}{2499}$.
		  \part $\d\prod_{k=2}^{100}\frac{k^2}{(k^2-1)} = \frac{4}{3}\cdot\frac{9}{8}\cdot\frac{16}{15}\cdots\frac{10000}{9999}$.
		  \part $\d\prod_{k=0}^n (2+3k) = (2)(5)(8)(11)(14)\cdots(2+3n)$.
		\end{parts}
	\end{answer}



\end{questions}
