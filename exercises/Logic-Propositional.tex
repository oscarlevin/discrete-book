\begin{questions}

  
\question Consider the statement about a party, ``If it's your birthday or there will be cake, then there will be cake.''
\begin{parts}
 \part Translate the above statement into symbols.  Clearly state which statement is $P$ and which is $Q$.
 \part Make a truth table for the statement.
 \part Assuming the statement is true, what (if anything) can you conclude if there will be cake?
 \part Assuming the statement is true, what (if anything) can you conclude if there will not be cake?
 \part Suppose you found out that the statement was a lie.  What can you conclude?
\end{parts}

  \begin{answer}
    \begin{parts}
      \part $P$: it's your birthday; $Q$: there will be cake.  $(P \vee Q) \imp Q$
      \part Hint: you should get three T's and one F.
      \part Only that there will be cake.
      \part It's NOT your birthday!
      \part It's your birthday, but the cake is a lie.
    \end{parts}
  \end{answer}


  
\question Suppose $P$ and $Q$ are the statements:
$P$: Jack passed math.
$Q$: Jill passed math.
\begin{parts}
 \part Translate ``Jack and Jill both passed math'' into symbols.
\part Translate ``If Jack passed math, then Jill did not'' into symbols.
\part Translate ``$P \vee Q$'' into English.
\part Translate ``$\neg(P \wedge Q) \imp Q$'' into English.
\part Suppose you know that if Jack passed math, then so did Jill.  What can you conclude if you know that:
\begin{subparts}
 \subpart Jill passed math?  
\subpart  Jill did not pass math?
\end{subparts}
\end{parts}

  \begin{answer}
    \begin{parts}
      \part $P \wedge Q$.
      \part $P \imp \neg Q$.
      \part Jack passed math or Jill passed math (or both).
      \part If Jack and Jill did not both pass math, then Jill did.
      \part 
	\begin{subparts}
	  \subpart Nothing else. 
	  \subpart  Jack did not pass math either.
	\end{subparts}
    \end{parts}
  \end{answer}



  
\question Geoff Poshingten is out at a fancy pizza joint, and decides to order a calzone.  When the waiter asks what he would like in it, he replies, ``I want either pepperoni or sausage, and if I have sausage, I must also include quail.  Oh, and if I have pepperoni or quail then I must also have ricotta cheese.''  
\begin{parts}
	\part Translate Geoff's order into logical symbols.
	\part The waiter knows that Geoff is either a liar or a truth-teller (so either everything he says is false, or everything is true).  Which is it?
	\part What, if anything, can the waiter conclude about the ingredients in Geoff's desired calzone?
\end{parts}

  \begin{answer}
    \begin{parts}
	\part Three statements: $P \vee S$, $S \imp Q$, $(P \vee Q) \imp R$.  You could also connect the first two with a $\wedge$.
	\part He cannot be lying about all three sentences, so he is telling the truth.
	\part No matter what, Geoff wants ricotta.  If he doesn't have quail, then he must have pepperoni but not sausage.
    \end{parts}
  \end{answer}




\question Make a truth table for the statement $(P \vee Q) \imp (P \wedge Q)$. 

\begin{answer}
 \begin{tabular}{c|c|c}
             $P$ & $Q$ & $(P \vee Q) \imp (P \wedge Q)$\\ \hline
             T & T & T \\
             T & F & F \\
             F & T & F \\
             F & F & T
          \end{tabular}
\end{answer}


\question Make a truth table for the statement $\neg P \wedge (Q \imp P)$.  What can you conclude about $P$ and $Q$ if you know the statement is true?

    \begin{answer}
      \begin{tabular}{c|c|c}
             $P$ & $Q$ & $\neg P \wedge (Q \imp P)$\\ \hline
             T & T & F \\
             T & F & F \\
             F & T & F \\
             F & F & T
          \end{tabular}
	If the statement is true, then both $P$ and $Q$ are false.
    \end{answer}


\question Make a truth table for the statement $\neg P \imp (Q \wedge R)$.

  \begin{answer}
    Hint: Like above, only now you will need 8 rows instead of just 4.
  \end{answer}



\question Determine if the following deduction rule is valid: 

\centerline{\begin{tabular}{rc} & $P \vee Q$ \\ & $\neg P$ \\ \hline $\therefore$ & $Q$\end{tabular}}

  \begin{answer}
    The rule is valid.  To see this, make a truth table which contains $P \vee Q$ and $\neg P$ (and $P$ and $Q$ of course).  Look at the truth value of $Q$ in each of the rows that have $P \vee Q$ and $\neg P$ true.  
  \end{answer}

  
  
  
\question Determine if the following is a valid deduction rule: 

\centerline{\begin{tabular}{rc} & $P \imp (Q \vee R)$ \\ & $\neg(P \imp Q)$ \\ \hline $\therefore$ & $R$\end{tabular}}

  \begin{answer}
    The deduction rule is valid.  Again, make a truth table containing the premises and conclusion.  Look at the rows for which the premises are true.
  \end{answer}


  
\question Determine if the following is a valid deduction rule: 

\centerline{\begin{tabular}{rc} & $(P \wedge Q) \imp R$ \\ & $\neg P \vee \neg Q$ \\ \hline $\therefore$ & $\neg R$\end{tabular}}

  \begin{answer}
    The rule is NOT valid.  If you make a truth table containing the premises and conclusion, there will be a row with both premises true but the conclusion false.  For example, if $P$ and $Q$ are false and $R$ is true, then $P \wedge Q$ is false, so $(P \wedge Q) \imp R$ is true.  Also $\neg P$ is true, so $\neg P \vee \neg Q$ is true.  However, $\neg R$ is false.
  \end{answer}




\end{questions}