\begin{questions}




\question Let $S = \{1, 2, 3, 4, 5, 6\}$
\begin{parts}
  \part How many subsets are there total? 
  \part How many subsets contain $\{2,3,5\}$ as a subset?
%  \part How many subsets of $S$ contain no prime numbers?
  \part How many subsets contain at least one odd number? 
  \part How many subsets contain exactly one even number?
\end{parts}

  \begin{answer}
    \begin{parts}
      \part $2^6 = 64$  %How many subsets are there total? 
      \part $2^3 = 8$.  We need to select yes/no for each of the remaining three elements.  %How many subsets contain $\{2,3,5\}$ as a subset?
%      \part $2^3 = 8$.  We need to decide yes/no for the three non-prime elements.  %How many subsets of $S$ contain no prime numbers?
      \part $2^6 - 2^3 = 56$.  There are 8 subsets which do not contain any odd numbers. %How many subsets contain at least one odd number? 
      \part $3\cdot 2^3 = 24$.  First pick the even number.  Then say yes or no to each of the odd numbers.
    \end{parts}
  \end{answer}

\question Let $S = \{1, 2, 3, 4, 5, 6\}$
\begin{parts}
  \part How many subsets are there of cardinality 4? 
  \part How many subsets of cardinality 4 contain $\{2,3,5\}$ as a subset?
%  \part How many subsets of $S$ contain no prime numbers?
  \part How many subsets of cardinality 4 contain at least one odd number? 
  \part How many subsets of cardinality 4 contain exactly one even number?
\end{parts}

  \begin{answer}
    \begin{parts}
      \part ${6\choose 4} = 15$  
      \part ${3 \choose 1} = 3$.  We need to select 1 of the 3 remaining elements to be in the subset.  

      \part ${6 \choose 4} = 15$.  All subsets of cardinality 4 must contain at least one odd number.
      \part ${3 \choose 1} = 3$.  Select 1 of the 3 even numbers.  The remaining three odd numbers of $S$ must all be in the set.
    \end{parts}
  \end{answer}
  
  
  
  
\question You break your piggy-bank to discover lots of pennies and nickels.  You start arranging these in rows of 6 coins.

\begin{parts}
\part You find yourself making rows containing and equal number of pennies and nickels.  For fun, you decide to lay out every possible different such row.  How many coins will you need?

\part How many coins would you need to make all possible different rows of 6 coins (not necessarily with equal number of pennies and nickels)?
\end{parts}

	\begin{answer}
	\part We can think of each row as a 6-bit string of weight 3 (since of the 6 coins, we require 3 to be pennies).  Thus there are ${6 \choose 3} = 20$ rows possible.  Each row requires 6 coins, so if we want to make all the rows at the same time, we will need 120 coins (60 of each).
	
	\part Now there are $2^6 = 64$ rows possible, which is also ${6 \choose 0} + {6\choose 1} + {6 \choose 2} + {6 \choose 3} + {6 \choose 4} + {6 \choose 5} + {6 \choose 6}$.  Thus we need $6 \cdot 64 = 384$ coins (192 of each).
	\end{answer}




\question How many 10-bit strings contain 6 or more 1's?

  \begin{answer}
    ${10 \choose 6} + {10\choose 7} + {10\choose 8} + {10 \choose 9} + {10\choose 10} = 386$ 
  \end{answer}

\question How many subsets of $\{0,1,\ldots, 9\}$ have cardinality 6 or more?

	\begin{answer}
	${10 \choose 6} + {10\choose 7} + {10\choose 8} + {10 \choose 9} + {10\choose 10} = 386$.  This is the same as the previous question, since we can think of each subset as a 10-bit string with a 1 representing that we include that element in the subset.
	\end{answer}



\question What is the coefficient of $x^{12}$ in $(x+2)^{15}$?

	\begin{answer}
		To get an $x^{12}$, we must pick 12 of the 15 factors to contribute an $x$, leaving the other 3 to contribute a 2.  There are ${15 \choose 12}$ ways to select these 12 factors.  So the term containing an $x^{12}$ will be ${15 \choose 12}x^{12}2^{3}$.  In other words the coefficient of $x^{12}$ is ${15\choose 12}2^3$.
	\end{answer}



\question What is the coefficient of $x^9$ in the expansion of $(x+1)^{14} + x^3(x+2)^{15}$?

  \begin{answer}
    Use the binomial theorem.  ${14\choose 9} + {15 \choose 6}2^9$.
  \end{answer}




   
\question How many shortest lattice paths start at (3,3) and
\begin{parts}
  \part end at (10,10)?
  \part end at (10,10) and pass through (5,7)?
  \part end at (10,10) and avoid (5,7)?
\end{parts}   

  \begin{answer}
    \begin{parts}
      \part ${14 \choose 7}$ %end at (10,10)?
      \part ${6 \choose 2}{8\choose 5}$ %end at (10,10) and pass through (5,7)?
      \part ${14 \choose 7} - {6\choose 2}{8 \choose 5}$ %end at (10,10) and avoid (5,7)?
    \end{parts} 
  \end{answer}



\question Suppose you are ordering a large pizza from \emph{D.P.\ Dough}.  You want 3 distinct toppings, chosen from their list of 11 vegetarian toppings.

\begin{parts}
\part How many choices do you have for your pizza?
\part How many choices do you have for your pizza if you refuse to have pineapple as one of your toppings?
\part How many choice do you have for your pizza if you \emph{insist} on having pineapple as one of your toppings?
\part How do the three questions above relate to each other?
\end{parts}


	\begin{answer}
	 \part ${11 \choose 3} = 165$, since you have to select a 3-element subset of the set of 11 toppings.
	 \part ${10 \choose 3} = 120$, since you must select 3 of the 10 non-pineapple toppings.
	 \part ${10 \choose 2} = 45$, since you must select 2 of the remaining 10 non-pineapple toppings to have in addition to the pineapple.
	 \part $165  = 120 + 45$, which makes sense because every 3-topping pizza either has pineapple or does not have pineapple as a topping.
	\end{answer}
	

\question Explain why the coefficient of $x^5y^3$ the same as the coefficient of $x^3y^5$ in the expansion of $(x+y)^8$?

	\begin{answer}
		The coefficient of $x^5y^3$ is ${8\choose 5}$, since we must pick 5 of the 8 factors to contribute an $x$.  The coefficient of $x^3y^5$ is ${8 \choose 3}$, since we pick 3 out of the 8 factors to contribute an $x$.  But ${8 \choose 5} = {8\choose 3}$, because we could just as easily have picked 5 out of the 8 factors to contribute a $y$.
	\end{answer} 
\end{questions}