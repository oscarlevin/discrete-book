\begin{questions}

\question Determine whether the following two statements are logically equivalent: $\neg(P \imp Q)$ and $P \wedge \neg Q$.  Explain how you know you are correct.

  \begin{answer}
    Make a truth table for each and compare.  The statements are logically equivalent.
  \end{answer}




\question Are the statements $P \imp (Q\vee R)$ and $(P \imp Q) \vee (P \imp R)$ logically equivalent?

  \begin{answer}
    Again, make two truth tables.  The statements are logically equivalent.
  \end{answer}




\question Consider the statement ``If Oscar eats Chinese food, then he drinks milk.''
\begin{parts}
 \part Write the converse of the statement.
 \part Write the contrapositive of the statement.
 \part Is it possible for the contrapositive to be false?  If it was, what would that tell you?
 \part Suppose the original statement is true, and that Oscar drinks milk.  Can you conclude anything (about his eating Chinese food)?  Explain.
 \part Suppose the original statement is true, and that Oscar does not drink milk.  Can you conclude anything (about his eating Chinese food)?  Explain.
\end{parts}

  \begin{answer}
    \begin{parts}
      \part If Oscar drinks milk, then he eats Chinese food.
      \part If Oscar does not drink milk, then he does not eat Chinese food.
      \part Yes.  The original statement would be false too.
      \part Nothing. The converse need not be true.
      \part He does not eat Chinese food. The contrapositive would be true.
    \end{parts}
  \end{answer}




\question Simplify the following statements (so that negation only appears right before variables).
\begin{parts}
  \part $\neg(P \imp \neg Q)$.
  \part $(\neg P \vee \neg Q) \imp \neg (\neg Q \wedge R)$.
  \part $\neg((P \imp \neg Q) \vee \neg (R \wedge \neg R))$.
  \part It is false that if Sam is not a man then Chris is a woman, and that Chris is not a woman.
\end{parts}

  \begin{answer}
    \begin{parts}
      \part $P \wedge Q$.
      \part $(P \wedge Q) \vee (Q \vee \neg R)$.
      \part F.  Or $(P \wedge Q) \wedge (R \wedge \neg R)$.
      \part Either Sam is a woman and Chris is a man, or Chris is a woman.
    \end{parts}
  \end{answer}



\question Which of the following statements are equivalent to the implication, ``if you win the lottery, then you will be rich,'' and which are equivalent to the converse of the implication?
\begin{parts}
 \part Either you win the lottery or else you are not rich.
 \part Either you don't win the lottery or else you are rich.
 \part You will win the lottery and be rich.
 \part You will be rich if you win the lottery.
 \part You will win the lottery if you are rich.
 \part It is necessary for you to win the lottery to be rich.

 \part It is sufficient to with the lottery to be rich.
 \part You will be rich only if you win the lottery.
 \part Unless you win the lottery, you won't be rich.
 \part If you are rich, you must have one the lottery.
 \part If you are not rich, then you did not win the lottery.
 \part You will win the lottery if and only if you are rich.
\end{parts}

  \begin{answer} The statements are equivalent to the\ldots
    \begin{parts}
      \part converse.
      \part implication.
      \part neither.
      \part implication.
      \part converse.
      \part converse.

      \part implication.
      \part converse.
      \part converse.
      \part converse (in fact, this IS the converse).
      \part implication (the statement is the contrapositive of the implication).
      \part neither.
    \end{parts}
  \end{answer}


\question Consider the implication, ``if you clean your room, then you can watch TV.''  Rephrase the implication in as many ways as possible.  Then do the same for the converse.

  \begin{answer}
    Hint: of course there are many answers.  It helps to assume that the statement is true and the converse is NOT true.  Think about what that means in the real world and then start saying it in different ways.  Some ideas: use necessary and sufficient language, use ``only if,'' consider negations, use ``or else'' language.
  \end{answer}



\end{questions}
