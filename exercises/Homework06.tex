\documentclass[11pt]{exam}

\usepackage{amssymb, amsmath, amsthm, mathrsfs, multicol, graphicx} 
\usepackage{tikz}

\def\d{\displaystyle}
\def\?{\reflectbox{?}}
\def\b#1{\mathbf{#1}}
\def\f#1{\mathfrak #1}
\def\c#1{\mathcal #1}
\def\s#1{\mathscr #1}
\def\r#1{\mathrm{#1}}
\def\N{\mathbb N}
\def\Z{\mathbb Z}
\def\Q{\mathbb Q}
\def\R{\mathbb R}
\def\C{\mathbb C}
\def\F{\mathbb F}
\def\A{\mathbb A}
\def\X{\mathbb X}
\def\E{\mathbb E}
\def\O{\mathbb O}
\def\pow{\mathscr P}
\def\inv{^{-1}}
\def\nrml{\triangleleft}
\def\st{:}
\def\~{\widetilde}
\def\rem{\mathcal R}
\def\iff{\leftrightarrow}
\def\Iff{\Leftrightarrow}
\def\and{\wedge}
\def\And{\bigwedge}
\def\AAnd{\d\bigwedge\mkern-18mu\bigwedge}
\def\Vee{\bigvee}
\def\VVee{\d\Vee\mkern-18mu\Vee}
\def\imp{\rightarrow}
\def\Imp{\Rightarrow}
\def\Fi{\Leftarrow}

\def\={\equiv}
\def\var{\mbox{var}}
\def\mod{\mbox{Mod}}
\def\Th{\mbox{Th}}
\def\sat{\mbox{Sat}}
\def\con{\mbox{Con}}
\def\bmodels{=\joinrel\mathrel|}
\def\iffmodels{\bmodels\models}
\def\dbland{\bigwedge \!\!\bigwedge}
\def\dom{\mbox{dom}}
\def\rng{\mbox{range}}
\DeclareMathOperator{\wgt}{wgt}

\def\circleA{(-.5,0) circle (1)}
\def\circleAlabel{(-1.5,.6) node[above]{$A$}}
\def\circleB{(.5,0) circle (1)}
\def\circleBlabel{(1.5,.6) node[above]{$B$}}
\def\circleC{(0,-1) circle (1)}
\def\circleClabel{(.5,-2) node[right]{$C$}}
\def\twosetbox{(-2,-1.5) rectangle (2,1.5)}
\def\threesetbox{(-2,-2.5) rectangle (2,1.5)}


\def\bar{\overline}

%\pointname{pts}
\pointsinmargin
\marginpointname{pts}
\marginbonuspointname{pts-bns}
\addpoints
\pagestyle{head}
%\printanswers

\firstpageheader{Math 228}{\bf Homework 6}{Due: Wed, March 6}

\def\vertexsize{4pt}
\newcommand{\vtx}[2]{node[fill,circle,inner sep=0pt, minimum size=\vertexsize,label=#1:#2]{}}
\newcommand{\va}[1]{\vtx{above}{#1}}
\newcommand{\vb}[1]{\vtx{below}{#1}}
\newcommand{\vr}[1]{\vtx{right}{#1}}
\newcommand{\vl}[1]{\vtx{left}{#1}}
\renewcommand{\v}{\vtx{above}{}}

\begin{document}
\noindent \textbf{Instructions}: Same rules as usual - turn in your work on separate sheets of paper.  You must justify all your answers for full credit.

\begin{questions}

\question[8] Gridtown USA, besides having excellent donut shoppes, is known for its precisely laid out grid of streets and avenues.  Streets run east-west, and avenues north-south, for the entire stretch of the town, never curving and never interrupted by parks or schools or the like.

Suppose you live on the corner of 1st and 1st and work on the corner of 12th and 12th.  Thus you must travel 22 blocks to get to work as quickly as possible.

\begin{parts}
\part How many different routes can you take to work, assuming you want to get there as quickly as possible?
\begin{solution}
  ${22 \choose 11}$ since you must choose 11 of the 22 blocks to travel east.
\end{solution}

\part Now suppose you want to stop and get a donut on the way to work, from your favorite donut shoppe on the corner of 8th st and 10th ave.  How many routes to work, via the donut shoppe, can you take (again, ensuring the shortest possible route)?
\begin{solution}
  The donut shoppe is 16 blocks away, 7 one way, 9 the other.  So to get from home to the donut shoppe, there are ${16 \choose 7}$ routes (or equivalently, ${16 \choose 9}$).  Then from the donut shopped to work, you need to travel 6 more blocks, 2 on way and 4 the other.  So there are ${6 \choose 2}$ (or ${6 \choose 4}$) routes from the donut shoppe to work.  
  
  For each of the ways to the donut shoppe, there are so many ways to work, so the multiplicative principle says the total number of ways from home to work via the donut shoppe is
  \[{16 \choose 7}{6 \choose 2}\]
\end{solution}

\part Disaster Strikes Gridtown: there is a pothole on 4th avenue between 5th and 6th street.  How many routes to work can you take avoiding that unsightly (and dangerous) stretch of road?
\begin{solution}
  Routes to work that hit the pothole: ${7 \choose 3}1{14 \choose 8}$.
  
  There for the number of routes to work which {\em avoid} the pothole are
  \[{22 \choose 11} - {7 \choose 3}{14 \choose 8}\]
\end{solution}

\part How many routes are there both avoiding the pothole and visiting the donut shoppe?
\begin{solution}
  First compute the number of routes to the donut shoppe avoiding the pothole:
  \[{16 \choose 7} - {7 \choose 3}{8 \choose 6}\]
  Then you still need to go to work from there.  Thus the answer is:
  \[\left({16 \choose 7} - {7 \choose 3}{8 \choose 6}\right){6 \choose 2}\]
\end{solution}

\end{parts}


\question[6] Suppose you own $a$ fezzes and $b$ bow ties.  Of course, $a$ and $b$ are both greater than 1.
\begin{parts}
  \part How many combinations of fez and bow tie can you make?  You can wear only one fez and one bow tie at a time.  Explain.
  \begin{solution}
    You have $a$ choices for the fez, and for each choice of fez you have $b$ choices for the bow tie.  Thus you have $a \cdot b$ choices for fez and bow tie combination.
  \end{solution}

  \part Explain why the answer is {\em also} ${a+b \choose 2} - {a \choose 2} - {b \choose 2}$.  (If this is what you claimed the answer was in part (a), try it again.)
  \begin{solution}
    Line up all $a+b$ quirky clothing items - the $a$ fezzes and $b$ bow ties.  Now pick 2 of them.  This can be done in ${a+b \choose 2}$ ways.  However, we might have picked 2 fezzes, which is not allowed.  There are ${a \choose 2}$ ways to pick 2 fezzes.  Similarly, the ${a+b \choose 2}$ ways to pick two items includes ${b \choose 2}$ ways to select 2 bow ties, also not allowed.  Thus the total number of ways to pick a fez and a bow ties is
    \[{a+b \choose 2} - {a \choose 2} - {b \choose 2}\]
  \end{solution}

  \part Use your answers to parts (a) and (b) to give a combinatorial proof of the identity
  \[{a+b \choose 2} - {a \choose 2} - {b \choose 2} = ab\]
  \begin{solution}
  \begin{proof}
       The question is how many ways can you select one of $a$ fezzes and one of $b$ bow ties.  We answer this question in two ways.  First, the answer could be $a\cdot b$. This is correct as described in part (a) above.  Second, the answer could be ${a+b \choose 2} - {a \choose 2} - {b \choose 2}$.  This is correct as described in part (b) above.  Therefore 
    \[{a+b \choose 2} - {a \choose 2} - {b \choose 2} = ab\]
  \end{proof}
  \end{solution}

\end{parts}


\question Consider the identity:
\[k{n\choose k} = n{n-1 \choose k-1}\]
\begin{parts}
  \part[2] Is this true?  Try it for a few values of $n$ and $k$.
  \begin{solution}
    Yes.  For example, if $n = 7$ and $k = 4$, we have \[4 \cdot {7 \choose 4} = 4 \cdot 35 = 140 = 7 \cdot 20 = 7 \cdot {6 \choose 3}\]
  \end{solution}

  \part[2] Use the formula for ${n \choose k}$ to give an algebraic proof of the identity.
  \begin{solution}
    \[k{n \choose k} = k \frac{n!}{(n-k)!\,k!} = \frac{n!}{(n-k)!(k-1)!} = n\frac{(n-1)!}{(n-1-(k-1))!(k-1)!} = n {n-1 \choose k-1}\]
  \end{solution}

  \part[4] Give a combinatorial proof of the identity. Hint: How many ways can you select a chaired committee of $k$ people from a group of $n$ people?  
  \begin{solution}
    \begin{proof}
      Question: How many ways can you select a chaired committee of $k$ people from a group of $n$ people?  That is, you need to select $k$ people to be on the committee and one of them needs to be in charge.  How many ways can this happen?
      
      Answer 1: First select $k$ of the $n$ people to be on the committee.  This can be done in ${n \choose k}$ ways.  Now select one of those $k$ people to be in charge - this can be done in $k$ ways.  So there are a total of $k {n \choose k}$ ways to select the chaired committee.
      
      Answer 2: First select the chair of the committee.  You have $n$ people to choose from, so this can be done in $n$ ways.  Now fill the rest of the committee.  There are $n-1$ people to choose from (you cannot select the person you picked to be the chair) and $k-1$ spots to fill (the chair's spot is already taken).  So this can be done in ${n-1 \choose k-1}$ ways.  Therefore there are $n{n-1 \choose k-1}$ ways to select the chaired committee.
    \end{proof}

  \end{solution}

\end{parts}

\vfill

\uplevel{{\bf Writing Assignment:} Please turn in the writing assignment below separately from the rest of your homework, NOT stapled to the other problems.}

\question[8] In this writing assignment, I would like you to reflect on how you solve counting problems, and how you decide if your answer is correct.  To do this, consider the following counting problem:

\begin{quote}
	How many 10-digit numbers contain exactly four 1's, three 2's, two 3's and one 4?
\end{quote} 

Write about how you personally go about answering this question.  I am more interested in your process than in the correct answer (although a correct answer will be worth 2 bonus points).  Tell me about what you think about when you see the question, what your first guess is, and how you decide whether your answer is correct.  This should take 1-2 pages.


\end{questions}




\end{document}


