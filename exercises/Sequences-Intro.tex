\begin{questions}

\question Find the closed formula for each of the following sequences by relating them to a well know sequence.  Assume the first term given is $a_1$.
\begin{parts}
  \part $2, 5, 10, 17, 26, \ldots$
  \part $0, 2, 5, 9, 14, 20, \ldots$
  \part $8, 12, 17, 23, 30,\ldots$
  \part $1, 5, 23, 119, 719,\ldots$
\end{parts}

	\begin{answer}
		\begin{parts}
		%$2, 5, 10, 17, 26, \ldots$
		  \part $a_n = n^2 + 1$.
		%   $0, 2, 5, 9, 14, 20, \ldots$
		  \part $a_n = \frac{n(n+1)}{2} - 1$.
		%   $8, 12, 17, 23, 30,\ldots$
		  \part $a_n = \frac{(n+2)(n+3)}{2} + 2$.
		%   $1, 5, 23, 119, 719,\ldots$
		  \part $a_n = (n+1)! - 1$ (where $n! = 1 \cdot 2 \cdot 3 \cdots n$).
		\end{parts}
	\end{answer}
	
	
	

\question The Fibonacci sequence is $0, 1, 1, 2, 3, 5, 8, 13, \ldots$ (where $F_0 = 0$).\index{Fibonacci sequence}
\begin{parts}
  \part Give the recursive definition for the sequence.
  \part Write out the first few terms of the sequence of partial sums.  
  \part Give a closed formula for the sequence of partial sums in terms of \gls{Fn}  (for example, you might say $F_0 + F_1 + \cdots + F_n = 3F_{n-1}^2 + n$, although that is definitely not correct).
\end{parts}

	\begin{answer}
		\begin{parts}
		%Give the recursive definition for the sequence.
		  \part $F_n = F_{n-1} + F_{n-2}$ with $F_0 = 0$ and $F_1 = 1$.
		%   Write out the first few terms of the sequence of partial sums. 
		  \part  $0, 1, 2, 4, 7, 12, 20, \ldots.$
		  %Give a closed formula for the sequence of partial sums in terms of $F_n$  (for example, you might say $F_0 + F_1 + \cdots + F_n = 3F_{n-1}^2 + n$, although that is definitely not correct).
		  \part $F_0 + F_1 + \cdots + F_n = F_{n+2} - 1.$ 
		\end{parts}
	\end{answer}
	


\question Consider the three sequences below.  For each, find a recursive definition.  How are these sequences related?

\begin{parts}
	\part $2, 4, 6, 10, 16, 26, 42, \ldots$.
	\part $5, 6, 11, 17, 28, 45, 73, \ldots$.
	\part $0, 0 , 0 , 0 , 0 , 0 , 0 ,\ldots$.
\end{parts}	

	\begin{answer}
		The sequences all have the same recurrence relation: $a_n = a_{n-1} + a_{n-2}$ (the same as the Fibonacci numbers).  The only difference is the initial conditions.
	\end{answer}
	


\question Write out the first few terms of the sequence given by $a_1 = 3$; $a_n = 2a_{n-1} + 4$.  Then find a recursive definition for the sequence $10, 24, 52, 108, \ldots$.

	\begin{answer}
		$3, 10, 24, 52, 108,\ldots$.  The recursive definition for $10, 24, 52, \ldots$ is $a_n = 2a_{n-1} + 4$ with $a_1 = 10$.
	\end{answer}
	
	
	


\question Write out the first few terms of the sequence given by $a_n = n^2 - 3n + 1$.  Then find a closed formula for the sequence (starting with $a_1$) $0, 2, 6, 12, 20, \ldots$.

	\begin{answer}
		$-1, -1, 1, 5, 11, 19,\ldots$  Thus the sequence $0, 2, 6, 12, 20,\ldots$ has closed formula $a_n = (n+1)^2 - 3(n+1) + 2$.
	\end{answer}
	
	
\question Find a closed formula for the sequence with recursive definition $a_n = 2a_{n-1} - a_{n-2}$ with $a_1 = 1$ and $a_2 = 2$.

	\begin{answer}
		Write out the first few terms of the sequence: $1, 2, 3, 4, 5, 6,\ldots$.  This is surprising at first, but note that we could write $2a_{n-1} - a_{n-2} = a_{n-1} + (a_{n-1} -a_{n-2})$, and $a_{n-1} - a_{n-2}$ is just the difference between the terms.  Initially, the difference between terms is 1, so each time we are just adding one.  So we see that $a_n = n$ is the closed formula.  
	\end{answer}
	

\question Find a recursive definition for the sequence with closed formula $a_n = 3 + 2n$.  Bonus points if you can give a recursive definition in which makes use of two previous terms and no constants.

	\begin{answer}
		The sequence we get is $3, 5, 7, 9, \ldots$.  One recursive definition for this is $a_n = a_{n-1} + 2$ with $a_0 = 3$.  Another option would be to take $a_n = 2a_{n-1} - a_{n-2}$ with $a_0 = 3$ and $a_1 = 5$.
	\end{answer}
	
\end{questions}