
\begin{questions}
\question Make a truth table for the statement $(P \vee Q) \imp (P \wedge Q)$. 

\begin{answer}
 \begin{tabular}{c|c|c}
             $P$ & $Q$ & $(P \vee Q) \imp (P \wedge Q)$\\ \hline
             T & T & T \\
             T & F & F \\
             F & T & F \\
             F & F & T
          \end{tabular}
\end{answer}


\question Make a truth table for the statement $\neg P \wedge (Q \imp P)$.  What can you conclude about $P$ and $Q$ if you know the statement is true?

    \begin{answer}
      \begin{tabular}{c|c|c}
             $P$ & $Q$ & $\neg P \wedge (Q \imp P)$\\ \hline
             T & T & F \\
             T & F & F \\
             F & T & F \\
             F & F & T
          \end{tabular}
	If the statement is true, then both $P$ and $Q$ are false.
    \end{answer}


\question Make a truth table for the statement $\neg P \imp (Q \wedge R)$.

  \begin{answer}
    Hint: Like above, only now you will need 8 rows instead of just 4.
  \end{answer}


\question Determine whether the following two statements are logically equivalent: $\neg(P \imp Q)$ and $P \wedge \neg Q$.  Explain how you know you are correct.

  \begin{answer}
    Make a truth table for each and compare.  The statements are logically equivalent.
  \end{answer}

  
  

\question Are the statements $P \imp (Q\vee R)$ and $(P \imp Q) \vee (P \imp R)$ logically equivalent?

  \begin{answer}
    Again, make two truth tables.  The statements are logically equivalent.
  \end{answer}


  
  
\question Determine if the following argument form is valid: \begin{tabular}{rc} & $P \vee Q$ \\ & $\neg P$ \\ \hline $\therefore$ & $Q$\end{tabular}.

  \begin{answer}
    The argument is valid.  To see this, make a truth table which contains $P \vee Q$ and $\neg P$ (and $P$ and $Q$ of course).  Look at the truth value of $Q$ in each of the rows that have $P \vee Q$ and $\neg P$ true.  
  \end{answer}

  
  
  
\question Determine if the following argument form is valid: \begin{tabular}{rc} & $P \imp (Q \vee R)$ \\ & $\neg(P \imp Q)$ \\ \hline $\therefore$ & $R$\end{tabular}

  \begin{answer}
    The argument form is valid.  Again, make a truth table containing the premises and conclusion - look at the rows for which the premises are true.
  \end{answer}


  
\question Determine if the following argument form is valid: \begin{tabular}{rc} & $(P \wedge Q) \imp R$ \\ & $\neg P \vee \neg Q$ \\ \hline $\therefore$ & $\neg R$\end{tabular}

  \begin{answer}
    The argument is NOT valid.  If you make a truth table containing the premises and conclusion, there will be a row with both premises true but the conclusion false.  For example, if $P$ and $Q$ are false and $R$ is true, then $P \wedge Q$ is false, so $(P \wedge Q) \imp R$ is true.  Also $\neg P$ is true, so $\neg P \vee \neg Q$ is true.  However, $\neg R$ is false.
  \end{answer}


  
\question Consider the statement about a party, ``If it's your birthday or there will be cake, then there will be cake.''
\begin{parts}
 \part Translate the above statement into symbols.  Clearly state which statement is $P$ and which is $Q$.
 \part Make a truth table for the statement.
 \part Assuming the statement is true, what (if anything) can you conclude if there will be cake?
 \part Assuming the statement is true, what (if anything) can you conclude if there will not be cake?
 \part Suppose you found out that the statement was a lie.  What can you conclude?
\end{parts}

  \begin{answer}
    \begin{parts}
      \part $P$: it's your birthday; $Q$: there will be cake.  $(P \vee Q) \imp Q$
      \part Hint: you should get three T's and one F.
      \part Only that there will be cake.
      \part It's NOT your birthday!
      \part It's your birthday, but the cake is a lie.
    \end{parts}
  \end{answer}


  
\question Suppose $P$ and $Q$ are the statements:
$P$: Jack passed math.
$Q$: Jill passed math.
\begin{parts}
 \part Translate ``Jack and Jill both passed math'' into symbols.
\part Translate ``If Jack passed math, then Jill did not'' into symbols.
\part Translate ``$P \vee Q$'' into English.
\part Translate ``$\neg(P \wedge Q) \imp Q$'' into English.
\part Suppose you know that if Jack passed math, then so did Jill.  What can you conclude if you know that:
\begin{subparts}
 \subpart Jill passed math?  
\subpart  Jill did not pass math?
\end{subparts}
\end{parts}

  \begin{answer}
    \begin{parts}
      \part $P \wedge Q$
      \part $P \imp \neg Q$
      \part Jack passed math or Jill passed math (or both).
      \part If Jack and Jill did not both pass math, then Jill did.
      \part 
	\begin{subparts}
	  \subpart Nothing else. 
	  \subpart  Jack did not pass math either.
	\end{subparts}
    \end{parts}
  \end{answer}



  
\question Geoff Poshingten is out at a fancy pizza joint, and decides to order a calzone.  When the waiter asks what he would like in it, he replies, ``I want either pepperoni or sausage, and if I have sausage, I must also include quail.  Oh, and if I have pepperoni or quail then I must also have ricotta cheese.''  
\begin{parts}
	\part Translate Geoff's order into logical symbols.
	\part The waiter knows that Geoff is either a liar or a truth-teller (so either everything he says is false, or everything is true).  Which is it?
	\part What, if anything, can the waiter conclude about the ingredients in Geoff's desired calzone?
\end{parts}

  \begin{answer}
    \begin{parts}
	\part Three statements: $P \vee S$, $S \imp Q$, $(P \vee Q) \imp R$.  You could also connect the first two with a $\wedge$.
	\part He cannot be lying about all three sentences, so he is telling the truth.
	\part No matter what, Geoff wants ricotta.  If he doesn't have quail, then he must have pepperoni but not sausage.
    \end{parts}
  \end{answer}


  
  
\question Consider the statement ``If Oscar eats Chinese food, then he drinks milk.''
\begin{parts}
 \part Write the converse of the statement.
 \part Write the contrapositive of the statement.
 \part Is it possible for the contrapositive to be false?  If it was, what would that tell you?
 \part Suppose the original statement is true, and that Oscar drinks milk.  Can you conclude anything (about his eating Chinese food)?  Explain.
 \part Suppose the original statement is true, and that Oscar does not drink milk.  Can you conclude anything (about his eating Chinese food)?  Explain.
\end{parts}

  \begin{answer}
    \begin{parts}
      \part If Oscar drinks milk, then he eats Chinese food.
      \part If Oscar does not drink milk, then he does not eat Chinese food.
      \part Yes.  The original statement would be false too.
      \part Nothing. The converse need not be true.
      \part He does not eat Chinese food. The contrapositive would be true.
    \end{parts}
  \end{answer}

  
  

\question Simplify the following statements (so that negation only appears right before variables).
\begin{parts}
  \part $\neg(P \imp \neg Q)$
  \part $(\neg P \vee \neg Q) \imp \neg (\neg Q \wedge R)$
  \part $\neg((P \imp \neg Q) \vee \neg (R \wedge \neg R))$
  \part It is false that if Sam is not a man then Chris is a woman, and that Chris is not a woman.
\end{parts}

  \begin{answer}
    \begin{parts}
      \part $P \wedge Q$ 
      \part $(P \vee Q) \vee (Q \wedge \neg R)$
      \part F.  Or $(P \wedge Q) \wedge (R \wedge \neg R)$ 
      \part Either Sam is a woman and Chris is a man, or Chris is a woman.
    \end{parts}
  \end{answer}


  
\question Which of the following statements are equivalent to the implication, ``if you win the lottery, then you will be rich,'' and which are equivalent to the converse of the implication?
\begin{parts}
 \part Either you win the lottery or else you are not rich.
 \part Either you don't win the lottery or else you are rich.
 \part You will win the lottery and be rich.
 \part You will be rich if you win the lottery.
 \part You will win the lottery if you are rich.
 \part It is necessary for you to win the lottery to be rich.
 
 \part It is sufficient to with the lottery to be rich.
 \part You will be rich only if you win the lottery.
 \part Unless you win the lottery, you won't be rich.
 \part If you are rich, you must have one the lottery.
 \part If you are not rich, then you did not win the lottery.
 \part You will win the lottery if and only if you are rich.
\end{parts}

  \begin{answer} The statements are equivalent to the\ldots
    \begin{parts}
      \part converse.
      \part implication.
      \part neither.
      \part implication.
      \part converse.
      \part converse.
      
      \part implication.
      \part converse.
      \part converse.
      \part converse (in fact, this IS the converse).
      \part implication (the statement is the contrapositive of the implication).
      \part neither.
    \end{parts}
  \end{answer}

  
\question Consider the implication, ``if you clean your room, then you can watch TV.''  Rephrase the implication in as many ways as possible.  Then do the same for the converse.

  \begin{answer}
    Hint: of course there are many answers.  It helps to assume that the statement is true and the converse is NOT true.  Think about what that means in the real world and then start saying it in different ways.  Some ideas: use necessary and sufficient language, use ``only if,'' consider negations, use ``or else'' language.
  \end{answer}

  
% \question Translate into symbols.  Use $E(x)$ for ``$x$ is even'' and $O(x)$ for ``$x$ is odd.''
%  \begin{parts}
%   \part No number is both even and odd.
% \part One more than any even number is an odd number.
% \part There is prime number that is even.
% \part Between any two numbers there is a third number.
% \part There is no number between a number and one more than that number.
%  \end{parts}
% 
%   \begin{answer}
%      \begin{parts}
% 	\part $\neg \exists x (E(x) \wedge O(x))$
% 	\part $\forall x (E(x) \imp O(x+1))$
% 	\part $\exists x(P(x) \wedge E(x))$ (where $P(x)$ means ``$x$ is prime'')
% 	\part $\forall x \forall y \exists z(x < z < y \vee y < z < x)$
% 	\part $\forall x \neg \exists y (x < y < x+1)$
%     \end{parts}
%   \end{answer}
% 
%   
%  
%  
% \question Translate into English:
% \begin{parts}
%  \part $\forall x (E(x) \imp E(x +2))$
% \part $\forall x \exists y (\sin(x) = y)$
% \part $\forall y \exists x (\sin(x) = y)$
% \part $\forall x \forall y (x^3 = y^3 \imp x = y)$
% \end{parts}
% 
%   \begin{answer}
%     \begin{parts}
% 	\part Any even number plus 2 is an even number.
% 	\part For any $x$ there is a $y$ such that $\sin(x) = y$.  In other words, every number $x$ is in the domain of sine. 
% 	\part For every $y$ there is an $x$ such that $\sin(x) = y$.  In other words, every number $y$ is in the range of sine (which is false).
% 	\part For any numbers, if the cubes of two numbers are equal, then the numbers are equal.
%       \end{parts}
%   \end{answer}
% 
%   
%   
% 
% \question Simplify the statements (so negation appears only directly next to predicates).
% \begin{parts}
%   \part $\neg \exists x \forall y (\neg O(x) \vee E(y))$
%   \part $\neg \forall x \neg \forall y \neg(x < y \wedge \exists z (x < z \vee y < z))$
%   \part There is a number $n$ for which no other number is either less $n$ than or equal to $n$.
%   \part It is false that for every number $n$ there are two other numbers which $n$ is between.
% \end{parts}
% 
%   \begin{answer}
%     \begin{parts}
% 	\part $\forall x \exists y (O(x) \wedge \neg E(y))$
% 	\part $\exists x \forall y (x \ge y \vee \forall z (x \ge z \wedge y \ge z))$
% 	\part There is a number $n$ for which every other number is strictly greater than $n$.
% 	\part There is a number $n$ which is not between any other two numbers.
%       \end{parts}
%   \end{answer}
% 
%   
% 
% 
% \question Consider the statement ``for all integers $a$ and $b$, if $a + b$ is even, then $a$ and $b$ are even''
% \begin{parts}
%  \part Write the contrapositive of the statement
%  \part Write the converse of the statement
%  \part Write the negation of the statement.
%  \part Is the original statement true or false?  Prove your answer.
%  \part Is the contrapositive of the original statement true or false?  Prove your answer.
%  \part Is the converse of the original statement true or false?  Prove your answer.
%  \part Is the negation of the original statement true or false?  Prove your answer.
% \end{parts}
% 
%   \begin{answer}
%     \begin{parts}
% 	\part For all integers $a$ and $b$, if $a$ or $b$ are not even, then $a+b$ is not even.
% 	\part For all integers $a$ and $b$, if $a$ and $b$ are even, then $a+b$ is even.
% 	\part There are numbers $a$ and $b$ such that $a+b$ is even but $a$ and $b$ are not both even.
% 	\part False.  For example, $a = 3$ and $b = 5$.  $a+b = 8$, but neither $a$ nor $b$ are even.
% 	\part False, since it is equivalent to the original statement.
% 	\part True.  Let $a$ and $b$ be integers.  Assume both are even.  Then $a = 2k$ and $b = 2j$ for some integers $k$ and $j$.  But then $a+b = 2k + 2j = 2(k+j)$ which is even.
% 	\part True, since the statement is false.
%       \end{parts}
%   \end{answer}
% 
% 
%   
%   
% \question Prove that $\sqrt 3$ is irrational.
% 
%   \begin{answer}
%     \begin{proof}
%      Suppose $\sqrt{3}$ were rational.  Then $\sqrt{3} = \frac{a}{b}$ for some integers $a$ and $b \ne 0$.  Without loss of generality, assume $\frac{a}{b}$ is reduced.  Now
% \[3 = \frac{a^2}{b^2}\]
% \[b^2 3 = a^2\]
% So $a^2$ is a multiple of 3.  This can only happen if $a$ is a multiple of 3, so $a = 3k$ for some integer $k$.  Then we have
% \[b^2 3 = 9k^2\]
% \[b^2 = 3k^2\]
% So $b^2$ is a multiple of 3, making $b$ a multiple of 3 as well.  But this contradicts our assumption that $\frac{a}{b}$ is in lowest terms.
%     \end{proof}
%   \end{answer}
% 
%  
 
 
\end{questions}



