\documentclass[11pt]{exam} 
\usepackage{answers, amsthm, amsmath, amssymb, mathrsfs} \pagestyle{head} \firstpageheader{Math 228}{\bf Practice Problems 6: Advanced Counting\\ Hints and Answers}{Spring 2013} \Newassociation{answer}{Ans}{Practice06-AdvancedCounting-solutions} 
 \def\d{\displaystyle}
\def\?{\reflectbox{?}}
\def\b#1{\mathbf{#1}}
\def\f#1{\mathfrak #1}
\def\c#1{\mathcal #1}
\def\s#1{\mathscr #1}
\def\r#1{\mathrm{#1}}
\def\N{\mathbb N}
\def\Z{\mathbb Z}
\def\Q{\mathbb Q}
\def\R{\mathbb R}
\def\C{\mathbb C}
\def\F{\mathbb F}
\def\A{\mathbb A}
\def\X{\mathbb X}
\def\E{\mathbb E}
\def\O{\mathbb O}
\def\U{\mathcal U}
\def\pow{\mathcal P}
\def\inv{^{-1}}
\def\nrml{\triangleleft}
\def\st{:}
\def\~{\widetilde}
\def\rem{\mathcal R}
\def\sigalg{$\sigma$-algebra }
\def\Gal{\mbox{Gal}}
\def\iff{\leftrightarrow}
\def\Iff{\Leftrightarrow}
\def\land{\wedge}
\def\And{\bigwedge}
\def\AAnd{\d\bigwedge\mkern-18mu\bigwedge}
\def\Vee{\bigvee}
\def\VVee{\d\Vee\mkern-18mu\Vee}
\def\imp{\rightarrow}
\def\Imp{\Rightarrow}
\def\Fi{\Leftarrow}

%\def\={\equiv}
\def\var{\mbox{var}}
\def\mod{\mbox{Mod}}
\def\Th{\mbox{Th}}
\def\sat{\mbox{Sat}}
\def\con{\mbox{Con}}
\def\bmodels{=\joinrel\mathrel|}
\def\iffmodels{\bmodels\models}
\def\dbland{\bigwedge \!\!\bigwedge}
\def\dom{\mbox{dom}}
\def\rng{\mbox{range}}
\DeclareMathOperator{\wgt}{wgt}


\def\bar{\overline}


\newcommand{\vtx}[2]{node[fill,circle,inner sep=0pt, minimum size=4pt,label=#1:#2]{}}
\newcommand{\va}[1]{\vtx{above}{#1}}
\newcommand{\vb}[1]{\vtx{below}{#1}}
\newcommand{\vr}[1]{\vtx{right}{#1}}
\newcommand{\vl}[1]{\vtx{left}{#1}}
\renewcommand{\v}{\vtx{above}{}}

\def\circleA{(-.5,0) circle (1)}
\def\circleAlabel{(-1.5,.6) node[above]{$A$}}
\def\circleB{(.5,0) circle (1)}
\def\circleBlabel{(1.5,.6) node[above]{$B$}}
\def\circleC{(0,-1) circle (1)}
\def\circleClabel{(.5,-2) node[right]{$C$}}
\def\twosetbox{(-2,-1.4) rectangle (2,1.4)}
\def\threesetbox{(-2.5,-2.4) rectangle (2.5,1.4)}
\newcommand{\twoline}[2]{\begin{pmatrix}#1 \\ #2 \end{pmatrix}}

\usepackage{tikz, multicol}
\renewenvironment{Ans}[1]{\setcounter{question}{#1}\addtocounter{question}{-1}\question }{}
\begin{document}
 \begin{questions}
\begin{Ans}{1}
	 \begin{parts}
	   \part ${18 \choose 4}$.  Each outcome can be represented by a sequence of 14 stars and 4 bars. %How many ways can you do this if there are no restrictions?
	   \part ${13 \choose 4}$.  First put one ball in each bin.  This leaves 9 stars and 4 bars.%How many ways can you do this if each bin must contain at least one dodge-ball?
	   \part ${18 \choose 4} - \left[ {5 \choose 1}{11 \choose 4} - {5 \choose 2}{4 \choose 4}\right]$.  Subtract all the distributions for which one or more bins contain 7 or more balls.  %How many ways can you do this if no bin can hold more than 6 balls?
	 \end{parts}
	
\end{Ans}
\begin{Ans}{2}
	\begin{parts}
	  \part ${7 \choose 2}$.  After each variable gets 1 star for free, we are left with 5 stars and 2 bars.  %$x$, $y$, and $z$ are all positive?
	  \part ${10 \choose 2}$.  We have 8 stars and 2 bars.  %$x$, $y$, and $z$ are all non-negative?
	  \part ${19 \choose 2}$.  This problem is equivalent to finding the number of solutions to $x' + y' + z' = 17$ where $x'$, $y'$ and $z'$ are non-negative.  (In fact, we really just do a substitution.  Let $x = x'- 3$, $y = y' - 3$ and $z = z' - 3$).  %$x$, $y$, and $z$ are all greater than $-3$.
	\end{parts}
	
\end{Ans}
\begin{Ans}{3}
	${10 \choose 5}$.  We have 5 stars (the five dice) and 5 bars (the five switches between the number 1-6).
	
\end{Ans}
\begin{Ans}{4}
	${18 \choose 3}$.  Distribute 10 units to the variables before finding all solutions to $x_1' + x_2' + x_3' + x_4' = 15$ in non-negative integers.
	
\end{Ans}
\begin{Ans}{5}
	There are 8 different functions.  For example, $f(1) = a$, $f(2) = a$, $f(3) = a$; or $f(1) = a$, $f(2) = b$, $f(3) = a$, and so on.  None of the functions are injective.  Exactly 6 of the functions are surjective.  No functions are both (since no functions here are injective).
	
\end{Ans}
\begin{Ans}{6}
	There are nine functions - you have a choice of three outputs for $f(1)$, and for each, you have three choices for the output $f(2)$.  Of these functions, 6 are injective, 0 are surjective, and 0 are both.
	
\end{Ans}
\begin{Ans}{7}
		\begin{parts}
		%Is $f$ injective?  Explain.
		\part $f$ is not injective, since $f(2) = f(5)$ - two different inputs have the same output.
		% Is $f$ surjective?  Explain.
		\part $f$ is surjective, since every element of the codomain is an element of the range.
		\end{parts}
	
\end{Ans}
\begin{Ans}{8}
		\begin{parts}
		%Is $f$ injective?  Explain.
		  \part $f$ is not injective, since $f(1) = 3$ and $f(4) = 3$.
		%   Is $f$ surjective?  Explain.
		  \part $f$ is not surjective, since there is no input which gives 2 as an output.
		\end{parts}
	
\end{Ans}
\begin{Ans}{9}
		\begin{parts}
		% $f:\N \to \N$ given by $f(n) = n+4$.
		  \part $f$ is injective, but not surjective.
		%   $f:\Z \to \Z$ given by $f(n) = n+4$.
		  \part $f$ is injective and surjective.
		  %$f:\Z \to \Z$ given by $f(n) = 5n - 8$.
		  \part $f$ is injective, but not surjective.
		%   $f:\Z \to \Z$ given by $f(n) = \begin{cases}
		%                                          n/2 & \mbox{ if $n$ is even}\\
		%                                          (n+1)/2 & \mbox{ if $n$ is odd}.
		%                                        \end{cases}$
		 \part $f$ is not injective, but is surjective.
		\end{parts}
	
\end{Ans}
\begin{Ans}{10}
		\begin{parts}
		% there is a injective function $f:X \to Y$.  Explain.
		  \part $|X| \le |Y|$ - otherwise two or more of the elements of $X$ would need to map to the same element of $Y$.
		%   there is a surjective function $f:X \to Y$.  Explain.
		  \part $|X| \ge |Y|$ - otherwise there would be one or more elements of $Y$ which were never an output.
		%   there is a bijection $f:X \to Y$.  Explain.
		  \part $|X| = |Y|$.  This is the only way for both of the above to occur.
		\end{parts}
	
\end{Ans}
\begin{Ans}{11}
		\begin{parts}
		% $f$ is injective but not surjective.
		  \part Yes. (Can you give an example?)
		  \part Yes. %$f$ is surjective but not injective.
		  \part Yes. %$|X| = |Y|$ and $f$ is injective but not surjective.
		  \part Yes. %$|X| = |Y|$ and $f$ is surjective but not injective.
		  \part No. %$|X| = |Y|$, $X$ and $Y$ are finite, and $f$ is injective but not surjective.
		  \part No. %$|X| = |Y|$, $X$ and $Y$ are finite, and $f$ is surjective but not injective.
		\end{parts}
	
\end{Ans}
\begin{Ans}{12}
	\begin{parts}
	  \part $6^4 = 1296$, since there are six choices of where to send each of the 4 elements of the domain. %How many functions are there total?
	  \part $P(6, 4) = 6 \cdot 5 \cdot 4 \cdot 3 = 360$, since outputs cannot be repeated.  %How many functions are injective?
	  \part None. %How many functions are surjective?
	  \part There are $5 \cdot 6^3$ functions for which $f(1) \ne a$ and another $5 \cdot 6^3$ functions for which $f(2) \ne b$.  There are $5^2 \cdot 6^2$ functions for which both $f(1) \ne a$ and $f(2) \ne b$.  So the total number of functions for which $f(1) \ne a$ or $f(2) \ne b$ or both is
	  \[5 \cdot 6^3 + 5 \cdot 6^3 - 5^2 \cdot 6^2 = 1260\] %How many functions have the property that $f(1) \ne a$ or $f(2) \ne b$, or both?
	\end{parts}
	
\end{Ans}
\begin{Ans}{13}
	\begin{parts}
	  \part $17^{10}$ %How many functions $f: A \to B$ are there?
	  \part $P(17, 10)$  %How many functions $f: A \to B$ are injective?
	\end{parts}
	
\end{Ans}
\begin{Ans}{14}
	$5^{10} - \left[{5 \choose 1}4^{10} - {5 \choose 2}3^{10} + {5 \choose 3}2^{10} - {5 \choose 4}1^{10}\right]$ %Consider sets $A$ and $B$ with $|A| = 10$ and $|B| = 5$.  How many functions $f: A \to B$ are surjective?
	
\end{Ans}
\begin{Ans}{15}
	$5! - \left[{5 \choose 1}4! - {5 \choose 2}3! + {5 \choose 3}2! - {5 \choose 4}1! + {5 \choose 5}0!\right]$.  This is a sneaky way to as for the number of derangements on 5 elements. %Let $A = \{1,2,3,4,5\}$.  How many injective functions $f:A \to A$ have the property that for each $x \in A$, $f(x) \ne x$?
	
\end{Ans}
\begin{Ans}{16}
	${10 \choose 6}\left(4! - \left[{4 \choose 1} 3! - {4 \choose 2}2! + {4 \choose 3}1! - {4 \choose 4}0!\right]\right)$.  We choose 6 of the 10 ladies to get their own hat, and the multiply by the number of ways the remaining hats can be deranged.
	
\end{Ans}
\begin{Ans}{17}
	\begin{parts}
	  \part ${8 \choose 3}$, after giving one present to each kid, you are left with 5 presents (stars) which need to be divide among the 4 kids (giving 3 bars). %The presents are identical, and each kid gets at least one present?
	  \part ${12 \choose 3}$.  You have 9 stars and 3 bars.  %The presents are identical, and some kids might get no presents?
	  \part $4^9$.  You have 4 choices for whom to give each present.  This is like making a function from the set of present to the set of kids. %The presents are unique, and some kids might get no presents?
	  \part $4^9 - \left[{4 \choose 1}3^9 - {4\choose 2}2^9 + {4 \choose 3}1^9 \right]$.  Now the function from the set of present to the set of kids must be surjective. %the presents are unique and each kid gets at least one present?
	\end{parts}
	
\end{Ans}
\begin{Ans}{18}
	\begin{parts}
	    \part Neither.  ${14 \choose 4}$.
	  \part ${10\choose 4}$
	  \part $P(10,4)$, since order is important.
	  \part Neither. Assuming you will wear each of the 4 ties on just 4 of the 7 days, without repeats: ${10\choose 4}P(7,4)$.
	  \part $P(10,4)$
	  \part ${10\choose 4}$
	  \part Neither. Since you could repeat letters: $10^4$. If no repeats are allowed, it would be $P(10,4)$.
	  \part Neither.  Actually, ``k'' is the 11th letter of the alphabet, so the answer is 0.  If ``k'' was among the first 10 letters, there would only be 1 way - write it down.
	  \part Neither.  Either ${9\choose 3}$ (if every kid gets an apple) or ${13 \choose 3}$ (if appleless kids are allowed).
	  \part Neither.  Note that this could not be ${10 \choose 4}$ since the 10 things and 4 things are from different groups.  $4^{10}$
	  \part ${10 \choose 4}$ - don't be fooled by the ``arrange'' in there - you are picking 4 out of 10 {\em spots} to put the 1's.
	  \part ${10 \choose 4}$ (assuming order is irrelevant).
	  \part Neither.  $16^{10}$ (each kid chooses yes or no to 4 varieties).
	  \part Neither.  0.
	  \part Neither.  $4^{10} - [{4\choose 1}3^{10} - {4\choose 2}2^{10} + {4 \choose 3}1^{10}]$
	  \part Neither.  $10\cdot 4$.
	  \part Neither. $4^{10}$.
	  \part ${10 \choose 4}$ (which is the same as ${10 \choose 6}$).
	  \part Neither.  If all the kids were identical, and you wanted no empty teams, it would be ${10 \choose 4}$.  Instead, this will be the same as the number of surjective functions from a set of size 11 to a set of size 5.
	  \part ${10 \choose 4}$
	  \part ${10 \choose 4}$
	  \part Neither.  $4!$.
	  \part Neither.  It's ${10 \choose 4}$ if you won't repeat any choices.  If repetition is allowed, then this becomes $x_1 + x_2 + \cdots +x_{10} = 4$, which has ${13 \choose 9}$ solutions in non-negative integers.
	  \part Neither.  Since repetition of cookie type is allowed, the answer is $10^4$.  Without repetition, you would have $P(10,4)$.
	  \part ${10 \choose 4}$ since that is equal to ${9 \choose 4} + {9 \choose 3}$.
	  \part Neither.  It will be a complicated (possibly PIE) counting problem.
	\end{parts}
	
\end{Ans}
 \end{questions} \par \end{document}
