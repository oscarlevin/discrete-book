\begin{questions}


	
	
\question Find $3 + 7 + 11+ \cdots + 427$.

  \begin{answer}
   $\frac{430\cdot 107}{2} = 23005$.
  \end{answer}


	
	
\question Consider the sum $4 + 11 + 18 + 25 + \cdots + 249$.
\begin{parts}
\part How many terms (summands) are in the sum?
\part Compute the sum.  Remember to show all your work.
\end{parts}

	\begin{answer}
		\begin{parts}
		\part 36.  %How many terms (summands) are in the sum?
		\part $\frac{253 \cdot 36}{2} = 4554$.  %Compute the sum.  Remember to show all your work.
		\end{parts}
	\end{answer}




\question Consider the sequence $2, 6, 10, 14, \ldots, 4n + 6$.  
\begin{parts}
\part How many terms are there in the sequence?
\part What is the second-to-last term?
\part Find the sum of all the terms in the sequence.
\end{parts}

  \begin{answer}
	 \begin{parts}
	 \part $n+2$ terms.
	 \part $4n+2$.
	 \part $\frac{(4n+8)(n+2)}{2}$.
	 \end{parts}
  \end{answer}



\question Consider the sequence $5, 9, 13, 17, 21, \ldots$ with $a_1 = 5$
\begin{parts}
  \part Give a recursive definition for the sequence.
  \part Give a closed formula for the $n$th term of the sequence.
  \part Is $2013$ a term in the sequence?  Explain.
  \part How many terms does the sequence $5, 9, 13, 17, 21, \ldots, 533$ have?
  \part Find the sum: $5 + 9 + 13 + 17 + 21 + \cdots + 533$.  Show your work.
  \part Use what you found above to find $b_n$, the $n^{th}$ term of $1, 6, 15, 28, 45, \ldots$, where $b_0 = 1$
\end{parts}

	\begin{answer}
		\begin{parts}
		  \part $a_n = a_{n-1} + 4$ with $a_1 = 5$.  %Give a recursive definition for the sequence.
		  \part $a_n = 5 + 4(n-1)$.  %Give a closed formula for the $n$th term of the sequence.
		  \part Yes, since $2013 = 5 + 4(503-1)$ (so $a_{503} = 2013$).
		  \part 133 %How many terms does the sequence $5, 9, 13, 17, 21, \ldots, 533$ have?
		  \part $\frac{538\cdot 133}{2} = 35777$.  %Find the sum: $5 + 9 + 13 + 17 + 21 + \cdots + 533$.  Show your work.
		  \part $b_n = 1 + \frac{(4n+1)n}{2}$.
		\end{parts}
	\end{answer}





%Sum of geometric sequence
\question Consider the sequence given by $a_n = 2\cdot 5^{n-1}$.
\begin{parts}
\part Find the first 4 terms of the sequence.  What sort of sequence is this?
\part Find the {\em sum} of the first 25 terms.  That is, compute $\d\sum_{k=1}^{25}a_k$.
\end{parts}

	\begin{answer}
		\begin{parts}
		\part $2, 10, 50, 250, \ldots$  The sequence is geometric. %Find the first 4 terms of the sequence.  What sort of sequence is this?
		\part $\frac{2 - 2\cdot 5^{25}}{-4}$.  %Find the {\em sum} of the first 25 terms.  That is, compute $\d\sum_{k=1}^{25}a_k$.
		\end{parts}
	\end{answer}





%Polynomial fitting
\question Use polynomial fitting to find a closed formula for the sequence:
$4, 11, 20, 31, 44, \ldots $
(assume $a_1 = 4$).

	\begin{answer}
		$a_n = n^2 + 4n - 1$.
	\end{answer}




\question Suppose the closed formula for a particular sequence is a degree 3 polynomial.  What can you say about the closed formula for:
\begin{parts}
  \part The sequence of partial sums.
  \part The sequence of second differences.
\end{parts}

  \begin{answer}
	 \begin{parts}
	  \part The sequence of partial sums will be a degree 4 polynomial (its sequence of differences will be the original sequence).
	  \part The sequence of second differences will be a degree 1 polynomial - an arithmetic sequence.
	 \end{parts}
  \end{answer}



%Recursive definition:
\question Consider the sequence given recursively by $a_1 = 4$, $a_2 = 6$ and $a_n = a_{n-1} + a_{n-2}$.
\begin{parts}
\part Write out the first 6 terms of the sequence.
\part Could the closed formula for $a_n$ be a polynomial?  Explain.
\end{parts}

	\begin{answer}
	 	\begin{parts}
	 	\part $4, 6, 10, 16, 26, 42, \ldots$.  %Write out the first 6 terms of the sequence.
	 	\part No, taking differences gives the original sequence back, so the differences will never be constant.  %Could the closed formula for $a_n$ be a polynomial?  Explain.
	 	\end{parts}
	\end{answer}





%new sequences from old:
\question The sequence $-1, 0, 2, 5, 9, 14\ldots$ has closed formula $a_n = \dfrac{(n+1)(n-2)}{2}$.  Use this fact to find a closed formula for the sequence $4, 10, 18, 28, 40, \ldots$. 

	\begin{answer}
		 $b_n = (n+3)n$.
	\end{answer}






\question Consider the recurrence relation $a_n = 3a_{n-1} + 10 a_{n-2}$ with first two terms $a_0 = 1$ and $a_1 = 2$.
\begin{parts}
 \part Write out the first 5 terms of the sequence defined by this recurrence relation.
 \part Solve the recurrence relation. That is, find a closed formula for $a_n$.
\end{parts}

	\begin{answer}
		\begin{parts}
		 \part $1, 2, 16,68, 364, \ldots$.  %Write out the first 5 terms of the sequence defined by this recurrence relation.
		 \part $a_n = \frac{3}{7}(-2)^n + \frac{4}{7}5^n$.  %Solve the recurrence relation. 
		\end{parts}
	\end{answer}





\question Consider the recurrence relation $a_n = 2a_{n-1} + 8a_{n-2}$, with initial terms $a_0 = 1$ and $a_1= 3$.
\begin{parts}
  \part Find the next two terms of the sequence ($a_2$ and $a_3$).
  \part Solve the recurrence relation.   That is, find a closed formula for the $n$th term of the sequence.
%  \part Find the generating function for the sequence.  Hint: use the recurrence relation.
\end{parts}

	\begin{answer}
		\begin{parts}
		  \part $a_2 = 14$.  $a_3 = 52$.  %Find the next two terms of the sequence ($a_2$ and $a_3$).
		  \part $a_n = \frac{1}{6}(-2)^n + \frac{5}{6}4^n$.  %Solve the recurrence relation.   That is, find a closed formula for the $n$th term of the sequence.
%		  \part $\frac{1+x}{1-2x-8x^2}$  %Find the generating function for the sequence.  Hint: use the recurrence relation.
		\end{parts}
	\end{answer}



%\question Each day your supply of magic chocolate covered espresso beans doubles (each one splits in half), but then you eat 5 of them.  You have 10 on day 0. 
%
%\begin{parts}
%	\part Write out the first few terms of the sequence.
%	\part Give a recursive definition of the sequence and explain why it is correct.
%	\part Prove, using induction, that the last digit of the number of beans you have on the $n$th day will always be a 5 for all $n \ge 1$.
%	\part Find a closed formula for the $n$th term of the sequence and prove it is correct by induction.
%\end{parts}
%
%	\begin{answer}
%		\begin{parts}
%			\part $10, 15, 25, 45, 85, \ldots$.
%			\part On day $n$, you will have twice the number of beans you had the previous day, less 5.  Thus $a_n = 2a_{n-1} - 5$ with $a_0 = 10$.
%			\part Let $P(n)$ be the statement, ``on day $n$, the number of beans you have ends in a 5.''  For the base case, not that on day 1, you have 15 beans, which ends in a 5.  Now assume that $P(k)$ is true.  That is, on the $k$th day, the number of beans ends in a 5.  What happens on the next day?  We double the number of beans, and subtract 5.  If you double a number ending in 5, you will get a number ending in a 0.  Then subtracting 5 will give you a number ending in 5.  Thus the number of beans on day $k+1$ ends in a 5.
%			\part Try dividing all the numbers by 5, to get $2, 3, 5, 9, 17,\ldots$, which are all one more than a power of 2.  So we guess that $a_n = 5(2^n + 1)$.  To prove that this is correct, note that $5(2^0 + 1) = 10$, the number of beans you have on day 0.  Now assume the formula for $a_n$ is correct on day $k$.  That is, $a_k = 5(2^k+1)$.  The next day we double this and subtract 5, giving $2\cdot 5(2^k+1) - 5 = 5\cdot 2\cdot 2^{k} + 10 - 5 = 5\cdot 2^{k+1} + 5 = 5(2^{k+1}+1) = a_{k+1}$.  So the proposed formula for $a_n$ is true for $n= k+1$ as well.
%		\end{parts}
%	\end{answer}


\question Your magic chocolate bunnies\index{magic chocolate bunnies} reproduce like rabbits: every large bunny produces 2 new mini bunnies each day, and each day every mini bunny born the previous day grows into a large bunny.  Assume you start with 2 mini bunnies and no bunny ever dies (or gets eaten).

\begin{parts}
	\part Write out the first few terms of the sequence.
	\part Give a recursive definition of the sequence and explain why it is correct.
	\part Find a closed formula for the $n$th term of the sequence.
\end{parts}

	\begin{answer}
	 \begin{parts}
	 	\part On the first day, your 2 mini bunnies become 2 large bunnies.  On day 2, your two large bunnies produce 4 mini bunnies.  On day 3, you have 4 mini bunnies (produced by your 2 large bunnies) plus 6 large bunnies (your original 2 plus the 4 newly matured bunnies).  On day 4, you will have $12$ mini bunnies (2 for each of the 6 large bunnies) plus 10 large bunnies (your previous 6 plus the 4 newly matured).  The sequence of total bunnies is $2, 2, 6, 10, 22, 42\ldots$ starting with $a_0 = 2$ and $a_1 = 2$.  
	 	\part $a_n = a_{n-1} + 2a_{n-2}$.  This is because the number of bunnies is equal to the number of bunnies you had the previous day (both mini and large) plus 2 times the number you had the day before that (since all bunnies you had 2 days ago are now large and producing 2 new bunnies each).
	 	\part Using the characteristic root technique, we find $a_n = a2^n + b(-1)^n$, and we can find $a$ and $b$ to give $a_n = \frac{4}{3}2^n + \frac{2}{3}(-1)^n$.
	 \end{parts}
	\end{answer}






\question Prove the following statements my mathematical induction: 
\begin{parts}
 \part $n! < n^n$ for $n \ge 2$
 \part $\d\frac{1}{1\cdot 2} + \frac{1}{2\cdot 3} +\frac{1}{3\cdot 4}+\cdots + \frac{1}{n\cdot(n+1)} = \d\frac{n}{n+1}$ for all $n \in \Z^+$.
 \part $4^n - 1$ is a multiple of 3 for all $n \in \N$.
 %\part $F_0 + F_2 + F_4 + \cdots + F_{2n} = F_{2n + 1} - 1$ for all $n = 0,1,2,\ldots$.  ($F_n$ is the $n$th Fibonacci numbers.)
 \part The {\em greatest} amount of postage you {\em cannot} make exactly using 4 and 9 cent stamps is 23 cents.
 \part Every even number squared is divisible by 4.
\end{parts}

	\begin{answer}
		\begin{parts}
		 \part Hint: $(n+1)^{n+1} > (n+1) \cdot n^{n}$.
		 \part Hint: This should be similar to the other sum proofs.  The last bit comes down to adding fractions.
		 \part Hint: Write $4^{k+1} - 1 = 4\cdot 4^k - 4 + 3$.
%		 \part Hint: Use the fact $F_{2n} + F_{2n+1} = F_{2n+2}$
		 \part Hint: one 9-cent stamp is 1 more than two 4-cent stamps, and seven 4-cent stamps is 1 more than three 9-cent stamps.
		 \part Careful to actually use induction here.  The base case: $2^2 = 4$.  The inductive case: assume $(2n)^2$ is divisible by 4 and consider $(2n+2)^2 = (2n)^2 + 4n + 4$.  This is divisible by 4 because $4n +4$ clearly is, and by our inductive hypothesis, so is $(2n)^2$.
		\end{parts}
	\end{answer}




\question Prove $1^3 + 2^3 + 3^3 + \cdots + n^3 = \left(\frac{n(n+1)}{2}\right)^2$ holds for all $n \ge 1$, by mathematical induction.

	\begin{answer}
		Hint: This is a straight forward induction proof.  Note you will need to simplify $\left(\frac{n(n+1)}{2}\right)^2 + (n+1)^3$ and get $\left(\frac{(n+1)(n+2)}{2}\right)^2$.
	\end{answer}
	



\question Suppose $a_0 = 1$, $a_1 = 1$ and $a_n = 3a_{n-1} - 2a_{n-1}$.  Prove, using strong induction, that $a_n = 1$ for all $n$.

	\begin{answer}
		Hint: there are two base cases $P(0)$ and $P(1)$.  Then, for the inductive case, assume $P(k)$ is true for all $k < n$.  This allows you to assume $a_{n-1} = 1$ and $a_{n-2} = 1$.  Apply the recurrence relation.
	\end{answer}
	
	
	



\question Prove, using strong induction, that every positive integer can be written as the sum of distinct powers of 2.  For example, $13 = 1 + 4 + 8 = 2^0 + 2^2 + 2^3$.

	\begin{answer}
		Note that $1 = 2^0$; this is your base case.  Now suppose $k$ can be written as the sum of distinct powers of 2 for all $1\le k \le n$.  We can then write $n$ as the sum of distinct powers of 2 as follows: subtract the largest power of 2 less than $n$ from $n$.  That is, write $n = 2^j + k$ for the largest possible $j$.  But $k$ is now less than $n$, and also less than $2^j$, so write $k$ as the sum of distinct powers of 2 (we can do so by the inductive hypothesis).  Thus $n$ can be written as the sum of distinct powers of 2 for all $n \ge 1$.
	\end{answer}
	
	

%\question  Prove, by induction, that for any natural numbers $n \ge 1$, if $a$ and $b$ are real numbers, then $(ab)^n = a^nb^n$.
%
%	\begin{answer}
%		Hint: For the inductive case, we will get that $(ab)^{k+1} = (ab)^k(ab) = a^kb^kab$.  Simplify.
%	\end{answer}

\question Prove using induction that every set containing $n$ elements has $2^n$ different subsets for any $n \ge 1$.

	\begin{answer}
		Let $P(n)$ be the statement, ``every set containing $n$ elements has $2^n$ different subsets.''  We will show $P(n)$ is true for all $n \ge 1$.
		
		\underline{Base case}: Any set with 1 element $\{a\}$ has exactly 2 subsets: the empty set and the set itself.  Thus the number of subsets is $2= 2^1$.  Thus $P(1)$ is true.
		
		\underline{Inductive case}: Suppose $P(k)$ is true for some arbitrary $k \ge 1$.  Thus every set containing exactly $k$ elements has $2^k$ different subsets.  Now consider a set containing $k+1$ elements: $A = \{a_1, a_2, \ldots, a_k, a_{k+1}\}$.  Any subset of $A$ must either contain $a_{k+1}$ or not.  In other words, a subset of $A$ is just a subset of $\{a_1, a_2,\ldots, a_k\}$ with or without $a_{k+1}$.  Thus there are $2^k$ subsets of $A$ which contain $a_{k+1}$ and another $2^{k+1}$ subsets of $A$ which do not contain $a^{k+1}$.  This gives a total of $2^k + 2^k = 2\cdot 2^k = 2^{k+1}$ subsets of $A$.  But our choice of $A$ was arbitrary, so this works for any subset containing $k+1$ elements, so $P(k+1)$ is true.
		
		Therefore, by the principle of mathematical induction, $P(n)$ is true for all $n \ge 1$.
	\end{answer}	
	
	
	
\end{questions}