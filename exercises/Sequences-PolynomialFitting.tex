\begin{questions}



\question Use polynomial fitting to find the formula for the $n$th term of the following sequences.  Assume the first term is $a_0$.
\begin{parts}
\part 2, 5, 11, 21, 36,\ldots
\part 0, 2, 6, 12, 20, \ldots
\end{parts}

	\begin{answer}
		\begin{parts}
		\part Hint: third differences are constant, so $a_n = an^3 + bn^2 + cn + d$.  Use the terms of the sequence to solve for $a, b, c,$ and $d$.  You should get $a_n = 1/6 (12+11 n+6 n^2+n^3)$.
		\part $a_n = n^2 - n$.
		\end{parts}
	\end{answer}



\question Consider the sequence $ 1, 3, 7, 13, 21, \ldots$.  Explain how you know the closed formula for the sequence will be quadratic.  Then guess the correct formula by comparing this sequence to the squares $1, 4, 9, 16, \ldots$.

	\begin{answer}
		The first differences are $2, 4, 6, 8, \ldots$, and the second differences are $2, 2, 2, \ldots$.  Thus the original sequence is $\Delta^2$-constant, so can be fit to a quadratic.
		
		Call the original sequence $a_n$.  Consider $a_n - n^2$. This gives $0, -1, -2, -3, \ldots$.  \emph{That} sequence has closed formula $1-n$ (starting at $n = 1$) so we have $a_n - n^2 = 1-n$ or equivalently $a_n = n^2 - n + 1$.
	\end{answer}


\question Use a similar technique as in the previous exercise to find a closed formula for the sequence $2, 11, 34, 77, 146, 247,\ldots$.

	\begin{answer}
	 This is a $\Delta^3$-constant sequence.  If we subtract off $n^3$, we are left with $1, 3, 7, 13, 21, \ldots$, the sequence from the previous question.  Thus here the closed formula is $n^3 + n^2 - n + 1$.
	\end{answer}

	

\question Suppose $a_n = n^2 + 3n + 4$.  Find a closed formula for the sequence of differences by computing $a_n - a_{n-1}$.

	\begin{answer}
		$a_{n-1} = (n-1)^2 + 3(n-1) + 4 = n^2 + n + 2$.  Thus $a_n - a_{n-1} = 2n+2$.  Note that this is linear (arithmetic).  We can check that we are correct.  The sequence $a_n$ is $4, 8, 14, 22, 32, \ldots$ and the sequence of differences is thus $4, 6, 8, 10,\ldots$ which agrees with $2n+2$ (if we start at $n = 1$).
	\end{answer}	
	

\question Repeat the above this time assuming $a_n = an^2 + bn + c$.  That is, prove that every quadratic sequence has arithmetic differences.

	\begin{answer}
		$a_{n-1} = a(n-1)^2 + b(n-1) + c = an^2 - 2an + a + bn - b + c$.  Therefore $a_n - a_{n-1} = 2an - a + b$, which is arithmetic.  Notice that this is not quite the derivative of $a_n$, which would be $2an + b$, but it is close.
	\end{answer}


\question Can you use polynomial fitting to find the formula for the $n$th term of the sequence 4, 7, 11, 18, 29, 47, \ldots?  Explain why or why not. 

	\begin{answer}
		No.  The sequence of differences is the same as the original sequence so no differences will be constant. 
	\end{answer}
	
	
\question Will the $n$th sequence of differences of $2, 6, 18, 54, 162, \ldots$ ever be constant?  Explain.

	\begin{answer}
		No.  The sequence is geometric, and in fact has closed formula $2\cdot 3^n$.  This is an exponential function, which is not equal to any polynomial of any degree.  If the $n$th sequence of differences was constant, then the closed formula for the original sequence would be a degree $n$ polynomial.
	\end{answer}
	




\end{questions}