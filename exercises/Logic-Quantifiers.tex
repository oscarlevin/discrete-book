\begin{questions}


\question Translate into symbols.  Use $E(x)$ for ``$x$ is even'' and $O(x)$ for ``$x$ is odd.''
 \begin{parts}
  \part No number is both even and odd.
\part One more than any even number is an odd number.
\part There is prime number that is even.
\part Between any two numbers there is a third number.
\part There is no number between a number and one more than that number.
 \end{parts}

  \begin{answer}
     \begin{parts}
	\part $\neg \exists x (E(x) \wedge O(x))$
	\part $\forall x (E(x) \imp O(x+1))$
	\part $\exists x(P(x) \wedge E(x))$ (where $P(x)$ means ``$x$ is prime'')
	\part $\forall x \forall y \exists z(x < z < y \vee y < z < x)$
	\part $\forall x \neg \exists y (x < y < x+1)$
    \end{parts}
  \end{answer}

  
 
 
\question Translate into English:
\begin{parts}
 \part $\forall x (E(x) \imp E(x +2))$
\part $\forall x \exists y (\sin(x) = y)$
\part $\forall y \exists x (\sin(x) = y)$
\part $\forall x \forall y (x^3 = y^3 \imp x = y)$
\end{parts}

  \begin{answer}
    \begin{parts}
	\part Any even number plus 2 is an even number.
	\part For any $x$ there is a $y$ such that $\sin(x) = y$.  In other words, every number $x$ is in the domain of sine. 
	\part For every $y$ there is an $x$ such that $\sin(x) = y$.  In other words, every number $y$ is in the range of sine (which is false).
	\part For any numbers, if the cubes of two numbers are equal, then the numbers are equal.
      \end{parts}
  \end{answer}

  
  

\question Simplify the statements (so negation appears only directly next to predicates).
\begin{parts}
  \part $\neg \exists x \forall y (\neg O(x) \vee E(y))$
  \part $\neg \forall x \neg \forall y \neg(x < y \wedge \exists z (x < z \vee y < z))$
  \part There is a number $n$ for which no other number is either less $n$ than or equal to $n$.
  \part It is false that for every number $n$ there are two other numbers which $n$ is between.
\end{parts}

  \begin{answer}
    \begin{parts}
	\part $\forall x \exists y (O(x) \wedge \neg E(y))$
	\part $\exists x \forall y (x \ge y \vee \forall z (x \ge z \wedge y \ge z))$
	\part There is a number $n$ for which every other number is strictly greater than $n$.
	\part There is a number $n$ which is not between any other two numbers.
      \end{parts}
  \end{answer}





 


\end{questions}