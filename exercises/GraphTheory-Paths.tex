\begin{questions}

		


\question You and your friends want to tour the southwest by car.  You will visit the nine states below, with the following rather odd rule: you must cross each border between neighboring states exactly once (so, for example, you must cross the Colorado-Utah border exactly once).  Can you do it?  If so, does it matter where you start your road trip?  What fact about graph theory solves this problem?

%\centerline{\includegraphics[height=1in]{images/southwest_map.png}}

\begin{center}
\tikz[scale=.2]{
\USA[every state={draw=white, line width = .7pt, fill=black!10}, CA={fill=gray}, NV={fill=gray},NM={fill=gray},AZ={fill=gray},UT={fill=gray},CO={fill=gray},TX={fill=gray},KS={fill=gray},OK={fill=gray}]
}
\end{center}

	\begin{answer}
		This is a question about finding Euler paths.  Draw a graph with a vertex in each state, and connect vertices if their states share a border.  Exactly two vertices will have odd degree - the vertices for Nevada and Utah.  Thus you must start your road trip at in one of those states and end it in the other. %You and your friends want to tour the southwest by car.  You will visit the nine states below, with the following rather odd rule: you must cross each border between neighboring states exactly once (so, for example, you must cross the Colorado-Utah border exactly once).  Can you do it?  If so, does it matter where you start your road trip?  What fact about graph theory solves this problem?
	\end{answer}
	
	
	





\question Which of the following graphs contain an Euler path?  Which contain an Euler circuit?
\begin{parts}
  \part $K_4$
  \part $K_5$.
  \part $K_{5,7}$
  \part $K_{2,7}$
  \part $C_7$
  \part $P_7$
\end{parts}

	\begin{answer}
		\begin{parts}
		  \part $K_4$ does not have an Euler path or circuit.
		  \part $K_5$ has an Euler circuit (so also an Euler path).
		  \part $K_{5,7}$ does not have an Euler path or circuit.
		  \part $K_{2,7}$ has an Euler path but not an Euler circuit.
		  \part $C_7$ has an Euler circuit (it is a circuit graph!)
		  \part $P_7$ has an Euler path but no Euler circuit.
		\end{parts}
	\end{answer}
	


\question For which $n$ does the graph $K_n$ contain an Euler circuit?  Explain.

	\begin{answer}
		When $n$ is odd, $K_n$ contains an Euler circuit.  This is because every vertex has degree $n-1$, so an odd $n$ results in all degrees being even.%For which $n$ does the graph $K_n$ contain an Euler circuit?  Explain.
	\end{answer}
	
	
	


\question For which $m$ and $n$ does the graph $K_{m,n}$ contain an Euler path?  An Euler circuit?  Explain.

	\begin{answer}
		If both $m$ and $n$ are even, then $K_{m,n}$ has an Euler circuit.  When both are odd, there is no Euler path or circuit.  If one is 2 and the other is odd, then there is an Euler path but not an Euler circuit. %For which $m$ and $n$ does the graph $K_{m,n}$ contain an Euler path?  An Euler circuit?  Explain
	\end{answer}



\question A bridge builder has come to K\"onigsberg and would like to add bridges so that it \emph{is} possible to travel over every bridge exactly once.  How many bridges must be built?

%\begin{center}
%\begin{tikzpicture}[scale=1, yscale=.5]
% \draw (-1,-2) \v to [out=120, in=240] (-1,0) \v to [out=120, in=240] (-1,2) \v to [out=300, in=60] (-1,0) to [out=300, in=60] (-1,-2);
%  \draw (1,0) \v -- (-1,2) (-1,0) -- (1,0) -- (-1,-2);
%  \end{tikzpicture}
%\end{center}

	\begin{answer}
		If we build one bridge, we can have an Euler path.  Two bridges must be built for an Euler circuit.
		\begin{center}
		\begin{tikzpicture}[scale=1, yscale=.5]
		 \draw (-1,-2) \v to [out=120, in=240] (-1,0) \v to [out=120, in=240] (-1,2) \v to [out=300, in=60] (-1,0) to [out=300, in=60] (-1,-2);
		  \draw (1,0) \v -- (-1,2) (-1,0) -- (1,0) -- (-1,-2);
		  \draw[dashed] (-1,-2) -- (-1,0);
		  \draw[dashed] (1,0) to [out=120, in=0] (-1,2);
		  \end{tikzpicture}	
		\end{center}
	\end{answer}



\question Below is a graph representing friendships between a group of students (each vertex is a student and each edge is a friendship).  Is it possible for the students to sit around a round table in such a way that every student sits between two friends?  What does this question have to do with paths?

\begin{center}
\tikz{
	\foreach \x in {1,...,9}{
	\coordinate (v\x) at (90-\x*360/9:1.5);
	\draw (v\x) \v;
	}
	\draw (v1) -- (v6) -- (v3) -- (v8) -- (v4) -- (v7) -- (v2) -- (v5) -- (v9) -- (v1);
	\draw (v1) -- (v3) -- (v5) (v4) -- (v5) (v4) -- (v7) -- (v6) -- (v9) (v3) -- (v7) (v9) -- (v3);
}
\end{center}


	\begin{answer}
		We are looking for a Hamiltonian cycle, and this graph does have one:
		
		\begin{center}
		\tikz{
			\foreach \x in {1,...,9}{
			\coordinate (v\x) at (90-\x*360/9:1.5);
			}			
			\draw[color=gray] (v1) -- (v3) -- (v5) (v4) -- (v5) (v4) -- (v7) -- (v6) -- (v9) (v3) -- (v7) (v9) -- (v3);
			\draw[line width=1.2pt, color=blue] (v1) -- (v6) -- (v3) -- (v8) (v8) -- (v4) (v4) -- (v7) -- (v2) -- (v5) -- (v9) -- (v1);
			\foreach \x in {1,...,9}{
			\draw (v\x) \v;
			}
		}
		\end{center}
	\end{answer}
\end{questions}