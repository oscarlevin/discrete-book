
\begin{questions}
\question Use induction to prove for all $n \in \N$ that $\d\sum_{k=0}^n 2^k = 2^{n+1} - 1$.

	\begin{answer}
		\begin{proof}
		 We must prove that $1 + 2 + 2^2 + 2^3 + \cdots +2^n = 2^{n+1} - 1$ for all $n \in \N$.  Thus let $P(n)$ be the statement $1 + 2 + 2^2 + \cdots + 2^n = 2^{n+1} - 1$.  We will prove that $P(n)$ is true for all $n \in \N$.
		 
		 First we establish the base case, $P(0)$, which claims that $1 = 2^{0+1} -1$.  Since $2^1 - 1 = 2 - 1 = 1$, we see that $P(0)$ is true.
		 
		 Now for the inductive case.  Assume that $P(k)$ is true for an arbitrary $k \in \N$.  That is, $1 + 2 + 2^2 + \cdots + 2^k = 2^{k+1} - 1$.  We must show that $P(k+1)$ is true (i.e., that $1 + 2 + 2^2 + \cdots + 2^{k+1} = 2^{k+2} - 1$).  To do this, we start with the left hand side of $P(k+1)$ and work to the right hand side:
		 \begin{align*}
		  1 + 2 + 2^2 + \cdots + 2^k + 2^{k+1} = &~ 2^{k+1} - 1 + 2^{k+1} & \mbox{ \footnotesize by the inductive hypothesis}\\
		   = & ~2\cdot 2^{k+1} - 1 & \\
		   = &~ 2^{k+2} - 1 &
		 \end{align*}
		Thus $P(k+1)$ is true so by the principle of mathematical induction, $P(n)$ is true for all $n \in \N$.
		\end{proof}
	\end{answer}
	
	
	
	
\question Prove that $7^n - 1$ is a multiple of 6 for all $n \in \N$.

	\begin{answer}
		\begin{proof}
		 Let $P(n)$ be the statement ``$7^n - 1$ is a multiple of 6.''  We will show $P(n)$ is true for all $n \in \N$.  
		 
		 First we establish the base case, $P(0)$.  Since $7^0 - 1 = 0$, and $0$ is a multiple of 6, $P(0)$ is true.
		 
		 Now for the inductive case.  Assume $P(k)$ holds for an arbitrary $k \in \N$.  That is, $7^k - 1$ is a multiple of 6, or in other words, $7^k - 1 = 6j$ for some integer $j$.  Now consider $7^{k+1} - 1$:
		 \begin{align*}
		  7^{k+1} - 1 ~ & = 7^{k+1} - 7 + 6 & \mbox{ \footnotesize by cleverness: $-1 = -7 + 6$}\\
		  & = 7(7^k - 1) + 6 & \mbox{ \footnotesize factor out a 7 from the first two terms}\\
		  & = 7(6j) + 6 & \mbox{ \footnotesize by the inductive hypothesis}\\
		  & = 6(7j + 1) & \mbox{ \footnotesize factor out a 6}
		 \end{align*}
		Therefore $7^{k+1} - 1$ is a multiple of 6, or in other words, $P(k+1)$ is true.  Therefore by the principle of mathematical induction, $P(n)$ is true for all $n \in \N$.
		\end{proof}
	\end{answer}
	
	
	

\question Prove that $1 + 3 + 5 + \cdots + (2n-1) = n^2$ for all $n \ge 1$.

	\begin{answer}
		\begin{proof}
		 Let $P(n)$ be the statement $1+3 +5 + \cdots + (2n-1) = n^2$.  We will prove that $P(n)$ is true for all $n \ge 1$.  
		 
		 First the base case, $P(1)$.  We have $ 1 = 1^2$ which is true, so $P(1)$ is established.
		 
		 Now the inductive case.  Assume that $P(k)$ is true for some fixed arbitrary $k \ge 1$.  That is, $1 + 3 + 5 + \cdots + (2k-1) = k^2$.  We will now prove that $P(k+1)$ is also true (i.e., that $1 + 3 + 5 + \cdots + (2k+1) = (k+1)^2$).  We start with the left hand side of $P(k+1)$ and work to the right hand side:
		 \begin{align*}
		  1 + 3 + 5 + \cdots + (2k-1) + (2k+1) ~ & = k^2 + (2k+1) & \mbox{ \footnotesize by the induction hypothesis}\\
		  & = (k+1)^2 & \mbox{ \footnotesize by factoring}
		 \end{align*}
		Thus $P(k+1)$ holds, so by the principle of mathematical induction, $P(n)$ is true for all $n \ge 1$.
		\end{proof}
	\end{answer}
	
	
	

\question Prove that $F_0 + F_2 + F_4 + \cdots + F_{2n} = F_{2n+1} - 1$ where $F_n$ is the $n$th Fibonacci number.

	\begin{answer}
		\begin{proof}
		 Let $P(n)$ be the statement $F_0 + F_2 + F_4 + \cdots + F_{2n} = F_{2n+1} - 1$.  We will show that $P(n)$ is true for all $n \ge 0$.  First the base case is easy because $F_0 = 0$ and $F_1 = 1$ so $F_0 = F_1 - 1$.  Now consider the inductive case.  Assume $P(k)$ is true, that is, assume $F_0 + F_2 + F_4 + \cdots + F_{2k} = F_{2k+1} - 1$.  To establish $P(k+1)$ we work from left to right:
		 \begin{align*}
		  F_0 + F_2 + F_4 + \cdots + F_{2k} + F_{2k+2} ~ & = F_{2k+1} - 1 + F_{2k+2} & \mbox{\footnotesize by the inductive hypothesis}\\
		  & = F_{2k+1} + F_{2k+2} - 1 & \\
		  & = F_{2k+3} - 1 & \mbox{\footnotesize by the recursive definition of the Fibonacci numbers}
		 \end{align*}
		Therefore $F_0 + F_2 + F_4 + \cdots + F_{2k+2} = F_{2k+3} - 1$, which is to say $P(k+1)$ holds.  Therefore by the principle of mathematical induction, $P(n)$ is true for all $n \ge 0$.
		\end{proof}
	\end{answer}
	
	
	

\question Prove that $2^n < n!$ for all $n \ge 4$.  (Recall, $n! = 1\cdot 2 \cdot 3 \cdot \cdots\cdot n$.)

	\begin{answer}
		\begin{proof}
		 Let $P(n)$ be the statement $2^n < n!$.  We will show $P(n)$ is true for all $n \ge 4$.  First, we check the base case and see that yes, $2^4 < 4!$ (as $16 < 24$) so $P(4)$ is true.  Now for the inductive case.  Assume $P(k)$ is true for an arbitrary $k \ge 4$.  That is, $2^k < k!$.  Now consider $P(k+1)$: $2^{k+1} < (k+1)!$.  To prove this, we start with the left side and work to the right side.
		 \begin{align*}
		  2^{k+1}~ & = 2\cdot 2^k & \\
		  & < 2\cdot k! & \mbox{ \footnotesize by the inductive hypothesis}\\
		  & < (k+1) \cdot k! & \mbox{ \footnotesize since $k+1 > 2$}\\
		  & = (k+1)! &
		 \end{align*}
		Therefore $2^{k+1} < (k+1)!$ so we have established $P(k+1)$.  Thus by the principle of mathematical induction $P(n)$ is true for all $n \ge 4$.
		\end{proof}
	\end{answer}
	
	
	


\question What is wrong with the following ``proof'' of the ``fact'' that $n+3 = n+7$ for all values of $n$ (besides of course that the thing it is claiming to prove is false)? 
  \begin{proof}
    Let $P(n)$ be the statement that $n + 3 = n + 7$.  We will prove that $P(n)$ is true for all $n \in \N$.  Assume, for induction that $P(k)$ is true.  That is, $k+3 = k+7$.  We must show that $P(k+1)$ is true.  Now since $k + 3 = k + 7$, add 1 to both sides.  This gives $k + 3 + 1 = k + 7 + 1$.  Regrouping $(k+1) + 3 = (k+1) + 7$.  But this is simply $P(k+1)$.  Thus by the principle of mathematical induction $P(n)$ is true for all $n \in \N$.
  \end{proof}
  
  	\begin{answer}
  		The only problem is that we never established the base case.  Of course, when $n = 0$, $0+3 \ne 0+7$.
  	\end{answer}
  	
  	
  	
  

\question The proof in the previous problem does not work.  But if we modify the ``fact,'' we can get a working proof.  Prove that $n + 3 < n + 7$ for all values of $n \in \N$.  You can do this proof with algebra (without induction), but the goal of this exercise is to write out a valid induction proof.

	\begin{answer}
		\begin{proof}
		    Let $P(n)$ be the statement that $n + 3 < n + 7$.  We will prove that $P(n)$ is true for all $n \in \N$.  First, note that the base case holds: $0+3 < 0+7$.  Now assume for induction that $P(k)$ is true.  That is, $k+3 < k+7$.  We must show that $P(k+1)$ is true.  Now since $k + 3 < k + 7$, add 1 to both sides.  This gives $k + 3 + 1 < k + 7 + 1$.  Regrouping $(k+1) + 3 < (k+1) + 7$.  But this is simply $P(k+1)$.  Thus by the principle of mathematical induction $P(n)$ is true for all $n \in \N$.
		\end{proof}
	\end{answer}
	
	
	

  
\question  Find the flaw in the following ``proof'' of the ``fact'' that $n < 100$ for every $n \in \N$.
 \begin{proof}
  Let $P(n)$ be the statement $n < 100$.  We will prove $P(n)$ is true for all $n \in \N$. First we establish the base case: when $n = 0$, $P(n)$ is true, because $0 < 100$.  Now for the inductive step, assume $P(k)$ is true.  That is, $k < 100$.  Now if $k < 100$, then $k$ is some number, like 80.  Of course $80+1 = 81$ which is still less than 100.  So $k +1 < 100$ as well.  But this is what $P(k+1)$ claims, so we have shown that $P(k) \imp P(k+1)$.  Thus by the principle of mathematical induction, $P(n)$ is true for all $n \in \N$.
 \end{proof}
 
 	\begin{answer}
 		The problem here is that while $P(0)$ is true, and while $P(k) \imp P(k+1)$ for {\em some} values of $k$, there is at least one value of $k$ (namely $k = 99$) when that implication fails.  For a valid proof by induction, $P(k) \imp P(k+1)$ must be true for all values of $k$ greater than or equal to the base case.
 	\end{answer}
 	
 	
 	
 

\question While the above proof does not work (it better not - the statement it is trying to prove is false!)  we can prove something similar.  Prove that there is a strictly increasing sequence $a_1, a_2, a_3, \ldots$ of numbers (not necessarily integers) such that $a_n < 100$ for all $n \in \N$.  (By {\em strictly increasing} we mean $a_n < a_{n+1}$ for all $n$ - so each term must be larger than the last.)

	\begin{answer}
		\begin{proof}
		 Let $P(n)$ be the statement ``there is a strictly increasing sequence $a_1, a_2, a_3, \ldots, a_n$ with $a_n < 100$.''  We will prove $P(n)$ is true for all $n \ge 1$. First we establish the base case: $P(1)$ says there is a single number $a_1$ with $a_1 < 100$.  This is true - take $a_1 = 0$.  Now for the inductive step, assume $P(k)$ is true.  That is there exists a strictly increasing sequence $a_1, a_2, a_3, \ldots, a_k$ with $a_k < 100$.  Now consider this sequence, plus one more term, $a_{k+1}$ which is greater than $a_k$ but less than $100$.  Such a number exists - for example, the average between $a_k$ and 100.  So then $P(k+1)$ is true, so we have shown that $P(k) \imp P(k+1)$.  Thus by the principle of mathematical induction, $P(n)$ is true for all $n \in \N$.
		\end{proof}
		
	\end{answer}
	
	
	



\question What is wrong with the following ``proof'' of the ``fact'' that for all $n \in \N$, the number $n^2 + n$ is odd?
  \begin{proof}
    Let $P(n)$ be the statement ``$n^2 + n$ is odd.''  We will prove that $P(n)$ is true for all $n \in \N$.  Suppose for induction that $P(k)$ is true, that is, that $k^2 + k$ is odd.  Now consider the statement $P(k+1)$.  Now $(k+1)^2 + (k+1) = k^2 + 2k + 1 + k + 1 = k^2 + k + 2k + 2$.  By the inductive hypothesis, $k^2 + k$ is odd, and of course $2k + 2$ is even.  An odd plus an even is always odd, so therefore $(k+1)^2 + (k+1)$ is odd.  Therefore by the principle of mathematical induction, $P(n)$ is true for all $n \in \N$.
  \end{proof}
  
  	\begin{answer}
  		We once again failed to establish the base case: when $n = 0$, $n^2 + n = 0$ which is even, not odd.
  	\end{answer}
  	
  	
  	
  

\question Now give a valid proof (by induction - even though you might be able to do so without using induction) of the statement, ``for all $n \in \N$, the number $n^2 + n$ is even.''

	\begin{answer}
		  \begin{proof}
		    Let $P(n)$ be the statement ``$n^2 + n$ is even.''  We will prove that $P(n)$ is true for all $n \in \N$.  First the base case: when $n = 0$, we have $n^2 + n = 0$ which is even, so $P(0)$ is true.  Now suppose for induction that $P(k)$ is true, that is, that $k^2 + k$ is even.  Now consider the statement $P(k+1)$.  Now $(k+1)^2 + (k+1) = k^2 + 2k + 1 + k + 1 = k^2 + k + 2k + 2$.  By the inductive hypothesis, $k^2 + k$ is even, and of course $2k + 2$ is even.  An even plus an even is always even, so therefore $(k+1)^2 + (k+1)$ is even.  Therefore by the principle of mathematical induction, $P(n)$ is true for all $n \in \N$.
		  \end{proof}
	\end{answer}
	
	
	


\question Prove that there is a sequence of positive real numbers $a_1, a_2, a_3, \ldots$ such that the partial sum $a_1 + a_2 + a_3 + \cdots + a_n$ is strictly less than $2$ for all $n \in \N$.  Hint: think about how you could define what $a_{k+1}$ is to make the induction argument work.

	\begin{answer}
		 Further hint: the idea is to define the sequence so that $a_n$ is less than the distance between the previous partial sum and 2.  That way when you add it into the next partial sum, the partial sum is still less than 2.  You could do this ahead of time, or use a clever $P(n)$ in the induction proof.  Let $P(n)$ be the statement, ``there is a sequence of positive real numbers $a_1, a_2, a_3, \ldots, a_n$ such that $a_1 + a_2 + a_3 + \cdots + a_n < 2$.''  The base case should be easy (just pick $a_1 < 2$).  For the inductive case, you know that $a_1 + a_2 + \cdots + a_k < 2$ so you just need to argue that you can find some $a_{k+1}$ small enough to have $a_1 + a_2 + \cdots +a_k + a_{k+1} < 2$.
	\end{answer}
	
	
\question Prove that every natural number is either a power of 2, or can be written as the sum of distinct powers of 2.

	\begin{answer}
		The base case should be easy - 0 is a power of 2.  For the inductive case, you actually want to use strong induction.  Suppose $k$ is either a power of 2 or can be written as the sum of distinct powers of 2, for any $k < n$.  Now if $n$ is a power of 2, we are done.  If not, subtract the largest power of 2 from $n$ possible.  You get $n - 2^x$, which is a smaller number, in fact smaller than both $n$ and $2^x$.  Thus $n-2^x$ is either a power of 2 or can be written as the sum of distinct powers of 2, but none of them are going to be $2^x$, so the together with $2^x$ we have written $n$ as the sum of distinct powers of 2.
	\end{answer}	

\question  Use induction to prove that if $n$ people all shake hands with each other, that the total number of handshakes is $\frac{n(n-1)}{2}$. 

	\begin{answer}
	  If $n = 2$, this should work out (so their's your base case).  If we assume it works for $k$ people (that the number of handshakes is $\frac{k(k-1)}{2}$, what happens if a $k+1$st person shows up.  How many {\em new} handshakes take place?  Now make this into a formal induction argument.
	  
	  Note, we have already proven this without using induction, but this is fun too.
	\end{answer}

\question Prove that if $G$ is a connected planar graph with $V$ vertices, $E$ edges, and $F$ faces, then $V - E + F = 2$.  Hint: use induction on the number of edges in the graph.

	\begin{answer}
		With just one edge, it should be easy to verify the equation.  If the equation holds with $k$ edges, what will happen to $V$, $E$, and $F$ if we add and edge?  This can be done in two ways - either adding a new vertex, or adding a new face.  Explain how this works, and what happens to the equation.
	\end{answer} 
	
\question Use induction to prove that $\d\sum_{k=0}^n {n \choose k} = 2^n$.  That is, the sum of the $n$th row of Pascal's Triangle is $2^n$.

	\begin{answer}
		Here's the idea: since every entry in Pascal's Triangle is the sum of the two entries above it, we can get the $k+1$st row by adding up all the pairs of entry from the $k$th row.  But doing this uses each entry on the $k$th row twice.  Thus each time we drop to the next row, we double the total.  Of course, row 0 has sum $1 = 2^0$ (the base case).  Now try to make this precise with a formal induction proof.  You will use the fact that ${n \choose k} = {n-1 \choose k-1} + {n-1 \choose k}$ for the inductive case.
	\end{answer}
	
\question Use induction to prove ${4 \choose 0} + {5 \choose 1} + {6 \choose 2} + \cdots + {4+n \choose n} = {5+n \choose n}$.  (This is an example of the hockey stick theorem.)

	\begin{answer}
		To see why this works, try it on a copy of Pascal's triangle.  We are adding up the entries along a diagonal, starting with the 1 on the left hand side of the 4th row.  Suppose we add up the first 5 entries on this diagonal.  The claim is that the sum is the entry below and to the left of the last of these 5 entries.  Note that if this is true, and we instead add up the first 6 entries, we will need to add the entry one spot to the right of the previous sum.  But these two together give the entry below them, which is below and left of the last of the 6 entries on the diagonal.
		
		If you follow that, you can see what is going on.  But it is not a great proof.  A formal induction proof is needed:
		
		\begin{proof}
			Let $P(n)$ be the statement ${4 \choose 0} + {5 \choose 1} + {6 \choose 2} + \cdots + {4+n \choose n} = {5+n \choose n}$.  For the base case, consider $n = 0$.  This says ${4 \choose 0} = {5 \choose 0}$.  Since these are both 1, the base case is true.  Now for the inductive case, suppose $P(k)$ is true.  That is, ${4 \choose 0} + {5 \choose 1} + {6 \choose 2} + \cdots + {4+k \choose k} = {5+k \choose k}$.  If we add ${4+k+1 \choose k+1}$ to both sides, we get \[{4 \choose 0} + {5 \choose 1} + {6 \choose 2} + \cdots + {4+k \choose k} + {5+k \choose k+1}= {5+k \choose k} + {5+k \choose k+1}\]
			But ${5+k \choose k} + {5+k \choose k+1} = {5+k+1 \choose k+1}$.  In other words, we have
			\[{4 \choose 0} + {5 \choose 1} + {6 \choose 2} + \cdots + {4+k \choose k} + {5+k \choose k+1} = {5+k+1 \choose k+1}\]
			which is to say that $P(k+1)$ is true.
			
			Therefore, by the principle of mathematical induction, $P(n)$ is true for all $n \ge 0$.
		\end{proof}
	\end{answer}
	
	
\question Use the product rule for logarithms ($\log(ab) = \log(a) + \log(b)$) to prove, by induction on $n$, that $\log(a^n) = n \log(a)$, for all natural numbers $n \ge 2$.

	\begin{answer}
		The idea here is that if we take the logarithm of $a^n$, we can increase $n$ by 1 if we multiply by another $a$ (inside the logarithm).  This results in adding 1 more $\log(a)$ to the total.
		
		\begin{proof}
			Let $P(n)$ be the statement $\log(a^n) = n \log(a)$.  The base case, $P(2)$ is true, because $\log(a^2) = \log(a\cdot a) = \log(a) + \log(a) = 2\log(a)$, by the product rule for logarithms.
			
			Now assume, for induction, that $P(k)$ is true.  That is, $\log(a^k) = k\log(a)$.  Consider $\log(a^{k+1})$.  We have
			\[\log(a^{k+1}) = \log(a^k\cdot a) = \log(a^k) + \log(a) = k\log(a) + \log(a)\]
			with the last equality due to the inductive hypothesis.  But this simplifies to $(k+1) \log(a)$, establishing $P(k+1)$.
			
			Therefore by the principle of mathematical induction, $P(n)$ is true for all $n \ge 2$.	
		\end{proof}
	\end{answer}
	
	
\question Let $f_1, f_2,\ldots, f_n$ be differentiable functions.  Prove, using induction, that
\[(f_1 + f_2 + \cdots + f_n)' = f_1' + f_2' + \cdots + f_n'\]
You may assume $(f+g)' = f' + g'$ for any differentiable functions $f$ and $g$.

	\begin{answer}
		Hint: You are allowed to assume the base case.  For the inductive case, group all but the last function together as one sum of functions, then apply the usual sum of derivatives rule, and then the inductive hypothesis.
	\end{answer}


\question Suppose $f_1, f_2, \ldots, f_n$ are differentiable functions.  Use mathematical induction to prove the generalized product rule: 
\[(f_1 f_2 f_3 \cdots f_n)' = f_1' f_2 f_3 \cdots f_n + f_1 f_2' f_3 \cdots f_n + f_1 f_2 f_3' \cdots f_n + \cdots + f_1 f_2 f_3 \cdots f_n'\]
You may assume the product rule for two functions is true.

	\begin{answer}
		Hint: for the inductive step, we know by the product rule for two functions that \[(f_1f_2f_3 \cdots f_k f_{k+1})' = (f_1f_2f_3\cdots f_k)'f_{k+1} + (f_1f_2f_3\cdots f_k)f_{k+1}'\]
		Then use the inductive hypothesis on the first summand, and distribute.
	\end{answer}


 
\end{questions}



