\documentclass[11pt]{exam} 
\usepackage{answers, amsthm, amsmath, amssymb, mathrsfs} \pagestyle{head} \firstpageheader{Math 228}{\bf Practice Problems 7: Sequences\\ Hints and Answers}{Spring 2013} \Newassociation{answer}{Ans}{Practice07-Sequences-solutions} 
 \def\d{\displaystyle}
\def\?{\reflectbox{?}}
\def\b#1{\mathbf{#1}}
\def\f#1{\mathfrak #1}
\def\c#1{\mathcal #1}
\def\s#1{\mathscr #1}
\def\r#1{\mathrm{#1}}
\def\N{\mathbb N}
\def\Z{\mathbb Z}
\def\Q{\mathbb Q}
\def\R{\mathbb R}
\def\C{\mathbb C}
\def\F{\mathbb F}
\def\A{\mathbb A}
\def\X{\mathbb X}
\def\E{\mathbb E}
\def\O{\mathbb O}
\def\U{\mathcal U}
\def\pow{\mathcal P}
\def\inv{^{-1}}
\def\nrml{\triangleleft}
\def\st{:}
\def\~{\widetilde}
\def\rem{\mathcal R}
\def\sigalg{$\sigma$-algebra }
\def\Gal{\mbox{Gal}}
\def\iff{\leftrightarrow}
\def\Iff{\Leftrightarrow}
\def\land{\wedge}
\def\And{\bigwedge}
\def\AAnd{\d\bigwedge\mkern-18mu\bigwedge}
\def\Vee{\bigvee}
\def\VVee{\d\Vee\mkern-18mu\Vee}
\def\imp{\rightarrow}
\def\Imp{\Rightarrow}
\def\Fi{\Leftarrow}

%\def\={\equiv}
\def\var{\mbox{var}}
\def\mod{\mbox{Mod}}
\def\Th{\mbox{Th}}
\def\sat{\mbox{Sat}}
\def\con{\mbox{Con}}
\def\bmodels{=\joinrel\mathrel|}
\def\iffmodels{\bmodels\models}
\def\dbland{\bigwedge \!\!\bigwedge}
\def\dom{\mbox{dom}}
\def\rng{\mbox{range}}
\DeclareMathOperator{\wgt}{wgt}


\def\bar{\overline}


\newcommand{\vtx}[2]{node[fill,circle,inner sep=0pt, minimum size=4pt,label=#1:#2]{}}
\newcommand{\va}[1]{\vtx{above}{#1}}
\newcommand{\vb}[1]{\vtx{below}{#1}}
\newcommand{\vr}[1]{\vtx{right}{#1}}
\newcommand{\vl}[1]{\vtx{left}{#1}}
\renewcommand{\v}{\vtx{above}{}}

\def\circleA{(-.5,0) circle (1)}
\def\circleAlabel{(-1.5,.6) node[above]{$A$}}
\def\circleB{(.5,0) circle (1)}
\def\circleBlabel{(1.5,.6) node[above]{$B$}}
\def\circleC{(0,-1) circle (1)}
\def\circleClabel{(.5,-2) node[right]{$C$}}
\def\twosetbox{(-2,-1.4) rectangle (2,1.4)}
\def\threesetbox{(-2.5,-2.4) rectangle (2.5,1.4)}
\newcommand{\twoline}[2]{\begin{pmatrix}#1 \\ #2 \end{pmatrix}}

\usepackage{tikz, multicol}
\renewenvironment{Ans}[1]{\setcounter{question}{#1}\addtocounter{question}{-1}\question }{}
\begin{document}
 \begin{questions}
\begin{Ans}{1}
		\begin{parts}
		%$2, 5, 10, 17, 26, \ldots$
		  \part $a_n = n^2 + 1$
		%   $0, 2, 5, 9, 14, 20, \ldots$
		  \part $a_n = \frac{n(n+1)}{2} - 1$
		%   $8, 12, 17, 23, 30,\ldots$
		  \part $a_n = \frac{(n+2)(n+3)}{2} + 2$
		%   $1, 5, 23, 119, 719,\ldots$
		  \part $a_n = (n+1)! - 1$ (where $n! = 1 \cdot 2 \cdot 3 \cdots n$)
		\end{parts}
	
\end{Ans}
\begin{Ans}{2}
		\begin{parts}
		%Give the recursive definition for the sequence.
		  \part $F_n = F_{n-1} + F_{n-2}$ with $F_0 = 0$ and $F_1 = 1$.
		%   Write out the first few terms of the sequence of partial sums.
		  \part  $0, 1, 2, 4, 7, 12, 20, \ldots$
		  %Give a closed formula for the sequence of partial sums in terms of $F_n$  (for example, you might say $F_0 + F_1 + \cdots + F_n = 3F_{n-1}^2 + n$, although that is definitely not correct).
		  \part $F_0 + F_1 + \cdots + F_n = F_{n+2} - 1$
		\end{parts}
	
\end{Ans}
\begin{Ans}{3}
		$3, 10, 24, 52, 108,\ldots$.  The recursive definition for $10, 24, 52, \ldots$ is $a_n = 2a_{n-1} + 4$ with $a_1 = 10$.
	
\end{Ans}
\begin{Ans}{4}
		$-1, -1, 1, 5, 11, 19,\ldots$  Thus the sequence $0, 2, 6, 12, 20,\ldots$ has closed formula $a_n = (n+1)^2 - 3(n+1) + 2$.
	
\end{Ans}
\begin{Ans}{5}
		\begin{parts}
		% What is the next term in the sequence?
		\part 32.
		% Find a formula for the $n$th term of this sequence, assuming $a_1 = 8$.
		\part $a_n = 8 + 6(n-1)$
		% Find the sum of the first 100 terms of the sequence: $\sum_{k=1}^{100}a_k$.
		\part $30500$.
		\end{parts}
	
\end{Ans}
\begin{Ans}{6}
		\begin{parts}
		% How many terms are there in the sequence?
		\part $n+2$ terms.
		\part $6n+1$. %second to last term
		%Find the sum of all the terms in the sequence.
		\part $\frac{(6n+8)(n+2)}{2}$
		\end{parts}
	
\end{Ans}
\begin{Ans}{7}
		68117
	
\end{Ans}
\begin{Ans}{8}
		$\frac{5-5\cdot 3^{21}}{-2}$
	
\end{Ans}
\begin{Ans}{9}
		$\frac{1 + \frac{2^{31}}{3^{31}}}{5/3}$
	
\end{Ans}
\begin{Ans}{10}
		For arithmetic: $x = 55/3$, $y = 29/3$.  For geometric: $x = 9$ and $y = 3$.
	
\end{Ans}
\begin{Ans}{11}
		\begin{parts}
		  \part $\d\sum_{k=1}^n 2k$		%$2 + 4 + 6 + 8 + \cdots + 2n$
		  \part $\d\sum_{k=1}^{107} (1 + 4(k-1))$		%$1 + 5 + 9 + 13 + \cdots + 425$
		  \part $\d\sum_{k=1}^{50} \frac{1}{k}$		%$1 + \frac{1}{2} + \frac{1}{3} + \frac{1}{4} + \cdots + \frac{1}{50}$
		  \part $\d\prod_{k=1}^n 2k$		%$2 \cdot 4 \cdot 6 \cdot \cdots \cdot 2n$
		  \part $\d\prod_{k=1}^{100} \frac{k}{k+1}$	%$(\frac{1}{2})(\frac{2}{3})(\frac{3}{4})\cdots(\frac{100}{101})$
		\end{parts}
	
\end{Ans}
\begin{Ans}{12}
		\begin{parts}
		  \part $\d\sum_{k=1}^{100} (3+4k) = 7 + 11 + 15 + \cdots + 403$
		  \part $\d\sum_{k=0}^n 2^k = 1 + 2 + 4 + 8 + \cdots + 2^n$
		  \part $\d\sum_{k=2}^{50}\frac{1}{(k^2 - 1)} = 1 + \frac{1}{3} + \frac{1}{8} + \frac{1}{15} + \cdots + \frac{1}{2499}$
		  \part $\d\prod_{k=2}^{100}\frac{k^2}{(k^2-1} = \frac{4}{3}\cdot\frac{9}{8}\cdot\frac{16}{15}\cdots\frac{10000}{9999}$
		  \part $\d\prod_{k=0}^n (2+3k) = (2)(5)(8)(11)(14)\cdots(2+3n)$
		\end{parts}
	
\end{Ans}
\begin{Ans}{13}
		\begin{parts}
		\part Hint: third differences are constant, so $a_n = an^3 + bn^2 + cn + d$.  Use the terms of the sequence to solve for $a, b, c,$ and $d$.
		\part $a_n = n^2 - n$
		\end{parts}
	
\end{Ans}
\begin{Ans}{14}
		No.  The sequence of differences is the same as the original sequence so no differences will be constant.
	
\end{Ans}
\begin{Ans}{15}
		171 and 341.  $a_n = a_{n-1} + 2a_{n-2}$ with $a_0 = 3$ and $a_1 = 5$.  Closed formula: $a_n = \frac{8}{3}2^n + \frac{1}{3}(-1)^n$
	
\end{Ans}
\begin{Ans}{16}
		By telescoping or iteration.  $a_n = 3 + 2^{n+1}$
	
\end{Ans}
\begin{Ans}{17}
		We claim $a_n = 4^n$ works.  Plug it in: $4^n = 3(4^{n-1}) + 4(4^{n-2})$.  This works - just simplify the right hand side.
	
\end{Ans}
\begin{Ans}{18}
		By the Characteristic Root Technique.  $a_n = 4^n + (-1)^n$.
	
\end{Ans}
\begin{Ans}{19}
		$a_n = \frac{13}{5} 4^n + \frac{12}{5} (-1)^n$
	
\end{Ans}
\begin{Ans}{20}
		The general solution is $a_n = a + bn$ where $a$ and $b$ depend on the initial conditions.  %Solve the recurrence relation $a_n = 2a_{n-1} - a_{n-2}$.
		\begin{parts}
		  \part $a_n = 1 + n$
		  %What is the solution if the initial terms are $a_0 = 1$ and $a_1 = 2$?
		  \part For example, we could have $a_0 = 21$ and $a_1 = 22$.  %What do the initial terms need to be in order for $a_9 = 30$?
		  \part For every $x$ - take $a_0 = x-9$ and $a_1 = x-8$.  %For which $x$ are there initial terms which make $a_9 = x$?
		\end{parts}
	
\end{Ans}
\begin{Ans}{21}
		$a_n = \frac{19}{7}(-2)^n + \frac{9}{7}5^n$
		%Solve the recurrence relation $a_n = 3a_{n-1} + 10a_{n-2}$ with initial terms $a_0 = 4$ and $a_1 = 1$.
	
\end{Ans}
 \end{questions} \par \end{document}
