\section{Exam 2 Study Guide}

Exam 2 (on Friday, March 15) will cover everything we have discussed since the last exam.  That means counting.  Also combinatorial proofs, sets and functions (which we used to count). Here is a more detailed list of topics:

\begin{itemize}
 \item Additive and Multiplicative Principles
 \item Principle of Inclusion/Exclusion
 \item Binomial coefficients: ${n \choose k}$
 \item Combinations and Permutations
 \item Stars and Bars
 \item Advanced counting using PIE
 \item Things to count:
 \begin{itemize}
 	\item Bit strings.
 	\item Subsets.
 	\item Lattice Paths.
 	\item Pizzas.
 	\item Functions.
 	\item Injections.
 	\item Surjections.
 	\item Derangements.
 \end{itemize}
 \item Combinatorial Proofs
\end{itemize}


The quizzes, worksheets, practice problems, and turn-in homework should give you a good idea of the types of questions to expect.  

Additionally, the questions below would all make fine exam questions.  In fact, some of them are taken directly off of discrete exams from last semester.\footnote{Disclaimer: Question on the actual exam may be easier or harder than those given her.  There might be types of questions on this study guide not covered on the exam and questions on the exam not covered in this study guide.  Questions on the exam might be asked in a different way than here.  If solving a question lasts longer than four hours, contact your professor immediately.}  




\begin{squestions}
\question Give a counting question where the answer is $8\cdot 3 \cdot 3 \cdot 5$.  Give another question where the answer is $8 + 3 + 3 + 5$.

	\begin{answer}
		You own 8 purple bow ties, 3 red bow ties, 3 blue bow ties and 5 green bow ties.  How many ways can you select one of each color bow tie to take with you on a trip?  $8 \cdot 3 \cdot 3 \cdot 5$.  How many choices do you have for a single bow tie to wear tomorrow?  $8 + 3 + 3 + 5$.
	\end{answer}
	
	

\question Consider five digit numbers $\alpha = a_1a_2a_3a_4a_5$, with each digit from the set $\{1,2,3,4\}$.
\begin{parts}
\part How many such numbers are there?
\part How many such numbers are there for which the {\em sum} of the digits is even?
\part How many such numbers contain more even digits than odd digits?
\end{parts}

	\begin{answer}
		\begin{parts}
		\part $4^5$. %How many such numbers are there?
		\part $4^4\cdot 2$ (choose any digits for the first four digits - then pick either an even or an odd last digit to make the sum even). %How many such numbers are there for which the {\em sum} of the digits is even?
		\part We need 3 or more even digits.  3 even digits: ${5 \choose 3}2^3 2^2$.  4 even digits: ${5 \choose 4}2^4 2$.  5 even digits: ${5 \choose 5}2^5$.  So all together: ${5 \choose 3}2^3 2^2 + {5 \choose 4}2^4 2 + {5 \choose 5}2^5$.  %  How many such numbers contain more even digits than odd digits?
		\end{parts}
	\end{answer}
	
	
	
\question Let $A$ and $B$ be finite sets.  Explain, in words, why $|A \cup B| \le |A| + |B|$ and why $|A \cup B| = |A| + |B| - |A \cap B|$.

	\begin{answer}
		$|A \cup B|$ is the number of things that are in $A$ or in $B$ or in both.  If you count up everything in each set independently, then anything which is in both sets (in $A \cap B$) is counted twice.
	\end{answer}
	
	
	
\question For how many $n \in \{1,2, \ldots, 500\}$ is $n$ a multiple of one or more of 5, 6, or 7?

	\begin{answer}
		215.  Use PIE: $100 + 83 + 71 - 16 - 14 -11 + 2 = 215$ or a Venn diagram.  To find out how many numbers are divisible by 6 and 7, for example, take $500/42$ and round down.
	\end{answer}
	
	


\question In a recent small survey of airline passengers, 25 said they had flown American in the last year, 30 had flown Jet Blue, and 20 had flown Continental.  Of those, 10 reported they had flown on American and Jet Blue, 12 had flown on Jet Blue and Continental, and 7 had flown on American and Continental.  5 passengers had flown on all three airlines. 

How many passengers were surveyed?  (Assume the results above make up the entire survey.)

	\begin{answer}
		51.
	\end{answer}
	
	


	

\question Recall, by $8$-bit strings, we mean strings of binary digits, of length 8.
\begin{parts}
  \part How many $8$-bit strings are there total?
  \part How many $8$-bit strings have weight 5?
  \part How many subsets of the set $\{a,b,c,d,e,f,g,h\}$ contain exactly 5 elements?
  \part Explain why your answers to parts (b) and (c) are the same.  Why are these questions equivalent?
\end{parts}

	\begin{answer}
		\begin{parts}
		  \part $2^8$. %How many $8$-bit strings are there total?
		  \part ${8 \choose 5}$  %How many $8$-bit strings have weight 5?
		  \part ${8 \choose 5}$ %How many subsets of the set $\{a,b,c,d,e,f,g,h\}$ contain exactly 5 elements?
		  \part There is a bijection between subsets and bit strings: a 1 means that element in is the subset, a 0 means that element is not in the subset.  To get a subset of an 8 element set we have a 8-bit string.  To make sure the subset contains exactly 5 elements, there must be 5 1's, so the weight must be 5. %Explain why your answers to parts (b) and (c) are the same.  Why are these questions equivalent?
		\end{parts}
	\end{answer}
	
	

\question What is the coefficient of $x^{10}$ in the expansion of $(x+1)^{13} + x^2(x+1)^{17}$?

	\begin{answer}
		${13 \choose 10} + {17 \choose 8}$
	\end{answer}
	
	


\question How many 8-letter words contain exactly 5 vowels (a,e,i,o,u)?  What if repeated letters were not allowed?

	\begin{answer}
		 With repeated letters allowed: ${8 \choose 5}5^5 21^3$.  Without repeats: ${8 \choose 5}5! P(21, 3)$.
	\end{answer}
	
	


\question For each of the following, find the number of shortest lattice paths from $(0,0)$ to $(8,8)$ which:
\begin{parts}
  \part pass through the point $(2,3)$.
  \part avoid (do not pass through) the point $(7,5)$.
  \part either pass through $(2,3)$ or $(5,7)$ (or both).
\end{parts}

	\begin{answer}
		\begin{parts}
		  \part ${5 \choose 2}{11 \choose 6}$ %pass through the point $(2,3)$.
		  \part ${16 \choose 8} - {12 \choose 7}{4 \choose 1}$   %avoid (do not pass through) the point $(7,5)$.
		  \part ${5 \choose 2}{11 \choose 6} + {12 \choose 5}{4 \choose 3} - {5 \choose 2}{7 \choose 3}{4 \choose 3}$ %either pass through $(2,3)$ or $(5,7)$ (or both).
		\end{parts}
	\end{answer}
	
	


\question You live in Grid-Town on the corner of 2nd and 3rd, and work in a building on the corner of 10th and 13th.  How many routes are there which take you from home to work and then back home, but by a different route?

	\begin{answer}
		 ${18 \choose 8}\left({18 \choose 8} - 1\right)$ 
	\end{answer}
	
	


\question Give an example of a problem for which $P(n,k)$ is the solution.  Give another example of a problem for which ${n\choose k}$ is the solution.

	\begin{answer}
		 A test had $n$ questions on it, of which you must answer any $k$ questions.  How many choices do you have as to what order you answer the questions on the test?  $P(n,k)$.  When grading the test, how many different combinations of question might the professor see?  ${n \choose k}$.
	\end{answer}
	
	


\question How many 10-bit strings contain 6 or more 1's?

	\begin{answer}
		${10 \choose 6} + {10 \choose 7} + {10 \choose 8} + {10 \choose 9} + {10 \choose 10}$ 
	\end{answer}
	
	


\question How many 10-bit strings start with $111$ or end with $101$ or both?

	\begin{answer}
		 $2^7 + 2^7 - 2^4$.
	\end{answer}
	
	


\question How many 10-bit strings of weight 6 start with $111$ or end with $101$ or both?

	\begin{answer}
		${7 \choose 3} + {7 \choose 4} - {4 \choose 1}$.
	\end{answer}
	
	


\question How many 6 letter words made from the letters $a,b,c,d,e,f$ without repeats do not contain the sub-word ``bad'' in (a) consecutive letters? or (b) not-necessarily consecutive letters (but in order)? 

	\begin{answer}
		(a) $6! - 4\cdot 3!$.  (b) $6! - {6 \choose 3}3!$.
	\end{answer}
	
	


\question Explain using lattice paths why $\sum_{k=0}^n {n \choose k} = 2^n$

	\begin{answer}
		$2^n$ is the number of lattice paths which have length $n$, since for each step you can go up or right.  Such a path would end along the line $x + y = n$.  So you will end at $(0,n)$, or $(1,n-1)$ or $(2, n-2)$ or \ldots or $(n,0)$.  Counting the paths to each of these points separately, give ${n \choose 0}$, ${n \choose 1}$, ${n \choose 2}$, \ldots, ${n \choose n}$ (each time choosing which of the $n$ steps to be to the right).
	\end{answer}
	
	

 
\question Explain the relationship between $\d{n\choose k}$ and $P(n,k)$.  Be sure to say both how the formulas for each are related, and why that relationship makes sense.

	\begin{answer}
		Hint: give a combinatorial proof for the identity $P(n,k) = {n \choose k} k!$.
	\end{answer}
	
	


\question Give your favorite argument for why Pascal's Triangle is symmetric.  That is, explain why \[\d {n \choose k} = {n \choose n-k}\]

	\begin{answer}
		Of your $n$ bow ties, you decide to give $k$ away to charity.  How many ways can you do this?  On one hand, you can choose $k$ of the $n$ bow ties to give away in ${n \choose k}$ ways.  Alternatively, you can choose which bow ties to keep.  You must keep $n -k$ of them if you will give $k$ away, so you can choose the bow ties to keep in ${n \choose n-k}$ ways.  This gives a combinatorial proof for the identity.
	\end{answer}
	
	



\question Suppose you have 20 one-dollar bills to give out as prizes to your top 5 discrete math students.  How many ways can you do this if:
\begin{parts}
  \part each of the 5 students gets at least 1 dollar?
  \part some students might get nothing?
  \part each student gets at least 1 dollar but no more than 7 dollars?
\end{parts}

	\begin{answer}
		Hint: stars and bars%Suppose you have 20 one-dollar bills to give out as prizes to your top 5 discrete math students.  How many ways can you do this if:
		\begin{parts}
		  \part ${19 \choose 4}$ %each of the 5 students gets at least 1 dollar?
		  \part ${24 \choose 4}$ %some students might get nothing?
		  \part ${19 \choose 4} - \left[{5 \choose 1}{12 \choose 4} - {5 \choose 2}{5 \choose 4}  \right]$ %each student gets at least 1 dollar but no more than 7 dollars?
		\end{parts}
	\end{answer}
	


\question What does it mean for a function to be surjective?  What does it mean to be injective.

	\begin{answer}
		A function is surjective if the codomain is equal to the range.  A function is injective if every element of the codomain is the image (output) of at most one element from the domain.
	\end{answer}
	
	
	
	
	
\question If $X$ is a finite set, and $f: X \to Y$ is a injective function, must it also be surjective?

	\begin{answer}
		No, $Y$ could be larger than $X$.
	\end{answer}
	
	
	



\question If $X$ is a finite set, and $f: X \to Y$ is both injective and surjective, what can you say about $Y$?

	\begin{answer}
		You can say that $|Y| = |X|$ (so $Y$ is finite as well).  In fact, you can say $|Y| = |X|$ in this case even if $X$ is not finite (the sets would have the same infinite cardinality).
	\end{answer}
	
	
	

	


\question How many functions $f: \{1,2,3,4,5\} \to \{a,b,c,d,e\}$ are there for which
\begin{parts}
  \part $f(1) = a$ or $f(2) = b$ (or both)?
  \part $f(1) \ne a$ or $f(2) \ne b$ (or both)?
  \part $f(1) \ne a$ {\em and} $f(2) \ne b$, and are also one-to-one?
  \part are onto but have $f(1) \ne a$, $f(2) \ne b$, $f(3) \ne c$, $f(4) \ne d$ and $f(5) \ne e$?
\end{parts}

	\begin{answer}
		\begin{parts}
		  \part $5^4 + 5^4 - 5^3$ %$f(1) = a$ or $f(2) = b$ (or both)?
		  \part $4\cdot 5^4 + 5 \cdot 4 \cdot 5^3 - 4 \cdot 4 \cdot 5^3$ %$f(1) \ne a$ or $f(2) \ne b$ (or both)?
		  \part $5! - \left[ 4! + 4! - 3! \right]$ %$f(1) \ne a$ {\em and} $f(2) \ne b$, and are also one-to-one?
		  \part $5! - \left[{5 \choose 1}4! - {5 \choose 2}3! + {5 \choose 3}2! - {5 \choose 4}1! + {5 \choose 5} 0!\right]$ %are onto but have $f(1) \ne a$, $f(2) \ne b$, $f(3) \ne c$, $f(4) \ne d$ and $f(5) \ne e$?
		\end{parts}
	\end{answer}
	
	


\question How many permutations of $\{1,2,3,4,5\}$ leave exactly 1 element fixed?

	\begin{answer}
		 ${5 \choose 1}\left( 4! - \left[{4 \choose 1}3! - {4 \choose 2}2! + {4 \choose 3} 1! - {4 \choose 4} 0!\right] \right)$
	\end{answer}
	
	


\question How many functions map $\{1,2,3,4,5,6\}$ {\em onto} $\{a,b,c,d\}$

	\begin{answer}
		$4^6 - \left[{4 \choose 1}3^6 - {4 \choose 2}2^6 + {4 \choose 3} 1^6 \right]$
	\end{answer}
	
	


\question To thank your math professor for doing such an amazing job all semester, you decide to bake him (or her) cookies.  You know how to make 10 different types of cookies.
\begin{parts}
 \part If you want to give your professor 4 different types of cookies, how many different combinations of cookie type can you select?  Explain your answer.
 \part To keep things interesting, you decide to make a different number of each type of cookie.  If again you want to select 4 cookie types, how many ways can you select the cookie types and decide for which there will be the most, second most, etc.  Explain your answer.
 \part You change your mind again.  This time you decide you will make a total of 12 cookies.  Each cookie could be any one of the 10 types of cookies you know how to bake (and it's okay if you leave some types out).  How many choices do you have?  Explain.
 \part You realize that the previous plan did not account for presentation.  This time, you once again want to make 12 cookies, each one could be any one of the 10 types of cookies.  However, now you plan to shape the cookies into the numerals 1, 2, \ldots, 12 (and probably arrange them to make a giant clock - but you haven't decided on that yet).  How many choices do you have for which types of cookies to bake into which numerals?  Explain.
\end{parts}

	\begin{answer}
		\begin{parts}
		 \part ${10 \choose 4}$.  You need to choose 4 of the 10 cookie types.  Order doesn't matter. %If you want to give your professor 4 different types of cookies, how many different combinations of cookie type can you select?  Explain your answer.
		 \part $P(10, 4) = 10 \cdot 9 \cdot 8 \cdot 7$.  You are choosing and arranging 4 out of 10 cookies.  Order matters now.  %To keep things interesting, you decide to make a different number of each type of cookie.  If again you want to select 4 cookie types, how many ways can you select the cookie types and decide for which there will be the most, second most, etc.  Explain your answer.
		 \part ${21 \choose 9}$.  You must switch between cookie type 9 times as you make your 12 cookies.  The cookies are the stars, the switches between cookie types are the bars. %You change your mind again.  This time you decide you will make a total of 12 cookies.  Each cookie could be any one of the 10 types of cookies you know how to bake (and it's okay if you leave some types out).  How many choices do you have?  Explain.
		 \part $10^{12}$.  You have 10 choices for the ``1'' cookie, 10 choices for the ``2'' cookie, and so on. %You realize that the previous plan did not account for presentation.  This time, you once again want to make 12 cookies, each one could be any one of the 10 types of cookies.  However, now you plan to shape the cookies into the numerals 1, 2, \ldots, 12 (and probably arrange them to make a giant clock - but you haven't decided on that yet).  How many choices do you have for which types of cookies to bake into which numerals?  Explain.
		\end{parts}
	\end{answer}
	
	


\end{squestions}




