

\section{Exam 1 Study Guide}

The first exam (on Friday, February 15) will cover everything we have discussed so far this semester.  That is, logic, including quantifiers and set theory, and graph theory.  Here is a more detailed list of topics:

\begin{itemize}
 \item Sentential (or Propositional) Logic:
 \begin{itemize}
  \item Meaning of the five logical connectives.
  \item Truth tables.
  \item Logical equivalence of statements (use truth tables).
  \item Valid argument forms (again, use truth tables).
  \item What it means for a statement to be false - how to simplify with negations.
  \item Converse and contrapositive of an implication.
 \end{itemize}
  \item Predicate Logic:
  \begin{itemize}
   \item Meaning of quantifiers.
   \item How to translate between English and symbols.
   \item Negation of quantified statements (how to simplify).
  \end{itemize}
  \item Set Theory:
  \begin{itemize}
   \item Use set theory language to describe a situation (e.g., all squares are rectangles).
   \item How to read and make sense of set theory notation.
   \item Unions, intersections and compliments.
   \item Elements vs. Subsets ($\in$ vs. $\subseteq$).
   \item The power set.
   \item Venn diagrams.
  \end{itemize}
  \item Graph Theory:
  \begin{itemize}
   \item Euler paths a circuits.
   \item Vertex degrees.
   \item Named graphs - $K_n$, $K_{m,n}$, $C_n$, and $P_n$.  
   \item Counting edges.
   \item Planar graphs.
   \item Euler's formula for planar graphs.
   \item Vertex coloring - the chromatic number of a graph.
  \end{itemize}

\end{itemize}


The quizzes, worksheets, practice problems, and turn-in homework should give you a good idea of the types of questions to expect.  Copies of these assignments, with solutions, as well as the course notes are available on Blackboard.

Additionally, the questions below would all make fine exam questions.  In fact, some of them are taken directly off of discrete exams from previous semesters.  Of course, it is possible that the actual exam questions will be easier or harder than those below.  Also, there might be types of questions on this study guide not covered on the exam and questions on the exam not covered in this study guide.  Finally, questions on the exam might be asked in a different way than here - for instance, fill in the blank or true/false and explain.  These are really just to get you started.  




\begin{squestions}


\question\label{tt} Complete a truth table for the statement $\neg P \imp (Q \wedge R)$

  \begin{answer}
    \begin{tabular}{c|c|c||c}
                     $P$&$Q$&$R$& $\neg P \imp (Q \and R)$ \\ \hline
                     T & T & T & T\\
                     T & T & F & T\\
                     T & F & T & T\\
                     T & F & F & T \\
                     F & T & T & T\\
                     F & T & F & F\\
                     F & F & T & F\\
                     F & F & F & F
                    \end{tabular}
  \end{answer}

  
\question Suppose you know that the statement ``if Peter is not tall, then Quincy is fat and Robert is skinny'' is \underline{false}.  What, if anything, can you conclude about Peter and Robert if you know that Quincy is indeed fat?  Explain (you may reference problem \ref{tt}).

  \begin{answer}
    Peter is not tall and Robert is not skinny.  You must be in row 6 in the truth table above.
  \end{answer}


\question Are the statement $P \imp (Q \vee R)$ and $(P \imp Q) \vee (P \imp R)$ logically equivalent?  Explain your answer.

  \begin{answer}
    Yes.  To see this, make a truth table for each statement and compare.
  \end{answer}


  
\question Is the following a valid argument form: \begin{tabular}{rl} & $P \imp Q$ \\ & $P\imp R$ \\ \hline $\therefore$ & $P \imp (Q \and R)$.\end{tabular}  Explain.

  \begin{answer}
    Make a truth table that includes all three statements in the argument:
    
    \begin{tabular}{c|c|c||c|c|c}
     $P$ & $Q$ & $R$ & $P \imp Q$ & $P \imp R$ & $P \imp (Q \and R)$ \\ \hline
      T  &  T  &  T  &      T     &      T     &   T \\
      T  &  T  &  F  &      T     &      F     &   F \\
      T  &  F  &  T  &      F     &      T     &   F \\
      T  &  F  &  F  &      F     &      F     &   F \\
      F  &  T  &  T  &      T     &      T     &   T \\
      F  &  T  &  F  &      T     &      T     &   T \\
      F  &  F  &  T  &      T     &      T     &   T \\
      F  &  F  &  F  &      T     &      T     &   T 
    \end{tabular}
  
  Notice that in every row for which both $P \imp Q$ and $P \imp R$ is true, so is $P \imp (Q \and R)$.  Therefore, whenever the premises of the argument are true, so is the conclusion.  In other words, the argument form is valid.
  \end{answer}


\question Consider the statement: for all integers $n$, if $n$ is even and $n \le 7$ then $n$ is negative or $n \in \{0,2,4,6\}$.
\begin{parts}
 \part Is the statement true?  Explain why.
 \part Write the negation of the statement.  Is it true?  Explain.
 \part State the contrapositive of the statement.  Is it true?  Explain.
 \part State the converse of the statement.  Is it true?  Explain.
\end{parts}

  \begin{answer}
      \begin{parts}
	% Is the statement true?  Explain why.
	\part The statement is true.  If $n$ is an even integer less than or equal to 8, then the only way it could not be negative is if $n$ was equal to 0, 2, 4, or 6.
	% Write the negation of the statement.  Is it true?  Explain.
	\part There is an integer $n$ such that $n$ is even and $n \le 7$ but $n$ is not negative and $n \not\in \{0,2,4,6\}$.  This is false, since the original statement is true.
	%  State the contrapositive of the statement.  Is it true?  Explain.
	\part For all integers $n$, if $n$ is not negative and $n \not\in\{0,2,4,6\}$ then $n$ is odd or $n > 7$.  This is true, since the contrapositive is equivalent to the original statement (which is true).
	%  State the converse of the statement.  Is it true?  Explain.
	\part For all integers $n$, if $n$ is negative or $n \in \{0,2,4,6\}$ then $n$ is even and $n \le 7$.  This is false.  $n = -3$ is a counter-example.
      \end{parts}
  \end{answer}


\question Consider the statement: $\forall x (\forall y (x + y = y) \imp \forall z (x\cdot z = 0))$
\begin{parts}
   \part Explain what the statement says in words.  Is this statement true?  Be sure to state what you are taking the universe of discourse to be.
   \part Write the converse of the statement, both in words and in symbols.  Is the converse true?
   \part Write the contrapositive of the statement, both in words and in symbols.  Is the contrapositive true?
   \part Write the negation of the statement, both in words and in symbols.  Is the negation true?
\end{parts}

  \begin{answer}
      \begin{parts}
	\part For any number $x$, if it is the case that adding any number to $x$ gives that number back, then multiplying any number by $x$ will give 0.  This is true (of the integers or the reals) - the ``if'' part only holds if $x = 0$, and in that case, anything times $x$ will be 0.
	\part The converse in words is this: for any number $x$, if everything times $x$ is zero, then everything added to $x$ gives itself.  Or in symbols: $\forall x (\forall z (x \cdot z = 0) \imp \forall y (x + y = y))$.  The converse is true - the only number which when multiplied by any other number gives 0 is $x = 0$.  And if $x = 0$, then $x + y = y$.
	\part The contrapositive in words is: for any number $x$, if there is some number which when multiplied by $x$ does not give zero, then there is some number which when added to $x$ does not give that number.  In symbols: $\forall x (\exists z (x\cdot z \ne 0) \imp \exists y (x + y \ne y))$.  We know the contrapositive must be true because the original implication is true.
	\part The negation: there is a number $x$ such that any number added to $x$ gives the number back again, but there is a number you can multiply $x$ by and not get 0.  In symbols: $\exists x (\forall y (x + y = y) \and \exists z (x \cdot z \ne 0))$.  Of course since the original implication is true, the negation is false.
      \end{parts}
  \end{answer}


  
\question Write each of the following statements in the form, ``if \ldots, then \ldots.''  Careful, some of the statements might be false (which is alright for the purposes of this question).
\begin{parts}
  \part To loose weight, you must exercise.
  \part To loose weight, all you need to do is exercise.
  \part Every American is patriotic.
  \part You are patriotic only if you are American.
  \part The set of rational numbers is a subset of the real numbers.
  \part A number is prime if it is not even.
  \part Either the Broncos will win the Super Bowl, or they won't play in the Super Bowl.
\end{parts}

  \begin{answer}
      \begin{parts}
	  \part If you have lost weight, then you exercised.
	  \part If you exercise, then you will lose weight.
	  \part If you are American, then you are patriotic.
	  \part If you are patriotic, then you are American.
	  \part If a number is rational, then it is real.
	  \part If a number is not even, then it is prime.  (Or the contrapositive: if a number is not prime, then it is even.)
	  \part If the Broncos don't win the Super Bowl, then they didn't play in the Super Bowl.  Alternatively, if the Broncos play in the Super Bowl, then they will win the Super Bowl.
      \end{parts}
  \end{answer}


\question Simplify the following.
\begin{parts}
 \part $\neg (\neg (P \and \neg Q) \imp \neg(\neg R \vee \neg(P \imp R)))$
 \part $\neg \exists x \neg \forall y \neg \exists z (z = x + y \imp \exists w (x - y = w))$
 \part $\bar{\bar{(A \cap \bar B)} \cup (\bar C \cap D)}$
\end{parts}

  \begin{answer}
      \begin{parts}
	    % $\neg (\neg (P \and \neg Q) \imp \neg(\neg R \vee \neg(P \imp R)))$
	    \part $(\neg P \vee Q) \and (\neg R \vee (P \and \neg R))$
	    %  $\neg \exists x \neg \forall y \neg \exists z (z = x + y \imp \exists w (x - y = w))$
	    \part $\forall x \forall y \forall z (z = x+y \and \forall w (x-y \ne w))$
	    %  $\bar{\bar{(A \cap \bar B)} \cup (\bar C \cap D)}$
	    \part $(A \cap \bar B) \cap (C \cup \bar D)$
      \end{parts}
  \end{answer}




\question Let $A = \{1, 2, 3, 4, 5\}$ and $B = \{2, 3, 5, 7\}$ be subsets of the universal set $\U = \{1, 2, \ldots, 10\}$.  Find:
\begin{parts}
\part $A \cup B$
\part $A \cap B$
\part $\bar A$
\part $A \cap \bar B$
\end{parts}

  \begin{answer}
      \begin{parts}
	\part $A \cup B = \{1,2,3,4,5,7\}$
	\part $A \cap B = \{2,3,5\}$
	\part $\bar A = \{6,7,8,9,10\}$
	\part $A \cap \bar B = \{1,4\}$
      \end{parts}    
  \end{answer}




\question Give an example of two sets $A$ and $B$ such that $A \subset B$ and $A \in B$.  Give another example for which $A \subseteq B$ but $A \notin B$.  Give a third example for which $A \not\subseteq B$ but $A \in B$.

  \begin{answer}
    If $A = \{1,2\}$ and $B = \{1,2,3,\{1,2\}\}$ then $A \subset B$ and $A \in B$.  If $A = \{1,2\}$ and $B = \{1,2,3\}$ then $A \subseteq B$ but $A \notin B$.  If $A = \{1,2\}$ and $B = \{1, \{1,2\}\}$ then $A \not\subseteq B$ but $A \in B$.
  \end{answer}


 
\question Draw a Venn diagram to represent the set $(\bar A \cap B) \cup \bar{(B \cup C)}$. 

  \begin{answer}
    $(\bar A \cap B) \cup \bar{(B \cup C)}$\\
	\begin{tikzpicture}[fill=gray!50]

	\begin{scope}
	\clip \threesetbox \circleC;
	\fill \threesetbox;
	\end{scope}
	\fill \circleB;
	\begin{scope}
	  \clip \circleB;
	  \fill[white] \circleA;
	\end{scope}

	\draw[thick] \circleA \circleAlabel \circleB \circleBlabel \circleC \circleClabel \threesetbox;
	\end{tikzpicture}
  \end{answer}


 
\question Let $A$ and $B$ be sets with $|A| = 9$ and $|B| = 16$, and $|A \cup B| = 25$.  Find $A \cap B$.  Explain how you know your answer is correct.

  \begin{answer}
     $A \cap B = \emptyset$.  We know this because the set $A \cup B$ contains 25 elements, each of which is either from $A$ or from $B$, or from both.  But there can't be any from both, because $9 + 16 = 25$.  So $A \cap B$ contains no elements - it is the empty set.
  \end{answer}


\question At a school dance, 6 girls and 4 boys take turns dancing (as couples) with each other.
\begin{parts}
  \part How many couples danced if every girl dances with ever boy?
  \part How many couples danced if every one danced with everyone else (regardless of gender)?
  \part Explain what graphs can be used to represent these situations.
\end{parts}

  \begin{answer}
  \begin{parts}
	 \part There were 24 couples - 6 choices for the girl and 4 choices for the boy.
	 \part There were 45 couples - ${10 \choose 2}$ since we must choose two of the 10 people to dance together.
	 \part For part (a), we are counting the number of edges in $K_{4,6}$.  In part (b) we count the edges of $K_{10}$.
  \end{parts}
  \end{answer}
  
  

\question Among a group of $n$ people, is it possible for everyone to be friends with an odd number of people in the group?  If so, what can you say about $n$?

  \begin{answer}
  Yes, as long as $n$ is even.  If $n$ were odd, then corresponding graph would have an odd number of odd degree vertices, which is impossible.
  \end{answer}

 
 \question How many edges does the graph $K_{n,n}$ have?  For which values of $n$ does the graph contain an Euler circuit?  For which values of $n$ is the graph planar?

  \begin{answer}
  $K_{n,n}$ has $n^2$ edges.  The graph will have an Euler circuit when $n$ is even.  The graph will be planar only when $n < 3$.
  \end{answer}
 
 
 
 \question The graph $G$ has 6 vertices with degrees $1, 2, 2, 3, 3, 5$.  How many edges does $G$ have?  If $G$ was planar how many faces would it have?  Does $G$ have an Euler path?

  \begin{answer}
  $G$ has 8 edges (since the sum of the degrees is 16).  If $G$ is planar, then it will have 4 faces (since $6 - 8 + 4 = 2$).  $G$ does not have an Euler path since there are more than 2 vertices of odd degree.
  \end{answer}
 

  
\question What is the smallest number of colors you need to properly color the vertices of $K_{7}$.  Can you say whether $K_7$ is planar based on your answer?

  \begin{answer}
  $7$ colors.  Thus $K_7$ is not planar (by the contrapositive of the Four Color Theorem).
  \end{answer}


\question What is the smallest number of colors you need to properly color the vertices of $K_{3,4}$?  Can you say whether $K_{3,4}$ is planar based on your answer?

  \begin{answer}
  The chromatic number of $K_{3,4}$ is 2, since the graph is bipartite.  You cannot say whether the graph is planar based on this coloring (the converse of the Four Color Theorem is not true).  In fact, the graph is {\em not} planar, since it contains $K_{3,3}$ as a subgraph.
  \end{answer}


\question If a planar graph $G$ with $7$ vertices divides the the plane into 8 regions, how many edges must $G$ have?

  \begin{answer}
  $G$ has $13$ edges, since we need $7 - E + 8 = 2$.
  \end{answer}



\question Consider the graph below:
\begin{center}
  \begin{tikzpicture}[scale=.4]
    \draw[thick] (0,0) \v -- (-1.5, .5) \v -- (0,1.5) \v -- (1.5,.5) \v -- (0,0) -- (-1,2) \v -- (0,1.5) -- (1,2) \v -- (0,0) -- (0, 1.5);
  \end{tikzpicture}
\end{center}

\begin{parts}
  \part Does the graph have an Euler path or circuit?  Explain.
  \part Is the graph planar?  Explain.
  \part Is the graph bipartite?  Complete?  Complete bipartite?
  \part What is the chromatic number of the graph.
\end{parts}

  \begin{answer}
  \begin{parts}
	 \part The graph does have an Euler path, but not an Euler circuit.  There are exactly two vertices with odd degree - the path starts at one and ends at the other.
	 \part The graph is planar.  Even though as it is drawn edges cross, it is easy to redraw it without edges crossing.
	 \part The graph is not bipartite (there is an odd cycle), nor complete.
	 \part The chromatic number of the graph is 3.
  \end{parts}
  \end{answer}



\question For each part below, say whether the statement is true or false.  Explain why the true statements are true, and given counter-examples for the false statements.
\begin{parts}
  \part Every bipartite graph is planar.
  \part Every bipartite graph has chromatic number 2.
  \part Every bipartite graph has an Euler path.
  \part Every vertex of a bipartite graph has even degree.
  \part A graph is bipartite if and only if the sum of the degrees of all the vertices is even.
\end{parts}

  \begin{answer}
  \begin{parts}
	 \part False.  For example, $K_{3,3}$ is not planar.
	 \part True.  The graph is bipartite so it is possible to divide the vertices into two groups with no edges between vertices in the same group.  Thus we can color all the vertices of one group red and the other group blue.
	 \part False.  $K_{3,3}$ has 6 vertices with degree 3, so contains no Euler path.
	 \part False.  $K_{3,3}$ again.
	 \part False.  The sum of the degrees of all vertices is even for {\em all} graphs so this property does not imply that the graph is bipartite.
  \end{parts}
  \end{answer}



\question Consider the statement ``If a graph is planar, then it has an Euler path.''
\begin{parts}
 \part Write the converse of the statement.
 \part Write the contrapositive of the statement.
 \part Write the negation of the statement.
 \part Is it possible for the contrapositive to be false?  If it was, what would that tell you?
 \part Is the original statement true or false?  Prove your answer.
 \part Is the converse of the statement true or false?  Prove your answer.
\end{parts}

  \begin{answer}
  \begin{parts}
  \part If a graph has an Euler path, then it is planar.
  \part If a graph does not have an Euler path, then it is not planar.
  \part There is a graph which is planar and does not have an Euler path.
  \part Yes.  In fact, in this case it is because the original statement is false.
  \part False.  $K_4$ is planar but does not have an Euler path.
  \part False.  $K_5$ has an Euler path but is not planar.
  \end{parts}
  \end{answer}

 
\end{squestions}





