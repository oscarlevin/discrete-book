\section{Exam 3 Study Guide}

Exam 3 (on Friday, April 19) will cover everything we have discussed since the last exam.  That means sequences (including recurrence relations and generating functions) and induction.  Here is a more detailed list of topics:

\begin{itemize}
 \item Closed vs. recursive formulas
 \item Relationship between sequences, relationships between their formulas.
 \item Arithmetic and geometric sequences
 \item Sums of arithmetic sequences (reverse and add)
 \item Sums of geometric sequences (multiply by $r$, shift and subtract)
 \item Polynomial fitting (finite differences)
 \item The characteristic root technique
 \item Generating functions:
 \begin{itemize}
 	\item What they are and how they work
 	\item Transforming one into another
 	\item Partial sums, sequence of differences
 	\item Differencing and using recurrence relations
 \end{itemize}
 \item Mathematical induction:
 \begin{itemize}
 	\item Explain the idea of a proof - inductive thinking
 	\item Write formal proofs
 	\item Strong induction 
 \end{itemize}
\end{itemize}


The quizzes, worksheets, practice problems, and turn-in homework should give you a good idea of the types of questions to expect.  

Additionally, the questions below would all make fine exam questions.  In fact, some of them are taken directly off of discrete exams from previous semesters.\footnote{Disclaimer: Question on the actual exam may be easier or harder than those given her.  There might be types of questions on this study guide not covered on the exam and questions on the exam not covered in this study guide.  Questions on the exam might be asked in a different way than here.  If solving a question lasts longer than four hours, contact your professor immediately.}  



\begin{squestions}
\question Consider the sum $4 + 11 + 18 + 25 + \cdots + 249$.
\begin{parts}
\part How many terms (summands) are in the sum?
\part Compute the sum.  Remember to show all your work.
\end{parts}

	\begin{answer}
		\begin{parts}
		\part 36.  %How many terms (summands) are in the sum?
		\part $\frac{253 \cdot 36}{2} = 4554$.  %Compute the sum.  Remember to show all your work.
		\end{parts}
	\end{answer}




\question Consider the sequence $5, 9, 13, 17, 21, \ldots$
\begin{parts}
  \part Give a recursive definition for the sequence.
  \part Give a closed formula for the $n$th term of the sequence.
  \part Is $2013$ a term in the sequence?  Explain.
  \part How many terms does the sequence $5, 9, 13, 17, 21, \ldots, 533$ have?
  \part Find the sum: $5 + 9 + 13 + 17 + 21 + \cdots + 533$.  Show your work.
\end{parts}

	\begin{answer}
		\begin{parts}
		  \part $a_n = a_{n-1} + 4$ with $a_1 = 5$.  %Give a recursive definition for the sequence.
		  \part $a_n = 5 + 4(n-1)$  %Give a closed formula for the $n$th term of the sequence.
		  \part Yes, since $2013 = 5 + 4(503-1)$ (so $a_{503} = 2013$).
		  \part 133 %How many terms does the sequence $5, 9, 13, 17, 21, \ldots, 533$ have?
		  \part $\frac{538\cdot 133}{2} = 35777$  %Find the sum: $5 + 9 + 13 + 17 + 21 + \cdots + 533$.  Show your work.
		\end{parts}
	\end{answer}





%Sum of geometric sequence
\question Consider the sequence given by $a_n = 2\cdot 5^{n-1}$.
\begin{parts}
\part Find the first 4 terms of the sequence.  What sort of sequence is this?
\part Find the {\em sum} of the first 25 terms.  That is, compute $\d\sum_{k=1}^{25}a_k$.
\end{parts}

	\begin{answer}
		\begin{parts}
		\part $2, 10, 50, 250, \ldots$  The sequence is geometric. %Find the first 4 terms of the sequence.  What sort of sequence is this?
		\part $\frac{2 - 2\cdot 5^{25}}{-4}$.  %Find the {\em sum} of the first 25 terms.  That is, compute $\d\sum_{k=1}^{25}a_k$.
		\end{parts}
	\end{answer}





%Polynomial fitting
\question Use polynomial fitting to find a closed formula for the sequence:
$4, 11, 20, 31, 44, \ldots $
(assume $a_1 = 4$).

	\begin{answer}
		$a_n = n^2 + 4n - 1$
	\end{answer}






%Recursive definition:
\question Consider the sequence given recursively by $a_1 = 4$, $a_2 = 6$ and $a_n = a_{n-1} + a_{n-2}$.
\begin{parts}
\part Write out the first 6 terms of the sequence.
\part Could the closed formula for $a_n$ be a polynomial?  Explain.
\end{parts}

	\begin{answer}
	 	\begin{parts}
	 	\part $4, 6, 10, 16, 26, 42, \ldots$  %Write out the first 6 terms of the sequence.
	 	\part No, taking differences gives the original sequence back, so the differences will never be constant.  %Could the closed formula for $a_n$ be a polynomial?  Explain.
	 	\end{parts}
	\end{answer}





%new sequences from old:
\question The sequence $-1, 0, 2, 5, 9, 14\ldots$ has closed formula $a_n = \dfrac{(n+1)(n-2)}{2}$.  Use this fact to find a closed formula for the sequence $4, 10, 18, 28, 40, \ldots$ 

	\begin{answer}
		 $b_n = (n+3)n$
	\end{answer}





\question Consider the recurrence relation $a_n = 3a_{n-1} + 10 a_{n-2}$ with first two terms $a_0 = 1$ and $a_1 = 2$.
\begin{parts}
 \part Write out the first 5 terms of the sequence defined by this recurrence relation.
 \part Solve the recurrence relation. 
\end{parts}

	\begin{answer}
		\begin{parts}
		 \part $1, 2, 16,68, 364, \ldots$  %Write out the first 5 terms of the sequence defined by this recurrence relation.
		 \part $a_n = \frac{3}{7}(-2)^n + \frac{4}{7}5^n$  %Solve the recurrence relation. 
		\end{parts}
	\end{answer}





\question Consider the recurrence relation $a_n = 2a_{n-1} + 8a_{n-2}$, with initial terms $a_0 = 1$ and $a_1= 3$.
\begin{parts}
  \part Find the next two terms of the sequence ($a_2$ and $a_3$).
  \part Solve the recurrence relation.   That is, find a closed formula for the $n$th term of the sequence.
  \part Find the generating function for the sequence.  Hint: use the recurrence relation.
\end{parts}

	\begin{answer}
		\begin{parts}
		  \part $a_2 = 14$.  $a_3 = 52$  %Find the next two terms of the sequence ($a_2$ and $a_3$).
		  \part $a_n = \frac{1}{6}(-2)^n + \frac{5}{6}4^n$  %Solve the recurrence relation.   That is, find a closed formula for the $n$th term of the sequence.
		  \part $\frac{1+x}{1-2x-8x^2}$  %Find the generating function for the sequence.  Hint: use the recurrence relation.
		\end{parts}
	\end{answer}





\question Prove the following statements my mathematical induction: 
\begin{parts}
 \part $n! < n^n$ for $n \ge 2$
 \part $\d\frac{1}{1\cdot 2} + \frac{1}{2\cdot 3} +\frac{1}{3\cdot 4}+\cdots + \frac{1}{n\cdot(n+1)} = \d\frac{n}{n+1}$ for all $n \in \Z^+$.
 \part $4^n - 1$ is a multiple of 3 for all $n \in \N$.
 \part $F_0 + F_2 + F_4 + \cdots + F_{2n} = F_{2n + 1} - 1$ for all $n = 0,1,2,\ldots$.  ($F_n$ is the $n$th Fibonacci numbers.)
 \part The {\em greatest} amount of postage you {\em cannot} make exactly using 4 and 9 cent stamps is 23 cents.
\end{parts}

	\begin{answer}
		\begin{parts}
		 \part Hint: $(n+1)^{n+1} > (n+1) \cdot n^{n}$.
		 \part Hint: This should be similar to the other sum proofs - the last bit comes down to adding fractions.
		 \part Hint: Write $4^{k+1} - 1 = 4\cdot 4^k - 4 + 3$.
		 \part Hint: Use the fact $F_{2n} + F_{2n+1} = F_{2n+2}$
		 \part Hint: one 9-cent stamp is 1 more than two 4-cent stamps, and seven 4-cent stamps is 1 more than three 9-cent stamps.
		\end{parts}
	\end{answer}





\question Explain how we know that $\dfrac{1}{(1-x)^2}$ is the generating function for the sequence $1, 2, 3, 4, \ldots$

	\begin{answer}
		Starting with $\frac{1}{1-x} = 1 + x + x^2 + x^3 +\cdots$, we can take derivatives of both sides, given $\frac{1}{(1-x)^2} = 1 + 2x + 3x^2 + \cdots$.  By the definition of generating functions, this says that $\frac{1}{(1-x)^2}$ generates the {\em sequence} 1, 2, 3\ldots.  You can also find this using differencing or by multiplying. 
	\end{answer}





\question Starting with the generating function for $1,2,3,4, \ldots$, find a generating function for each of the following sequences.
\begin{parts}
 \part $1, 0, 2, 0, 3, 0, 4,\ldots$
 \part $1, -2, 3, -4, 5, -6, \ldots$
 \part $0, 3, 6, 9, 12, 15, 18, \ldots$
 \part $0, 3, 9, 18, 30, 45, 63,\ldots$ (Hint: relate this sequence to the previous one.)
\end{parts}

	\begin{answer}
		\begin{parts}
		 \part $\frac{1}{(1-x^2)^2}$  %$1, 0, 2, 0, 3, 0, 4,\ldots$
		 \part $\frac{1}{(1+x)^2}$  %$1, -2, 3, -4, 5, -6, \ldots$
		 \part $\frac{3x}{(1-x)^2}$  %$0, 3, 6, 9, 12, 15, 18, \ldots$
		 \part $\frac{3x}{(1-x)^3}$  (partial sums)  %$0, 3, 9, 18, 30, 45, 63,\ldots$ (Hint: relate this sequence to the previous one.)
		 \end{parts}
	\end{answer}






\question You may assume that the sequence $1, 1, 2, 3, 5, 8,\ldots$ has generating function $\dfrac{1}{1-x-x^2}$ (because it does).  Use this fact to find the sequence generated by each of the following generating functions.
\begin{parts}
  \part $\frac{x^2}{1-x-x^2}$
  \part $\frac{1}{1-x^2-x^4}$
  \part $\frac{1}{1-3x-9x^2}$
  \part $\frac{1}{(1-x-x^2)(1-x)}$
\end{parts}

	\begin{answer}
		\begin{parts}
		  \part $0,0,1,1,2,3,5,8, \ldots$   %$\frac{x^2}{1-x-x^2}$
		  \part $1, 0, 1, 0, 2, 0, 3, 0, 5, 0, 8, 0, \ldots$  %$\frac{1}{1-x^2-x^4}$
		  \part $1, 3, 18, 81, 405, \ldots$  %$\frac{1}{1-3x-9x^2}$
		  \part $1, 2, 4, 7, 12, 20$  %$\frac{1}{(1-x-x^2)(1-x)}$
		\end{parts}
	\end{answer}





\question Find the generating function for the sequence $1, -2, 4, -8, 16, \ldots$

	\begin{answer}
		$\frac{1}{1+2x}$ 
	\end{answer}




\question Find the generating function for the sequence $1, 1, 1, 2, 3, 4, 5, 6, \ldots$

	\begin{answer}
		$\frac{x^3}{(1-x)^2} + \frac{1}{1-x}$
	\end{answer}





\question Suppose $A$ is the generating function for the sequence $3, 5, 9, 15, 23, 33, \ldots$
\begin{parts}
 \part Find a generating function (in terms of $A$) for the sequence of differences between terms.
 \part Write the sequence of differences between terms and find a generating function for it (without referencing $A$).
 \part Use your answers to parts (a) and (b) to find the generating function for the original sequence.
\end{parts}

	\begin{answer}
		\begin{parts}
		 \part $(1-x)A = 3 + 2x + 4x^2 + 6x^3 + \cdots$ which is almost right.  We can fix it like this:
		 $2 + 4x + 6x^2 + \cdots = \frac{(1-x)A - 3}{x}$
		 \part We know $2 + 4x + 6x^3 + \cdots = \frac{2}{(1-x)^2}$
		 \part $A = \frac{2x}{(1-x)^3} + \frac{3}{1-x} = \frac{3 -4x + 3x^2}{(1-x)^3}$
		\end{parts}
	\end{answer}



\question Prove the following statements my mathematical induction: 
\begin{parts}
 \part $n! < n^n$ for $n \ge 2$
 \part $\d\frac{1}{1\cdot 2} + \frac{1}{2\cdot 3} +\frac{1}{3\cdot 4}+\cdots + \frac{1}{n\cdot(n+1)} = \d\frac{n}{n+1}$ for all $n \in \Z^+$.
 \part $4^n - 1$ is a multiple of 3 for all $n \in \N$.
 \part $F_0 + F_2 + F_4 + \cdots + F_{2n} = F_{2n + 1} - 1$ for all $n = 0,1,2,\ldots$.  ($F_n$ is the $n$th Fibonacci numbers.)
 \part The {\em greatest} amount of postage you {\em cannot} make exactly using 4 and 9 cent stamps is 23 cents.
\end{parts}

	\begin{answer}
		\begin{parts}
		 \part Hint: $(n+1)^{n+1} > (n+1) \cdot n^{n}$.
		 \part Hint: This should be similar to the other sum proofs - the last bit comes down to adding fractions.
		 \part Hint: Write $4^{k+1} - 1 = 4\cdot 4^k - 4 + 3$.
		 \part Hint: Use the fact $F_{2n} + F_{2n+1} = F_{2n+2}$
		 \part Hint: one 9-cent stamp is 1 more than two 4-cent stamps, and seven 4-cent stamps is 1 more than three 9-cent stamps.
		\end{parts}
	\end{answer}

\question  Prove, by induction, that for any natural numbers $n \ge 1$, if $a$ and $b$ are real numbers, then $(ab)^n = a^nb^n$.

	\begin{answer}
		Hint: For the inductive case, we will get that $(ab)^{k+1} = (ab)^k(ab) = a^kb^kab$.  Simplify.
	\end{answer}
	
	
	
\end{squestions}





