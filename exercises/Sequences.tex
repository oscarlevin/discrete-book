
\begin{questions}
\question Find the closed formula for each of the following sequences by relating them to a well know sequence.  Assume the first term given is $a_1$.
\begin{parts}
  \part $2, 5, 10, 17, 26, \ldots$
  \part $0, 2, 5, 9, 14, 20, \ldots$
  \part $8, 12, 17, 23, 30,\ldots$
  \part $1, 5, 23, 119, 719,\ldots$
\end{parts}

	\begin{answer}
		\begin{parts}
		%$2, 5, 10, 17, 26, \ldots$
		  \part $a_n = n^2 + 1$
		%   $0, 2, 5, 9, 14, 20, \ldots$
		  \part $a_n = \frac{n(n+1)}{2} - 1$
		%   $8, 12, 17, 23, 30,\ldots$
		  \part $a_n = \frac{(n+2)(n+3)}{2} + 2$
		%   $1, 5, 23, 119, 719,\ldots$
		  \part $a_n = (n+1)! - 1$ (where $n! = 1 \cdot 2 \cdot 3 \cdots n$)
		\end{parts}
	\end{answer}
	
	
	

\question The Fibonacci sequence is $0, 1, 1, 2, 3, 5, 8, 13, \ldots$ (where $F_0 = 0$).
\begin{parts}
  \part Give the recursive definition for the sequence.
  \part Write out the first few terms of the sequence of partial sums.  
  \part Give a closed formula for the sequence of partial sums in terms of $F_n$  (for example, you might say $F_0 + F_1 + \cdots + F_n = 3F_{n-1}^2 + n$, although that is definitely not correct).
\end{parts}

	\begin{answer}
		\begin{parts}
		%Give the recursive definition for the sequence.
		  \part $F_n = F_{n-1} + F_{n-2}$ with $F_0 = 0$ and $F_1 = 1$.
		%   Write out the first few terms of the sequence of partial sums. 
		  \part  $0, 1, 2, 4, 7, 12, 20, \ldots$
		  %Give a closed formula for the sequence of partial sums in terms of $F_n$  (for example, you might say $F_0 + F_1 + \cdots + F_n = 3F_{n-1}^2 + n$, although that is definitely not correct).
		  \part $F_0 + F_1 + \cdots + F_n = F_{n+2} - 1$ 
		\end{parts}
	\end{answer}
	
	
	


\question Write out the first few terms of the sequence given by $a_1 = 3$; $a_n = 2a_{n-1} + 4$.  Then find a recursive definition for the sequence $10, 24, 52, 108, \ldots$.

	\begin{answer}
		$3, 10, 24, 52, 108,\ldots$.  The recursive definition for $10, 24, 52, \ldots$ is $a_n = 2a_{n-1} + 4$ with $a_1 = 10$.
	\end{answer}
	
	
	


\question Write out the first few terms of the sequence given by $a_n = n^2 - 3n + 1$.  Then find a closed formula for the sequence (starting with $a_1$) $0, 2, 6, 12, 20, \ldots$.

	\begin{answer}
		$-1, -1, 1, 5, 11, 19,\ldots$  Thus the sequence $0, 2, 6, 12, 20,\ldots$ has closed formula $a_n = (n+1)^2 - 3(n+1) + 2$.
	\end{answer}
	
	
	


\question Consider the sequence $8, 14, 20, 26, \ldots, $. 
\begin{parts}
\part What is the next term in the sequence?  
\part Find a formula for the $n$th term of this sequence, assuming $a_1 = 8$.
\part Find the sum of the first 100 terms of the sequence: $\sum_{k=1}^{100}a_k$.
\end{parts}

	\begin{answer}
		\begin{parts}
		% What is the next term in the sequence?  
		\part 32.
		% Find a formula for the $n$th term of this sequence, assuming $a_1 = 8$.
		\part $a_n = 8 + 6(n-1)$
		% Find the sum of the first 100 terms of the sequence: $\sum_{k=1}^{100}a_k$.
		\part $30500$.
		\end{parts}
	\end{answer}
	
	
	


\question Consider the sequence $1, 7, 13, 19, \ldots, 6n + 7$.  
\begin{parts}
\part How many terms are there in the sequence?
\part What is the second-to-last term?
\part Find the sum of all the terms in the sequence.
\end{parts}

	\begin{answer}
		\begin{parts}
		% How many terms are there in the sequence?
		\part $n+2$ terms.
		\part $6n+1$. %second to last term
		%Find the sum of all the terms in the sequence.
		\part $\frac{(6n+8)(n+2)}{2}$ 
		\end{parts}
	\end{answer}
	
	
	



\question Find $5 + 7 + 9 + 11+ \cdots + 521$.

	\begin{answer}
		68117
	\end{answer}
	
	
	


\question Find $5 + 15 + 45 + \cdots + 5\cdot 3^{20}$

	\begin{answer}
		$\frac{5-5\cdot 3^{21}}{-2}$
	\end{answer}
	
	
	


\question Find $1 - \frac{2}{3} + \frac{4}{9} - \cdots + \frac{2^{30}}{3^{30}}$

	\begin{answer}
		$\frac{1 + \frac{2^{31}}{3^{31}}}{5/3}$
	\end{answer}
	
	
	


\question Find $x$ and $y$ such that $27, x, y, 1$ is part of an arithmetic sequence.  Then find $x$ and $y$ so that the sequence is part of a geometric sequence.  ($x$ and $y$ might not be integers.) 

	\begin{answer}
		For arithmetic: $x = 55/3$, $y = 29/3$.  For geometric: $x = 9$ and $y = 3$.
	\end{answer}
	
	
	


\question Use summation ($\sum$) or product ($\prod$) notation to rewrite the following.
\begin{parts}
  \part $2 + 4 + 6 + 8 + \cdots + 2n$
  \part $1 + 5 + 9 + 13 + \cdots + 425$
  \part $1 + \frac{1}{2} + \frac{1}{3} + \frac{1}{4} + \cdots + \frac{1}{50}$
  \part $2 \cdot 4 \cdot 6 \cdot \cdots \cdot 2n$
  \part $(\frac{1}{2})(\frac{2}{3})(\frac{3}{4})\cdots(\frac{100}{101})$
\end{parts}

	\begin{answer}
		\begin{parts}
		  \part $\d\sum_{k=1}^n 2k$		%$2 + 4 + 6 + 8 + \cdots + 2n$
		  \part $\d\sum_{k=1}^{107} (1 + 4(k-1))$		%$1 + 5 + 9 + 13 + \cdots + 425$
		  \part $\d\sum_{k=1}^{50} \frac{1}{k}$		%$1 + \frac{1}{2} + \frac{1}{3} + \frac{1}{4} + \cdots + \frac{1}{50}$
		  \part $\d\prod_{k=1}^n 2k$		%$2 \cdot 4 \cdot 6 \cdot \cdots \cdot 2n$
		  \part $\d\prod_{k=1}^{100} \frac{k}{k+1}$	%$(\frac{1}{2})(\frac{2}{3})(\frac{3}{4})\cdots(\frac{100}{101})$
		\end{parts}
	\end{answer}
	
	
	


\question Expand the following sums and products.  That is, write them out the long way.
\begin{parts}
  \part $\d\sum_{k=1}^{100} (3+4k)$
  \part $\d\sum_{k=0}^n 2^k$
  \part $\d\sum_{k=2}^{50}\frac{1}{(k^2 - 1)}$
  \part $\d\prod_{k=2}^{100}\frac{k^2}{(k^2-1}$
  \part $\d\prod_{k=0}^n (2+3k)$
\end{parts}

	\begin{answer}
		\begin{parts}
		  \part $\d\sum_{k=1}^{100} (3+4k) = 7 + 11 + 15 + \cdots + 403$
		  \part $\d\sum_{k=0}^n 2^k = 1 + 2 + 4 + 8 + \cdots + 2^n$
		  \part $\d\sum_{k=2}^{50}\frac{1}{(k^2 - 1)} = 1 + \frac{1}{3} + \frac{1}{8} + \frac{1}{15} + \cdots + \frac{1}{2499}$
		  \part $\d\prod_{k=2}^{100}\frac{k^2}{(k^2-1} = \frac{4}{3}\cdot\frac{9}{8}\cdot\frac{16}{15}\cdots\frac{10000}{9999}$
		  \part $\d\prod_{k=0}^n (2+3k) = (2)(5)(8)(11)(14)\cdots(2+3n)$
		\end{parts}
	\end{answer}
	
	
	



\question Use polynomial fitting to find the formula for the $n$th term of the following sequences:
\begin{parts}
\part 2, 5, 11, 21, 36,\ldots
\part 0, 2, 6, 12, 20, \ldots
\end{parts}

	\begin{answer}
		\begin{parts}
		\part Hint: third differences are constant, so $a_n = an^3 + bn^2 + cn + d$.  Use the terms of the sequence to solve for $a, b, c,$ and $d$.
		\part $a_n = n^2 - n$
		\end{parts}
	\end{answer}
	
	
	


\question Can you use polynomial fitting to find the formula for the $n$th term of the sequence 4, 7, 11, 18, 29, 47, \ldots?  Explain why or why not. 

	\begin{answer}
		No.  The sequence of differences is the same as the original sequence so no differences will be constant. 
	\end{answer}
	
	
	


\question Find the next 2 terms in the sequence $3, 5, 11, 21, 43, 85\ldots.$.  Then give a recursive definition for the sequence.  Finally, use the characteristic root technique to find a closed formula for the sequence.

	\begin{answer}
		171 and 341.  $a_n = a_{n-1} + 2a_{n-2}$ with $a_0 = 3$ and $a_1 = 5$.  Closed formula: $a_n = \frac{8}{3}2^n + \frac{1}{3}(-1)^n$
	\end{answer}
	
	
	


\question Solve the recurrence relation $a_n = a_{n-1} + 2^n$ with $a_0 = 5$.

	\begin{answer}
		By telescoping or iteration.  $a_n = 3 + 2^{n+1}$
	\end{answer}
	
	
	


\question Show that $4^n$ is a solution to the recurrence relation $a_n = 3a_{n-1} + 4a_{n-2}$.

	\begin{answer}
		We claim $a_n = 4^n$ works.  Plug it in: $4^n = 3(4^{n-1}) + 4(4^{n-2})$.  This works - just simplify the right hand side.
	\end{answer}
	
	
	


\question Find the solution to the recurrence relation $a_n = 3a_{n-1} + 4a_{n-2}$ with initial terms $a_0 = 2$ and $a_1 = 3$.

	\begin{answer}
		By the Characteristic Root Technique.  $a_n = 4^n + (-1)^n$.
	\end{answer}
	
	
	


\question Find the solution to the recurrence relation $a_n = 3a_{n-1} + 4a_{n-2}$ with initial terms $a_0 = 5$ and $a_1 = 8$.

	\begin{answer}
		$a_n = \frac{13}{5} 4^n + \frac{12}{5} (-1)^n$
	\end{answer}
	
	
	


\question Solve the recurrence relation $a_n = 2a_{n-1} - a_{n-2}$.
\begin{parts}
  \part What is the solution if the initial terms are $a_0 = 1$ and $a_1 = 2$?
  \part What do the initial terms need to be in order for $a_9 = 30$?
  \part For which $x$ are there initial terms which make $a_9 = x$?
\end{parts}

	\begin{answer}
		The general solution is $a_n = a + bn$ where $a$ and $b$ depend on the initial conditions.  %Solve the recurrence relation $a_n = 2a_{n-1} - a_{n-2}$.
		\begin{parts}
		  \part $a_n = 1 + n$
		  %What is the solution if the initial terms are $a_0 = 1$ and $a_1 = 2$?
		  \part For example, we could have $a_0 = 21$ and $a_1 = 22$.  %What do the initial terms need to be in order for $a_9 = 30$?
		  \part For every $x$ - take $a_0 = x-9$ and $a_1 = x-8$.  %For which $x$ are there initial terms which make $a_9 = x$?
		\end{parts}
	\end{answer}
	
	
	


\question Solve the recurrence relation $a_n = 3a_{n-1} + 10a_{n-2}$ with initial terms $a_0 = 4$ and $a_1 = 1$.  

	\begin{answer}
		$a_n = \frac{19}{7}(-2)^n + \frac{9}{7}5^n$
		%Solve the recurrence relation $a_n = 3a_{n-1} + 10a_{n-2}$ with initial terms $a_0 = 4$ and $a_1 = 1$.  
	\end{answer}
	
	
	

\question Find the generating function for each of the following sequences by relating them back to a sequence with known generating function.
\begin{parts}
  \part $4,4,4,4,4,\ldots$
  \part $2, 4, 6, 8, 10, \ldots$
  \part $0,0,0,2,4,6,8,10,\ldots$
  \part $1, 5, 25, 125, \ldots$
  \part $1, -3, 9, -27, 81, \ldots$
  \part $1, 0, 5, 0, 25, 0, 125, 0, \ldots$
  \part $0, 1, 0, 0, 2, 0, 0, 3, 0, 0, 4, 0, 0, 5, \ldots$
\end{parts}

	\begin{answer}
		\begin{parts}
		  \part $\dfrac{4}{1-x}$  %$4,4,4,4,4,\ldots$
		  \part $\dfrac{2}{(1-x)^2}$  %$2, 4, 6, 8, 10, \ldots$
		  \part $\dfrac{2x^3}{(1-x}^2$  %$0,0,0,2,4,6,8,10,\ldots$
		  \part $\dfrac{1}{1-5x}$  %$1, 5, 25, 125, \ldots$
		  \part $\dfrac{1}{1+3x}$  %$1, -3, 9, -27, 81, \ldots$
		  \part $\dfrac{1}{1-5x^2}$  %$1, 0, 5, 0, 25, 0, 125, 0, \ldots$
		  \part $\dfrac{x}{(1-x^3)^2}$  %$0, 1, 0, 0, 2, 0, 0, 3, 0, 0, 4, 0, 0, 5, \ldots$
		\end{parts}
	\end{answer}
	
	
	
	

\question Find the sequence generated by the following generating functions:
\begin{parts}
  \part $\dfrac{4x}{1-x}$
  \part $\dfrac{1}{1-4x}$
  \part $\dfrac{x}{1+x}$
  \part $\dfrac{3x}{(1+x)^2}$
  \part $\dfrac{1+x+x^2}{(1-x)^2}$ (Hint: multiplication)
\end{parts}

	\begin{answer}
		\begin{parts}
		  \part $0, 4, 4, 4, 4, 4, \ldots$  %$\dfrac{4x}{1-x}$
		  \part $1, 4, 16, 64, 256, \ldots$  %$\dfrac{1}{1-4x}$
		  \part $0, 1, -1, 1, -1, 1, -1, \ldots$  %$\dfrac{x}{1+x}$
		  \part $0, 3, -6, 9, -12, 15, -18, \ldots $  %$\dfrac{3x}{(1+x)^2}$
		  \part $1, 3, 6, 9, 12, 15, \ldots$  %$\dfrac{1+x+x^2}{(1-x)^2}$ (Hint: multiplication)
		\end{parts}
	\end{answer}
	
	
	
	



\question Show how you can get the generating function for the triangular numbers in three different ways:
\begin{parts}
  \part Take two derivatives of the generating function for $1,1,1,1,1, \ldots$
  \part Use differencing.
  \part Multiply two known generating functions.
\end{parts}

	\begin{answer}
		\begin{parts}
		  \part The second derivative of $\dfrac{1}{1-x}$ is $\dfrac{2}{(1-x)^3}$ which expands to $2 + 6x + 12x^2 + 20x^3 + 30x^4 + \cdots$.  Dividing by 2 gives the generating function for the triangular numbers. %Take two derivatives of the generating function for $1,1,1,1,1, \ldots$
		  \part Compute $A - xA$ and you get $1 + 2x + 3x^2 + 4x^3 + \cdots$ which can be written as $\dfrac{1}{(1-x)^2}$.  Solving for $A$ gives the correct generating function. %Use differencing.
		  \part The triangular numbers are the sum of the first $n$ numbers $1,2,3,4, \ldots$.  To get the sequence of partial sums, we multiply by $\frac{1}{1-x}$. so this gives the correct generating function again. %Multiply two known generating functions.
		\end{parts}
	\end{answer}
	
	
	
	


\question Use differencing to find the generating function for $4, 5, 7, 10, 14, 19, 25, \ldots$

	\begin{answer}
		Call the generating function $A$.  Compute $A - xA = 4 + x + 2x^2 + 3x^3 + 4x^4 + \cdots$.  Thus $A - xA = 4 + \dfrac{x}{(1-x)^2}$.  Solving for $A$ gives $\d\frac{4}{1-x} + \frac{x}{(1-x)^3}$.  %Use differencing to find the generating function for $4, 5, 7, 10, 14, 19, 25, \ldots$
	\end{answer}
	
	
	
	


\question Find a generating function for the sequence with recurrence relation $a_n = 3a_{n-1} - a_{n-2}$ with initial terms $a_0 = 1$ and $a_1 = 5$.

	\begin{answer}
		$\dfrac{1+2x}{1-3x + x^2}$  %Find a generating function for the sequence with recurrence relation $a_n = 3a_{n-1} - a_{n-2}$ with initial terms $a_0 = 1$ and $a_1 = 5$.
	\end{answer}
	
	
	
	


\question Use the recurrence relation for the Fibonacci numbers to find the generating function for the Fibonacci sequence.

	\begin{answer}
		Compute $A - xA - x^2A$ and the solve for $A$.  The generating function will be $\dfrac{x}{1-x-x^2}$.  %Use the recurrence relation for the Fibonacci numbers to find the generating function for the Fibonacci sequence.
	\end{answer}
	
	
	
	


\question Use multiplication to find the generating function for the sequence of partial sums of Fibonacci numbers, $S_0, S_1, S_2, \ldots$ where $S_0 = F_0$, $S_1 = F_0 + F_1$, $S_2 = F_0 + F_1 + F_2$, $S_3 = F_0 + F_1 + F_2 + F_3$ and so on.

	\begin{answer}
		$\dfrac{x}{(1-x)(1-x-x^2)}$  %Use multiplication to find the generating function for the sequence of partial sums of Fibonacci numbers, $S_0, S_1, S_2, \ldots$ where $S_0 = F_0$, $S_1 = F_0 + F_1$, $S_2 = F_0 + F_1 + F_2$, $S_3 = F_0 + F_1 + F_2 + F_3$ and so on.
	\end{answer}
	
	
	
	


\question Find the generating function for the sequence with closed formula $a_n = 2(5^n) + 7(-3)^n$.

	\begin{answer}
		$\dfrac{2}{1-5x} + \dfrac{7}{1+3x}$.  %Find the generating function for the sequence with closed formula $a_n = 2(5^n) + 7(-3)^n$.
	\end{answer}
	
	
	
	


\question Find a closed formula for the $n$th term of the sequence with generating function $\dfrac{3x}{1-4x} + \dfrac{1}{1-x}$.

	\begin{answer}
		$a_n = 3\cdot 4^{n-1} + 1$  %Find a closed formula for the $n$th term of the sequence with generating function $\dfrac{3x}{1-4x} + \dfrac{1}{1-x}$.
	\end{answer}
	
	
	
	


\question Find $a_7$ for the sequence with generating function $\dfrac{2}{(1-x)^2}\cdot\dfrac{x}{1-x-x^2}$

	\begin{answer}
		Hint: you should ``multiply'' the two sequences.  Answer: 158.  %Find $a_7$ for the sequence with generating function $\dfrac{2}{(1-x)^2}\cdot\dfrac{x}{1-x-x^2}$
	\end{answer}
	





 
\end{questions}



