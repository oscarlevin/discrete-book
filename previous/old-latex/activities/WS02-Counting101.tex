\documentclass[11pt]{exam}

\usepackage{amsmath, amssymb, multicol}
\usepackage{graphicx}
\usepackage{textcomp}
\usepackage{chessboard}

\def\d{\displaystyle}
\def\b{\mathbf}
\def\R{\mathbf{R}}
\def\Z{\mathbf{Z}}
\def\st{~:~}
\def\bar{\overline}
\def\inv{^{-1}}

%Create ``defbox'' environment to highlight important definitions
\newenvironment{defbox}[1]{\begin{framed}\noindent{\bf #1\\}}{\end{framed}}

%\pointname{pts}
\pointsinmargin
\marginpointname{pts}
\addpoints
\pagestyle{headandfoot}
%\printanswers

\firstpageheader{Math 228}{\bf Counting 101}{January 21, 2015}
\firstpagefooter{}{\bf STOP!  Do not turn this page over until instructed to do so.}{}
\runningfooter{}{}{}

\begin{document}

%space for name
%\noindent {\large\bf Name:} \underline{\hspace{2.5in}}
%\vskip 1em

\begin{questions}
\question A restaurant offers 8 appetizers and 14 entr\'ees.  How many choices do you have if:
\begin{parts}
 \part you will eat one dish - either an appetizers or an entr\'ee?
 \vfill
 \part you are extra hungry and want to eat both an appetizer and an entr\'ee?
 \vfill
\end{parts}
\question Think about the methods you used to solve the counting problems above.  Write down the rules for these methods.
\vfill
\vfill
\question Do your rules work?  A standard deck of playing cards has 26 red cards and 12 face cards.
\begin{parts}
  
 \part How many ways can you select a card which is either red or a face card?
 \vfill
 \part How many ways can you select a card which is both red and a face card?
 \vfill
 \part How many ways can you select two cards so that the first one is red and the second one is a face card?
 \vfill
\end{parts}

\newpage
\begin{defbox}{Additive Principle}
  The {\em additive principle} states that if event $A$ can occur in $m$ ways, and event $B$ can occur in $n$ {\em disjoint} ways, then the event ``$A$ or $B$'' can occur in $m + n$ ways.  
\end{defbox}

\begin{defbox}{Multiplicative Principle}
  The {\em multiplicative principle} states that if event $A$ can occur in $m$ ways, and each possibility for $A$ allows for exactly $n$ ways for event $B$, then the event ``$A$ and $B$'' can occur in $m \cdot n$ ways.
\end{defbox}

\question You have a bunch of chips which come in colors red, blue, green and yellow. 
\begin{parts}
\part How many different two-chip stacks can you make if the bottom chip must be red or blue?  Explain your answer using both the additive and multiplicative principles.
\vfill
\vfill
\part How many different three-chip stacks can you make if the bottom chip must be red or blue and the top chip must be green or yellow?  How does this problem relate to the previous one?
\vfill
\vfill
\part How many different three-chip stacks are there in which no color is repeated (but otherwise any colors could be on the top or bottom)?
\vfill
\vfill
\part Suppose you wanted to take three different colored chips and put them in your pocket.  How many different choices do you have?  How does this problem relate to the previous one?
\vfill
\end{parts} 


%\question How many two letter ``words'' start with either A or B?
%\vfill
%\question How many two letter ``words'' start with a vowel?
%\vfill



\end{questions}

%\noindent{\bf For next time:}  A recent survey of Village Inn patrons revealed the following pie preferences.  People were asked whether they enjoyed Apple, Blueberry, or Cherry pie.
%  The following table shows how many people like which kinds of pie, in their various combinations.
%\begin{center}
%\begin{tabular}{|l|c|c|c|c|c|c|c|}
%\hline
% Pies enjoyed: & A & B & C & AB & AC & BC & ABC\\
%\hline
%Number of people: & 20 & 13 & 26 & 9 & 15 & 7 & 5\\
%\hline
%\end{tabular}
%\end{center}
%
%How many of those asked enjoy at least one of the kinds of pie?  Hint: The answer is not 95.
%

% Homework: A rook can move only in straight lines (not diagonally).  Fill in each square of the chess board below with the number of different shortest paths the rook in the upper left corner can take to get to the square.  For example, one square is already filled in - there are four paths from the rook to the square: DRRR, RDRR, RRDR and RRRD.
% 
% \cbDefineNewPiece{white}{x}{$4$}{$4$}
% \centerline{\chessboard[largeboard, borderwidth=.5px, showmover=false, labelleft=false, labelbottom=false, color=blue, setpieces={ra8, xd7}, blackfieldcolor=gray, setfontcolors]}
\end{document}


