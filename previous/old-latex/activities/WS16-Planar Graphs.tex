\documentclass[11pt]{exam}

\usepackage{amsmath, amssymb, multicol}
\usepackage{graphicx}
\usepackage{textcomp}
\usepackage{chessboard}
\usepackage{tikz}

\def\d{\displaystyle}
\def\b{\mathbf}
\def\R{\mathbf{R}}
\def\Z{\mathbf{Z}}
\def\st{~:~}
\def\bar{\overline}
\def\inv{^{-1}}
\def\r{2.5pt}
\def\v{circle (\r)}


%\pointname{pts}
\pointsinmargin
\marginpointname{pts}
\addpoints
\pagestyle{head}
%\printanswers

\firstpageheader{Math 228}{\bf Planar Graphs}{Friday, April 10}


\begin{document}

%space for name
%\noindent {\large\bf Name:} \underline{\hspace{2.5in}}
%\vskip 1em
\noindent{\bf Definition:} $K_n$ is the complete graph with $n$ vertices. Namely, a graph with edges connecting each of the $n$ vertices 

\begin{questions}
\question Using the above definition of $K_n$, we would like to be able to say something about the number of vertices, edges, and (if the graph is planar) faces. Let's first consider $K_3$
\begin{parts}
\part How many vertices does $K_3$ have? How many edges?
\vfill
\part If $K_3$ is planar, how many faces should it have?
\vfill
\end{parts} 
\question Repeat parts (a) and (b) from question 1 for $K_4$ 
\vfill
\question Repeat parts (a) and (b) from question 1 for $K_5$
\vfill
\question Repeat parts (a) and (b) from question 1 for $K_{23}$
\vfill
\newpage
\uplevel{$K_{m,n}$ is the complete bipartite graph with sets of $m$ and $n$ vertices. \\
{\bf Bipartite graph}: A graph for which it is possible to divide the vertices into two disjoint sets such that there are no edges between any two vertices in the same set.\\
{\bf Complete bipartite graph}: A bipartite graph for which every vertex in the first set is adjacent to every vertex in the second set.}
\question How many edges does $K_{7,4}$ have?
\vfill
\question For which values of $m$ and $n$ is $K_{m,n}$ planar?
\vfill
%That is, $K_{m,n}$ has $m+n$ vertices that are divided into two groups - one with $m$ vertices, the other with $n$. If I am standing on one of the $n$ vertices, I should not be able to immediately get to another vertex in the set of $n$ vertices. In fact, the only vertices I should be able to get to are those that are in the set with $m$ vertices. The same thing holds for if I'm standing on one of the $m$ vertices - I should only be able to directly get to one of the $n$ vertices. What is the smallest $m$ and $n$ that you can find such that $K_{m,n}$ is not planar? Can you prove it?
\vfill
\end{questions}
%
%\vfill
%  Some graphs are used more than others, and get special names.
%  \begin{itemize}
%    \item $K_n$: the complete graph on $n$ vertices.
%    \item $K_{m,n}$: the complete bipartite graph with sets of $m$ and $n$ vertices.
%    \item $C_n$: the cycle graph on $n$ vertices - just one big loop.
%    \item $P_n$: the path graph on $n$ vertices - just one long path.
%  \end{itemize}
% % Here are some typical examples:
%  
%\def\sb{.6}
%\begin{center}
%\hfill
%\begin{tikzpicture}[scale=\sb+.05]
%  \path (0,.9) +(18:1) coordinate (a);
%  \path (0,.9) +(90:1) coordinate (b);
%  \path (0,.9) +(162:1) coordinate (c);
%  \path (0,.9) +(234:1) coordinate (d);
%  \path (0,.9) +(306:1) coordinate (e);
%  \draw[thick,fill=black] (a) \v -- (b) \v -- (c) \v -- (d) \v -- (e) \v;
%  \draw[thick] (e) -- (a) -- (c) -- (e) -- (b) -- (d) -- (a);
%  \draw (0,-.5) node[below]{\large $K_5$};
%\end{tikzpicture}
%\hfill
%\begin{tikzpicture}[scale=\sb, xscale=1.5]
% \draw[thick, fill=black] (-1, 0) \v -- (-.5,2) \v -- (0,0) \v -- (.5, 2) \v -- (1,0) \v -- (-.5,2) (.5,2) -- (-1,0);
% \draw (0,-.5) node[below]{\large $K_{2,3}$};
%  \end{tikzpicture}
%\hfill
%\begin{tikzpicture}[scale=\sb]
%  \draw[thick, fill=black] (0:1) \v -- (60:1) \v -- (120:1) \v -- (180:1) \v -- (240:1) \v -- (300:1) \v -- (0:1);
%  \draw (270:1.5) node[below]{\large $C_6$};
%\end{tikzpicture}
%\hfill
%\begin{tikzpicture}[scale=\sb]
%  \draw[thick, fill=black] (-2,0) \v -- (-1,.5) \v -- (0,0) \v -- (1,.75) \v -- (.5,1.5) \v -- (2,2) \v;
%  \draw (0,-.5) node[below]{\large $P_6$};
%\end{tikzpicture}
%\hfill
%~
%\end{center}


\end{document}


