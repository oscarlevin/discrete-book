% % % % % % % % % % % % % % % % % % % % % % % % % % % % % % % % % % % % % % % % % % % % % % % % %
% % 	This is the prefice of Discrete Mathematics: and Open Introduction 	% % % % % % % % % % %
% % % % % % % % % % % % % % % % % % % % % % % % % % % % % % % % % % % % % % % % % % % % % % % % %

\chapter{Preface}

This text aims to give an introduction to select topics in discrete mathematics at a level appropriate for first or second year undergraduate math majors, especially those who intend to teach middle and high school mathematics.  The book began as a set of notes for the Discrete Mathematics course at the University of Northern Colorado.  This course serves both as a survey of the topics in discrete math and as the ``bridge'' course for math majors, as UNC does not offer a separate ``introduction to proofs'' course.  Most students who take the course plan to teach, although there are a handful of students who will go on to graduate school or study applied math and computer science.  For these students the current text hopefully is still of interest, but the intent is not to provide a solid mathematical foundation for computer science, unlike the majority of textbooks on the subject.

Another difference between this text and most other discrete math books is that this book is intended to be used in a class taught using problem oriented or inquiry based methods.  When I teach the class, I will assign sections for reading \emph{after} first introducing them in class by using a mix of group work and class discussion on a few interesting problems.  The text is meant to consolidate what we \emph{discover} in class and serve as a reference for students as they master the concepts and techniques covered in the unit.  None-the-less, every attempt has been made to make the text sufficient for self study as well, in a way that hopefully mimics an inquiry based classroom.

\todo[inline]{Perhaps a paragraph about topics and order of topics?}

\section*{How to use this book}

In addition to expository text, this book has a few features designed to encourage you to interact with the mathematics.

\subsubsection*{\textit{Investigate!} activities}  Sprinkled throughout the sections (usually at the very beginning of a topic) you will find activities designed to get you acquainted with the topic soon to be discussed.  These are similar (sometimes identical) to group activities I give students to introduce material.  You really should spend some time thinking about, or even working through, these problems before reading the section.  By priming yourself to the types of issues involved in the material you are about to read, you will better understand what's to come.  There are no provided solutions for these problems, but don't worry if you can't solve them or are not confident in your answers.  My hope is that you will take this frustration with you while you read the proceeding section.  By the time you are done with the section, things should be much clearer.

\subsubsection*{Examples}

I have tried to include the ``correct'' number of examples.  For those examples which include \emph{problems}, full solutions are included. Before reading the solution, try to at least have an understanding of what the problem is asking.  Unlike some textbooks, the examples are not meant to be all inclusive for problems you will see in the exercises.  They should not be used as a blueprint for solving other problems.  Instead, use the examples to deepen our understanding of the concepts and techniques discussed in each section.  Then use this understanding to solve the exercises at the end of each section.

\subsubsection*{Exercises}

You get good at math through practice.  Each section concludes with a small number of exercises meant to solidify concepts and basic skills presented in that section.  At the end of each chapter, a larger collection of similar exercises is included (as a sort of ``chapter review'') which might bridge material of different sections in that chapter.  Every exercise has either a hint, answer or full solution (which in the pdf version of the text can be found by clicking on the exercises number -- clicking on the solution number will bring you back to the exercise).  Readers are encouraged to try these exercises before looking at the solution.  When I teach with this book, I assign these exercises as practice and then use them, or similar problems, on quizzes and exams.

\subsubsection*{Homework Problems} 

Each chapter includes a small number of more involved problems -- the type I would assign as homework to be written up and collected each week.  As many of these are problems I assign, solutions are not included.  If you are using this book for self study, consider these additional \emph{Investigate!} problems.

\section*{Acknowledgments}

\todo[inline]{Add something about Richard's discrete book}

In Spring 2015, Alees Seehausen, a graduate student at the University of Northern Colorado, co-taught the Discrete Mathematics course with me and helped develop many of the \emph{Investigate!} activities  and other problems currently used in the text.  She also offered many suggestions for improvement of the expository text, for which I am quite grateful.


\begin{flushright}
Oscar Levin, Ph.D.
\end{flushright}
\newpage