\chapter{Preface}

This is some information about the book.

\section*{Features of this book}

In addition to expository text, this book has a few features designed to encourage you to interact with the mathematics.

\subsubsection*{\textit{Investigate!} activities}  Sprinkled throughout the sections (usually at the very beginning of a topic) you will find activities designed to get you acquainted with the topic soon to be discussed.  These are similar (sometimes identical) to group activities I give students to introduce material.  You really should spend some time thinking about, or even working through, these problems before reading the section.  By priming yourself to the types of issues involved in the material you are about to read, you will better understand what's to come.  There are no provided solutions for these problems, but don't worry if you can't solve them or are not confident in your answers.  My hope is that you will take this frustration with you while you read the proceeding section.  By the time you are done with the section, things should be much clearer.

\subsubsection*{Examples}

I have tried to include the ``correct'' number of examples.  For those examples which include \emph{problems}, full solutions are included. Before reading the solution, try to at least have an understanding of what the problem is asking.  Unlike some textbooks, the examples are not meant to be all inclusive for problems you will see in the exercises.  They should not be used as a blueprint for solving other problems.  Instead, use the examples to deepen our understanding of the concepts and techniques discussed in each section.  Then use this understanding to solve the exercises at the end of each section.

\subsubsection*{Exercises}

You get good at math through practice.  


\begin{flushright}
Oscar Levin, Ph.D.
\end{flushright}
\newpage