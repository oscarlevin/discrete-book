\documentclass[11pt]{exam}

\usepackage{amsmath, amssymb, multicol}
\usepackage{graphicx}
\usepackage{textcomp}
\usepackage{chessboard}

\def\d{\displaystyle}
\def\b{\mathbf}
\def\R{\mathbf{R}}
\def\Z{\mathbf{Z}}
\def\st{~:~}
\def\bar{\overline}
\def\inv{^{-1}}


%\pointname{pts}
\pointsinmargin
\marginpointname{pts}
\addpoints
\pagestyle{head}
%\printanswers

\firstpageheader{Math 228}{\bf Counting 101}{March 19, 2012}


\begin{document}

%space for name
%\noindent {\large\bf Name:} \underline{\hspace{2.5in}}
%\vskip 1em

\begin{questions}
\question A restaurant offers 8 appetizers and 14 entr\'ees.  How many choices do you have if:
\begin{parts}
 \part you will eat one dish - either an appetizers or an entr\'ee?
 \vfill
 \part you are extra hungry and want to eat both an appetizer and an entr\'ee?
 \vfill
\end{parts}
\question How many two letter ``words'' start with either A or B?
\vfill
\question How many two letter ``words'' start with a vowel?
\vfill
\question Think about the methods you used to solve the counting problems above.  Write down the rules for these methods, and why they work.
\vfill
\vfill
\vfill
\vfill
\newpage
\question Do your rules work?  A standard deck of playing cards has 26 red cards and 12 face cards.
\begin{parts}
  
 \part How many ways can you select a card which is either red or a face card?
 \vfill
 \part How many ways can you select a card which is both red and a face card?
 \vfill
 \part How many ways can you select two cards so that the first one is red and the second one is a face card?
 \vfill
\end{parts}

\end{questions}

Homework: A rook can move only in straight lines (not diagonally).  Fill in each square of the chess board below with the number of different shortest paths the rook in the upper left corner can take to get to the square.  For example, one square is already filled in - there are four paths from the rook to the square: DRRR, RDRR, RRDR and RRRD.

\cbDefineNewPiece{white}{x}{$4$}{$4$}
\centerline{\chessboard[largeboard, borderwidth=.5px, showmover=false, labelleft=false, labelbottom=false, color=blue, setpieces={ra8, xd7}, blackfieldcolor=gray, setfontcolors]}
\end{document}


