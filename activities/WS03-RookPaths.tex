\documentclass[11pt]{exam}

\usepackage{amsmath, amssymb, multicol}
\usepackage{graphicx}
\usepackage{textcomp}
\usepackage{chessboard}
\usepackage[top=.25in, bottom=.25in, left=1in, right=1in]{geometry}

\def\d{\displaystyle}
\def\b{\mathbf}
\def\R{\mathbf{R}}
\def\Z{\mathbf{Z}}
\def\st{~:~}
\def\bar{\overline}
\def\inv{^{-1}}


%\pointname{pts}
\pointsinmargin
\marginpointname{pts}
\addpoints
\pagestyle{head}
%\printanswers

%\firstpageheader{Math 228}{\bf Pre}{February 18, 2013}


\begin{document}

%space for name
%\noindent {\large\bf Name:} \underline{\hspace{2.5in}}
%\vskip 1em

\centerline{{\bf Math 228 \hfill Monday, Jan 26}}
\vskip 1em
 A rook can move only in straight lines (not diagonally).  Fill in each square of the chess board below with the number of different shortest paths the rook in the upper left corner can take to get to the square.  For example, one square is already filled in - there are four paths from the rook to the square: DRRR, RDRR, RRDR and RRRD.
 
 \cbDefineNewPiece{white}{x}{$4$}{$4$}
 \centerline{\chessboard[largeboard, borderwidth=.5px, showmover=false, labelleft=false, labelbottom=false, color=blue, setpieces={ra8, xd7}, blackfieldcolor=gray, setfontcolors]}
 
% \centerline{{\bf Math 228 \hfill Monday, Jan 26}}
%\vskip 1em
% A rook can move only in straight lines (not diagonally).  Fill in each square of the chess board below with the number of different shortest paths the rook in the upper left corner can take to get to the square.  For example, one square is already filled in - there are four paths from the rook to the square: DRRR, RDRR, RRDR and RRRD.
% 
%% \cbDefineNewPiece{white}{x}{$4$}{$4$}
% \centerline{\chessboard[largeboard, borderwidth=.5px, showmover=false, labelleft=false, labelbottom=false, color=blue, setpieces={ra8, xd7}, blackfieldcolor=gray, setfontcolors]}

\end{document}


