\documentclass[11pt]{exam}

\usepackage{amsmath, amssymb, multicol}
\usepackage{graphicx}
\usepackage{textcomp}
\usepackage{chessboard}

\def\d{\displaystyle}
\def\b{\mathbf}
\def\R{\mathbf{R}}
\def\Z{\mathbf{Z}}
\def\st{~:~}
\def\bar{\overline}
\def\inv{^{-1}}

\def\v{circle (3pt)}


%\pointname{pts}
\pointsinmargin
\marginpointname{pts}
\addpoints
\pagestyle{head}
%\printanswers

\firstpageheader{Math 228}{\bf Sequences of Candy}{Monday, Feb 23}


\begin{document}

%space for name
%\noindent {\large\bf Name:} \underline{\hspace{2.5in}}
%\vskip 1em

\subsection*{Activity 1: Skittle Machine}
King Soopers has a candy machine full of Skittles at the front of the store.
\begin{itemize}
\item[1)] Suppose that the candy machine currently holds exactly 650 Skittles, and every time someone inserts a quarter, exactly 7 Skittles come out of the machine. 
\begin{itemize}
\item[(a)]How many Skittles will be left in the machine after 20 quarters have been inserted?
\vskip 1 in
\item[(b)] Will there ever be exactly zero Skittles left in the machine? Explain.
\vskip 1 in
\end{itemize}
\item[2)] What if the candy machine gives 7 Skittles to the first customer who put in a quarter, 10 to the second, 13 to the third, 16 to the fourth, etc. How many Skittles has the machine given out after 20 quarters are put into the machine?
\vfill
\item[3)] Now, what if the machine gives 4 Skittles to the first customer, 7 to the second, 12 to the third, 19 to the fourth, etc. How many Skittles has the machine given out after 20 quarters are put into the machine?
\end{itemize}
\vfill
\end{document}


