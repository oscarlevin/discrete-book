\documentclass[11pt]{exam}

\usepackage{amsmath, amssymb, multicol}
\usepackage{graphicx}
\usepackage{textcomp}
\usepackage{chessboard}
\usepackage{tikz}
\usepackage{pdfpages}

\def\d{\displaystyle}
\def\b{\mathbf}
\def\R{\mathbf{R}}
\def\Z{\mathbf{Z}}
\def\st{~:~}
\def\bar{\overline}
\def\inv{^{-1}}
\def\r{2.5pt}
\def\v{circle (\r)}


%\pointname{pts}
\pointsinmargin
\marginpointname{pts}
\addpoints
\pagestyle{head}
%\printanswers

\firstpageheader{Math 228}{\bf Coloring Graphs}{Monday, April 13}


\begin{document}

%space for name
%\noindent {\large\bf Name:} \underline{\hspace{2.5in}}
%\vskip 1em
A \emph{proper vertex coloring} of a graph is a coloring in which no two adjacent vertices are colored the same color.  The \emph{chromatic number} of a graph is the smallest number of colors needed for a proper coloring of the graph.

For each graph below, find a proper coloring and determine the chromatic number of the graph.  As you are working on these colorings, think about the following ideas:
\begin{enumerate}
\item Before I even start coloring the graph, can I come up with an upper bound on the number of colors I will need (a bound that is better than simply the total number of vertices in the graph)? 

\item Before I even start coloring the graph, can I come up with a lower bound on the number of colors I will need (a bound that is better than simply 2)?  

\item Once you have colored the graph, make sure that you check your coloring with your tablemates to try to determine if you used the minimal number of colors.  Talk with each other about reasons why you couldn't have used fewer colors than you did.
  
\item For each graph, try to identify if it is a type of graph that we've talked about before (i.e.\ does it have a special name or special properties).  If the graph is a named graph, try to see if you can generalize the ideas you used in your coloring of this graph to other graphs of that same type.
\end{enumerate}

\begin{multicols}{2}
\includegraphics[scale=.75, page=1]{color.pdf}
\vskip 1em
\includegraphics[scale=.75, page=3]{color.pdf}

\includegraphics[scale=.75, page=2]{color.pdf}
\vskip 1em
\includegraphics[scale=.75, page=4]{color.pdf}
\end{multicols}
%\newpage
%\phantom{blah}


\includepdf[scale=.9, nup=2x4,delta=3cm 1cm,pages=5-last,noautoscale=false]{color.pdf}


\end{document}


