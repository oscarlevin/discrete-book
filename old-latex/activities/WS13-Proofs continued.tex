\documentclass[11pt]{exam}

\usepackage{amsmath, amssymb, multicol}
\usepackage{graphicx}
\usepackage{textcomp}
\usepackage{chessboard}

\def\d{\displaystyle}
\def\b{\mathbf}
\def\R{\mathbf{R}}
\def\Z{\mathbf{Z}}
\def\st{~:~}
\def\bar{\overline}
\def\inv{^{-1}}


%\pointname{pts}
\pointsinmargin
\marginpointname{pts}
\addpoints
\pagestyle{head}
%\printanswers

\firstpageheader{Math 228}{\bf More Proofs}{Wednesday, April 1}


\begin{document}

%space for name
%\noindent {\large\bf Name:} \underline{\hspace{2.5in}}
%\vskip 1em
Suppose you have a collection of 5-cent stamps and 8-cent stamps.  We saw earlier that it is possible to make any amount of postage greater than 27 cents using combinations of both these types of stamps.  But, let's ask some other questions:
\begin{questions}
\question What amounts of postage can you make if you only use an even number of both types of stamps? Prove your answer.
\vfill
\question Suppose you made an even amount of postage.  Prove that you used an even number of at least one of the types of stamps.
\vfill
\question Suppose you made exactly 72 cents of postage.  Prove that you used at least 6 of one type of stamp.
\vfill
\end{questions}

%April Fools!
%\noindent If there's time, prove: Every even number greater than 2 can be written as the sum of two primes.
\end{document}


