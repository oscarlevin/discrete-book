\documentclass[11pt]{exam}

\usepackage{amsmath, amssymb, multicol}
\usepackage{graphicx}
\usepackage{textcomp}
\usepackage{chessboard}

\def\d{\displaystyle}
\def\b{\mathbf}
\def\R{\mathbf{R}}
\def\Z{\mathbf{Z}}
\def\st{~:~}
\def\bar{\overline}
\def\inv{^{-1}}


%\pointname{pts}
\pointsinmargin
\marginpointname{pts}
\addpoints
\pagestyle{head}
%\printanswers

\firstpageheader{Math 228}{\bf Towers and Tiles}{Wednesday, Feb 18}


\begin{document}

%space for name
%\noindent {\large\bf Name:} \underline{\hspace{2.5in}}
%\vskip 1em

\subsection*{Activity 1: Tower of Hanoi}

As the legend goes, there is a monastery in Hanoi with a great hall containing 3 tall pillars.  Resting on the first pillar are 64 giant disks (or washers) - all different sizes, stacked from largest to smallest.  The monks are charged with the following task: they must move the entire stack of disks to the third pillar.  However, due to the size of the disks, the monks cannot move more than one at a time.  Each disk must be placed on one of the pillars before the next disk is moved.  And because the disks are so heavy and fragile, the monks may never place a larger disk on top of a smaller disk.

When the monks finally complete their task, the world shall come to an end.  Your task: figure out how long before we need to start worrying about the end of the world.

\begin{questions}
  \question First, let's find the minimum number of moves required for a smaller number of disks.  Collect some data. Make a table.
  \vfill
  \question Conjecture a formula for the minimum number of moves required to move $n$ disks.  Test your conjecture.  How do you know your formula is correct?
  \vfill
  \question If the monks were able to move one disk every second without ever stopping, how long before the world ends?
  
  \vfill
  
\end{questions}


\newpage

\subsection*{Activity 2: Counting Strips}

The goal: You have a large collection of $1\times 1$ squares and $1\times 2$ dominoes.  You want to arrange these to make a $1 \times 15$ strip.  How many ways can you do this?
\begin{questions}
  \question Again, start by collecting data: How many length $1\times 1$ strips can you make?  How many $1\times 2$ strips?  How many $1\times 3$ strips?  And so on.
  \vfill
  \question How are the $1\times 3$ and $1 \times 4$ strips related to the $1\times 5$ strips?  
  \vfill
  \question How many $1\times 15$ strips can you make?
  \vfill
  \vfill
  \question What if I asked you to find the number of $1\times 1000$ strips?  Would the method you used to calculate the number fo  $1 \times 15$ strips be helpful?  
%  \question Can you see a relationship between the number of length 3 paths, the number of length 4 paths, and the number of length 5 paths?  Does this relationship work for length 4, 5, and 6 paths as well?  Use this to find the number of length 10 paths.
%  \vfill
%  \question Does the relationship you found in part 2 always hold?  Why?
%  \vfill
%  \question If you were to build all possible length 1 paths, all possible length 2 paths, and so on up to all possible length $5$, how many paths will you have total?  What if you go up to all paths of length $n$?
%  \vfill
\end{questions}



\end{document}


