\documentclass[11pt]{exam}

\usepackage{amsmath, amssymb, multicol}
\usepackage{graphicx}
\usepackage{textcomp}
\usepackage{chessboard}

\def\d{\displaystyle}
\def\b{\mathbf}
\def\R{\mathbf{R}}
\def\Z{\mathbf{Z}}
\def\st{~:~}
\def\bar{\overline}
\def\inv{^{-1}}


%\pointname{pts}
\pointsinmargin
\marginpointname{pts}
\addpoints
\pagestyle{head}
%\printanswers

\firstpageheader{Math 228}{\bf Logic and Proofs}{Monday, March 23}


\begin{document}

%space for name
%\noindent {\large\bf Name:} \underline{\hspace{2.5in}}
%\vskip 1em
\begin{center}
Holmes owns two suits: one blue and one brown.  He always wears either a blue suit or white socks.  Whenever he wears his blue suit and a blue shirt, he also wears a blue tie.  He never wears the blue suit unless he is also wearing either a blue shirt or white socks.  Whenever he wears white socks, he also wears a blue shirt.  Today, Holmes is wearing a gold tie.  What else is he wearing?
\end{center}
\newpage
Consider the following statement: 
\begin{center}
If $a$ times $b$ is an even number, then $a$ is even or $b$ is even.
\end{center}
Based on the ideas that we discussed in class today, decide whether the following proofs of the above statement are valid or invalid.
\begin{questions}
\question
Suppose $a=2k+1$ ($a$ is odd) and $b=2m+1$ ($b$ is odd). Then 
\begin{align*} 
ab &=(2k+1)(2m+1)\\
&=4km+2k+2m+1\\
&=2(2km+k+m)+1
\end{align*}
Which proves that $ab$ is odd if $a$ and $b$ are odd. Therefore, if $ab$ is even, then $a$ or $b$ is be even.
\vfill

\question
Assume that $a$ or $b$ is even - say it is $a$. That is, $a=2k$ for some integer $k$. Then 
\begin{align*} 
ab &=(2k)b\\
&=2(kb)
\end{align*} 
Which means that $ab$ is even.  The case where $b$ is even is identical.  Therefore, if $ab$ is even then $a$ is even or $b$ is even.
\vfill

\question
Suppose that $ab$ is even but $a$ and $b$ are both odd. Namely, $ab = 2n$, $a=2k+1$ and $b=2j+1$ for some integers $n$, $k$, and $j$. Then 
\begin{align*} 
2n &=(2k+1)(2j+1)\\
2n &=4kj+2k+2j+1\\
n &= 2kj+k+j+\frac{1}{2}
\end{align*}
But since $2kj+k+j$ is an integer, this says that the integer $n$ is equal to a non-integer, which is impossible.
Therefore, if $ab$ is even then $a$ or $b$ must be even.
\vfill

\question 
Let $ab$ be an even number, $ab=2n$, and $a$ be an odd number, $a=2k+1$. Then
\begin{align*} 
ab &=(2k+1)b\\
2n &=2kb+b\\
2n-2kb&=b\\
2(n-kb)&=b
\end{align*}
Therefore, $b$ must be even. So, if $ab$ is even then either $a$ or $b$ must also be even.
\end{questions}
\end{document}


