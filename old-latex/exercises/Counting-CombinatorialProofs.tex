\begin{questions}

\question Prove the identity ${n\choose k} = {n-1 \choose k-1} + {n-1 \choose k}$ using a question about subsets.

	\begin{answer}
		\begin{proof}
			\underline{Question}: How many subsets of size $k$ are there of the set $\{1,2,\ldots, n\}$?

			\underline{Answer 1}: You must choose $k$ out of $n$ elements to put in the set, which can be done in ${n \choose k}$ ways.

			\underline{Answer 2}: First count the number of $k$-element subsets of $\{1,2,\ldots, n\}$ which contain the number $n$.  We must choose $k-1$ of the $n-1$ other element to include in this set.  Thus there are ${n-1\choose k-1}$ such subsets.  We have not yet counted all the $k$-element subsets of $\{1,2,\ldots, n\}$ though.  In fact, we have missed exactly those subsets which do NOT contain $n$.  To form one of these subsets, we need to choose $k$ of the other $n-1$ elements, so this can be done in ${n-1 \choose k}$ ways.    Thus the answer to the question is ${n-1 \choose k-1} + {n-1 \choose k}$.
		\end{proof}
	\end{answer}


\question Give a combinatorial proof of the identity $2+2+2 = 3\cdot 2$.

	\begin{answer}
		\begin{proof}
		\underline{Question}: How many 2-letter words start with \textit{a}, \textit{b}, or \textit{c} and end with either \textit{y} or \textit{z}?

		\underline{Answer 1}: There are two words that start with \textit{a}, two that start with \textit{b}, two that start with \textit{c}, for a total of $2+2+2$.

		\underline{Answer 2}:  There are three choices for the first letter and two choices for the second letter, for a total of $3 \cdot 2$.

		Since the two answer are both answers to the same question, they are equal.  Thus $2 + 2 + 2 = 3\cdot 2$.
		\end{proof}
	\end{answer}

\question Give a combinatorial proof for the identity $1 + 2 + 3 + \cdots + n = {n+1 \choose 2}$.

	\begin{answer}
	\begin{proof}
        \underline{Question}: How many subsets of $A = {1,2,3, \ldots, n+1}$ contain exactly two elements?

        \underline{Answer 1}: We must choose 2 elements from $n+1$ choices, so there are ${n+1 \choose 2}$ subsets.

        \underline{Answer 2}: We break this question down into cases, based on what the larger of the two elements in the subset is. The larger element can't be 1, since we need at least one element smaller than it.

        Larger element is 2: there is 1 choice for the smaller element.

        Larger element is 3: there are 2 choices for the smaller element.

        Larger element is 4: there are 3 choices for the smaller element.

        And so on.  When the larger element is $n+1$, there are $n$ choices for the smaller element.  Since each two element subset must be in exactly one of these cases, the total number of two element subsets is $1 + 2 + 3 + \cdots + n$.

        Answer 1 and answer 2 are both correct, so they must be equal.  Therefore
        \[1 + 2 + 3 + \cdots + n = {n+1 \choose 2}\]
       \end{proof}
	\end{answer}





\question A woman is getting married.  She has 15 best friends but can only select 6 of them to be her bridesmaids, one of which needs to be her maid of honor.  How many ways can she do this?
\begin{parts}
 \part What if she first selects the 6 bridesmaids, and then selects one of them to be the maid of honor?
 \part What if she first selects her maid of honor, and then 5 other bridemaids?
 \part Explain why $6 {15 \choose 6} = 15 {14 \choose 5}$.
\end{parts}

	\begin{answer}
		\begin{parts}
		 \part She has ${15 \choose 6}$ ways to select the 6 bridesmaids, and then for each way, has 6 choices for the maid of honor.  Thus she has ${15 \choose 6}6$ choices.  %What if she first selects the 6 bridesmaids, and then selects one of them to be the maid of honor?
		 \part She has 15 choices for who will be her maid of honor.  Then she needs to select 5 of the remaining 14 friends to be bridesmaids, which she can do in ${14 \choose 5}$ ways.  Thus she has $15 {14 \choose 5}$ choices.  %What if she first selects her maid of honor, and then 5 other bridesmaids?
		 \part We have answered the question (how many wedding parties can the bride choose from) in two ways.  The first way gives the left hand side of the identity and the second way gives the right-hand side of the identity.  Therefore the identity holds. %Explain why $6 {15 \choose 6} = 15 {14 \choose 5}$.
		\end{parts}
	\end{answer}


\question Give a combinatorial proof of the identity ${n \choose 2}{n-2 \choose k-2} = {n\choose k}{k \choose 2}$.

	\begin{answer}
		\begin{proof}
		\underline{Question}: You have a large container filled with ping-pong balls, all with a different number of them.  You must select $k$ of the balls, putting two of them in a jar and the others in a box.  How many ways can you do this?

		\underline{Answer 1}: First select 2 of the $n$ balls to put in the jar, then select $k-2$ of the remaining $n-2$ balls to put in the box.  The first task can be completed in ${n \choose 2}$ different ways, the second task in ${n-2 \choose k-2}$ ways.  Thus there are ${n \choose 2}{n-2 \choose k-2}$ ways to select the balls.

		\underline{Answer 2}:  First select $k$ balls from the $n$ in the container.  Then pick 2 of the $k$ balls you picked to put in the jar, placing the remaining $k-2$ in the box.  The first task can be completed in ${n \choose k}$ ways, the second task in ${k \choose 2}$ ways.  Thus there are ${n \choose k}{k \choose 2}$ ways to select the balls.

		Since both answers count the same thing, they must be equal.
		\end{proof}
	\end{answer}



\question Consider the bit strings in $\B^6_2$ (bit strings of length 6 and weight 2).
\begin{parts}
 \part How many of those bit strings start with 1?
 \part How many of those bit strings start with 01?
 \part How many of those bit strings start with 001?
 \part Are there any other strings we have not counted yet?  Which ones, and how many are there?
 \part How many bit strings are there total in $\B^6_2$?
 \part What binomial identity have you just given a combinatorial proof for?
\end{parts}

	\begin{answer}
		\begin{parts}
		 \part After the 1, we need to find a 5-bit string with one 1.  There are ${5 \choose 1}$ ways to do this. %How many of those bit strings start with 1?
		 \part ${4 \choose 1}$ (we need to pick 1 of the remaining 4 slots to be the second 1). %How many of those bit strings start with 01?
		 \part ${3 \choose 1}$ %How many of those bit strings start with 001?
		 \part Yes.  We still need strings starting with 0001 (there are ${2 \choose 1}$ of these) and strings starting 00001 (there is only ${1 \choose 1} = 1$ of these).  %Are there any other strings we have not counted yet?  Which ones, and how many are there?
		 \part ${6 \choose 2}$ %How many bit strings are there total in $\B^6_2$?
		 \part An example of the Hockey Stick Theorem:  %What binomial identity have you just given a combinatorial proof for?
		 \[{1 \choose 1} + {2 \choose 1} + {3 \choose 1} + {4 \choose 1} + {5 \choose 1} = {6 \choose 2}\]
		\end{parts}
	\end{answer}




\question Let's count {\em ternary} digit strings, that is, strings in which each digit can be 0, 1, or 2.
\begin{parts}
 \part How many ternary digit strings contain exactly $n$ digits?
 \part How many ternary digit strings contain exactly $n$ digits and $n$ 2's.
 \part How many ternary digit strings contain exactly $n$ digits and $n-1$ 2's.  (Hint: where can you put the non-2 digit, and then what could it be?)
 \part How many ternary digit strings contain exactly $n$ digits and $n-2$ 2's.  (Hint: see previous hint)
 \part How many ternary digit strings contain exactly $n$ digits and $n-k$ 2's.
 \part How many ternary digit strings contain exactly $n$ digits and no 2's. (Hint: what kind of a string is this?)
 \part Use the above parts to give a combinatorial proof for the identity
 \[{n \choose 0} + 2{n \choose 1} + 2^2{n \choose 2} + 2^3{n \choose 3} + \cdots + 2^n{n \choose n} = 3^n.\]
\end{parts}

	\begin{answer}
		\begin{parts}
		 \part $3^n$, since there are 3 choices for each of the $n$ digits.  %How many ternary digit strings contain exactly $n$ digits?
		 \part $1$, since all the digits need to be 2's.  However, we might write this as ${n \choose 0}$.  %How many ternary digit strings contain exactly $n$ digits and $n$ 2's.
		 \part There are ${n \choose 1}$ places to put the non-2 digit.  That digit can be either a 0 or a 1, so there are $2{n \choose 1}$ such strings.  %How many ternary digit strings contain exactly $n$ digits and $n-1$ 2's.  (Hint: where can you put the non-2 digit, and then what could it be?)
		 \part We must choose two slots to fill with 0's or 1's.  There are ${n \choose 2}$ ways to do that.  Once the slots are picked, we have two choices for the first slot (0 or 1) and two choices for the second slot (0 or 1).  So there are a total of $2^2{n \choose 2}$ such strings. %How many ternary digit strings contain exactly $n$ digits and $n-2$ 2's.  (Hint: see previous hint)
		 \part There are ${n \choose k}$ ways to pick which slots don't have the 2's.  Then those slots can be filled in $2^k$ ways (0 or 1 for each slot).  So there are $2^k{n \choose k}$ such strings. %How many ternary digit strings contain exactly $n$ digits and $n-k$ 2's.
		 \part These strings contain just 0's and 1's, so they are bit strings.  There are $2^n$ bit strings.  But keeping with the pattern above, we might write this as $2^n {n \choose n}$. %How many ternary digit strings contain exactly $n$ digits and no 2's. (Hint: what kind of a string is this?)
		 \part We answer the question of how many length $n$ ternary digit strings there are in two ways.  First, each digit can be one of three choices, so the total number of strings is $3^n$.  On the other hand, we could break the question down into cases by how many of the digits are 2's.  If they are all 2's, then there are ${n \choose 0}$ strings.  If all but one is a 2, then there are $2{n \choose 1}$ strings.  If all but 2 of the digits are 2's, then there are $2^2{n \choose 2}$ strings.  We choose 2 of the $n$ digits to be non-2, and then there are 2 choices for each of those digits.  And so on for every possible number of 2's in the string.  %Use the above parts to give a combinatorial proof for the identity
		 %\[{n \choose 0} + 2{n \choose 1} + 2^2{n \choose 2} + 2^3{n \choose 3} + \cdots + 2^n{n \choose n} = 3^n\]
		\end{parts}
	\end{answer}



\question How many ways are there to rearrange the letters in the word ``rearrange''? 	Answer this question in at least two different ways to establish a binomial identity.

	\begin{answer}
		The word contains 9 letters: 3 ``r''s, 2 ``a''s and 2 ``e''s, along with an ``n'' and a ``g''.  We could first select the positions for the ``r''s in ${9 \choose 3}$ ways, then the ``a''s in ${6 \choose 2}$ ways, the ``e''s in ${4 \choose 2}$ ways and then select one of the remaining two spots to put the ``n'' (placing the ``g'' in the last spot).  This gives the answer
		\[{9 \choose 3}{6 \choose 2}{4 \choose 2}{2\choose 1}{1\choose 1}.\]
		Alternatively, we could select the positions of the letters in the opposite order, which would give an answer
		\[{9 \choose 1}{8\choose 1}{7 \choose 2}{5\choose 2}{3\choose 3}.\]
		(where the 3 ``r''s go in the remaining 3 spots).  These two expressions are equal:
		\[{9 \choose 3}{6 \choose 2}{4 \choose 2}{2\choose 1}{1\choose 1} = {9 \choose 1}{8\choose 1}{7 \choose 2}{5\choose 2}{3\choose 3}.\]
	\end{answer}




\question Give a combinatorial proof for the identity $P(n,k) = {n \choose k}k!$

	\begin{answer}
		\begin{proof}
         \underline{Question}: How many $k$-letter words can you make using $n$ different letters without repeating any letter?

         \underline{Answer 1}: There are $n$ choices for the first letter, $n-1$ choices for the second letter, $n-2$ choices for the third letter, and so on until $n - (k-1)$ choices for the $k$th letter (since $k-1$ letters have already been assigned at that point).  The product of these numbers can be written $\frac{n!}{(n-k)!}$ which is $P(n,k)$.

         \underline{Answer 2}: First pick $k$ letters to be in the word from the $n$ choices.  This can be done in ${n \choose k}$ ways.  Now arrange those letters into a word.  There are $k$ choices for the first letter, $k-1$ choices for the second, and so on, for a total of $k!$ arrangements of the $k$ letters.  Thus the total number of words is ${n \choose k}k!$.
        \end{proof}
	\end{answer}



\question Establish the identity below using a combinatorial proof.

\[{2 \choose 2}{n \choose 2} + {3 \choose 2}{n-1 \choose 2} + {4\choose 2}{n-2 \choose 2} + \cdots + {n\choose 2}{2\choose 2} = {n+3 \choose 5}. \]


	\begin{answer}
		\begin{proof}
		\underline{Question}: How many 5-element subsets are there of the set $\{1,2,\ldots, n+3\}$.

		\underline{Answer 1}: We choose 5 out of the $n+3$ elements, so ${n+3 \choose 5}$.

		\underline{Answer 2}: Break this up into cases by what the ``middle'' (third smallest) element of the 5 element subset is.  The smallest this could be is a 3.  In that case, we have ${2 \choose 2}$ choices for the numbers below it, and ${n \choose 2}$ choices for the numbers above it.  Alternatively, the middle number could be a 4.  In this case there are ${3 \choose 2}$ choices for the bottom two numbers and ${n-1 \choose 2}$ choices for the top two numbers.  If the middle number is 5, then there are ${4 \choose 2}$ choices for the bottom two numbers and ${n-2 \choose 2}$ choices for the top two numbers.  An so on, all the way up to the largest the middle number could be, which is $n+1$.  In that case there are ${n \choose 2}$ choices for the bottom two numbers and ${2 \choose 2}$ choices for the top number.  Thus the number of 5 element subsets is.
		\[{2 \choose 2}{n \choose 2} + {3 \choose 2}{n-1 \choose 2} + {4\choose 2}{n-2 \choose 2} + \cdots + {n\choose 2}{2\choose 2}.\]
		\end{proof}
	\end{answer}



\end{questions}
