\documentclass[11pt]{exam}

\usepackage{amsmath, amssymb, multicol}
\usepackage{graphicx}
\usepackage{textcomp}


\def\d{\displaystyle}
\def\b{\mathbf}
\def\R{\mathbf{R}}
\def\Z{\mathbf{Z}}
\def\st{~:~}
\def\bar{\overline}
\def\inv{^{-1}}
\def\pow{\mathcal P}


%\pointname{pts}
\pointsinmargin
\marginpointname{pts}
\addpoints
\pagestyle{head}
%\printanswers

\firstpageheader{Math 228}{\bf Binomial Identities}{Spring 2013}


\begin{document}

%space for name
%\noindent {\large\bf Name:} \underline{\hspace{2.5in}}
%\vskip 1em

\begin{questions}
\question Explain why ${n \choose k} = {n \choose n-k}$.  There are at least 3 ways to prove this.
\vfill
\vfill
\question Conjecture a formula for the sum of the $n$th row of Pascal's triangle. How could you prove the formula is correct?
\vfill
\begin{parts}
  \part  How many subsets does a set of cardinality $n$ possess?  That is, if $|A| = n$, what is $|\pow(A)|$?
  \vfill
  \part Of all the subsets of $A$, how many have cardinality 0?  Cardinality 1?  Cardinality 2?  etc?  
  \vfill
  \part How are parts (a) and (b) related?  What does this have to do with Pascal's triangle?
  \vfill
\end{parts}

\newpage

\question The Stanley Cup is decided in a best of 7 tournament between two teams.  In how many ways can your team win?  Let's answer this question two ways:
\begin{parts}
  \part How many of the 7 games does your team need to win?  How many ways can this happen?
  \vfill
  \part What if the tournament goes all 7 games?  So you win the last game.  How many ways can the first 6 games go down?  
  \vfill
  \part What if the tournament goes just 6 games?  How many ways can this happen?  What about 5 games?  4 games?  
  \vfill
  \part What are the two different ways to compute the number of ways your team can win?  What pattern in Pascal's triangle is this an example of?
  \vfill
\end{parts}
\question Generalize.  That is, what if the rules changed and you played a best of $2n-1 $ tournament?  What {\em binomial identity} (i.e., pattern in Pascal's triangle) do you get?
\vfill



\end{questions}

\end{document}


