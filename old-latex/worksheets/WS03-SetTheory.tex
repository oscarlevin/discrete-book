\documentclass[11pt]{exam}

\usepackage{amsmath, amssymb, multicol}
\usepackage{graphicx}
\usepackage{textcomp}

 \def\d{\displaystyle}
\def\?{\reflectbox{?}}
\def\b#1{\mathbf{#1}}
\def\f#1{\mathfrak #1}
\def\c#1{\mathcal #1}
\def\s#1{\mathscr #1}
\def\r#1{\mathrm{#1}}
\def\N{\mathbb N}
\def\Z{\mathbb Z}
\def\Q{\mathbb Q}
\def\R{\mathbb R}
\def\C{\mathbb C}
\def\F{\mathbb F}
\def\A{\mathbb A}
\def\X{\mathbb X}
\def\E{\mathbb E}
\def\O{\mathbb O}
\def\U{\mathcal U}
\def\pow{\mathcal P}
\def\inv{^{-1}}
\def\nrml{\triangleleft}
\def\st{:}
\def\~{\widetilde}
\def\rem{\mathcal R}
\def\sigalg{$\sigma$-algebra }
\def\Gal{\mbox{Gal}}
\def\iff{\leftrightarrow}
\def\Iff{\Leftrightarrow}
\def\land{\wedge}
\def\And{\bigwedge}
\def\AAnd{\d\bigwedge\mkern-18mu\bigwedge}
\def\Vee{\bigvee}
\def\VVee{\d\Vee\mkern-18mu\Vee}
\def\imp{\rightarrow}
\def\Imp{\Rightarrow}
\def\Fi{\Leftarrow}

%\def\={\equiv}
\def\var{\mbox{var}}
\def\mod{\mbox{Mod}}
\def\Th{\mbox{Th}}
\def\sat{\mbox{Sat}}
\def\con{\mbox{Con}}
\def\bmodels{=\joinrel\mathrel|}
\def\iffmodels{\bmodels\models}
\def\dbland{\bigwedge \!\!\bigwedge}
\def\dom{\mbox{dom}}
\def\rng{\mbox{range}}
\DeclareMathOperator{\wgt}{wgt}


\def\bar{\overline}


\newcommand{\vtx}[2]{node[fill,circle,inner sep=0pt, minimum size=4pt,label=#1:#2]{}}
\newcommand{\va}[1]{\vtx{above}{#1}}
\newcommand{\vb}[1]{\vtx{below}{#1}}
\newcommand{\vr}[1]{\vtx{right}{#1}}
\newcommand{\vl}[1]{\vtx{left}{#1}}
\renewcommand{\v}{\vtx{above}{}}

\def\circleA{(-.5,0) circle (1)}
\def\circleAlabel{(-1.5,.6) node[above]{$A$}}
\def\circleB{(.5,0) circle (1)}
\def\circleBlabel{(1.5,.6) node[above]{$B$}}
\def\circleC{(0,-1) circle (1)}
\def\circleClabel{(.5,-2) node[right]{$C$}}
\def\twosetbox{(-2,-1.4) rectangle (2,1.4)}
\def\threesetbox{(-2.5,-2.4) rectangle (2.5,1.4)}
\newcommand{\twoline}[2]{\begin{pmatrix}#1 \\ #2 \end{pmatrix}}



%Create ``defbox'' environment to highlight important definitions
\newenvironment{defbox}[1]{\vskip 1em \begin{framed}\noindent{\bf #1\\}}{\end{framed}}


%\pointname{pts}
\pointsinmargin
\marginpointname{pts}
\addpoints
\pagestyle{head}
%\printanswers

\firstpageheader{Math 228}{\bf Set Theory Notation}{Spring 2013}


\begin{document}

%space for name
%\noindent {\large\bf Name:} \underline{\hspace{2.5in}}
%\vskip 1em


\begin{questions}
\question Let $A = \{x^2 \st x \in \N\}$.  
\begin{parts}
  \part Describe the set $A$ in a couple other ways - in words and by listing (some of) the elements.
  \vfill

\part Classify each of the following as true, false, or meaningless.
\begin{multicols}{4}
  \begin{subparts}
    \subpart $4 \in A$

    \subpart $4 \subseteq A$

    \subpart $\{4\} \in A$

    \subpart $\{4\} \subseteq A$

  \end{subparts}

\end{multicols}
\end{parts}
\vfill

\question Let $A$ be any set.  Classify each of the following as always true, sometimes false, or meaningless.
\begin{multicols}{3}
  \begin{parts}
    \part $\emptyset \in A$
    \vskip 4em
    \part $\emptyset \subseteq A$
    \vskip 4em
    \part $A \in \pow(A)$
    \vskip 4em
    \part $A \subseteq \pow(A)$
    \vskip 4em
    \part $\emptyset \in \pow(A)$
    \vskip 4em
    \part $\emptyset \subseteq \pow(A)$
  \end{parts}
\end{multicols}

\vfill

\question Are the following statements true for all sets $A$ and $B$?  If so, explain why.  If not, give a counter-example.
\begin{parts}
  \part $A \cup B \subseteq B$
  \vfill
  \part $A \cap B \subseteq B$
  \vfill
  \part $A \subseteq A \cup B$
  \vfill
  \part $A \subseteq A \cap B$
  \vfill
\end{parts}


\end{questions}
\newpage
\begin{defbox}{Set Theory Notation}

\noindent  \begin{tabular}{l p{1.5in} p{3.5in}}
    Symbol: & Read: & Example: \\ \hline \\[1ex]
    $\{$, $\}$ & braces & $\{1,2,3\}$.  The braces enclose the elements of a set.  This is the set which contains the numbers 1, 2 and 3.\\[1ex]
    $\st$ & such that & $\{x \st x > 2\}$ is the set of all $x$ such that $x$ is greater than 2.\\[1ex]
    $\in$ & is an element of & $2 \in \{1,2,3\}$ asserts that 2 is one of the elements in the set $\{1,2,3\}$.  However, $4 \notin\{1,2,3\}$.\\[1ex]
    $\subseteq$ & is a subset of & $A \subseteq B$ asserts that every element of $A$ is also an element of $B$.\\[1ex]
    $\subset$ &is a proper subset of & $A \subset B$ asserts that every element of $A$ is also an element of $B$, but $A \ne B$.\\[1ex]
    $\cap$ & intersection & $A \cap B$ is the {\em set} of all elements which are elements of both $A$ and $B$.\\[1ex]
    $\cup$ & union & $A \cup B$ is the {\em set} of all elements which are elements of $A$ or $B$ or both.\\[1ex]
    $\setminus$ & set difference & $A \setminus B$ is the {\em set} of all elements of $A$ which are not elements of $B$.\\[1ex]
    $\bar A$ & compliment (of $A$) & $\bar A$ is the set of everything which is not an element of $A$.  The $A$ can be any set here.\\[1ex]
    $|A|$ & cardinality (of $A$)& $|\{4,5,6\}| = 3$ because there are 3 elements in the set.  Sometimes we say $|A|$ is the {\em size} of $A$.\\[1ex]
    
  \end{tabular}

\noindent{\bf Special sets}

\begin{tabular}{l p{5in}}
  $\emptyset$ & The {\em empty set} is the set which contains no elements.\\[1ex]
  $\U$ & The {\em universe set} is the set of all elements.\\[1ex]
$\N$ & The set of natural numbers. That is, $\N = \{0, 1, 2, 3\ldots\}$ \\[1ex]
$\Z$ & The set of integers.  $\Z = \{\ldots, -2, -1, 0, 1, 2, 3, \ldots\}$\\[1ex]
$\Q$ & The set of rational numbers.\\[1ex]
$\R$ & The set of real numbers.\\[1ex]
$\pow(A)$ & The {\em power set} of any set $A$ is the set of all subsets of $A$.
\end{tabular}
\end{defbox}


\end{document}


