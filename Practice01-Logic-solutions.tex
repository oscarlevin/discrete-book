\begin{questions} \Newassociation{answer}{Ans}{Practice01-Logic-solutions}
\begin{Ans}\Ansparams 
 \begin{tabular}{c|c|c}
             $P$ & $Q$ & $(P \vee Q) \imp (P \and Q)$\\ \hline
             T & T & T \\
             T & F & F \\
             F & T & F \\
             F & F & T
          \end{tabular}
\end{Ans}
\begin{Ans}\Ansparams 
      \begin{tabular}{c|c|c}
             $P$ & $Q$ & $\neg P \and (Q \imp P)$\\ \hline
             T & T & F \\
             T & F & F \\
             F & T & F \\
             F & F & T
          \end{tabular}
	If the statement is true, then both $P$ and $Q$ are false.
    
\end{Ans}
\begin{Ans}\Ansparams 
    Hint: Like above, only now you will need 8 rows instead of just 4.
  
\end{Ans}
\begin{Ans}\Ansparams 
    Make a truth table for each and compare.  The statements are logically equivalent.
  
\end{Ans}
\begin{Ans}\Ansparams 
    Again, make two truth tables.  The statements are logically equivalent.
  
\end{Ans}
\begin{Ans}\Ansparams 
    The argument is valid.  To see this, make a truth table which contains $P \vee Q$ and $\neg P$ (and $P$ and $Q$ of course).  Look at the truth value of $Q$ in each of the rows that have $P \vee Q$ and $\neg P$ true.
  
\end{Ans}
\begin{Ans}\Ansparams 
    The argument form is valid.  Again, make a truth table containing the premises and conclusion - look at the rows for which the premises are true.
  
\end{Ans}
\begin{Ans}\Ansparams 
    The argument is NOT valid.  If you make a truth table containing the premises and conclusion, there will be a row with both premises true but the conclusion false.  For example, if $P$ and $Q$ are false and $R$ is true, then $P \and Q$ is false, so $(P \and Q) \imp R$ is true.  Also $\neg P$ is true, so $\neg P \vee \neg Q$ is true.  However, $\neg R$ is false.
  
\end{Ans}
\begin{Ans}\Ansparams 
    \begin{parts}
      \part $P$: it's your birthday; $Q$: there will be cake.  $(P \vee Q) \imp Q$
      \part Hint: you should get three T's and one F.
      \part Only that there will be cake.
      \part It's NOT your birthday!
      \part It's your birthday, but the cake is a lie.
    \end{parts}
  
\end{Ans}
\begin{Ans}\Ansparams 
    \begin{parts}
      \part $P \and Q$
      \part $P \imp \neg Q$
      \part Jack passed math or Jill passed math (or both).
      \part If Jack and Jill did not both pass math, then Jill did.
      \part
	\begin{subparts}
	  \subpart Nothing else.
	  \subpart  Jack did not pass math either.
	\end{subparts}
    \end{parts}
  
\end{Ans}
\begin{Ans}\Ansparams 
    \begin{parts}
	\part Three statements: $P \vee S$, $S \imp Q$, $(P \vee Q) \imp R$.  You could also connect the first two with a $\and$.
	\part He cannot be lying about all three sentences, so he is telling the truth.
	\part No matter what, Geoff wants ricotta.  If he doesn't have quail, then he must have pepperoni but not sausage.
    \end{parts}
  
\end{Ans}
\begin{Ans}\Ansparams 
    \begin{parts}
      \part If Oscar drinks milk, then he eats Chinese food.
      \part If Oscar does not drink milk, then he does not eat Chinese food.
      \part Yes.  The original statement would be false too.
      \part Nothing. The converse need not be true.
      \part He does not eat Chinese food. The contrapositive would be true.
    \end{parts}
  
\end{Ans}
\begin{Ans}\Ansparams 
    \begin{parts}
      \part $P \and Q$
      \part $(P \vee Q) \vee (Q \and \neg R)$
      \part F.  Or $(P \and Q) \and (R \and \neg R)$
      \part Either Sam is a woman and Chris is a man, or Chris is a woman.
    \end{parts}
  
\end{Ans}
\begin{Ans}\Ansparams 
 The statements are equivalent to the\ldots
    \begin{parts}
      \part converse.
      \part implication.
      \part neither.
      \part implication.
      \part converse.
      \part converse.

      \part implication.
      \part converse.
      \part converse.
      \part converse (in fact, this IS the converse).
      \part implication (the statement is the contrapositive of the implication).
      \part neither.
    \end{parts}
  
\end{Ans}
\begin{Ans}\Ansparams 
    Hint: of course there are many answers.  It helps to assume that the statement is true and the converse is NOT true.  Think about what that means in the real world and then start saying it in different ways.  Some ideas: use necessary and sufficient language, use ``only if,'' consider negations, use ``or else'' language.
  
\end{Ans}
 \end{questions} \par \end{document}
