\documentclass[11pt]{exam}

\usepackage{amsmath, amssymb, multicol}
\usepackage{graphicx}
\usepackage{textcomp}
\usepackage{chessboard}

\def\d{\displaystyle}
\def\b{\mathbf}
\def\R{\mathbf{R}}
\def\Z{\mathbf{Z}}
\def\st{~:~}
\def\bar{\overline}
\def\inv{^{-1}}
\def\imp{\rightarrow}
\def\Imp{\Rightarrow}
\def\iff{\Leftrightarrow}
\def\Iff{\Leftrightarrow}
\def\land{\wedge}

%\pointname{pts}
\pointsinmargin
\marginpointname{pts}
\addpoints
\pagestyle{head}
%\printanswers

\firstpageheader{Math 228}{\bf Trees and Forests}{Wednesday, September 5}
% \runningheader{}{\bf For next time:}{}


\begin{document}

%space for name
%\noindent {\large\bf Name:} \underline{\hspace{2.5in}}
%\vskip 1em

A \emph{tree} is a connected graph which contains no cycles.  A \emph{forest} is a graph which contains no cycles (so each connected component is a tree).  The goal of this activity is to prove some facts about trees and forests.

\begin{questions}
\question Consider the statement: \textbf{A graph is a forest if and only if there is at most one path between any pair of vertices.}  To prove this statement, you must prove \emph{two} implications.

\begin{parts}
  \part First, consider the implication: \textbf{If there is at most one path between any pair of vertices, then the graph is a forest.}  Write the first and last line of a proof of this statement provided you were to prove it \emph{directly}, \emph{by contrapositive} or \emph{by contradiction}.
  \begin{solution}
    Directly: Assume there is at most one path between any pair of vertices.  \ldots Therefore the graph is a forest.
    
    Contrapositive: Assume the graph is not a forest.  \ldots. Therefore there is more than one path between some pair of vertices.
    
    Contradiction: Assume there is at most one path between a pair of vertices AND the graph is not a forest.  \ldots.  This is a contradiction.
  \end{solution}
  \vfill
  \part Write the first and last line for each of the proof styles for proving the converse: \textbf{if the graph is a forest, then there is at most one path between any pair of vertices.}
  \begin{solution}
    Directly: Assume the graph is a forest.   \ldots. Therefore there is at most one path between any pair of vertices.
    
    Contrapositive: Assume there is more than one path between some pair of vertices. \ldots. Therefore the graph is not a forest. 
    
    Contradiction: Assume the graph is a forest but there is more than one path between some pair of vertices.  \ldots.  This is a contradiction.
  \end{solution}
  \vfill
  \clearpage
  \part For both \emph{directions}, complete the proof using an appropriate style.
  \begin{solution}
    We will give contrapositive proofs for both directions.
    \begin{proof}
      First, suppose the graph is not a forest.  This means there is a cycle.  Let $u$ and $v$ be vertices in that cycle.  Since you can get from $u$ to $v$ by going either clockwise or counterclockwise around the cycle, we see there is more than one path between this pair of vertices.  
      
      Conversely, assume there is some pair of vertices $u$ and $v$ for which there is more than one path from $u$ to $v$.  The two paths might start out the same, but since they are different paths, there is some vertex $u'$ after which the two paths diverge (for the first time).  Then, since both the paths eventually end at $v$, there is some first vertex $v'$ at which the two paths intersect again.  But the two paths from $u'$ to $v'$ form a cycle, so we can conclude the graph is not a forest.
    \end{proof}
  \end{solution}
  \vfill
\end{parts}
\question Prove that any tree with at least two vertices has at least one leaf (i.e., a degree 1 vertex).  Note this can be phrased as an implication: if $G$ is a tree with at least two vertices, then $G$ has at least one vertex of degree 1.
\begin{solution}
 Let $T$ be a tree, and assume, contrary to stipulation, that no vertex has degree one.  Let $P$ be the longest path in $T$.  What can we say about the endpoints of $P$?  Since $T$ has no degree one vertices, there must be edges coming off of the endpoints in $P$.  These edges cannot go to any additional vertices, since $P$ is the longest path in $T$.  But also, these edges cannot go to a vertex already part of $P$, because that would create a cycle.  So we have a contradiction, which shows that $T$ must have a vertex of degree one.
 
 In fact, both endpoints of $P$ must have degree one, so every tree has at least two degree one vertices (if the $T$ has at least two vertices).
\end{solution}
\vfill

\end{questions}

\end{document}
