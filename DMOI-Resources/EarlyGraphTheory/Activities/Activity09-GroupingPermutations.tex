\documentclass[11pt]{exam}

\usepackage{amsmath, amssymb, multicol}
\usepackage{graphicx}
\usepackage{textcomp}
\usepackage{chessboard}
\usepackage{tikz}
\usepackage{adjustbox}
\usepackage{pdfpages}

\def\d{\displaystyle}
\def\b{\mathbf}
\def\R{\mathbf{R}}
\def\Z{\mathbf{Z}}
\def\st{~:~}
\def\bar{\overline}
\def\inv{^{-1}}
\def\r{2.5pt}
% \def\v{circle (\r)}
\def\vo{node[circle, color=black, fill=white,  inner sep=2pt, minimum size = 17pt, draw]{}}
\newcommand{\vtx}[2]{node[fill,circle,inner sep=0pt, minimum size=4pt,label=#1:#2]{}}
\newcommand{\va}[1]{\vtx{above}{#1}}
\newcommand{\vb}[1]{\vtx{below}{#1}}
\newcommand{\vr}[1]{\vtx{right}{#1}}
\newcommand{\vl}[1]{\vtx{left}{#1}}
\renewcommand{\v}{\vtx{above}{}}

%\pointname{pts}
\pointsinmargin
\marginpointname{pts}
\addpoints
\pagestyle{head}
%\printanswers

\firstpageheader{Math 228}{\bf Grouping Permutations}{Monday, October 8}


\begin{document}

%space for name
%\noindent {\large\bf Name:} \underline{\hspace{2.5in}}
%\vskip 1em


\begin{questions}
\question Suppose you have all 13 cards of one suit (say Hearts).
\begin{parts}
  \part How many ways are there to arrange this stack of 13 cards?
  \vfill
  \part How many ways could you deal five cards from the 13?
  \vfill
  \part How many 5-card hands could you get from these 13 cards?  How is this different from the previous question?
  \vfill
  \part How many ways could you deal five cards from the 13 so that you deal them in increasing order?  Is this like part (b) or part (c)?
  \vfill
\end{parts}
\clearpage
\question Consider the set $A = \{a,b,c,d\}$.  We can ask two related questions: 
\begin{enumerate}
  \item How many sequences of three distinct elements drawn from $A$ are there?
  \item How many sets of three distinct elements drawn from $A$ are there?  
\end{enumerate}
How are these questions related?  Which is larger and by how much?  Think about these questions as you complete the following.
\begin{parts}
  \part List out all 3-element sequences drawn from $A$ without repeats.  You might want to do this in a table with 4 columns.
  \vfill
  \vfill
  \part List out all 3-element subsets of $A$ (which automatically do not include repeats).
  \vfill
  \part How does your list of sequences relate to your list of subsets? Why does it make sense to multiply the number of subsets by 6 to get the number of sequences?
\end{parts}
\end{questions}





\end{document}
