\documentclass[11pt]{exam}

\usepackage{amsmath, amssymb, multicol}
\usepackage{graphicx}
\usepackage{textcomp}
\usepackage{chessboard}

\def\d{\displaystyle}
\def\b{\mathbf}
\def\R{\mathbf{R}}
\def\Z{\mathbf{Z}}
\def\st{~:~}
\def\bar{\overline}
\def\inv{^{-1}}

\def\v{circle (3pt)}


%\pointname{pts}
\pointsinmargin
\marginpointname{pts}
\addpoints
\pagestyle{head}
%\printanswers

\firstpageheader{Math 228}{\bf Magic Beans and License Plates}{Monday, November 5}


\begin{document}

%space for name
%\noindent {\large\bf Name:} \underline{\hspace{2.5in}}
%\vskip 1em

\subsection*{Activity 1: Magic Beans}

You have traded your cow for 10 magic chocolate covered espresso beans.  Each night at midnight, each espresso beans splits into two beans.  To take advantage of this, you eat 5 beans each morning for breakfast.  You wonder how many beans you will have after breakfast 30 days after you traded your cow.

\begin{questions}
  \question Write out the first few terms of the sequences $(b_n)_{n \ge 0}$ and give a recursive definition for $b_n$.
  \vfill
  \question For $n \ge 1$, what will the last digit of $b_n$ be?  How can you prove this is correct?
  \vfill
  \question Find a closed formula for $b_n$ and prove it is correct using induction.
  \vfill
  \vfill
\end{questions}


\newpage

\subsection*{Activity 2: License Plates}

How many license plates consist of 6 symbols, using up to three numerals (1, 2, and 3) and four letters (a, b, c, and d), for which no letter can be placed before any numeral?  For example, 31daab, babadc, and 132311 are all acceptable plates, but 13ba2a is not.
\begin{questions}
  \question Try this: how many of the plates have zero numerals?  How many have 1?  etc.
\vfill
\vfill
  \question Is there a way to add up the cases in a nice closed form?
  \vfill
  \vfill
  \question How might you generalize this problem?  Is there any nice binomial identity you could prove with a combinatorial proof here?
  \vfill
\end{questions}


\end{document}
