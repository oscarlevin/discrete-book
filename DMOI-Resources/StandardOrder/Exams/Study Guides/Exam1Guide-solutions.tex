\protect \documentclass [10pt]{exam} 
\protect \usepackage {answers, amsthm, amsmath, amssymb, mathrsfs} \protect \pagestyle {head} \protect \firstpageheader {Math 228}{\protect \bf  Exam 1 Study Guide\protect \\ Hints and Answers}{Fall 2017} \protect \Newassociation {answer}{Ans}{Exam1Guide-solutions} 
 \def\d{\displaystyle}
\def\?{\reflectbox{?}}
\def\b#1{\mathbf{#1}}
\def\f#1{\mathfrak #1}
\def\c#1{\mathcal #1}
\def\s#1{\mathscr #1}
\def\r#1{\mathrm{#1}}
\def\N{\mathbb N}
\def\Z{\mathbb Z}
\def\Q{\mathbb Q}
\def\R{\mathbb R}
\def\C{\mathbb C}
\def\F{\mathbb F}
\def\A{\mathbb A}
\def\X{\mathbb X}
\def\E{\mathbb E}
\def\O{\mathbb O}
\def\U{\mathcal U}
\def\pow{\mathcal P}
\def\inv{^{-1}}
\def\nrml{\triangleleft}
\def\st{:}
\def\~{\widetilde}
\def\rem{\mathcal R}
\def\sigalg{$\sigma$-algebra }
\def\Gal{\mbox{Gal}}
\def\iff{\leftrightarrow}
\def\Iff{\Leftrightarrow}
\def\land{\wedge}
\def\And{\bigwedge}
\def\AAnd{\d\bigwedge\mkern-18mu\bigwedge}
\def\Vee{\bigvee}
\def\VVee{\d\Vee\mkern-18mu\Vee}
\def\imp{\rightarrow}
\def\Imp{\Rightarrow}
\def\Fi{\Leftarrow}

%\def\={\equiv}
\def\var{\mbox{var}}
\def\mod{\mbox{Mod}}
\def\Th{\mbox{Th}}
\def\sat{\mbox{Sat}}
\def\con{\mbox{Con}}
\def\bmodels{=\joinrel\mathrel|}
\def\iffmodels{\bmodels\models}
\def\dbland{\bigwedge \!\!\bigwedge}
\def\dom{\mbox{dom}}
\def\rng{\mbox{range}}
\DeclareMathOperator{\wgt}{wgt}


\def\bar{\overline}


\newcommand{\vtx}[2]{node[fill,circle,inner sep=0pt, minimum size=4pt,label=#1:#2]{}}
\newcommand{\va}[1]{\vtx{above}{#1}}
\newcommand{\vb}[1]{\vtx{below}{#1}}
\newcommand{\vr}[1]{\vtx{right}{#1}}
\newcommand{\vl}[1]{\vtx{left}{#1}}
\renewcommand{\v}{\vtx{above}{}}

\def\circleA{(-.5,0) circle (1)}
\def\circleAlabel{(-1.5,.6) node[above]{$A$}}
\def\circleB{(.5,0) circle (1)}
\def\circleBlabel{(1.5,.6) node[above]{$B$}}
\def\circleC{(0,-1) circle (1)}
\def\circleClabel{(.5,-2) node[right]{$C$}}
\def\twosetbox{(-2,-1.4) rectangle (2,1.4)}
\def\threesetbox{(-2.5,-2.4) rectangle (2.5,1.4)}
\newcommand{\twoline}[2]{\begin{pmatrix}#1 \\ #2 \end{pmatrix}}

\usepackage{tikz, multicol}
\renewenvironment{Ans}[1]{\setcounter{question}{#1}\addtocounter{question}{-1}\question }{}
\begin{document}
 \begin{questions}
\begin{Ans}{1}
  \begin{parts}
    \part The converse: For all sets $A$ and $B$, if $|A \cup B| = 2|A|$, then $|A| = |B|$.

    The contrapositive: For all sets $A$ and $B$, if $|A \cup B| \ne 2|A|$ then $|A| \ne |B|$.

    \part The original statement is false.  That means that there \emph{exists} sets $A$ and $B$ such that $|A| = |B|$ but $|A \cup B| \ne 2|A|$.  We can prove this by giving an example of such sets: $A = \{1,2\}$ and $B= \{2,3\}$.  These sets have the same cardinality, but $A \cup B = \{1,2,3\}$ has cardinality 3, not 4.

    \part The converse is false as well.  Take $A = \{1,2\}$ and $B = \{2,3,4\}$.  Now we have $|A \cup B| = 4 = 2|A|$ but $|A| = 2$ and $|B| = 3$.

    \part The new statement is about specific sets.  For these sets, $|A| \ne |B|$, so the hypothesis of the implication is false.  That makes the entire statement true automatically (even though the conclusion is false).  This one example does not prove that the original implication is true (after all, it is not).  It also cannot prove that the original statement is false, as it is not a counterexample.
  \end{parts}
\end{Ans}
\begin{Ans}{2}
     $A \cap B = \emptyset$.  We know this because the set $A \cup B$ contains 25 elements, each of which is either from $A$ or from $B$, or from both.  But there can't be any from both, because $9 + 16 = 25$.  So $A \cap B$ contains no elements - it is the empty set.
  
\end{Ans}
\begin{Ans}{3}
		No, $Y$ could be larger than $X$.
	
\end{Ans}
\begin{Ans}{4}
		You can say that $|Y| = |X|$ (so $Y$ is finite as well).  In fact, you can say $|Y| = |X|$ in this case even if $X$ is not finite (the sets would have the same infinite cardinality).
	
\end{Ans}
\begin{Ans}{5}
		You own 8 purple bow ties, 3 red bow ties, 3 blue bow ties and 5 green bow ties.  How many ways can you select one of each color bow tie to take with you on a trip?  $8 \cdot 3 \cdot 3 \cdot 5$.  How many choices do you have for a single bow tie to wear tomorrow?  $8 + 3 + 3 + 5$.
	
\end{Ans}
\begin{Ans}{6}
  \begin{parts}
    \part $A = \{(b_1,f_1), (b_2,f_1),(b_3,f_1),(b_4,f_1),(b_5,f_1),(b_6,f_1),(b_7,f_1),\\ (b_1,f_2),(b_2,f_2),(b_3,f_2),(b_4,f_2),(b_5,f_2),(b_6,f_2),(b_7,f_2)\}$
    \part Let $B = \{(b_1,f_1), (b_2,f_1),(b_3,f_1),(b_4,f_1),(b_5,f_1),(b_6,f_1),(b_7,f_1)\}$ and \\ $C = \{(b_1,f_2),(b_2,f_2),(b_3,f_2),(b_4,f_2),(b_5,f_2),(b_6,f_2),(b_7,f_2)\}$ (so $B$ contains all the outfits with the first fez and $C$ all the outfits with the second fez).  We have $A = B \cup C$, and since $B$ and $C$
    are disjoint, we see that $|A| = |B \cup C|$.
    \part Let $D = \{b_1, b_2, b_3, b_4, b_5, b_6, b_7\}$ and $E = \{f_1, f_2\}$.  Forming all ordered pairs gives us $A$ (we need to think of $A$ as containing ordered pairs, otherwise all we get is a bijection between $A$ and $D \times E$).  Then we have that $|A| = 7 \cdot 2 = |D| \cdot |E|$.
  \end{parts}
\end{Ans}
\begin{Ans}{7}
		\begin{parts}
		\part $4^5$. %How many such numbers are there?
		\part $4^4\cdot 2$ (choose any digits for the first four digits - then pick either an even or an odd last digit to make the sum even). %How many such numbers are there for which the {\em sum} of the digits is even?
		\part We need 3 or more even digits.  3 even digits: ${5 \choose 3}2^3 2^2$.  4 even digits: ${5 \choose 4}2^4 2$.  5 even digits: ${5 \choose 5}2^5$.  So all together: ${5 \choose 3}2^3 2^2 + {5 \choose 4}2^4 2 + {5 \choose 5}2^5$.  %  How many such numbers contain more even digits than odd digits?
		\end{parts}
	
\end{Ans}
\begin{Ans}{8}
		215.  Use PIE: $100 + 83 + 71 - 16 - 14 -11 + 2 = 215$ or a Venn diagram.  To find out how many numbers are divisible by 6 and 7, for example, take $500/42$ and round down.
	
\end{Ans}
\begin{Ans}{9}
		\begin{parts}
		  \part $2^8$. %How many $8$-bit strings are there total?
		  \part ${8 \choose 5}$  %How many $8$-bit strings have weight 5?
		  \part ${8 \choose 5}$ %How many subsets of the set $\{a,b,c,d,e,f,g,h\}$ contain exactly 5 elements?
		  \part There is a bijection between subsets and bit strings: a 1 means that element in is the subset, a 0 means that element is not in the subset.  To get a subset of an 8 element set we have a 8-bit string.  To make sure the subset contains exactly 5 elements, there must be 5 1's, so the weight must be 5. %Explain why your answers to parts (b) and (c) are the same.  Why are these questions equivalent?
		\end{parts}
	
\end{Ans}
\begin{Ans}{10}
		 With repeated letters allowed: ${8 \choose 5}5^5 21^3$.  Without repeats: ${8 \choose 5}5! P(21, 3)$.
	
\end{Ans}
\begin{Ans}{11}
		\begin{parts}
		  \part ${5 \choose 2}{11 \choose 6}$ %pass through the point $(2,3)$.
		  \part ${16 \choose 8} - {12 \choose 7}{4 \choose 1}$   %avoid (do not pass through) the point $(7,5)$.
		  \part ${5 \choose 2}{11 \choose 6} + {12 \choose 5}{4 \choose 3} - {5 \choose 2}{7 \choose 3}{4 \choose 3}$ %either pass through $(2,3)$ or $(5,7)$ (or both).
		\end{parts}
	
\end{Ans}
\begin{Ans}{12}
		 ${18 \choose 8}\left({18 \choose 8} - 1\right)$
	
\end{Ans}
\begin{Ans}{13}
		 One answer is ${10 \choose 6}$, the other is $P(10, 6)$.  These are different, in fact, $P(10,6)$ is $6!$ times larger than ${10 \choose 6}$.  This is because ${10 \choose 6}$ is the number of ways to select which 6 of the 10 questions you will answer, but then assumes you will answer them in the usual order.  Once you have selected the 6 questions, you could answer them (possibly) out of order in 6! different ways, and that is what $P(10,6)$ counts: There are 10 choices for which question to answer first, 9 for which to answer second, and so on until you have answered 6 questions.
	
\end{Ans}
\begin{Ans}{14}
  \begin{parts}
    \part Call the items B, C, P, and R.  There are 24 outcomes:

    \begin{tabular}{cccccc}
      BCP & BPC & CBP & CPB & PBC & PCB \\
      BCR & BRC & CBR & CRB & RBC & RCB \\
      BPR & BRP & PBR & PRB & RBP & RPB \\
      CPR & CRP & PCR & PRC & RCP & RPC
    \end{tabular}

    We know this is all of them because there are 4 choices for which item we choose first, then 3 choices for the second item, and 2 choices for the 3rd.  Order matters in the sense that different arrangements count as different outcomes.

    \part Now the set of outcomes is $\{ BCP, BCR, BPR, CPR\}$.  We just need to choose 1 item not to order.  Or equivalently, ${4 \choose 3} = 4$.  Notice that we are giving these in alphabetical order, but that is because we are NOT including the (repeat) outcomes when the same three items are listed in different orders.

    \part The 24 outcomes for part (a) are arranged in a table so that each row corresponds to a set of three items, and the columns in that row are the 6 different ways to permute those three items.  This illustrates that ${4 \choose 3} = P(4,3)/3!$
  \end{parts}
\end{Ans}
\begin{Ans}{15}
		 $2^7 + 2^7 - 2^4$.
	
\end{Ans}
\begin{Ans}{16}
		${7 \choose 3} + {7 \choose 4} - {4 \choose 1}$.
	
\end{Ans}
\begin{Ans}{17}
		Each step in our path adds 1 to either $x$ or $y$.  So to end at a point on $x+y = 5$, we must make $5$ steps, each being in the $x$ or $y$ direction.  Thus all together there are $2^5$ such paths.

    Answering this another way, notice that these paths end at $(0,5)$,  $(1,4)$, $(2, 3)$, $(3,2)$, $(4,1)$, or $(5,0)$.  Counting the paths to each of these points separately, we get ${5 \choose 0}$, ${5 \choose 1}$, ${5 \choose 2}$, \ldots, ${5 \choose 5}$ (each time choosing which of the $n$ steps are in the $x$ direction).  All together then we get
    \[{5 \choose 0} + {5 \choose 1} + {5\choose 2} + {5 \choose 3} + {5 \choose 4} + {5 \choose 5}\]

    These two answers are the same.  This is an example of the fact that the sum of the $n$th row in Pascal's triangle is $2^n$.
	
\end{Ans}
\begin{Ans}{18}
		This might remind you a little about the anagrams questions, so you could use as a question: how many $n$-letter words can you make using $k$ a's, $j$ b's and the rest of the letters c's?  One answer is to pick $k$ of the $n$ spots to fill with a's (in ${n \choose k}$ ways), then chose $j$ of the remaining $n-k$ spots to fill with b's, and fill the remaining spots with c's.  Answer 2 is to first pick the spots where the b's go: pick $j$ of the $n$ spots to fill with b's, then pick $k$ of the remaining $n-j$ spots to fill with a's, and fill the remaining spots with c's.

		Another question you could ask: how many ways are there to select a $k$-person team and a $j$-person team from a group of $n$ people.  The two answers depend on which team you pick first.
	
\end{Ans}
\begin{Ans}{19}
		Hint: stars and bars%Suppose you have 20 one-dollar bills to give out as prizes to your top 5 discrete math students.  How many ways can you do this if:
		\begin{parts}
		  \part ${19 \choose 4}$ %each of the 5 students gets at least 1 dollar?
		  \part ${24 \choose 4}$ %some students might get nothing?
		  \part ${19 \choose 4} - \left[{5 \choose 1}{12 \choose 4} - {5 \choose 2}{5 \choose 4}  \right]$ %each student gets at least 1 dollar but no more than 7 dollars?
		\end{parts}
	
\end{Ans}
\begin{Ans}{20}
		\begin{parts}
		  \part $5^4 + 5^4 - 5^3$ %$f(1) = a$ or $f(2) = b$ (or both)?
		  \part $4\cdot 5^4 + 5 \cdot 4 \cdot 5^3 - 4 \cdot 4 \cdot 5^3$ %$f(1) \ne a$ or $f(2) \ne b$ (or both)?
		  \part $5! - \left[ 4! + 4! - 3! \right]$ %$f(1) \ne a$ {\em and} $f(2) \ne b$, and are also one-to-one?
      \part Use PIE: $4! + 4! + 4! - 3! - 3! - 3! + 2!$
		  % \part $5! - \left[{5 \choose 1}4! - {5 \choose 2}3! + {5 \choose 3}2! - {5 \choose 4}1! + {5 \choose 5} 0!\right]$ %are onto but have $f(1) \ne a$, $f(2) \ne b$, $f(3) \ne c$, $f(4) \ne d$ and $f(5) \ne e$?
		\end{parts}
	
\end{Ans}
\begin{Ans}{21}
	\begin{parts}
		\part
			${10 \choose 5}$.  Note that a strictly increasing function is automatically injective.  So the five outputs must all be different.  So let's first pick which five outputs we will use: there are ${10 \choose 5}$ ways to do this.  Now how many ways are there to assign those outputs to the inputs $1$ through 5?  Only one way, since there is only one way to arrange numbers in increasing order.

		\part
			${14 \choose 5}$.  This is in fact a stars and bars problem.  The stars are the 5 inputs and the bars are the 9 spots between the 10 possible outputs.  Think of it this way - we will specify $f(1)$, then $f(2)$, then $f(3)$, and so on in that order.  Start with the possible output 0.  We can use it as the output of $f(1)$, or we can switch to 1 as a potential output.  Say we put $f(1) = 1$.  Now we are at 1 (can't go back to 0).  Should $f(2) = 1$?  If yes, then we are putting down another star.  If no, put down a bar and switch to 2.  Maybe you switch to 3, then assign $f(2) = 3$ and $f(3) = 3$ (two more stars) before switching to 4 as a possible output.  And so on.

	\end{parts}
	
\end{Ans}
\begin{Ans}{22}
		 ${5 \choose 1}\left( 4! - \left[{4 \choose 1}3! - {4 \choose 2}2! + {4 \choose 3} 1! - {4 \choose 4} 0!\right] \right)$
	
\end{Ans}
\begin{Ans}{23}
		$4^6 - \left[{4 \choose 1}3^6 - {4 \choose 2}2^6 + {4 \choose 3} 1^6 \right]$
	
\end{Ans}
\begin{Ans}{24}
\begin{parts}
\part $\d{22 \choose 6}$ -- there are 16 stars and 6 bars.
\part $\d{15 \choose 6}$ -- buy one of each item, using \$7.  This leaves you \$11 to distribute to the 7 items, so 11 stars and 6 bars.
\part \[{22 \choose 6} - \left[{7 \choose 1}{17 \choose 6} - {7 \choose 2}{12 \choose 6} + {7 \choose 3}{7 \choose 6} \right]\]
\part $\d{7 \choose 3}$.  This is just a combination: choose 3 of the 7 items.

\end{parts}
\end{Ans}
\begin{Ans}{25}
  Everyone is right, under different interpretations.  The 2520 number is correct if you assume that customers will pick 5 different items, but care about what order they eat them in (so eating a taco and then a burrito would be a different meal than eating a burrito then a taco).  Then the answer would be $P(7,5) = 7\cdot 6 \cdot 5 \cdot 4$.  If we care about the order we eat the items in but allow repeated items, we get $7^5 = 16807$.  If we just want to count the number of 5-item meals, with 5 distinct items, we get ${7 \choose 5} = 21$.  Another reasonable answer would be to count the number of 5-item meals, not distinguishing between different orders of consumption, but allowing for repeated items.  This would be a stars-and-bars problem, giving ${11 \choose 5}$ meals.
\end{Ans}
 \protect \end {questions} \par \protect \end {document}
