\protect \documentclass [10pt]{exam} 
\protect \usepackage {answers, amsthm, amsmath, amssymb, mathrsfs} \protect \pagestyle {head} \protect \firstpageheader {Math 228}{\protect \bf  Exam 2 Study Guide\protect \\ Hints and Answers}{Fall 2017} \protect \Newassociation {answer}{Ans}{Exam2Guide-solutions} 
 \def\d{\displaystyle}
\def\?{\reflectbox{?}}
\def\b#1{\mathbf{#1}}
\def\f#1{\mathfrak #1}
\def\c#1{\mathcal #1}
\def\s#1{\mathscr #1}
\def\r#1{\mathrm{#1}}
\def\N{\mathbb N}
\def\Z{\mathbb Z}
\def\Q{\mathbb Q}
\def\R{\mathbb R}
\def\C{\mathbb C}
\def\F{\mathbb F}
\def\A{\mathbb A}
\def\X{\mathbb X}
\def\E{\mathbb E}
\def\O{\mathbb O}
\def\U{\mathcal U}
\def\pow{\mathcal P}
\def\inv{^{-1}}
\def\nrml{\triangleleft}
\def\st{:}
\def\~{\widetilde}
\def\rem{\mathcal R}
\def\sigalg{$\sigma$-algebra }
\def\Gal{\mbox{Gal}}
\def\iff{\leftrightarrow}
\def\Iff{\Leftrightarrow}
\def\land{\wedge}
\def\And{\bigwedge}
\def\AAnd{\d\bigwedge\mkern-18mu\bigwedge}
\def\Vee{\bigvee}
\def\VVee{\d\Vee\mkern-18mu\Vee}
\def\imp{\rightarrow}
\def\Imp{\Rightarrow}
\def\Fi{\Leftarrow}

%\def\={\equiv}
\def\var{\mbox{var}}
\def\mod{\mbox{Mod}}
\def\Th{\mbox{Th}}
\def\sat{\mbox{Sat}}
\def\con{\mbox{Con}}
\def\bmodels{=\joinrel\mathrel|}
\def\iffmodels{\bmodels\models}
\def\dbland{\bigwedge \!\!\bigwedge}
\def\dom{\mbox{dom}}
\def\rng{\mbox{range}}
\DeclareMathOperator{\wgt}{wgt}


\def\bar{\overline}


\newcommand{\vtx}[2]{node[fill,circle,inner sep=0pt, minimum size=4pt,label=#1:#2]{}}
\newcommand{\va}[1]{\vtx{above}{#1}}
\newcommand{\vb}[1]{\vtx{below}{#1}}
\newcommand{\vr}[1]{\vtx{right}{#1}}
\newcommand{\vl}[1]{\vtx{left}{#1}}
\renewcommand{\v}{\vtx{above}{}}

\def\circleA{(-.5,0) circle (1)}
\def\circleAlabel{(-1.5,.6) node[above]{$A$}}
\def\circleB{(.5,0) circle (1)}
\def\circleBlabel{(1.5,.6) node[above]{$B$}}
\def\circleC{(0,-1) circle (1)}
\def\circleClabel{(.5,-2) node[right]{$C$}}
\def\twosetbox{(-2,-1.4) rectangle (2,1.4)}
\def\threesetbox{(-2.5,-2.4) rectangle (2.5,1.4)}
\newcommand{\twoline}[2]{\begin{pmatrix}#1 \\ #2 \end{pmatrix}}

\usepackage{tikz, multicol}
\renewenvironment{Ans}[1]{\setcounter{question}{#1}\addtocounter{question}{-1}\question }{}
\begin{document}
 \begin{questions}
\begin{Ans}{1}
		\begin{parts}
		\part 36.  %How many terms (summands) are in the sum?
		\part $\frac{253 \cdot 36}{2} = 4554$.  %Compute the sum.  Remember to show all your work.
		\end{parts}
	
\end{Ans}
\begin{Ans}{2}
		\begin{parts}
		  \part $a_n = a_{n-1} + 4$ with $a_1 = 5$.  %Give a recursive definition for the sequence.
		  \part $a_n = 5 + 4(n-1)$  %Give a closed formula for the $n$th term of the sequence.
		  \part Yes, since $2013 = 5 + 4(503-1)$ (so $a_{503} = 2013$).
		  \part 133 %How many terms does the sequence $5, 9, 13, 17, 21, \ldots, 533$ have?
		  \part $\frac{538\cdot 133}{2} = 35777$  %Find the sum: $5 + 9 + 13 + 17 + 21 + \cdots + 533$.  Show your work.
		  \part $b_n = 1 + \frac{(4n+6)n}{2}$.
		\end{parts}
	
\end{Ans}
\begin{Ans}{3}
		$a_n = n^2 + 4n - 1$
	
\end{Ans}
\begin{Ans}{4}
	 	\begin{parts}
	 	\part $4, 6, 10, 16, 26, 42, \ldots$  %Write out the first 6 terms of the sequence.
	 	\part No, taking differences gives the original sequence back, so the differences will never be constant.  %Could the closed formula for $a_n$ be a polynomial?  Explain.
	 	\end{parts}
	
\end{Ans}
\begin{Ans}{5}
		 $b_n = (n+3)n$
	
\end{Ans}
\begin{Ans}{6}
		\begin{parts}
		 \part $1, 2, 16,68, 364, \ldots$  %Write out the first 5 terms of the sequence defined by this recurrence relation.
		 \part $a_n = \frac{3}{7}(-2)^n + \frac{4}{7}5^n$  %Solve the recurrence relation.
		\end{parts}
	
\end{Ans}
\begin{Ans}{7}
		\begin{parts}
		  \part $a_2 = 14$.  $a_3 = 52$  %Find the next two terms of the sequence ($a_2$ and $a_3$).
		  \part $a_n = \frac{1}{6}(-2)^n + \frac{5}{6}4^n$  %Solve the recurrence relation.   That is, find a closed formula for the $n$th term of the sequence.
%		  \part $\frac{1+x}{1-2x-8x^2}$  %Find the generating function for the sequence.  Hint: use the recurrence relation.
		\end{parts}
	
\end{Ans}
\begin{Ans}{8}
		\begin{parts}
  			\part $5, 9, 21, 57, 165, 489, \ldots$.
			\part By the end of day $n$, you will have three the number of beans you had the previous day, less 6.  Thus $a_n = 3a_{n-1} - 6$ with $a_0 = 5$.
			\part Let $P(n)$ be the statement, ``at the end of day $n$, you have an odd number of beans.''  For the base case, not that on day 0, you have 5 beans, which is odd.  Now assume that $P(k)$ is true.  That is, at the end of the $k$th day, you have an odd number of beans.  What happens on the next day?  We triple the number of beans, and subtract 6.  If you triple an odd number, you will get an odd number.  Then subtracting 6 (an even number) will still give you an odd number.  Thus the number of beans at the end of day $k+1$ will be odd.  This concludes the inductive case.  Therefore, by the principle of mathematical induction, $P(n)$ is true of all $n \ge 0$.
			\part The inductive case works independent of the base case.  We can prove the IF you have an odd number of beans on day $k$, THEN you have an odd number of beans on day $k+1$.  But to go from this to the conclusion that you will always have an odd number of beans requires a starting spot.  In fact, if you started with 4 beans, then the next day you would have 6 beans, and then 12, and so on.  All these numbers would be even!
		\end{parts}
	
\end{Ans}
\begin{Ans}{9}
	 \begin{parts}
	 	\part On the first day, your 2 mini bunnies become 2 large bunnies.  On day 2, your two large bunnies produce 4 mini bunnies.  On day 3, you have 4 mini bunnies (produced by your 2 large bunnies) plus 6 large bunnies (your original 2 plus the 4 newly matured bunnies).  On day 4, you will have $12$ mini bunnies (2 for each of the 6 large bunnies) plus 10 large bunnies (your previous 6 plus the 4 newly matured).  The sequence of total bunnies is $2, 2, 6, 10, 22, 42\ldots$ starting with $a_0 = 2$ and $a_1 = 2$.
	 	\part $a_n = a_{n-1} + 2a_{n-2}$.  This is because the number of bunnies is equal to the number of bunnies you had the previous day (both mini and large) plus 2 times the number you had the day before that (since all bunnies you had 2 days ago are now large and producing 2 new bunnies each).
	 	\part Using the characteristic root technique, we find $a_n = a2^n + b(-1)^n$, and we can find $a$ and $b$ to give $a_n = \frac{4}{3}2^n + \frac{2}{3}(-1)^n$.
	 \end{parts}
	
\end{Ans}
\begin{Ans}{10}
		\begin{parts}
		 \part Hint: $(n+1)^{n+1} > (n+1) \cdot n^{n}$.
		 \begin{proof}
		 Let $P(n)$ be the statement $n! < n^n$.  We will prove $P(n)$ is true for all $n \ge 2$.  For the base case, note that $2! = 2$ while $2^2 = 4$, so $P(2)$ is true.  For the inductive case, assume $P(k)$ is true for some arbitrary $k \ge 2$.  That is, $k! < k^k$.  Consider $(k+1)! = (k+1)k!$.  This is less than $(k+1)k^k$ and $k^k$ is less than $(k+1)^k$.  That is,
		 \begin{align*}
		 (k+1)! & = (k+1)k! \\
		 & < (k+1)k^k \\
		 & < (k+1)(k+1)^k \\
		 & = (k+1)^{k+1}
		 \end{align*}
		 Thus $P(k+1)$ is true as well.  Therefore by the principle of mathematical induction $P(n)$ is true for all $n \ge 2$.
		 \end{proof}

		 \part \begin{proof}
		 Let $P(n)$ be the statement $\d\frac{1}{1\cdot 2} + \frac{1}{2\cdot 3} +\frac{1}{3\cdot 4}+\cdots + \frac{1}{n\cdot(n+1)} = \d\frac{n}{n+1}$.  We will show $P(n)$ is true for all $n \ge 1$.  For the base case, we have $\frac{1}{1\cdot 2} = \frac{1}{1+1}$, so $P(1)$ is true.

		 For the inductive case, assume $P(k)$ is true for some arbitrary $k \ge 1$.  That is, $\d\frac{1}{1\cdot 2} + \frac{1}{2\cdot 3} +\frac{1}{3\cdot 4}+\cdots + \frac{1}{k\cdot(k+1)} = \d\frac{k}{k+1}$.  Consider what happens when we add the next fraction, namely $\frac{1}{(k+1)(k+2)}$ to both sides:
		 \[\d\frac{1}{1\cdot 2} + \frac{1}{2\cdot 3} +\frac{1}{3\cdot 4}+\cdots + \frac{1}{k\cdot(k+1)} + \frac{1}{(k+1)(k+2)} = \d\frac{k}{k+1} + \frac{1}{(k+1)(k+2)}\]
		 The right hand side becomes
		 \[\frac{k(k+2)}{(k+1)(k+2)} + \frac{1}{(k+1)(k+2)} = \frac{k^2 + 2k + 1}{(k+1)(k+2)} = \frac{k+1}{k+2}\]
		 Thus $P(k+1)$ is true.  Therefore, by the principle of mathematical induction, $P(n)$ is true for all $n \ge 1$.

		 \end{proof}
		 \part Hint: Write $4^{k+1} - 1 = 4\cdot 4^k - 4 + 3$.

		 \begin{proof}
		 Let $P(n)$ be the statement ``$4^n-1$ is a multiple of 3.''  We will show $P(n)$ is true for all $n \ge 0$.

		 Base case: $4^0 -1 = 0$ which is a multiple of 3, so $P(0)$ is true.

		 Inductive case: Assume $P(k)$ is true for some arbitrary $k \ge 0$.  That is, $4^k - 1$ is a multiple of 3.  Consider $4^{k+1} - 1$.  We can write this as $4\cdot 4^k - 4 + 3$, or equivalently $4(4^k - 1) + 3$.  Since $4^k-1$ is a multiple of 3 (by the inductive hypothesis), $4$ times it is a multiple of 3, and adding 3 will still give a multiple of 3.  Thus $4^{k+1} - 1$ is a multiple of 3, so $P(k+1)$ is true.

		 Therefore by the principle of mathematical induction $P(n)$ is true for all $n \ge 0$.
		 \end{proof}

		 \part Hint: Use the fact $F_{2n} + F_{2n+1} = F_{2n+2}$
		 \part Hint: one 9-cent stamp is 1 more than two 4-cent stamps, and seven 4-cent stamps is 1 more than three 9-cent stamps.
		 \part Careful to actually use induction here.  The base case: $2^2 = 4$.  The inductive case: assume $(2n)^2$ is divisible by 4 and consider $(2n+2)^2 = (2n)^2 + 4n + 4$.  This is divisible by 4 because $4n +4$ clearly is, and by our inductive hypothesis, so is $(2n)^2$.
		\end{parts}
	
\end{Ans}
\begin{Ans}{11}
		Note that $1 = 2^0$ -- this is your base case.  Now suppose $k$ can be written as the sum of distinct powers of 2 for all $1\le k \le n$.  We can then write $n$ as the sum of distinct powers of 2 as follows: subtract the largest power of 2 less than $n$ from $n$.  That is, write $n = 2^j + k$ for the largest possible $j$.  But $k$ is now less than $n$, and also less than $2^j$, so write $k$ as the sum of distinct powers of 2 (we can do so by the inductive hypothesis).  Thus $n$ can be written as the sum of distinct powers of 2 for all $n \ge 1$.
	
\end{Ans}
\begin{Ans}{12}
		Let $P(n)$ be the statement, ``every set containing $n$ elements has $2^n$ different subsets.''  We will show $P(n)$ is true for all $n \ge 1$.

		\underline{Base case}: Any set with 1 element $\{a\}$ has exactly 2 subsets: the empty set and the set itself.  Thus the number of subsets is $2= 2^1$.  Thus $P(1)$ is true.

		\underline{Inductive case}: Suppose $P(k)$ is true for some arbitrary $k \ge 1$.  Thus every set containing exactly $k$ elements has $2^k$ different subsets.  Now consider a set containing $k+1$ elements: $A = \{a_1, a_2, \ldots, a_k, a_{k+1}\}$.  Any subset of $A$ must either contain $a_{k+1}$ or not.  In other words, a subset of $A$ is just a subset of $\{a_1, a_2,\ldots, a_k\}$ with or without $a_{k+1}$.  Thus there are $2^k$ subsets of $A$ which contain $a_{k+1}$ and another $2^{k+1}$ subsets of $A$ which do not contain $a^{k+1}$.  This gives a total of $2^k + 2^k = 2\cdot 2^k = 2^{k+1}$ subsets of $A$.  But our choice of $A$ was arbitrary, so this works for any subset containing $k+1$ elements, so $P(k+1)$ is true.

		Therefore, by the principle of mathematical induction, $P(n)$ is true for all $n \ge 1$.
	
\end{Ans}
 \protect \end {questions} \par \protect \end {document}
