\documentclass[12pt]{exam}

\usepackage{amsmath, amssymb, amsthm, multicol}
\usepackage{graphicx}
\usepackage{textcomp}

\def\d{\displaystyle}
\def\matrix#1{\begin{bmatrix}#1\end{bmatrix}}
\def\b{\mathbf}
\def\R{\mathbb{R}}
\def\Z{\mathbb{Z}}
\def\N{\mathbb{N}}
\def\and{\wedge}
\def\imp{\rightarrow}
\def\inv{^{-1}}
\def\st{~:~}



%\pointname{pts}
\pointsinmargin
\marginpointname{pts}
\addpoints
\pagestyle{head}
%\printanswers

\firstpageheader{Math 228}{\bf Group Induction ``Quiz''}{Friday, October 13}


\begin{document}

%space for name
 \noindent {\large\bf Names:} \underline{\hspace{2.5in}}, \qquad   \underline{\hspace{2.5in}},
 \vskip 1em
\hspace{.5in} \underline{\hspace{2.5in}}, \qquad\underline{\hspace{2.5in}},
  \vskip 1em
\hspace{.5in}   \underline{\hspace{2.5in}}, \qquad  \underline{\hspace{2.5in}}.
   \vskip 1em


Work with your table to write a valid induction proof of the following statement.  First agree on the idea of the proof (why and how you are going to use induction), then write down a formal proof.  Make sure everyone agrees that the proof is perfect.

\vskip 1ex

\begin{center}
$2^0 + 2^1 + \cdots + 2^n = 2^{n+1} - 1$ for all natural numbers $n$.
\end{center}

\begin{solution}
The reason induction is good to use here is that we are finding the sum of the first $n$ terms of a sequence.  It is easier to add on the $(n+1)$st term than to find the sum from scratch.

\begin{proof}
Let $P(n)$ be the statement ``$2^0 + 2^1 + \cdots + 2^n = 2^{n+1} - 1$.''  We will prove $P(n)$ is true for all natural numbers $n$.

\underline{Base case}: $P(0)$ is true because $2^0 = 2^1 - 1$.

\underline{Inductive case}:  Assume $P(k)$ is true for some arbitrary $k$ in the natural numbers.  That is, $2^0 + 2^1 + \cdots + 2^k = 2^{k+1} - 1$.  Since this is true, if we add $2^{k+1}$ to both sides, we will get a true equation.  This is:
\[2^0 + 2^1 + \cdots + 2^k + 2^{k+1} = 2^{k+1} - 1 + 2^{k+1}\]
but the right hand side is $2\cdot 2^{k+ 1} - 1 = 2^{k+2} - 1$.  Thus
\[2^0 + 2^1 + \cdots + 2^{k+1} = 2^{k+2} - 1\]
which is the statement $P(k+1)$, completing the inductive case.

Therefore, by the principle of mathematical induction, $P(n)$ is true for all natural numbers $n$.
\end{proof}
\end{solution}

\end{document}
