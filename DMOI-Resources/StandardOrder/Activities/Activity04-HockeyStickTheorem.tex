\documentclass[11pt]{exam}

\usepackage{amsmath, amssymb, multicol}
\usepackage{graphicx}
\usepackage{textcomp}


\def\d{\displaystyle}
\def\b{\mathbf}
\def\R{\mathbf{R}}
\def\Z{\mathbf{Z}}
\def\st{~:~}
\def\bar{\overline}
\def\inv{^{-1}}
\def\pow{\mathcal P}


%\pointname{pts}
\pointsinmargin
\marginpointname{pts}
\addpoints
\pagestyle{head}
%\printanswers

\firstpageheader{Math 228}{\bf The Hockey Stick Theorem}{Monday, September 11}
\runningheader{Math228}{\bf Combinatorial Proofs}{Wednesday, September 13}

\begin{document}

%space for name
%\noindent {\large\bf Name:} \underline{\hspace{2.5in}}
%\vskip 1em

\begin{questions}
%\question Explain why ${n \choose k} = {n \choose n-k}$.  There are at least 3 ways to prove this.
%\vfill
%\vfill
%\question Conjecture a formula for the sum of the $n$th row of Pascal's triangle. How could you prove the formula is correct?
%\vfill
%\begin{parts}
%  \part  How many subsets does a set of cardinality $n$ possess?  That is, if $|A| = n$, what is $|\pow(A)|$?
%  \vfill
%  \part Of all the subsets of $A$, how many have cardinality 0?  Cardinality 1?  Cardinality 2?  etc?
%  \vfill
%  \part How are parts (a) and (b) related?  What does this have to do with Pascal's triangle?
%  \vfill
%\end{parts}
%
%\newpage

\question The Stanley Cup is decided in a best of 7 tournament between two teams.  In how many ways can your team win?  Let's answer this question two ways:
\begin{parts}
  \part How many of the 7 games does your team need to win?  How many ways can this happen?
  \begin{solution}
  $\d{7 \choose 4}$.  You must choose 4 of the 7 games to win (the others you will lose or just not play if they occur after your 4th win).
  \end{solution}
  \vfill
  \part Let's give a second solution.  Divide the outcomes into cases, based on how long the series lasts (4 games, 5 games, 6 games, or all 7 games), and count each case separately, then adding them up.  Hint: if you win your 4th game in the $n$th game of the series, you must have won $3$ of the previous $n-1$ games.

  \begin{solution}
	  If the series goes 7 games (and you win), then you must have won 3 out of the first 6 games.  There are $\d\binom{6}{3}$ ways this can happen.  If you win the tournament (win your 4th game) in game 6, then you must have won some 3 of the previous 5 games.  This can happen in ${5 \choose 3}$ ways.  If the tournament only lasted 5 games, you would win that 5th game, plus need to win 3 of the first 4, which can happen in ${4 \choose 3}$ ways.  If the tournament ends in 4 games, you win all 4, which can only happen in 1 way.  But it might be better to write this as ${3 \choose 3}$, to fit the pattern.
  \end{solution}
  \vfill
  \vfill
  \vfill
\end{parts}
\question Generalize.  What if the rules changed and you played a best of $9$ tournament (5 wins required)?  What if you played an $n$ game tournament with $k$ wins required to be named champion?  What binomial identity have we established?
\begin{solution}
For a 9-game tournament, you would arrive at the identity:
\[{9 \choose 5} = {8 \choose 4} + {7 \choose 4} + {6 \choose 4} + {5 \choose 4} + {4\choose 4}\]

In general, the hockey stick theorem is:
\[{n \choose k} = {n-1 \choose k-1} + {n-2 \choose k-1} + {n-3 \choose k-1} + \cdots + {n-k-1 \choose k-1}\]

\end{solution}
\vfill
\vfill





\clearpage

\uplevel{Here are some binomial identities.  In each case, find a counting question you can answer it two ways: the left hand side and the right hand side.}


  \question $\d \binom{n}{0} + \binom{n}{1} + \binom{n}{2} + \cdots + \binom{n}{n} = 2^n$.  Hint: bit strings.
  \vfill

  \question $\d\binom{n}{0}^2 + \binom{n}{1}^2 + \binom{n}{2}^2 + \cdots + \binom{n}{n}^2 = \binom{2n}{n}$.  Hint: lattice paths.
  \vfill

  \question $\d 1 \cdot n + 2 \cdot (n-1) + 3\cdot (n-2)+ \cdots + (n-1)\cdot 2 + n\cdot 1 = \binom{n+2}{3}$.  Hint: subsets
  \vfill

  \question $\d \binom{n}{0} + \binom{n+1}{1} + \binom{n+2}{2} + \cdots + \binom{n+k}{k} = \binom{n+k+1}{k}$.  \\Bonus: give three proofs using bit strings, subsets, and lattice paths.
  \vfill

\end{questions}


\end{document}
